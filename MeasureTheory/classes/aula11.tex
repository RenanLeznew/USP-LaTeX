\documentclass[MeasureTheory/measure_theory.tex]{subfiles}
\begin{document}
\section{Aula 11 - 29/01/2024}
\subsection{Motivações}
\begin{itemize}
	\item Desigualdade de Hölder;
	\item Desigualdade de Minkowski;
	\item Espaços de Banach.
\end{itemize}
\subsection{O Espaço \(L^{p}\)}
\subsubsection*{Leonardo C. - Desigualdade de Hölder e Minkowski}
\begin{def*}
	Seja \((X, \mathcal{A}, \mu )\) um espaço de medida \(\sigma \)-finita.
	\begin{itemize}
		\item[i)] Se \(1\leq p<\infty\), defina a \textbf{norma }\(L^{p}\) de f por
		      \[
			      \Vert f \Vert_{p}=\biggl(\int_{}^{}|f|^{p}d\mu \biggr)^{\frac{1}{p}}.
		      \]
		\item[2)] Se \(p=\infty\), defina a \textbf{norma }\(L^{\infty}\) de f por
		      \[
			      \Vert f \Vert_{\infty}=\mathrm{essup}_{x\in X}|f(x)| = \inf_{}\{M\geq 0: \mu (\{x:|f(x)|>M\})=0\}.
		      \]
		      Se tal M não existir, então \(\Vert f \Vert_{\infty} = \infty.\) \(\square\)
	\end{itemize}
\end{def*}
É preciso fazermos algumas observações para melhor entendimento. A começar por notar que a norma \(L^{\infty}\) de uma função f é o menor número M tal que \(|f|\leq M\) quase toda parte (q.t.p.).

Para todo \(1\leq p\leq \infty\), o espaço \(L^{p}\) é o conjunto
\[
	\{f: f \text{ é mensurável e }\Vert f \Vert_{p}<\infty\}.
\]
Podemos escrever \(L^{p}(X)\) ou \(L^{p}(\mu )\) se desejamos enfatizar o espaço ou a medida. Com isto tudo,
\[
	\Vert f \Vert_{p} = 0 \Longleftrightarrow f = 0 \mathrm{q.t.p}
\]
\begin{def*}
	Seja \(1 < p < \infty\) e defina q tal que
	\[
		\frac{1}{p} + \frac{1}{q} = 1.
	\]
	Chamamos p e q de \textbf{expoentes conjugados}. Casos específicos são o caso \(p =1 \), para o qual \(q=\infty\), e \(p = \infty\), ao qual associamos \(q = 1\) como conjugado. \(\square\)
\end{def*}
\hypertarget{holder}{
	\begin{prop*}[Desigualdade de Hölder]
		Sejam \(1 < p, q < \infty\) expoentes conjugados. Se f e g são funções mensuráveis, então
		\[
			\int_{}^{}|fg|d\mu \leq \Vert f \Vert_{p}\Vert g \Vert_{q}.
		\]
		O resultado também vale se \(p=\infty\) e \(q=1\), ou se \(p = 1\) e \(q=\infty\).
	\end{prop*}
}
\begin{proof*}
	Primeiramente, seja \(p=\infty\) e \(q=1\). Caso \(M = \Vert f \Vert_{\infty},\) então \(|f|\leq M\) q.t.p. e, sendo assim,
	\[
		\int_{}^{}|fg|d\mu = \int_{}^{}|f||g|d\mu \leq \int_{}^{}M|g|d\mu = M \int_{}^{}|g|d\mu = M \Vert g \Vert_{1}.
	\]
	Analogamente, prova-se o caso \(p=1\) e \(q=\infty\).

	Agora, seja \(1 < p, q < \infty\). Se \(\Vert f \Vert_{p} = 0,\) acabou, pois \(f=0\) q.t.p. e \(\int_{}^{}|fg|d\mu  = 0,\) mostrando que o resultado é trivial neste caso e naquele em que \(\Vert g \Vert_{p} = 0.\) Como
	o resultado também é imediato nos casos em que um dos dois é infinito, vamos nos restringir ao caso que ambos são finitos não-nulos. Sejam
	\[
		F(x) = \frac{|f(x)|}{\Vert f \Vert_{p}}\quad\&\quad G(x) = \frac{|g(x)|}{\Vert g \Vert_{p}}.
	\]
	Observe que
	\[
		\Vert F \Vert_{p} = 1\quad\&\quad \Vert G \Vert_{p} = 1.
	\]
	Então, basta mostrarmos que
	\[
		\int_{}^{}FGd\mu \leq 1.
	\]
	Para começar, Lembre-se que a função \(e^{t}\) é \textit{convexa}, ou seja,
	\[
		e^{\lambda a + (1-\lambda )b}\leq \lambda e^{a}+(1-\lambda )e^{b},\quad \forall a\leq b\in \mathbb{R}\quad\&\quad 0\leq \lambda \leq 1.
	\]
	Caso \(F(x)\neq0\) e \(G(x)\neq0\), tome \(a = p \ln^{}{(F(x))}, b = q\ln^{}{(G(x))}, \lambda = 1/p\) e \(1-\lambda =1/q.\) Então,
	\[
		F(x)G(x)\leq \frac{F(x)^{p}}{p} + \frac{G(x)^{q}}{q}.
	\]
	Esta desigualdade vale trivialmente se \(F(x) = 0\) ou \(G(x) = 0.\) Integrando ambos os lados, então, chegamos em
	\[
		\int_{}FG d\mu_{}\leq \frac{\Vert F \Vert_{p}^{p}}{p}+\frac{\Vert G \Vert_{q}^{q}}{q} = \frac{1}{p} + \frac{1}{q} = 1.
	\]
	Portanto, provamos a forma equivalente da Desigualdade de Hölder. \qedsymbol
\end{proof*}
\begin{lemma*}
	Se \(a, b\geq 0\) e \(1\leq p < \infty\), então
	\[
		(a+b)^{p} \leq 2^{p-1}a^{p} + 2^{p-1}b^{p}.
	\]
\end{lemma*}
\begin{proof*}
	Os casos \(p=1\) e \(a=0\) são ambos óbvios, de forma que faz sentido pedir que \(p > 1\) e \(a > 0\). Divide-se os dois lados por \(a^{p}\), a fim de obter
	\[
		\biggl(\frac{a+b}{a}\biggr)^{p}\leq 2^{p-1} + 2^{p-1}\biggl(\frac{b}{a}\biggr)^{p}.
	\]
	Disto, fazendo \(x=b/a\), segue que
	\[
		(1+x)^{p}\leq 2^{p-1} + 2^{p-1}x^{p}.
	\]
	Ainda mais, se \(f(x) = 2^{p-1} + 2^{p-1}x^{p}-(1+x)^{p},\) então basta mostrar que \(f(x)\geq 0\) para todo \(x\geq 0\).

	Com efeito, temos
	\begin{itemize}
		\item \(f(0) = 2^{p-1}-1^{p} = \frac{2^{p}}{2}-1 > 0\), pois \(p>1\);
		\item \(f(1) = 2^{p-1} + 2^{p-1} - 2^{p} = \frac{2^{p}}{2} + \frac{2^{p}}{2} - 2^{p} = 2^{p}-2^{p} = 0\);
		\item \(\lim_{x\to \infty}f(x) = \infty\).
	\end{itemize}
	Como a única solução para \(f'(x) = 0\) em \((0, \infty)\) é \(x=1\), segue que f atinge seu mínimo em \(x=1\). Portanto, \(f(x)\geq 0\) para todo \(x\geq 0\).
\end{proof*}
\hypertarget{minkowski}{
	\begin{prop*}[Desigualdade de Minkowski]
		Se \(1\leq p\leq \infty\) e f, g são mensuráveis, então
		\[
			\Vert f + g \Vert_{p}\leq \Vert f \Vert_{p} + \Vert g \Vert_{p}.
		\]
	\end{prop*}
}
Gostaríamos de dizer, em virtude da Desigualdade de Minkowski, que \(L^{p}\) é um \textit{espaço normado}. No entanto, devemos ser cuidadosos, pois \(\Vert f \Vert_{p} = 0\) não implica que \(f\equiv 0\). Até o momento, só sabemos que
\[
	\Vert f \Vert_{p} = 0 \Longleftrightarrow f = 0 \text{ q.t.p.}
\]
Logo, a definição precisa de \(L^{p}\) requer a introdução de uma relação de equivalência com base em medida, isto é,
\[
	f\sim g \Longleftrightarrow f = g \text{ q.t.p.}
\]
Assim, o espaço \(L^{p}\) consiste das classes de equivalência de funções mensuráveis f que satisfazem \(\int_{}^{}|f|^{p}d\mu <\infty.\) Além disso, defina \(\Vert f \Vert_{p}\) como a norma \(L^{p}\) de qualquer função
na mesma classe de equivalência que f.
\begin{example}[Espaço \(\ell ^{p}\)]
	Seja \(X = \mathbb{N}\) equipado com a medida de contagem. Então, \(L^{p}(\mathbb{N})\) consiste em todas as sequências \(x=(x_1, x_2, \dotsc )\) tais que
	\[
		\int_{}^{}|x_{n}|^{p}d\mu = \sum\limits_{n=1}^{\infty}|x_{n}|^{p}<\infty.
	\]
	Denotamos este espaço de sequências por \(\ell ^{p}(\mathbb{N})\). Caso \(1\leq p < \infty\), então
	\[
		\Vert x \Vert_{\ell ^{p}} = \biggl(\sum\limits_{n=1}^{\infty}|x_{n}|^{p}\biggr)^{\frac{1}{p}}.
	\]
	Em contraste, se \(p = \infty\), definimos
	\[
		\Vert x \Vert_{\ell ^{\infty}} = \sup_{x\in \mathbb{N}}|x_{n}|.
	\]
\end{example}
\begin{theorem*}
	Se \(1\leq p\leq \infty\), então \(L^{p}\) é completo.
\end{theorem*}
\begin{proof*}
	Seja \(1\leq p\leq \infty\) fixado. Considere os seugintes passos:

	1. Seja \(f_{n}\) uma sequência arbitrária de Cauchy em \(L^{p}.\) Escolha \(n_1 < n_2 <\dotsc \) tais que
	\[
		m, n\geq n_{j}\Rightarrow \Vert f_{m} - f_{n} \Vert \leq 2^{-(j+1)}
	\]

	2. Seja \(n_{0} = 0\) e defina \(f_{0}\equiv 0\) (função identicamente nula). Nosso candidato para ser o limite de \(f_{n}\) será
	\[
		\sum\limits_{m=1}^{\infty}(f_{n_{m}} - f_{n_{m-1}}).
	\]
	Tome \(g_{j}(x) = \sum\limits_{m=1}^{j}|f_{n_{m}}(x) - f_{n_{m-1}}(x)|,\) tal que \(g_{j}\) cresça em j para cada x em X fixado. Seja
	\[
		g(x) = \sum\limits_{m=1}^{\infty}|f_{n_{m}}(x) - f_{n_{m-1}}(x)|,
	\]
	que pode inclusive ser infinito, o limite de g. Pela \hyperlink{minkowski}{\textit{Desigualdade de Minkowski}}, segue que
	\[
		\Vert g_{j} \Vert_{p}\leq \sum\limits_{m=1}^{j}\Vert f_{n_{m}}-f_{n_{m-1}} \Vert_{p} \leq \Vert f_{n_{1}}-f_{n_{0}} \Vert_{p} + \sum\limits_{m=2}^{j}2^{-m}\leq \Vert f_{n_{1}} \Vert_{p}+\frac{1}{2}.
	\]
	Pelo \hyperlink{fatou}{\textit{Lema de Fatou}},
	\[
		\int_{}^{}|g(x)|^{p}\mu (dx)\leq \lim_{j\to \infty}\int_{}^{}|g_{j}(x)|^{p}\mu (dx) = \lim_{j\to \infty}\Vert g_{j} \Vert_{p}^{p}\leq \biggl(\frac{1}{2}+\Vert f_{n_1} \Vert^{p}\biggr)^{p}.
	\]
	Então, g é finita q.t.p., o que garante a convergência absoluta q.t.p. da série \(\sum\limits_{m=1}^{\infty}|f_{n_{m}}(x) - f_{n_{m-1}}(x)|.\)

	3. Defina \(f(x) = \sum\limits_{m=1}^{\infty}[f_{n_{m}}(x) - f_{n_{m-1}}(x)]\), a qual sabemos convergir absolutamente q.t.p e, assim, estando bem-definida q.t.p. Seja \(f(x) = 0\) para todo x em que a convergência absoluta não ocorre. Então,
	\[
		f(x) = \lim_{K\to \infty}\sum\limits_{m=1}^{K}[f_{n_{m}}(x) - f_{n_{m-1}}(x)] = \lim_{K\to \infty}f_{n_{K}}(x).
	\]
	Pelo \hyperlink{fatou}{\textit{Lema de Fatou}},
	\[
		\Vert f - f_{n_{j}} \Vert_{p}^{p} = \int_{}^{}|f-f_{n_{j}}|^{p}\leq \liminf_{K\to \infty}\int_{}^{}|f_{n_{K}}-f_{n_{j}}|^{p} = \liminf_{K\to \infty}\Vert f_{n_{K}}-f_{n_{j}} \Vert_{p}^{p} \leq 2^{-(j+1)p}
	\]

	4. Finalmente, \(\Vert f - f_{n_{j}} \Vert_{p}\to 0\) sempre que j tende a infinito. Daí, como toda sequência de Cauchy com subsequência convergente converge para o mesmo limite, segue que \(f_{n}\to f\) sempre que \(n\to \infty\).
	Em particular, \(\Vert f_{n_{j}} - f_{m}\Vert_{p}<\varepsilon \) para j suficientemente grande e \(\varepsilon  > 0\). Pelo \hyperlink{fatou}{\textit{Lema de Fatou}} e se \(m\geq N\),
	\[
		\Vert f-f_{m} \Vert_{p}^{p} - \biggl(\int_{}^{}\liminf_{j\to \infty}|f_{n_{j}}-f_{m}|^{p}\biggr) \leq \liminf_{j\to \infty}\Vert f_{n_{j}}-f_{m} \Vert_{p}^{p}\leq \varepsilon^{p}.
	\]
	Isto prova que \(f_{n}\) converge para f na norma \(L^{p}\) e, logo, que \(L^{p}\) é completo para \(1\leq p < \infty.\)


	Para o último caso, tome \(p=\infty\). Considere os passos:

	1. Seja \(f_{n}\) uma sequência arbitrária de Cauchy em \(L^{p}.\) Para cada \(m, n\in \mathbb{N}\), deifna
	\[
		F_{m, n}\coloneqq \{x\in X: |f_{m}(x) - f_{n}(x)| > \Vert f_{m}-f_{n} \Vert_{\infty}\},
	\]
	de modo que \(\mu (F_{m, n})=0\) para todo m, n naturais. Seja \(F = \bigcup_{m, n}^{}F_{m, n}\) e \(E = F ^{\complement}\). Observe que \(\mu (E ^{\complement}) = \mu (F) = 0\) e, mais ainda,
	\begin{align*}
		E & =\bigcap_{m, n\in \mathbb{N}}^{}\{x\in X: |f_{m}(x) - f_{n}(x)| \leq \Vert f_{m}-f_{n} \Vert_{\infty}\}            \\
		  & =\{x\in X: |f_{m}(x) - f_{n}(x)| \leq \Vert f_{m} - f_{n}(x) \Vert_{\infty}\text{ para todo }m, n\in \mathbb{N}\}.
	\end{align*}
	2. Seja \(\varepsilon  > 0\). Existe \(N\in \mathbb{N}\) tal que
	\[
		m, n\geq N \Rightarrow \Vert f_{m}(x) - f_{n}(x) \Vert_{\infty} < \varepsilon .
	\]
	Com maior razão, para todo x em E e todos \(m, n\geq N\), temos
	\[
		|f_{m}(x) - f_{n}(x)|\leq \Vert f_{m} - f_{n} \Vert_{\infty} < \varepsilon .
	\]
	Provamos, com isso, que para todo x em E, \(f_{n}(x)\) é sequência de Cauchy em \(\mathbb{K},\) em que \(\mathbb{K}\in \{\mathbb{C}, \mathbb{R}\}\). Como \(\mathbb{K}\) é completo, existe o limite
	\[
		\lim_{n\to \infty}f_{n}(x) = f(x).
	\]
	É claro que \(f(x)\) está definida em \(E = F ^{\complement}\), donde, fazendo \(f(x) = 0\) para todo x em F, concluímos que \(f = \lim_{n\to \infty}f_{n}\chi_{E}\) é mensurável.

	3. Sendo \(f_{n}(x)\) uma sequência de Cauchy de números, exsite N natural tal que
	\[
		m, n \geq N \Rightarrow |f_{m}(x) - f_{n}(x)| < \varepsilon .
	\]
	Fazendo \(n\to \infty\), encontramos
	\[
		m\geq N \Rightarrow |f_{m}(x) - f(x)|\leq \varepsilon .
	\]
	Ainda mais,
	\[
		m\geq N \Rightarrow \Vert f_{m}-f \Vert_{\infty} = \inf_{}\{M\geq 0: |f_{m}(x) - f(x)|\leq \varepsilon \}\leq \varepsilon,
	\]
	provando que \(f_m\to f\) na norma \(L^{\infty}.\)

	4. Por fim, pela \hyperlink{minkowski}{\textit{Desigualdade de Minkowski}},
	\[
		\Vert f \Vert_{\infty}\leq \Vert f_{m} \Vert_{\infty} + \Vert f_m - f \Vert_{\infty}\leq \Vert f \Vert_{\infty} + \varepsilon  < \infty
	\]
	para todo \(m\geq N\). Portanto, \(f\in L^{\infty},\) donde concluímos a completude do espaço \(L^{\infty}.\) \qedsymbol

\end{proof*}
\begin{prop*}
	O conjunto das funções contínuas com suporte compacto é denso em \(L^{p}(\mathbb{R})\) para \(1\leq p < \infty\).
\end{prop*}
\begin{proof*}
	Seja \(f\in L^{p}(\mathbb{R})\). Como \(\lim_{n\to \infty}f \chi_{[-n, n]} = f,\) é claro que
	\[
		|f-f\chi_{[-n, n]}|^{p}\to 0
	\]
	sempre que n tender a infinito. Daí, sendo \(|f|^{p}\) integrável, como
	\[
		|f-f\chi_{[-n, n]}|^{p}\leq 2^{p}|f|^{p},
	\]
	segue do \hyperlink{dominated_convergence}{\textit{Teorema da Convergência Dominada}} que
	\[
		\Vert f - f\chi_{[-n, n]} \Vert_{p}^{p} = \int_{}^{}|f-f\chi_{[-n, n]}|^{p}\to 0
	\]
	sempre que \(n\to \infty\). Logo, é suficiente aproximar funções em \(L^{p}\) que possuem suporte compacto, já que
	\[
		\Vert f -g  \Vert \leq \Vert f - f\chi_{[-n, n]} \Vert + \Vert f\chi_{[-n, n]} - g \Vert \leq \frac{\varepsilon }{2} + \frac{\varepsilon }{2} = \varepsilon.
	\]
	Como podemos escrever \(f=f^{+}-f^{-}\), podemos supor \(f\geq 0\) e, daí, encontrarmos sequência crescente de funções simples \(s_{n}\)
	tais que \(s_{n}\to f\). Desta forma, \(s_{n}\chi_{[-n, n]}\to f\) sempre que \(n\to \infty\) e como \(|f-s_{n}\chi_{[-n, n]}|\leq 2|f|^{p},\) segue do TCD que
	\[
		\lim_{n\to \infty}\Vert f - s_{n}\chi_{[-n, n]} \Vert_{p}^{p} = \lim_{n\to \infty}\int_{}^{}|f-s_{n}\chi_{[-n, n]}|^{p} = 0.
	\]
	Então, basta aproximar funções simples com suporte compacto, denotadas por \(s_{n}\). Por linearidade, é suficiente aproximar funções características com suporte compacto. Sendo assim, seja E um conjunto de Borel em um intervalo limitado. Dado \(\varepsilon  > 0\), segue que existe uma função contínua com suporte compacto g e com valores em [0, 1] tal que
	\[
		\int_{}^{}|g-\chi_{E}|<\varepsilon .
	\]
	Como \(|g-\chi_{E}|\leq 1,\) devemos ter \(|g-\chi_{E}|^{p}\leq |g-\chi_{E}|,\) donde concluímos que
	\[
		\Vert g-\chi_{E} \Vert_{p}^{p} = \int_{}^{}|g-\chi_{E}|^{p} \leq \int_{}^{}|g-\chi_{E}| < \varepsilon .
	\]
	Sendo \(s_{n}\) uma sequência crescente de funções simples com suporte compacto, então para cada \(n\in \mathbb{N},\)
	\[
		s_{n} = \sum\limits_{i=1}^{p}a_{i}\chi_{A_{i}}.
	\]
	Segue do que acabamos de ver que, para cada função característica \(\chi_{A_{i}}, i\in \{1, \dotsc , p\}\), existe uma função contínua com suporte compacto \(g_{i}\) tal que
	\[
		\Vert g_{i}-\chi_{A_{i}} \Vert_{p}^{p}<\frac{\varepsilon ^{p}}{a_{i}2p}.
	\]
	Defina \(g = \sum\limits_{i=1}^{p}a_{i}g_{i},\) que é contínua com suporte compacto. Assim,
	\begin{align*}
		\Vert s_{n}-g \Vert_{p}^{p} & = \biggl\Vert \sum\limits_{i=1}^{p}a_{i}\chi_{A_{i}}-\sum\limits_{i=1}^{p}a_{i}g_{i} \biggr\Vert_{p}^{p}     \\
		                            & = \Vert a_1(\chi_{A_1}-g_1) + \dotsc + a_{p}(\chi_{A_p} - g_p) \Vert_{p}^{p}                                 \\
		                            & \leq \Vert a_1(\chi_{A_1}-g_1) \Vert_{p}^{p} + \dotsc + \Vert a_p(\chi_{A_p}-g_p) \Vert_{p}^{p}              \\
		                            & < a_1\frac{\varepsilon ^{p}}{a_12p} + \dotsc + a_p\frac{\varepsilon ^{p}}{a_p2p} = \frac{\varepsilon^{p}}{2}
	\end{align*}
	Finalmente, como \(s_{n}\to f\), podemos escolher \(s_{n}\) tal que
	\[
		\Vert f-g \Vert_{p}^{p}\leq \Vert f-s_{n} \Vert_{p}^{p}+\Vert s_{n}-g \Vert_{p}^{p}<\frac{\varepsilon ^{p}}{2}+\frac{\varepsilon^{p}}{2} = \varepsilon^{p}.
	\]
	Portanto, tomando a p-ésima raíz,
	\[
		\Vert f-g \Vert_{p} < \varepsilon .\quad \text{\qedsymbol}
	\]

	\begin{crl*}
		O conjunto das funções contínuas em [a, b] é denso no espaço \(L^{2}([a, b])\) com respeito à norma \(L^{2}([a, b])\).
	\end{crl*}

\end{proof*}

\subsubsection*{Leonardo A. - Convolução e Operadores Lineares Contínuos}
\begin{def*}
	A \textbf{convolução} de duas funções mensuráveis f, g é definida por
	\[
		(f*g)(x)\coloneqq \int_{}^{}f(x-y)g(y)dy.
	\]
	quando a integral existir. \(\square\)
\end{def*}
Vamos começar fixando algumas coisas para esta seção.
\begin{itemize}
	\item As funções serão \(f:\mathbb{R}^{n}\rightarrow \mathbb{R}\);
	\item Uma integral existe quando, por exemplo, \(f\in L^{1} \text{ e } g\in L^{*}\), ou \(g\in L^{1}\text{ e }f\in L^{1}\).
\end{itemize}
Observe que \(*\) é uma operação \(*:L^{1}\times L^{1}\rightarrow L^{1}\) que é associativa, comutativa e sem identidade - suponha que existe \(I\in L^{1}\) tal que
\[
	I*f = f,\quad \forall f\in L^{1}.
\]
Então, a transformada de Fourier da convolução (veremos que satisfaz \(\widehat{f*g}(x) = \hat{f}(x)\cdot \hat{g}(x)\)) implicaria
\[
	\hat{I}(x)\cdot \hat{f}(x) = \hat{f(x)} \Rightarrow \hat{I}(x) = 0.
\]
Mas, \(\hat{I}(x)\to 0,\) uma contradição.
\begin{prop*}
	\begin{itemize}
		\item[1)]Assuma \(f, g\in L^{1}.\) Então,  \(f*g\in L^{1}\) com \(|f*g|_1\leq |f|_1|g|_1\)
		\item[2)] Se \(1\leq p\leq \infty\), \(f\in L^{1}\) e \(g\in L^{p},\) então
		      \[
			      |f*g|_p \leq |f|_1|g|_p.
		      \]
	\end{itemize}
\end{prop*}
\begin{proof*}
	O primeiro passo é observar que
	\begin{align*}
		\int_{}^{}|f*g(x)|dx & = \int_{}^{}\biggl\vert \int_{}^{}f(x-y)g(y)dy \biggr\vert dx        \\
		                     & \leq \int_{}^{}\int_{}^{}\biggl\vert |f(x-y)||g(y)| \biggr\vert dydx \\
		                     & =\int_{}^{}\biggl(\int_{}^{}|f(x-y)|dx|g(y)|\biggr)dy                \\
		                     & = \int_{}^{}|f|_1|g(y)|dy = |f|_1|g|_1.
	\end{align*}
	em que utilizamos o \hyperlink{fubini_tonelli}{\textit{Teorema de Fubini-Tonelli}}.

	Para o segundo passo, note que se \(p = \infty\), não há nada ser feito. Então, assuma que \(p < \infty\) e seja q o seu conjugado \(\biggl(\frac{1}{p} + \frac{1}{q} = 1\biggr)\). Temos
	\begin{align*}
		\biggl\vert \int_{}^{}f(y)g(x-y)dy \biggr\vert & \leq \int_{}^{}|f(y)g(x-y)|dy                                                                                                \\
		                                               & \leq \int_{}^{}|f(y)|^{\frac{1}{q}}|f(y)|^{1-\frac{1}{q}}|g(x-y)|dy                                                          \\
		                                               & \leq \biggl(\int_{}^{}|f(y)|dy\biggr)^{\frac{1}{q}}\biggl(\int_{}^{}|f(y)|\biggr)^{p \biggl(1-\frac{1}{q}\biggr)}|g(x-y)|dy.
	\end{align*}
	Por \hyperlink{fubini_tonelli}{\textit{Fubini}} e o acima,
	\begin{align*}
		|f*g|_{p}^{p} = \int_{}^{}|f \cdot g(x)|^{p}dx & \leq \int_{}^{}\biggl(\int_{}^{}|f(y)|dy\biggr)^{\frac{p}{q}}\biggl(\int_{}^{}|f|g(x-y)dx\biggr)^{p} \\
		                                               & = |f|_{1}^{\frac{p}{q}}|g|_{p}^{p}\int_{}^{}f(y)dydx                                                 \\
		                                               & = |f|^{1 + \frac{p}{q}}|g|_{p}.
	\end{align*}
	tirando a p-ésima raíz, então, chegamos no desejado. \qedsymbol
\end{proof*}
Estudaremos agora a aproximação/regularização/molificação de funções. Fixe uma função \(\varphi : \mathbb{R}^{n}\rightarrow \mathbb{R}\) de classe \(C_{K}^{\infty}, \varphi \geq 0\) e \(\int_{}^{}\varphi dx = 1\). Por exemplo,
\[
	\varphi (x) = \left\{\begin{array}{ll}
		e^{-\frac{1}{|x|^{2}}},\quad & x\neq 0 \\
		0,\quad                      & x = 0
	\end{array}\right.
\]
Utilizaremos a notação, para cada \(\varepsilon  > 0\) e sendo n a dimensão do espaço euclidiano,
\[
	\varphi_{\varepsilon }(x) = \frac{1}{\varepsilon^{n} }\varphi (\varepsilon^{-1}x)
\]
Segue que, independente do \(\varepsilon \),
\[
	\int_{}^{}\varphi_{\varepsilon }(x) = 1.
\]
\begin{theorem*}
	Seja \(f\in L^{p}, 1\leq p\leq \infty\).
	\begin{itemize}
		\item[1)] Para cada \(\varepsilon > 0\), a função \(f*\varphi_{\varepsilon }\) é de classe \(C_{K}^{\infty}\). Além disso, se \(\alpha  = (\alpha_1, \alpha_2, \dotsc \alpha_n)\) e
		      \(x = (x_1, x_2, \dotsc , x_{n})\), vale que
		      \[
			      \frac{\partial^{\alpha }(f*\varphi_{\varepsilon })}{\partial x^{\alpha }} = f*\frac{\partial^{\alpha }\varphi_{\varepsilon }}{\partial x^{\alpha }} \Longleftrightarrow \frac{\partial^{\alpha_1 + \alpha_2 + \dotsc + \alpha_{n}}f*\varphi_{\varepsilon}}{\partial x_{1}^{\alpha_1}x_{2}^{\alpha_2}\dotsc x_{n}^{\alpha_{n}}} = f* \frac{\partial^{\alpha_1 + \alpha_2 + \dotsc \alpha_{n}}\varphi_{\varepsilon }}{\partial x_1^{\alpha_1}x_{2}^{\alpha_2}\dotsc x_{n}^{\alpha_{n}}}.
		      \]
		\item[2)] Segue que \(f*\varphi_{\varepsilon }\to f\) q.t.p. Quando \(\varepsilon \to 0\)
		\item[3)] Temos \(f*\varphi_{\varepsilon }\to f\) uniformemente em compactos, sempre que f for contínua.
		\item[4)] Para \(1\leq p < \infty\), \(f*\varphi_\varepsilon \to f\) no sentido \(L^{p}.\)
	\end{itemize}
\end{theorem*}
\begin{proof*}
	Vamos assumir que o suporte \(\mathrm{supp}(\varphi )\subseteq B(0, R)\) para \(R>0\) grande suficiente. Seja \(e_{i}\) o vetor unitário na i-ésima direção e escreva
	\[
		f*\varphi_{\varepsilon }(x+he_{i}) - f*\varphi_{\varepsilon }(x) = \int_{}^{}f(y)[\varphi_{\varepsilon }(x+he_{i}-y) - \varphi_{\varepsilon }(x-y)]dy.
	\]
	Como \(\varphi_{\varepsilon }\) é \(C^{1},\) existe \(c_{1}\) contínua tal que
	\[
		\frac{1}{n}\biggl\vert \varphi _{\varepsilon }(x_he_{i}-y) - \varphi _{\varepsilon }(x-y) \biggr\vert\leq c_1\frac{|h|}{n},\quad \forall x, y.
	\]
	Assim,
	\[
		\int_{}^{}f(y)[\varphi_{\varepsilon }(x+he_{i}-y) - \varphi_{\varepsilon }(x-y)]dy = \int_{}^{}f(y)\frac{\partial \varphi_{\varepsilon }(x-y)}{\partial x_{i}}dy.
	\]
	Os outros casos seguem por indução, finalizando a prova do primeiro item.

	Para o segundo item, considere
	\[
		f(x) = f(x)\cdot 1=f(x)\int_{}^{}\varphi (x-y)dy
	\]
	pontual em x. Logo,
	\begin{align*}
		f*\varphi_{\varepsilon }(x) - f(x) & = \int_{}^{}[f(y)-f(x)]\varphi_\varepsilon  (x-y)dy                                                       \\
		                                   & = \frac{1}{\varepsilon ^{n}}\int_{B(0, R)}^{}[f(y)-f(x)]\varphi \biggl(\frac{x-y}{\varepsilon }\biggr)dy.
	\end{align*}
	Fazendo uma mudança de variáveis \(w=\frac{x-y}{\varepsilon }\),
	\[
		|f*\varphi _{\varepsilon }(x)-f(x)|\leq |\varphi |_{\infty}m(B(0, R\varepsilon ))\frac{1}{m(B(0, R\varepsilon ))}\int_{B(0, R\varepsilon )}^{}f(x)-f(y)dy,
	\]
	que vai a 0 quando \(\varepsilon \to 0\), assim terminando o item 2.

	Para provar o item 3, seja \(N > 0\). Se
	\[
		|f*\varphi _{\varepsilon }(x)-f(x)|\leq |\varphi |_{\infty}m(B(0, R\varepsilon ))\frac{1}{m(B(0, R\varepsilon ))}\int_{B(0, R\varepsilon )}^{}f(x)-f(y)dy,
	\]
	temos
	\[
		\sup_{|x|\leq N}|f*\varphi_{\varepsilon }(x) - f(x)|\leq |\varphi |_{\infty}m(B(0, R))\sup_{|x|<M, |yx|\leq R\varepsilon }|f(x)-f(y)|.
	\]
	Como todos os termos à direita são constantes, obtivemos o que queríamos.

	Por fim, para o item 4, seja \(\varepsilon > 0, f\in L^{p}\) e g limitada suportada em \(B(0, N)\) tal que \(|f-g|_{p}< \varepsilon .\) Defina \(f_{n} = f\chi_{B(0, N)}\) e note que \(|f-f_{n}|\to 0\) quando \(N\to \infty\). Assim,
	\(|f-f_{N}|<\frac{\varepsilon }{2}\) para N suficientemente grande. Podemos tomar, sem perda de generalidade, g simples que aproxima \(f_{N}\) por \(\frac{\varepsilon }{2},\) ou seja, \(|f_{N}-g|<\frac{\varepsilon }{2}\). Como \(g\in L^{\infty}\),
	\[
		|g*\varphi_{\varepsilon }|\leq \Vert g \Vert_{\infty}|\varphi |_{\varepsilon } = \Vert g \Vert_{\infty}.
	\]
	Do item 2, \(g*\varphi_{\varepsilon }\to g\) q.t.p. quando \(\varepsilon \to 0\). Como \(\mathrm{supp}(\varphi_{\varepsilon })\subseteq B(0, R\varepsilon )\) e \(\mathrm{supp}(g)\subseteq B(0, N)\), obtemos
	\[
		\mathrm{supp}(g*\varphi_{\varepsilon })\subseteq B(0, N + R\varepsilon )
	\]
	Aplicando o \hyperlink{dominated_convergence}{\textit{Teorema da Convergência Dominada}},
	\[
		g*\varphi_{\varepsilon }\to g
	\]
	em \(L^{p}\) quando \(\varepsilon \to 0\), provando que o resultado vale para funções simples. Finalmente,
	\[
		|f-g|_{p} <\varepsilon \quad\&\quad |f \varphi_{\varepsilon }(x) - g \varphi_{\varepsilon }(x)| \leq |f-g|_{p}|\varphi_{\varepsilon }|_{1}.
	\]
	Logo,
	\[
		|f*\varphi_{\varepsilon } - f|_{p} = |f*\varphi_{\varepsilon } - g*\varphi_{\varepsilon } + g*\varphi_{\varepsilon } + g - g - f|_{p}\leq 2\varepsilon + |g*\varphi_{\varepsilon }-g|_{p}.
	\]
	Portanto,
	\[
		\limsup_{\varepsilon \to 0}|f*\varphi_{\varepsilon }-f|_p \leq 2\varepsilon.\quad \text{\qedsymbol}
	\]
\end{proof*}
\begin{def*}
	Um \textbf{funcional linear} é uma função \(H:L^{p}\rightarrow \mathbb{R}\) que é linear. \(\square\)
\end{def*}
\begin{def*}
	Um funcional linear é dito \textbf{limitado} se
	\[
		\Vert H \Vert\coloneqq \sup_{|f|_{p}=1}\{H(f)\} < \infty.\quad \square
	\]
\end{def*}
\begin{def*}
	O \textbf{espaço dual} de \(L^{p}\) é a coleção de todos os funcionais lineares limitados, denotado por \((L^{p})^{*}.\quad \square\)
\end{def*}
\begin{example}
	Por exemplo,
	\[
		\biggl(C([a, b])\biggr)^{*} = \{\mu : \mu \text{ é medida suportada}\}.
	\]
\end{example}
\begin{def*}
	A função \textbf{sinal} é definida como
	\[
		\mathrm{sgn}(x)  = \left\{\begin{array}{ll}
			-1,\quad x < 0 \\
			0,\quad x = 0  \\
			1,\quad x > 0.
		\end{array}\right..\quad \square
	\]
\end{def*}
Vale notar que o dual do \(L^{p}\) é \(L^{q}\) quando p é finito e \((L^{\infty})^{*} = BMO\) das funções de variação média.
\begin{theorem*}
	Seja \(1< p < \infty\). Se \(f\in L^{p}\),
	\[
		|f|_{p}=\sup_{|g|_{q}\leq 1}\biggl(\int_{}^{}f(x)g(x)dx\biggr).
	\]
\end{theorem*}
\begin{crl*}
	Basta olhar para as funções simples e a proposição continua válida.
\end{crl*}
\begin{prop*}
	Seja \(1 < p < \infty\) e fixe \(g\in L^{q}\). Defina
	\begin{align*}
		H: & L^{p}\rightarrow \mathbb{R}         \\
		   & H(f)\coloneqq \int_{}^{}f(x)g(x)dx.
	\end{align*}
	Então, \(H\in (L^{p})^{*}.\)
\end{prop*}
\end{document}

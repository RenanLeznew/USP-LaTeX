\documentclass[measure_theory.tex]{subfiles}
\begin{document}
\section{Aula 13 - 01/02/2024}
\subsection{Motivações}
\begin{itemize}
	\item Espaços de Hilbert;
	\item Série de Fourier.
\end{itemize}
\subsection{Laura V. B. - Espaços de Hilbert}
\begin{def*}
	Seja H um espaço vetorial sobre \(\mathbb{F}\in\{\mathbb{C}, \mathbb{R}\}\). Dizemos que H é um \textbf{espaço de produto interno} se existir um mapa
	\(\left< \cdot , \cdot  \right>:H \times H\rightarrow \mathbb{F}\) tal que
	\begin{itemize}
		\item[1)] \(\left< y, x \right> = \overline{\left< x, y \right>}\) para todos x, y em H;
		\item[2)] \(\left< x + y, z \right> = \left< x, z \right> + \left< y, z \right>\) para todos x, y, z em H;
		\item[3)] \(\left< \alpha x, y \right> = \alpha \left< x, y \right>\) para todos \(x, y\in H\) e \(\alpha \in \mathbb{F};\)
		\item[4)] \(\left< x, x \right> \geq 0\) para todo x em H;
		\item[5)] \(\left< x, x \right> = 0\) se, e somente se, \(x=0.\quad \square\)
	\end{itemize}
\end{def*}
Definimos \(\Vert x \Vert = \left< x, x \right>^{\frac{1}{2}},\) tal que \(\left< x, x \right> = \Vert x \Vert^{2}.\) Pelas definições dadas, \(\left< 0, y \right> = 0\) e
\(\left< x, \alpha y \right> = \overline{\alpha }\left< x, y \right>.\)

\hypertarget{cauchy_schwarz}{
	\begin{theorem*}[Desigualdade de Cauchy Schwarz]
		Para todos \(x, y\in H\), temos
		\[
			|\left< x, y \right>|\leq \Vert x \Vert \Vert y \Vert.
		\]
	\end{theorem*}}
\begin{proof*}
	Seja \(A = \Vert x \Vert^{2}, B = |\left< x, y \right>|\) e \(C = \Vert y \Vert^{2}\). Se \(C = 0,\) então \(y=0\) e \(\left< x, y \right> = 0\), tal que a Desigualdade continua verdadeira. Se B for 0,
	é um caso óbvio. Consequentemente, podemos assumir que \(C > 0\) e \(B\neq 0\).

	Caso \(\left< x, y \right> = R e^{i\theta },\) tome \(\alpha  = e^{i\theta }\), tal que \(|\alpha| = 1\) e \(\alpha \left< y, x \right> = |\left< x, y \right>| = B.\) Como B é real, vale que \(\overline{\alpha }\left< x, y \right> = \alpha \left< x, y \right> = |\left< x, y \right>|\). Assim,
	dado \(r\in \mathbb{R},\)
	\begin{align*}
		0 & \leq \Vert x-r\alpha y \Vert^{2}                                                                                        \\
		  & =\left< x-r\alpha y, x-r\alpha y \right>                                                                                \\
		  & = \left< x, x \right> - r\alpha \left< y, x \right> - r\overline{\alpha }\left< x, y \right> + r^{2}\left< y, y \right> \\
		  & = \Vert x \Vert^{2} - 2r|\left< x, y \right>| + r^{2}\Vert y \Vert^{2}.
	\end{align*}
	Desta forma,
	\[
		A - 2Br + Cr^{2} \geq 0
	\]
	para todo r real. Como supomos que \(C > 0\), faz sentido tomarmos \(r = \frac{B}{C}\), tal que obtemos \(B^{2} \leq AC.\) Portanto,
	\[
		\sqrt[]{B^{2}}\leq \sqrt[]{AC} \Longleftrightarrow |\left< x, y \right>|\leq \Vert x \Vert\Vert y \Vert.\quad \text{\qedsymbol}
	\]
\end{proof*}
Um caso particular, resultando da Desigualdade de Cauchy-Schwarz, é a \textit{desigualdade triangular}.
\begin{prop*}
	Para todo x, y em H, temos
	\[
		\Vert x + y \Vert \leq \Vert x \Vert + \Vert y \Vert.
	\]
\end{prop*}
\begin{proof*}
	Basta notar que
	\begin{align*}
		\Vert x + y \Vert^{2} & = \left< x + y, x + y \right>                                                                                   \\
		                      & = \left< x, x \right> + \left< x, y \right> + \left< y, x \right> + \left< y, y \right>                         \\
		                      & \leq \Vert x \Vert^{2} + 2\Vert x \Vert\Vert y \Vert + \Vert y \Vert^{2} = (\Vert x \Vert + \Vert y \Vert)^{2},
	\end{align*}
	como queríamos. \qedsymbol
\end{proof*}
\begin{def*}
	Um \textbf{espaço de Hilbert}, H, é um espaço de produto interno completo com respeito à métrica induzida pela norma, \(d(x, y) = \Vert x-y \Vert.\quad \square\)
\end{def*}
\begin{example}
	Seja \(\mu \) uma medida positiva em um conjunto X e tome \(H = L^{2}(\mu )\). Defina
	\[
		\left< f, g \right> = \int_{}^{}f \overline{g} d\mu .
	\]
	Então, H é um espaço de Hilbert com este produto interno.
	Se \(\mu \) for uma medida de contagem nos número naturais, então \(L^{2}(\mu )\) é um espaço levemente diferente, denotado por \(\ell ^{2}.\) Um elemento em \(\ell ^{2}\) é uma sequência \(a = (a_1, a_2, \dotsc )\)
	tal que \(\sum\limits_{n=1}^{\infty}|a_{n}|^{2} < \infty\) e que, se \(b = (b_1, b_2, \dotsc ),\) então
	\[
		\left< a,  b \right> = \sum\limits_{n=1}^{\infty}a_{n}\overline{b_{n}}.
	\]
	Além destes dois, podemos pedir que \(\mu \) seja uma medida de contagem sobre um conjunto de n elementos, o que resultaria em H ser o espaço euclidiano de dimensão n.
\end{example}
\begin{prop*}
	Seja \(y\in H\) fixo. Então, as funções \(x\mapsto \left< x, y \right>\) e \(x\mapsto \Vert x \Vert\) são contínuas.
\end{prop*}
\begin{proof*}
	Pela \hyperlink{cauchy_schwarz}{\textit{Desigualdade de Cauchy-Schwarz}},
	\[
		|\left< x, y \right> - \left< x', y \right>| = |\left< x -x', y \right>|\leq \Vert x - x' \Vert\Vert y \Vert,
	\]
	provando a continuidade de \(x\mapsto \left< x, y \right>\). Agora, pela desigualdade triangular,
	\[
		\Vert x \Vert \leq \Vert x-x' \Vert + \Vert x' \Vert \Longleftrightarrow \Vert x \Vert - \Vert x' \Vert \leq \Vert x-x' \Vert.
	\]
	Com isso,
	\[
		|\Vert x \Vert - \Vert x' \Vert| \leq \Vert x-x' \Vert.
	\]
	Portanto, a função \(x\mapsto \Vert x \Vert\) é contínua. \qedsymbol
\end{proof*}
\begin{def*}
	Um subconjunto M de um espaço vetorial é chamado de \textbf{subespaço} se M em si for um espaço vetorial com as mesmas operações de soma e multiplicação por escalar. Um
	\textbf{subespaço fechado} é um subespaço que é, também, um conjunto fechado com respeito à métrica induzida pelo produto interno \(\left< \cdot , \cdot  \right>.\quad \square\)
\end{def*}
Nem todo subespaço é fechado. Vejamos isto no exemplo a seguir
\begin{example}
	Considere \(\ell ^{2}\) e seja M a coleção de sequência que é nula para todos os índices, menos uma quantidade finita deles. Seja \(x_{n} = \biggl(1, \frac{1}{2}, \dotsc , \frac{1}{n}, 0, 0, \dotsc \biggr)\) e \(x =
	\biggl(1, \frac{1}{2}, \frac{1}{3}, \dotsc \biggr)\) dois elementos de \(\ell ^{2}\). Então, cada \(x_{n}\in M\), mas x não pertence a M, pois ele possui infinitos termos não nulos. Consequentemente, M não é fechado, pois
	\[
		\lim_{n\to \infty}\Vert x_{n} - x \Vert^{2} = \lim_{n\to \infty} \sum\limits_{j=n+1}^{\infty}\frac{1}{j^{2}} = 0.
	\]
\end{example}
Outra propriedade interessante das normas induzidas por produtos internos e que está presente nos espaços de Hilbert é a \hypertarget{parallelogram}{\textit{regra do paralelogramo}}
\[
	\Vert x+y \Vert^{2} + \Vert x-y \Vert^{2} = 2\Vert x \Vert^{2} + 2\Vert y \Vert^{2}.
\]
\begin{def*}
	Um conjunto \(E\subseteq H\) é dito \textbf{convexo} se \(\lambda x + (1-\lambda )y\in E\) para quaisquer \(0\leq \lambda \leq 1\) e \(x, y\in E.\quad \square\)
\end{def*}
\begin{prop*}
	Todos os conjuntos não-vazios, fechados e convexos E de H possuem um elemento cuja norma é a menor de todas.
\end{prop*}
\begin{proof*}
	Seja \(\delta = \inf_{}\{\Vert x \Vert: x\in E\}\). Se \(x, y\in E\), podemos dividir a expressão da regra do paralelogramo por 4 para obter
	\[
		\frac{1}{4}\Vert x-y \Vert^{2} = \frac{1}{2}\Vert x \Vert^{2} + \frac{1}{2}\Vert y \Vert^{2} - \biggl\Vert \frac{x+y}{2}\biggr\Vert^{2}.
	\]
	Por convexidade de E, dados \(x, y \in E\), \((x+y)/2\) sempre pertencerá ao conjunto E. Sendo assim,
	\[
		\Vert x-y \Vert^{2} \leq 2\Vert x \Vert^{2} + 2\Vert y \Vert^{2} - 4\delta ^{2}.
	\]
	Agora, escolhemos \(y_{n}\in E\) tal que \(\Vert y_{n} \Vert\to \delta .\) Consequentemente,
	\[
		\Vert y_{n} - y_{m} \Vert^{2} \leq 2\Vert y_{n} \Vert^{2} + 2\Vert y_{m} \Vert^{2} - 4\delta ^{2},
	\]
	mas o lado direito tende a 0 conforme m e n tendem a infinito. Isto implica que \(y_{n}\) é uma sequência de Cauchy e, por completude de H, converge para algum y dentro de H.
	Como \(y_{n}\in E\) e E é fechado, \(y\in E\) e, por continuidade da norma, \(\Vert y \Vert = \lim_{n\to \infty}\Vert y_{n} \Vert = \delta .\)

	Quanto à unicidade, dado outro ponto \(y'\) com \(\Vert y' \Vert = \delta ,\) então \(\Vert y - y' \Vert = 0\) e, portanto, \(y = y'\). \qedsymbol
\end{proof*}
\begin{def*}
	Diremos que x é \textbf{ortogonal} a y se \(\left< x,  y \right> = 0.\) Denotamos o conjunto de todos os elementos y ortogonais a um elemento x por
	\[
		x^{\perp }\coloneqq \{y\in X: \left< x, y \right> = 0\}.
	\]
	O \textbf{complemento ortogonal} de M, denotado por \(M^{\perp },\) é definido como
	\[
		M^{\perp } =  \{y\in X: \left< y, x \right> = 0 \forall x\in M\}.
	\]
\end{def*}
Como \(x^{\perp } = f^{-1}(\{0\}),\) em que \(f(x) = \left< x, y \right>\) é contínua, vale que \(x^{\perp }\) é um subespaço fechado de H. Além disso, \(M^{\perp }\) também é um subespaço fechado, já que
\[
	M^{\perp } = \bigcap_{x\in M}^{}x^{\perp }.
\]
Observe que, se \(z\in M\cap M^{\perp }\), então
\[
	\Vert z \Vert^{2} = \left< z, z \right> = 0 \Rightarrow z = 0.
\]
\begin{lemma*}
	Seja M um subespaço fechado de H com \(M\neq H\). Então, \(M^{\perp }\) contém um elemento não-nulo.
\end{lemma*}
\begin{proof*}
	Escolha \(x\in H\) com \(x\not\in M\). Seja \(E = \{w - x: w\in M\},\) o qual é convexo e fechado como subconjunto de H. Por isso, sabe-se que existe um elemento \(y\in E\) cuja norma
	é a menor de todas. Observe que \(y+x\in M\), donde concluímos que \(y\neq 0\) já que \(x\not\in M\). Com isto, o próximo passo é mostrar que, dado \(w\in M\), \(\left< w, y \right> = 0\), o que
	significa que \(y\in M^{\perp }\).

	Isto é óbvio caso \(w = 0\), então podemos supro que \(w\neq 0.\) Sabe-se que \(y+x\in M\), tal que, dado qualquer \(t\in \mathbb{R},\) temos \(tw + (y+x)\in M\). Assim,
	\[
		tw + (y+x) - x = tw + y\in E
	\]
	por definição de E. Como y é o elemento de E com menor norma,
	\begin{align*}
		\left< y, y \right> = \Vert y \Vert^{2} & \leq \Vert tw+y \Vert^{2}                                                             \\
		                                        & = \left< tw + y, tw + y \right>                                                       \\
		                                        & = t^{2}\left< w, w \right> + 2t \mathrm{Re}\left< w, y \right> + \left< y, y \right>,
	\end{align*}
	o que implica que, para todo t real,
	\[
		t^{2}\left< w, w \right> + 2t \mathrm{Re}\left< w, y \right>\geq 0.
	\]
	Em particular, tome \(t = -\mathrm{Re}\biggl(\frac{\left< w, y \right>}{\left< w, w \right>}\biggr)\). Chegamos em
	\[
		-\frac{|\mathrm{Re}\left< w, y \right>|^{2}}{\left< w, w \right>}\geq 0,
	\]
	donde segue que \(\mathrm{Re}\left< w, y \right> = 0\).

	Resta mostrar que o mesmo vale para números imaginários. Sendo assim, como \(w\in M\), \(iw\in M\) e, repetindo o mesmo argumento para iw no lugar de w, temos \(\mathrm{Re}\left< iw, y \right> = 0,\) de forma que
	\[
		\mathrm{Im}\left< w, y \right> = - \mathrm{Re}(i \left< w, y \right>) = -\mathrm{Re}\left< iw, y \right> = 0.
	\]
	Portanto, \(\left< w, y \right> = 0\). \qedsymbol
\end{proof*}
Em particular, a prova deste resultado fornece uma ferramenta que permite-nos escrever cada elemento de H unicamente como a soma de um elemento de M e um de \(M^{\perp }\). Sendo mais específico, colocando
\[
	Px = y + x\quad\&\quad Qx = -y
\]
na prova feita, então \(Px\in M\) e \(Qx\in M^{\perp }\), tal que \(x = Px + Qx\). Os termos \(Px\) e \(Qx\) são conhecidos como \textbf{projeções ortogonais} de x sobre M e \(M^{\perp }\), respectivamente. Para ver a unicidade, se
\[
	x = z + z',\quad z\in M,\quad z'\in M^{\perp },
\]
então \(Px - z = z' - Qx\) é um elemento tanto de M quanto de \(M^{\perp },\) o que significa que ambos os lados da igualdade são nulos. Logo, \(z = Px\) e \(z'=Qx.\)
\begin{theorem*}
	Se L é um funcional linear limitado em H, então existe um único y em H tal que \(Lx = \left< x, y \right>.\)
\end{theorem*}
\begin{proof*}
	Supondo que tenhamos provado a existência, a unicidade segue automaticamente. Se \(Lx = \left< x, y \right> = \left< x, y' \right>\), então \(\left< x, y - y' \right> = 0\) para todos os x. Em particular, até mesmo para
	\(x = y - y'.\)

	Com relação à existência, se \(Lx = 0\) para todo x, fazemos \(y = 0\). Caso contrário, tome
	\[
		M = \{x: Lx = 0\},
	\]
	\(z\neq 0\) em \(M^{\perp }\) e \(y = \alpha z\), em que \(\alpha  = \frac{\overline{Lz}}{\left< z, z \right>}.\) Note que \(y\in M^{\perp }\),
	\[
		Ly = \frac{\overline{Lz}}{\left< z, z \right>}Lz = \frac{|Lz|^{2}}{\left< z, z \right>} = \left< y, y \right>
	\]
	e que \(y\neq 0.\) Se \(x\in H\) e
	\[
		w = x - \frac{Lx}{\left< y, y \right>}y,
	\]
	então \(Lw = 0\), o que significa que \(w\in M\) e, além disso, \(\left< w, y \right> = 0\). Portanto,
	\[
		\left< x, y \right> = \left< x - w, y \right> = Lx. \quad \text{\qedsymbol}
	\]
\end{proof*}
\begin{def*}
	Um subconjunto \(\{u_{\alpha }\}_{\alpha \in A}\) de H é \textbf{ortonormal} se \(\Vert u_{\alpha } \Vert = 1\) para todo \(\alpha \) e \(\left< u_{\alpha }, u_{\beta } \right> = 0\) sempre que \(\alpha , \beta \in A\) e que eles são escalares distintos. \(\square\)
\end{def*}
Vale mencionar que o processo de ortonormalização de Gram-Schmidt vale, também, em dimensão infinita. Com efeito, suponha que \(\{x_{n}\}_{n=1}^{\infty}\) é uma sequência linearmente indepedente, o que quer dizer que não há combinação linear finita de \(x_{n}\)'s que vala 0. Coloque
\(u_1 = \frac{x_1}{\Vert x_1 \Vert}\) e defina, indutivamente,
\[
	v_{N} = x_{N} - \sum\limits_{i=1}^{n-1}\left< x_{N}, u_{i} \right>u_{i},\quad u_{N} = \frac{v_{N}}{\Vert v_{N} \Vert}.
\]
Deste modo, temos \(\left< v_{N}, u_{i} \right> = 0\) se \(i < N\). Portanto, \(u_1, \dotsc , u_{N}\) são ortonormais.

Uma propriedade boa que conjuntos ortonormais possuem é que, para qualquer elemento x do espaço de Hilbert, apenas uma quantidade finita de termos é relevante na soma
\[
	\sum\limits_{\alpha \in A}^{}|\left< x, u_{\alpha } \right>|^{2}\leq \Vert x \Vert^{2}.
\]
Esta desigualdade recebe o nome especial de \hypertarget{bessel_inequality}{\textit{Desigualdade de Bessel}}. Formalmente,
\begin{prop*}
	Se \(\{u_{\alpha }\}_{\alpha \in A}\) é um conjunto ortonormal, então, para cada x em H,
	\[
		\sum\limits_{\alpha \in A}^{}|\left< x, u_{\alpha } \right>|^{2}\leq \Vert x \Vert^{2}.
	\]
\end{prop*}
\begin{proof*}
	Seja F um subconjunto finito de A e
	\[
		y = \sum\limits_{\alpha \in F}^{}\left< x, u_{\alpha } \right>u_{\alpha }.
	\]
	Então,
	\[
		0 \leq \Vert x-y \Vert^{2} = \Vert x \Vert^{2}-\left< x, y \right> - \left< y, x \right> + \Vert y \Vert^{2}.
	\]
	Em particular,
	\[
		\left< y, x \right> = \biggl< \sum\limits_{\alpha\in F }^{}\left< x, u_{\alpha } \right>u_{\alpha }, x \biggr> = \sum\limits_{\alpha \in F}^{}\left< x, u_{\alpha } \right>\left< u_{\alpha }, x \right> = \sum\limits_{\alpha \in F}^{}|\left< x, u_{\alpha } \right>|^{2}.
	\]
	Como isto é um número real, \(\left< x, y \right> = \left< y, x \right>\). Além disso, como \(\{u_{\alpha }\}\) é um conjunto ortonormal
	\begin{align*}
		\Vert y \Vert^{2} = \left< y, y \right> & = \biggl<\sum\limits_{\alpha \in F}^{}\left< x, u_{\alpha } \right>u_{\alpha } , \sum\limits_{\beta \in F}^{}\left< x, u_{\beta } \right>u_{\beta } \biggr> \\
		                                        & = \sum\limits_{\alpha , \beta \in F}^{}\left< x, u_{\alpha } \right>\overline{\left< x, u_{\beta } \right>}\left< u_{\alpha }, u_{\beta } \right>           \\
		                                        & = \sum\limits_{\alpha \in F}^{}|\left< x, u_{\alpha } \right>|^{2}.
	\end{align*}
	Logo,
	\[
		0\leq \Vert y-x \Vert^{2} = \Vert x \Vert^{2} - \sum\limits_{\alpha \in F}^{}|\left< x, u_{\alpha } \right>|^{2}.
	\]
	Isolando os termos entre a desigualdade,
	\[
		\sum\limits_{\alpha \in F}^{}|\left< x, u_{\alpha } \right>|^{2}\leq \Vert x \Vert^{2}.
	\]
	Se N é um inteiro maior que \(n \Vert x \Vert^{2},\) então não ocorrerá \(|\left< x, u_{\alpha } \right>|^{2}> 1/n\) para mais do que N termos dos \(\alpha \)'s. Destarte,
	\(|\left< x, u_{\alpha } \right>|^{2}\neq 0\) apenas para uma quantidade enumerável de \(\alpha \). Chamando-os de \(\alpha _1, \alpha _2, \dotsc \), chegamos que
	\[
		\sum\limits_{\alpha \in A}^{}|\left< x, u_{\alpha } \right>|^{2} = \sum\limits_{j=1}^{\infty}|\left< x, u_{\alpha_{j}} \right>|^{2} = \lim_{m\to \infty}\sum\limits_{j=1}^{m} \leq \Vert x \Vert^{2}.
	\]
	Portanto,
	\[
		\sum\limits_{\alpha \in A}^{}|\left< x, u_{\alpha } \right>|^{2} \leq \Vert x \Vert^{2}.\quad \text{\qedsymbol}
	\]
\end{proof*}
\begin{prop*}
	Suponha que \(\{u_{\alpha }\}_{\alpha \in A}\) é ortonormal. São equivalentes:
	\begin{itemize}
		\item[1)] Se \(\left< x, u_{\alpha } \right> = 0\) para cada \(\alpha \in A\), então x = 0;
		\item[2)] Para todo x, \(\Vert x \Vert^{2} = \sum\limits_{\alpha \in A}^{}|\left< x, u_{\alpha } \right>|^{2}\);
		\item[3)] Para cada x em H, \(x = \sum\limits_{\alpha \in A}^{}\left< x, u_{\alpha } \right>u_{\alpha }\).
	\end{itemize}
\end{prop*}
Antes da prova, valem alguns comentários. Quando a primeira propriedade é verdadeira, diremos que o conjunto ortonormal é \textbf{completo}. O item 2 é conhecido como a \textit{identidade de Parseval}.
No terceiro, lidamos com convergência com respeito à norma de H, o que impica que apenas uma quantidade enumerável de termos da soma é não nula.
\begin{proof*}
	\(1) \Rightarrow 3): \) Tome x em H. Pela \hyperlink{bessel_inequality}{\textit{Desigualdade de Bessel}}, existe apenas uma quantidade contável de \(\alpha \)'s tais que \(|\left< x, u_{\alpha } \right>|^{2}\neq 0\). Seja
	\(\alpha_1, \alpha _2, \dotsc \) uma enumeração destes \(\alpha \)'s. Novamente, pela \hyperlink{bessel_inequality}{\textit{Desigualdade de Bessel}}, a série \(\sum\limits_{i=1}^{\infty}|\left< x, u_{\alpha_{i}} \right>|^{2}\) converge. Utilizando que \(\{u_{\alpha }\}\) é um conjunto ortonormal,
	\begin{align*}
		\biggl\Vert \sum\limits_{j=m}^{n}\left< x, u_{\alpha_{j}} \right>u_{\alpha_{j}}\biggr\Vert^{2} & = \sum\limits_{j, k = m}^{n}\left< x, u_{\alpha_{j}} \right>\overline{\left< x, u_{\alpha_{k}} \right>}\left< u_{\alpha_{j}}, u_{\alpha_{k}} \right> \\
		                                                                                               & = \sum\limits_{j=m}^{n}|\left< x, u_{\alpha_{j}} \right>|^{2}\to 0
	\end{align*}
	quando m e n tendem a infinito. Assim, \(\sum\limits_{j=1}^{n}\left< x, u_{\alpha_{j}} \right>u_{\alpha_{j}}\) é uma sequência de Cauchy, significando sua convergência. Tome \(z = \sum\limits_{j=1}^{\infty}\left< x, u_{\alpha_{j}} \right>u_{\alpha_{j}}.\) Então,
	\(\left< z-x, u_{\alpha_{j}} \right> = 0\) para todo \(\alpha_{j}\). Logo, \(z-x = 0.\)

	\(3)\Rightarrow 2):\) Segue da relação
	\[
		\Vert x \Vert^{2} - \sum\limits_{j=1}^{n}|\left< x, u_{\alpha_{j}} \right>|^{2} = \biggl\Vert x - \sum\limits_{j=1}^{n}\left< x, u_{\alpha_{j}} \right>u_{\alpha_{j}}\biggr\Vert^{2}\to 0.
	\]

	Por fim, que 2 implica 1 é automático. \qedsymbol
\end{proof*}
\begin{example}
	Tome \(H = \ell ^{2} = \{x = (x_1, x_2, \dotsc ): \sum\limits_{}^{}|x_{i}|^{2} < \infty\}\) com \(\left< x, y \right> = \sum\limits_{i}^{}x_{i}\overline{y_{i}}\). Então, \(\{e_{i}\}\) é um sistema ortonormal completa, em que cada \(e_{i}\) assume o valor 1 na i-ésima coordenada
	e 0 em todas as outras.
\end{example}
\begin{def*}
	Seja \(K\subseteq H\), em que H é um espaço de Hilbert. O conjunto das combinações lineares de finitos elementos de K é chamado de \textbf{conjunto gerado por K}.
	Uma coleção de elementos \(\{e_{\alpha }\}\) é uma \textbf{base} para H se o conjunto de combinações lineares finitas de \(e_{\alpha }\)'s for denso em H. \(\square\)
\end{def*}
Equivalentemente, uma base é um subconjunto de H cujo fecho tenha H como conjunto gerado.
\begin{prop*}
	Todo espaço de Hilbert tem uma base ortonormal.
\end{prop*}
\begin{proof*}
	Se \(B = \{u_{\alpha }\} \) for ortonormal, mas não uma base, tome V como o fecho do conjunto linearmente gerado por B, \textit{i. e.}, o fecho, com respeito à norma em H, do conjunto de combinações lineares finitas dos elementos de B.
	Assuma que \(V\neq H\) e escolha um x não-nulo em \(V^{\perp }.\) Tome \(B' = B \cup \biggl\{\frac{x}{\Vert x \Vert}\biggr\}\), tal que B' é uma base estritamente maior que B.
	Como a união de uma sequência crescente de conjunto ortonormais é, em si, um conjunto ortonormal, existe um deles que é maximal pelo \hyperlink{zornn}{\textit{Lema de Zornn}}. Pelo que já foi visto, este conjunto ortonormal maximal precisa ser uma base,
	caso contrário, conseguiríamos achar um maior, o que geraria uma contradição. \qedsymbol
\end{proof*}

\subsection{Ana Júlia G. A. - Séries de Fourier}
O sentido de estudar as séries de Fourier neste momento é a aplicação interessante que espaços de Hilbert e as técnicas que vimos até o momento têm para a teoria das séries. Elas em si são utilizadas em EDPs, filtragem, compressão de dados,
digitalização. Ao longo desta seção, tomaremos \(H = L^{2}([0, 2\pi ))\) e
\[
	u_{n} = \frac{1}{\sqrt[]{2\pi }}e^{inx},\quad n\in \mathbb{Z}.
\]
Faremos o produto interno ser
\[
	\left< f, g \right> = \int_{0}^{2\pi }f(x)\overline{g(x)}dx,\quad \Vert f \Vert^{2} = \int_{0}^{2\pi }|f(x)|^{2}dx.
\]
Em particular, para m e n distintos,
\[
	\left< u_{n}, u_{m} \right> = \int_{0}^{2\pi }e^{inx}e^{-imx}dx = \int_{0}^{2\pi }e^{i(n-m)x}dx = 0
\]
e, para m = n,
\[
	\left< u_{n}, u_{n} \right> = \int_{0}^{2\pi }e^{i(n-n)x}dx = 2\pi
\]
\begin{def*}
	Um \textbf{polinômio trigonométrico} é uma combinação linear finita dos termos \(u_{n}.\quad \square\)
\end{def*}
Se \(\mathcal{F}\) denota o conjunto gerado pelos \(\{u_{n}\}\), então nosso objetivo será mostrar que \(\mathcal{F}\) é denso em \(L^{2}([0, 2\pi )).\)

Seja \(\varphi_{n}(x) = c_{n}(1+\cos^{}{(x)})^{n},\) em que \(c_{n}\) é escolhido tal que
\[
	\int_{-\pi }^{\pi }\varphi_{n}(x)dx = 1,\quad \varphi_{n}(x)\geq 0, x\in [-\pi, \pi ].
\]
\begin{lemma*}
	\begin{itemize}
		\item[1)] Se \(f\in L^{1},\) então \(f*\varphi_{n}\) é um polinômio trigonométrico;
		\item[2)] Se \(\delta\in (0, \pi )\), então
		      \[
			      \int_{\delta \leq |x| \leq \pi }^{}\varphi_{n}(x)dx\to 0
		      \]
		      quando \(n\to \infty\).
	\end{itemize}
\end{lemma*}
\begin{proof*}
	1) Para ver que \(\varphi_{n}\) é um polinômio trigonométrico, observe que
	\[
		\varphi_{n}(x) = c_{n}\biggl(e^{i0x} + \frac{1}{2}e^{ix} + \frac{1}{2}e^{-ix}\biggr)^{n}
	\]
	Escreva \(\varphi_{n} = \sum\limits_{k=-n}^{n}b_{k}e^{ikx}\), tal que
	\[
		f*\varphi (x) = \sum\limits_{k=-n}^{n}b_{k}\int_{}^{}e^{ik(x-y)}f(y)dy = \sum\limits_{k=-n}^{n}a_{k}b_{k}e^{ikx},
	\]
	em que pedimos que \(a_{k} = \int_{}^{}e^{-iky}f(y)dy\). Isto conclui a prova do primeiro item, pois expressamos \(f*\varphi \) como um polinômio trigonométrico.

	Para o item 2, sejam
	\[
		\alpha_{n} = \int_{-\delta }^{\delta }\varphi_{n}(x)dx \quad\&\quad  \beta_{n} = \int_{\delta \leq |x|\leq \pi }^{}\varphi_{n}(x)dx.
	\]
	Dentro de \([-\delta , \delta ]\), a função \(1 + \cos^{}{(x)}\) assume seu máximo em \(|x| = \delta \). Por outro lado, em \(\biggl[-\frac{\delta }{2}, \frac{\delta }{2}\biggr]\), assume
	seu mínimo em \(|x| = \frac{\delta }{2}.\) Como \(\alpha_{n} \geq \int_{-\frac{\delta }{2}}^{\frac{\delta }{2}}dx,\) segue que
	\begin{align*}
		\frac{\beta_{n}}{\alpha_{n}} & \leq \frac{c_{n}2\pi (1+\cos^{}{(\delta )})^{n}}{c_{n}\delta (1+\cos^{}{\biggl(\frac{\delta }{2}\biggr)})^{n}}      \\
		                             & = \frac{2\pi }{\delta }\biggl(\frac{1+\cos^{}{(\delta )}}{1 + \cos^{}{\biggl(\frac{\delta }{2}\biggr)}}\biggr)^{n}.
	\end{align*}
	Como o lado direito tende a 0 quando \(n\to \infty\), concluímos a prova. \qedsymbol
\end{proof*}
\begin{theorem*}
	Suponha que f é contínua em \([0, 2\pi ]\) e que \(f(0) = f(2\pi ).\) Dado \(\varepsilon  > 0\), existe g em \(\mathcal{F}\) tal que
	\[
		\sup_{0 \leq x \leq 2\pi }|f(x)-g(x)| < \varepsilon .
	\]
\end{theorem*}
\begin{proof*}
	Dada tal função f, iremos periodicamente estender ela ao intervalo \([-4\pi , 4\pi ]\) e multiplicá-la por uma função contínua unitária em \([-3\pi , 3\pi ]\) e com suporte compacto. Como f em si tem suporte compacto, existe \(\delta \) tal que \(|f(x-y) - f(x)| < \frac{\varepsilon }{2}\)
	se \(|y|\leq \delta \). Seja \(M = \sup_{x}|f(x)|.\) Então, se \(|x| \leq \pi ,\) usamos que \(\int_{-\pi }^{\pi }\varphi_{n}(y)dy = 1\) e chegamos em
	\begin{align*}
		|f*\varphi_{n}(x) - f(x)| & =\biggl\vert \int_{}^{}[f(x-y)-f(x)]\varphi_{n}(y)dy \biggr\vert                                                      \\
		                          & \leq \int_{-\delta }^{\delta }|f(x-y) - f(x)|\varphi_{n}(y)dy + 2M \int_{\delta\leq |y|\leq \pi }^{}\varphi_{n}(y)dy.
	\end{align*}
	Como os dois termos somados são limitados por
	\[
		\frac{\varepsilon }{2}\int_{-\pi }^{\pi }\varphi_{n}(y)dy = \frac{\varepsilon }{2},
	\]
	chegamos em
	\[
		|f*\varphi_{n}(x) - f(x)| \leq \frac{\varepsilon }{2} + \frac{\varepsilon }{2} = \varepsilon,
	\]
	provando o teorema. \qedsymbol
\end{proof*}
\begin{theorem*}
	\(\mathcal{F}\) é denso em \(L^{2}([0, 2\pi )).\)
\end{theorem*}
\begin{proof*}
	Se \(f\in L^{2}([0, 2\pi ))\), então, pelo \hyperlink{dominated_convergence}{\textit{Teorema da Convergência Dominada}}, quando \(m\to \infty\),
	\[
		\int_{}^{}|f-f\chi_{[1/m, 2\pi -1/m]}|^{2}\to 0.
	\]
	Sabe-se que qualquer função em \(L^{2}([1/m, 2\pi - 1/m])\) pode ser aproximada em \(L^{2}\) por funções continuas com suporte no intervalo dentro do parenteses, tal que, pelo que foi
	dito acima, uma função contínua com suporte no conjunto em parêntese pode ser uniformemente aproximada em \([0, 2\pi )\) por elementos de \(\mathcal{F}\).

	Finalmente, se g é uma função contínua em \([0, 2\pi )\) e \(g_{m}\to g\) uniformemente em \([0, 2\pi )\), então \(g_{m}\to g\) em \(L^{2}([0, 2\pi ))\) por uma aplicação do \hyperlink{dominated_convergence}{\textit{Teorema da Convergência Dominada}}.
	Juntando isto tudo, portanto, segue que \(\mathcal{F}\) é denso em \(L^{2}([0, 2\pi )).\) \qedsymbol
\end{proof*}
Resta provarmos a afirmação de que \(u_{n}\) é completo. Com efeito, suponha que f é ortogonal a cada \(u_{n}\), o que, em particular, significa que ela é ortogonal a todas as combinações lineares finitas dos \(u_{n}\)'s e a todos os elementos de \(\mathcal{F}.\) Como este é denso em \(L^{2}([0, 2\pi ))\),
podemos encontrar \(f_{n}\in \mathcal{F}\) tendendo a f em \(L^{2}\), tal que
\[
	\Vert f \Vert^{2} = \left< f, f \right> = \lim_{n\to \infty}\left< f_{n}, f \right> = 0.
\]
Logo, \(\Vert f \Vert^{2} = 0 \) e f = 0, mostrando a completude de \(\{u_{n}\}\) como sistema ortonormal.

Outra propriedade interessante é que há uma forma de traduzir resultados dos \(u_{n}\)'s para séries trigonométricas de forma automática. Dado f em \(L^{2}([0, 2\pi )),\) escrevemos
\[
	c_{n} = \left< f, u_{n} \right> = \int_{0}^{2\pi }f\overline{u_{n}}dx = \frac{1}{\sqrt[]{2\pi }}\int_{0}^{2\pi }f(x)e^{-inx}dx,
\]
conhecidos como \textit{Coeficientes de Fourier de f.} Pela identidade de Parseval,
\[
	\Vert f \Vert^{2} = \sum\limits_{n}^{}|c_{n}|^{2}.
\]
Observe que, também, para qualquer f em \(L^{2},\) temos
\[
	\lim_{N\to \infty}\sum\limits_{|n|\leq N}^{}c_{n}u_{n} = f \Longleftrightarrow \biggl\Vert f - \sum\limits_{|n| \leq N}^{}c_{n}u_{n}\biggr\Vert\to 0, N\to \infty.
\]
Agora, usando que \(e^{inx} = \cos^{}{(nx)} + i \sin^{}{(nx)}\), temos
\[
	\sum\limits_{n=-\infty}^{\infty}c_{n}e^{inx} = A_{0} + \sum\limits_{n=1}^{\infty}B_{n}\cos^{}{(nx)} + \sum\limits_{n=1}^{\infty}C_{n}\sin^{}{(nx)},
\]
em que pedimos que \(A_{0} = c_{0}, B_{n} = c_{n} + c_{-n}\) e \(C_{n} = i(c_{n} - c_{-n}).\) Para mostra que o oposto também é verdadeiro, utilizamos que
\(\cos^{}{(nx)} = \frac{(e^{inx} + e^{-inx})}{2}\) e \(\sin^{}{(nx)} = \frac{(e^{inx} - e^{-inx})}{2i},\) tal que
\[
	A_{0} + \sum\limits_{n=1}^{\infty}B_{n}\cos^{}{(nx)} + \sum\limits_{n=1}^{\infty}C_{n}\sin^{}{(nx)} = \sum\limits_{n=-\infty}^{\infty}c_{n}e^{inx},
\]
no qual pedimos, desta vez, que \(c_{0} = A_{0}, c_{n} = \frac{B_{n}}{2} + \frac{C_{n}}{2i}\) para \(n > 0\) e \(c_{n} = \frac{B_{n}}{2} - \frac{C_{n}}{2i}\) para \(n < 0\), assim provando a possibilidade de equivalência dos resultados, como desejado.

No próximo passo do estudo das séries de Fourier, veremos como elas podem ser usadas para aprender coisas sobre as funções nas quais elas são baseadas - em outras palavras, provaremos que
a convergência \(S_{N}f\to f\) em \(L^{2}\) quando \(N\to \infty\) ocorre, também, pontualmente, \textit{i.e.}, \(S_{N}f(x)\to f(x)\) quando \(N\to \infty\) para todo x. Aqui, escrevemos
\[
	S_{N}f(x) = \frac{1}{\sqrt[]{2\pi }}\sum\limits_{|n|\leq N}^{}c_{n}e^{inx},\quad c_{n} = \frac{1}{\sqrt[]{2\pi }}\int_{0}^{2\pi }f(y)e^{-iny}dy.
\]
Isto conta como um complemento ao teorema principal da seção, pois mostramos que as funções contínuas que possuem mesmo valor em 0 e em \(2\pi \) podem ser aproximadas por polinômios trigonométricos, mas, aqui, mostraremos que eles são explicitamente as somas parciais
\(S_{N}f\). A primeira etapa é o Lema de Riemann-Lebesgue, enunciado e provado a seguir
\hypertarget{riemann_lebesgue}{
	\begin{lemma*}
		Se \(f\in L^{1}([0, 2\pi ))\), então
		\[
			\int_{0}^{2\pi }f(x)e^{-irx}dx\to 0
		\]
		quando \(|r|\to \infty.\)
	\end{lemma*}}
\begin{proof*}
	Se \(f(x) = \chi_{[a, b)]}(x)\) com \(0 \leq a < b \leq 2\pi \), então
	\[
		\int_{0}^{2\pi }f(x)e^{-irx}dx = \frac{e^{-ira} - e^{-irb}}{ir}\to 0,\quad |r|\to \infty
	\]
	e temos o resultado provado para funções características. Por linearidade, temos o resultado, também, para quando f for uma função degrau.
	Resta para os casos de f ser positiva e f ser uma função qualquer.

	Para tanto, suponha que \(f\in L^{1} \) e tome \(\varepsilon > 0\). Escreva
	\[
		\int_{0}^{2\pi }f(x)e^{-irx}dx = \int_{0}^{2\pi }(f(x)-g(x))e^{-irx}dx + \int_{0}^{2\pi }g(x)e^{-irx}dx,
	\]
	em que g é uma função degrau tal que \(\Vert f-g \Vert_{1} < \varepsilon .\) O primeiro termo à direita é limitado por \(\Vert f - g \Vert_{1} < \varepsilon \), enquanto que o segundo tende a 0 quando fazemos r
	tender a infinito em módulo. Logo,
	\[
		\limsup_{|r|\to \infty}\biggl\vert \int_{0}^{2\pi }f(x)e^{-irx}dx \biggr\vert \leq \varepsilon .
	\]
	Portanto, como \(\varepsilon > 0\) foi escolhido de forma arbitrária, completamos a prova. \qedsymbol
\end{proof*}
Com relação à convergência de \(S_{N}f(x)\), primeiramente note que
\[
	S_{N}f(x) = \sum\limits_{|n|\leq N}^{}\frac{1}{2\pi }e^{inx}\int_{0}^{2\pi }f(y)e^{-iny}dy = \frac{1}{2\pi }\int_{0}^{2\pi }f(y)\sum\limits_{|n|\leq N}^{}e^{in(x-y)}dy.
\]
Colocando \(D_{N}(t) = \frac{1}{2\pi }\sum\limits_{|n|\leq N}^{}e^{int},\) temos
\[
	S_{N}f(x) = \int_{0}^{2\pi }f(y)D_{N}(x-y)dy.
\]
Temos uma forma explícita para esta função \(D_{N}(t)\), a qual é conhecida como \textbf{Núcleo de Dirichlet}, ou \textbf{Kernel de Dirichlet}. Para encontrá-la, usamos as fórmulas para a série geométrica
e para a expressão de seno em termos da exponencial complexa. Assim, obtemos
\begin{align*}
	D_{N}(t) = \frac{1}{2\pi }\sum\limits_{|n|\leq N}^{}e^{int} & = \frac{1}{2\pi }e^{-iNt}\sum\limits_{n=0}^{2N}e^{int}                                                                                                  \\
	                                                            & = \frac{1}{2\pi }e^{-iNt}\frac{e^{i(2N+1)t}-1}{e^{it}-1}                                                                                                \\
	                                                            & = \frac{1}{2\pi }\frac{e^{i \frac{t}{2}}}{e^{i \frac{t}{2}}}\frac{e^{i(N+\frac{1}{2})t}-e^{-i(N+\frac{1}{2})t}}{e^{i \frac{t}{2}} - e^{-i \frac{t}{2}}} \\
	                                                            & = \frac{1}{2\pi }\frac{\sin^{}{\biggl(\biggl(N+\frac{1}{2}\biggr)t\biggr)}}{\sin^{}{\biggl(\frac{t}{2}\biggr)}}.
\end{align*}
Como cada exponencial tem período de \(2\pi \), este será o período de \(D_{N}\) também. Caso f seja periódica e com período de \(2\pi \), uma substituição de \(y = x - z\) fornece
\begin{align*}
	S_{N}f(x) & = \int_{0}^{2\pi }f(z)D_{N}(x-z)dz          \\
	          & = \int_{-\pi + x}^{\pi + x}f(z)D_{N}(x-z)dz \\
	          & = - \int_{\pi }^{-\pi }f(x-y)D_{N}(y)dy     \\
	          & = \int_{-\pi }^{\pi }f(x-y)D_{N}(y)dy.
\end{align*}
Fazendo com que \(f_{0}\) seja a função identicamente igual a 1, então todos os coeficientes de \(f_{0}\) são 0, menos o primeiro, \(c_{0}\), que precisará valer \(\sqrt[]{2\pi },\) tal que \(S_{N}f_{0}(0) = 1.\)
Com isso, trocando f por \(f_{0}\), obtemos
\[
	1 = \int_{-\pi }^{\pi }D_{N}(y)dy.
\]
Com manipulações algébricas, chegamos na expressão
\[
	S_{N}f(x) - f(x) = \int_{-\pi }^{\pi }[f(x-y) - f(x)]D_{N}(y)dy.
\]
\begin{theorem*}
	Suponha que f é limitada e mensurável em \([0, 2\pi ).\) Por periodicidade, é possível estender o domínio de f para toda a reta \(\mathbb{R}\). Além disso, fixado x
	em \([0, 2\pi )\) e assumindo que exista \(c_1 > 0\) satisfazendo, para todo h,
	\[
		|f(x+h) - f(x)|\leq c_{1}|h|,
	\]
	então
	\[
		\lim_{N\to \infty}S_{N}f(x) = f(x).
	\]
\end{theorem*}
\begin{proof*}
	ome
	\[
		g(y) = \frac{1}{2\pi }\frac{f(x-y)-f(x)}{\sin^{}{\biggl(\frac{y}{2}\biggr)}}.
	\]
	Para y em \([-\pi , \pi ],\) sabemos que \(\biggl\vert \sin^{}{\biggl(\frac{y}{2}\biggr)} \biggr\vert \geq \frac{2}{\pi }\frac{y}{2},\) de maneira que
	\[
		|g(y)|\leq \frac{1}{2\pi }\frac{c_y|y|}{|y|/\pi } = \frac{c_{1}}{2}
	\]
	é um valor absoluto limitado por uma constante. Finalmente, como
	\[
		S_{N}f(x) - f(x) = \frac{1}{2i}\int_{-\pi }^{\pi }g(y)(e^{i(N+\frac{1}{2})y} - e^{-i(N+\frac{1}{2})y})dy,
	\]
	segue do \hyperlink{riemann_lebesgue}{\textit{Lema de Riemann-Lebesgue}} que \(\lim_{N\to \infty}S_{N}f(x) - f(x)\ = 0.\) \qedsymbol
\end{proof*}
\end{document}

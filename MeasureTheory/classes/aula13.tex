\documentclass[MeasureTheory/measure_theory.tex]{subfiles}
\begin{document}
\section{Aula 13 - 01/02/2024}
\subsection{Motivações}
\begin{itemize}
	\item Espaços de Hilbert;
	\item Série de Fourier.
\end{itemize}
\subsection{Laura V. B. - Espaços de Hilbert}
\begin{def*}
	Seja H um espaço vetorial sobre \(\mathbb{F}\in\{\mathbb{C}, \mathbb{R}\}\). Dizemos que H é um \textbf{espaço de produto interno} se existir um mapa
	\(\left< \cdot , \cdot  \right>:H \times H\rightarrow \mathbb{F}\) tal que
	\begin{itemize}
		\item[1)] \(\left< y, x \right> = \overline{\left< x, y \right>}\) para todos x, y em H;
		\item[2)] \(\left< x + y, z \right> = \left< x, z \right> + \left< y, z \right>\) para todos x, y, z em H;
		\item[3)] \(\left< \alpha x, y \right> = \alpha \left< x, y \right>\) para todos \(x, y\in H\) e \(\alpha \in \mathbb{F};\)
		\item[4)] \(\left< x, x \right> \geq 0\) para todo x em H;
		\item[5)] \(\left< x, x \right> = 0\) se, e somente se, \(x=0.\quad \square\)
	\end{itemize}
\end{def*}
Definimos \(\Vert x \Vert = \left< x, x \right>^{\frac{1}{2}},\) tal que \(\left< x, x \right> = \Vert x \Vert^{2}.\) Pelas definições dadas, \(\left< 0, y \right> = 0\) e
\(\left< x, \alpha y \right> = \overline{\alpha }\left< x, y \right>.\)

\hypertarget{cauchy_schwarz}{
	\begin{theorem*}[Desigualdade de Cauchy Schwarz]
		Para todos \(x, y\in H\), temos
		\[
			|\left< x, y \right>|\leq \Vert x \Vert \Vert y \Vert.
		\]
	\end{theorem*}}

\subsection{Ana Júlia G. A. - Séries de Fourier}
página 214 + pedir slides
\end{document}

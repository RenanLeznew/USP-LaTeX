\documentclass[measure_theory.tex]{subfiles}
\begin{document}
\section{Aula 10 - 23/01/2024}
\subsection{Motivações}
\begin{itemize}
	\item Relacionando ao Cálculo - Diferenciação;
	\item Função de Variação Limitada.
\end{itemize}
\subsection{Diferenciação}
Trabalharemos em \(\mathbb{R},\) com m sendo a medida Lebesgue e \(B(x, r) = (x-r, x+r).\)
\begin{def*}
	Define-se a \textbf{antiderivada} ou \textbf{integral indefinida} de uma função integrável por
	\[
		F(x) = \int_{a}^{x}f(t)dt.\quad \square
	\]
\end{def*}
Relembremos, também, o \textit{Teorema Fundamental do Cálculo}
\hypertarget{calculus}{\begin{theorem*}[Teorema Fundamental do Cálculo]
		Seja \(f:\mathbb{R}\rightarrow \mathbb{R}\) integrável e \(a\in \mathbb{R}.\) Definimos
		\[
			F(x) = \int_{a}^{x}f(t)dt.
		\]
		Então, F é derivável quase sempre e \(F'(x) = f(x)\) quase sempre.
	\end{theorem*}}
\begin{proof*}
	Se \(h > 0,\) temos
	\[
		F(x+h) - F(x) = \int_{x}^{x+h}f(t)dt,
	\]
	tal que
	\begin{align*}
		\biggl\vert \frac{F(x+h) - F(x)}{h} - f(x) \biggr\vert & = \frac{1}{h}\biggl\vert \int_{x}^{x+h}[f(y) - f(x)]dy \biggr\vert \\
		                                                       & \leq \frac{2}{m(B(x, h))}\int_{x-h}^{x+h}|f(y) - f(x)|dy.
	\end{align*}
	Sabemos que o lado direito vai a 0 quando h tende a 0 para quase todo x e, assim, a derivada de F à direita existe e é igual a f em quase toda parte. Analogamente, o lado esquerdo também vai a 0 quando h vai 0, ou seja, temos
	\(F'(x) = f(x)\) quase sempre. \qedsymbol
\end{proof*}
Mostraremos que funções crescentes são deriváveis quase sempre.
\begin{lemma*}
	Suponha que \(H:\mathbb{R}\rightarrow \mathbb{R}\) é crescente, contínua à direita e constante para \(x\geq 1\) e \(x\leq 0.\) Seja \(\lambda \) a medida de Lebesgue-Stieltjes definida usando H e suponha que \(\lambda \) e m são mutualmente singulares. Então,
	\[
		\lim_{r\to 0}\frac{\lambda (B(x, r))}{m(B(x, r))} = 0
	\]
	quase sempre, x com relação a m.
\end{lemma*}
\begin{proof*}
	O primeiro passo é a mensurabilidade. Seja \(\tilde{H}(y) = \lim_{z\to y^{-}}H(z).\) Então,
	\[
		\tilde{H}(y) = \lim_{z\to y^{-}}\lambda ([0, z])=\lambda ([0, y)).
	\]
	Como H e \(\tilde{H}\) são crescentes, elas são Borel mensuráveis. Note que
	\[
		\lambda (B(x, r)) = \lambda ([0, x + r)) - \lambda ([0, x - r]) = \tilde{H}(x+r) - H(x-r).
	\]
	Assim, para cada r, a função \(x\mapsto \lambda (B(x, r))\) é Borel mensurável.

	Seja \(r_{j} = 2^{-j}.\) Se \(2^{-j-1}\leq r < 2^{-j}, \) então
	\[
		\frac{\lambda (B(x, r))}{m(B(x, r))} \leq \frac{\lambda (B(x, r_{j}))}{m(b(x, r_{j+1}))} = 2 \frac{\lambda (B(x, r_{j}))}{m(B(x, r_{j}))}.
	\]
	Para provar o resultado, agora, basta provar que
	\[
		\lim_{j\to \infty}\frac{\lambda (B(x, r_{j}))}{m(B(x, r_{j}))} = 0,\quad \text{quase sempre para todo x.}
	\]

	\textbf{\underline{Afirmação}:} O resultado é claro se x é menor que 0 ou maior que 1.
	Com efeito, como \(\lambda \perp m\), existem conjuntos E e F mensuráveis tais que \(\lambda (F) = 0, m(E) = 0\) e \(F = E ^{\complement}.\) Seja \(\varepsilon > 0\). Agora, existe G aberto limitado tal que \(F\cap [0, 1]\subseteq G\) e \(\lambda (G) < \varepsilon .\)
	Consequentemente, existe G' aberto contendo F tal que \(\lambda (G') < \varepsilon .\) Como H é constante em \((-\infty, 0]\cup [1, \infty),\) tomamos G sendo \(G = G'\cap (-1, 2)\).

	Seja \(\beta  > 0\) e
	\[
		A_{\beta } = \biggl\{x\in F\cap [0, 1]: \limsup_{j\to \infty}\frac{\lambda (B(x, r_{j}))}{m(B(x, r_{j}))}\biggr\}.
	\]
	\textbf{\underline{Afirmação}:} A medida de \(A_{\beta }\) é nula. De fato, se \(x\in A_{\beta },\) então \(x\in F\subseteq G\) e existe uma bola aberta \(B_{x}\), centrada em x e de raio \(2^{-j}\) para algum j, tal que \(B_{x}\subseteq G\) e \(\frac{\lambda (B_x)}{m(B_x)}>\beta .\)
	Com isso, existe uma sequência disjunta \(B_1, B_2, \dotsc \) tal que
	\[
		m(A_{\beta }) \leq 3\sum\limits_{i=1}^{\infty}m(B_{i}).
	\]
	Então,
	\[
		m(A_{\beta }) \leq 3\sum\limits_{i=1}^{\infty}m(B_{i}) \leq \frac{3}{\beta }\sum\limits_{i=1}^{\infty}\lambda (B_{i}) \leq \frac{3}{\beta }\lambda (G) \leq \frac{3}{\beta }\varepsilon .
	\]
	Como \(\varepsilon \) é arbitrário, a construção de G não depende de \(\beta \), do que segue que \(m(A_{\beta }) = 0.\)

	Finalmente, como \(m(A_{1/k}) = 0\) para todo k, então
	\[
		m \biggl(\biggl\{x\in F\cap [0, 1]: \limsup_{j\to \infty}\frac{\lambda (B(x, r_{j}))}{m(B(x, r_{j}))} > 0\biggr\}\biggr) = 0.
	\]
	Portanto, como \(m(E) = 0\), segue o resultado. \qedsymbol
\end{proof*}
\begin{prop*}
	Seja \(F:\mathbb{R}\rightarrow \mathbb{R}\) crescente e contínua à direita. Então, F' existe quase sempre. Além disso, F' é localmente integrável e, para todo \(a< b\),
	\[
		\int_{a}^{b}F'(x)dx \leq F(b) - F(a).
	\]
\end{prop*}
\begin{proof*}
	Provaremos que F é diferenciável quase sempre em [0, 1], o que pode ser generalizado pelo mesmo argumento para \([-N, N]\) para cada N, ou seja, será válido para \(\mathbb{R}.\)

	Redefina F pondo
	\[
		F(x) = \left\{\begin{array}{ll}
			\lim_{y\to 0^{+}}F(y),\quad x \leq 0 \\
			F(1),\quad x > 1
		\end{array}\right..
	\]
	Então, F ainda é contínua à direita e crescente, não afetando a diferenciabilidade em [0, 1] exceto possivelmente no extremos.

	Seja \(\nu \) a medida de Lebesgue-Stieltjes definida em termos de F. Da decomposição de Lebesgue, podemos escrever \(\nu = \lambda +\rho \), em que \(\lambda \perp m\) e \(\rho << m.\) Note que
	\[
		\rho ([0, 1]) \leq \nu ([0,1]) = F(1) - F(0).
	\]
	Do \hyperlink{radon_nikodym}{\textit{Teorema de Radon-Nikodym}}, existe uma função \(f\geq 0\) integrável tal qeu \(\rho (A) = \int_{A}^{}fdm\) para cada A mensurável. Seja
	\[
		H(x) = \lambda ((0, x]) = \nu ((0, x]) - \rho ((0, x]) = F(x) - F(0) - \int_{0}^{x}f(y)dy.
	\]
	Mas, pode-se provar que a função \(x\mapsto \int_{0}^{x}f(y)dy\) é contínua, o que torna H contínua à direita, crescente e \(\lambda \) é a medida de Lebesgue-Stieltjes definida em termos de H. Com isso,
	\begin{align*}
		\limsup_{h\to 0^{+}}\frac{H(x+h) - H(x)}{h} & \leq \limsup_{h\to 0^{+}}\frac{H(x+h) - H(x-h)}{h}                                         \\
		                                            & =\limsup_{h\to 0^{+}}\frac{\lambda ((x-h), (x+h)])}{h}                                     \\
		                                            & \leq 4\limsup_{h\to 0^{+}}\frac{\lambda (B(x, 2h))}{4h} = 0,\quad \text{q.s. para todo x.}
	\end{align*}
	O mesmo é verdadeiro para a derivada à esquerda. Assim, \(H'\) existe e é igual a 0 quase sempre para x. Vimos que a função \(x\mapsto \int_{a}^{x}f(y)dy\) é derivável quase sempre e concluímos que F é derivável quase sempre. Além disso, \(F'=f\) quase sempre. Para finalizar, se
	\(a < b\),
	\[
		\int_{a}^{b}F'(x)dx = \int_{a}^{b}f(x)dx = \rho ((a, b])\leq \nu ((a, b]) = F(b) - F(a).
	\]
	Portanto, a prova está completa. \qedsymbol
\end{proof*}
Podemos finalmente enunciar o resultado principal sobre diferenciabilidade de função crescente.
\begin{theorem*}
	Se \(F:\mathbb{R}\rightarrow \mathbb{R}\) é crescente, então \(F'\) existe quase sempre e
	\[
		\int_{a}^{b}F'(x)dx \leq F(b) - F(a)
	\]
	sempre que \(a < b.\)
\end{theorem*}
\begin{proof*}
	Seja \(G(x) = \lim_{y\to x^{+}}F(y).\) Como F é crescente, existe no máximo um número enumerável de pontos x em que F não é contínua e, assim, F(x) = G(x) quase sempre.
	Como G é crescente e contínua à direita, G é derivável quase sempre.

	Mostraremos, então, que se x é um ponto em que G é derivável e \(F(x) = G(x)\), então \(F'(x)\) existe e é igual a \(G'(x).\)

	Seja x um tal ponto e sejam \(L = G'(x)\), \(\varepsilon > 0.\) Sendo F e G crescentes, para todo \(h > 0\) existe ponto \(x_{h}\in (x+h, x + (1+\varepsilon )h\) no qual F e G coincidem e, assim,
	\[
		F(x+h) \leq F(x_{h}) = G(x_{h}) \leq G(x+(1+\varepsilon )h).
	\]
	Então,
	\begin{align*}
		\limsup_{h\to 0^{+}}\frac{F(x+h) - F(x)}{h} & \leq \limsup_{h\to 0^{+}}\frac{G(x+(1+\varepsilon )h)-G(x)}{h}                               \\
		                                            & = (1+\varepsilon )\limsup_{h\to 0^{+}}\frac{G(x+(1+\varepsilon )h - G(x)}{(1+\varepsilon )h} \\
		                                            & = (1+\varepsilon )L.
	\end{align*}
	Analogamente, \(\liminf_{h\to 0^{+}}\frac{F(x+h)-F(x)}{h}\geq (1+\varepsilon )L.\) Sendo \(\varepsilon \) arbitrário, a derivada à direita de F existe no ponto x e é igual a L. Similarmente, a derivada à esquerda é L.

	Como \(F'=G'\) quase sempre, então F' é localmente integrável. Se \(a < b,\) tome \(a_{n}\downarrow a\) e \(b_{n}\uparrow b\), de modo que F e G coincidem em \(a_{n}\) e \(b_{n}\). Então, obtemos
	\begin{align*}
		F(b) - F(a) & \geq F(b_{n}) - F(a_{n})         \\
		            & = G(b_{n}) - G(a_{n})            \\
		            & \geq \int_{a_{n}}^{b_{n}}G'(x)dx \\
		            & = \int_{a_{n}}^{b_{n}}F'(x)dx.
	\end{align*}
	Portanto, tomando \(n\to \infty\) e pelo \hyperlink{monotone_convergence}{\textit{Teorema da Convergência Monótona}}, temos o resultado. \qedsymbol
\end{proof*}
Observe que, se F é a função de Cantor-Lebesgue, então \(F'(x) = 0\) quase sempre. De fato, isto ocorre em \(C ^{\complement}\), sendo C o conjunto de Cantor. Com isso,
\[
	1 = F(1) - F(0) > 0 = \int_{0}^{1}F'(x)dx.
\]
Isto mostra que, em geral, não temos igualdade no Teorema anterior.
\begin{def*}
	Seja \(f:[a, b]\rightarrow \mathbb{R}\). Diremos que f é de \textbf{variação limitada} em [a, b] se
	\[
		V_f[a, b] = \sup_{\mathcal{P}}\biggl\{\sum\limits_{i=1}^{k}|f(x_{i}) - f(x_{i-1})|\biggr\} < \infty,
	\]
	em que o supremo é tomado em todas as partições \(\mathcal{P}: a = x_{0} < x_1 < \dotsc <x_{k} = b\) de [a, b]. \(\square\)
\end{def*}
Se f é crescente, então
\[
	\sum\limits_{i=1}^{k} |f(x_{i}) - f(x_{i-1})| = \sum\limits_{i=1}^{k}f(x_{i}) - f(x_{i-1}) = f(x_{k}) - f(x_{0}) < \infty,
\]
mostrando que toda função crescente é de variação limitada. Além disso, se f e g são de variação limitada e \(c\in \mathbb{R},\) então \(f\pm g,\) \(cf\) e \(fg\) são de variação limitada. Também, se f for Lipschitz, então vale
\[
	\sum\limits_{i=1}^{k}|f(x_{i}) - f(x_{i-1})| \leq C \sum\limits_{i=1}^{k}|x_{i}-x_{i-1}| \leq C(b-a).
\]
Logo, f é de variação limitada. Finalmente, se f é continuamente derivável, então ela é de variação limitada.
\begin{example}
	\begin{itemize}
		\item[i)] A função \(f(x) = [x], x\in [0, 2]\), é de variação limitada e não é contínua;
		\item[ii)] Se f é de variação limitada, então f é limitada;
		\item[iii)] Seja
		      \[
			      f(x) = \left\{\begin{array}{ll}
				      x\sin^{}{\biggl(\frac{\pi }{x}\biggr)},\quad & x\in (0 ,1) \\
				      0,                                           & \quad x = 0
			      \end{array}\right..
		      \]
		      Esta f é contínua, mas não é de variação limitada, pois a série \(\sum\limits_{}^{}\frac{1}{2n+1}\) diverge. Então, a \(S_{n}\) soma parcial não é limitada. Para ver isto, considere a partição \(P = \biggl\{0, \frac{2}{2n+1}, \frac{2}{2n-1}, \dotsc , \frac{2}{5}, \frac{2}{3}, q\biggr\}.\) Segue que \(S_{n} = \sum\limits_{i=1}^{k}|f(x_{i}) - f(x_{i-1})|\) não
		      é limitada.
		\item[iv)] Seja
		      \[
			      f(x) = \left\{\begin{array}{ll}
				      1, & \quad x\in \mathbb{Q}^{\complement} \\
				      0, & \quad x\in \mathbb{Q}
			      \end{array}\right..
		      \]
		      Como cada intervalo existe um irracional, se tomarmos uma partição com n subintervalos, tomemos um irracional e um racional em cada subintervalo. Daí, \(S_{n} = 1 + 1 +\dotsc +1\geq \frac{n}{2},\) em que \(S_{n}\) é ilimitada.
	\end{itemize}
\end{example}
Toda função de variação limitada pode ser escrita como uma diferença de duas funções crescentes.
\begin{lemma*}
	Se f é de variação limitada em \([a, b]\), então f pode ser escrita como \(f= f_1 - f_2,\) em que \(f_1, f_2\) são crescentes.
\end{lemma*}
\begin{proof*}
	Defina
	\[
		f_1(y) = \sup_{\mathcal{P}}\biggl\{\sum\limits_{i=1}^{k}|f(x_{i})-f(x_{i-1})|^{+}\biggr\}
	\]
	e
	\[
		f_2(y) = \sup_{\mathcal{P}}\biggl\{\sum\limits_{i=1}^{k}|f(x_{i})-f(x_{i-1})|^{-}\biggr\}
	\]
	em que o supremos é tomado em todas as partições \(\mathcal{P}: a = x_{0} < x_1 < \dotsc < x_{k} = y\) para \(y\in [a, b].\) Com isso, ambas \(f_1, f_2\) são mensuráveis por serem crescentes.
	Como
	\[
		\sum\limits_{i=1}^{k}|f(x_{i}) - f(x_{i-1})|^{+} = \sum\limits_{i=1}^{k}|f(x_{i}) - f(x_{i-1})|^{-} + f(y) - f(a),
	\]
	tomando o supremo sobre todas as partições \(a=x_{0} < x_1 <\dotsc <x_{k} = y\) com \(y\in [a, b]\), temos
	\[
		f_1(y) = f_2(y) + f(y) - f(a).
	\]
	Portanto, notando que as duas \(f_1, f_2\) são crescentes em y e isolando f(y), segue o resultado. \qedsymbol
\end{proof*}
Utilizando os resultados até o presente momento, obtivemos que toda função de variação limitada é derivável quase sempre. No entanto, a recíproca não é verdadeira. Basta notar que
\[
	f(x) = \sin^{}{\biggl(\frac{1}{x}\biggr)},\quad x\in [0, 1]
\]
é diferenciável quase sempre, mas não é de variação limitada.
\begin{lemma*}
	Se \([c, d]\subseteq [a, b]\), então \(f_1(d) - f_1(c)\leq V_f[c, d]\), em que
	\[
		V_f[a, b] = \sup_{\mathcal{P:a_{0}=x_1 < x_2 < \dotsc <x_{k}=b}}\biggl\{\sum\limits_{i=1}^{k}|f(x_{i}) - f(x_{i-1})|\biggr\}.
	\]
	O mesmo vale se trocarmos \(f_1\) por \(f_2\).
\end{lemma*}
\begin{proof*}
	Seja \(\mathcal{P}\) a partição de \([a, d]\) dada por \(a = x_{0} < x_1 < \dotsc < x_{k} = d\) em \([a, d].\) Seja \(\mathcal{P}_{0} = \mathcal{P}\cup \{c\},\) descrita por
	\(a = x_{0} < x_1 < \dotsc < x_{j-1}\leq c < x_{j} < \dotsc <x22 = b.\) Tome \(\mathcal{P}'\) como a partição \(a = x_{0} < x_1 <\dotsc x_{j-1}\leq c\).
	e \(\mathcal{P}'': x_{j} < \dotsc < x_{n}=b\).

	Como \((r+s)^{+} \leq r^{+} + s^{+}, \) temos
	\begin{align*}
		\sum\limits_{\mathcal{P}}^{}|f(x_{i}) - f(x_{i-1})|^{+} & \leq \sum\limits_{\mathcal{P}_{0}}^{}|f(x_{i}) - f(x_{i-1})|^{+}                                                    \\
		                                                        & = \sum\limits_{\mathcal{P}'}^{}|f(x_{i}) - f(x_{i-1})|^{+}\sum\limits_{\mathcal{P}''}^{}|f(x_{i}) - f(x_{i-1})|^{+} \\
		                                                        & \leq \sum\limits_{\mathcal{P}'}^{}|f(x_{i})-f(x_{i-1})|^{+} + \sum\limits_{\mathcal{P}''}^{}|f(x_{i}) - f(x_{i-1})| \\
		                                                        & \leq f_1(c) + V_f[c, d].
	\end{align*}
	Aqui, usamos a notação \(\sum\limits_{\mathcal{P}}^{} = \sum\limits_{i=1}^{k}.\) Tomando o supremo em todas as partições \(\mathcal{P},\) temos
	\[
		f_1(d) \leq f_1(c) + V_f[c, d]
	\]
	Portanto,
	\[
		f_1(d) - f_1(c) \leq V_f[c, d].
	\]
	Para \(f_2,\), é similar. \qedsymbol
\end{proof*}
\begin{def*}
	Seja f de variação limitada. Coloque, \(f = f_1 - f_2\), em que \(f_1\) e \(f_2\) são crescentes. Então, a quantidade
	\[
		f_1(b) + f_2(b) - (f_1(a) + f_2(a))
	\]
	é chamada \textbf{variação total de f em [a, b]}. A variação total f em \([a, b]\) e \([b, c]\), então, coincide com a variação total em \([a, c]. \quad \square\)
\end{def*}
Observe que, se f é crescente em [a, b] e contínua à direita, podemos escrever \(f = f_1 - f_2\), em que \(f_1\) é contínua à direita e \(f_2(x) = \sum\limits_{a < t < x}^{}(f(t) - f(t^{-})).\) No somatório, apenas um número enumerável é não-nulo. Cada termo da série é não-negativa e as oma é finita, já que ela é limitada por \(f(x)-f(a).\) Assim, podemos
decompor qualquer função de variação limitada que seja contínua à direita.
\begin{def*}
	Seja \(f:[a, b]\rightarrow \mathbb{R}\). Dizemos que f é \textbf{absolutamente contínua em [a, b]} se dado \(\varepsilon  > 0\), existir \(\delta  > 0\) tal que
	\[
		\sum\limits_{i=1}^{k}|f(b_{i}) - f(a_{i})| < \varepsilon
	\]
	sempre que
	\[
		\sum\limits_{i=1}^{k}|b_{i} - a_{i}|  < \delta,
	\]
	em que \(\{(a_{i}, b_{i})\}\) é uma coleção finita disjunta de intervalos. \(\square\)
\end{def*}
Toda função absolutamente contínua é contínua, mas a volta não é verdade. Para isso, utilize a função de Cantor:
\begin{example}
	A função de Cantor não é absolutamente contínua pois, tomando como partição no n-ésimo passo os extremos dos \(2^{n}\) subintervalos de comprimento \((1/3)^{n},\) então nos extremos, as funções de Cantor coincidem por serem constantes. Daí, lembrando que \(f(0) = 0 \) e \(f(1) = 1,\) temos
	\[
		S_{n} f(a_1) - f(0) + f(a_2) - f(a_1) + \dotsc  1 = f(1) - f(a_{n-1}) = f(1) = 1.
	\]
	Por outro lado, o comprimento total dos intervalos omitidos vale \(2^{n}\biggl(\frac{1}{3}\biggr)^{n} = \biggl(\frac{2}{3}\biggr)^{n},\) que vai a zero. Podemos fazer menor que \(\delta ,\) mas achamos \(S_{n} \geq \varepsilon \).
\end{example}
\begin{lemma*}
	Se f é absolutamente contínua, então f é de variação limitada.
\end{lemma*}
\begin{proof*}
	Da definição com \(\varepsilon  = 1\), existe \(\delta  \) tal que
	\[
		\sum\limits_{i=1}^{k}|f(b_{i}) - f(a_{i})| < 1
	\]
	sempre que \(\sum\limits_{i=1}^{k}(b_{i} - a_{i}) < \delta \) e que \((a_{i}, b_{i})\) são intervalos disjuntos.

	Portanto, para cada j, a variação total de f em \([a + j\delta, a + (j+1)]\) é menor ou igual a 1, donde segue que a variação total de f sobre [a, b] é finita e menor. \qedsymbol
\end{proof*}
\begin{lemma*}
	Suponha f de variação limitada. Sabemos que \( f = f_1 - f_2\), em que \(f_1, f_2\) são crescentes. Se f é absolutamente contínua, então \(f_1, f_2\) também serão.
\end{lemma*}
\begin{proof*}
	Dado \(\varepsilon \), existe \(\delta \) tal que \(\sum\limits_{\ell =1}^{m}|f(B_{\ell }) - f(A_{\ell })|\) sempre que \(\sum\limits_{i=1}^{m}(b_{i}-a_{i}) < \delta .\) Precisamos provar que \(\sum\limits_{i=1}^{m}|f_1(b_{i}) - f_1(a_{i})|\leq \varepsilon ,\) com o mesmo válido para \(f_2\).
	Seja a partição de \((a_{i}, b_{i})\) dada por \(a_{i} = s_{i0} < s_{i1} < \dotsc <s_{iJ_{i}} = b_{i}.\) Então,
	\[
		\sum\limits_{i=1}^{k}\sum\limits_{j=0}^{J_{i}-1}(s_{i, j+1} - s_{ij}) = \sum\limits_{i=1}^{k}(b_{i}-a_{i})\leq \delta .
	\]
	Aplicando o primeiro parágrafo com a coleção
	\[
		\{(s_{ij}, s_{i, j+1}), i = 1, 2, \dotsc , k\quad j = 0, \dotsc , J_{i}-1\},
	\]
	temos
	\[
		\sum\limits_{i=1}^{k}\sum\limits_{j=0}^{J_{i}-1}|f(s_{i, j+1}) - f(s_{ij})|\leq \varepsilon .
	\]
	Mantendo \(a_{i}, b_{i}\) fixos e tomando o supremo sobre todas partições, temos
	\[
		\sum\limits_{i=1}^{k}V_j[a_{i}, b_{i}] \leq \varepsilon .
	\]
	Portanto, segue a conclusão para \(f_1.\) Para \(f_2\), é similar. \qedsymbol
\end{proof*}
\begin{theorem*}
	Se F é absolutamente contínua, então \(F'\) existe quase sempre, é integrável e
	\[
		\int_{a}^{b}F'(x)dx = F(b) - F(a).
	\]
\end{theorem*}
\begin{proof*}
	Basta provarmos que F é crescente e absolutamente contínua. Seja \(\nu \) a medida de Lebesgue-Stieltjes definida em termos de F. Como F é contínua,
	\[
		F(d) - F(c) = \nu ((c, d)).
	\]
	Pela definição de absolutamente contínua, fazendo \(k\to \infty\), temos, para \(\varepsilon > 0\), a existência de \(\delta \) tal que
	\[
		\sum\limits_{i=1}^{\infty}|F(b_{i}) - F(a_{i})| < \varepsilon
	\]
	sempre que \(\sum\limits_{i=1}^{\infty}(b_{i}-a_{i})<\delta \) e que \((a_{i}, b_{i})\) forem intervalos disjuntos. Como qualquer aberto G pode ser escrito como união disjunta de intervalos abertos \((a_{i}, b_{i})\), podemos reescrever reordenando já que,
	dado \(\varepsilon > 0\) e garantida a existência de \(\delta > 0\), eles são tais que
	\[
		\nu (G) = \sum\limits_{i=1}^{\infty}\nu ((a_{i}, b_{i})) = \sum\limits_{i=1}^{\infty}|F(b_{i}) - F(a_{i})| < \varepsilon
	\]
	para todo G aberto e \(m(G) < \delta .\)

	Se \(m(G) < \delta \) e A é Borel mensurável, existe aberto G, \(G\supseteq A\), tal que \(m(G) < \delta \) e, assim, \(\nu (A) \leq \nu (G) \leq \varepsilon .\) Concluímos disso que \(\nu << m.\)
	Logo, existe uma função não-negativa integrável f, pelo \hyperlink{radon_nikodym}{\textit{Teorema de Radon-Nikodym}}, tal que
	\[
		\nu (A) = \int_{A}f dm,
	\]
	para todo conjunto A mensurável. Em particular, para cada \(x\in [a, b]\),
	\[
		F(x) - F(a) = \nu ((a, x)) = \int_{a}^{x}f(y)dy.
	\]
	Portanto, fazendo \(x=b\), obtemos a existência de \(F'\) e igualdade a f quase sempre e, além disso,
	\[
		F(b) - F(a) = \int_{a}^{b}F'(y)dy,
	\]
	como desejado. \qedsymbol
\end{proof*}
\end{document}

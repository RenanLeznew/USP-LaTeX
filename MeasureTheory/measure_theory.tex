  \documentclass{article}
 \usepackage{bookmark}
 \usepackage{amsmath}
 \usepackage{amsthm}
 \usepackage{amssymb}
 \usepackage{pgfplots}
 \usepackage[utf8]{inputenc}
 \usepackage{amsfonts}
 \usepackage[margin=2.5cm]{geometry}
 \usepackage{graphicx}
 \usepackage[export]{adjustbox}
 \usepackage{fancyhdr}
 \usepackage[portuguese]{babel}
 \usepackage{hyperref}
 \usepackage{multirow}
 \usepackage{lastpage}
 \usepackage{mathtools}
\usepackage[math]{kurier}
\usepackage[T1]{fontenc}
 \setcounter{section}{0}

 \pagestyle{fancy}
 \fancyhf{}

 \pgfplotsset{compat = 1.18}
 \newtheoremstyle{proof*}
 {3pt}
 {3pt}
 {}
 {}
 {\itshape}
 {:}
 {.5em}
 {}
 \hypersetup{
     colorlinks,
     citecolor=black,
     filecolor=black,
     linkcolor=black,
     urlcolor=black
 }
 \newtheorem*{def*}{\underline{Defini\c c\~ao}}
 \newtheorem*{theorem*}{\underline{Teorema}}
 \newtheorem*{lemma*}{\underline{Lema}}
 \newtheorem*{prop*}{\underline{Proposi\c c\~ao}}
 \newtheorem{example}{\underline{Exemplo}}
 \newtheorem*{proof*}{\underline{Prova}}
 \newtheorem*{crl*}{\underline{Corolário}}
 \renewcommand\qedsymbol{$\blacksquare$}

 \rfoot{P\'agina \thepage \hspace{1pt} de \pageref{LastPage}}

\begin{document}
\begin{center}
	\vspace{1cm}
	\LARGE
	UNIVERSIDADE FEDERAL DE S\~AO CARLOS

	\vspace{1.3cm}
	\LARGE
	DEPARTAMENTO DE MATEMÁTICA - DM

	\vspace{1.7cm}
	\Large
	\textbf{TEORIA DA MEDIDA}

	\vspace{1.3cm}
	\large
	\textbf{Renan Wenzel - 11169472}

	\vspace{1.3cm}
	\large
	\textbf{Professor(a): Olímpio Hiroshi Miyagaki}

	\textbf{E-mail: olimpio@ufscar.br}

	\vspace{1.3cm}
	\today
\end{center}

\newpage
\textbf{{\Huge Disclaimer}}
\vspace{5cm}

{\huge Essas notas não possuem relação com professor algum.

	Qualquer erro é responsabilidade solene do autor.

	Caso julgue necessário, contatar:

	renan.wenzel.rw@gmail.com}
\tableofcontents

\newpage



\newpage

\section{Aula 01 - 08/01/2024}
\subsection{Motivações}
\begin{itemize}
	\item Noções Iniciais e Notação;
	\item Álgebras e \(\sigma \)-álgebras;
	\item Álgebra de Borel.
\end{itemize}
\subsection{Notações}
\begin{itemize}
	\item \(A ^{\complement} = \{x\in X: x\not\in A\}\)
	\item \(A\backslash B = A\cap B ^{\complement}\)
	\item \(A\triangle B = (A\setminus{B})\cup (B\setminus{A})\)
	\item \(A_{i}\uparrow \) se \(A_{1}\subseteq A_{2}\subseteq \dotsc \)
	\item \(A_{i}\uparrow A\) se \(A_{1}\subseteq A_{2}\subseteq \dotsc \) e \(\bigcup_{i\geq 1}^{}A_{i} = A\)
	\item \(A_{i} \downarrow \) se \(A_{1}\supseteq A_{2}\supseteq \dotsc \)
	\item \(A_{i} \downarrow A\) se \(A_{1}\supseteq A_{2}\supseteq \dotsc \) e \(\bigcap_{i\geq 1}^{}A_{i} = A\)
	\item \(x\vee y = \max(x, y)\) e \(x\wedge y = \min(x, y), x^{\pm} = (\pm x)\vee 0\)
	\item \(f(x^{\pm}) = \lim_{y\to x^{\pm}}f(y)\)
	\item \(\limsup_{y\to x}f(y) = \inf_{\delta > 0}\sup_{|x-y|<\delta }f(y)\)
\end{itemize}

Suponha que X é um espaço métrico, \(r > 0\) e \(A\subseteq X\)
\begin{itemize}
	\item \(B(x, r) = \{y\in X: d(x, y) < r\}\)
	\item \(A^{\mathrm{o}} = \{x\in X: \exists r_{x} > 0: B(x, r_{x})\subseteq A\}\)
	\item \(\overline{a} = \{x\in X: B(x, r)\cap X \neq\emptyset\}\)
	\item  \(A = A ^{\mathrm{o}}\) aberto e \(A = \overline{A} \) fechado
	\item \(f:X\rightarrow \mathbb{R},\quad \mathrm{supp}(f)\coloneqq \overline{\{x: f(x)\neq 0\}}\)
\end{itemize}
\subsection{Compactos, Normas e Propriedades}
\begin{def*}
	Um espaço normado é um par \((X, \Vert \cdot  \Vert)\) tal que existe uma aplicação \(x \mapsto \Vert x \Vert\) tal que
	\begin{itemize}
		\item[i)] \(\Vert x \Vert > 0,\quad \Vert x \Vert = 0 \Longleftrightarrow x = 0\)
		\item[ii)] \(\Vert cx \Vert = |c|\Vert x \Vert, \quad c\in \mathfrak{F} = \{\mathbb{R}, \mathbb{C}\}, x\in X\)
		\item[iii)] \(\Vert x + y \Vert \leq \Vert x \Vert + \Vert y \Vert,\quad \forall x, y\in X.\quad \square\)
	\end{itemize}
\end{def*}
Todo espaço normado é métrico por meio de \(d(x, y) = \Vert x - y \Vert\).

A ordem em \(X = \mathcal{P}(A) = \{B\subseteq A\}\) (partes de A) é dada por
\[
	B\leq C,\quad \text{se } B, C\in X \quad\&\quad B\subseteq C.
\]
\begin{prop*}
	Se K é compacto, \(F\subseteq K\) e F é fechado, então F é compacto.
\end{prop*}
\begin{proof*}
	Seja \(F\subseteq \bigcup_{i=1}^{\infty}G_{i}.\) Então, \(F ^{\complement}\) é aberto e \(\bigcup_{i=1}^{n}G_{i}\cup F ^{\complement}\) é cobertura de K.
	Assim, existem \(F ^{\complement}, G_{1}, G_{2}, \dotsc , G_{n}\in \{G_{\alpha }\}_{\alpha \in I}\) que formam uma subcobertura finita de F. Note que \(F ^{\complement}\) pode ser
	descartado, pois é uma cobertura trivial de \(F ^{\complement}\). Portanto, \(G_{1}, \dotsc , G_{n}\) é uma cobertura finita por abertos de F. \qedsymbol
\end{proof*}
\begin{prop*}
	Se K é compacto e f é contínua em K, então existem \(x_{1}, x_{2}\in K\) tais que
	\[
		f(x_{1}) = \inf_{x\in K}f(x)\quad\&\quad f(x_{2}) = \sup_{x\in K}f(x).
	\]
\end{prop*}
\begin{proof*}
	Seja \(M = \sup_{x\in K}f(x)\) e suponha que \(f(x) < M\) para todo x em K. Para \(y\in K\), seja
	\[
		L_{y} = \frac{f(y) + M}{2}
	\]
	e seja
	\[
		\varepsilon_y = \frac{(M-f(y))}{2}.
	\]
	Pela continuidade da f, para \(\varepsilon_y\), existe \(\delta _y\) tal que \(|f(y) - f(z)| < \varepsilon_{y}\) se \(d(y, z)< \delta_y.\)
	Logo, \(G_{y} = B(y, \delta_y)\) é uma bola aberta contendo y, sendo f limitada superiormente por \(L_y\).

	Desta forma, \(\{G_y\}\) é uma cobertura por abertos de K. Seja, agora, \(\{G_{y_{i}}, i = 1, 2, \dotsc , n\}\) cobertura finita de K e \(L = \max\{L_{y_1}, \dotsc , L_{y_{n}}\}\),
	tal que \(L < M\). Caso \(x\in K\), então \(x\in G_{y_{i}}\) para algum \(y_{i}\) e, portanto, \(f(x) \leq L_{y_{i}} \leq L.\) Assim, L é limitante superior de \(\{f(x):x \in K\}\), uma
	contradição com a definição de M. Portanto, \(f(x) < M\) para todo x em K não pode valer. \qedsymbol
\end{proof*}
\begin{prop*}
	Se \(K\subseteq \mathbb{R}\) é fechado e K está contido no intervalo finito, então K é compacto
\end{prop*}
\begin{proof*}
	Basta provar que se \(K\subseteq [a, b]\) e \([a, b]\) é compacto, uma aplicação da última proposição garante o resultado.
	Para provar a compacidade de [a, b], é preciso usar o axioma enunciado abaixo. \qedsymbol
\end{proof*}

\underline{\textbf{Axioma}}: Se \(A\subseteq \mathbb{R}\) é limitado superiormente, então o supremo \(\sup_{x\in A}\) existe.

\begin{prop*}
	Suponha \(I_{1}\supseteq I_{2}\supseteq \dotsc \) são intervalos limitados, \(I_{i}\subseteq \mathbb{R}\) para todo j. Então,
	\[
		\bigcap_{i\geq 1}^{}I_{i}\neq\emptyset.
	\]
\end{prop*}
\begin{proof*}
	Escreva \(I_{i} = [a_{i}, b_{i}]\). Como \(I_{1}\supseteq I_{2}\supseteq \dotsc \), temos
	\[
		a_{1} \leq a_{2} \leq \dotsc \quad\&\quad b_{1} \geq b_{2} \geq  \dotsc .
	\]
	Como \(I_{i}\subseteq I_{1}\) para todo \(i \geq 1\), segue que \(a_{i} \leq b_{1}\) para todo \(i\geq 1,\) donde segue que \(A = \{a_{i}\}\subseteq \mathbb{R}\) é limitada superiormente.
	Pelo axioma, \(x = \sup_{i\geq 1}A\) existe. Suponha que \(x > b_{i_{0}}\) para algum \(i_{0}\). Para cada \(i \geq i_{0},\) temos \(a_{i}\leq b_{i} \leq b_{i_{0}}\) e, para
	\(i < i_{0}\), também temos \(a_{i}\leq a_{i_{0}}\leq b_{i_{0}}\). Logo, \(b_{i_{0}}\) é uma cota superior para A. Uma contradição, pois \(x\) é o supremo. Portanto,
	\(x\leq b_{i}\) para todo i e, sendo x o supremo de A, temos \(x \geq a_{i}\) para todo i, o que significa que \(x\in [a_{i}, b_{i}]\), finalizando a prova, já que \(x\in \bigcap_{i\geq 1}^{}I_{i}.\) \qedsymbol
\end{proof*}
\begin{prop*}
	Se \(-\infty< a < b < \infty,\) então \([a, b]\) é um conjunto compacto.
\end{prop*}
\begin{proof*}
	Seja \(I_{1} = [a, b]\) , defina \(a_{1} = a \) e \(b_{1} = b\). Seja \(\mathcal{G} = \{G_{\alpha }\}\) uma cobertura por abertos de \(I_{1}\) e suponha
	que \(\mathcal{G}\) não admita uma subcobertura finita. Divida o intervalo \(I_{1}\) em \(I_{1} = [a_{1}, c_{1}]\cup [c_{1}, b_{1}]\), sendo \(c_{1} = \frac{a_{1}+b_{1}}{2}\). Pelo menos
	um dos subintervalos não possui subcobertura finita, digamos que \(I_{2} = [a_{2}, b_{2}]\), sendo \(a_{2} = a_{1}\) e \(b_{2} = c_{1}.\) Divida o intervalo \(I_{2}\) em \(I_{2} = [a_{2}, c_{2}]\cup [c_{2}, b_{2}]\), sendo
	\(c_{2}\) o ponto médio do intervalo. Pelo menos um dos subintervalos não possui uma cobertura finita, digamos \(I_{2} = [a_{3}, b_{3}],\) em que \(a_{3} = a_{2}\) e \(b_{3} = c_{2}.\)

	Continuando, obtemos intervalos
	\[
		I_{1}\supseteq I_{2}\supseteq \dotsc ,\quad I_{j} = [a_{j}, b_{j}], |I_{j}| = 2^{-(j-1)}(b-a).
	\]
	Existe apenas um ponto \(x\in \bigcap_{i\geq 1}^{}I_{i}\). Agora, \(x\in I_{1}\) e \(\mathcal{G}\) é uma cobertura para \(I_{1}.\) Existe um aberto \(G_{\alpha_{0}}\in \mathcal{G}\) tal que
	\(x\in G_{\alpha_{0}}.\) Sendo \(G_{\alpha_{0}}\) aberto, existe n tal que \(x - 2^{-(n-1)}(b-a), x+2^{-(n-1)}(b-a))\subseteq G_{\alpha_{0}}.\) Mas, \(x\in I_{n}\) para todo n e o comprimento
	é \(|I_{n}| = 2^{-(n-1)}(b-a)\), o que implica que \(I_{n}\subseteq G_{\alpha_{0}}\). Mas, então, a cobertura com um único conjunto \(\{G_{\alpha_{0}}\} \) é uma cobertura finita de \(\mathcal{G}\) cobrindo
	\(I_{n},\) um absurdo. \qedsymbol
\end{proof*}
\begin{prop*}
	Suponha que \(G\subseteq \mathbb{R}\) é um aberto. Então, G pode ser escrito como uma união enumerável de intervalos abertos disjuntos.
\end{prop*}
\begin{proof*}
	Seja \(G\subseteq \mathbb{R}\) um aberto. Para cada \(x\in G\), defina
	\[
		A_{x} = \inf_{}\{a: \exists b\mid x\in (a, b)\subseteq G\}
	\]
	e
	\[
		B_{x} = \sup_{}\{d: \exists c\mid x\in (c, d)\subseteq G\}.
	\]
	Seja \(I_{x} = (A_{x}, B_{x}).\) Provaremos que \(x\in I_{x}\subseteq G\). Se \(y\in I_{x},\) então
	\begin{itemize}
		\item \(y > A_{x} \Rightarrow \) existem a, b tais que \(A_{x} < a < y\) e \(x\in (a, b)\subseteq G\)
		\item \(y < B_{x} \Rightarrow \) existem \(c, d\) tais que \(y < d < B_{x}\) e \(x\in (c, d)\subseteq G\).
	\end{itemize}
	Consequentemente, \(x\in (a, b)\cup (c, d) = (a\wedge b, c\vee d)\equiv J.\) Note que \(J\subseteq G\) e é um aberto. Além disso, ambos x, y são maiores
	do que \(a > A_{x}\) e menores do que \(d < B_{x}\), donde segue que \(x\in I_{x}\) e \(y\in J\subseteq G\). Logo, \(I_{x}\subseteq G.\)

	Provaremos, agora, que se \(x\neq y,\) então \(I_{x}\cap I_{y} = \emptyset \) ou \(I_{x} = I_{y}.\) Assuma que \(I_{x}\cap I_{y}\neq\emptyset.\) Então,
	\(H = I_{x}\cup I_{y}\) é um intervalo aberto, \(H \subseteq G\) e \(H = (A_{x}\wedge A_{y}, B_{x}\vee B_{y}).\) Agora, se \(x\in I_{x}\subseteq J = H \subseteq G,\) segue da definição que
	\[
		A_{x} \leq A_{x}\wedge A_{y} \Rightarrow A_{x} \leq A_{y}
	\]
	e, analogamente,
	\[
		B_{x} \geq B_{x}\vee B_{y}\Rightarrow B_{x}\geq B_{y}.
	\]
	Então, \(I_{y}\subseteq I_x\). Trocando x por y, prova-se que \(I_x \subseteq I_y,\) ou seja, \(I_{x} = I_{y}\).

	Disto, concluímos que, para \(I_{x}\) abertos dois-a-dois disjuntos, \(G = \cup_{x\in G}I_{x}.\) Finalmente, para \(x\in G\), escolha \(r_{x}\in \mathbb{Q}\) tal que \(r_{x}\in I_x\). Então,
	se \(x\neq y\), temos \(r_{x}\neq r_y,\) pois \(I_{x}\cap I_{y} = \emptyset \), ou seja,
	\begin{align*}
		\varphi : & G = \bigcup_{x\in G}^{}I_{x}\rightarrow \mathbb{Q} \\
		          & x \mapsto r_{x}
	\end{align*}
	é injetora. Portanto, G é enumerável, pois \(\mathbb{Q}\) é enumerável. \qedsymbol
\end{proof*}
\begin{prop*}
	Seja \(f:\mathbb{R}\rightarrow \mathbb{R}\) uma função crescente. Então, ambos \(\lim_{y\to x^{\pm}}f(y)\) existem para todo x. Além disso, o conjunto
	\[
		\{x\in \mathbb{R}: f \text{ não é contínua no ponto x}\}
	\]
	é enumerável.
\end{prop*}
\begin{proof*}
	Suponha que f seja crescente e fixe \(x_{0}\in \mathbb{R}.\) Defina
	\[
		A = \{f(x): x < x_{0}\},
	\]
	de modo que A é limitado superiormente por \(f(x_{0})\). Seja \(M = \sup_{x\in A}f(x).\) Então, dado \(\varepsilon > 0, M - \varepsilon \) não é supremo, ou seja,
	existe \(x_{1} < x_{0}\) tal que \(f(x_{1}) > M - \varepsilon .\) Seja \(\delta  = x_{0} - x_{1}.\) Caso x seja tal que \(x_{0} - \delta < x < x_{0},\) então \(f(x) \leq M\), pois
	M é supremo. Por outro lado, \(f(x) > M - \varepsilon \), visto que \(f(x) \geq f(x_{1}) > M - \varepsilon .\) Assim, para todo \(\varepsilon > 0\) e para todo \(x\in (x_{0}-\delta , x_{0}),\) temos
	\[
		M - \varepsilon \leq f(x) \leq M,
	\]
	ou seja, \(\lim_{x\to x_{0}^{-}}f(x)\) existe.
	Agora, se B é limitado inferiormente, então \(A = \{-x: x\in B\}\) é limitado superiormente e, se \(M = \sup_{}A,\) então \(-M = \inf_{}B.\) Procedendo como antes, chegamos na conclusão
	que \(\lim_{x\to x_{0}^{+}}f(x)\) existe.

	Finalmente, para cada x tal que \(f(x^{-}) < f(x^{+}),\) existe \(r_{x}\in \mathbb{Q}\) tal que \(r_{x}\in (f(x^{-}), f(x^{+}))\equiv I_{x}.\) Sendo f crescente, se \(x < y\), temos
	\(I_{x}\cap I_{y} = \emptyset.\) Denote por D o conjunto dos pontos de descontinuidade de f. Com isso,
	\begin{align*}
		\varphi : & D\rightarrow \mathbb{Q} \\
		          & x \mapsto r_{x}
	\end{align*}
	é injetora. Portanto, D é enumerável. \qedsymbol
\end{proof*}
\begin{prop*}
	Seja X espaço métrico compacto, \(\mathcal{A}\) uma coleção de \(f:X\rightarrow \mathbb{C}\) contínuas satisfazendo
	\begin{itemize}
		\item[i)] Se \(f, g\in \mathcal{A}\), então \(f + g, fg, cf\in \mathcal{A}\)
		\item[ii)] Se \(f\in \mathcal{A}\), então \(\overline{f}\in \mathcal{A}\)
		\item[iii)] Se x pertence a X, existe \(f\in \mathcal{A}\) tal que \(f(x)\neq 0\)
		\item[iv)] Se \(x, y\in X\), então existe \(f\in \mathcal{A}\) tal que \(f(x)\neq f(y)\)
	\end{itemize}
	Então, \(\overline{\mathcal{A}}\) é uma coleção de funções contínuas em X.
\end{prop*}
Observe que, se f é contínua em X e \(\varepsilon > 0\), existe \(g\in \mathcal{A}\) tal que
\[
	\sup_{x\in X}|f(x) - g(x)| < \varepsilon.
\]
\subsection{Introdução a Conjuntos Mensuráveis e Álgebra de Borel}
\begin{def*}
	Seja X um conjunto. Uma \textbf{álgebra} é uma coleção \(\mathcal{A}\) de subconjuntos de X tal que
	\begin{itemize}
		\item[1)] \(\emptyset \in \mathcal{A}, X\in \mathcal{A}\)
		\item[2)] Se \(A\in \mathcal{A},\) então \(A ^{\complement}\in \mathcal{A}\)
		\item[3)] Se \(A_{1}, A_{2}, \dotsc , A_{n}\in \mathcal{A}\), então \(\bigcup_{i=1}^{n}A_{i}\in \mathcal{A}\)
		\item[4)] Diremos que \(\mathcal{A}\) é \textbf{\(\sigma \)-álgebra} se
		      \[
			      A_{1}, \dotsc \in \mathcal{A} \Rightarrow  \bigcup_{i=1}^{\infty}A_{n}\in \mathcal{A}.
		      \]
	\end{itemize}
	O par \((X, \mathcal{A})\) é chamado \textbf{espaço mensurável}, e A é \textbf{mensurável} ou \(\sigma \)\textbf{-mensurável} se \(A\in \mathcal{A}.\)
\end{def*}
Como \(\bigcap_{}^{}A_{i} = \biggl(\bigcup_{}^{}A_{i}\biggr) ^{\complement}\), álgebras e \(\sigma \)-álgebras são fechadas pela interseção também.
\begin{example}
	\item[1)] \(X = \mathbb{R}, \mathcal{A}\) coleção de subconjuntos de \(\mathbb{R}\) é \(\sigma \)-álgebra.
	\item[2)] \(X = [0, 1], \mathcal{A} = \{\emptyset , X, [0, \frac{1}{2}], (\frac{1}{2}, 1]\}\) é \(\sigma \)-álgebra.
	\item[3)] \(X = \{1, 2, 3\}, \mathcal{A} = \{\emptyset , X, \{1\}, \{2, 3\}\}\) é \(\sigma\)-álgebra.
	\item[4)] \(X = [0, 1], B_{1}, B_{2}, \dotsc , B_{8}\subseteq X\) dois-a-dois disjuntos e \(\bigcup_{i=1}^{8}B_{i} = X.\) Seja \(\mathcal{A}\) a coleção de
	união finita de \(B_{i}\) junto com \(\emptyset \) e X. Então, \(\mathcal{A}\) é \(\sigma \)-álgebra.
	\item[5)] Se \(X = \mathbb{R}, \mathcal{A} = \{A\subseteq \mathbb{R}: \text{A é enumerável ou }A ^{\complement} \text{ é enumerável}\} \) é uma \(\sigma \)-álgebra. De fato,
	basta ver que, se \(A_{1}, \dotsc \in \mathcal{A}\) é enumerável para todo i, então \(\bigcup_{}^{}A_{i}\) é enumerável. Então,
	\[
		\biggl(\bigcup_{}^{}A_{i}\biggr) ^{\complement} = \bigcap_{}^{}A_{i}^{\complement} \subseteq A_{i_{0}}^{\complement}
	\]
	é enumerável. Portanto, em qualquer caso, \(\bigcup_{}^{}A_{i}\in \mathcal{A}.\) Notando que \(\bigcap_{}^{}A_{i} = (\bigcup_{}^{}A_{i}^{\complement})^{\complement}\), segue que \(\bigcap_{}^{}A_{i}\in \mathcal{A}.\)
\end{example}
\begin{lemma*}
	Se \(\mathcal{A}_{\alpha }\) é \(\sigma \)-álgebra, para cada \(\alpha \in I \neq\emptyset\), então \(\mathcal{B} = \bigcap_{\alpha \in I}^{}A_{\alpha }\) é \(\sigma \)-álgebra.
\end{lemma*}
\begin{proof*}
	Segue da definição: \(\emptyset , X\in \mathcal{B}\) é claro. Se \(A\in \mathcal{B}, \) então \(A ^{\complement}\in \mathcal{B}\) segue pois, se \(A\in \bigcap_{}^{}A_{\alpha }\), então
	\(A\in \mathcal{A}_{\alpha }\) para todo \(\alpha \in I.\) Logo, \(A ^{\complement}\in \mathcal{A}_{\alpha }\) para todo \(\alpha \) em I. Portanto, \(A ^{\complement}\in \bigcap_{}^{}\mathcal{A}_{\alpha }\).
	Os outros caso são análogos. \qedsymbol
\end{proof*}
\begin{def*}
	Seja \(\mathcal{C}\) a coleção de subconjuntos de X. Defina a \(\sigma \)-álgebra gerada por \(\mathcal{C}\) como
	\[
		\sigma (C) = \bigcap_{}^{}\biggl\{\mathcal{A}_{\alpha }: A_{\alpha }\text{ é }\alpha \text{-álgebra e } \mathcal{C}\subseteq \mathcal{A}_{\alpha }\biggr\}.\quad \square
	\]
\end{def*}
Note que \(\sigma (C)\neq\emptyset\), \(\sigma (C) \) é \(\sigma \)-álgebra, \(\sigma (\sigma (\mathcal{C})) = \sigma (\mathcal{C})\), pois isto indica que \(\sigma (\mathcal{C})\) gera \(\sigma (\sigma (\mathcal{C}))\), ou seja, \(\sigma (\mathcal{C}) \subseteq \sigma (\sigma (\mathcal{C}))\). Por outro lado, \(\sigma (\mathcal{C})\) é \(\sigma \)-álgebras, tal que a interseção
está \(\sigma (\sigma (\mathcal{C}))\) está contida em \(\sigma (\mathcal{C}).\) Finalmente, se \(\mathcal{C}_{1}\subseteq \mathcal{C}_{2},\) então
\(\sigma (\mathcal{C}_{1}) \subseteq \sigma (\mathcal{C}_{2})\).
\begin{def*}
	Seja X espaço métrico e \(\mathcal{G}\) a coleção de abertos de X. Denote a \(\sigma \)\textbf{-álgebras de Borel em X} por \(\mathcal{B}\equiv \sigma (\mathcal{G}).\) Os elementos de \(\mathcal{B}\) são
	chamados de \textbf{conjuntos de Borel}, ou seja, \textbf{Borel-mensurável}.
\end{def*}
Veremos que, se \(X = \mathbb{R}, \) então \(\mathcal{B}\) não é igual a todos os subconjuntos de X.
\begin{prop*}
	Seja \(X = \mathbb{R}.\) Então, \(\mathcal{B}\) é gerada por cada uma dos seguintes coleções:
	\begin{itemize}
		\item[a)] \(\mathcal{C}_{1} = \{(a, b):a, b\in \mathbb{R}\}\)
		\item[b)] \(\mathcal{C}_{2} = \{[a, b]:a, b\in \mathbb{R}\}\)
		\item[c)] \(\mathcal{C}_{3} = \{(a, b]:a, b\in \mathbb{R}\}\)
		\item[d)] \(\mathcal{C}_{4} = \{[a, \infty):a\in \mathbb{R}\}\)
	\end{itemize}
\end{prop*}
\begin{proof*}
	a) Seja \(\mathcal{G}\) a coleção de abertos. Então, \(\sigma (\mathcal{G})\) é a \(\sigma \)-álgebra de Borel. Como cada elemento de \(\mathcal{C}_{1}\) é aberto, então \(\mathcal{C}_{1}\subseteq \mathcal{G}.\) Logo,
	\(\sigma(\mathcal{C}_{1}) = \sigma (\mathcal{G}) = \mathcal{B}.\)

	Reciprocamente, se G é aberto, então \(G = \bigcup_{}^{}I_{j}\) em que \(I_{j}\) são intervalos abertos. Se for intervalo finito, terminamos. Caso contrário, como \((a, \infty) = \bigcup_{n=1}^{\infty}(a, a + n)\), então
	\((a, \infty)\in \sigma (\mathcal{C}_{1})\). Analogamente, \((-\infty, a)\in \sigma (\mathcal{C}_{1}).\) Com isso, se G é aberto, então \(G\in \sigma (\mathcal{C}_{1})\). Portanto, \(\mathcal{G}\subseteq \sigma (\mathcal{C}_{1})\), ou seja,
	\(\mathcal{B} = \sigma (\mathcal{G})\subseteq \sigma (\sigma (\mathcal{C}_{1})) = \sigma (\mathcal{C}_{1}).\)

	b) Se \([a, b]\) pertence a \(\mathcal{C}_{2},\) então \([a, b] = \bigcap_{i=1}^{\infty}(a-\frac{1}{n}, b+\frac{1}{n})\in \sigma (\mathcal{G}),\) donde segue que \(\mathcal{C}_{2}\subseteq \sigma (\mathcal{G}),\)
	ou seja, \(\sigma (\mathcal{C}_{2})\subseteq \sigma (\sigma (\mathcal{G})) = \sigma (\mathcal{G}) = \mathcal{B}.\) Caso \((a, b)\in \mathcal{C}_{1},\) escolha \(n_{0} > 2/(b-a).\) Basta notar que faz sentido porque
	\[
		\biggl(\biggl(b-\frac{1}{n}\biggr) - \biggl(a + \frac{1}{n}\biggr)\biggr) > 0 \Rightarrow b - a >2/n.
	\]
	Com isso, \((a, b) = \bigcup_{n=n_{0}}^{\infty}\biggl[a+\frac{1}{n}, b-\frac{1}{n}\biggr]\in \sigma (\mathcal{C}_{2}).\) Portanto, \(\mathcal{C}_{1}\subseteq \sigma (\mathcal{C}_{2})\) e
	segue que \(\mathcal{B} = \sigma (\mathcal{C}_{1})\subseteq \sigma (\sigma (\mathcal{C}_{2})) = \sigma (\mathcal{C}_{2})\).

	c) Como
	\[
		(a, b] = \bigcap_{n=1}^{\infty}\biggl(a, b+\frac{1}{n}\biggr),
	\]
	segue que \(\mathcal{C}_3 \subseteq \sigma (\mathcal{C}_1)\). Logo, \(\sigma (\mathcal{C}_3)\subseteq \sigma (\sigma (\mathcal{C}_1)) = \mathcal{B}.\) Usando que \((a, b) = \bigcup_{n=n_{0}}^{\infty}\biggl[a, b-\frac{1}{n}\biggr]\) para \(n_{0}\) grande,
	obtemos \(\mathcal{C}_{1} \subseteq \sigma (\mathcal{C}_{3}) \Rightarrow \mathcal{B} \subseteq \sigma (\mathcal{C}_3)\)

	d) Como
	\[
		(a, b] = (a, \infty)\setminus{(b, \infty)} \subseteq (a, \infty),
	\]
	temos \(\mathcal{C}_{3}\subseteq \sigma (\mathcal{C}_{4})\), o que implica, como antes, que \(\mathcal{C}_{3}\supseteq  \mathcal{B} = \sigma (\sigma (\mathcal{C}_{3}))\). Por outro lado, já que \((a, \infty) = \bigcup_{n=1}^{\infty}(a, a + n]\), chegamos em
	\(\mathcal{C}_{4} \subseteq \sigma (\mathcal{C}_{3}).\) Portanto, \(\sigma (\mathcal{C}_{4})\subseteq \mathcal{B}\) e \(\sigma (\mathcal{C}_{4}) = \mathcal{B}.\) \qedsymbol
\end{proof*}

\newpage

\section{Aula 02 - 09/01/2024}
\subsection{Motivações}
\begin{itemize}
	\item Classes Monótonas;
	\item Medida, Medida Exterior e Medida de Lebesgue-Stieltjes;
	\item Teorema de Caratheodory.
\end{itemize}
\subsection{Classes Monótonas}
\begin{def*}
	Uma \textbf{classe monótona} é uma coleção de sbuconjuntos \(\mathcal{M}\) de X tal que

	\begin{itemize}
		\item[1)] Se \(A_{i}\uparrow A\) e cada \(A_{i}\in \mathcal{M},\) então \(A\in \mathcal{M}\)
		\item[2)] Se \(A_{i}\downarrow A\) e cada \(A_{i}\in \mathcal{M}\), então \(A\in \mathcal{M}\).
	\end{itemize}\(\quad \square\)
\end{def*}
Observe que a interseção de classes monótonas é uma classe monótona e que a interseção de todas as classes monótonas contendo uma coleção de conjuntos é a menor
classe monótona contendo esta coleção. Essa segunda observação nos leva a postular o seguinte teorema:
\begin{theorem*}
	Suponha que \(\mathcal{A}_{0}\) seja uma álgebra, \(\mathcal{A}\) a menor \(\sigma \)-álgebra contendo \(\mathcal{A}_{0}\) e \(\mathcal{M} \) a menor classe monótona contendo \(\mathcal{A}_{0}\). Então, \(\mathcal{A} = \mathcal{M}.\)
\end{theorem*}
\begin{proof*}
	Uma \(\sigma \)-álgebra é uma classe monótona por definição. Assim, \(\mathcal{M}\subseteq \mathcal{A}\) pela observação feita. Mostraremos o outro lado. Primeiramente,
	seja \(\mathcal{N}_{1} = \{A\in \mathcal{M}: A ^{\complement}\in \mathcal{M}\}\) e note que \(\mathcal{N}_{1} \subseteq \mathcal{M}\) e que \(\mathcal{A}_{0} \subseteq \mathcal{N}_1.\) Se \(A_{i}\uparrow A\) e cada \(A_{i}\in \mathcal{N}_1\),
	então cada \(A_{i}\in \mathcal{M}\), e \(A_{i}^{\complement}\downarrow A ^{\complement}.\) Como \(\mathcal{M}\) é classe monótona, \(A ^{\complement}\in \mathcal{M}\), ou seja, \(A\in \mathcal{N}_{1}\)
	Analogamente, se \(A_{i}\downarrow A,\) com cada \(A_{i}\in \mathcal{N}_1,\) então cada \(A\in \mathcal{N}_1.\) Com isto, concluímos que \(\mathcal{N}_{1}\) é classe monótona e, assim,
	\(\mathcal{N}_1 = \mathcal{M}\). Desta forma, \(\mathcal{M}\) é fechado com relação à operação de tomar complementos.

	Em seguida, vamos mostrar que, se \(A, B\in \mathcal{M},\) então \(A\cap B\in \mathcal{M}.\) De fato, seja \(\mathcal{N}_{2} = \{A\in \mathcal{M}:A\cap B\in \mathcal{M} \text{ para todo }B\in \mathcal{A}_{0}\}\).
	Note que \(\mathcal{N}_2\subseteq \mathcal{M}\) e \(\mathcal{N}_2\supseteq \mathcal{A}_{0}\), pois \(\mathcal{A}_{0}\) é álgebra. Caso \(A_{i}\uparrow A,\) em que cada \(A_{i}\in \mathcal{N}_2\), então \(A\cap B = \bigcup_{i=1}^{\infty}(A_{i}\cap B).\)
	O fato de \(\mathcal{M}\) ser classe monótona implica que \(A\cap B\in \mathcal{M},\) donde segue que \(A\cap B\in \mathcal{N}_2\). Analogamente, se \(A_{i}\downarrow A, A\in \mathcal{N}_2\) e, portanto, \(\mathcal{N}_2\) é classe monótona, do que segue que \(\mathcal{N}_2 = \mathcal{M}.\)
	Em outras palavras, se \(B\in \mathcal{A}_{0}, \) então \(A\cap B\in \mathcal{M}.\)

	Finalmente, seja \(\mathcal{N}_{3} = \{A\in \mathcal{M}: A\cap B\in \mathcal{M},\text{ para todo }B\in \mathcal{M}\}.\) Assim como antes, \(\mathcal{N}_3\)
	é classe monótona contida em \(\mathcal{M},\) tal como \(\mathcal{N}_{3}\supseteq \mathcal{A}_{0},\) como no passo anterior. Disto, \(\mathcal{N}_3 = \mathcal{M}.\)
	Dessa forma, \(\mathcal{M}\) é uma classe monótona fechada com relação à tomada de complementos e interseções. De fato, se \(A_1, A_2, \dotsc \in \mathcal{M}\), então por \(\mathcal{N}_3\) temos
	\(B_{n} = \bigcap_{i=1}^{n}A_{i}\in \mathcal{M}\) para todo n e \(B_{n}\downarrow \bigcap_{i=1}^{\infty}A_{i}\).  Como \(\mathcal{M}\) é uma classe monótona, \(\bigcap_{i=1}^{\infty}A_{i}\in \mathcal{M}\).
	Por outro lado, se \(A_1, A_2, \dotsc \in \mathcal{M}\), segue de \(\mathcal{N}_1\) que \(A_{1}^{\complement}, A_{2}^{\complement}, \dotsc \in \mathcal{M}\) e, logo, \(\bigcap_{i=1}^{\infty}A_{i}^{\complement}\in \mathcal{M}\), do que segue que
	\(\bigcup_{i=1}^{\infty}A_{i}^{\complement}=\biggl(\bigcap_{i=1}^{\infty}A_{i}^{\complement}\biggr)^{\complement}\in \mathcal{M}.\)

	Portanto, \(\mathcal{M}\) é uma \(\sigma \)-álgebra e, assim, \(\mathcal{A} \subseteq \mathcal{M}.\) \qedsymbol
\end{proof*}
\begin{example}[Exercícios]
	\begin{itemize}
		\item[1)] Ache um exemplo de um conjunto X e uma classe monótona \(\mathcal{M}\) consistindo de subconjuntos de X, junto com \(\emptyset , X\in \mathcal{M}\), mas \(\mathcal{M}\) não é uma \(\sigma \)-álgebra. [R: \(\mathcal{M} = \{\emptyset , X, A\}, A\subsetneq X\)
		\item[2)] Seja \((Y, \mathcal{A})\) um espaço mensurável e \(f:X\rightarrow Y\) não injetora. Defina \(\mathcal{B} = \{f^{-1}(A): A\in \mathcal{A}\}.\) Prove que \(\mathcal{B}\) é uma \(\sigma \)-álgebra de subconjuntos de X.
	\end{itemize}
\end{example}
Passaremos a definir o que é uma medida e apresentaremos algumas propriedades. A ideia da medida busca generalizar comprimento, área e volume em dimensões 1, 2 e 3, respectivamente. Uma propriedade desejada é a de decompor uniões em somas, ou seja,
se \(A_{1}, A_2, \dotsc , A_{n}\) são dois-a-dois disjuntos, a medida da união deles será a soma da medida de cada componente.

\subsection{Medida}
\begin{def*}
	Seja X um conjunto, \(\mathcal{A}\) uma \(\sigma \)-álgebra consistindo de subconjuntos de X. Uma \textbf{medida} sobre \((X, \mathcal{A})\) é uma função \(\mu:A\rightarrow [0, \infty) \) tal que
	\begin{itemize}
		\item[1)] \(\mu (\emptyset ) = 0\)
		\item[2)] Se \(A_{i}\in \mathcal{A}, i = 1, 2, \dotsc \) são disjuntos dois-a-dois, então
		      \[
			      \mu \biggl(\bigcup_{i=1}^{\infty}A_{i}\biggr) = \sum\limits_{i=1}^{\infty}\mu (A_{i})\quad \text{\textbf{(aditividade enumerável) }} \square
		      \]
	\end{itemize}
\end{def*}
\begin{example}
	\begin{itemize}
		\item[1)] Se X é um conjunto e \(\mathcal{A}\) uma coleção de subconjuntos de X, então \(\mu (A)\) é o número de elementos de A é uma medida (contador)
		\item[2)] Se \(X = \mathbb{R}\), \(\mathcal{A}\) é uma coleção de subconjuntos de X, \(x_{1}, x_{2}, \dotsc \in \mathbb{R}\) e \(a_{1}, a_2, \dotsc \geq 0\), defina
		      \[
			      \mu (A) = \sum\limits_{\{i: x_{i}\in A\}}^{}a_{i}
		      \]
		      é medida
		\item[3)] Seja \(\delta_x (A) = 1\) se \(x\in A\) e 0 caso contrário. Essa é a medida concentrada de no ponto x.

	\end{itemize}
\end{example}
Valem as Propriedades:
\begin{prop*}
	Se \(A, B\in \mathcal{A}, A\subseteq B\), então \(\mu (A) \leq \mu (B)\)
\end{prop*}
\begin{proof*}
	Tome \(A_{1} = A, A_{2} = B\setminus{A}, A_{3} = A_{4} = \dotsc  = \emptyset \). Pela aditividade,
	\[
		\mu (B) = \mu (A) + \mu(B\setminus{A}) + 0 +\dotsc  \geq \mu (A). \quad \text{\qedsymbol}
	\]
\end{proof*}
\begin{prop*}
	Se \(A_{i}\in \mathcal{A}\) e \(A = \bigcup_{i=1}^{\infty}A_{i},\) então \(\mu (A) \leq \sum\limits_{i=1}^{\infty}\mu (A_{i})\)
\end{prop*}
\begin{proof*}
	Faça \(B_{1} = A_{1}, B_2 = A_2\setminus{A_1}, B_3 = A_3\setminus{(A_1\cup A_2)}, B_4 = A_4\setminus{(A_1\cup A_2\cup A_3)}.\)  Em geral, \(B_{i} = A_{i}\setminus{\bigcup_{j=1}^{i-1}A_{j}}\).
	Assim, \(B_{i}\) são dois-a-dois disjunto, \(B_{i}\subseteq A_{i}\) para cada i e \(\bigcup_{i=1}^{n}B_{i} = \bigcup_{i=1}^{n}A_{i}\) para cada n. Assim, \(\bigcup_{i=1}^{\infty}B_{i} = \bigcup_{i=1}^{\infty}A_{i}\) e, portanto,
	\[
		\mu(A) = \mu \biggl(\bigcup_{i=1}^{\infty}B_{i}\biggr) = \sum\limits_{i=1}^{\infty}\mu(B_{i}) \leq \sum\limits_{i=1}^{\infty}\mu (A_{i}).\quad \text{\qedsymbol}
	\]
\end{proof*}
\begin{prop*}
	Suponha \(A_{i}\in \mathcal{A}, A_{i}\uparrow A.\) Então, \(\mu (A) = \lim_{n\to \infty}\mu (A_{n})\).
\end{prop*}
\begin{proof*}
	Defina \(B_{i} = A_{i}\setminus{(\bigcup_{j=1}^{i-1}A_{j})}\). Como a união coincide, temos
	\begin{align*}
		\mu (A) & = \mu \biggl(\bigcup_{i=1}^{\infty}A_{i}\biggr) = \mu \biggl(\bigcup_{i=1}^{\infty}B_{i}\biggr) \\
		        & = \sum\limits_{i=1}^{\infty}\mu (B_{i}) = \lim_{n\to \infty}\sum\limits_{i=1}^{n}\mu (B_{i})    \\
		        & = \lim_{n\to \infty}\mu \biggl(\bigcup_{i=1}^{n}B_{i}\biggr)                                    \\
		        & = \lim_{n\to \infty}\mu \biggl(\bigcup_{i=1}^{n}A_{i}\biggr).\quad \text{\qedsymbol}
	\end{align*}
\end{proof*}
\begin{prop*}
	Suponha \(A_{i}\in \mathcal{A}, A_{i}\downarrow A.\) Se \(\mu (A_1) < \infty\), então \(\mu (A) = \lim_{n\to \infty}\mu (A_{n})\).
\end{prop*}
\begin{proof*}
	Aplicaremos a última proposição para \(A_1\setminus{A_{i}}, i = 1, \dotsc \). Note que \(A_1\setminus{A_{i}}\) cresce para \(A_1\setminus{A}\). Assim,
	\[
		\mu (A_1) - \mu (A) = \mu (A_1 - A) = \lim_{n\to \infty}(\mu (A_1) - \mu (A_{n})),
	\]
	pois, como \(A\subseteq A_1, \) então \(A_1 = (A_1\setminus{A})\cup A\) e \(A_1\setminus{A}\cap A = \emptyset \), tal que \(\mu (A_1) = \mu (A_1\setminus{A}) + \mu (A).\)
	Portanto, basta subtrair \(\mu (A_1)\) de ambos os membros e multiplicar por -1. \qedsymbol

\end{proof*}
\begin{example}
	A necessidade de \(\mu (A_1) < \infty\) segue pois, por exemplo, se \(X = 1, 2, \dotsc \) com medida contadora \(\mu \). Considere \(A_{i} = \{i, i+1, ..\}\), tal que \(A_{i}\) decresce,
	\(\mu (A_{i}) = \infty\), mas \(\mu (\bigcap_{i}^{}A_{i}) = \mu (\emptyset ) = 0.\)
\end{example}
\begin{def*}
	\begin{itemize}
		\item[a)] Uma medida \(\mu \) é \textbf{finita sobre X} se \(\mu (X) < \infty\);
		\item[b)] Uma medida \(\mu \) é \(\sigma \)\textbf{-finita} se existir uma sequência de conjuntos \(E_{i}\in \mathcal{A}\) para i = 1, 2, \(\dotsc \) tal que
		      \(\mu (E_{i}) < \infty\) e \(X = \bigcup_{i=1}^{\infty}E_{i};\)
		\item[c)] Se \(\mu \) é medida finita, então \((X, \mathcal{A}, \mu )\) é chamada \textbf{espaço de medida finita};
		\item[d)] Se \(\mu \) é \(\sigma \)-finita, entào \((X, \mathcal{A}, \mu )\) é chamado \textbf{espaço de medida} \(\sigma \)\textbf{-finita.}
	\end{itemize} \(\square\)

\end{def*}
Suponha que X seja \(\sigma \)-finita. Então, \(X = \bigcup_{i=1}^{\infty}E_{i}\) e \(E_{i}\in \mathcal{A}\), se \(F_{n} = \bigcup_{i=1}^{n}E_{i}\) com \(\mu (F_{n}) < \infty\) para cada n, \(F_{n}\uparrow X\), isso mostra
que podemos exigir que \(E_{i}\) seja crescente.
\begin{def*}
	\begin{itemize}
		\item[i)] Um conjunto \(A\subseteq X\) é \textbf{nulo}, ou tem \textbf{medida nula}, se existir um conjunto \(B\in \mathcal{A}\) com \(A\subseteq B\) e \(\mu (B) = 0.\)
		\item[ii)] Dizemos que \((X, \mathcal{A}, \mu )\) é um \textbf{espaço de medida completa} se \(\mathcal{A}\) contém todos os conjuntos nulos.
		\item[iii)] Um \textbf{completamento} de \(\mathcal{A}\) é a menor \(\sigma \)-álgebra \(\overline{\mathcal{A}}\) contedno \(\mathcal{A}\) tal que \((X, \overline{\mathcal{A}}, \overline{\mu })\) é completa,
		      sendo \(\overline{\mu} \) uma medida sobre \(\overline{\mathcal{A}}\) que é uma extensão de \(\mu \), ou seja, \(\mu(B) = \overline{\mu }(B)\) para todo \(B\in \mathcal{A}.\)
		\item[iv)] Uma \textbf{probabilidade} é uma medida \(\mu \) tal que \(\mu (X) = 1\). Escrevemos \((\Omega , \mathcal{F}, \mathbb{P})\) no lugar de \((X, \mathcal{A}, \mu )\) e \(\mathcal{F}\) é chamado de \(\sigma \)\textbf{-campo}.
	\end{itemize}
\end{def*}
\subsection{Construindo uma Medida}
Vamos construir uma medida. Para isso, utilizaremos da medida exterior. A mais conhecida é a medida de Lebesgue. Para construí-la, sendo \(m \) esta medida
\[
	m(I) = |I|,
\]
considere todo aberto da reta uma união enumerável de intervalos abertos disjuntos,
\[
	G = \bigcup_{i=1}^{\infty}(a_{i}, b_{i})
\]
Defina
\[
	m (E) = \inf_{}\{\lambda (G), G \text{ aberto }, E\subseteq G\}, \quad E\subseteq \mathbb{R}.
\]
Note que \(m \) n~ao é uma medida sobre \(\sigma \)-álgebra de todos os subconjuntos da reta. Isto será contornado definindo uma \(\sigma \)-álgebra estritamente menor a aplicando o teorema de Caratheodory
\begin{def*}
	Seja X um conjunto. Uma \textbf{medida exterior} é uma função \(\mu ^{*}\) definida na coleção de todos os subconjuntos de X tal que
	\begin{itemize}
		\item[a)] \(\mu ^{*}(\emptyset ) = 0\)
		\item[b)] Se \(A\subseteq B\), então \(\mu ^{*}(A) \leq \mu ^{*}(B)\)
		\item[c)] \(\mu ^{*}(\bigcup_{i=1}^{\infty}A_{i}) \leq \sum\limits_{i=1}^{\infty}\mu (A_{i})\) para todo subconjuntos de X.
	\end{itemize}
	Um conjnto N é nulo com relação a \(\mu ^{*}\) se \(\mu ^{*}(N) = 0\)
\end{def*}
\begin{prop*}
	Seja \(\mathcal{C} \) uma coleção de subconjuntos de X tal que \(\emptyset \in \mathcal{C}\) e existem \(D_{1}, D_2, \dotsc \in \mathcal{C}\) tais que \(X = \bigcup_{i=1}^{\infty}D_{i}\).
	Suponha que \(\ell :\mathcal{C}\rightarrow [0, \infty]\) com \(\ell (\emptyset )= 0\). Defina
	\[
		\mu ^{*}(E) = \inf_{}\biggl\{\sum\limits_{i=1}^{\infty}\ell (A_{i}): A_{i}\in \mathcal{C}, \forall i, E\subseteq \bigcup_{i=1}^{\infty}A_{i}\biggr\}.
	\]
	Então, \(\mu ^{*}\) é uma medida exterior.
\end{prop*}
\begin{proof*}
	Que \(\mu ^{*}(\emptyset ) = 0\) é claro. Se \(A\subseteq B\), então \(\inf_{}A \leq \inf_{}V\), tal que
	\[
		\mu ^{*}(A) \leq \mu ^{*}(B).
	\]
	Além disso, sejam \(A_{1}, A_{2},\dotsc \) subconjuntos de X, \(\varepsilon > 0\) dado. para cada i, existem \(C_{i1}, C_{i2}, \dotsc \in \mathcal{A}\) tais que
	\[
		A_{i}\subseteq \bigcup_{j=1}^{\infty}C_{ij},\quad \sum\limits_{j=1}^{\infty}\ell (C_{ij})\leq \mu ^{*}(A_{i}) + \frac{\varepsilon }{2^{i}}
	\]
	Assim, \(\bigcup_{i=1}^{\infty}A_{i}\subseteq \bigcup_{i=1}^{\infty}\bigcup_{j=1}^{\infty}C_{ij}\) e
	\[
		\mu ^{*}\biggl(\bigcup_{i=1}^{\infty}A_{i}\biggr) \leq \sum\limits_{i=1}^{\infty}\ell (C_{ij}) = \sum\limits_{i}^{}\sum\limits_{j}^{}\ell (C_{ij}) \leq \sum\limits_{i=1}^{\infty}\mu ^{*}(A_{i}) + \varepsilon .
	\]
	Sendo \(\varepsilon \) arbitrário, temos
	\[
		\mu ^{*}\biggl(\bigcup_{i=1}^{\infty}A_{i}\biggr) \leq \sum\limits_{i=1}^{\infty}\mu ^{*}(A_{i}).\quad \text{\qedsymbol}
	\]
\end{proof*}
\begin{example}
	Seja \(X = \mathbb{R}\) e \(\mathcal{C}\) a coleção de intervalos da forma \((a, b]\). Seja \(\ell (I) = b- a\), em que \(I = (a, b]\). Defina \(\mu ^{*}\) pondo
	\[
		\mu ^{*} = \inf_{}\biggl\{\sum\limits_{i=1}^{\infty}\ell (A_{i}): A_{i}\in \mathcal{C}, E\subseteq \bigcup_{i=1}^{\infty}A_{i}\biggr\}.
	\]
	Da proposição anterior, \(\mu ^{*}\) é medida exterior. Apesar disso, \(\mu ^{*}\) não é medida sobre todos os subconjuntos de \(\mathbb{R}\), mas se for restrito à \(\lambda \)-álgebra \(\mathcal{L}\), que é estritamente
	menor que a coleção de todos os subconjuntos, ela será medida sobre \(\mathcal{L}\), denominada medida de Lebesgue. \(\mathcal{L}\) é chamada \(\sigma \)-álgebra de Lebesgue.
\end{example}
\begin{example}
	Seja \(X = \mathbb{R}\) e \(\mathcal{C}\) todos os subconjuntos da forma \((a, b]\). Seja \(\alpha : \mathbb{R}\rightarrow \mathbb{R}\) crescente, contínua à direita. Assim, para cada x,
	\[
		\alpha (x) = \lim_{y\to x}\alpha (y),\quad \alpha (x)< \alpha (y) \text{ se} x < y.
	\]
	Seja \(\ell (I) = \alpha (b) - \alpha (a)\) para \(I = (a, b]\). Defina \(\mu ^{*}\) como no último exemplo, que seraá medida exterior sobre a \(\sigma \)-álgebra \(\mathcal{L}\). Essa medida
	é denominada medida de Lebesgue-Stieltjes em relação a \(\alpha \), coincidindo caso \(\alpha \) seja identidade.
\end{example}
Em geral, podemos restringir \(\mu ^{*}\) a uma \(\sigma \)-álgebra menor do que a coleção de todos os subconjuntos de \(\mathbb{R}\), mas nem sempre! Por exemplo,
\[
	\alpha (x)  = \left\{\begin{array}{ll}
		0\quad x < 0 \\
		1\quad x \geq 0
	\end{array}\right.
\]
A medida de Lebesgue-Stieltjes é um ponto de massa em 0 e a correspondente \(\sigma \)=álgebra é a coleção de todos os subconjuntos de \(\mathbb{R}. \) De fato, se \(\ell (I) = \alpha (b) - \alpha (a)\), então, caso \(0\in I\),
segue que \(a < 0, b> 0\) e \(\ell (I) = 1 - 0 = 1\). Caso \(0\not\in I\), então \(a, b\) são ambas positivas ou negativas e, assim, \(\ell (I) = 1 - 1 = 0\) ou \(\ell (I) = 0 - 0 = 0\), ou seja,
\[
	\ell (I) \equiv \delta_{x}(I)  = \left\{\begin{array}{ll}
		1\quad x\in I \\
		0\quad x\not\in I
	\end{array}\right.
\]
é a medida denominada massa centrada em 0.
\begin{def*}
	Seja \(\mu ^{*}\) uma medida exterior. Dizemos que \(A\subseteq X\) é \(\mu ^{*}\)\textbf{-mensurável} se
	\[
		\mu ^{*}(E) = \mu ^{*}(E\cap A) + \mu ^{*}(E\cap A ^{\complement}),\quad \forall E\subseteq X.
	\]
\end{def*}
\begin{theorem*}
	Seja \(\mu ^{*}\) uma medida exterior. então, a coleção \(\mathcal{A}\) de conjuntos \(\mu ^{*}\)-mensuráveis é uma \(\sigma \)-álgebra. Se \(\mu \) é restrição de \(\mu ^{*}\) à \(\mathcal{A}\), então \(\mu \) é medida. Além disso,
	\(\mathcal{A}\) é completa.
\end{theorem*}
\begin{proof*}
	Da definição,
	\[
		\mu ^{*}(E) \leq \mu ^{*}(E\cap A) + \mu ^{*}(E\cap A ^{\complement}),\quad \forall E\subseteq X
	\]
	pois \(E = (E\cap A)\cup (E\cap A ^{\complement})\). Assim, basta provar a desigualdade reversa:
	\[
		\mu ^{*}(E) \geq \mu ^{*}(E\cap A) + \mu ^{*}(E\cap A ^{\complement}).
	\]
	Se \(\mu ^{*}(E) = \infty\), acabou. Caso contrário,

	\textbf{\underline{Afirmação}:} \(\mathcal{A}\) é um álgebra.

	Com efeito, \(\emptyset, X\in \mathcal{A}\) pois ambos são mensuráveis. Se \(A\in \mathcal{A}\), então \(A ^{\complement}\in \mathcal{A}\) por simetria. Se \(A, B\in \mathcal{A}\) e \(E\subseteq X\), temos
	\begin{align*}
		\mu ^{*}(E) & = \mu ^{*}(E\cap A) + \mu ^{*}(E\cap A ^{\complement})                                                                                                                \\
		            & = \mu ^{*}(E\cap (A\cap B)) + \mu ^{*}(E\cap A \cap B ^{\complement}) +\mu ^{*}(E\cap A ^{\complement}\cap B) + \mu ^{*}(E\cap A ^{\complement}\cap B ^{\complement}) \\
		            & \geq \mu ^{*}(E\cap (A\cup B)) + \mu ^{*}(E\cap (A\cup B)^{\complement})
	\end{align*}
	Em que usamos que \((A ^{\complement}\cap B ^{\complement}) = (A\cup B) ^{\complement}\) junto de
	\[
		E\cap (A\cup B)\subseteq E\cap (A\cap B)\cup (E\cap(A\cap B^{\complement}))\cup (E\cap(A ^{\complement}\cap B)),
	\]
	pois, sendo \(\mu ^{*}\) uma medida externa, isso resulta em
	\[
		\mu ^{*}(E\cap (A\cup B)) \leq \mu ^{*}(E\cap (A\cap B)) + \mu ^{*}(E\cap (A\cap B ^{\complement})) + \mu ^{*}(E\cap (A ^{\complement}\cap B)).
	\]
	Assim, \(A\cup B\in \mathcal{A}\), mostrando que \(\mathcal{A}\) é um álgebra.

	\textbf{\underline{Afirmação}:} \(\mathcal{A}\) é uma \(\sigma \)-álgebra.

	De fato, dados \(A_{1}, A_{2}, \dotsc \in \mathcal{A}\) dois-a-dois disjuntos e \(E\subseteq X\). Defina \(B_{n} = \bigcup_{i=1}^{n}A_{i}\) e \(B = \bigcup_{i=1}^{\infty}A.\) Temos
	\begin{align*}
		\mu ^{*}(E\cap B_{n}) & = \mu ^{*}(E\cap B_{n}\cap A_{n}) + \mu ^{*}(E\cap B_{n}\cap A_{n}^{\complement} \\
		                      & = \mu ^{*}(E\cap A_{n}) + \mu ^{*}(E\cap B_{n-1}).
	\end{align*}
	Analogamente,
	\[
		\mu ^{*}(E\cap B_{n-1}) = \mu ^{*}(E\cap A_{n-1}) + \mu ^{*}(E\cap B_{n-2}),
	\]
	tal que
	\[
		\mu ^{*}(E\cap B_{n}) \geq \sum\limits_{i=1}^{n}\mu ^{*}(E\cap A).
	\]
	Fazendo \(n\to\infty\) e lembrando que \(\mu ^{*}\) é medida exterior,
	\begin{align*}
		\mu ^{*}(E) & \geq \sum\limits_{i=1}^{\infty}\mu ^{*}(E\cap A_{i}) + \mu ^{*}(E\cap B ^{\complement}) \\
		            & \geq \mu ^{*}(E\cap B) + \mu ^{*}(E\cap B ^{\complement})                               \\
		            & \geq \mu ^{*}(E),
	\end{align*}
	provando que \(B\in \mathcal{A}\). Tome, agora, \(C_{1}, C_{2}, \dotsc \in \mathcal{A}\). Provaremos que \(\bigcap_{i=1}^{\infty}C_{i}\in \mathcal{A}\).
	Defina \(A_{i} = C_i\setminus{\bigcup_{j=1}^{i-1}A_{j}}\). Como cada \(C_{i}\in \mathcal{A}\) e \(\mathcal{A}\) é álgebra, então
	\(A_{i} = C_{i}\cap (C_{1}\cup \dotsc \cup C_{i-1}) ^{\complement}\in \mathcal{A}.\) Da definição dos \(A_{i}\), eles são dois-a-dois
	disjuntos e
	\[
		\bigcup_{i=1}^{\infty}C_{i} = \bigcup_{i=1}^{\infty}A_{i}.
	\]
	Por outro lado,
	\[
		\bigcap_{i=1}^{\infty}C_{i} = \biggl(\bigcup_{i=1}^{\infty}C_{i}\biggr)^{\complement}\in \mathcal{A},
	\]
	mostrando que \(\mathcal{A}\) é \(\sigma \)-álgebra.

	Restra mostrar que a restrição de \(\mu ^{*}\) é medida. De fato, dados \(A_{1}, A_2, \dotsc \in \mathcal{A}\) dois-a-dois disjuntos,
	temos
	\[
		\mu ^{*}(B) = \sum\limits_{i=1}^{\infty}\mu ^{*}(B\cap A_{i}) + \mu ^{*}(B\cap B ^{\complement}) = \sum\limits_{i=1}^{\infty}\mu ^{*}(A_{i}).
	\]
	Mas, \(B = \bigcup_{i=1}^{\infty}A_{i}\). Logo, \(\mu ^{*}\) é aditiva e contável sobre \(\mathcal{A}\). Finalmente, se \(\mu ^{*}(A) = 0\) e \(E\subseteq X\), então
	\[
		\mu ^{*}(E\cap A) + \mu ^{*}(E\cap A ^{\complement}) = \mu ^{*}(E\cap A ^{\complement}) \leq \mu ^{*}(E).
	\]
	Como a reversa sempre vale, temos \(A\in \mathcal{A}\), mostrando que \(\mathcal{A}\) contém todos os conjuntos nulos. \qedsymbol
\end{proof*}
Seja \(X = \mathbb{R}\) e \(\mathcal{C}\) a coleção de todos os intervalos da forma \((a, b]\) e seja \(\alpha: \mathbb{R}\Longleftrightarrow \mathbb{R}\) crescente e contínua à direita. Assim, para cada x,
\[
	\alpha (x) = \lim_{y\to x^{+}}\alpha (y),\quad \forall x \quad\&\quad \alpha (x) < \alpha (y)\text{ se }x < y.
\]
Seja \(\ell (I) = \alpha (b) - \alpha (a)\) e defina \(m^{*}\) como antes. Pela proposição de antes, \(m^{*}\) é medida exterior. Ainda mais, pelo Teorema de Caratheodory, \(m^{*}\) é medida sobre
a coleção dos conjuntos \(m^{*}\)-mensuráveis. Note que, se K e L são adjacentes, digamos \(K = (a, b], L = (b, c]\), então \(K\cup L = (a, c]\) e
\[
	\ell (K) + \ell (L) = [\alpha (b) - \alpha (a)] + [\alpha (c) - \alpha (b)] = \alpha (c) - \alpha (a) = \ell (K\cup L).
\]
Provaremos que a medida de \((e, f]\) é, de fato, \(\alpha (f) - \alpha (e).\)
\begin{lemma*}
	Seja \(J_{k} = (a_{k}, b_{k}), k = 1, 2, \dotsc , n\) uma coleção finita de intervalos abertos limitados que cobrem \([C, D].\) Então,
	\[
		\sum\limits_{k=1}^{n}[\alpha (b_{k}) - \alpha (a_{k})] \geq \alpha (D) - \alpha (C).
	\]
\end{lemma*}
\begin{proof*}
	Sendo \(\{J_{k}\}\) uma cobertura de [C, D], existe pelo menos um intervalo, digamos \(J_{k_1} = (a_{k_{1}}, b_{k_{1}})\) tal que \(C\in J_{k_{1}}\). Caso \(J_{k_{1}}\) cubra [C, D],
	não resta nada a ser provado.
	Caso contrário, \(b_{k_{1}} \leq D\) e existe um intervalo, digamos \(J_{k_{2}} = (a_{k_{2}}, b_{k_{2}})\) tal que \(b_{k_{1}}\in J_{k_{2}}.\) Se \(J_{k_{1}}\cup J_{k_{2}}\)
	cobrir [C, D], então acabamos novamente.
	Caso contrário, \(b_{k_{1}} < b_{k_{2}} \leq D\) e existe um intervalo, digamos \(J_{k_{3}} = (a_{k_{3}}, b_{k_{3}})\) satisfazendo \(b_{k_{2}}\in J_{k_{3}}.\)
	Se \(J_{k_{1}}\cup J_{k_{2}}\cup J_{k_{3}}\) cobrir o intervalo [C, D], acabamos.

	Continuando este processo, existe \(J_{k_{m}}\) tal que \(b_{k_{m-1}}\in J_{k_{m}}\), tal que \(J_{k_{1}}\cup \dotsc \cup J_{k_{m}}\) cobrem [C, D]. Sendo \(\{J_{k}\} \) uma cobertura finita, paramos
	o processo para \(m \leq n\).

	Com esta construção, obtivemos
	\[
		a_{k_{1}} < C < b_{k_{1}},\quad a_{k_{m}} < D < b_{k_{m}}, \quad a_{k_{j}} < B_{k_{j-1}} < b_{k_{j}},\quad (j = 2, 3, \dotsc , m).
	\]
	Usando as desigualdades acima, chegamos em
	\begin{align*}
		\alpha (D) - \alpha (C) & \leq \alpha (b_{k_{m}}) - \alpha (a_{k_{1}})                                                          \\
		                        & = \alpha (b_{k_{m}}) - \alpha (b_{k_{m-1}}) + \alpha (b_{k_{m-1}}) - \alpha (b_{k_{m-2}}) + \dotsc    \\
		                        & \quad +\alpha (b_{k_{2}}) - \alpha (b_{k_{1}}) + \alpha (b_{k_{1}}) - \alpha (a_{k_{1}})              \\
		                        & \leq [\alpha (b_{k_{m}} - \alpha (a_{k_{m}})] + [\alpha (b_{k_{m-1}} - \alpha (a_{k_{m-1}})] + \dotsc \\
		                        & \quad + [\alpha (b_{k_{2}}) - \alpha (a_{k_{2}})] + [\alpha (b_{k_{1}}) - \alpha (a_{k_{1}})].
	\end{align*}
	Portanto, como \(\{J_{k_{1}},\dotsc ,J_{k_{m}}\}\subseteq \{J_{1},\dotsc ,J_{n}\}\), a prova da desigualdade desejada está completa. \qedsymbol
\end{proof*}
\newpage

\section{Aula 03 - 10/01/2024}
\subsection{Motivações}
\begin{itemize}
	\item A Medida de Lebesgue-Stieltjes;
	\item Conjunto de Cantor;
	\item Conjuntos não-mensuráveis;
	\item Teorema de Extensão de Caratheodory.
\end{itemize}
\subsection{Medida de Lebesgue-Stieltjes}
Ao fim da aula passada, provamos o seguinte lema:
\begin{lemma*}
	Seja \(J_{k} = (a_{k}, b_{k}), k = 1, 2, \dotsc , n\) uma coleção finita de intervalos abertos limitados que cobrem \([C, D].\) Então,
	\[
		\sum\limits_{k=1}^{n}[\alpha (b_{k}) - \alpha (a_{k})] \geq \alpha (D) - \alpha (C).
	\]
\end{lemma*}
Porém, umas das nossas afirmações era que a medida exterior de um intervalo é dada pela medida de Lebesgue-Stieltjes do mesmo:
\begin{prop*}
	Se \(I = (e, f]\) é um intervalo limitado, então
	\[
		m^{*}(I) = \ell (I).
	\]
\end{prop*}
\begin{proof*}
	Mostraremos primeiro a parte fácil, ou seja, \(m^{*}(I) \leq \ell (I)\). Seja \(A_{1} = I\) e \(A_2 = A_3 = \dotsc = \emptyset .\) Então, \(I\subseteq \bigcup_{i=1}^{\infty}A_{i}\) e
	\[
		m^{*}(I) \leq \sum\limits_{i=1}^{\infty}\ell (A_{i}) = \ell (A_1) = \ell (I).
	\]
	Para o outro lado, suponha \(I\subseteq \bigcup_{i=1}^{\infty}A_{i},\) em que \(A_{i} = (c_{i}, d_{i}].\) Dado \(\varepsilon > 0\), da continuidade à direita de \(\alpha \), podemos escolher
	\(C\in (e, f)\) tal que \(\alpha (C) - \alpha (e) < \frac{\varepsilon }{2}.\) Seja \(D = f\) e, para cada i, escolha \(d_{i}^\prime > d_{i}\) tal que \(\alpha (d_{i}^\prime) - \alpha (d_{i}) < \frac{\varepsilon }{2^{i+1}}.\) Defina
	\(B_{i} = (c_{i}, d_{i}^\prime).\) Assim, \(\{B_{i}\}\) é uma cobertura por abertos para o compacto [C, D]. Logo, podemos escolher uma cobertura por abertos finitos, digamos \(\{J_{1}, \dotsc , J_{n}\}\) de \(\{B_{i}\}\).

	Ao aplicar o Lema anterior, obtemos
	\begin{align*}
		\ell (I) \leq \alpha (D) - \alpha (C) + \frac{\varepsilon }{2} & \leq \sum\limits_{k=1}^{n}(\alpha (d_{k}\prime) - \alpha (c_{k})) + \frac{\varepsilon }{2} \\
		                                                               & \leq \sum\limits_{k=1}^{\infty}\ell (A_{k}) + \varepsilon,
	\end{align*}
	em que utilizamos a seguinte relação: Se \(I = (e, f]\), então \(\ell (I) = \alpha (f) - \alpha (e) = \alpha (D) - \alpha (e) = \alpha (D) - \alpha (C) + \alpha (C) - \alpha (e)\) e \(A = (c_{i}, d_{i}),\) tal que
	\(\ell (A_{i}) = \alpha (d_{i}) - \alpha (c_{i})\), tal que \([\alpha (d_{k}^\prime) - \alpha (c_{k})]  = [\alpha (d_{k}^{\prime}) - \alpha (d_{k})] + [ \alpha (d_{k}) - \alpha (c_{k})] .\)

	Tomando o ínfimo sobre todas as coleções contáveis \(\{A_{i}\}\) que cobrem I, obtemos
	\[
		\ell (I) \leq m^{*}(I) + \varepsilon .
	\]
	Portanto, como \(\varepsilon \) é arbitrário, chegamos em
	\[
		\ell (I) \leq m^{*}(I).\quad \text{\qedsymbol}
	\]
\end{proof*}
Na construção de medida de Lebesgue-Stieltjes com relação a \(\alpha \), há um passo muito importante que vale ser mencionado.
\begin{prop*}
	Todo conjunto em \(\sigma \)-álgebra de Borel em \(\mathbb{R}\) é \(m^{*}\)-mensurável.
\end{prop*}
\begin{proof*}
	Sendo a coleção dos conjuntos \(m^{*}\)-mensuráveis uma \(\sigma\)-álgebra, basta mostrar que todo intervalo J da forma (c, d] é \(m^{*}\)-mensurável, ou seja,
	\[
		m^{*}(E) = m^{*}(E\cap J) + m^{*}(E\cap J ^{\complement}),\quad E\subseteq X.
	\]
	Para isso, precisamos provar apenas um laad da desigualdade, o lado
	\[
		m^{*}(E) \geq m^{*}(E\cap J) + m^{*}(E\cap J ^{\complement}).
	\]
	Além disso, isto é trivialmente verdade para \(m^{*}(E) = \infty\), então podemos tomar \(E\subseteq X\) com \(m^{*}(E) < \infty\). Escola \(I_{1}, I_2, \dotsc \), da forma
	\(I_{i} = (a_{i}, b_{i}], i = 1, 2, \dotsc \), tal que \(E \subseteq \bigcup_{i=1}^{\infty}I_{i}.\) Da definição de ínfimo,
	\[
		m^{*}(E) \geq \sum\limits_{i=1}^{\infty}(\alpha (b_{i}) - \alpha (a_{i})) - \varepsilon .
	\]
	Como \(E \subseteq \bigcup_{i=1}^{\infty}I_{i},\) temos
	\[
		m^{*}(E\cap J) \leq \sum\limits_{i=1}^{\infty}m^{*}(I_{i}\cap J)\quad\&\quad m^{*}(E\cap J ^{\complement}) \leq \sum\limits_{i=1}^{\infty}m^{*}(I_{i}\cap J ^{\complement}).
	\]
	Somando, temos
	\[
		m^{*}(E\cap J) + m^{*}(E\cap J ^{\complement}) \leq \sum\limits_{k=1}^{\infty}[m^{*}(I_{i}\cap J) + m^{*}(I_{i}\cap J ^{\complement})]
	\]
	Como \(J = (c, d],\) temos \(J ^{\complement} = K_1 \cup K_2,\) em que \(K_1 = (-\infty, c]\) e \(K_2 = (d, \infty).\) Além disso, \(I_{i}\cap J, I_{i}\cap K_1\) e \(I_{i}\cap K_2\) são intervalos
	abertos à esquerda e fechados à direita, eventualmente vazios. Usando que \(\ell (K\cup L) = \ell (K) + \ell (L),\) Temos
	\begin{align*}
		m^{*}(I_{i}\cap J) + m^{*}(I_{i}\cap J ^{\complement}) & \leq m^{*}(I_{i}\cap K_1) + m^{*}(I_{i}\cap J) + m^{*}(I_{i}\cap K_2) \\
		                                                       & = \ell (I_{i}\cap K_1) + \ell (I_{i}\cap J) + \ell(I_{i}\cap K_2)     \\
		                                                       & \leq \ell (I_{i}) = m^{*}(I_{i}).
	\end{align*}
	Assim,
	\[
		m^{*}(E\cap J) + m^{*}(E\cap J ^{\complement}) \leq \sum\limits_{i=1}^{\infty}m^{*}(I_{i}) \leq m^{*}(E) + \varepsilon .
	\]
	Portanto, como \(\varepsilon \) é arbitrário, a prova está acabada. \qedsymbol
\end{proof*}
Valem algumas observações. Primeiramente, para Lebesgue-Stieltjes, denotaremos por apenas \(m\) ao invés de \(m^{*}.\) Quando \(\alpha (x) = x,\) m é medida de Lebesgue, e os conjuntos
\(m^{*}-\)mensuráveis serão chamados de Lebesgue \(\sigma \)-álgebra, de forma que um conjunto é Lebesgue mensurável se ele é um elementos da \(\sigma \)-álgebra de Lebesgue. Finalmente,
dada uma medida \(\mu \) sobre \(\mathbb{R}\) tal que \(\mu (K) < \infty\) para K compacto, defina \(\alpha (x) = \mu ((0, x])\) se \(x \geq 0\) e \(\alpha (x) = -\mu ((x, 0])\) se \(x < 0\). Então,
\(\alpha \) é crescente e contínua à direita. Pode-se provar que essa medida \(\mu \) é medida de Lebesgue-Stieltjes.
\begin{example}
	Seja m mediad de Lebesgue. Se \(x\in \mathbb{R}\), então \(\{x\}\) é fechado, logo Borel Mensurável. Além disso,
	\[
		m(\{x\}) = \lim_{n\to \infty}m \biggl(\biggl(x - \frac{1}{n}, x\biggr]\biggr) = \lim_{n\to \infty}\biggl(x - x + \frac{1}{n}\biggr) = 0,
	\]
	o que implica que
	\[
		m([a, b]) = m((a, b])) + m(\{a\}) = b-a + 0 = b-a
	\]
	e
	\[
		m((a, b)) = m((a, b]) - m(\{b\}) = b-a - 0 = b-a.
	\]
	Conclui-se, por este raciocínio, que \(m(A) = 0\) sempre que A é um conjunto enumerável.
\end{example}
Apesar de contra-intuitivo, existem conjuntos não enumeráveis com medida de Lebesgue NULA. Um exemplo clássico disso é o \textbf{Conjunto de Cantor}.
\begin{example}[Conjunto de Cantor]
	O conjunto de Cantor é construído da seguinte maneira: Sejam
	\begin{align*}
		 & F_{0} = [0, 1]                                                                                                                                                                                  \\
		 & F_{1} = F_{0}\setminus{\biggl(\frac{1}{3}, \frac{2}{3}\biggr)}\quad \text{Removido o terço médio}                                                                                               \\
		 & F_{2} = F_{1}\setminus{\biggl[\biggl(\frac{1}{3^{2}}, \frac{2}{3^{2}}\biggr)\cup \biggl(\frac{7}{3^{2}}, \frac{8}{3^{2}}\biggr)\biggr]}\quad \text{Removido o terço médio de cada subintervalo} \\
		 & \vdots
	\end{align*}
	Então, o conjunto \(C\equiv \bigcap_{n=0}^{\infty}F_{n}\) é o chamado \textbf{conjunto de Cantor}. Ele é fechado, não-enumerável, ele não contém intervalos e todo ponto deste conjunto é ponto de acumulação.
	Note que a medida de \(F_{1}\) é
	\[
		\mu (F_{1}) = \mu \biggl([0, 1]\setminus{\biggl(\frac{1}{3}, \frac{2}{3}\biggr)}\biggr) = \mu \biggl([0, \frac{1}{3})\cup (\frac{2}{3}, 1]\biggr) = \mu \biggl([0, \frac{1}{3}]\biggr) + \mu \biggl(\frac{2}{3}, 1]\biggr) = 1-\frac{1}{3} = \frac{2}{3}.
	\]
	A medida de \(F_{2}\) é \(\frac{2^{2}}{3^{2}},\) a de \(F_{3}\) é \(\frac{2^{3}}{3^{3}}\) e a medida de \(F_{n}\) é, por indução, \(\frac{2^{n}}{3^{n}} = \biggl(\frac{2}{3}\biggr)^{n}.\) Sendo C a interseção de todos eles, segue que
	\[
		\mu (C) = \mu \biggl(\lim_{n\to \infty}\bigcap_{i=1}^{n}F_{i}\biggr) = 0.
	\]
  Outra forma de construir este conjunto é por meio das \textbf{funções de Cantor}. Vamos definí-las. 

  Comece colocando \(f_{0} = \frac{1}{2}\) em \(\biggl(\frac{1}{3}, \frac{2}{3}\biggr)\). Continuamos definindo como \(f_{0} = \frac{1}{2^{3}}\) em \(\biggl(\frac{1}{3^{2}}, \frac{2}{3^{2}}\biggr)\), \(f_{0}=\frac{3}{2^{2}}\) em \(\biggl(\frac{7}{3^{2}}, \frac{8}{3^{2}}\biggr),\)
 \(f_{0} = \frac{1}{2^{3}}\) em \(\biggl(\frac{1}{3^{3}}, \frac{2}{3^{3}}\biggr)\), \(f_{0} = \frac{3}{2^{3}}\) em \(\biggl(\frac{7}{3^{3}}, \frac{8}{3^{3}}\biggr)\), \(f_{0} = \frac{5}{2^{3}}\) em \(\biggl(\frac{19}{3^{2}}, \frac{20}{3^{3}}\biggr),\) \(f_{0} = \frac{7}{2^{3}}\) 
 em \(\biggl(\frac{25}{3^{3}}, \frac{26}{3^{3}}\biggr), \dotsc \) e constante nos intervalos omitidos. 

 Agora, defina 
   \[
     f(x) = \inf_{}\{f_{0}(y): y \geq x, y\not\in C\},\quad x < 1.
   \]
   Esta f é crescente, \(f = f_{0}\) nos intervalos omitidos, \(f(1) = 1\) e f tem saltos pelo menos fora de C. Vemos que f é crescente no conjunto de Cantor C, que tem medida nula e 
   constante fora de \(C(f_{0})\), tal que f é contínua, onde definimos f nos pontos de \(x\in C\) como senod o limite dos valores laterais \(f(y)\) quando \(y\in C ^{\complement},y\to x\). De fato, 
   se \(f(x^{-}) < f(x^{+})\) denotam os limites laterais em x, então existe um racional \(\frac{k}{2^{n}}, k \leq 2^{n}\) que não está na imagem de f, mas, por construção, cada valor da forma \(\biggl\{\frac{k}{2^{n}}: k \leq 2^{n}\biggr\}\) é 
   assumido por \(f_{0}\), provando a continuidade de f.
\end{example}
\begin{example}[Conjunto de Cantor Generalizado]
  Esse conjunto tem a propriedade de ter medida \(\frac{1}{2}.\) Ao invés de retirar um terço médio, retira-se um quarto médio. No primeiro passo, é retirado um intervalo de tamanho \(\frac{1}{4}\), no segundo dois de \(\frac{1}{4^{2}}\) de tamanho, no 
terceiro passo retira-se quatro de tamanho \(\frac{1}{4^{3}},\) etc. Tal que o comprimento total dos intervalos será 
  \[
    \sum\limits_{n=1}^{\infty}\frac{2^{n-1}}{4^{n}} = \frac{1}{2}
  \]
\end{example}
\begin{example}
  Sejam \(q_1, q_2, \dotsc \) enumeração dos racionais e \(\varepsilon > 0\) dado. Coloque \(I_{i} = \biggl(q_{i} - \frac{\varepsilon }{2^{i}}, q_{i} + \frac{\varepsilon }{2^{i}}\biggr)\), tal que \(|I_{i}| = \frac{\varepsilon }{2^{i-1}}.\) 
  A medida de \(\bigcup_{i}^{}I_{i}\) é no máximo \(2\varepsilon\). Assim, \(A = [0, 1]\setminus{\bigcup_{i}^{}I_{i}}\) é maior que \(1 - 2\varepsilon \) e não contém racionais.
\end{example}
\begin{prop*}
  Suponha \(A\subseteq [0, 1]\) e que A é Lebesgue-mensurável. Seja m a medida de Lebesgue. 
 \begin{itemize}
   \item[1)] Dado \(\varepsilon > 0\), existe um aberto G tal que \(m(G\setminus{A}) < \varepsilon \) e \(A\subseteq G\);
     \item[2)] Dado \(\varepsilon > 0\), existe um fechado F tal que \(m(A\setminus{F}) < \varepsilon \) e \(A\subseteq G\);
       \item[3)] Existe um conjunto \(H \supseteq A\) que é uma interseção enumerável de uma sequência de abertos decrescentes e \(m(H\setminus{A}) = 0\). Denotamos eles por \(G_\delta \);
         \item[4)] Existe um conjunto \(F\subseteq A\) que é união enumerável de uma sequência crescente de fechados e \(m(A\setminus{F}) = 0\). Denotamos por \(F_{\sigma }\).
 \end{itemize}
\end{prop*}
\begin{proof*}
  A 1 segue da deifnição de m. Existe \(E = \bigcup_{j=1}^{\infty}(a_{j}, b_{j}]\) tal que \(A\subseteq E\) e \(m(E\setminus{A}) < \frac{\varepsilon }{2},\) em que usamos que \(m(E) = m(E\setminus{A}) + m(A),\) sendo m(A) o ínifmo de m(E), 
  \(m(A)\sim m(E).\) Seja \(G = \bigcup_{j=1}^{\infty}(a_{j}, b_{j}+\varepsilon 2^{-j-1}.\) Então, G é aberto e contém A, e 
    \[
      m(G\setminus{A}) < \sum\limits_{j=1}^{\infty}\varepsilon 2^{-j-1} = \frac{\varepsilon }{2}.
    \]
  Portanto, 
    \[
      m(G\setminus{A})\leq m(G\setminus{E}) + m(E\setminus{A}) < \varepsilon .
    \]

    A 2 segue, usando a primeira parte, tomando G aperto tal que \(m(G\setminus{A}^{\prime}) < \varepsilon \) e \(A\prime \subseteq G\), em que \(A\prime = [0, 1]\setminus{A}.\) Seja \(F = [0, 1]\setminus{G}\), tal que F é fechado, \(F\subseteq A\)
e 
  \[
    m(A\setminus{F}) \leq m(G\setminus{A}^{\prime}) < \varepsilon .
  \]
  Aqui, foi usado que \(A\setminus{F})\subseteq (G\setminus{A}^{\prime})\).

  Quanto aos itens 3 e 4, usando a primeira parte, tome um aberto \(G_{i}\) par acada i tal que \(m(G_{i}\setminus{A}) < 2^{-i}\) e \(A\subseteq G_{i}\). Então, \(H_{i} = \bigcap_{j=1}^{i}G_{j}\supseteq A\), é aberto, é contido em \(G_{i}\) e 
    \[
      m(H_{i}\setminus{A}) < 2^{-i}.
    \]
  Tome \(H = \bigcap_{i=1}^{\infty}H_{i},\) sendo H não necessariamente aberto, mas interseção enumerável deles. O conjunto H é um conjunto de Borel que contém A e que satisfaz 
    \[
      m(H\setminus{A}) \leq m(H_{i}\setminus{A}) < 2^{-i}
    \]
  para cada i e, portanto, 
    \[
      m(H\setminus{A}) = 0.
    \]
    Isto basta para o item 3. Finalmente, se \(A^{\prime} = [0, 1]\setminus{A}\), aplique 3 para \(A^{\prime}\) para obter H contendo \(A^{\prime}\), que é a interseção enumerável de sequências decrescentes de abertos tais que 
   \(m(H\setminus{A^{\prime}}) = 0.\) Seja \(J = [0, 1]\setminus{H}\) fechado tal que \(J = [0, 1]\cap H ^{\complement} = \cup ([0,1]\cap H_{i}^{\complement}) \subseteq A\). Então, 
     \[
       m(A\setminus{J}) \leq m(H\setminus{A^{\prime}}),
     \]
     pois \(A\setminus{J} \subseteq H\setminus{A^{\prime}.}\) Portanto, tomando \(J = F\) finaliza a prova. \qedsymbol
\end{proof*}
\begin{crl*}
  Seja \(\mu \) uma medida de Lebesgue-Stieltjes sobre a reta \(\mathbb{R}\). Então, as conclusões das propsições anteriores valem para \(\mu \) no lugar de m.
\end{crl*}
\begin{proof*}
  Sejam A e \(E = \bigcup_{j}^{}(a_{j}, b_{j}]\) escolhidos na prova do primeiro item, com m trocado por \(\mu \). Podemos escolher \(c_{j} > b_{j}\) tal que 
    \[
      \mu ((a_{j}, c_{j})) \leq \mu ((a_{j}, b_{j}]) + \varepsilon 2^{-j-1}.
    \]
  Tome \(G = \bigcup_{j=1}^{\infty}(a_{j}, c_{j})\) e, da construção, \(E\subseteq G\), além de que 
    \[
      \mu (G\setminus{E}) \leq \sum\limits_{j=1}^{\infty}(\mu ((a_{j}, c_{j})) - \mu ((a_{j}, b_{j}]) \leq \sum\limits_{j=1}^{\infty}\varepsilon 2^{-j-1} = \frac{\varepsilon }{2}.
    \] 
  Como na prova do item 1, temos \(A\subseteq E\) e \(\mu (E\setminus{A}) < \frac{\varepsilon }{2}\) e, da inclusão, \((G\setminus{A})\subseteq (G\setminus{E})\cup (E\setminus{A}),\) donde segue que 
    \[
      \mu (G\setminus{A}) < \frac{\varepsilon }{2}.
    \]
  Por fim, basta proceder como na prova da proposição. \qedsymbol
\end{proof*}
\begin{theorem*}
  Seja \(m^{*}\) definida por 
    \[
      \mu ^{*}(E) = \inf_{}\biggl\{\sum\limits_{i=1}^{\infty}\ell (A_{i}): A_{i}\in \mathcal{C}, E \subseteq \bigcup_{i=1}^{\infty}A_{i}\biggr\},
    \]
    em que \(\mathcal{C}\) é a coleção de intervalos da forma (a, b] e \(\ell ((a, b]) = b-a\). Então, \(m^{*}\) não é uma medida sobre a coleção dos subconjuntos de \(\mathbb{R}.\)
\end{theorem*}
\begin{proof*}
  Suponha que \(m^{*}\) seja medida e defina a relação 
    \[
      x\sim y \text{ se } x-y\in \mathbb{Q}.
    \]
    Então, \(\sim\) é relação de equivalência em [0, 1]. Para cada classe de equivalência, escolha um representante chamado A. Mostraremos que A não é \(m^{*}\)-mensurável.

    Dado B, defina 
      \[
        B + x = \{y + x: y\in B\}.
      \]
    Note que 
      \[
        \ell ((a+q, b+q)) = b-a = \ell ((a, b]),\quad \forall a, b, q.
      \]
    Da definição de \(m^{*}\), 
      \[
        m^{*}(A + q) = m^{*}(A),\quad \forall A, q.
      \]
    Note que os conjuntos A + q são disjuntos para diferentes racionais q. De fato, se \(x = a + q = a'+ q',\) então \(a, a'\in A,\) ou seja, \(a - a'= q'-q\in \mathbb{Q}, \) o que significa que 
    \(a \sim a'\) e \(q = q'\). Agora,
      \[
        [0, 1]\subseteq \bigcup_{q\in \mathbb{Q}\cap [-1, 1]}^{}(A+q),
      \]
    pois dado \(x\in [0,1]\) com x equivalente a a, então \(x-a = q\in \mathbb{Q}\). Da inclusão, temos 
      \[
        1 \leq \sum\limits_{q\in [-1, 1]\cap \mathbb{Q}}^{}m^{*}(A+q),
      \]
    tal que \(m^{*}(A) = m^{*}(A+q) > 0\), mas 
      \[
        \bigcup_{q\in [-1, 1]\cap \mathbb{Q}}^{}(A+q)\subseteq [-1, 2] \Rightarrow 3 \geq \sum\limits_{q\in [-1, 1]\cap \mathbb{Q}}^{}m^{*}(A+q),
      \]
    do que segue que 
      \[
        m^{*}(A) = 0
      \]
    pois a série converge. Absurdo. \qedsymbol
\end{proof*}
\subsection{Teorema de Extensão de Caratheodory}
  Essa ferramente abstrata permite a construção de novas medidas. Seja \(\mathcal{A}_{0}\) uma álgebra (não necessariamente \(\sigma \)-álgebra). Seja \(\ell \) uma medida sobre \(\mathcal{A}_{0}\), chamada de \textbf{pré-medida}, satisfazendo 
quase todas as propriedades de medida: 
\begin{itemize}
  \item[1)]\(\ell (\emptyset ) = 0\)
    \item[2)] Se \(A_1, A_2, \dotsc \) são elementos de \(\mathcal{A}_{0}\) dois-a-dois disjuntos e \(\bigcup_{i}^{}A_{i}\in \mathcal{A}_{0}\), então 
      \[
        \ell (\bigcup_{i=1}^{\infty}A_{i}) =\sum\limits_{i=1}^{\infty}\ell (A_{i}).
      \]
\end{itemize}
  Denotamos por \(\sigma (\mathcal{A}_{0})\) a \(\sigma \)-álgebra gerada por \(\mathcal{A}_{0}.\)
\begin{theorem*}
  Suponha \(\mathcal{A}_{0}\) uma álgebra e \(\ell :\mathcal{A}_{0}\rightarrow [0, \infty]\) é uma medida sobre \(\mathcal{A}_{0}\), defina 
    \[
      \mu ^{*}(E) = \inf_{}\biggl\{\sum\limits_{i=1}^{\infty}\ell (A_{i}): A_{i}\in \mathcal{A}_{0}, E\subseteq \bigcup_{i=1}^{\infty}A_{i}\biggr\}, \quad E\subseteq X,
    \]
    então 
   \begin{itemize}
    \item[i)] \(m^{*}\) é medida exterior
    \item[ii)] \(\mu ^{*}(A) = \ell (A)\) se \(A\in \mathcal{A}_{0}\)
    \item[iii)] Para todo conjunto em \(\mathcal{A}_{0}\) e todo conjunto de \(\mu^{*}\)-medida não nula, eles são mensuráveis.
    \item[iv)] Se \(\ell \) é \(\sigma \)-finita, então existe uma única extensão a \(\sigma (\mathcal{A}_{0}).\)
   \end{itemize}
\end{theorem*}
\begin{proof*}
  O item (1) já foi feito. Para o 2, suponha que \(E\in \mathcal{A}_{0}\). Tomando \(A_1 = E, A_2 = A_3 = \dotsc = \emptyset \), da definição de \(\mu ^{*}\) segue que 
    \[
      \mu ^{*}(E) \leq \ell (E).
    \]
    Se \(E\subseteq \bigcup_{i=1}^{\infty}A_{i},\) com \(A_{i}\in \mathcal{A}_{0}\), seja 
      \[
        B_{n} = E\cap (A_{n}\setminus{(\bigcup_{i=1}^{\infty}A_{i}})).
      \]
    Como \(B_{n} = E\cap (A_{n}\cap (\bigcup_{j=1}^{i=1}A_{j})^{\complement})\), segue que \(B_{n}\in \mathcal{A}_{0}\) e são dois a dois disjuntos. Além disso, \(E = \bigcup_{i=1}^{\infty}B_{i}.\) Logo, 
  como \(B_{n}\subseteq A_{n}\), temos 
    \[
      \ell (E) = \sum\limits_{i=1}^{\infty}\ell (B_{i}) \leq \sum\limits_{i=1}^{\infty}\ell (A_{i}).
    \]
    Tomando o ínfimo sobre toda sequência \(A_1, A_2, \dotsc \), obtemos 
      \[
        \ell (E) \leq \mu ^{*}(E).
      \]

  Quanto ao item 3, suponha \(A\in \mathcal{A}_{0}\) e sejam \(\varepsilon > 0\), \(E\subseteq X\). Tome \(B_1, B_2, \dotsc \in \mathcal{A}_{0}\) tal que \(E\subseteq \bigcup_{i=1}^{\infty}B_{i}\) e \(\sum\limits_{i}^{}\ell (B_{i}) \leq \mu ^{*}(E) + \varepsilon .\) Então, 
 \begin{align*}
   \mu ^{*}(E) \geq \sum\limits_{i}^{}\ell (B_{i}) &= \sum\limits_{i}^{}\ell (B_{i}\cap A) + \sum\limits_{i}^{}\ell (B_{i}\cap A ^{\complement})\\
                                                   &\geq \mu ^{*}(E\cap A) + \mu ^{*}(E\cap A ^{\complement}).
 \end{align*}
 Sendo \(\varepsilon \) arbitrário, temos 
   \[
     \mu^{*}(E) \geq \mu ^{*}(E\cap A) + \mu ^{*}(E\cap A ^{\complement}).
   \] 
   Assim, \(A\) é \(\mu ^{*}\)-mensurável. Agora, da definição, \(A\subseteq B\) implica que \(\mu ^{*}(A) \leq \mu ^{*}(B)\) e, se \(\mu ^{*}(A) = 0\) com \(E\subseteq X\), 
  temos 
    \[
      0 \leq \mu ^{*}(E) \leq \mu ^{*}(E\cap A) + \mu ^{*}(E\cap A ^{\complement}) = \mu ^{*}(E\cap A) \leq \mu ^{*}(A) = 0,
    \]
    ou seja, vale a igualdade e \(A\) é \(\mu ^{*}\)-mensurável.

   Finalmente, quanto ao item 4, suponha que existam duas extensões \(\mu ^{*}\) e \(\nu \) para \(\sigma (\mathcal{A}_{0}),\) a menor \(\sigma \)-álgebra contendo \(\mathcal{A}_{0}.\) Suponha que 
   \(\mu ^{*}\) é medida finita. Pela definição dela, podemos supor que \(\ell \) é finita, de forma que o conjunto \(\mu ^{*}\)-mensuráveis formam uma \(\sigma \)-álgebra contendo \(\mathcal{A}_{0},\) pois, se \(E \in \sigma (\mathcal{A}_{0}),\) então E deve ser \(\mu ^{*}\)-mensurável. 
   Como \(\ell \) é finita, podemos definir 
     \[
       \mu ^{*}(E) = \inf_{}\biggl\{\sum\limits_{i=1}^{\infty}\ell (A_{i}), A_{i}\in \mathcal{A}_{0}, E\subseteq \bigcup_{i=1}^{\infty}A_{i}\biggr\}.
     \]
    Pelo item 2, \(\ell = \nu\) sobre \(\mathcal{A}_{0}\), de forma que 
      \[
        \sum\limits_{i}^{}\ell (A_{i}) = \sum\limits_{i}^{}\nu(A_{i}).
      \]
    Logo, se \(E\subseteq \bigcup_{i=1}^{\infty}A_{i}\), \(A_{i}\in \mathcal{A}_{0}\), tal que 
      \[
        \nu(E) \leq \sum\limits_{i}^{}\nu(A_{i}) = \sum\limits_{i}^{}\ell (A_{i}),
      \]
    resultando em 
      \[
        \nu(E) \leq \mu ^{*}(E).
      \]
    Para provar a desigualdade reversa, seja \(\varepsilon > 0\) e escolha \(A_{i}\in \mathcal{A}_{0}\) tal que 
      \[
        \mu ^{*}(E) + \varepsilon  \geq \sum\limits_{i}^{}\ell (A_{i})\text{ e } E\subseteq \bigcup_{i}^{}A_{i}.
      \]
    Seja \(A = \bigcup_{i=1}^{\infty}A_{i}\) e \(B_{k} = \bigcup_{i=1}^{k}A_{i}.\) Observe que 
      \[
        \mu ^{*}(E) + \varepsilon  \geq \sum\limits_{i}^{}\ell (A_{i}) = \sum\limits_{i}^{}\mu ^{*}(A_{i}) \geq \mu ^{*}(\bigcup_{i}^{}A_{i}) = \mu ^{*}(A).
      \]
    Consequentemente, \(\mu ^{*}(A\setminus{E}) < \varepsilon ,\) pois \(A = (A\setminus{E})\cup E\). Agora, a partir do segundo item do teorema, temos 
      \[
        \mu ^{*}(A) = \lim_{k\to \infty}\mu ^{*}(B_{k}) = \lim_{k\to \infty}\nu^{*}(B_{k}) = \eta (A).
      \]
    Como \(E\subseteq A\), 
   \begin{align*}
     \mu ^{*}(E) \leq \mu ^{*}(A) = \nu(A) &=\nu(E) + \nu(A\setminus{E})\\ 
                                           &\leq \nu(E) + \mu ^{*}(A\setminus{E})\\ 
                                           &\leq \nu(E) + \varepsilon .
   \end{align*}
   Como \(\varepsilon \) é arbitrário, a prova está completa. Resta o caso em que \(\ell \) é \(\sigma \)-finita. Escreva \(X = \bigcup_{i}^{}K_{i}, \) em que \(K_{i}\uparrow X\) e 
   \(\ell (K_{i}) < \infty\)  para todo i. No passo anterior, temos unicidade para a medida restrita a \(\ell_{i} = \ell (A\cap K_{i}). \) Se \(\mu \) e \(\nu\) são duas extensões de \(\ell \) e \(A\in \sigma (\mathcal{A}_{0}),\) então 
     \[
       \mu (A) = \lim_{i\to \infty}\mu (A\cap K_{i}) = \lim_{i\to \infty}\ell_{i}(A) = \lim_{i\to \infty}\nu(A\cap K_{i}) = \nu(A).
     \]
    Portanto, \(\mu = \nu.\) \qedsymbol
\end{proof*}
\newpage
\section{Aula 04 - 11/01/2024}
\subsection{Motivações}
\begin{itemize}
  \item Consequências de Caratheodory;
\end{itemize}
\subsection{Funções Mensuráveis}
\begin{def*}
  Uma função \(f:X\rightarrow \mathbb{R}\) é \textbf{mensurável} ou \(\mathcal{A}\)\textbf{-mensurável} se \(\{x: f(x) > a\}\in \mathcal{A}\) para todo \(a\in \mathbb{R}.\) Uma função 
a valores complexos é mensurável se a parte real e a imaginária assim forem. \(\square\)
\end{def*}
\begin{example}
 \begin{itemize}
   \item[1)] Se \(f:X\rightarrow \mathbb{R}\) é dada por \(f(x) = c,\) então \(\{x: f(x) > a\}\) é igual a X ou \(\emptyset .\) Logo, f é mensurável
     \item[2)] Defina 
       \[
         f(x)  = \left\{\begin{array}{ll}
             1,\quad x\in A\\ 
             0,\quad x\not\in A
           \end{array}\right.\equiv \chi_{A}.
       \]
      Então, \(\{x: f(x) > a\}\) é igual a X, A ou \(\emptyset .\) Com isso, f é mensurável se, e somente se, \(A\in \mathcal{A}.\)
      \item[3)] Suponha \(X = \mathbb{R}\) com \(\sigma \)-álgebra de Borel e \(f(x) = x\). Então, \(\{x: f(x) > a\} = (a, \infty),\) do que 
        segue que f é mensurável.
 \end{itemize}
\end{example}
\begin{prop*}
  Seja \(f:X\rightarrow \mathbb{R}.\) As seguintes condições são equivalentes: 
 \begin{itemize}
   \item[i)] \(\{x: f(x) > a\}\in \mathcal{A}\) para todo \(a\in \mathbb{R}.\)
   \item[ii)] \(\{x: f(x) \leq  a\}\in \mathcal{A}\) para todo \(a\in \mathbb{R}.\)
   \item[iii)] \(\{x: f(x) < a\}\in \mathcal{A}\) para todo \(a\in \mathbb{R}.\)
   \item[iv)] \(\{x: f(x) \geq  a\}\in \mathcal{A}\) para todo \(a\in \mathbb{R}.\)
 \end{itemize}
\end{prop*}
\begin{proof*}
  \((1) \Longleftrightarrow (2)\) Segue de \(\{x: f(x) \leq a\} = \{x: f(x) > a\}^{\complement}\) junto com as propriedades de \(\mathcal{A}\) como \(\sigma \)-álgebra.

  \((3) \Longleftrightarrow (4)\) Decorre de \(\{x: f(x) \geq a\} = \{x: f(x) < a\}^{\complement}.\)

  \((1) \Rightarrow (4)\) Ocorre pois \(\{x: f(x)\geq a\} = \bigcap_{i=1}^{\infty}\biggl\{x: f(x) > a - \frac{1}{n}\biggr\}\)

  \((4) \Rightarrow (1)\)Finalmente, é análogo ao item anterior, pois \(\{x: f(x) > a\} = \bigcup_{i=1}^{\infty}\biggl\{x: f(x)\geq a + \frac{1}{n}\biggr\}\). \qedsymbol
\end{proof*}
\begin{prop*}
  Seja X um espaço métrico e suponha que \( \mathcal{A}\) contém todos abertos e \(f:X\rightarrow \mathbb{R}\) é contínua. Então, f é mensurável.
\end{prop*}
\begin{proof*}
  Basta notar que \(\{x: f(x) > a\} = f^{-1}((a, \infty))\), o qual é aberto por continuidade. Portanto, \(\{x: f(x) > a\}\in \mathcal{A}\). \qedsymbol
\end{proof*}
\begin{prop*}
  Seja \(c\in \mathbb{R}.\) Se \(f, g:X\rightarrow \mathbb{R}\) são mensuráveis, então f + g, f, cf, fg, \(\max_{}(f, g)\) e \(\min_{}(f, g)\) são mensuráveis.
\end{prop*}
\begin{proof*}
  Suponha que \(f(x) + g(x) < a,\) ou seja, \(f(x) < a - g(x)\), e que existe \(r\in \mathbb{Q}\) tal que \(f(x) < r < a-g(x).\) Logo,
    \[
      \{x: f(x) + g(x) < a\} = \bigcup_{r\in \mathbb{Q}}^{}(\{x: f(x) < r\}\cap \{x: g(x) < a -r\}),
    \]
  donde Conclui-se que f + g é mensurável. 

  Para -f, basta notar que \(\{x: -f(x) > a\} = \{x: f(x) < -a,\}\). Agora, dado \(c > 0\), então \(\{x: cf(x) > a\} = \{x: f(x) > \frac{a}{c}\},\) tal que cf é mensurável. 
Caso \(c=0\), cf será uma função constante, que já vimos ser mensurável. Se \(c < 0\), segue que \(cf = -(|c|f),\) que é mensurável pelas propriedades anteriores. 

Agora, observe que \(\{x: f^{2}(x) > a\} = X\) se \(a < 0\) e, para \(a \geq 0\),
  \[
    \{x: f^{2}(x) > a\} = \{x: f(x) > \sqrt[]{a}\}\cup \{x: f(x) < -\sqrt[]{a}\}.
  \]
  Em ambos os casos, f é mensurável, do que decorre, também, a mensurabilidade de fg via 
    \[
      fg = \frac{1}{2}[(f+g)^{2} - f^{2}-g^{2}].
    \]
  A igualdade 
    \[
      \{x:\max_{}(f(x), g(x)) > a\} = \{x: f(x) > a\}\cup \{x: g(x) > a\}
    \]
  e, para o mínimo, basta notar que \(\min_{}(f, g) = -\max_{}(-f, -g).\) Portanto, concluímos as propriedades. \qedsymbol
\end{proof*}
\begin{prop*}
  Se \(f_{i}:X\rightarrow \mathbb{R} \) é mensurável para cada i, então \(F(x) = \sup_{i}f_{i}\), \(f(x) = \inf_{i}f_{i}, F^{*}(x) = \limsup_{i\to \infty}f_{i}\)
  e \(f^{*}(x) = \liminf_{i\to \infty}f_{i} = \sup_{n\geq 1}\{\inf_{m\geq n}f_{m}(x)\}\) são todas mensuráveis, desde que sejam finita. [Se considerar \(f_{i}:X\rightarrow \overline{\mathbb{R}}=[-\infty, \infty]\), pode 
  ser infinita.].
\end{prop*}
\begin{proof*}
  Comece por notar que 
 \begin{align*}
   &\{x\in X: f(x) \geq a\} = \bigcap_{n}^{}\{x\in X: f_{n}(x) \geq a\}\\ 
   &\{x\in X: F(x) \geq a\} = \bigcup_{n}^{}\{x\in X: f_{n}(x) \geq a\}.
 \end{align*}
Como cada \(f_{n}\) é mensurável, utilizando a propriedade do fechamento das \(\sigma \)-álgebras para uniões e interseções contáveis garante-nos que f e F são mensuráveis. Portanto, pela definição de 
sup e inf, segue que \(f^{*}\) e \(F^{*}\) também são mensuráveis.
\end{proof*}
\begin{def*}
  Dizemos que f = g \textbf{quase sempre}, ou \textbf{quase toda parte}, e denotamos \(f= g \mathrm{q.s.}\) ou \(f = g \mathrm{q.t.p.}\), se \(\{x: f(x)\neq g(x)\}\) tem \textit{medida nula.} Analogamente, dizemos que 
 \(f_{i}\) converge para f q.s., ou q.t.p., denotado \(f_{n}\overbracket[0pt]{\longrightarrow}^{n\to \infty}f \mathrm{q.t.p.}/\mathrm{q.s.}\), se \(\{x: f_{n}(x) \text{ não converge para }f(x)\}\) tem \textit{medida nula}. \(\square\)
\end{def*}
\begin{def*}
  Se X é um espaço métrico, \(\mathcal{B}\) é uma \(\sigma \)-álgebra de Borel, e \(f:X\rightarrow \mathbb{R}\) é mensurável com relação a \(\mathcal{B},\) dizemos que f é \textbf{Borel mensurável}. Caso \(f:X\rightarrow \mathbb{R}\) seja 
  mensurável com relação à Lebesgue \(\sigma \)-álgebra, dizemos que f é \textbf{Lebesgue mensurável.} \(\square\)
\end{def*}
 Vimos que toda função contínua é Borel mensurável, assim como funções crescentes na reta também são.
\begin{prop*}
  Se \(f:X\rightarrow \mathbb{R}\) é monótona, então f é Borel mensurável.
\end{prop*}
\begin{proof*}
  Suponha que f seja crescente. Caso contrário, faça -f. Dado \(a\in \mathbb{R},\) seja \(x_{0}=\sup_{}\{y:f(y) \leq a\}.\) Se \(f(x_{0}) \leq a\), então 
    \[
      \{x: f(x) > a\} = (x_{0}, \infty).
    \]
  Se \(f(x_{0}) > a\), então 
    \[
      \{x:f(x) > a\}  = [x_{0}, \infty).
    \]
  Em qualquer caso, \(\{x: f(x) > a\}\) é um conjunto de Borel. Portanto, \(f\) é Borel-mensurável. \qedsymbol
\end{proof*}
\begin{prop*}
  Seja \((X, \mathcal{A})\) um espaço mensurável e seja \(f:X\rightarrow \mathbb{R}\) uma função \(\mathcal{A}\)-mensurável. Seja A um elemento de uma \(\sigma \)-álgebra de Borel em \(\mathbb{R}.\) Então, \(f^{-1}(A)\in \mathcal{A}.\)
\end{prop*}
\begin{proof*}
  Seja \(\mathcal{B}\) a \(\sigma \)-álgebra de Borel sobre \(\mathbb{R}\) e \(\mathcal{C} = \{A\subseteq \mathbb{R}: f^{-1}(A)\in \mathcal{A}\}.\) Se \(A_{1}, A_2,\dotsc \in \mathcal{C},\) então, como 
    \[
      f^{-1}\biggl(\bigcup_{i}^{}A_{i}\biggr) = \bigcup_{i}^{}f^{-1}(A_{i})\in \mathcal{A},
    \]
  segue que \(\mathcal{C}\) é fechado para a união. Analogamente, conclui-se que \(\mathcal{C}\) e fechado com relação à interseção e complementos, o que faz com que \(\mathcal{C}\) seja uma \(\sigma \)-álgebra. 
Como f é mensurável, \(\mathcal{C}\) contém \((a, \infty),\) que é a pré-imagem sob f de algum conjunto, para todo \(a\in \mathbb{R}.\) Desta forma, \(\mathcal{C}\) contém a \(\sigma \)-álgebra gerada por esses intervalos, 
ou seja, \(\mathcal{C}\) contém \(\mathcal{B}\). Portanto, toda pré-imagem de um conjunto de Borel é mensurável. \qedsymbol
\end{proof*}
  Existem conjuntos que são Lebesgue mensuráveis, mas não são Borel mensuráveis. Vejamos um deles a seguir, baseado no conjunto de Cantor.
\begin{example}
  Seja f a função de Cantor-Lebesgue e defina 
    \[
      F(x) = \inf_{}\{y: f(y)\geq x\}.
    \]
  Já vimos que F é estritamente crescente, apesar de não ser contínua. Lembre-se que f é constante nos intervalos omitidos na definição e deixa de ser contínua nos pontos de \(C\). Em particular, f é constante no intervalo 
  \(\biggl(\frac{1}{3}, \frac{2}{3}\biggr)\), intervalo no qual f fica menor que a identidade. Nesse intervalo, F é constante e dará um salto após isso, o que impede F de ser contínua. Além disso, ela é injetora, tal que, da definição de f, \(F([0, 1])\subseteq C,\) sendo C 
  o conjunto de Cantor. Como F é crescente, \(F^{-1}\) leva o conjunto Borel mensurável em um conjunto Borel mensurável. 

  Agora, seja m a medida de Lebesgue e A conjunto não mensurável construído previamente. Coloque \(B = F(A)\). Como \(F(A)\subseteq C\) e \(m(C) = 0\), 
vale que \(m(F(A)) = 0\), fazendo com que F(A) seja Lebesgue mensurável. Por outro lado, F(A) não pode ser Borel mensurável, pois, se fosse, \(A = F^{-1}(F(A))\) seria Borel mensurável, donde sairia uma contradição.
\end{example}
\begin{def*}
  Seja \((X, \mathcal{A})\) um espaço mensurável. Se \(E\in \mathcal{A},\) definimos a \textbf{função característica de E} como 
    \[
      \chi_{E}(x)  = \left\{\begin{array}{ll}
          1,\quad x\in E \\ 
          0,\quad x\not\in E
        \end{array}\right.\quad \square
    \]
\end{def*}
\begin{def*}
  Uma \textbf{função simples} s é uma função da forma 
    \[
      s(x) = \sum\limits_{i=1}^{n}a_{i}\chi_{E_{i}}(x),
    \]
  em que \(a_{i}\in \mathbb{R}\) e \(E_{i}\) são conjuntos mensuráveis. \(\square\)
\end{def*}
\begin{prop*}
  Suponha que f é não negativa e mensurável. Então, existe uma sequência de funções simples não negativas \(s_{n}\) crescendo para f \((s_1\leq s_2\leq \dotsc \leq f\).
\end{prop*}
\begin{proof*}
  Seja 
    \[
      A_{in} = \biggl\{x: \frac{(i-1)}{2^{n}} \leq f(x) \leq \frac{i}{2^{n}}\biggr\}
    \]
  e seja 
    \[
      B_{n} = \{x: f(x) \geq n\}, n =1, 2,\dotsc , i = 1, 2, \dotsc n2^{n}.
    \]
  Defina 
    \[
      s_{n} = \sum\limits_{i=1}^{n2^{n}}\frac{i-1}{2^{n}}\chi_{A_{in}} + n\chi_{B_{n}}.
    \]
  Com isso, \(s_{n}(x) = n \) se \(f(x)\geq n\) e, se \(f(x)\in \biggl(\frac{i-1}{2^{n}}, \frac{i}{2^{n}}\biggr)\) para \(\frac{i}{2^{n}} \leq n\), temos \(s_{n}(x) = \frac{(i-1)}{2^{n}}.\) Segue que 
 \begin{itemize}
   \item \(s_{n} \leq s_{n+1}\)
   \item \(S_{n} \leq f\) por definição.
 \end{itemize}
 Portanto, tomando o limite, segue que \(s_{n}(x)\overbracket[0pt]{\longrightarrow}^{n\to \infty}f_{n}(x)\).
\end{proof*}
Veremos agora o Teorema de Lusin, que, essencialmente, afirma que toda função mensurável é 
contínua a menos de um conjunto tão pequeno quanto se queira.
\begin{theorem*}[Lusin]
  Suponha que \(f:[0, 1]\rightarrow \mathbb{R}\) é Lebesgue mensurável, m é a medida de Lebesgue e \(\varepsilon > 0\) é dado. Então, existe um fechado \(F\subseteq [0, 1]\) tal que 
  \(m([0, 1]\setminus{F}) < \varepsilon \) e a restrição de f a F é uma função contínua sobre F.
\end{theorem*}
\begin{proof*}
  Suponha, primeiro, que \(f = \chi_{A}, A\subseteq [0, 1]\) Lebesgue mensurável. Então, existem E fechado e G aberto tais que \(E\subseteq A\subseteq G\) e \(m(G\setminus{A}) < \frac{\varepsilon }{2}, m(A\setminus{E}) < \frac{\varepsilon }{2}\). 
Seja \(\delta = \inf_{}\{|x-y|: x\in E, y\in G ^{\complement}\}.\) Como \(E\subseteq A\subseteq [0, 1],\) temos E como um compacto e \(\delta  > 0.\) Coloque 
  \[
    g(x) = \biggl(1 - \frac{d(x, E)}{\delta }\biggr),
  \]
sendo \(y = \max_{}(y, 0)\) e \(d(x, E) = \inf_{}\{|x-y|:y\in E\}.\) Então, g é contínua, assume valores em [0, 1] e é igual a 1 sobre E, mas 0 sobre \(G ^{\complement}.\) Tome 
 \(F = (E\cup G ^{\complement})\cap [0, 1].\) Então, 
   \[
     m([0, 1]\setminus{F}) \leq m(G\setminus{E}) < \varepsilon,
   \]
   e f = g sobre F, pois \(([0, 1]\setminus{F})\subseteq (G\setminus{E}).\)

   Agora, suponha que f é uma função simples \(f = \sum\limits_{i=1}^{\infty}a_{i}\chi_{A_{i}},\) sendo \(A_{i}\subseteq [0, 1]\) Lebesgue mensuráveis, \(a_{i} \geq 0.\) Da primeira parte, escolha 
   F fechado tal que \(m([0, 1]\setminus{F_{i}}) < \frac{\varepsilon }{M}\) e \(\chi_{A}\) restria à \(F_{i}\) é contínua para \(i=1,2,\dotsc , M.\) Faça \(F = \bigcap_{i=1}^{M}F_{i},\) de maneira que \(F\) é fechado, \(m([0,1]\setminus{F}) < \varepsilon \)
   e f restrita a F será contínua. 

   Suponha, a seguir, que \(f\geq 0\), limitada por K e \(\mathrm{supp}(f)\subseteq [0, 1].\) Seja 
     \[
       A_{in} = \biggl\{x: \frac{(i-1)}{2^{n}} \leq f(x) \leq \frac{i}{2^{n}}\biggr\}
     \]
     e defina 
       \[
         f_{n}(x) = \sum\limits_{i=1}^{K2^{n}+1}\frac{i-1}{2^{n}}\chi_{A_{in}(x)},
       \]
  de maneira que cada \(f_{n}\) é simples e \(f_{n}\uparrow f.\)

  Note que 
    \[
      h_{n}(x) = f_{n-1}(x) - f_{n}(x)
    \]
  é simples e limitado por \(2^{-n}.\) Escolha \(F_{0}\) fechado, pois \(m([0, 1]\setminus{F_{0}}) < \frac{\varepsilon }{2}\) e \(f_{0}\) restrito a \(F_{0}\) será contínua pelo passo 2. 
Para \(n\geq 1\), escolha \(F_{n}\) fechado tal que \(m([0, 1]\setminus{F_{n}}) < \frac{\varepsilon }{2^{n-1}}\) e \(h_{n}\) restrita a \(F_{n}\) é contínua. Coloque, então, \(F = \bigcap_{i=1}^{\infty}F_{n},\) o qual 
será fechado pois a interseção arbitrária de fechados permanece fechado. Com isso, 
  \[
    m([0,1]\setminus{F}) \leq \sum\limits_{n=1}^{\infty}m([0, 1]\setminus{F_{n}}) < \frac{\varepsilon }{2} < \varepsilon .
  \]
  Neste conjunto F, como \(h_{n} = f_{n+1}-f_{n},\) temos a convergência uniforme para f da função 
    \[
      f_{0} + \sum\limits_{i=0}^{\infty}h_{n}(x),
    \]
  já que cada \(h_{n}\) é limitada por \(2^{-n}.\) Como convergência uniforme preserva continuidade, segue que f é contínua sobre F. 

  Em seguida, assuma que \(f\geq 0\) e seja \(B_{K} = \{x: f(x) \leq K\}.\) Como f é limitado, então \(B_{K}\uparrow [0, 1]\) quando \(K\to \infty,\) tal que 
    \[
      m(B_{K}) > 1 - \frac{\varepsilon }{3}
    \]
    para K suficientemente grande. Escolha \(D\subseteq B_{K}\) tal que D é fechado e \(m(B_{K}\setminus{D}) < \frac{\varepsilon }{3}\) e \(E\subseteq [0, 1]\) fechado, de maneira que 
    \(f \cdot \chi_{D}\) restrita a E é contínua, com medida \(m([0, 1]\setminus{E}) < \frac{\varepsilon }{3}.\) Assim, \(F = D\cap E\) é fechado e \(m([0, 1]\setminus{F}) < \varepsilon ,\) 
    além de f restrita a F ser contínua. [Aqui, foi usado que \([0,1]\setminus{F}\subseteq B_{K}^{\complement}\cup (B_{K}\setminus{D})\cup ([0, 1]\setminus{E})]\)

    Finalmente, para o caso geral, suponha f mensurável, escreva \(f = f^{+} - f^{-}, f^{\pm} \geq 0.\) Existem \(F^{+}\) fechados com \(m([0, 1]\setminus{F^{\pm}}) < \frac{\varepsilon }{2}\) e com a continuidade 
    de \(f^{+}\) restrita a \(F^{+}.\) Tome \(F = F^{+}\cap F^{-},\) nosso fechado procurado. Suponha, primeiramente, que \(f=\chi_{B},\) em que \(B = [0, 1]\cap \mathbb{Q}^{\complement}.\) Esta f é Borel mensurável, pois 
    \([0, 1]\setminus{B}\) é enumerável, sendo este conjunto a imagem inversa de \((0, a)\) para \(a < 1\). Assim, a união enumerável de pontos [fechados], f assume valores 0 e 1 na vizinhança de \(a\in [0, 1].\) Portanto, 
    todo ponto \(a\in [0, 1]\) é ponto de descontinuidade. Agora, se \(q_1, q_2, \dotsc \) é a enumeração dos racionais e \(I_{j}\) é vizinhança aberta de \(q_{j},\) então \(f=1\) no fechado \(A = [0, 1]\setminus{\bigcup_{i}^{}I_{i}.}\)
    Logo, f é contínua neste conjunto. Portanto, provamos o Teorema de Lusin.
\end{proof*}
\subsection{Integral de Lebesgue}
\begin{def*}
  Seja \((X, \mathcal{A}, \mu )\) um espaço de medida. 
 \begin{itemize}
   \item[1)] Se 
     \[
       s=\sum\limits_{i=1}^{n}a_{i}\chi_{E_{i}}
     \]
     é uma função simples, não negativa e mensurável, define-se a integral de Lebesgue de s como sendo 
       \[
         \int_{}^{}sd\mu = \sum\limits_{i=1}^{n}a_{i}\mu (E_{i}).
       \]
      Aqui, se \(a_{i} = 0\) e \(\mu (E_{i}) = \infty\), usamos a convenção \(a_{i}\cdot \mu (E_{i}) =0.\)
  \item[2)] Se \(f\geq 0\) é uma função mensurável, define-se a integral de Lebesgue de f como sendo:
    \[
      \int_{}f d\mu_{} = \sup_{}\biggl\{\int_{}^{}s d\mu : 0 \leq s\leq f, s \text{ simples}\biggr\}
    \]
  \item[3)] Se f é mensurável, seja \(f^{\pm} = \max_{}(\pm f, 0)\). Suponha que \(\int_{}^{}f^{\pm}d\mu \) não seja infinito simultaneamente. Define-se, então, 
    \[
      \int_{}f d\mu_{} = \int_{}f^{+} d\mu_{} - \int_{}f^{-} d\mu_{}.
    \]
  \item[4)] Se \(f = u + iv\) com valores complexos é mensurável, com \(\int_{}|u|+|v| d\mu_{} < \infty\), define-se
    \[
      \int_{}f d\mu_{} = \int_{}u d\mu_{} + i\int_{}v d\mu_{}. \quad \square
    \]
 \end{itemize}
\end{def*}
  Vale mencionar que as representações de uma função simples não são únicas.
 \begin{itemize}
   \item \(s = \chi_{A\cup B} = \chi_{A} + \chi_{B}\) se \(A\cap B = \emptyset \)
   \item \(s = \sum\limits_{i=1}^{m}a_{i}\chi_{A_{i}} = \sum\limits_{j=1}^{n}b_{j}\chi_{B_{j}} \Rightarrow \sum\limits_{i=1}^{m}a_{i}\mu (A_{i}) = \sum\limits_{j=1}^{n}b_{j}\mu (B_{j})\).
 \end{itemize}
 \begin{prop*}
  \begin{itemize}
    \item[1)] Se \(c\geq 0, \int_{}^{}c\varphi d\mu  = c\int_{}\varphi  d\mu_{}\);
      \item[2)] \(\int_{}(\varphi + \psi) d\mu_{} = \int_{}^{}\varphi d\mu + \int_{}\psi d\mu_{}\);
        \item[3)] Se \(\varphi \leq \psi\), então \(\int_{}\varphi  d\mu_{} \leq \int_{}\psi d\mu_{}\)
          \item[4)] A aplicação \(A\mapsto \int_{A}\varphi  d\mu_{}\) é uma medida sobre X. 
  \end{itemize}
 \end{prop*}
 \begin{proof*}
   1 - Trivial. 

   2 - Sejam \(\varphi  = \sum\limits_{j=1}^{n}a_{i}\chi_{E_{i}}\) e \(\psi = \sum\limits_{k=1}^{m}b_{k}\chi_{F_{k}}\)  representações das funções simples. Como 
     \[
       E_{i} = \bigcup_{k=1}^{m}(E_{j}\cap F_{k}), \quad F_{k} = \bigcup_{j=1}^{n}(E_{j}\cap F_{k}),
     \]
    em que a união é disjunta. Da aditividade finita, temos 
   \begin{align*}
     \int_{}(\varphi +\psi) d\mu_{} &= \sum\limits_{j, k}^{}(a_{j} + b_{k})\mu (E_{j}\cap F_{k})\\ 
                                    &= \sum\limits_{j, k}^{}(a_{j} + b_{k})\mu \biggl(\biggl(\bigcup_{k=1}^{m}(E_{j}\cap F_{k})\biggr)\cap \biggl(\bigcup_{j=1}^{n}(E_{j}\cap F_{k})\biggr)\biggr) \\ 
                                    &= \sum\limits_{j=1}^{n}\sum\limits_{k=1}^{m}a_{j}\mu (E_{j}\cap F_{k}) + \sum\limits_{j=1}^{n}\sum\limits_{k=1}^{m}b_{k}\mu (E_{j}\cap F_{k})\\ 
                                    &= \sum\limits_{j}^{}a_{j}\mu (e_{j}) + \sum\limits_{k}^{}b_{k}\mu (F_{k})\\ 
                                    &= \int_{}\varphi  d\mu_{} + \int_{}\psi d\mu_{},
   \end{align*}
   em que foi usado que \(\mu (E_{j}) = \sum\limits_{k=1}^{m}\mu (E_{j}\cap F_{k})\) e \(\mu (F_{k}) = \sum\limits_{j=1}^{n}\mu (E_{j}\cap F_{k})\).

   3 - Se \(\varphi \leq \psi,\) então \(a_{j}\leq b_{k}\) sempre que \(E_{j}\cap F_{i}\neq\emptyset\) e, da aditividade, vem 
     \[
       \int_{}\varphi  d\mu_{} = \sum\limits_{j}^{}a_{j}\mu (e_{j}) = \sum\limits_{j, k}^{}a_{j}\mu (E_{j}\cap F_{k}) \leq \sum\limits_{j, k}^{}b_{k}\mu (E_{j}\cap F_{k}) = \sum\limits_{k}^{}b_{k}\mu (F_{k}) = \int_{}\psi d\mu_{}
     \]

  4 - Defina \(\nu(A)\equiv \int_{A}^{}\varphi d\mu .\) Note que \(\nu(\emptyset )=0.\) Se \(\{A_{i}\}_{i=1}^{\infty}\) é uma sequência de conjuntos dois-a-dois disjunto e, se \(A = \bigcup_{k=1}^{\infty}A_{k},\) temos 
    \[
      \int_{}\varphi  d\mu_{} = \sum\limits_{j}^{}a_{j}\mu (A\cap E_{j}) = \sum\limits_{j, k}^{}a_{j}\mu (A_{k}\cap E_{j}) = \sum\limits_{k}^{}\int_{A_{k}}\varphi  d\mu_{},
    \]
  mostrando que 
    \[
      \mu \biggl(\bigcup_{j}^{}A_{j}\biggr) = \sum\limits_{j}^{}\nu(A_{j}).
    \]
  Portanto, \(\nu\) é uma medida. \qedsymbol
 \end{proof*}
\newpage

\section{Aula 05 - 15/01/2024}
\subsection{Motivações}
\begin{itemize}
  \item Teoremas de Limites; 
  \item Relação entre Lebesgue e Riemann.
\end{itemize}
\subsection{Teoremas de Limites} 
  Vimos que 
    \[
      \int_{}f d\mu_{} = \sup_{}\biggl\{\int_{}s d\mu_{}: 0 \leq s\leq f, \text{ s simples}\biggr\}.
    \]
  Isto elucida a seguinte definição: 
 \begin{def*}
   Se f é mensurável e \(\int_{}|f| d\mu_{} < \infty,\) diremos que \textbf{f é integrável.} \(\square\)
 \end{def*}
\begin{prop*}
  Suponha que f e g são integráveis. 
 \begin{itemize}
   \item[a)] Se \(c \geq 0\) e \(f\geq 0\), \(\int_{}c\varphi  d\mu_{} = c \int_{}\varphi d\mu_{}\)
     \item[b)] Se \(f\leq g\) com \(f, g \geq 0\), então \(\int_{}f d\mu_{} \leq \int_{}g d\mu_{}\)
       \item[c)] Se \(0\leq a\leq f(x)\leq b\) para todo x e \(\mu (X) < \infty,\) então \(a\mu (X) \leq \int_{}f d\mu_{} \leq b\mu (X)\)
         \item[d)] Se \(f\geq 0\) e \(\mu (A) = 0\), então \(\int_{A}f d\mu_{}\equiv \int_{}f \chi_{A} d\mu_{} = 0.\)
 \end{itemize}
\end{prop*}
\begin{proof*}
  A linearidade e outras propriedades serão posteriormente vistas com outros teoremas. 

  1. Vimos que, se \(c \geq 0\) e \(\varphi \) é simples, então \(\int_{}c \varphi  d\mu_{} = c \int_{}\varphi  d\mu_{}\). O mesmo vale se \(\varphi \leq f,\) então vale a igualdade. 

  Para o item 2, vimos que, se \(\varphi, \psi\) são simples   \(\varphi \leq \psi,\) então \(\int_{}\varphi  d\mu_{} \leq \int_{}\psi d\mu_{}\). O mesmo vale se tomarmos \(\varphi \leq f\) e \(\psi \leq g\),
  tal que 
    \[
      \sup_{}\biggl\{\int_{}\varphi  d\mu_{}: \varphi \leq f\biggr\} \leq \sup_{}\biggl\{\int_{}\psi d\mu_{}: \psi \leq g\biggr\},
    \]
  donde segue que \(\int_{}f d\mu_{} \leq \int_{}g d\mu_{}.\) 

  O item 3 segue do item 2. 

  Por fim, o item 4 é feito tomando \(\varphi \) simples tal que \(\varphi  \leq f \chi_{A}.\) Note que, de fato, \(\varphi \) é simples em A. Tome \(A = \bigcup_{i=1}^{n}A_{i}.\)
  Então,
    \[
      \varphi = \sum\limits_{i}^{}a_{i}A_{i} \quad\&\quad \int_{}\varphi  d\mu_{} = \sum\limits_{}^{}a_{i}\mu (A_{i}) \leq \sum\limits_{}^{}a_{i}\mu (A) = 0.
    \]
  Portanto, tomando supremos, temos \(\int_{}f \chi_{A} d\mu_{} = 0.\) \qedsymbol
\end{proof*}
  Veremos alguns teoremas de limites que são fundamentais na teoria de medida. Em ordem que serão vistos, temos 
 \begin{itemize}
   \item Teorema da Convergência Monótona;
   \item Lema de Fatou;
   \item Teorema da Convergência Dominada;
   \item Teorema de Fubini.
 \end{itemize}
 \hypertarget{monotone_convergence}{\begin{theorem*}[Teorema da Convergência Monótona]
   Sejam \(f_{n}\geq 0\) mensuráveis, \(f_{1}(x) \leq f_2(x) \leq \dotsc \) para todo x em X e \(\lim_{n\to \infty}f_{n}(x) = f(x)\), também para todo x em X. Então,
     \[
       \int_{X}f_{n} d\mu_{}\longrightarrow \int_{X}f d\mu_{}
     \]
  \end{theorem*}}
  \begin{proof*}
   Note que 
     \[
       f_{n}\leq f_{n+1} \Rightarrow \int_{}^{}f_{n}\leq \int_{}^{}f_{n+1},
     \]
  ou seja, a sequência das integrais de \(f_{n}\) com respeito a \(\mu \) é uma crescente. Com isso, sabe-se que o limite f é mensurável. Por outro lado, como 
    \[
      f_1(x)\leq f_2(x)\leq \dotsc \leq f,
    \]
    segue que f é limitada. Daí, se 
      \[
        L\equiv \lim_{n\to \infty}\int_{}^{}f_{n}d\mu ,\quad f_{n}\leq f \Rightarrow L \leq \int_{}f d\mu_{}.
      \]
    Seja 
      \[
        s = \sum\limits_{i=1}^{m}a_{i}\chi_{E_{i}},\quad a_{i}\geq 0, \text{ s simples, com } s\leq f.
      \]
    Com \(c\in (0, 1),\) defina 
      \[
        A_{n} = \{x: f_{n}(x) \geq cs(x)\},
      \]
    de modo que, como \(f_{n}\leq f_{n+1}\) e \(c < 1,\) chegamos em 
      \[
        A_{n}\uparrow X,\quad A_1\subseteq A_2\subseteq \dotsc \Rightarrow \mu (X) = \lim_{n\to \infty}\mu (A_{n}).
      \]
    Para cada n, 
      \[
        \int_{}f_{n} d\mu_{} \geq \int_{A_{n}}f_{n} d\mu_{} \geq c \int_{A_{n}}s d\mu_{} = c \int_{}\sum\limits_{i=1}^{m}a_{i}\chi_{E_{i}} d\mu_{} = c \sum\limits_{i=1}^{m}a_{i}\mu (E_{i}\cap A_{i}).
      \]
    Fazendo \(n\to \infty\), 
      \[
        L \geq c \sum\limits_{i=1}^{m}\mu (A_{i}) = c \int_{}^{}s d\mu .
      \]
    Por \(c\in (0, 1)\) ser arbitrário, podemos concluir que 
      \[
        L \geq \int_{}^{}s d\mu, 
      \]
    Portanto, tomando o supremo para \(s \leq f\), 
      \[
        L \geq \int_{}f d\mu_{}.\quad \text{\qedsymbol}
      \]
 \end{proof*}
 \begin{example}
   Coloque \(X = [0, \infty), f_{n}(x) = \frac{-1}{n}\) para todo x em X. Então, 
     \[
       \int_{}^{}f_{n}d\mu = -\infty,
     \]
    mas \(f_{n}\uparrow f,\) em que \(f=0\) e \(\int_{}f d\mu_{} = 0.\) Conclui-se que não é aplicável o Teorema da Convergência Monótona, pois \(f_{n}\) não é não-negativa.
 \end{example}
 \begin{example}
   Agora, coloque \(X = [0, \infty), f_{n} = n \chi_{(0, \frac{1}{n})}\) para x em X. Com isso, 
     \[
       f_{n}\geq 0,\quad \int_{}^{}f_{n}d\mu = 1,
     \]
    mas \(f_{n}\) não converge para \(\int_{}^{}f d\mu = 0\), em que \(f=0\). Neste caso, o Teorema não aplica pois a sequência não é crescente.
 \end{example}
\begin{theorem*}
  Se \(f, g \geq 0\) são mensuráveis ou integráveis, então 
    \[
      \int_{}(f+g) d\mu_{} = \int_{}f d\mu_{} + \int_{}g d\mu_{}.
    \]
\end{theorem*}
\begin{proof*}
  Vimos que se f e g são simples, o resultado já é verificado. Agora, suponha que \(f, g \geq 0\) e tome sequências \(\{s_{n}\}\) e \(\{t_{n}\}\) simples tais que 
  \(s_{n}\uparrow f\) e \(t_{n}\uparrow g\), de modo que \(s_{n}+t_{n}\) são simples e \(s_{n}+t_{n}\uparrow f+g\). Portanto, pelo Teorema da Convergência Monótona,
    \[
      \int_{}(f+g) d\mu_{} = \lim_{n\to \infty}\int_{}(s_{n} + t_{n}) d\mu_{} = \lim_{n\to \infty}\int_{}s_{n} d\mu_{} + \lim_{n\to \infty}\int_{}t_{n} d\mu_{} = \int_{}f d\mu_{} + \int_{}g d\mu_{}    
    \]
  Suponha, agora, que \(f = f^{+} - f^{-}\) e \(g = g^{+}-g^{-}.\) Observe que f + g é integrável, pois 
    \[
      \int_{}|f+g| d\mu_{} \leq \int_{}|f| + |g| d\mu_{} = \int_{}|f| d\mu_{} + \int_{}|g| d\mu_{} < \infty.
    \]  
  Note que 
    \[
      (f+g)^{+} - (f+g)^{-} = f^{+} - f^{-} + g^{+}-g^{-} \Rightarrow (f+g)^{+} + f^{-} + g^{-} = f^{+} + g^{+} + (f+g)^{-}.
    \]
  Pelos resultados para funções não-negativas, segue que 
    \[
      \int_{}(f+g)^{+} d\mu_{} + \int_{}f^{-} d\mu_{} + \int_{}g^{-} d\mu_{} = \int_{}f^{+} d\mu_{} + \int_{}g^{+} d\mu_{} + \int_{}(f+g)^{-} d\mu_{}.
    \]
  Rearranjando, chegamos em 
    \[
      \int_{}(f+g) d\mu_{} = \int_{}f d\mu_{} + \int_{}g d\mu_{}.
    \]
  Caso f e g tenham valores complexos, aplica-se para a parte real e a parte imaginária. \qedsymbol
\end{proof*}
\begin{prop*}
  Suponha f e g integráveis. 
 \begin{itemize}
   \item[i)] Se \(c\geq 0\), então \(\int_{}cf d\mu_{} = c \int_{}f d\mu_{}\)
     \item[ii)] Se \(f \leq g,\) então \(\int_{}f d\mu_{} \leq \int_{}g d\mu_{}\)
       \item[iii)] Se \(a\leq f(x) \leq b\) para todo x e \(\mu (X) < \infty\), então \(a\mu (X) \leq \int_{}f d\mu_{} \leq b\mu (X)\)
         \item[iv)] Se \(\mu (A) = 0\), então \(\int_{A}f d\mu_{}\equiv \int_{}f \chi_{A} d\mu_{} = 0\)
 \end{itemize}
\end{prop*}
\begin{proof*}
  Podemos considerar as funções reais. 

  Para a parte 1, é imediato, basta tomar \(\{s_{n}\}\) simples tal que \(s_{n}\uparrow f\), do que segue que \(cs_{n}\) é simples e \(cs_{n}\uparrow cf.\)
Pelo TCM, temos 
  \[
    \int_{}cf d\mu_{} = \lim_{n\to \infty}\int_{}cs_{n} d\mu_{} = c \lim_{n\to \infty}\int_{}s_{n} d\mu_{} = c \int_{}f d\mu_{}
  \]
  Para o caso geral, basta separar f em parte positiva e negativa como antes. ]

  2. Note que 
    \[
      f = f^{+} - f^{-} \leq g^{+} - g^{-} = g \Rightarrow  0 \leq f^{+} + g^{-} \leq g^{+} + f^{-}
    \]
  Integrando, para o caso não negativo e usando linearidade, temos 
    \[
      \int_{}(f^{+} + g^{-}) d\mu_{} \leq \int_{}(g^{+}+f^{-}) d\mu_{} \Rightarrow \int_{}f^{+} d\mu_{} + \int_{}g^{-} d\mu_{} \leq \int_{}g^{+} d\mu_{} + \int_{} f^{-} d\mu_{},
    \]
  donde conclui-se que 
    \[
      \int_{}f d\mu_{} \leq \int_{}g d\mu_{}.
    \]
  O item 3 segue do item 2 e, por fim, para o item 4, como \(\int_{}f^{\pm}\chi_{A} d\mu_{} = 0\), segue da linearidade que 
    \[
      \int_{}f \chi_{A} d\mu_{} = \int_{}f^{+}\chi_{A} d\mu_{} - \int_{}f^{-}\chi_{A} d\mu_{} = 0. \quad \text{\qedsymbol}
    \]
\end{proof*}
\begin{prop*}
  Suponha \(f_{n}\geq 0\) mensuráveis. Então, 
    \[
      \int_{}\sum\limits_{i=1}^{\infty}f_{i} d\mu_{} = \sum\limits_{i=1}^{\infty}\int_{}f_{i} d\mu_{}
    \]
\end{prop*}
\begin{proof*}
  Seja \(F_{N} = \sum\limits_{i=1}^{N}f_{i}.\) Como \(0 \leq F_{n}(x)\uparrow \sum\limits_{i=1}^{\infty}f_{i}(x),\) temos 
 \begin{align*}
   \int_{}\sum\limits_{i=1}^{\infty}f_{i} d\mu_{} &= \int_{}\lim_{N\to \infty} d\mu_{} \sum\limits_{i=1}^{N}f_{i} \\ 
                                                  &= \int_{}^{}\lim_{N\to \infty}F_{N}d\mu  = \lim_{N\to \infty}\int_{}F_{N} d\mu_{}.\\ 
                                                  &= \lim_{N\to \infty}\sum\limits_{i=1}^{N}\int_{}f_{n} d\mu_{} = \sum\limits_{i=1}^{\infty}f_{n},
 \end{align*}
em que usamos tanto o Teorema da Convergência Monótona e a linearidade. \qedsymbol
\end{proof*}
\begin{prop*}
  Se f é integrável, então 
    \[
      \biggl\vert \int_{}f d\mu_{} \biggr\vert \leq \int_{}|f| d\mu_{}
    \].
\end{prop*}
\begin{proof*}
  No caso real, \(f \leq |f| \Rightarrow \int_{}f d\mu_{} \leq \int_{}|f| d\mu_{}\) e o análogo vale para -f. No caso complexo, \(\int_{}f d\mu_{}\in \mathbb{C}.\)

  Se \(\int_{}f d\mu_{} = 0\), não há nada a fazer. Caso contrário, \(\int_{}f d\mu_{} = re^{i\theta }\) para algum r e \(\theta \). Com isso, 
    \[
      \biggl\vert \int_{}f d\mu_{} \biggr\vert = r = e^{-i\theta }\int_{}f d\mu_{} = \int_{}e^{-i\theta }f d\mu_{}
    \]
  Agora, note que \(\mathrm{Re}(\int_{}f d\mu_{}) = \int_{}\mathrm{Re}(f) d\mu_{}.\) Como \(|\int_{}f d\mu_{}|\in \mathbb{R},\) temos o que queríamos: 
    \[
      \biggl\vert \int_{}f d\mu_{}  \biggr\vert = \mathrm{Re}\biggl(\int_{}e^{-\theta }f d\mu_{}\biggr) = \int_{}\mathrm{Re}(e^{-\theta }f) d\mu_{} \leq \int_{}|f| d\mu_{}.\quad \text{\qedsymbol}
    \]
\end{proof*}
\hypertarget{fatou}{ 
  \begin{lemma*}[Fatou]
 Sejam \(f_{n}\geq 0\) mensuráveis. Então, 
   \[
     \int_{}\lim_{n\to \infty}f_{n} d\mu_{} \leq \liminf_{n\to \infty}\int_{}f_{n} d\mu_{}.
   \]
\end{lemma*}}
\begin{proof*}
  Defina \(g_{n} = \inf_{i\geq n}f_{i},\) tal que \(g_{n}\geq 0\),  \(g_{n}\uparrow \liminf_{n\to \infty}f_{n}\). Obesrva-se de cada que, para \(i\geq n\), 
  \(g_{n}\leq f_{i}\), tal que \(\int_{}g_{n} d\mu_{} \leq \int_{}f_{i} d\mu_{}.\) Portanto, 
    \[
      \int_{}g_{n} d\mu_{} \leq \inf_{i\geq n} \int_{}f_{i} d\mu_{}
    \]
  Fazendo n tender a infinito na desigualdade acima, o lado direito fica \(\liminf_{n\to \infty}\int_{}f_{n} d\mu_{},\) enquanto o lado esquerdo, via Teorema da Convergência Monótona, fornece 
    \[
      \int_{}\liminf_{n\to \infty}f_{n} d\mu_{}.
    \]
  Portanto, 
    \[
      \int_{}\liminf_{n\to \infty}f_{n} d\mu_{} \leq \liminf_{n\to \infty}\int_{}f_{n} d\mu_{}.\quad \text{\qedsymbol}
    \]
\end{proof*}
\begin{example}
  Suponha \(f_{n}\longrightarrow f\) e \(\sup_{n}\int_{}|f_{n}| d\mu_{} \leq C < \infty.\) Então, \(|f_{n}|\to |f|,\) do que segue, pelo \hyperlink{fatou}{\textit{Lema de Fatou}}, que 
    \[
      \int_{}|f| d\mu_{} \leq C.
    \]
\end{example}
 \hypertarget{dominated_convergence}{
   \begin{theorem*}[Convergência Dominada de Lebesgue]
   Suponha as seguintes coisas: 
  \begin{itemize}
    \item \(f_{n}\) são funções reais mensuráveis 
    \item \(f_{n}(x)\to f(x)\) para cada x q.t.p.
    \item Eosite \(g\geq 0\) integrável tal que \(|f_{n}(x)| \leq g(x)\) para todo x.
  \end{itemize}
  Então, 
    \[
      \lim_{n\to \infty}\int_{}f_{n} d\mu_{} = \int_{}f d\mu_{}
    \]
 \end{theorem*}}
\begin{proof*}
  Como \(f_{n} +g \geq 0\), segue do \hyperlink{fatou}{\textit{Lema de Fatou}} que 
    \[
      \int_{}f d\mu_{} + \int_{}g d\mu_{} = \int_{}(f+g) d\mu_{} \leq \liminf_{n\to \infty}\int_{}(f_{n}+g) d\mu_{} = \liminf_{n\to s8}\int_{}f_{n} d\mu_{} + \int_{}g d\mu_{}.
    \]
  Por g ser integrável, 
    \[
      \int_{}f d\mu_{} \leq \liminf_{n\to \infty}\int_{}f_{n} d\mu_{}.
    \]
  Analogamente, \(g-f_{n}\geq 0\), tal que 
    \[
      \int_{}g d\mu_{} - \int_{}f d\mu_{} = \int_{}(g-f) d\mu_{} \leq \liminf_{n\to \infty}\int_{}(g-f_{n}) d\mu_{} = \int_{}g d\mu_{} + \liminf_{n\to \infty}\int_{}(-f_{n}) d\mu_{}.
    \]
  Logo, 
    \[
      -\int_{}f d\mu_{} \leq \liminf_{n\to \infty}\int_{}(-f_{n}) d\mu_{} = - \limsup_{n\to \infty}\int_{}f_{n} d\mu_{},
    \]
  donde 
    \[
      \int_{}f d\mu_{} \geq \limsup_{n\to \infty}\int_{}f_{n} d\mu_{}.
    \]
  Portanto, ao juntar as duas desigualdades, obtemos o resultado. \qedsymbol
\end{proof*}
  Vale uma observação - nos teoremas da convergência monótona ou dominada, a convergência \(f_{n}(x)\to f(x)\) pode ser quase sempre. De fato, defina \(A = \{x:f_{n}(x)\to f(x)\}.\) Feito isso, 
  \(f_{n}\chi_{A}(x)\uparrow f \chi_{A}(x)\) para cada x. Como \(\mu (A ^{\complement}) = 0, \) uma aplicação do \hypertarget{dominated_convergence}{Teorema da Convergência Dominada} garante que 
    \[
      \lim_{n\to \infty}\int_{}f_{n} d\mu_{} = \lim_{n\to \infty}\biggl(\int_{}f_{n} \chi_{A} d\mu_{} + \int_{}f_{n}\chi_{A ^{\complement}} d\mu_{}\biggr) = \lim_{n\to \infty}\int_{}f_{n}\chi_{A} d\mu_{} = \int_{}f \chi_{A} d\mu_{} = \int_{}f d\mu_{}.
    \]
\subsection{Integral de Lebesgue Relacionada à de Riemann}
 \begin{prop*}
   Seja f não-negativa, mensurável e \(\int_{}f d\mu_{} = 0.\) Então, \(f=0\) quase sempre.
 \end{prop*}
 \begin{proof*}
   Se f não fosse 0 quase sempre, existe n tal que \(\mu (A_{n}) > 0\), sendo \(A_{n} = \{s: f(x) > 1/n\}\). No entanto, \(f\geq 0\),
     \[
       0 = \int_{}f d\mu_{} \geq \int_{A_{n}}f d\mu_{} \geq \frac{1}{n}\mu (A_{n}),\quad n = 1, 2, \dotsc .
     \]
    Uma contradição. Portanto, f deve ser nula quase sempre. \qedsymbol
 \end{proof*}
\begin{prop*}
  Seja f com valores reais, integrável  e \(\int_{A}f d\mu_{} = 0\) para todo conjunto mensurável A. Então, \(f=0\) quase sempre.
\end{prop*}
\begin{proof*}
  Seja \(A = \{x: f(x) > \varepsilon \}.\) Segue que 
    \[
      0 = \int_{A}f d\mu_{} \geq \int_{A}\varepsilon  d\mu_{} = \varepsilon \mu (A),
    \]
  pois \(f \chi_{A} \geq \varepsilon \chi_{A}\). Logo, \(\mu (A) = 0\). Fazendo \(\varepsilon  = \frac{1}{n}, n = 1, 2, \dotsc ,\) temos 
 \begin{align*}
   \mu (\{x:f(x) > 0\}) &= \mu \biggl(\bigcup_{i=1}^{\infty}\biggl\{x: f(x) > \frac{1}{n}\biggr\}\biggr)\\ 
                        &\leq \sum\limits_{i=1}^{\infty}\mu \biggl(\biggl\{x:f(x)>\frac{1}{n}\biggr\}\biggr) = 0.
 \end{align*}
 Analogamente, \(\mu (\{x:f(x)<0\}) = 0.\) \qedsymbol
\end{proof*}
\begin{crl*}
  Seja m a medida de Lebesgue e \(a\in \mathbb{R}\). Suponha \(f:\mathbb{R}\rightarrow \mathbb{R}\) integrável e \(\int_{a}^{x}f(y)dy = 0\) para todo x. Então, \(f=0\) quase sempre.
\end{crl*}
\begin{proof*}
  Para todo intervalo [c, d], temos 
    \[
      \int_{c}^{d}f dm = \int_{a}^{d} fdm - \int_{a}^{c} fdm = 0.
    \]
  Por linearidade, se G é aberto, digamos \(G = \bigcup_{n=1}^{m}I_{n}\), sendo \(I_{n}\) intervalos disjuntos, então 
    \[
      \int_{G}f dm = \int_{}f \chi_{G} dm = \int_{}f \chi_{\bigcup_{i}^{}I_{i}} dm = \sum\limits_{}^{}\int_{}f \chi_{I_{n}} dm = \sum\limits_{}^{}\int_{I_{n}}f dm = 0.
    \]
  Usando o \hyperlink{dominated_convergence}{\textit{Teorema da Convergência Dominada}}, \(\int_{G}f dm = 0\) para qualquer aberto G que seja a união infinita de intervalos abertos. 

  Se \(I_{n}\) for uma sequência de abertos decrescentes para H, então o \hyperlink{dominated_convergence}{\textit{TCD}} implica que \(\int_{G}f dm = 0,\) pois note que 
  \(f_{n} = f \chi_{I_{n}}\to f \chi_{H}\) e \(|f \chi_{I_{n}}| \leq f.\) Com isso,
    \[
      \int_{H}f dm = \lim_{n\to } \int_{I_{n}}f dm = 0.
    \]
  Caso E seja um conjunto Borel mensurável, existe uma sequência de abertos \(G_{n}\) que decresce para H, sendo H = E quase sempre. Consequentemente, 
    \[
      \int_{E}f dm = \int_{}^{}f \chi_{E} dm = \int_{}f \chi_{H} dm = \int_{H}f dm = 0.
    \]
  Portanto, \(f = 0\) quase sempre. \qedsymbol
\end{proof*}
\newpage

\section{Aula 06 - 16/01/2024}
\subsection{Motivações}
\begin{itemize}
  \item Aproximação por uma Função Contínua; 
  \item Relação Entre Lebesgue e Riemann, Parte 2;
  \item Tipos de Convergências;
  \item Lema de Chebyshev e Teorema de Egorov.
\end{itemize}
\subsection{Relação Entre Riemann e Lebesgue, Parte 2}
\begin{theorem*}
  Suponha f Lebesgue mensurável, função real e integrável sobre \(\mathbb{R}.\) Seja \(\varepsilon >0\). 
Com isso, existe uma função contínua g com suporte compacto tal que 
  \[
    \int_{}|f-g| dm < \varepsilon .
  \]
\end{theorem*}
\begin{proof*}
  Escrevemos \(f = f^{+} - f^{-}.\) Se existir \(g^{\pm} \) contínuas com suporte compacto satisfazendo \(\int_{}|f^{\pm} - g^{\pm}| dm < \varepsilon /2,\) 
tomando \(g = g^{+} - g^{-},\) temos que g é contínua, com suporte compacto e \(\int_{}|f-g| dm < \varepsilon .\) Assim, podemos assumir que \(f\geq 0\) sem grandes problemas. 

Por outro lado, pelo \hyperlink{monotone_convergence}{\textit{Teorema da Convergência Monótona}},
  \[
    \int_{}f \chi_{[-n, n]} dm \uparrow \int_{}f dm,
  \]
o qual é finito. Assim, para n grande suficiente,
  \[
    \biggl\vert \int_{}f dm - \int_{}f \chi_{[-n, n]} dm \biggr\vert < \frac{\varepsilon }{2}.
  \]
No entanto, como \(f - f \chi_{[-n, n]} \geq 0\), temos
  \[
    \int_{}|f - f \chi_{[-n, n]}| dm < \frac{\varepsilon }{2}.
  \]
  Se encontrarmos g contínua de suporte compacto tal que
    \[
      \int_{}|f \chi_{[-n, n]} - g| dm < \frac{\varepsilon }{2},
    \]
  concluímos que por um argumento de dois epsilons,
    \[
      \int_{}|f-g| dm < \varepsilon.
    \]
  Usando o \hyperlink{monotone_convergence}{\textit{TCD,}} pode-se concluir que \(h_{n} = |f \chi_{[-n, n]} - g|\to |f-g|\) quase sempre e \(|h_{n}| \leq |f-g|\) é, assim, integrável. Agora, ganhamos a liberdade de supor que \(f \geq 0\) e 
  f zero fora de um intervalo limitado.  

  Basta supor \(f=\chi_{A},\) sendo A um conjunto limitado Lebesgue mensurável. Escolha G aberto e F fechado tal que \(F\subseteq A\subseteq G\) e \(m(G\setminus{F}) < \varepsilon.\) Na realidade,
  \(m(G\setminus{A}),\) \(m(A\setminus{F}) < \varepsilon /2.\) Use que \((G\setminus{A})\cup (A\setminus{F}).\) Podemos assumir G limitado, o que faz F compacto e, assim, existe uma distância mínima entre F e \(G^{\complement}\), digamos que ela é \(\delta \). Seja 
    \[
      g(x) = \biggl(1 - \frac{\mathrm{dist}(x, F)}{\delta }\biggr)^{+}.
    \]
  Então, g é contínua, \(0\leq g \leq 1, \) g é 1 sobre F, 0 sobre \(G ^{\complement}\) e g tem suporte compacto. Com isso, já que \(F\subseteq A\),
    \[
      |g- \chi_{A}|\leq \chi_{G} - \chi_{F}.
    \]
  Além disso,
    \[
      \int_{}|g-\chi_{A}| dm \leq \int_{}(\chi_{G} - \chi_{F}) dm = m(G\setminus{F})<\varepsilon .
    \]
  Provamos, então, que o resultado é verdadeiro para funções características sobre conjuntos limitados. Faremos as provas para funções simples, não-negativas e gerais. Se 
    \[
      f= \sum\limits_{i=1}^{p}a_{i}\chi_{A_{i}},
    \]
  em que \(A_{i}\subseteq I_{i}, I_{i}\) intervalo limitado e \(a_{i} > 0,\) então existe função contínua \(g_{i}\) de suporte compacto tal que 
    \[
      \int_{}|\chi_{A_{i}} - g_{i}| dm < \frac{\varepsilon }{a_{i}p_{i}},
    \]
  de modo que \(g=\sum\limits_{i=1}^{p}g_{i}\) é a função procurada. 

  Para o caso geral, suponha f não-negativa e com suporte compacto. Existe uma sequência de funções simples \(s_{n}\) com suporte num intervalo limitado e crescendo
para f, \(s_{n}\leq f,\) cuja integral \(\int_{}s_{n} dm \uparrow \int_{}f dm.\) Podemos assumir 
  \[
    \int_{}s_{n} dm \geq \int_{}f dm - \frac{\varepsilon }{2}.
  \]
  Tome g contínua com suporte compacto tal que \(\int_{}|s_{m} - g| dm < \frac{\varepsilon }{2}.\) Portanto, como \(f - s_{n}\geq 0,\)
    \[
      \int_{}|f-g| dm < \varepsilon .\quad \square
    \]
\end{proof*}
  A partir deste instante, denotaremos \(\int_{}f dm\) pra ser a integral de Lebesgue e \(R(f)\) como a integral de Riemann. Recordando a definição da integral de Riemann de \(f:[a, b]\rightarrow \mathbb{R}.\) Seja \(\mathcal{P} = \{x_{0} = a, x_1, \dotsc , x_{n} = b\}\) de [a, b]. 
  Defina 
 \begin{align*}
   &U(\mathcal{P}, f) = \sum\limits_{i=1}^{n}\sup_{x_{i-1}\leq x \leq x_{i}}f(x)[x_{i} - x_{i-1}]\\
   &L(\mathcal{P}, f) = \sum\limits_{i=1}^{n}\inf_{x_{i-1}\leq x \leq x_{i}}f(x)[x_{i} - x_{i-1}]\\
   &\overline{R}(f) = \inf_{}\{U(\mathcal{P}, f): \mathcal{P} \text{ partição}\}
   &\underline{R}(f) = \inf_{}\{L(\mathcal{P}, f): \mathcal{P} \text{ partição}\}.
 \end{align*}
  Dizemos que a integral de Riemann existe se \(\overline{R}(f) = \underline{R}_{}(f)\), sendo o valor comum denotado por R(f).
 \begin{theorem*}
   Seja \(f:[a, b]\rightarrow \mathbb{R}\) limitada. Então, f é Riemann integrável se, e somente se, \(m(D_f) = 0\), em que \(D_f = \{x: f \text{ é descontínua no ponto x}\}\) é o conjunto dos pontos 
   de descontinuidade de f 

   Neste caso, f é Lebesgue-mensurável e a integral de Riemann de f coincidindo com o integral, ou seja,
     \[
       \int_{}f dm = R(f),\quad \text{\qedsymbol}.
     \]
 \end{theorem*}
\begin{proof*}
  Passo 1. Mostraremos que se f é Riemann integrável, então f é contínua quase sempre e \(R(f) = \int_{}f dm.\) Se \(\mathcal{P}\) é uma partição, defina 
    \[
      T_{\mathcal{P}}(x) = \sum\limits_{i=1}^{n}\chi_{[x_{i-1}, x_{i})}(x)\sup_{x_{i-1}\leq y\leq x_{i}}f(y) \quad\&\quad S_{\mathcal{P}}(x) = \sum\limits_{i=1}^{n}\chi_{[x_{i-1}, x_{i})}(x)\inf_{x_{i-1}\leq y\leq x_{i}}f(y).
    \]
  Note que 
    \[
      \int_{}T_{\mathcal{P}} dm = U(\mathcal{P}, f) \quad\&\quad \int_{}S_{\mathcal{P}} dm = L(\mathcal{P}, f).
    \]
  Se f é Riemann integrável, existe uma sequência de partições \(Q_{i}\) tal que \(U(Q_{i}, f)\downarrow R(f)\) e uma sequência \(Q_{i}'\) tal que \(L(Q_{i}', f)\uparrow R(f).\) Note que, ao adicionar pontos a uma partição L, ela cresce e, 
  na partição U, ela decresce. Assim, seja \(P_{i} = \bigcup_{j\geq i}^{}(Q_{j}\cup Q_{j}')\), de modo que \(P_{i}\) é uma sequência crescente de partições tais que 
    \[
      U(P_{i}, f)\downarrow R(f)\quad\&\quad L(P_{i}, f)\uparrow R(f)>
    \]
  Observe que existem T(x) e S(x) satisfazendo 
    \[
      T_{P_{i}}(x)\downarrow T(x) \quad\&\quad S_{P_{i}}(x)\uparrow S(x),\quad T(x)\geq f(x)\geq S(x).
    \]
  Como f é limitada, pelo \hyperlink{dominated_convergence}{\textit{Teorema da Convergência Dominada}}, temos 
    \[
      \int_{}(T-S) dm = \lim_{i\to \infty}\int_{}(T_{P_{i}} - S_{P_{i}}) dm = \lim_{i\to \infty}(U(P_{i}, f) - L(P_{i}, f)) = 0.
    \]
  Consequentemente, \(T = S = f\) quase sempre. 

  Agora, note que T é limite de funções escada, o que a torna Borel mensurável e, para cada a, 
    \[
      \{x:f(x)>a\}\quad\&\quad \{x:T(x)>a\}
    \]
  diferem somente num conjunto de medida nula, donde conclui-se que f é Lebesgue Mensurável. Se x não está no conjunto de medida nula em que \(T(x)\neq S(x)\) nem em \(U_{i}, P_{i}\), 
o qual é enumerável e, consequentemente, de medida nula, então 
  \[
    T_{P_{i}}(x)\downarrow f(x) \quad\&\quad S_{P_{i}}(x)\uparrow f(x).
  \]
  Dado \(\varepsilon \), escolha i suficientemente grande para que 
    \[
      T_{P_{i}} - S_{P_{i}} < \varepsilon 
    \]
  e escolha \(\delta \) pequeno tal que \((x-\delta , x+\delta )\) esteja contido num subintervalo de \(P_{i}\) contendo x. Aplicando o \hyperlink{dominated_convergence}{\textit{Teorema da Convergência Dominada}} e notando que 
    \[
      R(f) = \lim_{i\to \infty}U(P_{i}, f) = \lim_{i\to \infty}\int_{}T_{P_{i}} dm = \int_{}f dm,
    \]
  concluímos que as duas integrais são iguais.

  Para o segundo passo, suponha que f seja contínua quase sempre e tome \(\varepsilon >0\) qualquer. Seja \(P_{i}\) partição que divide \([a, b]\) em \(2^{i}\) partes iguais. Se x não está no conjunto de medida nula onde f é descontínua, nem 
  em \(U_{i}, P_{i}\), então 
    \[
      T_{P_{i}}(x)\downarrow f(x)\quad\&\quad S_{P_{i}}(x)\uparrow f(x).
    \]
  Assim como no passo 1, f é mensurável, e o TCD nos fornece 
    \[
      U(P_{i}, f) = \int_{}T_{P_{i}} dm\to \int_{}f dm,\quad L(P_{i}, f) = \int_{}S_{P_{i}} dm\to \int_{}f. dm
    \]
  Portanto, \(\overline{R}(f) = \underline{R}(f) = R(f),\) ou seja, \(R(f)\) é integrável. \qedsymbol
\end{proof*}
\begin{example}
  Considere 
    \[
      f(x) = \left\{\begin{array}{ll}
          1,\quad x\in [0,1]\cap \mathbb{Q}^{\complement}\\ 
          0,\quad x\in [0,1]\cap \mathbb{Q}.
        \end{array}\right.
    \]
    Note que \(D_{f} = [0, 1]\) é o conjunto de descontinuidade da f e \(m(D_{f}) = 1,\) donde conclui-se que f não é Riemann integrável. No entanto, 
  \(f=1\) quase sempre, tal que 
    \[
      \int_{}f dm = 1.
    \]
  Portanto, f é Lebesgue integrável, mas não é Riemann integrável. 
\end{example}
\begin{example}
  Coloque 
    \[
      f(x) = \left\{\begin{array}{ll}
          1,\quad x\in \mathbb{R} \quad \&\quad x>0\\ 
          0,\quad x\in \mathbb{R} \quad \&\quad  x < 0. 
        \end{array}\right.
    \]
  Aqui, \(D_{f} = 0\), tal que \(m(D_{f}) = 0\) e, assim, f não é Riemann integrável, pois f tem domínio ilimitado. Além disso,
    \[
      \int_{}f dm = \int_{}1 \chi_{\mathbb{R}_{x > 0}} dm + \int_{}0\chi_{\mathbb{R}_{x < 0}} dm = \infty,
    \]
  tal que f não é Lebesgue integrável.
\end{example}
\begin{example}
  Coloque 
    \[
      f(x) = \left\{\begin{array}{ll}
          1,\quad x\in [0, 1]\cap \mathbb{Q}^{\complement}\\
          \frac{1}{q},\quad x\in[0,1], x = \frac{p}{q}\text{ (irredutível)}
        \end{array}\right..
    \]
    Observe que o domínio de descontinuidade de f é \(D_f = \mathbb{Q}\), tal que \(m(D_{f}) = 0\) e, portanto, f é Riemann Integrável. Obesrva-se que f = 0 quase sempre, tal que 
      \[
        \int_{}f dm = 0,
      \]
    mostrando que ela também é Lebesgue integrável. 

    Vale uma obesrvação: Note que \(\lim_{x\to a}f(x) = 0\) para todo \(a\in \mathbb{R},\) de forma que f é contínua em \(a\in \mathbb{Q}^{\complement}\). Para ver que isso é verdade, se \(a\in \mathbb{R}\) e \(\varepsilon > 0\) for dado, 
    o conjunto \(A_{\varepsilon } = \{q: q \leq \varepsilon^{-1}\}\) é finito. Assim, tomando \(\delta  < \mathrm{dist}(a, A_{\varepsilon }),\) esse conjunto de tamanho \(\delta \) não contém \(x = \frac{p}{q}\) com \(q\in A_{\varepsilon },\) isto é, 
    \(q\not\in a_{\varepsilon }.\) Mais exatamente, 
      \[
        q > \frac{1}{\varepsilon } \Longleftrightarrow \frac{1}{q} < \varepsilon .
      \]
\end{example}
\subsection{Tipos de Medidas}
  Seja, ao longo dessa seção, \(\mu \) uma medida, \(f_{n}, f\) mensuráveis. 
 \begin{def*}
   Dizemos que \(f_{n}\) \textbf{converge quase sempre} para f se existir um conjunto de medida nula A tal que, se \(x\not\in A\), temos \(f_{n}(x)\to f(x).\) Notação: \(f_{n}\to f\) q.s. \(\square\)
 \end{def*}
 \begin{def*}
   Dizemos que \(f_{n}\) \textbf{converge em medida} para f, se para cada \(\varepsilon  > 0\),
     \[
       \mu (\{x: |f_{n}(x) - f(x)| > \varepsilon \})\to 0,\quad n\to \infty.\quad \square 
     \]
 \end{def*}
 \begin{def*}
   Seja \(1\leq p < \infty. f_{n}\)\textbf{ converge em }\(L^{p}\)\textbf{ para f}, se 
     \[
       \int_{}|f_{n} - f|^{p} d\mu\to 0,\quad n\to \infty.
     \]
 \end{def*}
\begin{prop*}
  \begin{itemize}
    \item[i)]Suponha que \(\mu \) seja uma medida finita. Se \(f_{n}\to f\) q.s., então \(f_{n}\) converge para f em medida.
    \item[ii)] Suponha que \(\mu \) é uma medida não necessariamente finita. Se \(f_{n}\to f\) em medida, então existe uma subsequência \(n_{j}\) tal que \(f_{n_{j}}\) converge para f q.s.
  \end{itemize}
\end{prop*}
\begin{proof*}
  Seja \(\varepsilon > 0\) e suponha que \(f_{n}\to f\) q.s. Se 
    \[
      A_{n} = \mu (\{x: |f_{n}(x) - f(x)| > \varepsilon \}),
    \]
  então \(\chi_{A_{n}}\to 0\) q.s. e, pelo \hyperlink{dominated_convergence}{\textit{Teorema da Convergência Dominada}}, 
    \[
      \mu (A_{n}) = \int_{}\chi_{A_{n}}(x) \mu_{}(dx)\to 0,
    \]
  provando (1).

  Para o (2), suponha que \(f_{n}\to f\) em medida, seja \(n_1 = 1\) e escolha \(n_{j} > n_{j-1}\) por indução, tal que 
    \[
      \mu (\{x: |f_{n_{j}}(x) - f(x)| > 1/j\} \leq 2^{-j}.
    \]
  Seja \(A_{j} = \{x: |f_{n_{j}}(x) - f(x)| > 1/j\}.\) Se colocarmos 
    \[
      A = \bigcap_{k=1}^{\infty}\bigcup_{j=k}^{\infty}A_{j},
    \]
  teremos 
    \[
      \mu (A) = \lim_{k\to \infty}\mu \biggl(\bigcup_{j=k}^{\infty}A_{j}\biggr) \leq \lim_{k\to \infty}\sum\limits_{j=k}^{\infty}\mu (A_{j}) \leq \lim_{k\to \infty}2^{-k+1} = 0.
    \]
  Logo, A tem medida nula. Se \(x\not\in A,\) então \(x\not\in \bigcup_{j=k}^{\infty}A_{j}\) para algum k e, assim, 
    \[
      |f_{n}(x)-f(x)|\leq \frac{1}{j},\quad \forall j\geq k
    \]
  Portanto, \(f_{n}\to f\) sobre \(A ^{\complement}\). \qedsymbol
\end{proof*}
  Notação: Se \(A = \bigcap_{k=1}^{\infty}\bigcup_{j=k}^{\infty}A_{j},\) então \(x\in A\) se, e somente se, \(x\in A_{j}\) para uma infinidade de j. Denotamos por \(A = \{A_{j}\mathrm{i.o.}\}.\)
\begin{example}
  Se \(\mu (X) = \infty\), não vale que ``Se \(f_{n}\to f\) q.s., então \(f_{n}\) converge para f em medida''. De fato, tome \(X = \mathbb{R}\) e seja \(f_{n} = \chi_{(n, n+1)}.\) Temos \(f_{n}\to 0\) quase sempre, mas \(f_{n}\) não converge em medida, pois 
    \[
      \mu (\{x: |f_{n}(x) - f(x)|> \varepsilon \}) = \mu (\{x: |\chi_{(n, n+1)}| > \varepsilon \} = \mu (n, n+1)\not\to 0.
    \]
\end{example}
 \hypertarget{chebyshev}{
   \begin{lemma*}[Chebyshev]
    Se \(a\leq p < \infty\), então 
      \[
        \mu (\{x: |f(x)| \geq a\}) \leq \frac{\int_{}|f|^{p} d\mu_{}}{a^{p}}.
      \]
 \end{lemma*}}
 \begin{proof*}
   Seja \(A = \{x: |f(x)| \geq a\}.\) Como \(\chi_{A} \leq \frac{|f|^{p}\chi_{A}}{a^{p}},\) temos 
     \[
       \mu (A) \leq \int_{A}\frac{|f|^{p}}{a^{p}} d\mu_{} \leq \frac{1}{a^{p}}\int_{}|f|^{p} d\mu_{}. \quad \text{\qedsymbol}
     \]
 \end{proof*}
\begin{prop*}
  Se \(f_{n}\) converge para f em \(L^{p},\) então \(f_{n}\) converge para f em medida.
\end{prop*} 
\begin{proof*}
  Da \hyperlink{chebyshev}{\textit{desigualdade de Chebyshev}}, para \(\varepsilon >0\), 
    \[
      \mu (\{x: |f_{n}(x) - f(x)| > \varepsilon \}) \leq \frac{1}{\varepsilon ^{p}}\int_{}|f_{n}-f|^{p} d\mu_{}\to 0,
    \]
  como desejado. \qedsymbol
\end{proof*}
\begin{example}
  Seja \(f_{n} = n^{2}\chi_{(0, 1/n)}\) sobre \([0, 1]\) e seja \(\mu \) a medida de Lebesgue. Note que \(f_{n}\) converge a 0 quase sempre e em medida, mas não converge em \(L^{p}\) para \(p \geq 1\).

  Com efeito, dado x, existe n tal que \(\frac{1}{n} < x\) e, então, \(f_{n}(x) = 0.\) Assim, \(f_{n}\) converge a 0 quase sempre. Agora, 
    \[
      \mu (\{x: |f_{n}(x) - f(x)| > \varepsilon \} = \mu (\{x: |f_{n}(x)| > \varepsilon \}) \leq \mu \biggl(\biggl[0, \frac{1}{n}\biggr]\biggr)\to 0
    \]
  converge em medida. No entanto, 
    \[
      \int_{}|f_{n}-0|^{p} d\mu_{} = \int_{0}^{\frac{1}{n}}n^{2p}dx = n^{2p-1}\to \infty,
    \]
  ou seja, não converge em \(L^{p}.\)
\end{example}
\begin{example}
  Também podemos construir um exemplo em que a sequência converge em medida e em \(L^{p},\) mas não quase sempre. 

  De fato, tome \(S = \{e^{i\theta }: 0 \leq \theta < 2\pi \}\) como o círculo unitário no plano complexo e defina 
    \[
      \mu (A) = m(\{\theta \in [0, 2\pi ): e^{i\theta }\in A\}
    \]
  como a medida do comprimento de arco sobre S, sendo m a medida de Lebesgue sobre \([0, 2\pi ).\) Seja \(X = S\) e \(f_{n}(x) = \chi_{F_{n}}(x),\) em que 
    \[
      F_{n} = \biggl\{e^{i\theta }: \sum\limits_{j=1}^{n}\frac{1}{j} \leq \theta \leq \sum\limits_{j=1}^{n+1}\frac{1}{j}\biggr\} \Rightarrow \mu (F_{n}) \leq \frac{1}{n+1}\to 0,
    \]
  ou seja, \(f(e^{i\theta }) = 0\) para todo \(\theta \), do que concluímos que \(f_{n}\to f\) em medida. Além disto, como \(f_{n}\) é 1 ou 0, 
    \[
      \int_{}|f_{n}-f|^{p} d\mu_{} = \int_{}\chi_{F_{n}} d\mu_{} = \mu (F_{n})\to 0.
    \]
  Porém, como \(\sum\limits_{j=1}^{\infty}\frac{1}{j} = \infty,\) cada ponto de S está infinitamente em \(F_{n}\), e cada ponto de S está em \(S\setminus{F_{n}}\) para uma infinidade de n. Assim, 
  \(f_{n}\) não converge para f em qualquer ponto, pois podemos construir duas subsequência \(fn_{j}(x) = 1, x\in F_{n}\) e \(fn_{k}(x) = 0, x\in S\setminus{F_{n}}.\) Logo, a subsequência não pode convergir. 

  Note que \(F_{n}\) são arcos cujos comprimentos tendem a 0, mas tais que \(\bigcup_{n\geq m}^{F_{n}}\) contém S para cada m.
\end{example}
 \hypertarget{egorov}{
   \begin{theorem*}[Teorema de Egorov]
  Suponha \(\mu \) medida finita, \(\varepsilon > 0\) dado e \(f_{n}\to f\) q.s. Então, existe um conjunto enumerável A tal que \(\mu (A) < \varepsilon \) e \(f_{n}\to f\) uniformemente em \(A ^{\complement}\).
\end{theorem*}}
\begin{proof*}
  Seja 
    \[
      A_{nk} = \bigcup_{m=n}^{\infty}\biggl\{x: |f_{m}(x) - f(x)| > \frac{1}{k}\biggr\}.
    \]
  Fixado k, \(A_{nk}\) decresce quando n cresce. A interseção \(\bigcap_{n}^{}A_{nk}\) tem medida 0 pois quase todo x satisfaz 
    \[
      |f_{m}(x) - f(x)| \leq \frac{1}{k}
    \]
  para m suficientemente grande. Logo, \(\mu (X) < \infty\) e \(\lim_{} \mu (A_{nk}) = \mu (\bigcap_{n}^{}A_{nk}) = 0, \mu (A_{nk})\to 0\) quando \(n\to \infty\). Desta forma, existe \(n_{k}\) tal que 
 \(\mu (A_{n_kk}) < \varepsilon 2^{-k}. \) Coloque 
   \[
     A = \bigcup_{k=1}^{\infty}A_{n_kk}.
   \]
  Com isso, \(\mu (A) < \varepsilon \) e, se \(x\not\in A\), então \(x\not\in A_{n_{k}k}\) e, assim, 
    \[
      |f_{n}(x) - f(x)| \leq \frac{1}{k}
    \]
  se \(n \geq n_{k}.\) Portanto, \(f_{n}\to f\) uniformemente em \(A ^{\complement}.\) \qedsymbol
\end{proof*}

\newpage

\section{Aula 07 - 17/01/2024}
\subsection{Motivações}
\begin{itemize}
  \item Produto de Medidas; 
  \item Teorema de Fubini-Tonelli;
  \item Medidas com Sinais;
  \item Teorema de Hahn.
\end{itemize}
\subsection{Produto de Medidas}
\begin{def*}
  Sejam \((X, \mathcal{A}, \mu )\) e \((Y, \mathcal{B}, \nu)\) dois espaços de medida. Um \textbf{retângulo mensurável} é um 
  conjunto da forma \(A\times B\), em que \(A\in \mathcal{A}\) e \(B\in \mathcal{B}.\quad \square\)
\end{def*}
Seja \(\mathcal{C}_{0}\) a coleção da união finita e disjunta de retângulos mensuráveis, ou seja, todo elemento de \(\mathcal{C}_{0}\) é da forma 
  \[
    \bigcup_{i=1}^{n}(A_{i}\times B_{i}),\quad A_{i}\in \mathcal{A}, B_{i}\in \mathcal{B}
  \]
e 
  \[
    (A_{i}\times B_{i})\cap (A_{j}\times B_{j}) = \emptyset , i\neq j.
  \]
  Como 
    \[
      (A\times B)^{\complement} = (X\times B ^{\complement})\cup (A ^{\complement}\times Y)
    \]
    e a interseção de dois retângulos mensuráveis é um retângulo mensurável, então é fácil ver que \(\mathcal{C}_{0}\) é uma álgebra de conjunto. Definimos a \(\sigma \)-álgebra produto como sendo 
      \[
        \mathcal{A}\times \mathcal{B} = \sigma (\mathcal{C}_{0}).
      \]
    Se \(E\subseteq X\times Y\), definimos a x-seção de E por 
      \[
        s_x(E) = \{y\in Y: (x, y)\in E\},
      \]
    analogamente a y-seção de E por 
      \[
        t_y(E) = \{x\in X: (x, y)\in E\}.
      \]
    Dada \(f:X\times Y\rightarrow \mathbb{R},\) para cada x e y, definimos 
      \[
        S_{x}f :Y\rightarrow \mathbb{R}, \quad T_{y}f:X\rightarrow \mathbb{R}
      \]
    descritas por \(S_xf(y) = f(x, y)\) e \(T_yf(x) = f(x, y).\)
\begin{lemma*}
 \begin{itemize}
   \item[1)] Se \(E\in \mathcal{A}\times \mathcal{B},\) então \(s_x(E)\in \mathcal{B}\) para cada x e \(t_y(E)\in \mathcal{A}\) para cada y.
   \item[2)] Se f é \(\mathcal{A}\times \mathcal{B}\)-mensurável, então \(S_{x}f\) é \(\mathcal{B}\)-mensurável para cada x e \(T_yf\) é \(\mathcal{A}\)-mensurável para cada y.
 \end{itemize}
\end{lemma*}
\begin{proof*}
  Para o item 1, seja \(\mathcal{C}\) a coleção de conjuntos em \(\mathcal{A}\times \mathcal{B}\) tais que \(s_x(E)\in \mathcal{B}\) para cada x. Se 
    \[
      E = A\times B \Rightarrow s_x(E)  = \left\{\begin{array}{ll}
          B,\quad x\in A\\ 
          \emptyset ,\quad x\not\in A.
        \end{array}\right.
    \]
    Logo, \(s_x(E)\in \mathcal{B}\) para cada x quando E é um retângulo mensurável. 

    Se \(E\in \mathcal{C}\) e \(x\in X\), note que, caso \(s_x(E) = \emptyset \), então \(s_x(E ^{\complement}) = s_x(E)^{\complement} = Y.\) Por outro lado, caso \(s_x(E)\neq\emptyset\),
então \(y\in s_x(E ^{\complement})\) se, e somente se, \((x, y)\in E ^{\complement}\), ou seja, \(y\not\in s_x(E).\) Destarte, 
  \[
    s_x(E ^{\complement}) = (s_x(E))^{\complement}.
  \]
  Além disso, \(\mathcal{C}\) é fechado com relação a complementos. Analogamente, 
    \[
      s_x \biggl( \bigcup_{i=1}^{\infty}E_{i}\biggr) = \bigcup_{i=1}^{\infty}s_x(E_{i}),
    \]
    pois dado \(y\in s_x(\cup_i E_{i}\), isto significa que \((x, y)\in \cup_i E_{i}\), ou seja, \((x, y)\in E_{i}\) para algum i. Logo, \(y\in s_x(E_{i})\) e \(y\in \bigcup_{i}^{}s_{x}(E_{i}).\)
  Assim, \(\mathcal{C}\) é fechado com relação à união contável. Logo, \(\mathcal{C}\) é uma \(\sigma \)-álgebra contendo retângulos mensuráveis, tal que 
    \[
      \mathcal{C} = \mathcal{A}\times \mathcal{B}.
    \]
    Analogamente para \(t_y(E).\)

    Para o item (2), fixe x. Se \(f=\chi_{E}\) para \(E\in \mathcal{A}\times \mathcal{B}.\) Note que 
      \[
        S_xf(y) = \chi_{S_x(E)}(y),
      \]
    o qual é \(\mathcal{B}\)-mensurável. De fato, seja \(a\in \mathbb{R},\) \(x\in X\), 
      \[
        \{y: S_{x}f(y) > a\} = \{y:f(x, y) > a\} = \{(x, y)\in A\times B: f(x, y) > a\}_{x}.
      \]
    Pela linearidade, \(S_{x}f\) é \(\mathcal{B}\)-mensurável, quando f é simples. Se \(f\geq 0\), uma sequência \(\mathcal{A}\times \mathcal{B}\)-mensurável simples, \(\{r_{n}\}, r_{n}\uparrow f\), e como 
      \[
        S_xr_{n}\uparrow S_{x}f,
      \]
    conclui-se que \(S_xf\) é \(\mathcal{B}\)-mensurável. Escrevendo \(f=f^{+}-f^{-}\) e por meio da linearidade, mostra-se que \(S_{x}f\) é \(\mathcal{B}\)-mensurável. Analogamente, prova-se as mesmas coisas
    para \(T_{y}f\). \qedsymbol
\end{proof*}
\begin{prop*}

  Suponha \(\mu \) e \(\nu\) são medidas \(\sigma \)-finitas. Seja \(E\in \mathcal{A}\times \mathcal{B}\) e sejam
    \[
      h(x) = \nu(s_x(E)),\quad\&\quad k(y) = \mu (t_y(E)).
    \]
  Então, 
 \begin{itemize}
   \item[i)] h é \(\mathcal{A}\)-mensurável e k é \(\mathcal{B}\)-mensurável
     \item[ii)] Vale 
       \[
         \int_{}h(x) \mu (dx) = \int_{}^{}k(y)\nu(dy),
       \]
      o qual é equivalente a 
        \[
          \int_{}^{}\biggl(\int_{}^{}\chi_{E}(x, y)\nu(dy)\biggr)\mu (dx) = \int_{}^{}\biggl(\int_{}^{}\chi_{E}(x, y)\mu (dx)\biggr)\nu(dy).
        \]
 \end{itemize}
\end{prop*}
 \begin{proof*}
   A equivalência vem do seguinte: Note que 
     \[
       h(x) = \nu(s_x(E)) =
     \]
     aqui usamos \(y\in s_x(E)\) e, e somente se, \((x, y)\in E\) 

    Suponha que \(\nu\) e \(\mu \) são finitas. Seja \(\mathcal{C}\) a coleção de conjuntos em \(\mathcal{A}\times \mathcal{B}\) para o qual (1)
  e (2) valem. Sendo \(\mathcal{C}_{0}\) a coleção da união disjunta de retângulos, provaremos que \(\mathcal{C}\supseteq \mathcal{C}_{0}\) e monótona. 
  Com efeito, se \(E = A\times B\), com \(A\in \mathcal{A}\) e \(B\in \mathcal{B}\), então \(h(x) = \chi_{A}(x)\nu(B)\), o qual é \(\mathcal{A}\)-mensurável e 
    \[
      \int_{}h(x) \mu (dx) = \mu (A)\nu(B).
    \]
  Aqui, \(h(x) = \nu(s_x(E)) = \nu(B),\) \(x\in A = \nu(B)\chi_{A},\) \(E = A\times B\), pois \(y\in s_x(E) \) equivale a \(x\in  A, y\in B\). Analogamente, 
    \[
      k(y) = \mu (A)\chi_{B}(y)
    \]
  é \(\mathcal{B}\)-mensurável e 
    \[
      \int_{}k(y)\nu( dy) = \mu (A)\nu(B).
    \]
  Com isso, provamos que 1 e 2 valem para retângulos mensuráveis. Se \(E = \bigcup_{i=1}^{n}E_{i},\) em que cada \(E_{i}\) é retângulo mensurável e \(E_{i}\) são disjuntos, então \(s_x(E) = \bigcup_{i=1}^{n}s_x(E_{i})\)
e, como \(s_x(E_{i})\) são disjuntos, temos 
  \[
    h(x) = \nu(s_x(E)) = \nu \biggl(\bigcup_{i=1}^{n}s_x(E_{i})\biggr) = \sum\limits_{i=1}^{n}\nu(s_x(E_{i})).
  \]
  Isso prova que h é \(\mathcal{A}\)-mensurável e k é \(\mathcal{B}\)-mensurável, ambos por serem somas de funções mensuráveis. Se colocarmos \(h_{i}(x) = \nu(s_x(E_{i}))\) e k(y) similarmente, então 
    \[
      \int_{}h_{i}(x)\mu ( dx) = \int_{}k_{i}(y)\nu( dy)
    \]
  provando que \(\mathcal{C}\) contém \(\mathcal{C}_{0}\). Agora, suponha que \(E_{n}\uparrow E\) e cada \(E_{n}\in \mathcal{C}\). Ponha \(h_{n}(x) = \nu(s_x(E_{n}))\) e \(k_{n}(y) = \mu (t_y(E_{n}))\), tal que 
  \(h_{n}\uparrow h\) e \(k_{n}\uparrow k\). Portanto, \(h_{n}\) é \(\mathcal{A}\)-mensurável e \(k_{n}\) é \(\mathcal{B}\)-mensurável, donde segue que (2) vale e 
    \[
      \int_{}h_{n}(x)\mu ( dx) = \int_{}k_{n}(y)\nu( dy).
    \]
  Passando o limite de n tendendo a infinito e aplicando o \hyperlink{monotone_convergence}{\textit{Teorema da Convergência Monótona}}, segue que (2) vale para h e k. 

  Agora, assuma que \(E_{n}\downarrow E\) e cada \(E_{n}\in \mathcal{C}.\) Procedendo como acima, mas usando o \hyperlink{dominated_convergence}{\textit{Teorema da Convergência Dominada}} junto com a finitude de \(\nu\) e \(\mu \), obtemos o resultado. 
Assim, \(\mathcal{C}\) é classe moonótona contendo \(\mathcal{C}_{0}\) e contido em \(\mathcal{A}\times \mathcal{B}\), tal que \(\mathcal{C} = \sigma (\mathcal{C}_{0})\), o qual é \(\mathcal{A}\times \mathcal{B}.\)

  Suponha, agora, que \(\mu \) e \(\nu\) são \(\sigma \)-finitas. Existem \(F_{i}\uparrow X\) e \(G_{i}\uparrow Y\), cada \(F_{i}\) e \(G_{i}\) sendo \(\mathcal{A}, \mathcal{B}\)-mensuráveis e \(\mu (F_{i}), \nu(G_{i})< \infty\).
Sejam \(\mu_{i}(A) = \mu (A\cap F_{i})\) para cada \(A\in \mathcal{A}\) e \(\nu_{i}(B) = \nu(B\cap G_{i})\) para cada \(B\in \mathcal{B}\). Seja 
  \[
    h_{i}(x) = \nu_{i}(s_x(E)) = \nu(s_x(E)\cap G_{i})
  \]
  e defina \(k_{i}(y)\) de forma análogo, donde segue que 
    \[
      \int_{}h_{i}(x)\mu_{i}( dx) = \int_{}h_{i}(x)\chi_{F_{i}}(x)\mu ( dx),
    \]
  sendo o análogo verdadeiro para \(k_{i}, G_{i}\) e \(\nu_{i}\).

  Com o que vimos até agora, \(h_{i}\) é \(\mathcal{A}\)-mensurável e \(k_{i}\) é \(\mathcal{B}\)-mensurável, do que segue que (2) vale para h, k trocados por \(h_{i}, k_{i}\). Tomando, agora, 
  \(h_{i}\uparrow h\) e \(k_{i}\uparrow k\), obtemos a mensurabilidade de h e k. Portanto, pelo TCM, concluímos a prova. \qedsymbol
\end{proof*}
  Vale uma observação. Defina \(\mu \times \nu\) pondo 
    \[
      \mu \times \nu (E) = \int_{}h(x)\mu ( dx) = \int_{}k(y)\nu( dy),
    \]
  em que \(h(x) =\nu(s_x(E)) \) e \(k(y) = \mu(t_y(E)).\) Para ver que isso é medida mesmo, temos \(\mu \times \nu(\emptyset ) = 0\). Se \(E_1, E_2, \dotsc , E_{n}\) são disjuntos em \(\mathcal{A}\times \mathcal{B}\) e \(E = \bigcup_{i=1}^{n}E_{i},\) então 
    \[
      \nu(s_x(E)) = \sum\limits_{i=1}^{n}\nu(s_x(E_{i})).
    \]
    Daí, 
   \begin{align*}
     \mu \times \nu(E) &= \int_{}\nu(s_x(E))\mu ( dx) = \sum\limits_{i=1}^{n}\int_{}\nu(s_x(E_{i})) \mu(dx) \\ 
                       &= \sum\limits_{i=1}^{n}\mu \times \nu (E_{i}),
   \end{align*}
   provando a aditividade finita de \(\mu \times \nu\).

   Se \(E_{n}\uparrow E\) com \(E_{n}\in \mathcal{A}\times \mathcal{B},\) defina \(h_{n}(x) = \nu(s_x(E_{n})),\) donde segue, pelo TCM, que \(h_{n}\uparrow h\) e 
     \[
       \mu \times \nu(E_{n})\uparrow \mu \times \nu(E).
     \]
  Portanto, \(\mu \times \nu\) é, de fato, medida. 

  Note que, se \(E = A\times B\) é retângulo mensurável, então \(h(x) = \chi_{A}(x)\nu(B)\) e, assim, como na intuição, 
    \[
      \mu \times \nu(A\times B) = \mu (A) \nu(B).
    \]
    \hypertarget{fubini_tonelli}{
   \begin{theorem*}
     Suponha que \(f:X\times Y\rightarrow \mathbb{R}\) é mensurável com relação a \(\mathcal{A}\times \mathcal{B}\) e que \(\mu \) e \(\nu\) são medidas \(\sigma \)-finitas sobre X e Y, respectivamente. Se ocorrer que 
    \begin{itemize}
      \item[a)] \(f\geq 0\), ou 
        \item[b)] \(\int_{}|f(x, y)| d(\mu\times \nu_{})(x, y) < \infty,\)
    \end{itemize}
    então 
   \begin{itemize}
     \item[1)] Para cada x, a função \(y\mapsto f(x, y)\) é mensurável com relação a \(\mathcal{B}\);
       \item[2)] Para cada y, a função \(x\mapsto f(x, y)\) é mensurável com relação a \(\mathcal{A}\);
         \item[3)] A função \(h(x) = \int_{}f(x, y)\nu( dy)\) é mensurável com relação a \(\mathcal{A}\);
         \item[4)] A função \(k(x) = \int_{}f(x, y)\mu( dy)\) é mensurável com relação a \(\mathcal{B}\);
           \item[5)] Vale 
            \begin{align*}
              \int_{}f(x, y) d(\mu\times \nu)(x, y)_{} &= \int_{}\biggl(\int_{}^{}f(x, y)\mu (dx)\biggr) \nu(dy)\\ 
                                                       &= \int_{}^{}\biggl(\int_{}^{}f(x, y)\nu(dy)\biggr)\mu (dx).
            \end{align*}
   \end{itemize} 
   \end{theorem*}}
  A última integral deve ser interpretada como 
    \[
      \int_{}^{}\biggl(\int_{}^{}k(y)\nu(dy)\biggr)\mu (dx),
  \]
  analogamente para a outra integral. 
 \begin{proof*}
   Se \(f = \chi_{E}, E\subseteq \mathcal{A}\times \mathcal{B},\) então (1)-(5) são reescritas em resultados anteriores. 
Por linearidade, vale para f simples. Sendo o limite de funções mensuráveis crescentes também mensurável, então, escrevendo uma função \(f\geq 0\) como um limite de funções crescentes de funções 
simples, e usando o TCM, os itens (1)-(5) valem para \(f\geq 0\). No caso \(\int_{}|f| d(\mu\times\nu_{}) < \infty\), escrevendo \(f=f^{+}-f^{-}, f^{\pm}\geq 0\) e procedendo como no caso das não-negativas,
as propriedades (1)-(5) valem. \qedsymbol
 \end{proof*}
 Suponha que 
  \[
    \int_{}^{}\int_{}^{}|f(x, y)|\mu (dx)\nu(dy) < \infty,
  \]
e, como \(|f(x, y)| \geq 0,\) o \hyperlink{fubini_tonelli}{\textit{Teorema de Fubini}} nos diz que 
  \[
    \int_{}^{}|f(x, y)|d(\mu \times \nu) = \int_{}\int_{}^{}|f(x, y)|\mu (dx)\nu(dy) < \infty.
  \]
  Outra aplicação de \hyperlink{fubini_tonelli}{\textit{Fubini}} fornece 
    \[
      \int_{}^{}f(x, y)d(\mu \times \nu) = \int_{}^{}\int_{}^{}f(x, y)\mu (dx)\nu(dy) = \int_{}^{}\int_{}^{}f(x, y)\nu (dy)\mu (dx) < \infty.
    \]
  Desta forma, na hipótese do Teorema, poderíamos supor 
    \[
      \int_{}^{}\int_{}^{}f(x, y)\mu (dx)\nu(dy) < \infty \quad \text{ou}\quad \int_{}^{}\int_{}^{}f(x, y)\nu(dy)\mu (dx) < \infty.
    \]
  Este teorema pode ser estendido para m medidas \(\mu_1, \mu _2, \dotsc , \mu_m\).

  \underline{\textbf{Observação:}} Se \((X, \mathcal{A}, \mu )\) e \((Y, \mathcal{B}, \nu)\) são completos, não necessariamente \((X\times Y, \mathcal{A}\times \mathcal{B}, \mu \times \nu)\) será completo. 
  Por exemplo, seja \((X, \mathcal{A}, \mu ) = (Y, \mathcal{B}, \nu)\) a medida de Lebesgue em \([0, 1]\) com \(\sigma \)-álgebra de Lebesgue. Tome A não mensurável em [0, 1] e seja 
  \(E = A\times \biggl\{\frac{1}{2}\biggr\}\), tal que E é não mensurável com relação a \(\mathcal{A}\times \mathcal{B},\) ou que \(A = t_{1/2}(E)\) deveria estar em \(\mathcal{A}.\) Por outro lado, \(E\subseteq [0, 1]\times \biggl\{\frac{1}{2}\biggr\}\), 
  que tem medida zero com relação a \(\mu \times \nu\). Assim, E tem medida nula e, em geral, \(\mathcal{A}\times \mathcal{B}\) não é completa. No entanto, poderíamos completar.

  Em outra linha, vejamos um exemplo que ilustra a necessidade das hipóteses do Teorema de Fubini. 
 \begin{example}
   Existe um conjunto X munido com a ordem parcial, ``\(leq \)", tal que X é não enumerável, mas, para todo \(y\in X\), o conjunto \(\{x\in X: x \leq y\}\) é enumerável. (X pode ser um conjunto enumerável ordinal, tipo os naturais)). 
   A \(\sigma \)-álgebra é a coleção de subconjuntos A de X tal que um dos dois A ou \(A ^{\complement}\) é enumerável. Defina \(\mu \) sobre X pondo \(\mu (A) = 0\) se A é enumerável e 1 se A é não enumerável. Defina f sobre \(X\times X\) como sendo 
   \(f = \chi_{E}\), em que E é não um subconjunto de \(X\times X\) formado por pares \((x, y)\) com \(x\leq y\), contável sobre toda a reta horizontal com complemento contável sobre toda reta vertical, mas não mensurável, pois nem \(E \) e nem \(E ^{\complement}\) são enumeráveis. 
   Então, fixado y, como na reta horizontal é contável, então \(\mu = 0\), ou seja, \(\int_{}^{}\int_{}^{}f(x, y)\mu (dx)\mu (dy) = 0, \) enquanto fixado x, todo ponto (x, y) está em E. Assim, \(f=1\) e \(\int_{}^{}\int_{}^{}f\mu (dy)\mu (dx) = 1.\)
   Aqui, f não é mensurável no \(\sigma \)-produto. Poderíamos tomar f não mensurável, pelo axioma da escolha, mas, note que em \(X^{2} = [0, 1]^{2},\) a reta horizontal é contável na região \(x\leq y\) e a reta vertical não é contável, então o complementar deveria ser contável.
 \end{example}
 \begin{example}
   Seja \(X = Y = [0, 1]\) com \(\mu, \nu\) sendo medida de Lebesgue. Sejam \(g_{i}\) funções com suporte em \(I_{i} = \biggl(\frac{1}{i+1}, \frac{1}{i}\biggr)\) tais que \(\int_{0}^{1}g_{i}(x)dx = 1, i = 1, 2, \dotsc .\)Seja 
     \[
       f(x, y) = \sum\limits_{i=1}^{\infty}[g_{i}(x)-g_{i+1}(x)]g_{i}(y).
     \]
  Para cada ponto \((x, y)\), no máximo dois termos da soma são não nulos, tornando-a finita. Note que 
    \[
      \int_{0}^{1}f(x, y)dy = \sum\limits_{i=1}^{\infty}(g_{i}(x)-g_{i+1}(x))
    \]
  é a série telescópica, sendo sua soma \(g_1(x).\) Assim, 
    \[
      \int_{0}^{1}\int_{0}^{1}f(x, y)dydx = \int_{0}^{1}g_1(x)dx = 1.
    \]
  Porém, integrando primeiro em x, chegamos em 0, tal que 
    \[
      \int_{0}^{1}\int_{0}^{1}f(x, y)dxdy = 0.
    \]
  Mas aqui, tome Q partição dada pela definição e, para simplificar, tome \(g_{i}\) tendo como gráfico um triângulo de base em \(I_{i}\) e altura \(\frac{2}{\frac{1}{i}-\frac{1}{i+1}}.\) Assim, 
 \begin{align*}
   \int_{}^{}\int_{}^{}|f(x, y)|dxdy &= \lim_{n\to \infty}\sum\limits_{i=1}^{n}|f(x_{i}, y_{i})|\mathrm{vol}(Q) \\ 
                                     &= \lim_{n\to \infty}\sum\limits_{j=1}^{n}\biggl(\sum\limits_{i=1}^{\infty}\frac{2}{\biggl(\frac{1}{i}-\frac{1}{i+1}\biggr)}\frac{2}{\biggl(\frac{1}{i}-\frac{1}{i+1}\biggr)}\biggl(\frac{1}{i}-\frac{1}{i+1}\biggr)^{2}\biggr)\\
                                     &+ \lim_{n\to \infty}\sum\limits_{j=1}^{n}\biggl(\sum\limits_{i=1}^{\infty}\frac{2}{\biggl(\frac{1}{i}-\frac{1}{i+1}\biggr)}\frac{2}{\biggl(\frac{1}{i+1}-\frac{1}{i+2}\biggr)}\biggl(\frac{1}{i}-\frac{1}{i+1}\biggr)\biggl(\frac{1}{i+1}-\frac{1}{i+2}\biggr)\biggr)\\
                                     &= \infty.
 \end{align*}
 \end{example}
\begin{example}
  Coloque \(I=[0, 1]\) com \(\lambda , m\) medidas de Lebesgue e contador (não é \(\sigma \)-finita). Seja o fechado 
    \[
      \Delta = \{(x, x): x\in I\}
    \]
  e defina \(f=\chi_{\Delta }.\) segue que 
    \[
      \int_{}^{}\int_{}^{}fdmd\lambda = \int_{}^{}m(\{x\})d\lambda = \lambda (I) = 1
    \]
  e 
    \[
      \int_{}^{}\int_{}^{}fd\lambda dm = \int_{}^{}\lambda (\{x\})dm = 0.
    \]
\end{example}
  Sejam \(X = Y\) inteiros positivos e \(\mu  = \nu\) medida contador. Escreva \(c_{ij}\) para \(f(i, j).\) Então, pelo \hyperlink{fubini_tonelli}{\textit{Teorema de Fubini}}, temos 
    \[
      \sum\limits_{i=1}^{\infty}\sum\limits_{j=1}^{\infty}c_{ij} = \sum\limits_{j=1}^{\infty}\sum\limits_{i=1}^{\infty}c_{ij},
    \]
  sempre que \(c_{ij}\geq 0\) ou que \(\sum\limits_{i}^{}\sum\limits_{j}^{}|c_{ij}| < \infty\).
\subsection{Medidas com Sinal}
  Um tipo de medidas possui aplicações no estudo de cargas da física, mas cargas podem ser negativas e positivas, enquanto que as medidas que vimos até agora eram estritamente não-negativas. Para lidar melhor com essa situação, introduzimos a ideia de medidas 
com sinal, que possuem a propriedade de aditividade enumerável, mas podendo tomas valores dos dois negativos e/ou positivos. Por exemplo,
  \[
    \nu(A) = \int_{A}f d\mu_{},\quad f \text{ integrável}
  \]
  toma valores positivos e negativos. Essencialmente, são dois resultados que usaremos para lidar com essas medidas - o \textit{Teorema de Hahn} e o \textit{Teorema de Decomposição de Jordan}. O primeiro diz que, 
  dado uma medida com sinal \(\mu \), então \(X = P\cup N\), com \(P\cap N = \emptyset \), tal que \(\mu \) e \(-\mu \) são medidas positivas em P e N, respectivamente. O outro diz que \(\mu  = \mu ^{+} - \mu ^{-},\) em que \(\mu ^{\pm}\) são ambas positivas.
 \begin{def*}
   Seja \(\mathcal{A}\) uma \(\sigma \)-álgebra. Uma \textbf{medida com sinal} é uma função \(\mu : \mathcal{A}\rightarrow (-\infty, \infty]\) tal que \(\mu (\emptyset ) = 0\) e, se \(A_{1}, A_2, \dotsc \) são dois-a-dois disjuntos com \(A_{i}\in \mathcal{A}, i = 1, 2, \dotsc \), então 
     \[
       \mu \biggl(\bigcup_{i=1}^{\infty}A_{i}\biggr) = \sum\limits_{i=1}^{\infty}\mu (A_{i}),
     \]
  em que assume-se convergência absoluta da série se \(\mu (\bigcup_{i=1}^{\infty}A_{i})\) é finita. \(\square\)
 \end{def*}
 Como é requerida a convergência absoluta, a ordem no somatório não importa - lembrando que \(\mu :\mathcal{A}\rightarrow [0, \infty]\) denota uma medida positiva.
\begin{def*}
  Seja \(\mu \) uma medida com sinal. Um conjunto \(A\in \mathcal{A}\) é chamado \textbf{positivo} se \(\mu (B) \geq 0\) para todo \(B\subseteq A\) e \(B\in \mathcal{A}.\) Analogamente, \(A\in \mathcal{A}\) é chamada \textbf{negativa} se \(\mu (B)\leq 0\) para 
todo \(B\subseteq A\) e \(B\in \mathcal{A}.\) Finalmente, um conjunto \(A\in \mathcal{A}\) é chamado \textbf{nulo}/\textbf{de medida nula} se \(\mu (B) = 0 \) para todo \(B\subseteq A\) e \(B\in \mathcal{A}.\quad \square\)
\end{def*}
  A antiga definição de conjunto nulo, para medida positiva, coincide com conjunto nulo como definido acima. Se \(\mu \) é uma medida com sinal, argumentando como no caso de medida positiva, temos 
    \[
      \mu \biggl(\bigcup_{i=1}^{\infty}A_{i}\biggr) = \lim_{n\to \infty}\mu \biggl(\bigcup_{i=1}^{n}A_{i}\biggr).
    \]
 \begin{example}
   Suponha que m é a medida de Lebesgue e 
     \[
       \mu (A) = \int_{A}f dm,
     \]
  para algum f integrável. Se f muda de sinal, então \(\mu \) é uma medida com sinal.
  Se \(P = \{x: f(x)\geq 0  \}\), então P é um conjunto positivo e, se \(N = \{x:f(x)<0\},\) então N é um conjunto negativo. A Decomposição de Hahn dará a decomposição de \(\mathbb{R}\) como união de P e N unicamente exceto no conjunto \(C = \{x: f(x) = 0\},\) que 
poderia ser incluída em N ao invés de P, ou vice-versa, ou uma parte para cada um. A decomposição de Jordan será \(\mu  = \mu ^{+} - \mu ^{-}\), em que \(\mu ^{\pm}(A) = \int_{A}f^{\pm} dm.\)
 \end{example}
 \begin{prop*}
   Seja \(\mu \) medida com sinal, tomando valores em \((-\infty, \infty].\) Seja E mensurável com \(\mu(E) < 0\). Então, existe um subconjunto mensurável \(F\subseteq E\) negativa com \(\mu (F) < 0.\)
 \end{prop*}
\begin{proof*}
  Se E é um conjunto negativo, não há nada a provar. Caso contrário, existe um subconjunto mensurável com medida positiva. Seja \(n_1\) menor inteiro positivo tal que existe \(E_1\subseteq E\) mensurável com 
  \(\mu (E_1)\geq \frac{1}{n_1}\). Construímos, assim, uma sequência \(E_2, E_3, \dotsc \) dois-a-dois disjuntos. Seja \(k\geq 2\) e suponha que \(E_1, \dotsc , E_{k-1}\) são subconjuntos dois-a-dois disjuntos de E com 
  \(\mu (E_{i}) > 0\) para \(i=1, 2, \dotsc , k-1\). Se \(F_{k}=E\setminus{(E_1\cup E_2\cup \dotsc \cup E_{k-1})}\) é um conjunto negativo, então 
    \[
      \mu (F_{k}) = \mu (E) - \sum\limits_{i=1}^{k-1}\mu (E_{i}) \leq \mu (E) < 0
    \]
  e \(F_{k}\) é o conjunto desejado F do enunciado.

  Se \(F_{k}\) não é negativa, tome \(n_{k}\) como o menor inteiro positivo tal que existe \(E_{k}\subseteq F_{k}\) mensurável com \(\mu (E_{k})\geq \frac{1}{n_{k}}\). 
Paramos a construção quando existir k tal que \(F_{k}\) é negativa com \(\mu (F_{k}) < 0.\) Se não, continuamos a construção. Coloque 
  \[
    F= \bigcap_{i=k}^{}F_{k} = E\setminus{(\cup _k E_{k})}.
  \]
  Como \(0 > \mu (E) > -\infty\) e \(\mu (E_{k})\geq 0\), então 
    \[
      \mu (E) = \mu (F) + \sum\limits_{k=1}^{\infty}\mu (E_{k}).
    \]
  Ainda mais, \(\mu (F) \leq \mu (E) < 0\), tal que a soma converge. Isso implica que 
    \[
      \mu (E_{k})\to 0,\quad n_{k}\to \infty.
    \]
  Resta provar que F é negativa. Suponha que \(G\subseteq F\) é mensurável com \(\mu (G) > 0\). Então, \(\mu (G) \geq \frac{1}{N}\) para algum N. No entanto, isto vai contra a construção feita, pois, para algum k, \(n_{k}> N\), e poderíamos ter escolhido
G ao invés de \(E_{k}\) no k-ésimo passo. Portanto, F deve ser conjunto negativo. \qedsymbol
\end{proof*}
 \hypertarget{hahn}{
\begin{theorem*}
 \begin{itemize}
   \item[1)] Seja m medida com sinal tomando valores em \((-\infty, \infty]\). Existem conjuntos disjuntos e mensuráveis E e F em \(\mathcal{A}\) com \(X = E \cup F,\) tal que E é negativa e F é positiva.
    \item[2)] Se \(E'\) e \(F'\) é outro par, então \(E\Delta E' = F\Delta F'\) é conjunto nulo com relação a \(\mu \);
      \item[3)] Se \(\mu \) não é medida positiva, então \(\mu (E) < 0\). Se \(-\mu \) não é uma medida positiva, então \(\mu (F) > 0.\)
 \end{itemize}
\end{theorem*}}
\begin{proof*}
  (1) - Note que existe pelo menos um conjunto negativo - \(\emptyset \). Seja 
    \[
      L = \inf_{}\{\mu (A): A \text{ é um conjunto negativo}\}.
    \]
  Escolha conjuntos negativos \(A_{n}\) tal que \(\mu (A_{n})\to L.\)  Seja \(E = \bigcup_{n=1}^{\infty}A_{n}.\) Seja \(B_1 = A_1\) e seja \(B_{n} = A_{n}\setminus{(B_1\cup B_2\cup \dotsc \cup B_{n-1}}\) para cada n. Como \(A_{n}\) é um conjunto negativo, \(B_{n}\) também é, já que \(B_{n}\) é parte de \(A_{n}\). 
Além disso, \(B_{n}\) são disjuntos e \(\cup _n B_{n} = \cup_{n}^{}A_{n} = E.\)

  Se \(C\subseteq E\), então 
    \[
      \mu (C) = \lim_{n\to \infty}\mu \biggl(C\cap \biggl(\bigcup_{i=1}^{n}B_{i}\biggr)\biggr) = \lim_{n\to \infty}\sum\limits_{i=1}^{n}\mu (C\cap B_{i}) \leq 0.
    \]
  Assim, E é negativa, tal que 
    \[
      \mu (E) = \mu (A_{n}) + \mu (E\setminus{A_{n}}) \leq \mu (A_{n}) + \mu (E) \leq \mu (A_{n}).
    \]
  Fazendo n tender a infinito, temos \(\mu (E) = L,\) donde segue que \(L> -\infty.\) 

  Seja \(F = E ^{\complement}.\) Se F não fosse positiva, existiria \(B\subseteq F\) com \(\mu (B) < 0\), de forma que existe um conjunto negativo \(C\subseteq B\) com \(\mu (C) < 0.\) No entanto, isto significaria que 
 \(E\cup C\) deve ser negativo com \(\mu (E\cup C) < \mu (E) = L.\) Contradição. Portanto, F é positiva. \qedsymbol

  (2) - Para provar a unicidade, se \(E', F'\) é outro par de conjuntos e \(A\subseteq E\setminus{E'}\subseteq E\), então \(\mu (A) \leq 0\). Mas, \(A\subseteq E\setminus{E'} = F\setminus{F'}\subseteq F'\), do que segue que \(\mu (A)\geq 0\) e, logo, 
  \(\mu (A)=0.\) O mesmo argumento funciona se \(A\subseteq E'\setminus{E}\) e qualquer subconjunto de \(E\Delta E'\) pode ser escrita como a união de \(A_1, A_2\), em que \(A_1\subseteq E\setminus{E'}\) e \(A_2\subseteq E'\setminus{E}.\)

  (3) - Suponha que \(\mu \) não seja positiva e que \(\mu (E) = 0\). Se \(A\in \mathcal{A},\) então 
    \[
      \mu (A) = \mu (A\cap E) + \mu (A\cap F) \geq \mu (E) + \mu (A\cap F)\geq 0,
    \]
  o que implica que \(\mu \) deve ser positiva, uma contradição. Usamos aqui que 
    \[
      \mu (E) = \mu (E\cap A) + \mu (E\cap A ^{\complement}) \leq \mu (E\cap A).
    \]
  Um argumento similar aplica-se para \(-\mu \). \qedsymbol
\end{proof*}
\begin{def*}
  Dizemos que \(\mu \) e \(\nu\) são \textbf{mutualmente singulares} se existirem dois conjuntos E e F em \(\mathcal{A}\) tais que \(X = E\cup F\) e \(E\cap F = \emptyset \), com \(\mu (E) = \nu(F) = 0.\) Notação: \(\mu \perp \nu\) denota que \(\mu \) e \(\nu\) são mutualmente singulares. \(\square\)
\end{def*}
\begin{def*}
  A medida 
    \[
      |\mu | = \mu ^{+} + \mu ^{-}
    \]
  é chamada \textbf{medida variação total} de \(\mu \), e \(|\mu |(X)\) é chamada \textbf{variação total de }\(\mu \). \(\square\)
\end{def*}
\begin{example}
  Se \(\mu \) é medida de Lebesgue restrita a \(\biggl[0, \frac{1}{2}\biggr],\) tal que \(\mu (A) = m \biggl(A \cap \biggl[0, \frac{1}{2}\biggr]\biggr) \) e \(\nu\) é medida de Lebesgue restrita a \(\biggl[\frac{1}{2}, 1\biggr].\) Então, \(\mu \) e \(\nu\) são mutualmente singulares. Aqui, basta tomar \(E = \biggl(\frac{1}{2}, 1\biggr]\) e \(F = \biggl[0, \frac{1}{2}\biggr]\), o que funciona pois 
  \(\biggl\{\frac{1}{2}\biggr\}\) tem medida de Lebesgue zero.
\end{example}
\begin{example}
  Seja f a função de Cantor-Lebesgue, em que definimos 
    \[
      f(x) = \left\{\begin{array}{ll}
          1,\quad x\geq 1 \\ 
          0,\quad x < 0
        \end{array}\right.
    \]
  e seja \(\nu\) a medida de Lebesgue-Stieltjes associada a f. Seja m a medida de Lebesgue. Então, \(m\perp \nu\).

  De fato, seja \(E = C\), em que C é o conjunto de Cantor, e \(F= C ^{\complement}.\) Já sabemos que \(m(E) = 0\), necessitamos provar que 
  \(\nu(F) = 0.\) Isto equivale a provar que \(\nu(I) = 0\) para todo intervalo aberto contido em F. Note que isso segue se verificarmos para intervalos abertos \(J = (a, b]\) contidos em F, pois, como 
F é constante sobre tais intervalos, \(f(b) = f(a)\), e portanto, 
  \[
    \nu(J) = f(b)-f(a) = 0.
  \]
\end{example}
\newpage

\newpage
\section{Aula 08 - 22/01/2024}
\subsection{Motivações}
\begin{itemize}
  \item Decomposição de Jordan;
  \item O Teorema de Radon-Nikodym.
\end{itemize}
\subsection{Decomposição de Jordan}
 \hypertarget{jordan_decomposition}{
   \begin{theorem*}[Decomposição de Jordan]
  Se \(\mu \) é uma medida com sinal num espaço mensurável \((X, \mathcal{A})\), então existem medidas positivas \(\mu ^{\pm}\) tais que \(\mu = \mu ^{+} - \mu ^{-}\) e \(\mu ^{+}\perp \mu ^{-}.\) Esta decomposição é única.
\end{theorem*}}
\begin{proof*}
  Sejam E e F conjuntos negativo e positivo, respectivamente, com relação a \(\mu \), tal que \(X = E\cup F\) e \(E\cap F = \emptyset \). Seja \(\mu ^{+}(A) = \mu (A\cap F)\) e \(\mu ^{-}(A)=-\mu (A\cap E).\) Isso 
  fornece a decomposição desejada. Se \(\mu = \nu^{+}-\nu^{-}\) é outra decomposição com \(\nu^{+}\perp \nu^{-},\) seja \(E'\) tal que \(\nu^{+}(E') = 0\) e 
  \(\nu^{-}((E')^{\complement}) = 0.\) Seja \(F'=(E')^{\complement}\), de modo que \(X = E' \cup F'\) e \(E'\cap F' = \emptyset \). Se \(A\subseteq F',\) então \(\nu^{-}(A)\leq \nu^{-}(F') = 0\) e, assim,
    \[
      \mu (A) = \nu ^{+}(A) - \nu ^{-}(A) \geq \nu ^{+}(A) \geq 0,
    \]
  do que segue que \(F'\) é positiva para \(\mu \). Analogamente, \(e'\) é negativa para \(\mu .\) 

  Com isso, \(E', F'\) dá uma outra decomposição de Hahn sobre X. Pela unicidade do \hyperlink{hahn}{\textit{Teorema da Decomposição de Hahn}}, \(F\Delta F'\) tem medida nula com relação a \(\mu \). Como \(\nu ^{+}(E') = 0\) e \(\nu^{-}(F') = 0\)
se \(A\in \mathcal{A}\), temos 
 \begin{align*}
   \nu ^{+}(A) &= \nu ^{+}(A\cap F') = \nu ^{+}(A\cap F') - \nu ^{-}(A\cap F')\\ 
               &= \mu (A\cap F') = \mu (A\cap F) = \mu ^{+}(A).
 \end{align*}
 Analogamente, \(\nu^{-}=\mu ^{-}.\) Portanto, \(\mu^{+}-\mu ^{-}= \nu^{+}-\nu ^{-}. \) \qedsymbol
\end{proof*}
   Suponha \(f\geq 0\) integrável com relação a \(\mu \). Defina \(\nu \) pondo 
     \[
       \nu (A) = \int_{A}f d\mu_{}.
     \]
    Afirmamos que \(\nu \) é uma medida. De fato, apenas precisamos chegar a aditividade contável. Sejam \(A_{n}\) conjuntos mensuráveis disjuntos. Então,
      \[
        \nu \biggl(\bigcup_{n}^{}A_{n}\biggr) = \int_{\bigcup_{n}^{}A_{n}}f d\mu_{} = \sum\limits_{i=1}^{\infty}\int_{A_{i}}f d\mu_{} = \sum\limits_{i=1}^{\infty}\nu (A_{i}).
      \]
    em que usamos que, como \(f\geq 0\), podemos trocar a integral com a soma. Note que \(\nu (A) = 0\) sempre que \(\mu (A) = 0\). 

    Agora, fica a questão da ``recíproca'' - dads duas medidas \(\mu \) e \(\nu \), existe f tal que 
      \[
        \nu (A) = \int_{A}f d\mu_{}?
      \]
    Esta pergunta é respondida pelo Teorema de Radon-Nikodym. Sendo mais específico, se \(\nu (A) = 0\) sempre que \(\mu (A) = 0\), então existe f tal que 
      \[
        \nu (A) = \int_{A}f d\mu_{}.
      \]
    Medidas que possuem essa propriedade recebem uma definição própria.
\begin{def*}
  Uma medida \(\nu \) é dita ser \textbf{absolutamente contínua} com relação à medida \(\mu \) se 
\(\nu (A) = 0\) sempre que \(\mu (A) = 0\). Denotamos isso por \(\nu <<\mu .\) \(\square\)
\end{def*}
\begin{prop*}
  Seja \(\nu \) medida finita. Então, \(\nu \) é absolutamente contínua com relação a \(\mu \) se, e somente se, para todo \(\varepsilon \) existe \(\delta \) tal que \(\mu (A) < \delta \) implica que \(\nu (A) < \varepsilon .\)
\end{prop*} 
\begin{proof*}
  Suponha que, para todo \(\varepsilon > 0\), existe \(\delta > 0\) tal que 
    \[
      \mu (A) < \delta \Rightarrow \nu (A) < \varepsilon .
    \]
  Se \(\mu (A) = 0\), então \(\nu (A) = 0\) para todo \(\varepsilon \) e, logo, \(\nu (A) = 0\), donde segue que \(\nu << \mu .\)

  Por outro lado, assuma que \(\nu <<\mu .\) Se existe \(\varepsilon > 0\), então para todo \(\delta_{\varepsilon } > 0\), \(E_{\varepsilon }\) com \(\mu (E_{\varepsilon }) < \delta_{\varepsilon } \) e \(\nu (E_{\varepsilon })\geq \varepsilon \), podemos supo que existe 
 \(E_{k}\) tal que \(\mu (E_{k}) < 2^{-k},\) mas \(\nu (E_{k})\geq \varepsilon .\) Seja \(F = \bigcap_{n=1}^{\infty}\bigcup_{k=n}^{\infty}E_{k}.\) Então,
   \[
     \mu (F)=\lim_{n\to \infty}\mu \biggl(\bigcup_{k=n}^{\infty}E_{k}\biggr)\leq \lim_{n\to \infty}\sum\limits_{k=n}^{\infty}2^{-k} = 0,
   \]
  mas 
    \[
      \nu (F) = \lim_{n\to \infty}\nu \biggl(\bigcup_{k=n}^{\infty}E_{k}\biggr)\geq \varepsilon .
    \]
  Aqui, utilizamos que \(\nu \) é finito para obter a igualdade acima. Com isso, obtivemos uma contradição à definição de continuidade absoluta.\qedsymbol
\end{proof*}
\begin{lemma*}
  Sejam \(\mu \) e \(\nu \) medidas positivas finitas sobre o espaço mensurável \((X, \mathcal{A}).\) Então, \(\mu \perp \nu \), ou existe \(\varepsilon > 0\) e \(G\in \mathcal{A}\) tais que \(\mu (G) > 0\) e G é positiva para \(\nu - \varepsilon \mu .\)
\end{lemma*}
\begin{proof*}
  Considere a decomposição de Hahn para \(\nu -\frac{1}{n}\mu .\) Existem, então, uma sequência de conjuntos negativos \(E_{n}\) e positivos \(F_{n}\) para essa medida, sendo \(E_{n}\) e \(F_{n}\) disjuntos, com \(E_{n}\cup F_{n} = X\) para todo n. 
Coloque \(F = \cup _n F_{n}\) e \(E = \cap_{n} E_{n}\) e note que 
  \[
    E ^{\complement} = \bigcup_{n}^{}E_{n}^{\complement} = \bigcup_{n}^{}F_{n} = F.
  \]
Para cada n, \(E\subseteq E_{n}\), tal que 
  \[
    \nu (E) \leq \nu (E_{n}) \leq \frac{1}{n}\mu (E_{n})\leq \frac{1}{n}\mu (X).
  \]
Como \(\nu \) é uma medida positiva, \(\nu (E) = 0.\) 

Uma das possibilidades é que \(\mu (E ^{\complement}) = 0\) no caso de \(\mu \perp \nu \). A outra, é que \(\mu (E ^{\complement})>0,\) no qual \(\mu (F_{n}) > 0\) para algum n. Seja \(\varepsilon  = \frac{1}{n}\) e \(G = F_{n}.\) Portanto, 
pela definição de \(F_{n}\), G é positiva para \(\nu - \varepsilon \mu .\) \qedsymbol
\end{proof*}
\subsection{O Teorema de Radon-Nikodym}
 \hypertarget{radon_nikodym}{
   \begin{theorem*}[Radon-Nikodym]
    Suponha que \(\mu \) é uma medida \(\sigma \)-finita e positiva sobre o espaço mensurável \((X, \mathcal{A})\), e \(\nu \) é uma medida finita sobre o espaço mensurável \((X, \mathcal{A},)\) tal que \(\nu \) é absolutamente
    contínua com relação a \(\mu \). Então, existe uma função \(\mu \)-integrável não negativa f mensurável com relação a \(\mathcal{A}\) tal que 
      \[
        \nu (A) = \int_{A}f d\mu_{},\quad \forall A\in \mathcal{A}.
      \]
    Além disso, se g for outra função satisfazendo a mesma coisa, então \(f = g\) quase sempre com relação a \(\mu \).
  \end{theorem*}}
  A função f é chamada \textbf{derivada de Radon-Nikodym} de \(\nu \) com relação a \(\mu \), ou também chamada de \textbf{densidade de }\(\nu \) \textbf{com relação a }\(\mu \), e escrevemos 
    \[
      f= \frac{d\nu }{d\mu },
    \]
  ou 
    \[
      d\nu = f\mu.
    \]
 \begin{proof*}
   A ideia da prova é olhar para o conjunto de f tal que 
     \[
       \int_{A}f d\mu_{} \leq \nu(A),\quad \forall A\in \mathcal{A}.
     \]
    A partir disso, escolhe-se um dos conjuntos, X, tal que \(\int_{X}f d\mu_{}\) seja o maior deles. 

  \textit{\underline{Unicidade}:} Suponha que f e g são duas funções tais que 
    \[
      \int_{A}f d\mu_{} = \nu (A) = \int_{A}g d\nu_{},\quad \forall A\in \mathcal{A}.
    \]
  Para todo \(A\in \mathcal{A},\) temos 
    \[
      \int_{A}(f-g) d\mu_{} = \nu (A) - \nu(A) = 0.
    \]
  Logo, \(f = g = 0\) quase sempre com relação a \(\mu \), provando a unicidade.

  Agora, suponha que \(\mu \) seja uma medida finita. Vamos definir a f, já que sabemos que ela é única. Coloque 
    \[
      \mathcal{F} = \{g \text{ mensurável: } g\geq 0, \int_{A}g d\mu_{}\leq \nu (A), \forall A\in \mathcal{A}.\}
    \]
  Assim, \(\mathcal{F}\neq\emptyset\) pois \(0\in \mathcal{F}.\) Seja 
    \[
      L = \sup_{}\biggl\{\int_{}g d\mu_{}: g\in \mathcal{F}\biggr\},
    \]
  tal que \(L\leq \nu (X) < \infty\), e seja \(g_{n}\) sequência em \(\mathcal{F}\), tal que 
    \[
      \int_{}g_{n} d\mu_{}\to L.
    \]
    Tome \(h_{n} = \max_{}(g_1, \dotsc , g_{n})\) e \(B = \{x: g_1(x) \geq g_2(x)\}\). Escrevemos 
   \begin{align*}
     \int_{A}h_2 d\mu_{} &= \int_{A\cap B}h_2 d\mu_{} + \int_{A\cap B ^{\complement}}h_2 d\mu_{}\\ 
                         &= \int_{A\cap }g_1 d\mu_{} + \int_{A\cap B ^{\complement}}g_2 d\mu_{}\\ 
                         &\leq \nu (A\cap B) + \nu (A\cap B ^{\complement}) = \nu (A).
   \end{align*}
  Portanto, \(h_2=\max_{}(g_1, g_2)\in \mathcal{F}\). Repetindo o processo e aplicando indução, \(h_{n}\in \mathcal{F}.\) A função \(h_{n}\) cresce, digamos para f. Pelo \hyperlink{monotone_convergence}{\textit{Teorema da Convergência Monótona}},
    \[
      \int_{A}f d\mu_{}\leq \nu (A), \quad \forall A\in \mathcal{A}
    \]
  e 
    \[
      \int_{}f d\mu_{} \geq \int_{}h_{n} d\mu_{}\geq \int_{}g_{n} d\mu_{},\quad \forall n\in \mathbb{N}.
    \]
  Com isto, 
    \[
      \int_{}f d\mu_{} = L,
    \]
  pois L é o supremo por hipótese. 

  A seguir, mostramos que essa é a f que procuramos. Com efeito, defina a medida \(\lambda \) pondo 
    \[
      \lambda (A) = \nu (A) - \int_{A}f d\mu_{}.
    \]
  Note que \(\lambda \) é medida positiva, pois \(f\in \mathcal{F}.\) Suponha que \(\lambda \) não é mutualmente singular com \(\mu \). Então, existe \(\varepsilon >0\) e G tais que G é mensurável, \(\mu (G) > 0\)
e G é positiva para a medida \(\lambda -\varepsilon \mu .\) Para todo \(A\in \mathcal{A},\)
  \[
    \nu (A) - \int_{A}f d\mu_{} = \lambda (A) \geq \lambda (A\cap G) \geq \varepsilon \mu (A\cap G) = \int_{A}\varepsilon \chi_{G} d\mu_{},
  \]
ou seja, 
  \[
    \nu (A) \geq \int_{A}(f+\varepsilon \chi_{G}) d\mu_{}.
  \]
  Logo, \(f+\varepsilon \chi_{G}\in \mathcal{F}.\) No entanto, 
    \[
      \int_{X}(f+\varepsilon \chi_{G}) d\mu_{} = L + \varepsilon \mu (G) > L.
    \]
  Contradição, pois L é o supremo. Destarte, \(\lambda \perp \mu ,\) tal que existe \(h\in \mathcal{A}\) tal que \(\mu (H) = 0\) e \(\lambda (H ^{\complement}) = 0\). Como \(\nu<<\mu \), então \(\nu (H) = 0\) e, portanto, 
    \[
      \lambda (H) = \nu (H) - \int_{H}f d\mu_{} = 0.
    \]
  Isto implica que \(\lambda  = 0\) ou que \(\nu (A) = \int_{A}f d\mu_{}\) para todo A.

  Finalmente, mostramos que \(\mu \) é \(\sigma \)-finita. De fato, existe \(F_{i}\uparrow X\) tal que \(\mu (F_{i}) < \infty\) para todo i. Seja \(\mu_{i}\) a restrição de \(\mu \) em \(F_{i}\), ou seja, \(\mu_{i}(A) = \mu(A\cap F_{i})\).
Coloque \(\nu_{i}\) analogamente para \(\nu \). Segue que \(\nu_{i}\) é absolutamente contínua com relação a \(\nu_{i}\), pois se \(\mu(A\cap F_{i}) = 0\), então \(\nu (A\cap F_{i})=0,\) o que equivale a \(\mu_{i}(A) = \nu_{i}(A) = 0.\) 

  Se \(f_{i}\) é a função tal que \(d\nu_{i} = f_{i}d\mu_{i}\), então \(f_{i} = f_{j}\) para sobre \(F_{i}\) para todo \(j\leq i\) por unicidade. Defina f pondo \(f(x) = f_{i}(x)\) se \(x\in F_{i}.\) Com esta definição, para cada \(A\in \mathcal{A},\) 
    \[
      \nu (A\cap F_{i}) = \nu_{i}(A) = \int_{A}f_{i} d\mu_{i} = \int_{A\cap F_{i}} d\mu_{}.
    \]
  Portanto, fazendo \(i\to \infty\), encontrarmos a f procurada. \qedsymbol
 \end{proof*}
 A prova da decomposição de Lebesgue usa esses mesmos argumentos: 
\begin{theorem*}
  Suponha \(\mu \) uma medida \(\sigma \)-finita positiva e \(\nu \) uma medida finita positiva. Então, existem medidas positivas \(\lambda , \rho \) tais que \(\nu = \lambda + \rho \), com \(\rho \) absolutamente contínua com relação a \(\mu \), e \(\lambda \) e \(\mu \) mutualmente singulares.
\end{theorem*}
\begin{proof*}
  Vamos reduzir \(\mu \) ao caso de medida finita e definir \(\mathcal{F}\) e L, assim como a construção de f, como na prova do Radon-Nikodym.

  Seja \(\rho (A) = \int_{A}f d\mu_{}\) e seja \(\lambda = \nu - \rho\). Da construção, como na prova do Radon-Nikodym, 
    \[
      \int_{A}f d\mu_{}\leq \nu (A).
    \]
  Assim, \(\lambda (A) \geq 0\) para todo A. Temos, então, \(\rho + \lambda = \nu \). Resta provar a singularidade mútua. 

  Se não fosse, então existe \(\varepsilon > 0\) e \(F\in \mathcal{A}\) tal que \(\mu (F) > 0\) e F é positiva para a medida \(\lambda -\varepsilon \mu .\) Obtemos uma contradição na prova do Radon-Nikodym. Portanto, \(\lambda \perp \mu \) e a decomposição é única. \qedsymbol
\end{proof*}
 Seja I um intervalo de \(\mathbb{R}.\) Uma função \(f:I\rightarrow \mathbb{R}\) é absolutamente contínua em I se, para cada \(\varepsilon  > 0\), existe um \(\delta > 0\) de modo que, sempre que uma sequência finita de subintervalos disjuntos aos pares \((x_{k}, y_{k})\in I\) com \(x_{k}, y_{k}\in I\) satisfazendo 
   \[
     \sum\limits_{k}^{}(y_{k}-x_{k}) < \delta \Rightarrow \sum\limits_{k}^{}|f(y_{k}) - f(x_{k})| < \varepsilon .
   \]
   A coleção de todas as funções absolutamente contínuas em I é denotado \(\mathbb{AC}(I).\) Esta definição é equivalente a 
  \begin{itemize}
    \item[i)] f é absolutamente contínua; 
    \item[ii)] f tem uma derivada f'quase em todos os lugares, a derivada é Lebesgue integrável e \(f(x) = f(a) + \int_{a}^{x}f'(t)dt\) para todo \(x\in [a, b]\)
    \item[iii)] Existe uma função integrável Lebesgue g em [a, b] tal que \(f(x) = f(a) + \int_{a}^{x}g'(t)dt\) quase para todo x em [a, b].
  \end{itemize}
 \begin{lemma*}
   Denote m a medida de Lebesgue em \(\mathbb{R}^{n}.\) Suponha que \(E\subseteq \mathbb{R}^{n}\) seja coberta por uma coleção de bolas \(\{B_{\alpha }\} \) e existe \(R> 0\) tal que o diâmetro de cada bola \(\{B_{\alpha }\}\) é limitada por R. 
  Então, existe uma sequência \(B_{1}, B_{2}, \dotsc \) de elementos de \(\{B_{\alpha }\}\), disjuntas, tal que 
    \[
      m(E) \leq 3^{n}\sum\limits_{k}^{}m(B_{k}).
    \]
 \end{lemma*}
 \begin{proof*}
   Seja \(d(B_{\alpha })\) diâmetro de \(B_{\alpha }.\) Escolha \(B_1\) tal que 
     \[
       d(B_{1}) \geq \frac{1}{2}\sup_{\alpha }d(B_{\alpha }).
     \]
    Uma vez escolhidos, \(B_1, B_2, \dotsc , B_{k},\) escolhemos \(B_{k+1}\) disjunto de todos os outros e tal que 
      \[
        d(B_{k+1}) \geq \frac{1}{2}\sup_{\alpha }\{d(B_{\alpha }): B_{\alpha } \text{ é disjunto de }B_1,\dotsc ,B_{k}\}.
      \]
    Se \(\sum\limits_{k}^{}m(B_{k}) = \infty,\) então acabou. Suponha que \(\sum\limits_{k}^{}m(B_{k}) < \infty\). Seja \(B_{k}^{*}\) uma bola com o mesmo centro de \(B_{k},\) mas com o triplo do raio. Afirmamos que 
  \(E\subseteq \bigcup_{k}^{}B_{k}^{*}\). Assumindo a afirmação, temos 
    \[
      m(E) \leq m \biggl(\bigcup_{k}^{}B_{k}^{*}\biggr) \leq \sum\limits_{k}^{}m(B_{k}^{*}) = 3^{n}\sum\limits_{k}^{}m(B_{k}).
    \]
    Agora, provemos a afirmação. Basta provar que \(B_{\alpha }\subseteq \bigcup_{k}^{}B_{k}^{*}, \) já que \(\{B_{\alpha }\}\) cobrem E. Fixado \(\alpha \), se \(B_{\alpha }\) é um dos \(B_{k},\) estamos feitos. Caso contrário, 
  se \(\sum\limits_{k}^{}m(B_{k}) < \infty,\) então \(d(B_{k})\to 0\). Seja k o menor inteiro tal que \(d(B_{k+1}) < \frac{1}{2}d(B_{\alpha ).}\) Então, \(B_{\alpha }\) deve interceptar um dos \(B_{i}, i = 1, 2, \dotsc, k\). Caso contrário, poderíamos escolher \(B_{k+1}\) como sendo \(B_{\alpha }.\)

    Seja \(j_{0}\) o menor inteiro positivo menor ou igual a k tal que \(B_{j_{0}}\cap B_{\alpha }\neq\emptyset.\) Sabemos, da escolha e definição, que 
      \[
        \frac{1}{2}d(B_{\alpha }) \leq d(B_{j_{0}}),
      \]
  o que implica que \(B_{\alpha }\subseteq B_{j_{0}}^{*}\). Com efeito, seja \(x_{j_{0}}\) o centro de \(B_{j_{0}}\) e \(y\in B_{\alpha }\cap B_{j_{0}}.\) Se \(x\in B_{\alpha},\) então 
    \[
      |x-x_{0}|\leq |x-y| + |y-x_{0}| < d(B_{\alpha }) + d(B_{j_{0}})/2 \leq \frac{5}{2}d(B_{j_{0}}),
    \]
  ou \(x\in B_{j_{0}}^{*},\) em que usamos que \(d(B_\alpha ) \leq 2d(B_{j_{0}}).\) Portanto, \(B_{\alpha }\subseteq B_{j_{0}}^{*}.\) \qedsymbol
 \end{proof*}
 \begin{def*}
  \begin{itemize}
    \item[1)] f é \textbf{localmente integrável} se \(\int_{K}|f(x)| dx < \infty\) para todo K compacto.
      \item[2)] Se f é localmente integrável, definimos a \textbf{função maximal de f} como
        \[
          Mf(x) = \sup_{r > 0}\frac{1}{m(B(x, r))}\int_{B(x, r)}|f(y)| dy.
        \]
  \end{itemize}
 \end{def*}
  Note que, sem o supremo, isto seria apenas a média do módulo de f sobre a bola \(B(x, r).\) A estratégia para provar que Mf é função mensurável é a seguinte: 

  Se f é localmente integrável, então, para cada r, a função 
    \[
      x\mapsto \int_{B(x, r)}|f(y)| dy
    \]
  é contínua pelo \hyperlink{dominated_convergence}{\textit{Teorema da Convergência Dominada}}, pois quando \(x_{n}\) tende a \(x_{0}\), 
    \[
      h(x_{n}) = \int_{R^{n}}\chi_{B(x_{n}, r)}|f(y)| dy.
    \]
  Decorre, então, que para cada r e cada A, o conjunto 
    \[
      \biggl\{x: \int_{B(x, r)}|f(y)| dy  > A\biggr\}
    \]
  é aberto. Como \(Mf(x) > a\), para algum r, acontecerá 
    \[
      \int_{B(x, r)}|f(y)| dy > am(B(x, r)),
    \]
  tal que 
    \[
      \{x: Mf(x) > a\} = \bigcup_{r>0}^{}\biggl\{x: \int_{B(x, r)}|f(y)| dy > am(B(x, r))\biggr\}
    \]
  e, assim, \(\{x: Mf(x) > a\}\) é a união de conjuntos abertos, logo um aberto. Portanto, Mf é mensurável.

  Apesar disso, a M não leva funções integráveis em funções integráveis. Mas, Hardy e Littlewood provaram um resultado próximo: 
 \begin{theorem*}
   Se f é integrável, então, para todo \(\beta > 0,\) 
     \[
       m(\{x: Mf(x) > \beta \})\leq \frac{3^{n}}{\beta }\int_{}^{}|f(x)|dx.
     \]
 \end{theorem*}
\begin{proof*}
  Fixado \(\beta \), seja \(E_{\beta } = \{x: Mf(x) > \beta \}.\) Se \(x\in E_{\beta },\) então existe bola \(B_{x}\) centrada em x tal que 
    \[
      \int_{B_{x}}|f(y)| dy > \beta m (B_{x}).
    \]
  Então, 
    \[
      m(B_{x}) \leq \frac{\int_{}^{}|f|}{\beta }
    \]
  e, assim, \(\{B_{\alpha }\}\) é cobertura de \(E_{\beta }\) pelas bolas cujos diâmetros são limitados por um número independente de x. Extraindo uma sequência disjunta \(B_{1}, B_2, \dotsc \) tal que \(m(E_{\beta }) \leq 3^{n}\sum\limits_{k}^{}m(B_{k}),\) teremos 
 \begin{align*}
   m(E_{\beta }) &\leq 3^{n}\sum\limits_{k}^{}m(B_{k}) \leq \frac{3^{n}}{\beta }\sum\limits_{k}^{}\int_{B_{k}}|f|\\ 
                 &= \frac{3^{n}}{\beta }\int_{\bigcup_{k}^{}B_{k}}|f| \leq \frac{3^{n}}{\beta }\int_{}^{}|f|,
 \end{align*}
  como desejado. \qedsymbol
\end{proof*}
 \begin{example}
  Outro detalhe é que M \textbf{NÃO} leva funções integráveis em funções integráveis. Para ver isso, seja \(f = \chi_{B},\) em que B é a bola unitária. Então, \(Mf(x)\sim C|x|^{-n},\) x grande. Logo, Mf não é integrável, pois 
    \[
      Mf = \sup_{r}\frac{1}{w_{n}r^{n}}|B_1| = \frac{1}{r^{n}}.
    \]
 \end{example}
 \begin{theorem*}
   Seja 
     \[
       f_{r}(x) = \frac{1}{m(B(x, r))}\int_{B(x, r)}f(y)dy.
     \]
  Se f é localmente integrável, então 
    \[
      f_{r}(x)\to f(x)\quad \text{q. s. quando }r\to 0.
    \]
 \end{theorem*}
\begin{proof*}
  Basta provar que, para cada N, \(f_{r}(x)\to f(x)\) quase sempre para \(x\in B(0, N)\) quando \(r\to 0.\) Fixado N, suponha, sem perda de generalidade, que \(f = 0\) em \(B(0, 2N)^{\complement}\).
Assim, podemos supor que f seja integrável, já que ela pode ser aproximada por funções simples ou de suporte compacto. 

  Fixado \(\beta > 0\), seja \(\varepsilon > 0\). Vimos que podemos tomar g de suporte compacto tal que 
    \[
      \int_{}|f-g| dm < \varepsilon .
    \]
  Se \(g_r\), definida igual a \(f_r\) como uma medida, então 
 \begin{align*}
   |g_r(x) - g(x)| &= \biggl\vert \frac{1}{m(B(x, r))} \int_{B(x, r)}^{}|g(y) - g(x)||dy|\biggr\vert\\ 
                   &\leq \frac{1}{m(B(x, r))}\int_{B(x, r)}^{}|g(y)-g(x)|dy\to 0.
 \end{align*}
 quando \(r\to 0\) pela continuidade de g. Temos, então, 
\begin{align*}
  \limsup_{r\to 0}|f_r(x) - f(x)| &\leq \limsup_{r\to 0}|f_r(x) - f(x)|\\ 
                                  &+ \limsup_{r\to 0}|g_r(x) - g(x)| + |g(x) - f(x)|.
\end{align*}
O segundo termo do lado direito tende a zero pela estimativa feita, donde segue que podemos escrever 
\begin{align*}
  m(\{x:\limsup_{r\to 0}|f_r(x) - f(x)| > \beta \}) &\leq m(\{x:\limsup_{r\to 0}|f_r(x) - g_r(x)|>\beta /2\}) + m(\{x: |f(x)-g(x)| > \beta/2\})\\ 
                                                    &\leq m(\{x: |M(f-g)(x)|>\beta /2\}) + \frac{\int_{}^{}|f-g|}{\beta/2}\\
                                                    &\leq \frac{2(3^{n}+1)}{\beta }\int_{}^{}|f-g|\\ 
                                                    &< \frac{2(3^{n}+1)\varepsilon }{\beta },
\end{align*}
em que utilizamos a definição de função maximal para estimar 
  \[
    |f_r(x) - g_r(x)| \leq M(f-g)(x),\quad \forall r,
  \]
o que é verdadeiro para todo \(\varepsilon \). Assim,
  \[
    m(\{x:\limsup_{r\to 0}|f_r(x) - f(x)| > \beta \}) = 0.
  \]
Aplicando com \(\beta  = 1/j\) para cada inteiro positivo j, concluímos que 
\begin{align*}
  m(\{x:\limsup_{r\to 0}|f_r(x)-f(x)| > 0\}) \leq \sum\limits_{j=1}^{\infty}m(\{x:\limsup_{r\to 0}|f_r(x) - f(x)| > 1/j\}) = 0.
\end{align*}
  Portanto,
    \[
      f_r(x)\to f(x),\quad \text{q.s. quando }r\to 0.\quad \text{\qedsymbol}
    \]
\end{proof*}
  Temos, ainda, um resultado um pouco mais forte
 \begin{theorem*}
   Para quase todo x, 
     \[
       \frac{1}{m(B(x, r))}\int_{B(x, r)}^{}|f(y) - f(x)|dy\to 0
     \]
  quando \(r\to 0\).
 \end{theorem*}
\begin{proof*}
  Para cada racional c, existe um conjunto \(N_{c}\) de medida 0 tal que 
    \[
      \frac{1}{m(B(x, r))}\int_{B(x, r)}^{}|f(y)-c|dy\to |f(x)-c|,
    \]
para \(x\not\in N_{c}.\) Seja \(N = \bigcup_{c\in \mathbb{Q}}^{}N_{c}\) e suponha que \(x\not\in N\). Tome \(\varepsilon > 0\) e escolha c racional tal que \(|f(x) - c| < \varepsilon .\)
Com isso, 
 \begin{align*}
   \frac{1}{m(B(x, r))}\int_{B(x, r)}^{}|f(y) - f(x)|dy &\leq \frac{1}{m(B(x, r))}\int_{B(x, r)}^{}|f(y)-c|dy + \frac{1}{m(B(x, r))}\int_{B(x, r)}^{}|f(x)-c|dy\\ 
                                                        &= \frac{1}{m(B(x, r))}\int_{B(x, r)}^{}|f(y)-c|dy + |f(x) - c|
 \end{align*}
e, portanto, 
  \[
    \limsup_{r\to 0}\frac{1}{m(B(x, r))}\int_{B(x, r)}^{}|f(y) - f(x)|dy \leq 2|f(x)-c|< 2\varepsilon.
  \]
donde segue o resultado pela arbitrariedade do \(\varepsilon \). \qedsymbol
\end{proof*}
 Se \(f = \chi_{E},\) pelo resultado acima, para todo x em E, 
   \[
     \frac{1}{m(B(x, r))}\int_{B(x, r)}^{}\chi_{E}dy = \frac{m(E\cap B(x, r))}{m(B(x, r))}\to 1
   \]
quando r tende a 0. Analogamente, para quase todo \(x\not\in E,\) o raio tende a 0. Os pontos em que o raio tende a 1 são chamados \textbf{pontos de densidade de E.}
\newpage

\section{Aula 10 - 23/01/2024}
\subsection{Motivações}
\begin{itemize}
  \item Relacionando ao Cálculo - Diferenciação;
  \item Função de Variação Limitada.
\end{itemize}
\subsection{Diferenciação} 
  Trabalharemos em \(\mathbb{R},\) com m sendo a medida Lebesgue e \(B(x, r) = (x-r, x+r).\)
 \begin{def*}
   Define-se a \textbf{antiderivada} ou \textbf{integral indefinida} de uma função integrável por 
     \[
       F(x) = \int_{a}^{x}f(t)dt.\quad \square
     \]
 \end{def*}
 Relembremos, também, o \textit{Teorema Fundamental do Cálculo} 
 \hypertarget{calculus}{\begin{theorem*}[Teorema Fundamental do Cálculo]
    Seja \(f:\mathbb{R}\rightarrow \mathbb{R}\) integrável e \(a\in \mathbb{R}.\) Definimos 
      \[
        F(x) = \int_{a}^{x}f(t)dt.
      \]
   Então, F é derivável quase sempre e \(F'(x) = f(x)\) quase sempre.
  \end{theorem*}}
 \begin{proof*}
   Se \(h > 0,\) temos 
     \[
       F(x+h) - F(x) = \int_{x}^{x+h}f(t)dt,
     \]
  tal que 
 \begin{align*}
   \biggl\vert \frac{F(x+h) - F(x)}{h} - f(x) \biggr\vert &= \frac{1}{h}\biggl\vert \int_{x}^{x+h}[f(y) - f(x)]dy \biggr\vert\\ 
                                                          &\leq \frac{2}{m(B(x, h))}\int_{x-h}^{x+h}|f(y) - f(x)|dy.
 \end{align*}
  Sabemos que o lado direito vai a 0 quando h tende a 0 para quase todo x e, assim, a derivada de F à direita existe e é igual a f em quase toda parte. Analogamente, o lado esquerdo também vai a 0 quando h vai 0, ou seja, temos 
  \(F'(x) = f(x)\) quase sempre. \qedsymbol
 \end{proof*}
  Mostraremos que funções crescentes são deriváveis quase sempre. 
 \begin{lemma*}
   Suponha que \(H:\mathbb{R}\rightarrow \mathbb{R}\) é crescente, contínua à direita e constante para \(x\geq 1\) e \(x\leq 0.\) Seja \(\lambda \) a medida de Lebesgue-Stieltjes definida usando H e suponha que \(\lambda \) e m são mutualmente singulares. Então,
     \[
       \lim_{r\to 0}\frac{\lambda (B(x, r))}{m(B(x, r))} = 0
     \]
    quase sempre, x com relação a m.
 \end{lemma*}
 \begin{proof*}
   O primeiro passo é a mensurabilidade. Seja \(\tilde{H}(y) = \lim_{z\to y^{-}}H(z).\) Então, 
     \[
       \tilde{H}(y) = \lim_{z\to y^{-}}\lambda ([0, z])=\lambda ([0, y)).
     \]
    Como H e \(\tilde{H}\) são crescentes, elas são Borel mensuráveis. Note que 
      \[
        \lambda (B(x, r)) = \lambda ([0, x + r)) - \lambda ([0, x - r]) = \tilde{H}(x+r) - H(x-r).
      \]
   Assim, para cada r, a função \(x\mapsto \lambda (B(x, r))\) é Borel mensurável.

   Seja \(r_{j} = 2^{-j}.\) Se \(2^{-j-1}\leq r < 2^{-j}, \) então 
     \[
       \frac{\lambda (B(x, r))}{m(B(x, r))} \leq \frac{\lambda (B(x, r_{j}))}{m(b(x, r_{j+1}))} = 2 \frac{\lambda (B(x, r_{j}))}{m(B(x, r_{j}))}.
     \]
    Para provar o resultado, agora, basta provar que 
      \[
        \lim_{j\to \infty}\frac{\lambda (B(x, r_{j}))}{m(B(x, r_{j}))} = 0,\quad \text{quase sempre para todo x.}
      \]

    \textbf{\underline{Afirmação}:} O resultado é claro se x é menor que 0 ou maior que 1. 
    Com efeito, como \(\lambda \perp m\), existem conjuntos E e F mensuráveis tais que \(\lambda (F) = 0, m(E) = 0\) e \(F = E ^{\complement}.\) Seja \(\varepsilon > 0\). Agora, existe G aberto limitado tal que \(F\cap [0, 1]\subseteq G\) e \(\lambda (G) < \varepsilon .\)  
    Consequentemente, existe G' aberto contendo F tal que \(\lambda (G') < \varepsilon .\) Como H é constante em \((-\infty, 0]\cup [1, \infty),\) tomamos G sendo \(G = G'\cap (-1, 2)\).

    Seja \(\beta  > 0\) e 
      \[
      A_{\beta } = \biggl\{x\in F\cap [0, 1]: \limsup_{j\to \infty}\frac{\lambda (B(x, r_{j}))}{m(B(x, r_{j}))}\biggr\}.
      \]
    \textbf{\underline{Afirmação}:} A medida de \(A_{\beta }\) é nula. De fato, se \(x\in A_{\beta },\) então \(x\in F\subseteq G\) e existe uma bola aberta \(B_{x}\), centrada em x e de raio \(2^{-j}\) para algum j, tal que \(B_{x}\subseteq G\) e \(\frac{\lambda (B_x)}{m(B_x)}>\beta .\)
Com isso, existe uma sequência disjunta \(B_1, B_2, \dotsc \) tal que 
  \[
    m(A_{\beta }) \leq 3\sum\limits_{i=1}^{\infty}m(B_{i}).
  \]
  Então, 
    \[
      m(A_{\beta }) \leq 3\sum\limits_{i=1}^{\infty}m(B_{i}) \leq \frac{3}{\beta }\sum\limits_{i=1}^{\infty}\lambda (B_{i}) \leq \frac{3}{\beta }\lambda (G) \leq \frac{3}{\beta }\varepsilon .
    \]
  Como \(\varepsilon \) é arbitrário, a construção de G não depende de \(\beta \), do que segue que \(m(A_{\beta }) = 0.\)

  Finalmente, como \(m(A_{1/k}) = 0\) para todo k, então 
    \[
      m \biggl(\biggl\{x\in F\cap [0, 1]: \limsup_{j\to \infty}\frac{\lambda (B(x, r_{j}))}{m(B(x, r_{j}))} > 0\biggr\}\biggr) = 0.
    \]
    Portanto, como \(m(E) = 0\), segue o resultado. \qedsymbol
 \end{proof*}
\begin{prop*}
  Seja \(F:\mathbb{R}\rightarrow \mathbb{R}\) crescente e contínua à direita. Então, F' existe quase sempre. Além disso, F' é localmente integrável e, para todo \(a< b\), 
    \[
      \int_{a}^{b}F'(x)dx \leq F(b) - F(a).
    \]
\end{prop*}
\begin{proof*}
  Provaremos que F é diferenciável quase sempre em [0, 1], o que pode ser generalizado pelo mesmo argumento para \([-N, N]\) para cada N, ou seja, será válido para \(\mathbb{R}.\)

  Redefina F pondo 
    \[
      F(x) = \left\{\begin{array}{ll}
          \lim_{y\to 0^{+}}F(y),\quad x \leq 0\\ 
          F(1),\quad x > 1
        \end{array}\right..
    \]
  Então, F ainda é contínua à direita e crescente, não afetando a diferenciabilidade em [0, 1] exceto possivelmente no extremos. 

  Seja \(\nu \) a medida de Lebesgue-Stieltjes definida em termos de F. Da decomposição de Lebesgue, podemos escrever \(\nu = \lambda +\rho \), em que \(\lambda \perp m\) e \(\rho << m.\) Note que 
    \[
      \rho ([0, 1]) \leq \nu ([0,1]) = F(1) - F(0).
    \]
  Do \hyperlink{radon_nikodym}{\textit{Teorema de Radon-Nikodym}}, existe uma função \(f\geq 0\) integrável tal qeu \(\rho (A) = \int_{A}^{}fdm\) para cada A mensurável. Seja 
    \[
      H(x) = \lambda ((0, x]) = \nu ((0, x]) - \rho ((0, x]) = F(x) - F(0) - \int_{0}^{x}f(y)dy.
    \]
  Mas, pode-se provar que a função \(x\mapsto \int_{0}^{x}f(y)dy\) é contínua, o que torna H contínua à direita, crescente e \(\lambda \) é a medida de Lebesgue-Stieltjes definida em termos de H. Com isso, 
 \begin{align*}
   \limsup_{h\to 0^{+}}\frac{H(x+h) - H(x)}{h} &\leq \limsup_{h\to 0^{+}}\frac{H(x+h) - H(x-h)}{h}\\ 
                                               &=\limsup_{h\to 0^{+}}\frac{\lambda ((x-h), (x+h)])}{h}\\ 
                                               &\leq 4\limsup_{h\to 0^{+}}\frac{\lambda (B(x, 2h))}{4h} = 0,\quad \text{q.s. para todo x.}
 \end{align*}
 O mesmo é verdadeiro para a derivada à esquerda. Assim, \(H'\) existe e é igual a 0 quase sempre para x. Vimos que a função \(x\mapsto \int_{a}^{x}f(y)dy\) é derivável quase sempre e concluímos que F é derivável quase sempre. Além disso, \(F'=f\) quase sempre. Para finalizar, se 
 \(a < b\), 
   \[
     \int_{a}^{b}F'(x)dx = \int_{a}^{b}f(x)dx = \rho ((a, b])\leq \nu ((a, b]) = F(b) - F(a).
   \]
   Portanto, a prova está completa. \qedsymbol
\end{proof*}
  Podemos finalmente enunciar o resultado principal sobre diferenciabilidade de função crescente. 
 \begin{theorem*}
   Se \(F:\mathbb{R}\rightarrow \mathbb{R}\) é crescente, então \(F'\) existe quase sempre e 
     \[
       \int_{a}^{b}F'(x)dx \leq F(b) - F(a)
     \]
  sempre que \(a < b.\)
 \end{theorem*}
\begin{proof*}
  Seja \(G(x) = \lim_{y\to x^{+}}F(y).\) Como F é crescente, existe no máximo um número enumerável de pontos x em que F não é contínua e, assim, F(x) = G(x) quase sempre.
Como G é crescente e contínua à direita, G é derivável quase sempre. 

Mostraremos, então, que se x é um ponto em que G é derivável e \(F(x) = G(x)\), então \(F'(x)\) existe e é igual a \(G'(x).\)

Seja x um tal ponto e sejam \(L = G'(x)\), \(\varepsilon > 0.\) Sendo F e G crescentes, para todo \(h > 0\) existe ponto \(x_{h}\in (x+h, x + (1+\varepsilon )h\) no qual F e G coincidem e, assim, 
  \[
    F(x+h) \leq F(x_{h}) = G(x_{h}) \leq G(x+(1+\varepsilon )h).
  \]
Então, 
\begin{align*}
  \limsup_{h\to 0^{+}}\frac{F(x+h) - F(x)}{h} &\leq \limsup_{h\to 0^{+}}\frac{G(x+(1+\varepsilon )h)-G(x)}{h}\\ 
                                              &= (1+\varepsilon )\limsup_{h\to 0^{+}}\frac{G(x+(1+\varepsilon )h - G(x)}{(1+\varepsilon )h}\\ 
                                              &= (1+\varepsilon )L.
\end{align*}
Analogamente, \(\liminf_{h\to 0^{+}}\frac{F(x+h)-F(x)}{h}\geq (1+\varepsilon )L.\) Sendo \(\varepsilon \) arbitrário, a derivada à direita de F existe no ponto x e é igual a L. Similarmente, a derivada à esquerda é L. 

Como \(F'=G'\) quase sempre, então F' é localmente integrável. Se \(a < b,\) tome \(a_{n}\downarrow a\) e \(b_{n}\uparrow b\), de modo que F e G coincidem em \(a_{n}\) e \(b_{n}\). Então, obtemos 
\begin{align*}
  F(b) - F(a) &\geq F(b_{n}) - F(a_{n})\\ 
              &= G(b_{n}) - G(a_{n})\\ 
              &\geq \int_{a_{n}}^{b_{n}}G'(x)dx\\ 
              &= \int_{a_{n}}^{b_{n}}F'(x)dx.
\end{align*}
Portanto, tomando \(n\to \infty\) e pelo \hyperlink{monotone_convergence}{\textit{Teorema da Convergência Monótona}}, temos o resultado. \qedsymbol
\end{proof*}
  Observe que, se F é a função de Cantor-Lebesgue, então \(F'(x) = 0\) quase sempre. De fato, isto ocorre em \(C ^{\complement}\), sendo C o conjunto de Cantor. Com isso, 
    \[
      1 = F(1) - F(0) > 0 = \int_{0}^{1}F'(x)dx.
    \]
  Isto mostra que, em geral, não temos igualdade no Teorema anterior.
 \begin{def*}
   Seja \(f:[a, b]\rightarrow \mathbb{R}\). Diremos que f é de \textbf{variação limitada} em [a, b] se 
     \[
       V_f[a, b] = \sup_{\mathcal{P}}\biggl\{\sum\limits_{i=1}^{k}|f(x_{i}) - f(x_{i-1})|\biggr\} < \infty,
     \]
  em que o supremo é tomado em todas as partições \(\mathcal{P}: a = x_{0} < x_1 < \dotsc <x_{k} = b\) de [a, b]. \(\square\)
 \end{def*}
 Se f é crescente, então 
   \[
     \sum\limits_{i=1}^{k} |f(x_{i}) - f(x_{i-1})| = \sum\limits_{i=1}^{k}f(x_{i}) - f(x_{i-1}) = f(x_{k}) - f(x_{0}) < \infty,
   \]
  mostrando que toda função crescente é de variação limitada. Além disso, se f e g são de variação limitada e \(c\in \mathbb{R},\) então \(f\pm g,\) \(cf\) e \(fg\) são de variação limitada. Também, se f for Lipschitz, então vale 
    \[
      \sum\limits_{i=1}^{k}|f(x_{i}) - f(x_{i-1})| \leq C \sum\limits_{i=1}^{k}|x_{i}-x_{i-1}| \leq C(b-a).
    \]
  Logo, f é de variação limitada. Finalmente, se f é continuamente derivável, então ela é de variação limitada.
 \begin{example}
  \begin{itemize}
    \item[i)] A função \(f(x) = [x], x\in [0, 2]\), é de variação limitada e não é contínua;
    \item[ii)] Se f é de variação limitada, então f é limitada;
    \item[iii)] Seja 
        \[
          f(x) = \left\{\begin{array}{ll}
              x\sin^{}{\biggl(\frac{\pi }{x}\biggr)},\quad &x\in (0 ,1)\\ 
              0,&\quad x = 0
            \end{array}\right..
        \]
    Esta f é contínua, mas não é de variação limitada, pois a série \(\sum\limits_{}^{}\frac{1}{2n+1}\) diverge. Então, a \(S_{n}\) soma parcial não é limitada. Para ver isto, considere a partição \(P = \biggl\{0, \frac{2}{2n+1}, \frac{2}{2n-1}, \dotsc , \frac{2}{5}, \frac{2}{3}, q\biggr\}.\) Segue que \(S_{n} = \sum\limits_{i=1}^{k}|f(x_{i}) - f(x_{i-1})|\) não 
  é limitada.
  \item[iv)] Seja 
    \[
      f(x) = \left\{\begin{array}{ll}
          1,&\quad x\in \mathbb{Q}^{\complement}\\
          0,&\quad x\in \mathbb{Q}
        \end{array}\right..
    \]
    Como cada intervalo existe um irracional, se tomarmos uma partição com n subintervalos, tomemos um irracional e um racional em cada subintervalo. Daí, \(S_{n} = 1 + 1 +\dotsc +1\geq \frac{n}{2},\) em que \(S_{n}\) é ilimitada.
  \end{itemize}
 \end{example}
  Toda função de variação limitada pode ser escrita como uma diferença de duas funções crescentes.
 \begin{lemma*}
   Se f é de variação limitada em \([a, b]\), então f pode ser escrita como \(f= f_1 - f_2,\) em que \(f_1, f_2\) são crescentes.
 \end{lemma*}
\begin{proof*}
  Defina 
    \[
      f_1(y) = \sup_{\mathcal{P}}\biggl\{\sum\limits_{i=1}^{k}|f(x_{i})-f(x_{i-1})|^{+}\biggr\}
    \]
  e 
    \[
      f_2(y) = \sup_{\mathcal{P}}\biggl\{\sum\limits_{i=1}^{k}|f(x_{i})-f(x_{i-1})|^{-}\biggr\}
    \]
    em que o supremos é tomado em todas as partições \(\mathcal{P}: a = x_{0} < x_1 < \dotsc < x_{k} = y\) para \(y\in [a, b].\) Com isso, ambas \(f_1, f_2\) são mensuráveis por serem crescentes. 
Como 
  \[
    \sum\limits_{i=1}^{k}|f(x_{i}) - f(x_{i-1})|^{+} = \sum\limits_{i=1}^{k}|f(x_{i}) - f(x_{i-1})|^{-} + f(y) - f(a),
  \]
  tomando o supremo sobre todas as partições \(a=x_{0} < x_1 <\dotsc <x_{k} = y\) com \(y\in [a, b]\), temos 
    \[
      f_1(y) = f_2(y) + f(y) - f(a).
    \]
    Portanto, notando que as duas \(f_1, f_2\) são crescentes em y e isolando f(y), segue o resultado. \qedsymbol
\end{proof*}
  Utilizando os resultados até o presente momento, obtivemos que toda função de variação limitada é derivável quase sempre. No entanto, a recíproca não é verdadeira. Basta notar que 
    \[
      f(x) = \sin^{}{\biggl(\frac{1}{x}\biggr)},\quad x\in [0, 1]
    \]
  é diferenciável quase sempre, mas não é de variação limitada.
 \begin{lemma*}
   Se \([c, d]\subseteq [a, b]\), então \(f_1(d) - f_1(c)\leq V_f[c, d]\), em que 
     \[
       V_f[a, b] = \sup_{\mathcal{P:a_{0}=x_1 < x_2 < \dotsc <x_{k}=b}}\biggl\{\sum\limits_{i=1}^{k}|f(x_{i}) - f(x_{i-1})|\biggr\}.
     \]
  O mesmo vale se trocarmos \(f_1\) por \(f_2\).
 \end{lemma*}
\begin{proof*}
  Seja \(\mathcal{P}\) a partição de \([a, d]\) dada por \(a = x_{0} < x_1 < \dotsc < x_{k} = d\) em \([a, d].\) Seja \(\mathcal{P}_{0} = \mathcal{P}\cup \{c\},\) descrita por 
  \(a = x_{0} < x_1 < \dotsc < x_{j-1}\leq c < x_{j} < \dotsc <x22 = b.\) Tome \(\mathcal{P}'\) como a partição \(a = x_{0} < x_1 <\dotsc x_{j-1}\leq c\). 
  e \(\mathcal{P}'': x_{j} < \dotsc < x_{n}=b\).
    
Como \((r+s)^{+} \leq r^{+} + s^{+}, \) temos
 \begin{align*}
   \sum\limits_{\mathcal{P}}^{}|f(x_{i}) - f(x_{i-1})|^{+} &\leq \sum\limits_{\mathcal{P}_{0}}^{}|f(x_{i}) - f(x_{i-1})|^{+} \\ 
                                                           &= \sum\limits_{\mathcal{P}'}^{}|f(x_{i}) - f(x_{i-1})|^{+}\sum\limits_{\mathcal{P}''}^{}|f(x_{i}) - f(x_{i-1})|^{+}\\ 
                                                           &\leq \sum\limits_{\mathcal{P}'}^{}|f(x_{i})-f(x_{i-1})|^{+} + \sum\limits_{\mathcal{P}''}^{}|f(x_{i}) - f(x_{i-1})|\\ 
                                                           &\leq f_1(c) + V_f[c, d].
 \end{align*}
  Aqui, usamos a notação \(\sum\limits_{\mathcal{P}}^{} = \sum\limits_{i=1}^{k}.\) Tomando o supremo em todas as partições \(\mathcal{P},\) temos 
    \[
      f_1(d) \leq f_1(c) + V_f[c, d]
    \]
  Portanto, 
    \[
      f_1(d) - f_1(c) \leq V_f[c, d].
    \]
  Para \(f_2,\), é similar. \qedsymbol
\end{proof*}
 \begin{def*}
   Seja f de variação limitada. Coloque, \(f = f_1 - f_2\), em que \(f_1\) e \(f_2\) são crescentes. Então, a quantidade 
     \[
       f_1(b) + f_2(b) - (f_1(a) + f_2(a))
     \]
     é chamada \textbf{variação total de f em [a, b]}. A variação total f em \([a, b]\) e \([b, c]\), então, coincide com a variação total em \([a, c]. \quad \square\)
 \end{def*}
 Observe que, se f é crescente em [a, b] e contínua à direita, podemos escrever \(f = f_1 - f_2\), em que \(f_1\) é contínua à direita e \(f_2(x) = \sum\limits_{a < t < x}^{}(f(t) - f(t^{-})).\) No somatório, apenas um número enumerável é não-nulo. Cada termo da série é não-negativa e as oma é finita, já que ela é limitada por \(f(x)-f(a).\) Assim, podemos 
decompor qualquer função de variação limitada que seja contínua à direita.
 \begin{def*}
   Seja \(f:[a, b]\rightarrow \mathbb{R}\). Dizemos que f é \textbf{absolutamente contínua em [a, b]} se dado \(\varepsilon  > 0\), existir \(\delta  > 0\) tal que 
     \[
       \sum\limits_{i=1}^{k}|f(b_{i}) - f(a_{i})| < \varepsilon 
     \]
  sempre que 
    \[
      \sum\limits_{i=1}^{k}|b_{i} - a_{i}|  < \delta,
    \]
  em que \(\{(a_{i}, b_{i})\}\) é uma coleção finita disjunta de intervalos. \(\square\)
 \end{def*}
  Toda função absolutamente contínua é contínua, mas a volta não é verdade. Para isso, utilize a função de Cantor: 
 \begin{example}
   A função de Cantor não é absolutamente contínua pois, tomando como partição no n-ésimo passo os extremos dos \(2^{n}\) subintervalos de comprimento \((1/3)^{n},\) então nos extremos, as funções de Cantor coincidem por serem constantes. Daí, lembrando que \(f(0) = 0 \) e \(f(1) = 1,\) temos 
     \[
       S_{n} f(a_1) - f(0) + f(a_2) - f(a_1) + \dotsc  1 = f(1) - f(a_{n-1}) = f(1) = 1. 
     \]
    Por outro lado, o comprimento total dos intervalos omitidos vale \(2^{n}\biggl(\frac{1}{3}\biggr)^{n} = \biggl(\frac{2}{3}\biggr)^{n},\) que vai a zero. Podemos fazer menor que \(\delta ,\) mas achamos \(S_{n} \geq \varepsilon \).
 \end{example}
 \begin{lemma*}
   Se f é absolutamente contínua, então f é de variação limitada.
 \end{lemma*}
\begin{proof*}
  Da definição com \(\varepsilon  = 1\), existe \(\delta  \) tal que 
    \[
      \sum\limits_{i=1}^{k}|f(b_{i}) - f(a_{i})| < 1
    \]
  sempre que \(\sum\limits_{i=1}^{k}(b_{i} - a_{i}) < \delta \) e que \((a_{i}, b_{i})\) são intervalos disjuntos. 

  Portanto, para cada j, a variação total de f em \([a + j\delta, a + (j+1)]\) é menor ou igual a 1, donde segue que a variação total de f sobre [a, b] é finita e menor. \qedsymbol
\end{proof*}
\begin{lemma*}
  Suponha f de variação limitada. Sabemos que \( f = f_1 - f_2\), em que \(f_1, f_2\) são crescentes. Se f é absolutamente contínua, então \(f_1, f_2\) também serão.
\end{lemma*}
 \begin{proof*}
   Dado \(\varepsilon \), existe \(\delta \) tal que \(\sum\limits_{\ell =1}^{m}|f(B_{\ell }) - f(A_{\ell })|\) sempre que \(\sum\limits_{i=1}^{m}(b_{i}-a_{i}) < \delta .\) Precisamos provar que \(\sum\limits_{i=1}^{m}|f_1(b_{i}) - f_1(a_{i})|\leq \varepsilon ,\) com o mesmo válido para \(f_2\).
   Seja a partição de \((a_{i}, b_{i})\) dada por \(a_{i} = s_{i0} < s_{i1} < \dotsc <s_{iJ_{i}} = b_{i}.\) Então, 
     \[
       \sum\limits_{i=1}^{k}\sum\limits_{j=0}^{J_{i}-1}(s_{i, j+1} - s_{ij}) = \sum\limits_{i=1}^{k}(b_{i}-a_{i})\leq \delta .
     \]
    Aplicando o primeiro parágrafo com a coleção 
      \[
        \{(s_{ij}, s_{i, j+1}), i = 1, 2, \dotsc , k\quad j = 0, \dotsc , J_{i}-1\},
      \]
    temos 
      \[
        \sum\limits_{i=1}^{k}\sum\limits_{j=0}^{J_{i}-1}|f(s_{i, j+1}) - f(s_{ij})|\leq \varepsilon .
      \]
    Mantendo \(a_{i}, b_{i}\) fixos e tomando o supremo sobre todas partições, temos 
      \[
        \sum\limits_{i=1}^{k}V_j[a_{i}, b_{i}] \leq \varepsilon .
      \]
    Portanto, segue a conclusão para \(f_1.\) Para \(f_2\), é similar. \qedsymbol
 \end{proof*}
\begin{theorem*}
  Se F é absolutamente contínua, então \(F'\) existe quase sempre, é integrável e 
    \[
      \int_{a}^{b}F'(x)dx = F(b) - F(a).
    \]
\end{theorem*}
 \begin{proof*}
   Basta provarmos que F é crescente e absolutamente contínua. Seja \(\nu \) a medida de Lebesgue-Stieltjes definida em termos de F. Como F é contínua,
     \[
       F(d) - F(c) = \nu ((c, d)).
     \]
    Pela definição de absolutamente contínua, fazendo \(k\to \infty\), temos, para \(\varepsilon > 0\), a existência de \(\delta \) tal que 
      \[
        \sum\limits_{i=1}^{\infty}|F(b_{i}) - F(a_{i})| < \varepsilon 
      \]
    sempre que \(\sum\limits_{i=1}^{\infty}(b_{i}-a_{i})<\delta \) e que \((a_{i}, b_{i})\) forem intervalos disjuntos. Como qualquer aberto G pode ser escrito como união disjunta de intervalos abertos \((a_{i}, b_{i})\), podemos reescrever reordenando já que, 
  dado \(\varepsilon > 0\) e garantida a existência de \(\delta > 0\), eles são tais que 
    \[
      \nu (G) = \sum\limits_{i=1}^{\infty}\nu ((a_{i}, b_{i})) = \sum\limits_{i=1}^{\infty}|F(b_{i}) - F(a_{i})| < \varepsilon 
    \]
  para todo G aberto e \(m(G) < \delta .\)

  Se \(m(G) < \delta \) e A é Borel mensurável, existe aberto G, \(G\supseteq A\), tal que \(m(G) < \delta \) e, assim, \(\nu (A) \leq \nu (G) \leq \varepsilon .\) Concluímos disso que \(\nu << m.\) 
Logo, existe uma função não-negativa integrável f, pelo \hyperlink{radon_nikodym}{\textit{Teorema de Radon-Nikodym}}, tal que 
  \[
    \nu (A) = \int_{A}f dm,
  \]
  para todo conjunto A mensurável. Em particular, para cada \(x\in [a, b]\), 
    \[
      F(x) - F(a) = \nu ((a, x)) = \int_{a}^{x}f(y)dy.
    \]
    Portanto, fazendo \(x=b\), obtemos a existência de \(F'\) e igualdade a f quase sempre e, além disso, 
      \[
        F(b) - F(a) = \int_{a}^{b}F'(y)dy,
      \]
    como desejado. \qedsymbol
 \end{proof*}
 \newpage 
\section*{Nota ao Leitor}
  A partir deste ponto, as aulas foram apresentações por parte dos alunos. O devido crédito é dado a cada um que apresentou. Caso isto não ocorra, peço que contate-me em meu e-mail: renan.wenzel.rw\@ gmail.com.
\newpage

\section{Aula 11 - 29/01/2024}
\subsection{Motivações} 
\begin{itemize}
  \item Desigualdade de Hölder;
  \item Desigualdade de Minkowski;
  \item Espaços de Banach.
\end{itemize}
\subsection{O Espaço \(L^{p}\)}
\subsubsection*{Leonardo C. - Desigualdade de Hölder e Minkowski}
 \begin{def*}
   Seja \((X, \mathcal{A}, \mu )\) um espaço de medida \(\sigma \)-finita. 
  \begin{itemize}
    \item[i)] Se \(1\leq p<\infty\), defina a \textbf{norma }\(L^{p}\) de f por 
      \[
        \Vert f \Vert_{p}=\biggl(\int_{}^{}|f|^{p}d\mu \biggr)^{\frac{1}{p}}.
      \]
    \item[2)] Se \(p=\infty\), defina a \textbf{norma }\(L^{\infty}\) de f por 
      \[
        \Vert f \Vert_{\infty}=\mathrm{essup}_{x\in X}|f(x)| = \inf_{}\{M\geq 0: \mu (\{x:|f(x)|>M\})=0\}.
      \]
      Se tal M não existir, então \(\Vert f \Vert_{\infty} = \infty.\) \(\square\)
  \end{itemize}
 \end{def*}
 É preciso fazermos algumas observações para melhor entendimento. A começar por notar que a norma \(L^{\infty}\) de uma função f é o menor número M tal que \(|f|\leq M\) quase toda parte (q.t.p.). 

 Para todo \(1\leq p\leq \infty\), o espaço \(L^{p}\) é o conjunto 
   \[
     \{f: f \text{ é mensurável e }\Vert f \Vert_{p}<\infty\}.
   \]
   Podemos escrever \(L^{p}(X)\) ou \(L^{p}(\mu )\) se desejamos enfatizar o espaço ou a medida. Com isto tudo, 
     \[
       \Vert f \Vert_{p} = 0 \Longleftrightarrow f = 0 \mathrm{q.t.p}
     \]
 \begin{def*}
   Seja \(1 < p < \infty\) e defina q tal que 
     \[
       \frac{1}{p} + \frac{1}{q} = 1.
     \]
  Chamamos p e q de \textbf{expoentes conjugados}. Casos específicos são o caso \(p =1 \), para o qual \(q=\infty\), e \(p = \infty\), ao qual associamos \(q = 1\) como conjugado. \(\square\)
 \end{def*}
 \hypertarget{holder}{
   \begin{prop*}[Desigualdade de Hölder]
    Sejam \(1 < p, q < \infty\) expoentes conjugados. Se f e g são funções mensuráveis, então 
      \[
        \int_{}^{}|fg|d\mu \leq \Vert f \Vert_{p}\Vert g \Vert_{q}.
      \]
    O resultado também vale se \(p=\infty\) e \(q=1\), ou se \(p = 1\) e \(q=\infty\).
  \end{prop*}
 }
\begin{proof*}
  Primeiramente, seja \(p=\infty\) e \(q=1\). Caso \(M = \Vert f \Vert_{\infty},\) então \(|f|\leq M\) q.t.p. e, sendo assim, 
    \[
      \int_{}^{}|fg|d\mu = \int_{}^{}|f||g|d\mu \leq \int_{}^{}M|g|d\mu = M \int_{}^{}|g|d\mu = M \Vert g \Vert_{1}.
    \]
  Analogamente, prova-se o caso \(p=1\) e \(q=\infty\). 

  Agora, seja \(1 < p, q < \infty\). Se \(\Vert f \Vert_{p} = 0,\) acabou, pois \(f=0\) q.t.p. e \(\int_{}^{}|fg|d\mu  = 0,\) mostrando que o resultado é trivial neste caso e naquele em que \(\Vert g \Vert_{p} = 0.\) Como 
o resultado também é imediato nos casos em que um dos dois é infinito, vamos nos restringir ao caso que ambos são finitos não-nulos. Sejam 
  \[
    F(x) = \frac{|f(x)|}{\Vert f \Vert_{p}}\quad\&\quad G(x) = \frac{|g(x)|}{\Vert g \Vert_{p}}.
  \]
  Observe que 
    \[
      \Vert F \Vert_{p} = 1\quad\&\quad \Vert G \Vert_{p} = 1.
    \]
  Então, basta mostrarmos que 
    \[
      \int_{}^{}FGd\mu \leq 1.
    \]
    Para começar, Lembre-se que a função \(e^{t}\) é \textit{convexa}, ou seja,
      \[
        e^{\lambda a + (1-\lambda )b}\leq \lambda e^{a}+(1-\lambda )e^{b},\quad \forall a\leq b\in \mathbb{R}\quad\&\quad 0\leq \lambda \leq 1.
      \]
  Caso \(F(x)\neq0\) e \(G(x)\neq0\), tome \(a = p \ln^{}{(F(x))}, b = q\ln^{}{(G(x))}, \lambda = 1/p\) e \(1-\lambda =1/q.\) Então,
    \[
      F(x)G(x)\leq \frac{F(x)^{p}}{p} + \frac{G(x)^{q}}{q}.
    \]
  Esta desigualdade vale trivialmente se \(F(x) = 0\) ou \(G(x) = 0.\) Integrando ambos os lados, então, chegamos em 
    \[
      \int_{}FG d\mu_{}\leq \frac{\Vert F \Vert_{p}^{p}}{p}+\frac{\Vert G \Vert_{q}^{q}}{q} = \frac{1}{p} + \frac{1}{q} = 1.
    \]
  Portanto, provamos a forma equivalente da Desigualdade de Hölder. \qedsymbol
\end{proof*}
 \begin{lemma*}
   Se \(a, b\geq 0\) e \(1\leq p < \infty\), então 
     \[
       (a+b)^{p} \leq 2^{p-1}a^{p} + 2^{p-1}b^{p}.
     \]
 \end{lemma*}
\begin{proof*}
  Os casos \(p=1\) e \(a=0\) são ambos óbvios, de forma que faz sentido pedir que \(p > 1\) e \(a > 0\). Divide-se os dois lados por \(a^{p}\), a fim de obter 
    \[
      \biggl(\frac{a+b}{a}\biggr)^{p}\leq 2^{p-1} + 2^{p-1}\biggl(\frac{b}{a}\biggr)^{p}.
    \]
  Disto, fazendo \(x=b/a\), segue que 
    \[
      (1+x)^{p}\leq 2^{p-1} + 2^{p-1}x^{p}.
    \]
  Ainda mais, se \(f(x) = 2^{p-1} + 2^{p-1}x^{p}-(1+x)^{p},\) então basta mostrar que \(f(x)\geq 0\) para todo \(x\geq 0\). 

  Com efeito, temos 
 \begin{itemize}
   \item \(f(0) = 2^{p-1}-1^{p} = \frac{2^{p}}{2}-1 > 0\), pois \(p>1\); 
   \item \(f(1) = 2^{p-1} + 2^{p-1} - 2^{p} = \frac{2^{p}}{2} + \frac{2^{p}}{2} - 2^{p} = 2^{p}-2^{p} = 0\);
   \item \(\lim_{x\to \infty}f(x) = \infty\).
 \end{itemize}
 Como a única solução para \(f'(x) = 0\) em \((0, \infty)\) é \(x=1\), segue que f atinge seu mínimo em \(x=1\). Portanto, \(f(x)\geq 0\) para todo \(x\geq 0\).
\end{proof*}
\hypertarget{minkowski}{
  \begin{prop*}[Desigualdade de Minkowski]
   Se \(1\leq p\leq \infty\) e f, g são mensuráveis, então 
     \[
       \Vert f + g \Vert_{p}\leq \Vert f \Vert_{p} + \Vert g \Vert_{p}.
     \]
 \end{prop*}
}
  Gostaríamos de dizer, em virtude da Desigualdade de Minkowski, que \(L^{p}\) é um \textit{espaço normado}. No entanto, devemos ser cuidadosos, pois \(\Vert f \Vert_{p} = 0\) não implica que \(f\equiv 0\). Até o momento, só sabemos que 
    \[
      \Vert f \Vert_{p} = 0 \Longleftrightarrow f = 0 \text{ q.t.p.}
    \]
  Logo, a definição precisa de \(L^{p}\) requer a introdução de uma relação de equivalência com base em medida, isto é,
    \[
      f\sim g \Longleftrightarrow f = g \text{ q.t.p.}
    \]
  Assim, o espaço \(L^{p}\) consiste das classes de equivalência de funções mensuráveis f que satisfazem \(\int_{}^{}|f|^{p}d\mu <\infty.\) Além disso, defina \(\Vert f \Vert_{p}\) como a norma \(L^{p}\) de qualquer função 
  na mesma classe de equivalência que f.
\begin{example}[Espaço \(\ell ^{p}\)]
  Seja \(X = \mathbb{N}\) equipado com a medida de contagem. Então, \(L^{p}(\mathbb{N})\) consiste em todas as sequências \(x=(x_1, x_2, \dotsc )\) tais que 
    \[
      \int_{}^{}|x_{n}|^{p}d\mu = \sum\limits_{n=1}^{\infty}|x_{n}|^{p}<\infty.
    \]
  Denotamos este espaço de sequências por \(\ell ^{p}(\mathbb{N})\). Caso \(1\leq p < \infty\), então 
    \[
      \Vert x \Vert_{\ell ^{p}} = \biggl(\sum\limits_{n=1}^{\infty}|x_{n}|^{p}\biggr)^{\frac{1}{p}}.
    \]
  Em contraste, se \(p = \infty\), definimos 
    \[
      \Vert x \Vert_{\ell ^{\infty}} = \sup_{x\in \mathbb{N}}|x_{n}|.
    \]
 \end{example}
 \begin{theorem*}
   Se \(1\leq p\leq \infty\), então \(L^{p}\) é completo.
 \end{theorem*}
\begin{proof*}
  Seja \(1\leq p\leq \infty\) fixado. Considere os seugintes passos: 
  
  1. Seja \(f_{n}\) uma sequência arbitrária de Cauchy em \(L^{p}.\) Escolha \(n_1 < n_2 <\dotsc \) tais que 
    \[
      m, n\geq n_{j}\Rightarrow \Vert f_{m} - f_{n} \Vert \leq 2^{-(j+1)}
    \]

  2. Seja \(n_{0} = 0\) e defina \(f_{0}\equiv 0\) (função identicamente nula). Nosso candidato para ser o limite de \(f_{n}\) será 
    \[
      \sum\limits_{m=1}^{\infty}(f_{n_{m}} - f_{n_{m-1}}).
    \]
  Tome \(g_{j}(x) = \sum\limits_{m=1}^{j}|f_{n_{m}}(x) - f_{n_{m-1}}(x)|,\) tal que \(g_{j}\) cresça em j para cada x em X fixado. Seja 
    \[
      g(x) = \sum\limits_{m=1}^{\infty}|f_{n_{m}}(x) - f_{n_{m-1}}(x)|,
    \]
  que pode inclusive ser infinito, o limite de g. Pela \hyperlink{minkowski}{\textit{Desigualdade de Minkowski}}, segue que 
    \[
      \Vert g_{j} \Vert_{p}\leq \sum\limits_{m=1}^{j}\Vert f_{n_{m}}-f_{n_{m-1}} \Vert_{p} \leq \Vert f_{n_{1}}-f_{n_{0}} \Vert_{p} + \sum\limits_{m=2}^{j}2^{-m}\leq \Vert f_{n_{1}} \Vert_{p}+\frac{1}{2}.
    \]
  Pelo \hyperlink{fatou}{\textit{Lema de Fatou}}, 
    \[
      \int_{}^{}|g(x)|^{p}\mu (dx)\leq \lim_{j\to \infty}\int_{}^{}|g_{j}(x)|^{p}\mu (dx) = \lim_{j\to \infty}\Vert g_{j} \Vert_{p}^{p}\leq \biggl(\frac{1}{2}+\Vert f_{n_1} \Vert^{p}\biggr)^{p}.
    \]
  Então, g é finita q.t.p., o que garante a convergência absoluta q.t.p. da série \(\sum\limits_{m=1}^{\infty}|f_{n_{m}}(x) - f_{n_{m-1}}(x)|.\)

  3. Defina \(f(x) = \sum\limits_{m=1}^{\infty}[f_{n_{m}}(x) - f_{n_{m-1}}(x)]\), a qual sabemos convergir absolutamente q.t.p e, assim, estando bem-definida q.t.p. Seja \(f(x) = 0\) para todo x em que a convergência absoluta não ocorre. Então,
    \[
      f(x) = \lim_{K\to \infty}\sum\limits_{m=1}^{K}[f_{n_{m}}(x) - f_{n_{m-1}}(x)] = \lim_{K\to \infty}f_{n_{K}}(x).
    \] 
  Pelo \hyperlink{fatou}{\textit{Lema de Fatou}}, 
    \[
      \Vert f - f_{n_{j}} \Vert_{p}^{p} = \int_{}^{}|f-f_{n_{j}}|^{p}\leq \liminf_{K\to \infty}\int_{}^{}|f_{n_{K}}-f_{n_{j}}|^{p} = \liminf_{K\to \infty}\Vert f_{n_{K}}-f_{n_{j}} \Vert_{p}^{p} \leq 2^{-(j+1)p}
    \]
  
  4. Finalmente, \(\Vert f - f_{n_{j}} \Vert_{p}\to 0\) sempre que j tende a infinito. Daí, como toda sequência de Cauchy com subsequência convergente converge para o mesmo limite, segue que \(f_{n}\to f\) sempre que \(n\to \infty\).
Em particular, \(\Vert f_{n_{j}} - f_{m}\Vert_{p}<\varepsilon \) para j suficientemente grande e \(\varepsilon  > 0\). Pelo \hyperlink{fatou}{\textit{Lema de Fatou}} e se \(m\geq N\),
  \[
    \Vert f-f_{m} \Vert_{p}^{p} - \biggl(\int_{}^{}\liminf_{j\to \infty}|f_{n_{j}}-f_{m}|^{p}\biggr) \leq \liminf_{j\to \infty}\Vert f_{n_{j}}-f_{m} \Vert_{p}^{p}\leq \varepsilon^{p}.
  \]
  Isto prova que \(f_{n}\) converge para f na norma \(L^{p}\) e, logo, que \(L^{p}\) é completo para \(1\leq p < \infty.\)

  
  Para o último caso, tome \(p=\infty\). Considere os passos: 

  1. Seja \(f_{n}\) uma sequência arbitrária de Cauchy em \(L^{p}.\) Para cada \(m, n\in \mathbb{N}\), deifna 
    \[
      F_{m, n}\coloneqq \{x\in X: |f_{m}(x) - f_{n}(x)| > \Vert f_{m}-f_{n} \Vert_{\infty}\},
    \]
  de modo que \(\mu (F_{m, n})=0\) para todo m, n naturais. Seja \(F = \bigcup_{m, n}^{}F_{m, n}\) e \(E = F ^{\complement}\). Observe que \(\mu (E ^{\complement}) = \mu (F) = 0\) e, mais ainda, 
 \begin{align*}
   E &=\bigcap_{m, n\in \mathbb{N}}^{}\{x\in X: |f_{m}(x) - f_{n}(x)| \leq \Vert f_{m}-f_{n} \Vert_{\infty}\}\\
     &=\{x\in X: |f_{m}(x) - f_{n}(x)| \leq \Vert f_{m} - f_{n}(x) \Vert_{\infty}\text{ para todo }m, n\in \mathbb{N}\}.
 \end{align*}
  2. Seja \(\varepsilon  > 0\). Existe \(N\in \mathbb{N}\) tal que 
    \[
      m, n\geq N \Rightarrow \Vert f_{m}(x) - f_{n}(x) \Vert_{\infty} < \varepsilon .
    \]
  Com maior razão, para todo x em E e todos \(m, n\geq N\), temos 
    \[
      |f_{m}(x) - f_{n}(x)|\leq \Vert f_{m} - f_{n} \Vert_{\infty} < \varepsilon .
    \]
    Provamos, com isso, que para todo x em E, \(f_{n}(x)\) é sequência de Cauchy em \(\mathbb{K},\) em que \(\mathbb{K}\in \{\mathbb{C}, \mathbb{R}\}\). Como \(\mathbb{K}\) é completo, existe o limite 
      \[
        \lim_{n\to \infty}f_{n}(x) = f(x).
      \]
    É claro que \(f(x)\) está definida em \(E = F ^{\complement}\), donde, fazendo \(f(x) = 0\) para todo x em F, concluímos que \(f = \lim_{n\to \infty}f_{n}\chi_{E}\) é mensurável.

  3. Sendo \(f_{n}(x)\) uma sequência de Cauchy de números, exsite N natural tal que 
    \[
      m, n \geq N \Rightarrow |f_{m}(x) - f_{n}(x)| < \varepsilon .
    \]
  Fazendo \(n\to \infty\), encontramos 
    \[
      m\geq N \Rightarrow |f_{m}(x) - f(x)|\leq \varepsilon .
    \]
  Ainda mais, 
    \[
    m\geq N \Rightarrow \Vert f_{m}-f \Vert_{\infty} = \inf_{}\{M\geq 0: |f_{m}(x) - f(x)|\leq \varepsilon \}\leq \varepsilon,
    \]
    provando que \(f_m\to f\) na norma \(L^{\infty}.\)

  4. Por fim, pela \hyperlink{minkowski}{\textit{Desigualdade de Minkowski}}, 
    \[
      \Vert f \Vert_{\infty}\leq \Vert f_{m} \Vert_{\infty} + \Vert f_m - f \Vert_{\infty}\leq \Vert f \Vert_{\infty} + \varepsilon  < \infty
    \]
  para todo \(m\geq N\). Portanto, \(f\in L^{\infty},\) donde concluímos a completude do espaço \(L^{\infty}.\) \qedsymbol

\end{proof*}
\begin{prop*}
  O conjunto das funções contínuas com suporte compacto é denso em \(L^{p}(\mathbb{R})\) para \(1\leq p < \infty\).
\end{prop*}
\begin{proof*}
  Seja \(f\in L^{p}(\mathbb{R})\). Como \(\lim_{n\to \infty}f \chi_{[-n, n]} = f,\) é claro que 
    \[
      |f-f\chi_{[-n, n]}|^{p}\to 0
    \]
  sempre que n tender a infinito. Daí, sendo \(|f|^{p}\) integrável, como 
    \[
      |f-f\chi_{[-n, n]}|^{p}\leq 2^{p}|f|^{p},
    \]
  segue do \hyperlink{dominated_convergence}{\textit{Teorema da Convergência Dominada}} que 
    \[
      \Vert f - f\chi_{[-n, n]} \Vert_{p}^{p} = \int_{}^{}|f-f\chi_{[-n, n]}|^{p}\to 0
    \]
  sempre que \(n\to \infty\). Logo, é suficiente aproximar funções em \(L^{p}\) que possuem suporte compacto, já que 
    \[
      \Vert f -g  \Vert \leq \Vert f - f\chi_{[-n, n]} \Vert + \Vert f\chi_{[-n, n]} - g \Vert \leq \frac{\varepsilon }{2} + \frac{\varepsilon }{2} = \varepsilon.
    \]
  Como podemos escrever \(f=f^{+}-f^{-}\), podemos supor \(f\geq 0\) e, daí, encontrarmos sequência crescente de funções simples \(s_{n}\)
  tais que \(s_{n}\to f\). Desta forma, \(s_{n}\chi_{[-n, n]}\to f\) sempre que \(n\to \infty\) e como \(|f-s_{n}\chi_{[-n, n]}|\leq 2|f|^{p},\) segue do TCD que 
    \[
      \lim_{n\to \infty}\Vert f - s_{n}\chi_{[-n, n]} \Vert_{p}^{p} = \lim_{n\to \infty}\int_{}^{}|f-s_{n}\chi_{[-n, n]}|^{p} = 0.
    \]
    Então, basta aproximar funções simples com suporte compacto, denotadas por \(s_{n}\). Por linearidade, é suficiente aproximar funções características com suporte compacto. Sendo assim, seja E um conjunto de Borel em um intervalo limitado. Dado \(\varepsilon  > 0\), segue que existe uma função contínua com suporte compacto g e com valores em [0, 1] tal que 
      \[
        \int_{}^{}|g-\chi_{E}|<\varepsilon .
      \]
    Como \(|g-\chi_{E}|\leq 1,\) devemos ter \(|g-\chi_{E}|^{p}\leq |g-\chi_{E}|,\) donde concluímos que 
      \[
        \Vert g-\chi_{E} \Vert_{p}^{p} = \int_{}^{}|g-\chi_{E}|^{p} \leq \int_{}^{}|g-\chi_{E}| < \varepsilon .
      \]
  Sendo \(s_{n}\) uma sequência crescente de funções simples com suporte compacto, então para cada \(n\in \mathbb{N},\) 
    \[
      s_{n} = \sum\limits_{i=1}^{p}a_{i}\chi_{A_{i}}.
    \]
    Segue do que acabamos de ver que, para cada função característica \(\chi_{A_{i}}, i\in \{1, \dotsc , p\}\), existe uma função contínua com suporte compacto \(g_{i}\) tal que 
      \[
        \Vert g_{i}-\chi_{A_{i}} \Vert_{p}^{p}<\frac{\varepsilon ^{p}}{a_{i}2p}.
      \]
  Defina \(g = \sum\limits_{i=1}^{p}a_{i}g_{i},\) que é contínua com suporte compacto. Assim,
 \begin{align*}
   \Vert s_{n}-g \Vert_{p}^{p} &= \biggl\Vert \sum\limits_{i=1}^{p}a_{i}\chi_{A_{i}}-\sum\limits_{i=1}^{p}a_{i}g_{i} \biggr\Vert_{p}^{p}\\ 
                               &= \Vert a_1(\chi_{A_1}-g_1) + \dotsc + a_{p}(\chi_{A_p} - g_p) \Vert_{p}^{p} \\ 
                               &\leq \Vert a_1(\chi_{A_1}-g_1) \Vert_{p}^{p} + \dotsc + \Vert a_p(\chi_{A_p}-g_p) \Vert_{p}^{p}\\ 
                               &< a_1\frac{\varepsilon ^{p}}{a_12p} + \dotsc + a_p\frac{\varepsilon ^{p}}{a_p2p} = \frac{\varepsilon^{p}}{2}
 \end{align*}
  Finalmente, como \(s_{n}\to f\), podemos escolher \(s_{n}\) tal que 
    \[
      \Vert f-g \Vert_{p}^{p}\leq \Vert f-s_{n} \Vert_{p}^{p}+\Vert s_{n}-g \Vert_{p}^{p}<\frac{\varepsilon ^{p}}{2}+\frac{\varepsilon^{p}}{2} = \varepsilon^{p}.
    \]
  Portanto, tomando a p-ésima raíz,
    \[
      \Vert f-g \Vert_{p} < \varepsilon .\quad \text{\qedsymbol}
    \]

\begin{crl*}
  O conjunto das funções contínuas em [a, b] é denso no espaço \(L^{2}([a, b])\) com respeito à norma \(L^{2}([a, b])\).
\end{crl*}
  
\end{proof*}

\subsubsection*{Leonardo A. - Convolução e Operadores Lineares Contínuos}
  \begin{def*}
    A \textbf{convolução} de duas funções mensuráveis f, g é definida por 
      \[
        (f*g)(x)\coloneqq \int_{}^{}f(x-y)g(y)dy.
      \]
    quando a integral existir. \(\square\)
  \end{def*}
  Vamos começar fixando algumas coisas para esta seção. 
 \begin{itemize}
   \item As funções serão \(f:\mathbb{R}^{n}\rightarrow \mathbb{R}\);
   \item Uma integral existe quando, por exemplo, \(f\in L^{1} \text{ e } g\in L^{*}\), ou \(g\in L^{1}\text{ e }f\in L^{1}\).
 \end{itemize}
  Observe que \(*\) é uma operação \(*:L^{1}\times L^{1}\rightarrow L^{1}\) que é associativa, comutativa e sem identidade - suponha que existe \(I\in L^{1}\) tal que 
    \[
      I*f = f,\quad \forall f\in L^{1}.
    \]
  Então, a transformada de Fourier da convolução (veremos que satisfaz \(\widehat{f*g}(x) = \hat{f}(x)\cdot \hat{g}(x)\)) implicaria 
    \[
      \hat{I}(x)\cdot \hat{f}(x) = \hat{f(x)} \Rightarrow \hat{I}(x) = 0.
    \]
    Mas, \(\hat{I}(x)\to 0,\) uma contradição.
   \begin{prop*}
    \begin{itemize}
      \item[1)]Assuma \(f, g\in L^{1}.\) Então,  \(f*g\in L^{1}\) com \(|f*g|_1\leq |f|_1|g|_1\)
        \item[2)] Se \(1\leq p\leq \infty\), \(f\in L^{1}\) e \(g\in L^{p},\) então 
          \[
            |f*g|_p \leq |f|_1|g|_p.
          \]
    \end{itemize}
   \end{prop*}
  \begin{proof*}
    O primeiro passo é observar que 
    \begin{align*}
      \int_{}^{}|f*g(x)|dx &= \int_{}^{}\biggl\vert \int_{}^{}f(x-y)g(y)dy \biggr\vert dx \\ 
                           &\leq \int_{}^{}\int_{}^{}\biggl\vert |f(x-y)||g(y)| \biggr\vert dydx\\ 
                           &=\int_{}^{}\biggl(\int_{}^{}|f(x-y)|dx|g(y)|\biggr)dy\\ 
                           &= \int_{}^{}|f|_1|g(y)|dy = |f|_1|g|_1.
    \end{align*}
    em que utilizamos o \hyperlink{fubini_tonelli}{\textit{Teorema de Fubini-Tonelli}}.

    Para o segundo passo, note que se \(p = \infty\), não há nada ser feito. Então, assuma que \(p < \infty\) e seja q o seu conjugado \(\biggl(\frac{1}{p} + \frac{1}{q} = 1\biggr)\). Temos 
    \begin{align*}
      \biggl\vert \int_{}^{}f(y)g(x-y)dy \biggr\vert &\leq \int_{}^{}|f(y)g(x-y)|dy \\ 
                                                     &\leq \int_{}^{}|f(y)|^{\frac{1}{q}}|f(y)|^{1-\frac{1}{q}}|g(x-y)|dy\\ 
                                                     &\leq \biggl(\int_{}^{}|f(y)|dy\biggr)^{\frac{1}{q}}\biggl(\int_{}^{}|f(y)|\biggr)^{p \biggl(1-\frac{1}{q}\biggr)}|g(x-y)|dy.
    \end{align*}
    Por \hyperlink{fubini_tonelli}{\textit{Fubini}} e o acima, 
    \begin{align*}
      |f*g|_{p}^{p} = \int_{}^{}|f \cdot g(x)|^{p}dx &\leq \int_{}^{}\biggl(\int_{}^{}|f(y)|dy\biggr)^{\frac{p}{q}}\biggl(\int_{}^{}|f|g(x-y)dx\biggr)^{p}\\ 
                                                     &= |f|_{1}^{\frac{p}{q}}|g|_{p}^{p}\int_{}^{}f(y)dydx\\ 
                                                     &= |f|^{1 + \frac{p}{q}}|g|_{p}.
    \end{align*}
    tirando a p-ésima raíz, então, chegamos no desejado. \qedsymbol
  \end{proof*}
  Estudaremos agora a aproximação/regularização/molificação de funções. Fixe uma função \(\varphi : \mathbb{R}^{n}\rightarrow \mathbb{R}\) de classe \(C_{K}^{\infty}, \varphi \geq 0\) e \(\int_{}^{}\varphi dx = 1\). Por exemplo,
    \[
      \varphi (x) = \left\{\begin{array}{ll}
          e^{-\frac{1}{|x|^{2}}},\quad &x\neq 0\\ 
          0,\quad &x = 0
        \end{array}\right.
    \]
  Utilizaremos a notação, para cada \(\varepsilon  > 0\) e sendo n a dimensão do espaço euclidiano, 
    \[
      \varphi_{\varepsilon }(x) = \frac{1}{\varepsilon^{n} }\varphi (\varepsilon^{-1}x)
    \]
  Segue que, independente do \(\varepsilon \), 
    \[
      \int_{}^{}\varphi_{\varepsilon }(x) = 1.
    \]
 \begin{theorem*}
   Seja \(f\in L^{p}, 1\leq p\leq \infty\).
  \begin{itemize}
    \item[1)] Para cada \(\varepsilon > 0\), a função \(f*\varphi_{\varepsilon }\) é de classe \(C_{K}^{\infty}\). Além disso, se \(\alpha  = (\alpha_1, \alpha_2, \dotsc \alpha_n)\) e 
  \(x = (x_1, x_2, \dotsc , x_{n})\), vale que 
    \[
      \frac{\partial^{\alpha }(f*\varphi_{\varepsilon })}{\partial x^{\alpha }} = f*\frac{\partial^{\alpha }\varphi_{\varepsilon }}{\partial x^{\alpha }} \Longleftrightarrow \frac{\partial^{\alpha_1 + \alpha_2 + \dotsc + \alpha_{n}}f*\varphi_{\varepsilon}}{\partial x_{1}^{\alpha_1}x_{2}^{\alpha_2}\dotsc x_{n}^{\alpha_{n}}} = f* \frac{\partial^{\alpha_1 + \alpha_2 + \dotsc \alpha_{n}}\varphi_{\varepsilon }}{\partial x_1^{\alpha_1}x_{2}^{\alpha_2}\dotsc x_{n}^{\alpha_{n}}}.
    \]
    \item[2)] Segue que \(f*\varphi_{\varepsilon }\to f\) q.t.p. Quando \(\varepsilon \to 0\)
    \item[3)] Temos \(f*\varphi_{\varepsilon }\to f\) uniformemente em compactos, sempre que f for contínua.
    \item[4)] Para \(1\leq p < \infty\), \(f*\varphi_\varepsilon \to f\) no sentido \(L^{p}.\)
  \end{itemize}
 \end{theorem*}
\begin{proof*}
  Vamos assumir que o suporte \(\mathrm{supp}(\varphi )\subseteq B(0, R)\) para \(R>0\) grande suficiente. Seja \(e_{i}\) o vetor unitário na i-ésima direção e escreva 
  \[
    f*\varphi_{\varepsilon }(x+he_{i}) - f*\varphi_{\varepsilon }(x) = \int_{}^{}f(y)[\varphi_{\varepsilon }(x+he_{i}-y) - \varphi_{\varepsilon }(x-y)]dy.
  \]
  Como \(\varphi_{\varepsilon }\) é \(C^{1},\) existe \(c_{1}\) contínua tal que 
  \[
    \frac{1}{n}\biggl\vert \varphi _{\varepsilon }(x_he_{i}-y) - \varphi _{\varepsilon }(x-y) \biggr\vert\leq c_1\frac{|h|}{n},\quad \forall x, y.
  \]
  Assim, 
    \[
       \int_{}^{}f(y)[\varphi_{\varepsilon }(x+he_{i}-y) - \varphi_{\varepsilon }(x-y)]dy = \int_{}^{}f(y)\frac{\partial \varphi_{\varepsilon }(x-y)}{\partial x_{i}}dy.
    \]
  Os outros casos seguem por indução, finalizando a prova do primeiro item.

  Para o segundo item, considere
    \[
      f(x) = f(x)\cdot 1=f(x)\int_{}^{}\varphi (x-y)dy 
    \]
  pontual em x. Logo,
  \begin{align*}
    f*\varphi_{\varepsilon }(x) - f(x) &= \int_{}^{}[f(y)-f(x)]\varphi_\varepsilon  (x-y)dy\\ 
                                       &= \frac{1}{\varepsilon ^{n}}\int_{B(0, R)}^{}[f(y)-f(x)]\varphi \biggl(\frac{x-y}{\varepsilon }\biggr)dy.
  \end{align*}
  Fazendo uma mudança de variáveis \(w=\frac{x-y}{\varepsilon }\), 
  \[
   |f*\varphi _{\varepsilon }(x)-f(x)|\leq |\varphi |_{\infty}m(B(0, R\varepsilon ))\frac{1}{m(B(0, R\varepsilon ))}\int_{B(0, R\varepsilon )}^{}f(x)-f(y)dy,
  \]
  que vai a 0 quando \(\varepsilon \to 0\), assim terminando o item 2. 

  Para provar o item 3, seja \(N > 0\). Se 
  \[
   |f*\varphi _{\varepsilon }(x)-f(x)|\leq |\varphi |_{\infty}m(B(0, R\varepsilon ))\frac{1}{m(B(0, R\varepsilon ))}\int_{B(0, R\varepsilon )}^{}f(x)-f(y)dy,
  \]
  temos 
    \[
      \sup_{|x|\leq N}|f*\varphi_{\varepsilon }(x) - f(x)|\leq |\varphi |_{\infty}m(B(0, R))\sup_{|x|<M, |yx|\leq R\varepsilon }|f(x)-f(y)|.
    \]
  Como todos os termos à direita são constantes, obtivemos o que queríamos.
  
  Por fim, para o item 4, seja \(\varepsilon > 0, f\in L^{p}\) e g limitada suportada em \(B(0, N)\) tal que \(|f-g|_{p}< \varepsilon .\) Defina \(f_{n} = f\chi_{B(0, N)}\) e note que \(|f-f_{n}|\to 0\) quando \(N\to \infty\). Assim, 
  \(|f-f_{N}|<\frac{\varepsilon }{2}\) para N suficientemente grande. Podemos tomar, sem perda de generalidade, g simples que aproxima \(f_{N}\) por \(\frac{\varepsilon }{2},\) ou seja, \(|f_{N}-g|<\frac{\varepsilon }{2}\). Como \(g\in L^{\infty}\), 
  \[
    |g*\varphi_{\varepsilon }|\leq \Vert g \Vert_{\infty}|\varphi |_{\varepsilon } = \Vert g \Vert_{\infty}.
  \]
  Do item 2, \(g*\varphi_{\varepsilon }\to g\) q.t.p. quando \(\varepsilon \to 0\). Como \(\mathrm{supp}(\varphi_{\varepsilon })\subseteq B(0, R\varepsilon )\) e \(\mathrm{supp}(g)\subseteq B(0, N)\), obtemos 
    \[
      \mathrm{supp}(g*\varphi_{\varepsilon })\subseteq B(0, N + R\varepsilon )
    \]
  Aplicando o \hyperlink{dominated_convergence}{\textit{Teorema da Convergência Dominada}}, 
    \[
      g*\varphi_{\varepsilon }\to g
    \]
  em \(L^{p}\) quando \(\varepsilon \to 0\), provando que o resultado vale para funções simples. Finalmente, 
    \[
      |f-g|_{p} <\varepsilon \quad\&\quad |f \varphi_{\varepsilon }(x) - g \varphi_{\varepsilon }(x)| \leq |f-g|_{p}|\varphi_{\varepsilon }|_{1}.
    \]
  Logo, 
    \[
      |f*\varphi_{\varepsilon } - f|_{p} = |f*\varphi_{\varepsilon } - g*\varphi_{\varepsilon } + g*\varphi_{\varepsilon } + g - g - f|_{p}\leq 2\varepsilon + |g*\varphi_{\varepsilon }-g|_{p}.
    \]
  Portanto,
    \[
      \limsup_{\varepsilon \to 0}|f*\varphi_{\varepsilon }-f|_p \leq 2\varepsilon.\quad \text{\qedsymbol}
    \]
\end{proof*}
 \begin{def*}
   Um \textbf{funcional linear} é uma função \(H:L^{p}\rightarrow \mathbb{R}\) que é linear. \(\square\)
 \end{def*}
\begin{def*}
  Um funcional linear é dito \textbf{limitado} se 
    \[
      \Vert H \Vert\coloneqq \sup_{|f|_{p}=1}\{H(f)\} < \infty.\quad \square
    \]
\end{def*}
 \begin{def*}
   O \textbf{espaço dual} de \(L^{p}\) é a coleção de todos os funcionais lineares limitados, denotado por \((L^{p})^{*}.\quad \square\)
 \end{def*}
\begin{example}
  Por exemplo,
    \[
      \biggl(C([a, b])\biggr)^{*} = \{\mu : \mu \text{ é medida suportada}\}.
    \]
\end{example}
 \begin{def*}
   A função \textbf{sinal} é definida como 
   \[
   \mathrm{sgn}(x)  = \left\{\begin{array}{ll}
       -1,\quad x < 0\\ 
       0,\quad x = 0\\ 
       1,\quad x > 0.
     \end{array}\right..\quad \square
 \]
 \end{def*}
  Vale notar que o dual do \(L^{p}\) é \(L^{q}\) quando p é finito e \((L^{\infty})^{*} = BMO\) das funções de variação média.
 \begin{theorem*}
   Seja \(1< p < \infty\). Se \(f\in L^{p}\), 
     \[
       |f|_{p}=\sup_{|g|_{q}\leq 1}\biggl(\int_{}^{}f(x)g(x)dx\biggr).
     \]
 \end{theorem*}
 \begin{crl*}
   Basta olhar para as funções simples e a proposição continua válida.
 \end{crl*}
\begin{prop*}
  Seja \(1 < p < \infty\) e fixe \(g\in L^{q}\). Defina 
 \begin{align*}
   H:&L^{p}\rightarrow \mathbb{R}\\ 
     &H(f)\coloneqq \int_{}^{}f(x)g(x)dx.
 \end{align*}
 Então, \(H\in (L^{p})^{*}.\)
\end{prop*}
\newpage 

\section{Aula 12 - 31/01/2024}
\subsection{Motivações} 
 \begin{itemize}
   \item Transformada de Fourier e Teorema de Plancherel
   \item Espaços de Banach.
 \end{itemize}
\subsection{Luca Maciel Alexander - Transformada de Fourier}  
 \begin{def*}
   Se f é uma função a valores complexos e \(f\in L^{1}(\mathbb{R}^{n})\), defina a \textbf{Transformada de Fourier de f}, \(\hat{f}\), como a função \(\hat{f}:\mathbb{R}^{n}\rightarrow \mathbb{C}\) dada por 
     \[
       \hat{f}(u) = \int_{\mathbb{R}^{n}}^{}e^{iu \cdot x}f(x)dx,\quad u\in \mathbb{R}^{n}.\quad \square
     \]
 \end{def*}
  Neste capítulo, \(u \cdot x\) denota o \textit{produto interno} de \(\mathbb{R}^{n}\)
 \begin{prop*}
   Suponha que \(f, g, f \cdot x_{j}\in L^{1}.\) Então,
  \begin{itemize}
    \item[1)] \(\frac{\partial \hat{f}}{\partial u_{j}}(u) = \widehat{(ix_{j}f(x))}(u)\)
      \item[2)] Se \(f:\mathbb{R}\rightarrow \mathbb{R}\) é absolutamente contínua, então 
        \[
          \hat{f'(x)}(u) = -iu_{j}\hat{f}(u)
        \]
      \item[3)] Temos \(\widehat{(f*g)}(u) = \hat{f}\cdot \hat{g}(u)\)
  \end{itemize}
 \end{prop*}
\begin{proof*}
  A prova para a 1 segue da definição e do \hyperlink{dominated_convergence}{\textit{Teorema da Convergência Dominada}}. Basta fazer 
    \[
      \lim_{h\to 0}\frac{\hat{f}(u+he_{j}) - \hat{f}(u)}{h} = \lim_{h\to 0} \frac{1}{h}\int_{}^{}(e^{i(u+he_{j})x}-e^{iux})f(x)dx = \int_{}^{}e^{iux}\biggl(\frac{e^{ihx_{j}} -1}{h}\biggr)f(x)dx.
    \]
  Como 
    \[
      \biggl\vert \frac{1}{h}\biggl(e^{ihx_{j}}-1\biggr) \biggr\vert\leq |x_{j}|,
    \]
  e como \(x_{j}f(x)\in L^{1},\) o lado direito converge para \(\int_{}^{}e^{iux}ix_{j}f(x)dx\). Portanto, o lado esquerdo converge e o limite é exatamente \(\frac{\partial \hat{f}}{\partial u_{j}}\).
\end{proof*}
\begin{lemma*}[Gaussiana Ponto Fixo]
  Suponha que \(f_1:\mathbb{R}\rightarrow \mathbb{R}\) é definida como 
    \[
      f_1(x) = \frac{1}{\sqrt[]{2\pi }}e^{-\frac{x^{2}}{2}}.
    \]
  Então, \(\hat{f}_{1}(u) = e^{-\frac{u^{2}}{2}}\).
  Mais ainda, se \(f_{n}:\mathbb{R}^{n}\rightarrow \mathbb{R}\) é dada por 
    \[
      f_{n}(x) = \frac{1}{(2\pi )^{\frac{n}{2}}}e^{\frac{-|x|^{2}}{2}}.
    \]
  Então, 
    \[
      \hat{f}_{n}(u) = e^{\frac{-|u|^{2}}{2}}.
    \]
\end{lemma*}
 \begin{lemma*}
   Suponha que \(\varphi \) pertence a \(L^{1}\) e que \(\int_{}^{}\varphi (x)dx = 1\). Coloque \(\varphi_{\delta }(x) = \delta ^{-n}\varphi \biggl(\frac{x}{\delta }\biggr)\)
  \begin{itemize}
    \item[1)] Se \(f\in L^{1}\), então \(\varphi_\delta * f\to f\) em \(L^{1}\)
      \item[2)] Se \(f\in C_{K}^{0}\), então \(\varphi_{\delta }*f\to f\) para todo x.
  \end{itemize}
 \end{lemma*}
 \hypertarget{inversion}{
   \begin{theorem*}[Teorema da Inversão]
    Suponha que \(f, \hat{f}\in L^{1}.\) Então,
      \[
        f(y) = \frac{1}{(2\pi )^{n}}\int_{}^{}e^{-iu \cdot y}\hat{f}(u)du \quad \text{q.s.}
      \]
\end{theorem*}}
 \begin{proof*}
   Seja \(g(x) = a^{-n}k \biggl(\frac{x}{a}\biggr).\) Então, a transformada de Fourier de g é \(\hat{k}(au),\) de modo que a transformada de Fourier de 
     \[
       \frac{1}{a^{n}}\frac{1}{(2\pi )^{\frac{n}{2}}}e^{-\frac{x^{2}}{2a^{2}}}
     \]
  é simplesmente \(e^{-a^{2}\frac{u^{2}}{2}}\) pelo Lema do Ponto Fixo. Colocando 
    \[
      H_{a}(x) = \frac{1}{(2\pi )^{n}}e^{-\frac{|x|^{2}}{2a^{2}}},
    \]
  obtemos 
    \[
      \hat{H}_{a}(u) = (2\pi )^{-\frac{n}{2}}a^{n}e^{-a^{2}\frac{|u|^{2}}{2}}.
    \]
  Note que 
 \begin{align*}
   \int_{}^{}\hat{f}(u)e^{-iu \cdot y}h_{a}(u)du &= \int_{}^{}\int_{}^{}e^{iu \cdot x}f(x)e^{-iu \cdot y}H_{a}(u)dxdu \\ 
                                                 &= \int_{}^{}\int_{}^{}e^{iu \cdot (x-y)}H_{a}(u)duf(x)dx\\ 
                                                 &= \hat{H}_{a}(x-y)f(x)dx,
 \end{align*}
 em que pudemos trocar a ordem de integração pois \(|e^{iu \cdot x}| = 1\) e 
   \[
     \int_{}^{}\int_{}^{}|f(x)||H_a(u)|dxdu < \infty.
   \]
   Temos a convergência
    \[
      \lim_{a\to \infty}\int_{}^{}\hat{f}(u)e^{-iu \cdot y}h_{a}(u)du =(2\pi )^{-n}\int_{}^{}\hat{f}(u)e^{-iu \cdot y}dy
    \]
    que ocorre devido ao \hyperlink{dominated_convergence}{\textit{Teorema da Convergência Dominada}}, \(\hat{f}\in L^{1}\) e a \(H_{a}(u)\to (2\pi )^{-n}\). Como 
      \[
        \int_{}^{}\hat{H}_{a}(y-x)f(x)dx = f*\hat{H}_{a}(y)
      \]
    por simetria de \(\hat{H}_{a}\), colocando \(\delta  = a^{-1},\) obtemos, portanto, que 
      \[
        \lim_{a\to \infty, L^{1}}f*\hat{H}_{a} = f
      \]
 \end{proof*}
 \hypertarget{plancherel}{
   \begin{theorem*}[Teorema de Plancherel]
    Suponha que f é contínua com suporte compacto. Então, \(\hat{f}\in L^{2}\) e 
      \[
        \Vert f \Vert_{2} = (2\pi )^{-\frac{n}{2}}\Vert \hat{f} \Vert_{2}.
      \]
  \end{theorem*}
 }
 \begin{proof*}
   Primeiramente, note que 
     \[
       \int_{}^{}\hat{f}(u)e^{iu \cdot y}H_{a}(u)du = f*\hat{H}_{a}(y).
     \]
  Tomando \(y=0\) e pela simetria de \(\hat{H}_{a},\) chegamos em 
    \[
      \int_{}^{}\hat{f}(y)H_{a}(u)du = f* \hat{H}_{a}(0).
    \]
  Tome \(g(x) = \overline{f(-x)},\) sendo a barra a conjugação complexa de a. Como \(\overline{ab} = \overline{a}\overline{b},\)
 \begin{align*}
   \hat{g}(u) = \int_{}^{}e^{iu \cdot x}\overline{f(-x)}dx &= \overline{\int_{}^{}e^{-i u \cdot x}f(-x)dx}\\
                                                           &= \overline{\int_{}^{}e^{iu \cdot x}f(x)dx} = \overline{\hat{f}(u)}.
 \end{align*}
 Assim, trocando \(f\) por \(f*g\), 
   \[
     \int_{}^{}\widehat{f*g}(u)H_{a}(u)du = f*g*\hat{H}_{a}(0).
   \]
   Note que \(\widehat{f*g}(u) = \hat{f}(u)\hat{g}(u) = |\hat{f}(u)|^{2}\). Pelo \hyperlink{monotone_convergence}{\textit{Teorema da Convergência Monótona},} então, 
     \[
       \lim_{a\to \infty}\int_{}^{}\widehat{f*g}(u)H_{a}(u)du = (2\pi )^{-n}\int_{}^{}|\hat{f}(u)|^{2}du.
     \]
  Como tanto f quanto g são contínuas com suporte compacto, \(f*g\) também é e, consequentemente, 
    \[
      \lim_{a\to \infty}f*g*\hat{H}_{a}(0) = f*g(0) = \int_{}^{}f(y)g(-y)dy = \int_{}^{}|f(y)|^{2}dy.
    \]
  Portanto, 
    \[
      \Vert f \Vert_{2} = \hat{H}_{a}*f(0).\quad \text{\qedsymbol} 
    \]
 \end{proof*}
\subsection{Wagnysson Moura Luz - Espaços de Banach}
 \begin{def*}
   X é um espaço normado linear sobre um corpo F se existe um mapa \(x\mapsto \Vert x \Vert\) tal que 
  \begin{itemize}
    \item[1)] \(\Vert x \Vert \geq 0\) e \(\Vert x \Vert = 0\) se, e somente se \(x=\)
      \item[2)] \(\Vert \alpha x \Vert = |\alpha |\Vert x \Vert\) para todos \(\alpha \in F\) e \(x\in X\)
        \item[3)] \(\Vert x + y \Vert \leq \Vert x \Vert + \Vert y \Vert. \quad \square\)
  \end{itemize}
 \end{def*}
 \begin{def*}
   Um \textbf{Espaço de Banach} é um espaço linear normado no qual toda sequência de Cauchy converge, ou seja, que é completo. \(\square\) 
 \end{def*}
\begin{def*}
  Um \textbf{mapa linear} é um mapa L de um espaço linear normado X par aum espaço linear normado Y satisfazendo 
    \[
      L(x+y) = L(x) + L(y)\quad L(\alpha x) = \alpha L(x),\quad \forall x, y\in X, \alpha \in F
    \]
  Escrevemos \(L(x) = Lx. \quad \square\)
\end{def*}
 \begin{def*}
   Um mapa linear \(f:X\rightarrow \mathbb{R}\) é um \textbf{funcional linear real.} Se ele vai de X para \(\mathbb{C}\) no lugar, é um \textbf{funcional linear complexo}. Diremos que o funcional linear é \textbf{limitado } se 
     \[
       \Vert f \Vert = \sup_{}\{|f(x):x\in X, \Vert x \Vert \leq 1\} < \infty. \quad \square
     \]
 \end{def*}
 \begin{prop*}
   As seguintes são equivalentes: 
  \begin{itemize}
    \item[1)] O funcional linear é limitado
    \item[2)] O funcional linear é contínuo
    \item[1)] O funcional linear é contínuo em 0.
  \end{itemize}
 \end{prop*}
 \begin{proof*}
   Como 
     \[
       |f(x) - f(y)| = |f(x-y)|\leq \Vert f \Vert\Vert x-y \Vert,
     \]
     1 implica 2. A implicação de 2 em 3 é automática. Para mostrar que 3 implica 1, suponha que f não é limitada. Existe \((x_{n})\in X\) tal que \(\Vert x_{n} \Vert = 1\) para cada n, mas \(|f(x_{n})|\to \infty.\) Se colocarmos \(y_{n} = \frac{x_{n}}{|f(x_{n})|},\) então \(y_{n}\to 0\), mas 
     \(|f(y_{n})| = 1\not\to 0\), contradizendo (3). Portanto, provamos as equivalências. \qedsymbol
 \end{proof*}
  Nosso objetivo é mostrar que existem muitos funcionais lineares. Utilizaremos o axioma da escolha/Lema de Zorn: Se tivermos um conjunto ordenado parcialmente \((Y, \leq )\), um \textbf{subconjunto linearmente ordenado} \(X\subseteq Y\) é tal que, se \(x, y\in X\), então ou \(x\leq y\), ou \(y \leq x\), ou ambos, é verdadeiro. 
Um subconjunto do tipo tem uma \textbf{cota superior} se existe \(z\in Y\) tal que \(x\leq z\) para todo x em X.  Um elemento z de Y é \textbf{maximal} se \(z \leq y\) para \(y\in Y\) implica que \(y = z.\)
  \hypertarget{zornn}{
    \begin{lemma*}[Lema de Zornn]
     Se Y é um conjunto parcialmente ordenado e todo subconjunto linearmente ordenado de Y tem cota superior, então Y tem um elemento maximal.
   \end{lemma*}
  }
  Veremos agora o Teorema de Hahn-Banach Real. 
  \hypertarget{real_hahn_banach}{
    \begin{theorem*}[Hahn-Banach Real]
      Se M é um subespaço de um espaço linear normado X e f é um funcional linear real limitado em M, então f pode ser estendido para um funcional linear limitado F em X, tal que \(\Vert F \Vert = \Vert f \Vert.\) 
   \end{theorem*}
  }
  Dizer que F é uma extensão de f significa que o domínio de F contém o de f e \(F(x) = f(x)\) para todo x no domínio de f.
 \begin{proof*}
   Suponha que \(\Vert f \Vert = 0\). Então, para \(F\equiv 0\) completa o teorema. Suponha, então, que \(\Vert f \Vert\neq 0\). Multiplicar por uma constante faz com que \(\Vert f \Vert = 1\) possa ser assumido. 

   Agora, escolha \(x_{0}\in X\setminus{M}\) e tome \(M_1\) como um espaço vetorial gerado por M e \(x_{0}\), de forma que \(M_1\) consiste de todos os vetores da forma \(x+\lambda x_{0}\), em que \(x\in M\) e \(\lambda \) é real. Para todos \(x, y\in M\), 
     \[
       f(x) - f(y) = f(x-y) \leq \Vert x-y \Vert \leq \Vert x-x_{0} \Vert + \Vert y - x_{0} \Vert.
     \]
    Logo,
      \[
        f(x)-\Vert x-x_{0} \Vert \leq f(y) + \Vert y-x_{0} \Vert,\quad \forall x, y\in M.
      \]
    Escolha \(\alpha \in \mathbb{R}\) tal que 
      \[
        f(x) - \Vert x-x_{0} \Vert \leq \alpha \leq f(y) + \Vert y-x_{0} \Vert,\quad \forall x, y\in M.
      \]
    Defina \(f_1(x+\lambda x_{0}) = f(x) + \lambda \alpha \), a qual é uma extensão de f para \(M_1\). Dado \(x\in M \) e \(\lambda \in \mathbb{R},\) nossa escolha de \(\alpha\) garante que ou \(f(x) - \Vert x - x_{0} \Vert \leq \alpha \), ou \(f(x) - a \leq \Vert x-x_{0} \Vert\) e que 
    \(\alpha \leq f(x) + \Vert x-x_{0} \Vert\), ou \(f(x) - \alpha \geq -\Vert x-x_{0} \Vert\). Assim,
      \[
        |f(x) - \alpha | \leq \Vert x-x_{0} \Vert.
      \] 
    Trocando x por \(-\frac{x}{\lambda }\) e multiplicando por \(|\lambda |\), chegamos em 
      \[
        |\lambda ||-\frac{f(x)}{\lambda }-\alpha | \leq |\lambda | \Vert -\frac{x}{\lambda }-x_{0} \Vert,
      \]
    ou 
      \[
        |f_1(x+\lambda x_{0})| = |f(x) + \lambda \alpha |\leq \Vert x + \lambda x_{0} \Vert,
      \]
    que é o que queríamos provar. Agora, para estabelecer que isso vale para uma extensão a todo X, tome \(\mathcal{F}\) como a coleção de todas as extensões lineares F de f satisfazendo \(\Vert F \Vert\leq 1\), que é parcialmente ordenada pela inclusão. 
    Como a união de uma família crescente de subespaços de X é um subespaço, a união de uma subfamília linearmente ordenada de \(\mathcal{F}\) continua em \(\mathcal{F}\) e, pelo \hyperlink{zornn}{\textit{Lema de Zorn}}, \(\mathcal{F}\) tem um elemento maximal. 
    Digamos que ele é \(F_1\). Então, pela construção, se o domínio de \(F_1\) não for todo o X, existe uma extensão maior pelo mesmo processo de antes. Contradição. Portanto, \(F_1\) prova o Teorema. \qedsymbol
 \end{proof*}
 Lembremos que \(f(x) = u(x) + iv(x)\) para qualquer função complexa.
 \hypertarget{complex_hahn_banach}{
   \begin{theorem*}[Hahn-Banach Complexo]
    Se M é um subespaço de um espaço linear normado X e f é um funcional linear complexo limitado em M, então f pode ser estendida a um funcional linear limitado F em X tal que \(\Vert F \Vert = \Vert f \Vert\).
  \end{theorem*}
 }
 \begin{proof*}
   Sem perda de generalidade, assuma que \(\Vert f \Vert = 1.\) tome \(u= \mathrm{Re}(f)\) e note que \(|u(x)|\leq |f(x)|\leq \Vert x \Vert.\) Utilizando a \hyperlink{real_hahn_banach}{\textit{Versão Real do Teorema de Hahn-Banach}}, encontramos uma extensão U de u para X tal que \(\Vert U \Vert \leq 1.\) Tome 
     \[
       F(x) = U(x) - iU(ix).
     \]
    Resta provarmos que a norma de F é no máximo 1. Para isso, fixado x, escreva \(F(x) = re^{i\theta }.\) Então, 
      \[
        |F(x)| = r = e^{-i\theta }F(x) = F(e^{-i\theta }x).
      \]
    Como este número é real e não-negativo,
      \[
        |F(x)| = U(e^{-i\theta }x)\leq \Vert U \Vert\Vert e^{-i\theta }x \Vert\leq \Vert x \Vert.
      \]
    Portanto, como isto vale para todo x, \(\Vert F \Vert \leq 1\). \qedsymbol
 \end{proof*}
  Uma aplicação de Hahn-Banach é que, dado um subespaço M e um elemento \(x_{0}\) fora de M tal que 
    \[
      \inf_{x\in M}\Vert x-x_{0} \Vert > 0,
    \]
  podemos definir \(f(x+\lambda x_{0}) = \lambda \) para \(x\in M\) e estender isto para todo o X. Assim, \(f\) será 0 em M, mas não-nula em \(x_{0}\). Outra aplicação é que, fixado \(x_{0}\neq 0\), tome 
    \[
      f(\lambda x_{0}) = \lambda \Vert x_{0} \Vert.
    \]
  Estendendo f para todo o X, encontramos um funcional linear f tal que \(f(x_{0}) = \Vert x_{0} \Vert \) e \(\Vert f \Vert = 1\).
    Veremos, agora, o Teorema Categórico de Baire. 
  \hypertarget{baire_category}{
    \begin{theorem*}[Teorema Categórico de Baire]
     Seja X um espaço métrico completo. 
    \begin{itemize}
      \item[1)] Se \(G_{n}\) são todos abertos densos em X, então \(\bigcap_{n}^{}G_{n}\) é denso em X;
        \item[2)] X não pode ser escrito como a união contável de conjuntos densos em nenhum lugar.
    \end{itemize}
   \end{theorem*}
  }
 \begin{proof*}
   Mostraremos primeiro que (1) implica (2). Suponha que possamos escrever X como a união contável de conjuntos nunca densos, ou seja, \(X = \bigcup_{n}^{}E_{n}\), em que \((\overline{E}_{n})^{\circ } = \emptyset .\) Coloque \(F_{n} = \overline{E}_{n}\), o qual é fechado, tal que 
   \(F_{n}^{\circ } = \emptyset \) e \(X = \bigcup_{n}^{}F_{n}.\) Escreva \(G_{n} = F_{n}^{\complement}\), que é um aberto. Como \(F_{n}^{\circ } = \emptyset \), vale que \(\overline{G}_{n} = X.\) Começando como \(X = \bigcup_{n}^{}F_{n}\) e tomando complementos, vimos que \(\emptyset = \bigcap_{n}^{}G_{n}\) contradiz (1).

   Agora, precisamos apenas provar o item 1. Suponha que \(G_1, G_2, \dotsc \) são conjuntos abertos e densos em X. Seja H um aberto não-vazio de X e 
     \[
       B(z, r) = \{y\in X: d(z, y) < r\},
     \]
em que d é a métrica de X. Como \(G_{1}\) é denso em X, \(H\cap G_1\) é não-vazio e aberto, tal que podemos achar \(x_1\) e \(r_1\) para os quais \(\overline{B(x_1, r_1)}\subseteq H\cap G_1\) e \(0 < r_1 < 1.\) Suponha que escolhemos 
 \(x_{n-1}\) e \(r_{n-1}\) para algum \(n\geq 2\). Como \(G_{n}\) é denso, \(G_{n}\cap B(x_{n-1}, r_{n-1})\) é aberto e não-vazio, então existem \(x_{n}\) e \(r_{n}\) tais que \(\overline{B(x_{n}, r_{n})}\subseteq G_{n}\cap B(x_{n-1}, r_{n-1})\) e \(0< r_{n} < 2^{-n}.\) Continuando assim, encontramos uma sequência \(x_{n}\) em X. 
 Note que, se \(m, n > N\), então \(x_{m}\) e \(x_{n}\) ficam ambos em \(B(x_{N}, r_{N})\), tal que \(d(x_{m}, x_{n}) < 2r_{N} < 2^{-N+1}.\) Portanto, \(x_{n}\) é uma sequência de Cauchy e, por X ser completo, \(x_{n}\) deve convergir a um ponto \(x\in X\). 

  Resta provar que \(x\in H \cap (\bigcap_{n}^{}G_{n})\). Com efeito, como \(x_{n}\in \overline{B(x_{N}, r_{N})}\) se \(n > N\), então \(x\in \overline{B(X_{N}, r_{N})} \) e, consequentemente, em cada \(G_{N},\) o que equivale a \(x\in \bigcap_{n}^{}G_{n}\). Além disso, 
    \[
      x\in \overline{B(x_{n}, r_{n})}\subseteq B(x_{n-1}, r_{n-1})\subseteq \dotsc \subseteq B(x_1, r_1)\subseteq H.
    \]
  Portanto, \(x\in H\cap (\bigcap_{n}^{}G_{n})\). \qedsymbol
 \end{proof*}
\begin{def*}
  Um conjunto \(A\subseteq X\) é dito \textbf{meager} (``mesquinho?"), ou de \textbf{primeira categoria}, se ele é a união enumerável de conjuntos nunca densos. Caso contrário, digamos que ele é de \textbf{segunda categoria}. \(\square\)
\end{def*}
  Uma aplicação do Teorema Categórico de Baire é o Teorema de Banach-Steinhaus, também conhecido como Teorema da Limitação Uniforme.
  \hypertarget{banach_steinhauss}{
    \begin{theorem*}[Teorema de Banach-Steinhaus]
      Suponha que X é um espaço de Banach e que Y é um espaço linear normado. Seja A um conjunto de índices e \(\{L_{\alpha }: \alpha \in A \}\) uma coleção de mapas lineares limitados de X para Y. Então, ou existe 
um número real positivo \(M<\infty\) tal que \(\Vert L_{\alpha } \Vert\leq M\) para todo \(\alpha \in A\), ou \(\sup_{\alpha }\Vert L_{\alpha }x \Vert = \infty\) para algum x.
   \end{theorem*}
  }
 \begin{proof*}
   Sejam \(\ell (x) = \sup_{\alpha \in A}\Vert L_{\alpha }x \Vert\) e \(G_{n} = \{x:\ell (x) > n\}\). O mapa \(x\mapsto \Vert L_{\alpha }x \Vert\) é uma função contínua para cada \(\alpha \), já que \(l_{\alpha }\) é um funcional linear limitado. Isto implica que, para cada \(\alpha \), o conjunto 
   \(\{x: \Vert L_{\alpha }x > n\Vert\}\) é aberto. Como \(x\in G_{n}\) se, e somente se \(\Vert L_{\alpha }x \Vert > n\) para algum \(\alpha \in A\), concluímos que \(G_{n}\) é a união de conjuntos abertos e, consequentemente, um aberto em si. 

   Agora, suponha que existe N tal que \(G_{N}\) não é denso em X. Então, existem \(x_{0}\) e r tais que \(\overline{B(x_{0}, r)}\cap G_{N} = \emptyset \), o que equivale a dizer que, se \(\Vert x-x_{0} \Vert\leq r\), então \(\Vert L_{\alpha }(x) \Vert\leq N\) para todo \(\alpha \in A\). 
  Se \(\Vert y \Vert, r,\) temos \(y=(x_{0}+y) - x_{0},\) tal que \(\Vert (x_{0}+y) - x_{0} \Vert = \Vert y \Vert\leq r\) e, assim, \(\Vert L_{\alpha }(x_{0} + y) \Vert\leq N\) para todo \(\alpha \). Além disto, \(\Vert x_{0}-x_{0} \Vert = 0 \leq r\), o que implica que \(\Vert L_{\alpha }(x_{0}) \Vert\leq N\) para todo \(\alpha \). 
  Concluímos, então, que, se \(\Vert y \Vert\leq r\) e \(\alpha \in A\),
    \[
      \Vert L_{\alpha }y \Vert = \Vert L_{\alpha }((x_{0}+y) - x_{0}) \Vert\leq \Vert L_{\alpha }(x_{0}-y) \Vert + \Vert L_{\alpha }x_{0} \Vert\leq 2N.
    \]
  Logo, \(\sup_{\alpha \in A}\Vert L_{\alpha } \Vert\leq M\) com \(M = \frac{2N}{r}\).

  A outra possibilidade é que todo \(G_{n}\) seja denso em X. Neste, \(\bigcap_{n}^{}G_{n}\) é denso em X, mas \(\ell (x) = \infty\) para todo \(x\in \bigcap_{n}^{}G_{n}\). Portanto, a conclusão do Teorema permanece. \qedsymbol
 \end{proof*}
  Outro resultado fundamental destes estudos é o Teorema da Aplicação Aberta. Aqui, é importante que L seja sobrejetora. 
 \begin{def*}
   Uma aplicação \(L:X\rightarrow Y\) é \textbf{aberto} se \(L(U)\) é aberto em Y sempre que U for aberto em X. \(\square\)
 \end{def*}
 Para um conjunto mensurável A, colocamos \(L(A) = \{Lx: x \in A\}\).
  \hypertarget{open_mapping}{
    \begin{theorem*}[Teorema da Aplicação Aberta]
     Sejam X e Y espaços de Banach. Uma aplicação linear limitada L de X sobre Y, ou seja, \(L:X\rightarrow Y\) sobrejetora, é um mapa aberto.
   \end{theorem*}
  }
 \begin{proof*}
   Precisamos provar que se \(B(x, r)\subseteq X\), então \(L(B(x, r))\) contém uma bola em Y. Como L é sobrejetora, \(Y = \bigcup_{n=1}^{\infty}L(B(0, n))\). Pelo \hyperlink{baire_category}{\textit{Teorema Categórico de Baire}}, 
pelo menos um desses conjuntos \(L(B(0, n))\) não pode ser nunca-denso. Como L é linear, \(L(B(0, 1))\) é um conjunto que não pode ser denso nunca, tal que existem \(y_{0}\) e \(r\) tais que \(B(y_{0}, 4)\subseteq \overline{L(B(0, 1))}\).  

  Agora, escolha \(y_1\in l(B(0, 1))\) tal que \(\Vert y_1 - y_{0} \Vert < 2r\) e seja \(z_1\in B(0, 1)\) tal que \(y_1 = Lz_1.\) Então, \(B(y_1, 2r)\subseteq B(y_{0}, 4r)\subseteq \overline{L(B(0, 1))}.\) Assim, se \(\Vert y \Vert < 2r,\) então 
  \(y + y_1\in B(y_1, 2r)\), tal que 
    \[
      y= -Lz_1 + (y + y_1)\in \overline{L(-z_1 + B(0, 1))}.
    \]
  Como \(z_1\in B(0, 1)\), segue que \(-z_1 + B(0, 1)\subseteq B(0, 2)\), implicando em 
    \[
      y\in \overline{L(-z_1 + B(0, 1))}\subseteq \overline{L(B(0, 2))}.
    \]
  Por linearidade de L, se \(\Vert y \Vert < r\), então \(y\in \overline{L(B(0, 1))}.\) Segue por linearidade que, se \(\Vert y \Vert < r2^{-n},\) então \(y\in \overline{L(B(0, 2^{-n}))}.\) Equivalentemente, 
se \(\Vert y \Vert < r2^{-n}\) e \(\varepsilon  > 0\), então existe x tal que \(\Vert x \Vert < 2^{-n} \) e \(\Vert y - Lx \Vert < \varepsilon .\)

  Agora, suponha que \(\Vert y \Vert < \frac{r}{2}.\) Pelo primeiro passo, com \(\varepsilon  = \frac{r}{4}, \) podemos achar \(x_1\in B \biggl(0, \frac{1}{2}\biggr)\) tal que \(\Vert y - Lx_1 \Vert < \frac{r}{4}.\) Supondo que escolhemos \(x_1, \dotsc , x_{n-1}\) satisfazendo 
    \[
      \biggl\Vert y - \sum\limits_{j=1}^{n-1}Lx_{j} \biggr\Vert < r2^{-n}.
    \]
  Seja \(\varepsilon  = r2^{-(n+1)}.\) Pela primeira parte, podemos achar \(x_{n}\) tal que \(\Vert x_{n} \Vert < 2^{-n}\) e 
    \[
      \biggl\Vert y - \sum\limits_{j=1}^{n}Lx_{j} \biggr\Vert = \biggl\Vert \biggl(y - \sum\limits_{j=1}^{n-1}Lx_{j}\biggr) - Lx_{n} \biggr\Vert < r2^{-(n+1)}.
    \]
  Por indução, construímos uma sequência \(\{x_{j}\}\) com estes x's. Tome \(w_{n} = \sum\limits_{j=1}^{n}x_{j}.\) Como \(\Vert x_{j} \Vert < 2^{-j},\) vale que \(w_{n}\) é de Cauchy. Por completude de X, \(w_{n}\) converge para algum elemento, digamos x. Mas, assim, \(\Vert x \Vert < 
  \sum\limits_{j=1}^{\infty} 2^{-j} = 1\) e, por continuidade de L, \(y = Lx\). Em outras palavras, se \(y\in B \biggl(0, \frac{r}{2}\biggr)\), então \(y\in L(B(0, 1))\). Por linearidade de L, esta bola pode ser passada para uma genérica centrada em zero, \textit{i.e.}, \(L(B(0, r))\), a qual contém 
  uma bola centrada em 0 em Y. Portanto, \(L(B(x, r))\) é aberto e L é um mapa aberto. \qedsymbol
 \end{proof*}
  Quando \(Y = F\) (um corpo), então \(\mathcal{L}\coloneqq \{L:X\rightarrow Y: L \text{ é linear e limitado}\}\) é o conjunto dos funcionais lineares limitados de X. Neste caso, escrevemos \(X^{*}\) no lugar de \(\mathcal{L}\) e chamamos \(X^{*}\) de \textbf{espaço dual} de X.
\newpage

\section{Aula 12 - 01/02/2024}
\subsection{Motivações}
 \begin{itemize}
   \item a
 \end{itemize}
\subsection{Laura V. B. - Espaços de Hilbert}
 \begin{def*}
   Seja H um espaço vetorial sobre \(\mathbb{F}\in\{\mathbb{C}, \mathbb{R}\}\). Dizemos que H é um \textbf{espaço de produto interno} se existir um mapa
   \(\left< \cdot , \cdot  \right>:H \times H\rightarrow \mathbb{F}\) tal que 
  \begin{itemize}
    \item[1)] \(\left< y, x \right> = \overline{\left< x, y \right>}\) para todos x, y em H; 
    \item[2)] \(\left< x + y, z \right> = \left< x, z \right> + \left< y, z \right>\) para todos x, y, z em H;
    \item[3)] \(\left< \alpha x, y \right> = \alpha \left< x, y \right>\) para todos \(x, y\in H\) e \(\alpha \in \mathbb{F};\)
    \item[4)] \(\left< x, x \right> \geq 0\) para todo x em H; 
    \item[5)] \(\left< x, x \right> = 0\) se, e somente se, \(x=0.\quad \square\)
  \end{itemize}
 \end{def*}
  Definimos \(\Vert x \Vert = \left< x, x \right>^{\frac{1}{2}},\) tal que \(\left< x, x \right> = \Vert x \Vert^{2}.\) Pelas definições dadas, \(\left< 0, y \right> = 0\) e 
 \(\left< x, \alpha y \right> = \overline{\alpha }\left< x, y \right>.\) 

 \hypertarget{cauchy_schwarz}{
  \begin{theorem*}[Desigualdade de Cauchy Schwarz]
    Para todos \(x, y\in H\), temos 
      \[
        |\left< x, y \right>|\leq \Vert x \Vert \Vert y \Vert.
      \]  
  \end{theorem*}}

\subsection{Ana Júlia G. A. - Séries de Fourier}
  página 214 + pedir slides
\newpage

\section{Aula 13 - 05/02/2024}
\subsection{Motivações} 
 \begin{itemize}
   \item b
 \end{itemize}
\subsection{Felipe W. S. C. - Representação de Riesz}
\subsection{Claudinei C. J. - Topologia}
\newpage 

\section{Aula 13 - 05/02/2024}
\subsection{Motivações} 
 \begin{itemize}
   \item c
 \end{itemize}
\subsection{Hugo O. B. - Espaços de Sobolev}
\subsection{Renan W. - Distribuições}
\subsection{Kalel B. - Teoria Espectral}
\end{document}

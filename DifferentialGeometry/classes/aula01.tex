\documentclass[../differential_geometry.tex]{subfiles}
\begin{document}
\section{Aula 01 - 24 de Fevereiro, 2026}
\subsection{Motivações}
\begin{itemize}
	\item Curvatura;
	\item Teorema dos Quatro Vértices.
\end{itemize}
\subsection{Introdução à Geometria Diferencial}
Quais são as diferenças entre a reta e a circunferência? Existem várias, mas a diferença que nos interessa aqui é que \textit{a reta é plana}, enquanto que \textit{a circunferência é arredondada}. Será que é possível medir, de alguma forma, o quanto uma certa curva é plana ou arredondada em cada um de seus pontos, tal que definimos uma função que localmente fornece a informação desejada em cada ponto dela?

A ideia que exploraremos é que sim, tal função pode ser definida e usualmente é denotada por \(\kappa (p)\), onde p é um ponto numa curva qualquer; intuitivamente, a função deve ser nula (\(\kappa \equiv 0\)) em todos os pontos da reta, constante para o círculo (\(\kappa = \mathrm{cte}\neq 0\)) e, considerando os círculos tangentes a uma reta em um dado ponto, podemos lançar uma hipótese de que, se r é o raio do círculo resultante quando a tangente é traçada, então
\[
	\kappa = \frac{1}{r}.
\]
Por conta do que ela representa, a função \(\kappa \) é chamada \textbf{curvatura} da curva, e \textit{mede quanto a curva se afasta da sua reta tangente em cada um de seus pontos}!

Para introduzir algumas nomenclaturas e conceitos da ontologia permeando a geometria diferencial, vamos considerar o exemplo de uma elipse:
\[
	\alpha (u) = (a \cos^{}{u}, b \sin^{}{u}),\quad u\in [0,2\pi ],\; 0<b<a
\]
e suponhamos que a curvatura \(\kappa \) é uma função diferenciável. Nos pontos de máximo e mínimo horizontais da elipse, denotados por A e C, e nos máximo e mínimo verticais, denotados por B e D, a função \(\kappa'(u)\) vale 0, permitindo definirmos um conceito chamado \textbf{vértice da curva}; sobre eles, existe um teorema muito importante, conhecido como \textit{\hypertarget{four_vertices}{Teorema dos Quatro Vértices}}, que pode ser enunciado como:
\begin{quote}
	``Qualquer curva \textit{fechada}, \textit{simples} e \textit{convexa} possui pelo menos 4 vértices.'';
\end{quote}
claro, pode ser que algum dos termos acima sejam estranhos à formalização matemática por agora, mas temos pelo menos uma noção intuitivo do que cada coisa quer dizer, então já temos uma apreciação inicial sobre a potência desse resultado.
Além disso, esse teorema mostra uma possibilidade muito boa: conceitos locais, como a curvatura, podem ser usados para deduzir propriedades globais, como o número de vértices!

Em vista da última ponderação, essa introdução é finalizada com uma pergunta para o estudante, que servirá como guia ao longo do curso: \textbf{quais são as propriedades globais de curvas que podemos deduzir a partir de conceitos locais?}

\end{document}

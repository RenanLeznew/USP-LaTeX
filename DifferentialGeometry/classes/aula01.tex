\documentclass[../differential_geometry.tex]{subfiles}
\begin{document}
\section{Aula 01 - 24 de Fevereiro, 2026}
\subsection{Motivações}
\begin{itemize}
	\item Curvatura;
	\item Teorema dos Quatro Vértices.
\end{itemize}
\subsection{Introdução à Geometria Diferencial}
Quais são as diferenças entre a reta e a circunferência? Existem várias, mas a diferença que nos interessa aqui é que \textit{a reta é plana}, enquanto que \textit{a circunferência é arredondada}. Será que é possível medir, de alguma forma, o quanto uma certa curva é plana ou arredondada em cada um de seus pontos, tal que definimos uma função que localmente fornece a informação desejada em cada ponto dela?

A ideia que exploraremos é que sim, tal função pode ser definida e usualmente é denotada por \(\kappa (p)\), onde p é um ponto numa curva qualquer; intuitivamente, a função deve ser nula (\(\kappa \equiv 0\)) em todos os pontos da reta, constante para o círculo (\(\kappa = \mathrm{cte}\neq 0\)) e, considerando os círculos tangentes a uma reta em um dado ponto, podemos lançar uma hipótese de que, se r é o raio do círculo resultante quando a tangente é traçada, então
\[
	\kappa = \frac{1}{r}.
\]
\begin{figure}[H]
	\begin{center}
		\includegraphics[height=0.5\textheight, width=0.5\textwidth, keepaspectratio]{./Images/tangents_circle_01.png}
	\end{center}
	\caption{As tangentes em cada ponto traçam um círculo de raio inversamente proporcional à curvatura.}
\end{figure}

Por conta do que ela representa, a função \(\kappa \) é chamada \textbf{curvatura} da curva, e \textit{mede quanto a curva se afasta da sua reta tangente em cada um de seus pontos}!

Para introduzir algumas nomenclaturas e conceitos da ontologia permeando a geometria diferencial, vamos considerar o exemplo de uma elipse:
\[
	\alpha (u) = (a \cos^{}{u}, b \sin^{}{u}),\quad u\in [0,2\pi ],\; 0<b<a
\]
e suponhamos que a curvatura \(\kappa \) é uma função diferenciável.
\begin{figure}[H]
	\begin{center}
		\includegraphics[height=0.5\textheight, width=0.5\textwidth, keepaspectratio]{./Images/ellipse_01.png}
	\end{center}
	\caption{Elipse descrita por \(\alpha (u)\) com os máximos e mínimos denotados.}
\end{figure}

Nos pontos de máximo e mínimo horizontais da elipse, denotados por A e C, e nos máximo e mínimo verticais, denotados por B e D, a função \(\kappa'(u)\) vale 0, permitindo definirmos um conceito chamado \textbf{vértice da curva}; sobre eles, existe um teorema muito importante, conhecido como \textit{\hyperlink{four_vertices}{Teorema dos Quatro Vértices}}, que pode ser enunciado como:
\begin{quote}
	``Qualquer curva \textit{fechada}, \textit{simples} e \textit{convexa} possui pelo menos 4 vértices.'';
\end{quote}
claro, pode ser que algum dos termos acima sejam estranhos à formalização matemática por agora, mas temos pelo menos uma noção intuitivo do que cada coisa quer dizer, então já temos uma apreciação inicial sobre a potência desse resultado.
Além disso, esse teorema mostra uma possibilidade muito boa: conceitos locais, como a curvatura, podem ser usados para deduzir propriedades globais, como o número de vértices!

Em vista da última ponderação, surge uma pergunta para o estudante, que servirá como guia ao longo do curso: \textbf{quais são as propriedades globais de curvas que podemos deduzir a partir de conceitos locais?}

\subsection{Superfícies no Espaço}
Para tratarmos de curvatura em curvas planas, vamos estabelecer uma convenção sobre o sinal dela, conforme a imagem abaixo.
\begin{figure}[H]
	\begin{center}
		\includegraphics[height=\textheight, width=\textwidth, keepaspectratio]{./Images/curvature_sign_02.png}
	\end{center}
	\caption{A curva à esquerda mora localmente no semiplano determinado pela reta tangente que \textbf{não contém} o vetor normal ao ponto u, enquanto que a da direita mora localmente no semiplano determinado pela reta tangente e que \textbf{contém} o vetor normal ao ponto.}
\end{figure}
Quando a curva cai no caso um e a curva não fica contida no semiplano que é determinado pela tangente e que contém a normal (se traçarmos o plano formado pela tangente que contém a normal, a curva no ficaria ``abaixo dele'', sem nenhum pedaço dela contido), a curvatura será \textbf{negativa}; no segundo caso, onde a curva mora \textit{localmente} dentro do simplano determinado pela tangente e contendo o vetor normal ao ponto (se traçarmos o semiplano na tangente, um pedaço da curva vai estar dentro dele), então a curvatura será \textbf{positiva}.

Feita a convenção acima, outro problema que abordaremos consiste em considerar as segunintes superfícies em um dado ponto p no espaço \(\mathbb{R}^{3}\):
\begin{figure}[H]
	\begin{center}
		\includegraphics[height=0.7\textheight, width=0.7\textwidth, keepaspectratio]{./Images/different_surfaces_02.png}
	\end{center}
\end{figure}
As formas das superfícies são claramente distintas, mas como poderíamos explicar isso matematicamente?

Mais ainda, aproveitando as imagens, se cortarmos a superfície com planos passando pelo ponto p, gerados por um vetor tangente unitário v e o vetor normal \(N(p)\), então o braço da superfície neste plano cortando o ponto p é uma curva plana que tem curvatura \(\kappa_{v}(p)\); assim, variando v no círculo unitário no plano tangente a M no ponto p, \(T_{p}M\), temos
\[
	\kappa_{v_{\mathrm{min}}}(p)\leq \kappa_{v}(p)\leq \kappa_{v_{\mathrm{max}}}(p).
\]
A partir disso, definimos:
\begin{def*}
	Para uma superfície M, a sua \textbf{Curvatura Gaussiana no ponto p} é definida pelo produto
	\[
		\kappa (p)\coloneqq \kappa _{v_{\mathrm{min}}}(p)\kappa_{v_{\mathrm{max}}}(p). \; \square
	\]
\end{def*}

O sinal da curvatura gaussiana nos fornece uma informação sobre a forma local da superfície:
\begin{itemize}
	\item Se \(\kappa (p)>0\), a superfície tem localmente a forma de um ovo;
	\item Se \(\kappa (p)<0\), a superfície tem localmente a forma de uma sela de cavalo; e
	\item Se \(\kappa (p)=0\), a superfície tem localmente a forma plana.
\end{itemize}
No caso de uma esfera de raio r, o valor de sua curvatura em qualquer ponto p é \(\kappa (p)=\frac{1}{r^{2}}\). Essa caracterização leva, naturalmente, à próxima pergunta que gostaríamos de responder ao longo do curso: se \(\kappa \equiv 0\), então a superfície é um plano? E se \(\kappa \equiv \frac{1}{r^{2}}\), ela é necessariamente uma esfera?
Obsrerve que, para a esfera \(\mathbb{S}^{2}(r)\) de raio r, temos
\begin{align*}
	\iint_{\mathbb{S}^{2}(r)}\kappa \mathrm{d}A = \frac{1}{r^{2}}\iint_{\mathbb{S}^{2}(r)} \mathrm{d}A & = \frac{1}{r^{2}} \mathrm{Area}(\mathbb{S}^{2}(r)) \\
	                                                                                                   & = \frac{1}{r^{2}}4\pi r^{2}                        \\
	                                                                                                   & = 4\pi.
\end{align*}
Na verdade, esse resultado é um caso particular do \hyperlink{gauss_bonnet}{\textit{Teorema de Gauss-Bonnet}}:
\begin{theorem*}
	Para uma superfície fechada M em \(\mathbb{R}^{3}\), temos
	\[
		\iint_{M}\kappa (p)\mathrm{d}A = 2\pi \chi (M).
	\]
\end{theorem*}
A função \(\chi (M)\) é um conceito topológico chamado \textbf{característica de Euler}, de forma que o teorema de Gauss-Bonnet relaciona um conceito geométrico, a curvatura de uma superfície, com um topológico, a característica de Euler! Para o caso da esfera, \(\chi (\mathbb{S}^{2})=2\), e por isso mesmo
\[
	\iint_{\mathbb{S}^{2}(r)}\kappa (p)\mathrm{d}A = 4\pi = 2\pi \chi (M).
\]

Em suma, nessa disciplina, planejamos:
\begin{itemize}
	\item Definir a curvatura de uma curva e de uma superfície;
	\item Estudar a geometria de curvas planas e espaciais, assim como a geometria de superfícies;
	\item Fazer cálculos sobre curvas e superfícies;
	\item Obter propriedades globais a partir de conceitos locais; e
	\item Procurar propriedades intrínsecas de suprefícies, permitindo-nos diferenciá-las.
\end{itemize}

\end{document}

\documentclass[../category_theory.tex]{subfiles}
\begin{document}
\section{Class 22 - November 7th, 2024}
\subsection{Motivations}
\begin{itemize}
	\item Continuation of Inverse Limits;
	\item I-adic completion.
\end{itemize}
\subsection{Inverse Limits}
Since last class was dedicated to defining am examplifying inverse limits, today well look at some. Before, however, a quick remark is that an inverse system
\[
	F:I\rightarrow \mathcal{C}
\]
can be seen as a directed system by:
\[
	F^{\mathrm{op}}:I^{\mathrm{op}}\rightarrow \mathcal{C}
\]
\begin{example}
	Let p be a prime number, \(I=\mathbb{N}\). For  \(i\leq j\) in I, define
	\begin{align*}
		\varphi_{i}^{j}: & \mathbb{Z}/p^{j}\mathbb{Z}\rightarrow \mathbb{Z}/p ^{i}\mathbb{Z} \\
		                 & a+p ^{j}\mathbb{Z}\mapsto a+p ^{i}\mathbb{Z}.
	\end{align*}
	Then, \(\{\mathbb{Z}/p ^{i}\mathbb{Z}, \varphi_{j}^{i}\}\) defines an inverse system in Ab (which, as proven last class, has inverse limit). With this, we denote
	\[
		\mathbb{Z}_{p}\coloneqq \varprojlim_{}\frac{\mathbb{Z}}{p ^{i}\mathbb{Z}}.
	\]
	Now we have a collection of four different \(\mathbb{Z}\)'s with a p subscript!
	\begin{align*}
		 & \mathbb{Z}/p = \mathbb{Z}/p\mathbb{Z}                                                                                   \\
		 & \mathbb{Z}_{(p)}=\biggl\{\frac{a}{b}\in \mathbb{Q}: p \not|\; b\biggr\}                                                 \\
		 & \mathbb{Z}_{p ^{\infty}}\coloneqq \frac{\mathbb{Z}\biggl[\frac{1}{p}\biggr]}{\mathbb{Z}}=\varinjlim_{}\mathbb{Z}/p ^{n} \\
		 & \mathbb{Z}/p \hookrightarrow \mathbb{Z}/p ^{2} \hookrightarrow \mathbb{Z}/p ^{3}                                        \\
		 & \overline{1}\longmapsto \overline{p}\longmapsto \overline{p}                                                            \\
		 & \mathbb{Z}_{p}\coloneqq \varprojlim_{}\frac{\mathbb{Z}}{p ^{i}\mathbb{Z}} \rightsquigarrow\;\text{p-adic integers}.
	\end{align*}
	A few properties of the p-adic integers, which can be made explict by
	\[
		\mathbb{Z}_{p}=\biggl\{(a_{i})\in \prod\limits_{i\in \mathbb{N}}^{}\mathbb{Z}/p ^{i}\mathbb{Z}: \varphi_{j}^{i}(a_{j})=a_{i}\: \forall i\leq j\biggr\}.
	\]
	are:
	\begin{itemize}
		\item[1)] \(\mathbb{Z}_{p}\) is a domain, \(\mathbb{Z}\hookrightarrow \mathbb{Z}_{p}\) via \(n\mapsto (n)_{i\in \mathbb{N}}\);
		\item[2)] It is a local domain;
		\item[3)] it's field of fractions is \(\mathbb{Q}(\mathbb{Z}_{p})=\mathbb{Z}_{p}\)
	\end{itemize}
	\begin{exr}
		If I is directed (for all i, j in I, there exists a k such that \(i,\; j\leq k\)), \(F_{i}\) are rings, and \(\varphi_{j}^{i}\) are ring homomorphisms, then
		\[
			\varprojlim_{}F_{i}
		\]
		is a ring.
	\end{exr}
	It follows that \(\mathbb{Z}_{p}\) is a ring via the operations defined below for \((a_{i}), (b_{i})\) in \(\mathbb{Z}_{p}\):
	\begin{align*}
		 & (\overline{a}_{i})+(\overline{b}_{i})=(\overline{a}_{i}+\overline{b}_{i}) \\
		 & (\overline{a}_{i})\cdot (\overline{b}_{i})=(\overline{a_{i}b_{i}})        \\
		 & 1 _{\mathbb{Z}_{p}}= (1)_{i\in I}.
	\end{align*}
	which motivates the exercise mentioned above. Moreover, we can write \(x=(\overline{a}_{i})_{i\in \mathbb{N}}\) as
	\[
		x\coloneqq \sum\limits_{i=0}^{\infty}a_{i}p ^{i},\quad 0\leq i\leq p-1.
	\]
	Hence, with notation \(x=\sum\limits_{i=0}^{\infty}a_{i}p ^{i},\: y=\sum\limits_{i=0}^{\infty}b_{i}p ^{i}\), we can put
	\begin{align*}
		 & x+y=\sum\limits_{i=0}^{\infty}c_{i}p ^{i},   \\
		 & xy = \sum\limits_{i=0}^{\infty}d_{i} p ^{i},
	\end{align*}
	in which the process to buid the coefficents \(c_{i}, d_{i}\) is
	\begin{align*}
		 & a_{0}+b_{0}= q_{0}p + c_{0},\quad 0\leq c_{0}\leq p-1     \\
		 & a_{1}+b_{1}+q_{0}= q_{1}p+c_{1},\quad 0\leq c_{1}\leq p-1 \\
		 & a_{2}+b_{2}+q_{1}= q_{2}p+c_{2},\quad 0\leq c_{2}\leq p-1 \\
		 & \vdots
	\end{align*}
	and, using this, one can do analysis on \(\mathbb{Z}_{p}\), defining Cauchy sequences, a topology, and so on, so much so as to make it a complete space.

	Let's explore these ideas a bit more, and make the \textbf{I-adic completion}. Let R be a commutative ring, \(\mathfrak{i} \trianglelefteq R \), and M be an R-modulus. Consider the inverse system \(\{F_{i}, \varphi_{j}^{i}\}_{i\in \mathbb{N}}\), where
	\begin{align*}
		F_{i}= \frac{M}{\mathfrak{i}^{i}M}, \varphi_{j}^{i}: & \frac{M}{\mathfrak{i}^{i}M}\twoheadrightarrow \frac{M}{\mathfrak{i}^{i}M} \\
		                                                     & \overline{m}=m+\mathfrak{i}^{j}M\mapsto \overline{m}=m+\mathfrak{i}^{i}M  \\
		                                                     & \mathfrak{i}^{j}M\subseteq \mathfrak{i}^{i}M.
	\end{align*}
	The inverse limit is denoted by \(\hat{M}\):
	\[
		\hat{M}\coloneqq \varprojlim_{}\frac{M}{\mathfrak{i}^{i}M}=\biggl\{(m_{i})_{i}: \varphi_{j}^{i}(\overline{m}_{j})=\overline{m}_{i}\biggr\}.
	\]
	To see that \(\hat{M}\) is an R-modulus, define for \(x=(\overline{m}_{i})_{i\in \mathbb{N}}\) and a in R,
	\[
		a \cdot x\coloneqq (\overline{am_{i})}_{i\in I}.
	\]
	From this, we have an R-homomorphism through
	\[
		\theta:M\rightarrow \hat{M}, m\mapsto (\overline{m})_{i\in \mathbb{N}}.
	\]
	\begin{exr}
		\(\theta \) is injective iff \(\bigcap_{i\in \mathbb{N}}^{}\mathfrak{i}^{i}M = 0\).
	\end{exr}
	This \(\hat{M}\) is called the \textbf{I-adic completion of M}. If \(M=R\), then \(\hat{M}=\hat{R}\), which in particular means
	\[
		\hat{R}=\varprojlim_{}\frac{R}{\mathfrak{i}^{i}},
	\]
	and that is a commutative ring. With this in mind, we see that \(\hat{M}\) is an \(\hat{R}\)-modulus: if \(m=(\overline{m}_{i})_{i\in \mathbb{N},\; a=(\overline{a}_{i})_{i\in \mathbb{N}}}\), then
	\[
		a \cdot m_{i}\coloneqq (\overline{a_{i}m_{i}})_{i\in \mathbb{N}}.
	\]
	For example, if \(\mathfrak{j}=\mathfrak{i}^{2024}\), then \(\hat{M}_{\mathfrak{j}}\cong \hat{M}_{\mathfrak{i}}\). This is generalized via the following exercise
	\begin{exr}
		Let \(J\subseteq I\), where both J and I are directed sets, and assume that, for any i in I, there is a j in J such that \(i\leq j\). If \(\{F_{i}, \varphi_{j}^{i}\}\) is an inverse system, then
		\[
			\underbrace{\varprojlim_{I}F_{i}}_{\{F_{i}, \varphi_{j}^{i}\}_{i\in I}}\cong \underbrace{\varprojlim_{J}F_{j}}_{\{F_{j}, \varphi_{j}^{j'}\}_{j\in J}}
		\]
		In other words, the next diagram is commutative:
		\begin{center}
			\begin{tikzpicture}[
					observed/.style = {rectangle, thick, text centered, draw, text width = 6em},
					latent/.style = {ellipse, thick, draw, text centered, text width = 6em},
					error/.style ={circle, thick, draw, text centered},
					confounding/.style = {rectangle, thick, text centered, draw, text width = 6em, minimum width = 5.5in},
					outcome/.style = {rectangle, thick, draw, text centered, minimum height = 3.5in, text width = 6em},
				]
				\node(TL) at (-2,-1){\(F_{j}\)};
				\node(BL) at (0,1){\(\varprojlim_{I}F_{j}\)};
				\node(TR) at (2,-1){\(F_{j'}\)};
				\node(BR) at (0,3){\(\varprojlim_{I}F_{i}\)};

				\draw[Arrow](TL)--node[midway, above] {\(\varphi_{j}^{j'}\)}(TR);
				\draw[Arrow, dashed](BR)--node[midway, right] {\(\alpha \)}(BL);
				\draw[Arrow](BL)--node[midway, right] {\(\alpha_{j}\)}(TL);
				\draw[Arrow](BL)--node[midway, left] {\(\alpha_{j'}\)}(TR);
				\draw[Arrow, -stealth](BR)to[out = 200, in = 130, edge node={node[midway, left] {\(\alpha_{j}\)}}](TL); % To use in/out, imagine a circle around the node. The angles are with respect to the node as the center.
				\draw[Arrow, -stealth](BR)to[out = 340, in = 60, edge node={node[midway, right] {\(\alpha_{j'}\)}}](TR);
			\end{tikzpicture}
		\end{center}
	\end{exr}
	Next, consider \(h:M\rightarrow N\) to be a homomorphism of R-moduli. For all i in \(\mathbb{N}\), put
	\begin{align*}
		f_{i}: & \frac{M}{\mathfrak{i}^{i}M}\rightarrow \frac{N}{\mathfrak{i}^{i}N} \\
		       & \overline{m}\mapsto \overline{h(m)}.
	\end{align*}
	We want to combine the two diagrams below
	\begin{center}
		\begin{tikzpicture}[
				observed/.style = {rectangle, thick, text centered, draw, text width = 6em},
				latent/.style = {ellipse, thick, draw, text centered, text width = 6em},
				error/.style ={circle, thick, draw, text centered},
				confounding/.style = {rectangle, thick, text centered, draw, text width = 6em, minimum width = 5.5in},
				outcome/.style = {rectangle, thick, draw, text centered, minimum height = 3.5in, text width = 6em},
			]
			\node(T) at (-2,1){\(\hat{N}\)};
			\node(BL) at (-4,-1){\(N/\mathfrak{i}^{j}N\)};
			\node(BR) at (0,-1){\(N/\mathfrak{i}^{i}N\)};

			\node(TT) at (4,1){\(\hat{M}\)};
			\node(BBL) at (2,-1){\(M/\mathfrak{i}^{j}M\)};
			\node(BBR) at (6,-1){\(M/\mathfrak{i}^{i}M\)};

			\draw[Arrow](T)--node[midway, above] {\(\alpha_{j}\)}(BL);
			\draw[Arrow](T)--node[midway, above] {\(\alpha_{i}\)}(BR);
			\draw[Arrow](BL)--node[midway, below] {\(\varphi_{j}^{i}\)}(BR);

			\draw[Arrow](TT)--node[midway, above] {\(\alpha_{j}'\)}(BBL);
			\draw[Arrow](TT)--node[midway, above] {\(\alpha_{i}'\)}(BBR);
			\draw[Arrow](BBL)--node[midway, below] {\(\varphi_{j}^{i'}\)}(BBR);
		\end{tikzpicture}
	\end{center}
	To do that, we consider
	\begin{center}
		\begin{tikzpicture}[
				observed/.style = {rectangle, thick, text centered, draw, text width = 6em},
				latent/.style = {ellipse, thick, draw, text centered, text width = 6em},
				error/.style ={circle, thick, draw, text centered},
				confounding/.style = {rectangle, thick, text centered, draw, text width = 6em, minimum width = 5.5in},
				outcome/.style = {rectangle, thick, draw, text centered, minimum height = 3.5in, text width = 6em},
			]
			\node(TL) at (-2,-1){\(N/\mathfrak{i}^{j}N\)};
			\node(BL) at (0,1){\(\hat{N}\)};
			\node(TR) at (2,-1){\(N/\mathfrak{i}^{i}N\)};
			\node(BR) at (0,3){\(\hat{M}\)};

			\draw[Arrow](TL)--node[midway, above] {\(\varphi_{j}^{i}\)}(TR);
			\draw[Arrow, dashed](BR)--node[midway, right] {\(\hat{h}\)}(BL);
			\draw[Arrow](BL)--node[midway, right] {}(TL);
			\draw[Arrow](BL)--node[midway, left] {}(TR);
			\draw[Arrow](BR)to[out = 200, in = 130, edge node={node[midway, left] {\(h_{j}\circ \alpha_{j}'\)}}](TL); % To use in/out, imagine a circle around the node. The angles are with respect to the node as the center.
			\draw[Arrow](BR)to[out = 340, in = 60, edge node={node[midway, right] {\(h_{i}\circ \alpha_{i}'\)}}](TR);
		\end{tikzpicture}
	\end{center}
	where \(\hat{h}\) is an R-homomorphism defined by
	\begin{align*}
		\hat{h}: & \hat{M}\rightarrow \hat{N}                                                           \\
		         & (\overline{m}_{i})_{i\in \mathbb{N}}\mapsto (\overline{h(m_{i})})_{i\in \mathbb{N}}.
	\end{align*}
	To truly be able to do analysis, we must define some sort of norm/distance in this space. To do it, consider
	\[
		(\mathbb{Q}, |\cdot |_{p}),\quad |x|_{p}=\frac{1}{p ^{v(x)}},
	\]
	where v(x) is given by the power \(\alpha\) below:
	\begin{align*}
		x = \frac{a}{b} & = p ^{\alpha}\frac{r}{s},\quad p\not|rs,\: \alpha\in \mathbb{Z} \\
		                & a = p ^{a_{1}}r, \: p\not|r                                     \\
		                & b = p ^{a_{2}}s, \: p\not|s                                     \\
	\end{align*}
	Then, for all \(x, y\in \mathbb{Q}\), define \(|x-y|_{p}\) as metric on \(\mathbb{Q}\), so that
	\begin{align*}
		 & |xy|_{p}=|x|_{p}|y|_{p}                                                                 \\
		 & |x|_{p}=0 \Longleftrightarrow x=0                                                       \\
		 & \underbrace{|x+y|_{p}\leq \max_{}\{|x|_{p}, |y|_{p}\}}_{\text{lacks Archimedean prop.}}
	\end{align*}
	This serves as a generalization of how to start analysis.

	One more concrete example would be to take \(R=\mathbb{K}[x]\), \(\mathfrak{i}= \langle X \rangle\). Then,
	\[
		\hat{R}\cong h[[x]]= \biggl\{\sum\limits_{i=0}^{\infty}a_{i}x^{i}: a_{i}\in \mathbb{K}\biggr\},
	\]
	and
	\[
		\hat{M}=\biggl\{\sum\limits_{i=1}^{\infty}a_{i}x^{i}: a_{i}\in \mathbb{K}\biggr\},\quad \mathbb{Q}_{p, c}\rightarrow \hat{\mathbb{Q}}_{p, c}.
	\]
\end{example}
\end{document}

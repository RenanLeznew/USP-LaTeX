\documentclass[../category_theory.tex]{subfiles}
\begin{document}
\section{Class 25 - November 19th, 2024}
\subsection{Motivations}
\begin{itemize}
	\item \(K_{0}\) of Categories;
	\item ``Devissage'' Theorem.
\end{itemize}
\subsection{\(K_{0}\) of a Category}
First of all, we set the context of our category. One should recall the definitions of \textbf{pre-additive, additive, abelian, cokernel, kernel, and full subcategory}.
Given that, we can define the space on which we'll work.
\begin{def*}
	Let \(\mathcal{A}\) be an abelian category. A \textbf{category with exact sequences} is a full additive subcategory \(\mathcal{P}\) of \(\mathcal{A}\), with the following properties:
	\begin{itemize}
		\item[i)] \(\mathcal{P}\) is \textbf{closed under extensions}, i.e., if
		      \[
			      0\rightarrow P_{1}\rightarrow P\rightarrow P_{2}\rightarrow 0
		      \]
		      is an exact sequence in \(\mathcal{A}\) and \(P_{1}, P_{2}\in \mathrm{Obj}(\mathcal{P})\), then \(P\in \mathrm{Obj}(\mathcal{P})\).
		\item[ii)] \(\mathcal{P}\) has a \textbf{small skeletonk} in \(\mathcal{A}\), i.e., \(\mathcal{P}\) has a full subcategory \(\mathcal{P}_{0}\) which is small. \(\square\)
	\end{itemize}
\end{def*}
It's nice to add the following remark: the exact sequences in such a category are defined to be the exact sequences in the ambient category \(\mathcal{A}\) with objects and morphisms from \(\mathcal{P}.\)
\begin{example}
	Small abelian categories or abelian categories with a small skeleton are categories with exact sequences.
\end{example}
\begin{example}
	Let R be a ring. Then, the category \(\mathrm{Proj R}\) discussed last class is a category with exact sequences, where the small skeleton is the set of direct summands in \(\{R^{m}: m\in \mathbb{N}\}.\)
\end{example}
\begin{def*}
	Let \(\mathcal{P}\) be a category with exact sequences and whose small skeleton is denoted by \(\mathcal{P}_{0}\). We define \(K_{0}(\mathcal{P})\) to be the free abelian group on \(\mathrm{Obj}(\mathcal{P}_{0})\) modulo the following relations:
	\begin{itemize}
		\item[i)] \([P]=[P']\) if \(P\cong P'\) in \(\mathcal{P}\);
		\item[ii)] \([P]=[P_{1}]+[P_{2}]\) if there is a short exact sequence
		      \[
			      0\rightarrow P_{1}\rightarrow P\rightarrow P_{2}\rightarrow 0
		      \]
		      in \(\mathcal{P}.\:\square\)
	\end{itemize}
\end{def*}
Here, \([P]\) denotes the element of \(K_{0}(P)\) corresponding to \(P\in \mathrm{Obj}(\mathcal{P}_{0})\). Moreover, (i) is just a particular case of (ii) when \(P_{1} = 0\), so that
\[
	0\rightarrow P \hookrightarrow\mathrel{\mspace{-15mu}}\twoheadrightarrow P_{2}\rightarrow 0 \Rightarrow P\cong P_{2}.
\]
\begin{theorem*}
	Consider the category \(\mathrm{ProjR}\), with R a ring. Then, \(K_{0}(R)\cong K_{0}(\mathrm{Proj R})\).
\end{theorem*}
Next, let's study the ``Devissage'' theorem, which comes from a French word for ``\textit{unscrewing}'' and is an algebraic geometry technique introduced by Alexander Grothendieck to prove statements about coherent sheaves on Noetherian Schemes:
\begin{def*}
	A \textbf{simple object} in an abelian category \(\mathcal{A}\)  is an object \(0\neq M\in \mathrm{Obj}(\mathcal{A})\) such that any monomorphism \(N\rightarrowtail M\) is either 0 or an isomorphism. \(\square\)
\end{def*}
\begin{def*}
	Let \(\mathcal{A}\) be an abelian category. The simple objects in \(\mathcal{A}\) are \textbf{objects of length one}. By induction, we define the \textbf{objects of length n} to be those \(M\in \mathrm{Obj}(\mathcal{A})\) for which there is an exact sequence in \(\mathcal{A}\)
	\[
		0\rightarrow M_{1}\rightarrow M\rightarrow M_{2}\rightarrow 0,
	\]
	with \(M_{1}, M_{2}\in \mathrm{Obj}(\mathcal{A})\) and \(M_{1}\) of length n-1, \(n\geq 2\: \square\)
\end{def*}
Define \(\mathcal{A}_{fi}\) to be the \textit{category of objects of finite length}, that is, objects M of length less than or equal to n for some natural number \(n\geq 1\).
\begin{prop*}
	The full subcategory \(\mathcal{A}_{fi}\) of \(\mathcal{A}\) is a category with exact sequences.
\end{prop*}
\begin{theorem*}[``Devissage'']
	Let \(\mathcal{A}\) be an abelian category in which every simple object is isomorphic to one, and only one, element of some set \(S\subseteq \mathrm{Obj}(\mathcal{A})\). Then, \(K_{0}(\mathcal{A}_{fi})\) is canonically isomorphic to the free abelian group on S.
\end{theorem*}
\end{document}

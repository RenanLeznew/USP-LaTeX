\documentclass[PhysicsI/physics_notes.tex]{subfiles}
\begin{document}
\section{Aula 19 - 18/05/2023}
\subsection{O que esperar?}
\begin{itemize}
	\item Outras forças na perspectiva de trabalho;
	\item Energia Potencial;
\end{itemize}
\subsection{Trabalho e Energia Cinética - Parte II}
Vimos o trabalho de algumas forças explicitamente, como a força gravitacional,
e estudamos também o tipo de força conhecida como conservativa. Analisaremos
a situação de algumas outras a seguir.

Para começar, qual é o trabalho da força atrito? Lembrando que existem dois tipos,
é notável que, como a força de atrito estático está relacionada com a ausência de movimento,
i.e., um objeto estático, o trabalho dela será nulo. Com relação ao
atrito cinético, no entanto, a questão deve ser elaborada.

Suponha que um bloco realiza duas trajetórias - primeiro de \(x_1\) até \(x_2\)
e em segundo de \(x_2\) até \(x_1\) - tal que a trajetória total é a ``soma das duas trajetórias.''
Assim,
\[
	W_{F_{at}}^{TT} = W_{F_{at}}^{T_1} + W_{F_{at}}^{T_2} = \int_{x_{1}}^{x_{2}}(-F_{at})\cdot dx + \int_{x_{2}}^{x_{1}}F_{at}\cdot dx
\]
Como \(|F_{at}^{c}| = \mu_{c}N = \mu_{c}mg\) , concluímos que
\begin{align*}
	W_{F_{at}}^{TT} & = -\int_{x_{1}}^{x_{2}}\mu_{c}mgdx + \int_{x_{2}}^{x_{1}}\mu_{c}mgdx            \\
	                & = -\mu_{c}mg(x_{2}-x_{1}) + \mu_{c}mg(x_{1}-x_{2}) = -2\mu_{c}mg\Delta x\neq 0.
\end{align*}
Portanto, a força de atrito não é conservativa, ou seja, ela é dissipativa.

Voltando às forças conservativas, podemos representá-la como a variação de uma quantidade.
Essa quantidade será chamada de Energia Potencial, e escrevemos
\[
	W_{1\rightarrow2}^{C} = \int_{C}^{}\vec{F}\cdot \vec{dr} = -\Delta U = -(U(\vec{r}_{2}) - U(\vec{r}_{1})).
\]
Formalmente, iremos definir a energia potencial como
\[
	\hypertarget{potential_energy}{\boxed{U(\vec{r}) = -\int_{\vec{r}_{0}}^{\vec{r}}\vec{F}\cdot \vec{dr}}}
\]
sendo \(\vec{r}_{0}\) o ponto no espaço para o qual a energia potencial é nula \((U(\vec{r}_{0} = 0))\).
\begin{example}
	Vamos considerar um objeto caindo livremente de uma altura h. Encontraremos sua energia potencial.
	Nesta situação, defina \(\vec{r}_{0}\) = (0, 0), tal que \(U(0, 0) = 0.\) Assim,
	\[
		U(P) = -\int_{0}^{h}\vec{P}\cdot dy\hat{j} = \int_{0}^{h}(-mg)dy = mgh.
	\]
	Vejamos o que ocorre se mudarmos o ponto inicial para metade do caminho - \(\vec{r}_{0} = \frac{h}{2}\). Neste caso,
	\[
		U(P) = -\int_{\frac{h}{2}}^{h}(-mg)dy = mgy \biggl|_{\frac{h}{2}}^{h}\biggr. = mg \frac{h}{2}.
	\]
	No entanto, apesar do resultado diferente, ao calcularmos a variação da energia potencial,
	sempre chegaremos ao mesmo resultado - \(\Delta U = mgh.\)
\end{example}
\begin{example}
	Agora, suponha um sistema massa-mola, com ponto \(x_{0} = 0, U(x_{0}) = 0\)
	e F = 0. Então, como a força da mola é -kx, a energia potencial será
	\[
		U(x) = -\int_{0}^{x}F_{mola}dx'= - \int_{0}^{x}(-kx')dx' = \frac{kx^{2}}{2}
	\]
\end{example}

Observe que, pelo \hyperlink{work_kin_3d}{Teorema da Energia Cinética},
\[
	W_{1\rightarrow2}^{C} = -\Delta U = \Delta E_{cin}.
\]
Manipulando esta equação, obtemos
\[
	\Delta E_{cin} + \Delta U = 0 \Rightarrow (E_{cin}^{x_2} + U^{x_2}) - (E_{cin}^{x_1} + U^{x_1}) = 0
\]
Definimos os termos da forma \((E_{cin}^{x_2} + U^{x_2})\) como a Energia Mecânica do sistema. Mais precisamente,
\[
	\hypertarget{mechanical_energy}{E_{T} = E_{cin} + U_{R}}
\]
Em particular, observe que \(\Delta E_{T} = 0\), ou seja, a energia
mecânica é conservada no sistema. A quantidade \(U_{R}\) é análoga à
forma que escrevemos a força resultante, no sentido de que ela é a soma
de todas as energias potenciais. Vamos ver agora como aplicar a conservação de energia
\begin{example}
	Considere uma bolinha que começa com \(v_{0} = 0m/s\) e atinge o chão com velocidade
	\(v_{2}\). Buscamos o valor dessa velocidade 2. Temos
	\[
		\Delta E = E_{2} - E_{1} = 0.
	\]
	Sabemos que
	\[
		E_{1} = E_{cin} + U_{1} = 0 + mgh = mgh
	\]
	e
	\[
		E_{2} = E_{cin} + U_{2} = \frac{1}{2}mv_{2}^{2}.
	\]
	Logo,
	\[
		E_{1} = \frac{1}{2}mv_{2}^{2} - mgh = 0 \Rightarrow v_{2}^{2} = 2gh \Rightarrow v_{2} = \sqrt[]{2gh}.
	\]
\end{example}
\end{document}

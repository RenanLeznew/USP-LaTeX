\documentclass[PhysicsI/physics_notes.tex]{subfiles}
\begin{document}
\section{Aula 05 - 12/04/2023}
\subsection{O que esperar?}
\begin{itemize}
	\item Lançamento de Projéteis;
	\item Alcance e altura máximos.
\end{itemize}
\subsection{Lançamento de Projéteis}
Para descrevermos a trajetória de um objeto lançado, utilizamos a equação
$$
	y = v_{0}\sin{(\theta )}\frac{x}{v_{0}\cos{(\theta _0)}j} - \frac{1}{2}g\biggl[\frac{x}{v_{0}\cos{(\theta _0)}}\biggr]^{2} = \tan{(\theta _{0})}x-\frac{1}{2}\frac{g}{v_{0}\cos{(\theta _{0})}}x^{2}.
$$
Desta equação, podemos deduzir algumas informações sobre o movimento de tal objeto. Por exemplo, qual é a altura máxima
que um objeto atingirá? E onde ele estará neste instante? Quanto tempo levará para atingir este ponto máximo de altura?
Vamos responder cada uma dessas situações. Para a terceira, note que, ao chegar no ponto máximo, $v_{y}=0.$ Assim,
$$
	0 = v_{0}\sin{(\theta_{0})}-gt_{m} \Rightarrow \boxed{t_{m} = \frac{v_{0}\sin{(\theta_{0})}}{g}.}
$$
Sabendo o tempo, podemos encontrar o valor dela com reação ao eixo y:
$$
	y_{m} = v_{0}\sin{(\theta )}\biggl[\frac{v_{0}\sin{(\theta )}}{g}\biggr] - \frac{1}{2}g\biggl[\frac{v_{0}\sin{(\theta )}}{g}\biggr]^{2} \Rightarrow
$$
$$
	\boxed{y_{m}=\frac{v_{0}^{2}\sin{(\theta )}}{g}}
$$
Tendo estas duas informações, somos capazes de encontrar a informação restante - qual o valor da posição no eixo x.
$$
	x_{m} = v_{0}\cos{(\theta )}\frac{v_{0}\sin{(\theta )}}{g} \Rightarrow \boxed{x_{m} = \frac{v_{0}^{2}}{g}\cos{(\theta )}\sin{(\theta )}}
$$
Ao lançar um objeto, não apenas terá uma altura máxima, mas alguma hora ele atingirá o chão ou um obstáculo (espero). Esse valor é
conhecido como alcance e descreve, como indica o nome, o alcance que o arremesso terá. Com um processo similar ao da
altura máxima, podemos encontrar valores para o tempo que ele leva até atingir este alcance e a posição. A ideia por trás deste
raciocínio é pegar o caso em que o objeto atinge o chão, pois, assim, teremos y=0. Com isso,
$$
	0 = v_{0}\sin{(\theta )}t - \frac{1}{2}gt^{2} \Rightarrow t_{a}(v_{0}\sin{(\theta )}-\frac{1}{2}gt)
$$
Para esta equação mais à direita zerar, duas coisas podem ocorrer - Ou $t_{a} = 0$, ou $v_{0}\sin{(\theta )} - \frac{1}{2}gt_{a}=0.$
Portanto,
$$
	\boxed{t_{a} = \frac{2v_{0}\sin{(\theta )}}{g}}
$$
E quanto à posição? Agora que encontramos o valor do tempo, é possível resolver este problema sem muitas complicações, pois
$$
	x_{a} = v_{0}\cos{(\theta )}2\frac{v_{0}\sin{(\theta )}}{g} \Rightarrow x_{a} = \frac{2v_{0}^{2}}{g}\cos{(\theta )}\sin{(\theta )} = 2x_{m}.
$$
Isso responde uma parte das nossas questões com relação ao deslocamento no espaço. No entanto, além disso, como o ângulo que
lançamos o objeto altera o alcance?

Dada uma função qualquer f(t), no ponto p' tal que f(t') tenha seu valor máximo, podemos utilizar a derivada, especificamente
o momento em que a derivada é nula. Neste raciocínio, o anglo máximo pode ser encontrado através de
$$
	\frac{dx_{a}}{d\theta _{0}} \biggl|_{\theta_{0}=\theta_{ao}}^{}\biggr. = 0.
$$
Logo, utilizando $x_{a}(\theta_{0}) = \frac{2v_{0}^{2}}{g}\cos{(\theta_{0})}\sin{(\theta_{0})}$ juntamente da regra do produto,
\begin{align*}
	\frac{dx_{a}}{d\theta _{0}} & = -\frac{2v_{0}^{2}}{g}\sin{(\theta_{0})}\cdot \sin{(\theta_{0})} + \frac{2v_{0}^{2}}{g}\cos{(\theta_{0})}\cos{(\theta_{0})}       \\
	                            & = \biggl[\frac{2v_{0}^{2}}{g}(-\sin^{2}{(\theta_{0})}+\cos^{2}{(\theta_{0})})\biggr]\biggl|_{\theta_{0}=\theta_{am}}^{}\biggr. = 0
	                            & \Rightarrow \sin^{2}{(\theta_{am})}=\pm\cos^{2}{(\theta_{am})} \Rightarrow \boxed{\theta =45^{\circ}}
\end{align*}
Tendo as informações de deslocamento, conseguimos, por fim, encontrar a velocidade com que atingirá o solo. Temos
$t_{A} = \frac{2v_{0}\sin{(\theta_{0})}}{g}$, tal que
\begin{align*}
	 & v_{x}(t_{a}) = v_{x_{0}} = v_{0}\cos{(\theta_{0})}                                                                                                          \\
	 & v_{y}(t_{a}) = v_{0}\sin{(\theta_{0})}-gt_{a} = v_{0}\sin{(\theta )} - \frac{g2v_{0}\sin{(\theta _0)}}{g}                                                   \\
	 & \Rightarrow  v_{y}(t_{a}) = -v_{0}\sin{(\theta )} = -v_{y_{0}} \Rightarrow \vec{v}(t_{a}) = v_{0}\cos{(\theta_{0})}\hat{i} - v_{0}\sin{(\theta_{0})}\hat{j} \\
	 & |\vec{v}(t_{a})| = \sqrt[]{v_{0}^{2}[\cos^{2}{(\theta )}+\sin^{2}{(\theta )}]} = v_{0}.
\end{align*}

\subsection{Movimento Uniformemente Variado em Duas Dimensões}


\end{document}

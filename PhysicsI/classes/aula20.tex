\documentclass[physics_notes.tex]{subfiles}
\begin{document}
\section{Aula 20 - 29/05/2023}
\subsection{O que esperar?}
\begin{itemize}
	\item Potência;
	\item Forças não-conservativas.
\end{itemize}
\subsection{Potência}
Suponha um sistema em que há um bloquinho de massa m sofrendo a ação de uma força F
em direção ao eixo x positivo. Ele sai de uma posição \(x_{1}\) até outra
\(x_{2}\), um deslocamento de \(\Delta x\). Sabemos que o trabalho dessa força será
dado por
\[
	W_{1\rightarrow2} = \int_{x_{1}}^{x_{2}}.
\]
Nessa linha, pergunta-se: Qual é o tempo para realizar esse trabalho?

É para essa pergunta que definimos a noção de potência, sendo ela
\[
	\hypertarget{power}{\boxed{P = \frac{dW}{dt},\quad P_{m} = \frac{\Delta W}{\Delta t}.}}
\]
As dimensões da potência são \([P]=\frac{[W]}{[T]}\), o que, no S.I., corresponde ao Watt (W), dado por
\(W = \frac{J}{s}.\). Sabemos, em particular, que \(dW = Fdx.\) Utilizando isso,
\[
	P = \frac{dW}{dt} = F \frac{dx}{dt} = Fv.
\]
Ainda mais, como \(F = m a = m \frac{dv}{dt}, P=m \frac{dv}{dt}v = mv \frac{dv}{dt} = \frac{d}{dt}\biggl(\frac{1}{2}mv^{2}\biggr) = \frac{dE_{cin}}{dt}\)
\subsection{Forças Não-Conservativas}
As forças não-conservativas têm a tendência de dissipar a energia do sistema, donde vem o
outro nome delas, forças dissipativas. Imagine que temos um corpo qualquer, no qual
agem forças conservatias e não-conservativas. A força resultante nesse corpo
é dado pela soma de todas elas, ou seja,
\[
	\vec{F_{R}} = \vec{F}_{1}^{c} + \vec{F}_{2}^{c} + \cdots +\vec{F}_{1}^{nc} + \vec{F}_{2}^{nc} + \cdots.
\]
Como o \hyperlink{kin_en_theo}{Teorema do Trabalho-Energia Cinética} é válido, temos
\[
	W_{1\rightarrow2} = \Delta E_{cin}.
\]
Deste modo,
\[
	\int_{\vec{r}_{1}}^{\vec{r}_{2}}\vec{F}_{R}\vec{dr} = \int_{\vec{r}_{1}}^{\vec{r}_{2}}( \vec{F}_{1}^{c} + \vec{F}_{2}^{c} + \cdots +\vec{F}_{1}^{nc} + \vec{F}_{2}^{nc} + \cdots)\vec{dr}.
\]
Elaboremos esta expressão:
\begin{align*}
	\int_{\vec{r}_{1}}^{\vec{r}_{2}}\vec{F}_{R}\vec{dr} & = \int_{\vec{r}_{1}}^{\vec{r}_{2}}( \vec{F}_{1}^{c} + \vec{F}_{2}^{c} + \cdots +\vec{F}_{1}^{nc} + \vec{F}_{2}^{nc} + \cdots)\vec{dr}.                  \\
	                                                    & = \sum\limits_{i}^{}W_{1\rightarrow2}^{c_{i}} + \sum\limits_{i}^{}W_{1\rightarrow2}W^{nc_{i}} = \Delta E_{cin}                                          \\
	                                                    & W_{1\rightarrow2}^{c_{i}} = -\Delta U_{i} \Rightarrow \sum\limits_{i}^{}(-\Delta U_{i}) + \sum\limits_{i}^{}W_{1\rightarrow2}^{nc_{i}} = \Delta E_{cin} \\
	                                                    & \sum\limits_{i}^{}W_{1\rightarrow2}^{nc_{i}}=\Delta E_{cin}+\sum\limits_{i}^{}\Delta U_{i} = \Delta (E_{cin}+U_{total}) = \Delta E.
\end{align*}
Portanto, sempre que temos forças dissipativas, a energia do sistema não se conserva, visto que
\[
	\sum\limits_{i}^{}W_{1\rightarrow2}^{nc_{i}} = \Delta E\neq 0.
\]
\begin{example}
	Vamos considerar um bloco acima de um plano inclinado, preso por uma mola e sendo sua massa 10kg.
	Ele faz um ângulo de \(\theta =45^{\circ}\) com o eixo horizontal. A mola contrai e o bloco desliza pelo plano em uma distância \(d=2m\),
	com atrito \(\mu_{c}=0.5\) e constante elástica \(k=800N/m\). Neste ponto, a velocidade do bloco é nula e
	sua energia cinética anula-se. Qual é o tamanho da contração da mola?

	Vamos começar expressando a posição do bloco em x e y.

	\textbf{Eixo x}: \(\hat{x}: -P\sin{(\theta )} + F_{at} = ma_{x}\)

	\textbf{Eixo y}: \(\hat{y}: N = P\cos{(\theta )} \Rightarrow F_{at}=\mu_{c}N = \mu_{c}mg\cos{(\theta )}.\)

	Com isso, a energia potencial no primeiro instante tem a expressão
	\[
		U_{1} = mg(h_{1}+h_{2}) = mg(d+\Delta x)\sin{(\theta )},
	\]
	sendo \(h_{1}\)a altura entre a mola e o bloco e \(h_{2}\) a altura do bloco. Logo,
	\[
		E_{1} = E_{cin}^{1} + U_{1} = 0 + U_{1} = mg(d + \Delta x)\sin(\theta )
	\]
	Agora, no segundo instante,
	\[
		E_{2} = E_{cin}^{2} + U_{2} = 0 + U_{el} = \frac{1}{2}k\Delta x^{2}.
	\]
	Assim,
	\begin{align*}
		\Delta E = W_{F_{at}} & = \int_{1}^{2} F_{at}\cdot dx = -\mu_{c}mg\cos{(\theta )}(d+\Delta x)                                                       \\
		                      & \frac{1}{2}k\Delta x^{2} - mg(d+\Delta x)\sin{(\theta )} = -\mu_{c}mg\cos{(\theta )}(d+\Delta x)                            \\
		                      & \frac{1}{2}k\Delta x^{2} - mg\Delta x(\sin{(\theta )}-\mu_c\cos{(\theta )}) - mgd(\sin{(\theta )}-\mu_{c}\cos{(\theta )})=0 \\
		                      & \Rightarrow \Delta x=0.46m
	\end{align*}
\end{example}
\newpage

\end{document}

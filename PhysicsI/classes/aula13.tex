\documentclass[physics_notes.tex]{subfiles}
\begin{document}
\section{Aula 13 - 03/05/2023}
\subsection{O que Esperar?}
\begin{itemize}
	\item Revisão Pré-Prova
\end{itemize}
\subsection{Cinemática}
\subsubsection{Movimento Unidimensional}
As equações do movimento unidimensional são
\begin{align*}
	 & \vec{r}(t)\Rightarrow \vec{v}(t)=\frac{d \vec{r}(t)}{dt} \Rightarrow \vec{a}(t) = \frac{d \vec{v}(t)}{dt}                         \\
	 & \vec{v}(t)-\vec{v}(t_{0}) = \int_{t_{0}}^{t} \vec{a}(t)dt \Rightarrow \vec{r}(t) - \vec{r}(t_{0}) = \int_{t_{0}}^{t} \vec{v}(t)dt
\end{align*}
\subsubsection{Movimento Bidimensional}
Em duas dimensões ou mais,
$$
	\vec{r}(t) = x(t)\hat{i} + y(t)\hat{j} + z(t)\hat{k} (\text{ O mesmo para v(t) e a(t)}).
$$
Neste caso, decompomos o movimento em cada eixo e tratamos como movimento unidimensional. No caso do movimento bidimensional uniforme,
as velocidades em ambos os eixos são constantes, tal que
$$
	\left\{\begin{array}{ll}
		x(t) = x_{0} + v_{x}t - t = \frac{x-x_{0}}{v_{x}} \\
		y(t) = y_{0} + v_{y}t.
	\end{array}\right.
$$
A trajetória será $y=y_{0} + \frac{v_{y}}{v_{x}}(x-x_{0}).$
\subsubsection{Trajetória de Projéteis}
Vimos também sobre o movimento de projéteis. Nele, a velocidade no eixo x é constante; No eixo y, $v_{y}(t) = v_{y}^{0} + at$,
$a\neq0$. Assim, $x(t) = x_{0} + v_{x}^{0}t$ e $y(t) = y_{0} + v_{y}^{0}t$. Além disso, a trajetória é dada por
$$
	y = \tan{\theta }x - \frac{gx^{2}}{2|v_{0}|^{2}\cos^{2}{(\theta )}}
$$
Utilizando esses dados, a altura máxima de um objeto em trajetória de projétil, $(x_{m}, y_{m})$, ocorre quando
$v_{y}(t_{m}) = 0$, tal que $t_{m}=\frac{|\vec{v}_{0}|\sin{(\theta )}}{g}, y_{m}=\frac{1}{2}\frac{|v_{0}|^{2}\sin^{2}{(\theta )}}{g}.$
Finalmente, vimos o alcance de um objeto em lançamento, o qual é dado pelo ponto cartesiano $(x_{r}, 0)$:
$$
	x_{r} = \frac{v_{0}^{2}}{g}\sin{\theta }\cos{\theta },\quad t_{m} = \frac{2|\vec{v}_{0}|\sin{(\theta)}}{g} = 2t_{m}
$$
\subsubsection{Movimento Circular}
Além desses tipos de movimentos, aprendemos sobre o movimento circular, sumarizado na tabela
\begin{table}[h!]
	\centering
	\begin{tabular}{|c|c|c|c|}
		\hline
		                    & \textbf{Vetoriais} & \textbf{Variáveis angulares}                                    & \textbf{Variáveis escalares}                                                        \\
		\hline
		\textbf{Posição}    & $\vec{r}(t)$       & $\theta(t)$                                                     & $s(t) = R\theta(t)$                                                                 \\
		\hline
		\textbf{Velocidade} & $\vec{v}(t)$       & $\omega(t) = \frac{d \theta(t)}{dt}$                            & $v(t) = \frac{d s(t)}{dt} = R\omega(t)$                                             \\
		\hline
		\textbf{Aceleração} & $\vec{a}(t)$       & $\alpha(t) = \frac{d\omega(t)}{dt} = \frac{d^2\theta(t)}{dt^2}$ & $|\vec{a}(t)| = \frac{dv(t)}{dt} = R\alpha(t),\quad |\vec{a}_{cp}| = \frac{v^2}{R}$ \\
		\hline
	\end{tabular}
	\caption{Resumo movimento circular.}
	\label{Resumo2MCU}
\end{table}

\subsection{Revisão de Dinâmica}
A dinâmica é a relação entre o movimento e as interações/forças.
\subsubsection{Leis de Newton}
A primeira Lei de Newton é a seguinte:
\begin{quote}
	\hypertarget{first_newton}{``Um corpo em repouso ou em Movimento Uniforme matném o seu estado a não ser que haja uma força.''}
\end{quote}
A segunda relaciona massa e aceleração para descrever a interação:
$$
	\hypertarget{second_newton}{\vec{F}_{r} = m \vec{a}.}
$$
A terceira e última é a chamada lei da ação e reação
\begin{quote}
	\hypertarget{third_newton}{``Para toda força que é aplicada, ocorrerá uma força de reação com mesma intensidade, mas direção oposta''}
\end{quote}
\subsubsection{Tipos de Força}
Em seguida, vimos os tipos de força, sendo elas
\begin{itemize}
	\item[1)] Força Peso: $\vec{P} = m \vec{g}.$
	\item[2)] Força Normal: Aparece em corpos que estão em contato, responsável por impedir objetos de entrarem um no outro.
	      Caso o contato suma, a normal também desparece, ou seja, se perde o contato, $\vec{N} = 0.$
	\item[3)] Força de Tração $(\vec{T})$: Atua no contexto de cordas, polias, etc.
	\item[4)] Força Elástica: Força restaurativa quando uma mola é distorcida. $\vec{F} = -k \vec{x}.$
\end{itemize}
\subsection{Exemplos}
\begin{example}
	Considere um forno esquentado até 200 graus. Num pedaço deste forno, há um pequeno buraco. Além disso, há algo dentro do forno.
	Através desse buraco, há um feixe de partículas saindo com velocidade $\vec{v}_{0}$. Na direção pela qual elas estão saindo,
	há um gradiente de campo magnético $B'$ com uma força de origem magnética agindo nele, dada por $\vec{F}\approx \mu_{B}\vec{B}'$,
	em que $\mu_{B}$ é uma constante chamada magneton de Bohr com valor $\mu_{B}=9.27 \cdot 10^{-24}J/T.$ Quando a partícula passa pelo
	campo após percorrer a distância D, a força fará com que ela mude sua trajetória e comece a mover-se para baixo. No fim do sistema, tem uma máquina
	que registrará a posição em que a partícula irá parar, ou seja, um desvio $\Delta y$ de onde ela originalmente
	teria parado caso continuasse a seguir reto.
	Temos $|v_{0}|=400m/s, D = 10m, \Delta y = 19cm, B' = 2.5T/m.$ Qual é o átomo?

	Vamos separar em eixos. No eixo x,
	$$
		v_{x} = v_{0}^{x} = |\vec{v}_{0}|,\quad x(t) = |\vec{v}_{0}|t.
	$$
	Disto segue que $t = \frac{D}{|\vec{v}_{0}|} = 0.025s.$

	No eixo y,
	$$
		y(t) = \frac{1}{2}at^{2}.
	$$
	Podemos encontrar a aceleração notando que $F = ma = \mu_{B}B',$ ou seja, $\vec{a} = \frac{\mu_{B}B'}{m}$. Assim,
	$$
		y(t) = \frac{1}{2}\frac{\mu_{B}B'}{m}t^{2}.
	$$
	No instante final,
	$$
		\Delta y=\frac{1}{2}\frac{\mu_{B}B'}{m}t_{f}^{2} \Rightarrow m = \frac{1}{2}\frac{\mu_{B}B'}{\Delta y}t_{f}^{2}.
	$$
	Podemos resolver essa conta com os dados do enunciado:
	$$
		m = \frac{1}{2}\frac{9.27 \cdot 10^{-24}\cdot 2.5}{0.19}(0.025)^{2} = 3.81 \cdot 10^{-26}kg.
	$$
	Pela tabela periódica, utilizando a unidade de massa atômica como $1.66 \cdot 10^{-27}kg,$ temos
	$$
		m = m^{*}\cdot u,
	$$
	sendo $m^{*}$ o número de massa atômica. Assim,
	$$
		m^{*} = \frac{m}{u}\approx 22.9\approx 2 \cdot 11.
	$$
	Portanto, os átomos saindo pelo fogão são de sódio.
\end{example}
\begin{example}
	Seuponha que tem um jogador de futebol a uma distância d do gol. Ele chuta a bola, a qual faz um ângulo $\theta $ com o
	eixo horizontal. O gol possui uma altura h. Qual é a menor e a maior velocidade para fazer o gol? Dados d = 9m, h = 2.4m, $\theta =30^{\circ}, g=9.8m/s^{2}.$

	Para entrar no gol, a bola deve percorrer toda a d e ficar no chão, ou acertar um lugar menor que h. Em outras palavras,
	temos uma restrição com relação à trajetória (Sempre que houver uma restrição de trajetória, utilize a equação da trajetória.).
	Segue que
	$$
		y = \tan{(\theta)}x - \frac{gx^{2}}{2v_{0}^{2}\cos^{2}{(\theta )}}.
	$$
	Em x = d, $0\leq y\leq h$, tal que
	$$
		0\leq \tan{(\theta)}d - \frac{gd^{2}}{2v_{0}^{2}\cos^{2}{(\theta )}}\leq h.
	$$
	Vamos quebrar o problema em duas partes. Na primeira,
	\begin{align*}
		 & \frac{gd^{2}}{2v_{0}^{2}\cos^{2}{\theta }}\leq \tan{(\theta )}d \Rightarrow gd^{2}\leq \tan{(\theta )}d \cdot 2v_{0}^{2}\cos^{2}{(\theta)}             \\
		 & \Rightarrow \frac{gd^{2}}{\tan{(\theta )}d \cdot 2\cos^{2}{(\theta )}}\leq v_{0}^{2} \Rightarrow  \frac{gd}{2\sin{\theta }\cos{\theta }}\leq v_{0}^{2} \\
		 & \Rightarrow v_{0}\geq 10.09m/s.
	\end{align*}
	Para a parte 2,
	\begin{align*}
		 & \tan{(\theta )}d - \frac{gd^{2}}{2v_{0}^{2}\cos^{2}{(\theta )}}\leq  h \Rightarrow \tan{(\theta )}d - h\leq \frac{gd^{2}}{2v_{0}^{2}\cos^{2}{(\theta )}}               \\
		 & \Rightarrow v_{0}^{2}(\tan{(\theta )}d - h)\leq \frac{gd^{2}}{2\cos^{2}{(\theta )}} \Rightarrow  v_{0}^{2}\leq \frac{gd^{2}}{2\cos^{2}{(\theta )}(\tan{(\theta )}d-h)} \\
		 & v_{0}\leq 13.76m/s.
	\end{align*}
	Portanto, $10.09m/s\leq v_{0}\leq 13.76m/s.$
\end{example}
\end{document}

\documentclass[PhysicsI/physics_notes.tex]{subfiles}
\begin{document}
\section{Aula 11 - 25/04/2023}
\subsection{O que esperar?}
\begin{itemize}
	\item Força de Tensão;
	\item Roldanas e Polias.
\end{itemize}

\subsection{Força de Tensão}
A força de tensão é a força sofirda por cordas e roldanas. Assume-se que a corda é inextensível e possui massa
desprezível. Um exemplo em que ela aparece bastante é quando há roldanas/polias, ou seja, mecanismos que mudam a direção
da força. Vejamos exemplos
\begin{example}
	Suponha que há um bloco suspenso ao teto por uma corda com massa m e em equilíbrio. Qual é a força de tração?

	Como o bloco está em repouso, sabemos que a resultante vale 0. Note que as forças que agem sobre o bloco são a tração
	da corda e a foça peso. Logo,
	$$
		0 = \vec{F}_{r} = T - P \Rightarrow T = P = mg.
	$$
\end{example}
\begin{example}
	Agora, imagine que há um bloco suspenso por duas cordas próximo à quinta de um teto. As duas cordas encontram-se num
	ponto p em que há um nó. A primeira delas sai horizontalmente da parede, fazendo um ângulo reto com a mesma. A segunda
	surge pelo teto com um ângulo $\theta .$ Qual é a força de tração que age sobre a corda que sai do ponto de nó
	delas e prende-se ao bloco? E nas cordas 1 e 2?

	Quanto ao primeiro item, note que o bloco está em equilíbrio, logo a força resultante em p e no bloco deve ser 0.
	Assim,
	$$
		T_{3}-P = 0 \Rightarrow  T_{3} = m \cdot g.
	$$

	Para o segundo, decompõe-se a tração nas suas componentes x e y:
	\begin{itemize}
		\item[Em x:] $T_{1}\cos{(\theta )}-T_{2}=0$
		\item[Em y:] $T_{1}\sin{(\theta )}-T_{3}=0.$
	\end{itemize}
	Pela segunda equação, utilizando o valor para $T_{3}$ previamente encontrado, segue que
	$$
		T_{1}\sin{(\theta )} - T_{3}\Rightarrow T_{1}=\frac{mg}{\sin{(\theta )}}.
	$$
	Logo, combinando isso com a primeira equação,
	$$
		T_{2}=T_{1}\cos{(\theta )} - \frac{mg}{\sin{(\theta )}}\cos{(\theta )} = \frac{mg}{\tan{(\theta )}}.
	$$
	Portanto, usando $\alpha =90^{\circ} - \theta $, segue que
	$$
		\left\{\begin{array}{ll}
			T_{1}=\frac{mg}{\sin{(\theta )}} \Rightarrow 1.29mg \\
			T_{2}=\frac{mg}{\tan{(\theta )}} \Rightarrow 0.83mg \\
			T_{3}=mg.
		\end{array}\right.
	$$
\end{example}
O próximo exemplo ilustra as situações de polias num de seus contextos mais comuns - Máquina de Atwood.
\begin{example}
	A máquina de Atwood consiste em uma polia presa ao teto com um bloco em cada ponta. O primeiro bloco tem massa
	$m_{1}$ e o segundo tem $m_{2}$. A corda exerce uma tração T e os blocos têm pesos $P_{1}, P_{2}.$

	Se $m_{1}>m_{2}$, segue que $m_{1}$ desce e $m_{2}$ sobe. Também é possível relacionar cada quantidade.
	Se o deslocamento, a velocidade e a aceleração do primeiro bloco são $\Delta d_{1}, v_{1}, a_{1}$ e, do segundo bloco,
	$\Delta d_{2}, v_{2}, a_{2}$, essas quantias vinculam-se da seguinte forma:
	\begin{align*}
		 & |\Delta d_{1}| = |\Delta d_{2}| \\
		 & |v_{1}| = |v_{2}|               \\
		 & |a_{1}| = |a_{2}|.
	\end{align*}
	As forças atuando no bloco 1 são a tração para cima, a peso $P_{1}$ para baixo e a aceleração como consequência da
	diferença de massas entre os blocos, a qual aponta para baixo também. Por outro lado, no bloco 2, as forças são
	a tração para cima, a peso $P_{2}$ para baixo e, como consequência da diferença de massas, a aceleração que surge,
	dessa vez apontando para cima, já que este sobe. Em outras palavras, as seguintes equações descrevem o movimento no
	eixo y:
	\begin{align*}
		 & 1:\quad T-P_{1} = -m_{1}a \\
		 & 2:\quad T-P_{2} = m_{2}a.
	\end{align*}
	As equações que descrevem o bloco subindo podem ser encontradas notando que $-P_{1}+P_{2}=-m_{1}a-m_{2}a \Longleftrightarrow
		-m_{1}g + m_{2}g = -(m_{1}+m_{2})a \Longleftrightarrow  g(m_{1}-m_{2}) = (m_{1}+m_{2})a.$ Disto, concluímos que
	$$
		a = \frac{(m_{1}-m_{2})}{(m_{1}+m_{2})},\quad g>0, m_{1}>m_{2}.
	$$
	Além disso,
	$$
		T = \frac{2m_{1}m_{2}}{(m_{1}+m_{2})}g
	$$
\end{example}
\begin{example}
	O próximo exemplo que veremos é o de blocos numa mesa: Há dois blocos presos por uma corda que passa por uma polia.
	O primeiro bloco encontra-se na parte superior da mesa, possuindo massa $m_{1}$. O segundo bloco está suspenso, sendo
	sua massa $m_{2}.$ Nesse contexto, o bloco 2 puxa o bloco 1, que desloca-se $\Delta x_{1}$ unidades com aceleração em sua direção
	$a_{1},$ resultando no próprio movimento de $\Delta y_{2}$ com uma aceleração de $a_{2}$ para baixo.

	Para lidar com esse sistema, observe as forças que atuam em cada bloco - No bloco 1, a gravidade resulta na força
	peso $P_{1}$ para baixo e, pela Terceira Lei, há a consequência da normal $N_{1}.$ Além delas, há a tração T da corda.
	Com relação ao bloco 2, não há força de contato, então ele é governado pelo par ação reação da Tração T e da força
	peso $P_{2}.$

	Agora, analisando os vínculos dos valores, observa-se que $|\Delta x_{1}|=|\Delta y_{2}|$ e $|a_{1}| = |a_{2}| = a.$
	Podemos dar continuidade e estudar as equações. Começando pelo bloco 1,
	\begin{align*}
		 & \text{Eixo x: } F_{r} = T = m_{1}\cdot a                  \\
		 & \text{Eixo y: } N_{1} - P_{1} = 0 (\text{Sem movimento.})
	\end{align*}
	Estudando, então, o bloco 2, o único movimento sendo no eixo y:
	\begin{align*}
		F_{r} & = T - P_{2} = -m_{2}a                       \\
		      & \Rightarrow m_{1}a - P_{2} = -m_{2}a        \\
		      & \Rightarrow m_{1}a - m_{2}g = -m_{2}a       \\
		      & \Rightarrow (m_{1}+m_{2})a = m_{2}g         \\
		      & \Rightarrow a = \frac{m_{2}}{m_{1}+m_{2}}g.
	\end{align*}

\end{example}
\end{document}

\documentclass[PhysicsI/physics_notes.tex]{subfiles}
\begin{document}
\section{Aula 01 - 27/03/2023}
\begin{itemize}
	\item Revisar propriedades de derivadas;
	\item Aplicar derivadas em movimento 1D.
\end{itemize}
\subsection{Movimentos 1D}
Dada uma partícula com posição descrita por $x = x(t)$, em que t é a variável de tempo, denotamos seu deslocamento
por $\Delta x = x_{2} - x_{1} = x(t_{2}) - x(t_{1}).$ Analogamente, o intervalo de tmepo é definido por $\Delta t = t_{2} - t_{1}$.
Com essas ferramentas, já podemos definir a velocidade média de um objeto em uma dimensão como $\vec{v} = \frac{\Delta x}{\Delta t}.$
Observe que, quanto menor o intervalo de tempo, mais momentâneo se torna essa definição, de modo que a velocidade instantânea
pode ser encontrada como
$$
	\lim_{\Delta t\to0}\frac{x(t + \Delta t) - x(t)}{\Delta t} = \vec{v}(t).
$$
Regras de derivadas:
\begin{align*}
	 & f(t) = c \Rightarrow \frac{df}{dt} = 0 \text{ Derivada de uma constante é sempre nula;} \\
	 & f(t) = x^{n} \Rightarrow \frac{df}{dt} = nx^{n-1} \text{ Regra do tombo;}               \\
	 & f(t) = A\sin{(t)} \Rightarrow \frac{df}{dt} = A\cos{(t)};                               \\
	 & f(t) = B\cos{(t)} \Rightarrow \frac{df}{dt} = -B\sin{(t)};                              \\
	 & f(t) = C e^{t} \Rightarrow \frac{df}{dt} = C e^{t}.
\end{align*}
\begin{example}
	\begin{align*}
		 & i)f(t) = 3t^{4} + t^{2} \Rightarrow \frac{df}{dt} = 12t^{3} + 2t                                         \\
		 & ii) f(t) = 5\sin{(t)} + 3(t^{2}+1) = 5\sin{(t)} + 3t^{2} + 3 \Rightarrow \frac{df}{dt} = 5\cos{(t)} + 6t
	\end{align*}
\end{example}

A partir deste ponto, tome t como tempo, x(t) como posição e v(t) a velocidade instantânea.

\begin{tikzpicture}
	\begin{axis}[
			axis lines = left,
			xlabel = \(t\),
			ylabel = {\(x(t)\)},
		]
		%Below the red parabola is defined
		\addplot [
			domain=0:20,
			samples=100,
			color=red,
		]
		{3*x - 20};
		\addlegendentry{\(x(t) = 3t - 20\)}
		%Here the blue parabola is defined
		\addplot [
			domain=20:40,
			samples=100,
			color=blue,
		]
		{-3*x + 100};
		\addlegendentry{\(x(t) = -3t + 100\)}
	\end{axis}
\end{tikzpicture}

Esse movimento em que a velocidade é descrita por uma linha reta é conhecido como movimento retilíneo uniforme, pois
a velocidade $v(t)$ muda de forma linear, i.e., $\frac{dx}{dt} = c$, em que c é uma constante.

Por outro lado, há outro tipo de movimento, o movimento retilíneo uniformemente variado, em que a velocidade não é constante.
A ação responsável por mudar a velocidade é conhecida como aceleração, e os gráficos tendem a assumir o seguinte formato

\begin{tikzpicture}
	\begin{axis}[
			axis lines = left,
			xlabel = \(t\),
			ylabel = {\(x(t)\)},
		]
		%Below the red parabola is defined
		\addplot [
			domain=0:20,
			samples=100,
			color=red,
		]
		{-x^2 + 10*x};
		\addlegendentry{\(x(t) = -x^2+10x\)}
		%Here the blue parabola is defined
	\end{axis}
\end{tikzpicture}

Ou, caso a velocidade cresça com o tempo,

\begin{tikzpicture}
	\begin{axis}[
			axis lines = left,
			xlabel = \(t\),
			ylabel = {\(x(t)\)},
		]
		%Below the red parabola is defined
		\addplot [
			domain=0:20,
			samples=100,
			color=red,
		]
		{x^2};
		\addlegendentry{\(x(t) = x^2\)}
		%Here the blue parabola is defined
	\end{axis}
\end{tikzpicture}

Há ainda o caso em que a velocidade cresce por um tempo e diminui depois, com gráficos como o que segue

\begin{tikzpicture}
	\begin{axis}[
			axis lines = left,
			xlabel = \(t\),
			ylabel = {\(x(t)\)},
		]
		%Below the red parabola is defined
		\addplot [
			domain=0:20,
			samples=100,
			color=red,
		]
		{x^2 - 20*x};
		\addlegendentry{\(x(t) = x^2 - 20x\)}
		%Here the blue parabola is defined
	\end{axis}
\end{tikzpicture}

Nestes casos, para calcular o deslocamento da particula, precisamos somar muito mais intervalos de tempo. Para isso, observe que
cada instante, a posição da partícula pode ser encontrada multiplicando-se o intervalo de tempo pela velocidade instanânea, i.e.,
$\Delta x_{i}' = v_{i}'\Delta t_{i}'$. Quebrando os intervalos desta forma, o deslocamento de um ponto a outro é denotado por
$$
	\Delta x_{1, 2} = x(t_{2}) - x(t_{1})\approx \sum\limits_{k=1}^{N}\Delta x_{i}' = \sum\limits_{k=1}^{N}v_{i}'\Delta t_{i}'
$$
Assim como para a velocidade instantânea, quanto menor tomarmos o intervalo de tempo, mais preciso é o valor encontrado para $\Delta x_{1, 2}$,
o que indica uma boa oportunidade para o uso do limite novamente. Com isso, definimos
$$
	x(t_{2}) - x(t_{1}) = \lim_{\Delta t'\to0} \sum\limits_{i=1}^{N}v(t_{i}')\Delta t_{i}' = \int_{t_{1}}^{t_{2}}v(t)dt
$$
Este último símbolo, chamado integral, descreve a área ``embaixo'' da curva da função f(t) dentro do intervalo $[t_{1}, t_{2}].$
Supondo que c e k são constantes quaisquer, seguem abaixo algumas das regras de integração:
\begin{align*}
	 & i)f(t) = ct^{n} \Rightarrow \frac{df}{dt} = nct^{n-1} \Rightarrow F(t) = \frac{ct^{n+1}}{n+1}\text{ (Primitiva de f)}                      \\
	 & ii) \int_{t_{1}}^{t_{2}}f(t)dt = F(t_{2}) - F(t_{1}) = \frac{c}{n+1}t_{2}^{n+1} - \frac{c}{n+1}t_{1}^{n+1}\text{ (Integral definida de f)} \\
	 & iii) \int_{}^{}f(t)dt = \frac{c}{n+1}t^{n+1} + k\text{ (Integral indefinida de f)}
\end{align*}
Para conferir se a integral está correta, é preciso derivar a função F e, se obter como resultado a função f, significa que está correto.
Com este conhecimento em mente, segue que
$$
	\boxed{ x(t) = \int_{}^{}v_{0}dt = v_{0}t + x_{0}}
$$
Algumas outras regras importantes:
\begin{align*}
	 & iv) \frac{d\sin{(t)}}{dt} = \cos{(t)} \Rightarrow \int_{}^{}\cos{(t)}dt = \sin{(t)} + c \\
	 & v) \frac{d\cos{(t)}}{dt} = -\sin{(t)} \Rightarrow \int_{}^{}\sin{(t)}dt = \cos{(t)} + c \\
	 & vi) \frac{d e^{t}}{dt} = e^{t} \Rightarrow \int_{}^{} e^{t}dt = e^{t} + c
\end{align*}
Ou seja, em certo sentido, a integral e a derivada são dois lados da mesma moeda, assim como mulitplicação e divisão ou adição e subtração.
\end{document}

\documentclass[physics_notes.tex]{subfiles}
\begin{document}
\section{Aula 12 - 27/04/2023}
\subsection{O que esperar?}
\begin{itemize}
	\item Força de Tensão;
	\item Força de mola/Força Elástica.
\end{itemize}
\subsection{Força de Mola}
Sendo $x_{0}$ a posição da equação da mola, ao aplicarmos uma força na mola e deformamos, a reação
é uma força chamada restauradora, pois ela é responsável por recuperar o estado de equilíbrio da mola. Sua fórmula é
$$
	\vec{F} = -k\Delta \vec{x},
$$
em que k é a constante elástica da mola. Podemos elaborar nessa fórmula um pouco mais utilizando a segunda lei de Newton, pois
$F = -kx = ma = m \frac{dv}{dt}.$ Assim, $-kx = m \frac{d^{2}x}{dt^{2}}$, de onde encontramos uma fórmula para a posição em função de tempo
$$
	\frac{d^{2}x}{dt^{2}} = -\frac{k}{m}x = -\omega^{2}x \Rightarrow \boxed{x(t) = \frac{d^{2}x}{dt^{2}} = A\sin{\omega t}}
$$
Se derivarmos essa função, a velocidade do movimento será
$$
	\frac{d}{dt}\sin{\omega t} = \frac{d\sin{u}}{du}\frac{du}{dt} = \cos{u}\omega = \omega \cos{\omega t}
$$
e a aceleração
$$
	\frac{d}{dt}\cos{u} \omega = \frac{d}{du}\cos{u}\omega \frac{du}{dt} = \omega^{2}\sin{\omega t}.
$$
Vejamos um exemplo
\begin{example}
	Considere um objeto em repouso preso a uma mola com posição de equilíbrio $x_{0}$. Ocorre um
	deslocamento da mola em $\Delta y$, aplicando uma força $\vec{F}_{el}$ ao bloquinho. Qual é a força normal?

	Note que todas as forças estão atuando no eixo y. Como o corpo está parado, a resultante no eixo y é nula, tal que
	$F_{r}^{y} = |F_{el}| + N - P$. Com isso, $N = P - |F_{d}| = mg - k|\Delta y|.$
\end{example}
\subsection{Força de Tração}
\begin{example}
	Considere um elevador manual segurado por uma roldana com uma pessoa dentro dela. Essa pessoa está puxando a corda, exercendo
	uma força $\vec{F}$ para baixo. O sistema pessoa-elevador tem uma massa de $m=95kg$. Utilize $g=9.81m/s^{2}$.
	\begin{itemize}
		\item[a)] Qual é o módulo de $\vec{F}$ para subir com velocidade constante?
		\item[b)] Qual é a força resultante para uma aceleração $a_{y}=1.3m/s^{2}?$
	\end{itemize}

	(a) Observe que, ao puxar a corda, a reação gerada à força do puxão é a força de tração. Além disso,
	a resultante no ponto em que a pessoa puxa a corda é igual à tração. Assim, como há a tração no elevador também,
	$$
		\vec{F}_{R}^{y}= -P + T + T = -P + 2T
	$$
	Como é pedido que o elevador suba com velocidade constante, $\vec{F}_{R}^{y} = 0$, donde segue que
	$$
		T = \frac{P}{2} = \frac{95 \cdot 9.8}{2} = 466N.
	$$

	(b) Como encontramos a forma da resultante no último item, temos
	$$
		\vec{F}_{R}^{y} = - P + 2T = ma_{y} \Rightarrow T = F = \frac{ma_{y}+P}{2}.
	$$
	Assim, $F = \frac{m}{2}(a_{y}+g)^{2} = 528N.$
\end{example}

\begin{example}
	Dado um bloco e multiplas polias, sendo o bloco com um peso de $2670N.$ Suponha que o sistema está em equilíbrio.
	\begin{itemize}
		\item[a)] Quanto vale a tração no sistema da primeira roldana?
		\item[b)] Agora, quanto vale a tração considerando também a segunda roldana no sistema como separada?
		\item[c)] Suponha que a polia passa por uma nova polia ligada ao teto antes de chegar ao bloco. Qual é a tração nesse caso?
	\end{itemize}

	(a) Como a polia redireciona a direção da tração uma vez, $F_{r}^{y} = 2T - P = 0,$ segue que $T = \frac{P}{2} = 1385N.$

	(b) Apesar de ter adicionado uma nova polia, ainda assim, como só há dois pontos de sustentação, $F_{r}^{y} = 2T - P \Rightarrow T = \frac{P}{2}.$

	(c) Diferente do item (b), como ela passa por mais uma polia, há um novo ponto de sustentação além do segundo, sendo eles
	o teto, a polia presa ao teto e a polia preso ao bloco. Logo, $F_{r}^{y} = 3T - P = 0$, ou seja, $T = \frac{P}{3} = 890N.$
\end{example}
\end{document}

\documentclass[physics_notes.tex]{subfiles}
\begin{document}
\section{Revisão P2 - 31/05/2023}
\subsection{Dinâmica}
\subsubsection{Leis de Newton}
As equações presentes nas duas leis de Newton são
\[
	\vec{F} = m \vec{a},\quad \vec{a} = \frac{d \vec{v}}{dt}= \frac{d^{2}\vec{r}}{dt^{2}}.
\]
\subsubsection{Força de Atrito}
Existem dois tipos de forças de atrito - o estático, quando o objeto está parado, para o qual
\[
	F_{at}^{e(max)} = \mu_{e}N
\]
e o cinético, quando o objeto está em movimento, dado por
\[
	F_{at}^{c} = \mu_{c}N.
\]
Lembrando que \(\mu_{c} < \mu_{e}\). Ela é responsável, por exemplo, por fazer
o carro andar, resultante do atrito dele com o chão. Outras formas incluem a força de
arraste, como a resistência do ar.

\subsubsection{Trabalho e Energia Cinética}
\subsubsection{Teorema do Trabalho e Energia Cinética}
Definimos o trabalho realizado por uma força para deslocar um objeto de um caminho 1 até
um ponto 2 como
\[
	W_{1\rightarrow2}^{C} = \int_{1}^{2}\vec{F}\cdot \vec{dr}.
\]
Além disso, o Teorema do trabalho e da energia cinética afirma que, para um objeto indo de um
caminho 1 até um caminho 2 sofrendo a ação de alguma força F, então
\[
	W_{1\rightarrow2} = \Delta E_{cin},
\]
ou seja, o trabalho realizado pela força \(\vec{F}\) é igual à variação
da energia cinética do corpo. Vale lembrar que foi definido que
\[
	E_{cin}=\frac{1}{2}mv^{2}.
\]
\subsubsection{Forças Conservativas}
São aquelas para as quais
\begin{itemize}
	\item[1)] O trabalho não depende do caminho;
	\item[2)] O trabalho em um caminho fechado é nulo;
	\item[3)] Tem energia potencial associada a ela.
\end{itemize}
Essa energia potencial associada é dada por
\[
	U(r) = -\int_{\vec{r}_{0}}^{\vec{r}}\vec{F}\cdot \vec{dr}
\]
sendo \(\vec{r}_{0}\) a posição para \(U(\vec{r}_{0}) = 0.\)

Alguns exemplos de forças conservativas incluem a força peso e a elástica.
\subsubsection{Trabalho e Energia Potencial}
Segue que
\[
	W_{1\rightarrow2}^{C} = -\Delta U = \Delta E_{cin}.
\]
Em particular,
\[
	\Delta U + \Delta E_{cin} = 0.
\]
Definindo a quantidade \(E = U + E_{cin}\) como a energia mecânica, vale que ela
é conservada
\[
	\Delta U + \Delta E_{cin} = \Delta (U + E_{cin}) = \Delta E = 0.
\]
No caso unidimensional, sempre que a força puder ser escrita como \(F = F(x),\)
ela será conservativa (e.g. Força Elástica). Além disso,
\[
	U(x) = - \int_{x_{0}}^{x}F dx \Rightarrow F(x) = -\frac{dU(x)}{dx}
\]
Disto conseguimos encontrar os chamados pontos de equilíbrio, dados pelos valores
de x nos quais \(\frac{dU}{dx} = 0.\) Ademais, diremos que o ponto de equilíbrio é
\begin{itemize}
	\item Estável se \(\frac{d^{2}U}{dx^{2}} > 0\);
	\item Instável se \(\frac{d^{2}U}{dx^{2}} < 0\).
\end{itemize}
\subsubsection{Forças Dissipativas}
Também conhecidas como forças não-conservativas, para elas vale que
\[
	\Delta E = W^{nc}\neq 0.
\]
Forças dissipativas incluem atrito, resistência do ar, entre outras.
\subsection{Potência}
A potência é a taxa de trabalho feita por tempo, ou seja,
\[
	P = \frac{dW}{dt}.
\]
\subsection{Conceitos Necessários}
\subsubsection{Equilíbrio}
Um corpo é dito em equilíbrio se \(F_{R} = 0\). Isso ocorre quando ele
está parado ou se move com velocidade constante.
\subsubsection{Perdeu o contato}
Diremos que um objeto perdeu o contato quando a força normal atuando sobre ele fica nula.
\subsubsection{Movimento circular uniforme}
Um corpo que está em movimento circular uniforme tem como força resultante o
produto da massa pela aceleração centrípeta
\[
	F_{R} = ma_{c}.
\]

\subsection{Exercícios Pré-Prova}
\subsubsection{Ex. 1 - Movimento no Loop}
Temos um carrinho no topo de uma rampa que leva num loop. Suponha que este lugar
em que ele encontra-se está a uma altura H do chão. Além disso, ele parte do repouso, com \(v_{0}=0m/s\). O loop é suposto ter raio R. Qual
é o menor H para o corpo fazer o loop?

Começamos escrevendo as energias presentes no sistema. Chamando de instante 1 o
ponto que o carro está no topo e de instante 2 o ponto crítico do sistema, isto é, o ponto que o carrinho encontra-se no
topo do loop, temos a energia mecânica como
\[
	E_{1} = E_{cin} + U_{1} = 0 + mgH.
\]
No instante 2, ela vale
\[
	E_{2} = E_{cin}^{2} + U_{2} = \frac{1}{2}mv^{2} + mg(2R)
\]
Utilizando a conservação de energia, sabemos que
\[
	E_{1} = E_{2} \Rightarrow mgH = \frac{1}{2}mv^{2} + mg(2R).
\]
Assim,
\begin{align*}
	mgH = \frac{1}{2}mv^{2} + mg(2R) & \Rightarrow g(H-2R)=\frac{1}{2}v^{2} \\
	                                 & \Rightarrow v^{2} = 2g(H-2R).
\end{align*}
Analisando as forças que atuam sobre o carrinho no ponto 2, a peso e a normal estão na mesma direção e,
como a força resultante no mcu é \(F_{R} = ma_{c},\) vale que
\[
	P + N = m_{ac} \Longleftrightarrow mg + N = m \frac{v^{2}}{R}.
\]
Logo, \(N = m\biggl(\frac{v^{2}}{R} - g\biggr)\geq 0\). Já que a massa é sempre positiva,
a condição buscada é totalmente determinada pelo termo entre parênteses. Ao
observá-lo, portanto, chega-se que \(v\geq \sqrt[]{Rg}\) no alto do loop.

Voltando agora à equação anterior, temos , por fim,
\begin{align*}
	 & v^{2} = 2g(H-2R)\geq Rg \\
	 & (H-2R)\geq \frac{R}{2}  \\
	 & H\geq \frac{5R}{2}.
\end{align*}


\end{document}

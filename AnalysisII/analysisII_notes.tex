\documentclass[12pt]{article}
 \usepackage{bookmark}
 \usepackage{amsmath}
 \usepackage[x11names]{xcolor}
 \usepackage{amsthm}
 \usepackage{amssymb}
 \usepackage{tikz}
 \usepackage{pgfplots}
 \usepackage[utf8]{inputenc}
 \usepackage{amsfonts}
 \usepackage{geometry}
 \usepackage[many]{tcolorbox}
 \usepackage{graphicx}
 \usepackage{graphics}
 \usepackage[export]{adjustbox}
 \usepackage{fancyhdr}
 \usepackage[portuguese]{babel}
 \usepackage{hyperref}
 \usepackage{multirow}
 \usepackage{lastpage}
 \usepackage{mathtools}
 \usepackage{newtxsf}
 \usepackage{subfiles}
 \usepackage{flafter}
 \usepackage{float}
 \usepackage{accents}
 \usepackage[T1]{fontenc}
 \setcounter{section}{-1}

 \pagestyle{fancy}
 \fancyhf{}

 \pgfplotsset{compat = 1.18}

 \hypersetup{
     colorlinks,
     citecolor=black,
     filecolor=black,
     linkcolor=blue,
     urlcolor=black
 }
 \newtheorem*{theorem*}{\underline{Teorema}}
 \newtheorem*{lemma*}{\underline{Lema}}
 \newtheorem*{crl*}{\underline{Corolário}}
 \newtheorem*{prop*}{\underline{Proposição}}
 \theoremstyle{definition}
 \newtheorem*{def*}{\underline{Definição}}
 \newtheorem{example}{\underline{Exemplo}}
 \newtheorem*{proof*}{\underline{Prova}}
 \newtheorem{exr}{\underline{Exercício}}
 \renewcommand\qedsymbol{$\blacksquare$}

 \rfoot{Página \thepage \hspace{1pt} de \pageref{LastPage}}

 \geometry{a4paper, left=3cm, top=3cm, right=3cm, bottom=3cm}

\begin{document}
\begin{figure}[ht]
	\minipage{0.76\textwidth}
	\includegraphics[width=4cm]{../icmc.png}
	\hspace{7cm}
	\includegraphics[height=4.9cm,width=4cm]{../brasao_usp_cor.jpg}
	\endminipage
\end{figure}

\begin{center}
	\vspace{1cm}
	\LARGE
	UNIVERSIDADE DE SÃO PAULO

	\vspace{1.3cm}
	\LARGE
	INSTITUTO DE CIÊNCIAS MATEMÁTICAS E COMPUTACIONAIS - ICMC

	\vspace{1.7cm}
	\Large
	\textbf{Notas de Análise II}

	\vspace{1.3cm}
	\large
	\textbf{Renan Wenzel - 11169472}

	\vspace{1.3cm}
	\large
	\textbf{Professor(a): Éder Ritis Aragão Costa}

	\textbf{E-mail: ritis@icmc.usp.br}

	\vspace{1.3cm}
	\today
\end{center}

\newpage
\textbf{{\Huge Avisos}}

{\huge Essas notas não possuem relação com professor algum.

	Qualquer erro é responsabilidade solene do autor.

	Caso julgue necessário, contatar:

	renan.wenzel.rw@gmail.com.

	Além disso, alguns textos em itálico são clicáveis - normalmente, a fim de facilitar o encontro de um resultado, definição ou uma continuação.
}

\tableofcontents

\newpage

\section{Informações Sobre o Curso}
\paragraph{}\textbf{Monitor: Átila Correia, atilapc1989@gmail.com. Monitorias às sextas-feiras, às 18h.}

Serão feitas 2 provas e uma sub (do bem), com datas
\begin{itemize}
	\item[Prova 1)] 30/04/2025 - Quarta-feira;
	\item[Prova 2)] 07/07/2025 - Segunda-feira;
	\item[Prova S)] 14/07/2025 - Segunda-feira.
\end{itemize}

A ementa que esperamos cobrir é composta por:
\begin{itemize}
	\item[1)] Integral de Riemann;
	\item[2)] Sequências e Séries de Funções;
	\item[3)] Diferenciabilidade de Funções de \(\mathbb{R}^{m}\) em \(\mathbb{R}^{n}\).
\end{itemize}

O propósito do curso é, no fim, fazer um tratamento rigoroso das técnicas de manipulação e questões de existência apresentadas durante os cursos usuais de Cáculo, ou seja, é um curso sobre demonstrações de Teoremas.

Comunicação com o professor pode ser feita pelo e-mail (ritis@icmc.usp.br), ou marcando horário e encontrando o professor em sua sala (4-137, ou bloco quatro, andar 1, sala 37). Além disso, tem um grupo no Telegram da disciplina.

\subfile{classes/aula01.tex}
\newpage
\subfile{classes/aula02.tex}
\newpage
\subfile{classes/aula03.tex}
\newpage
\subfile{classes/aula04.tex}
\newpage
\subfile{classes/aula05.tex}
\newpage
\subfile{classes/aula06.tex}
\newpage
\subfile{classes/aula07.tex}
\newpage
\subfile{classes/aula08.tex}
\newpage
\subfile{classes/aula09.tex}
\newpage
\subfile{classes/aula10.tex}
\newpage
\subfile{classes/aula11.tex}
\newpage
\subfile{classes/aula12.tex}
\newpage
\subfile{classes/aula13.tex}
\newpage
\subfile{classes/aula14.tex}
\newpage
\subfile{classes/aula15.tex}
\newpage
\subfile{classes/aula16.tex}
\newpage
\subfile{classes/aula17.tex}
\newpage
\subfile{classes/aula18.tex}
\newpage
\subfile{classes/aula19.tex}
\newpage
\subfile{classes/aula20.tex}
\newpage
\subfile{classes/aula21.tex}
\newpage
\subfile{classes/aula22.tex}
\newpage
\subfile{classes/aula23.tex}
\newpage
\subfile{classes/aula24.tex}
\newpage
\subfile{classes/aula25.tex}
\newpage
\subfile{classes/aula26.tex}
\newpage

\begin{thebibliography}{99}
	\bibitem{apostol} APOSTOL, T. M. \textbf{Análisis matemático}. Espanha: Reverté, 1976.
	\bibitem{elon1} LIMA, E. L. \textbf{Curso de Análise}, Volume I. Rio de Janeiro: IMPA, 2019.
	\bibitem{elon2} LIMA, E. L. \textbf{Curso de Análise}, Volume II. Rio de Janeiro: IMPA, 2020.
	\bibitem{elon3} LIMA, E. L. \textbf{Análise Real}, Volume II. Rio de Janeiro: IMPA, 2016.
	\bibitem{elon4} LIMA, E. L. \textbf{Análise no Espaço} \(\mathbb{R}^{n}\). Rio de Janeiro: IMPA, 2016.
	\bibitem{royden} ROYDEN, H. L; FITZPATRICK, P. M. \textbf{Real Analysis}. New Jersey: Pearson Education, 2010.
	\bibitem{rudin} RUDIN, W. \textbf{Principles of Mathematical Analysis}. Nova Iorque: Mcgraw-Hill, 1976.
\end{thebibliography}
 
\end{document}

\documentclass[../analysisII_notes.tex]{subfiles}
\begin{document}
\section{Aula 08 - 31 de Março, 2025}
\subsection{Motivações}
\begin{itemize}
	\item Critérios de Integrabilidade de Riemann;
	\item Classes de Funções Integráveis.
\end{itemize}
\subsection{Critérios para Existência da Integral - Parte 2}
Conforme vimos na aula passada, dada uma função limitada \(f:[a, b]\rightarrow \mathbb{R}\), ao colocar
\[
	A = \sigma (f)\quad\&\quad B=\Sigma (f),
\]
vimos que para quaisquer elementos de A e B, tem-se \(a\leq b\), em particular para
\[
	a = L(f; \mathcal{P}),\quad b=U(f; \mathcal{Q}).
\]
Logo, \(\sigma (f)\) e \(\Sigma (f)\) estão nas hipóteses do lema, donde segue que
\[
	\underline{\intup_{a}^{b}}f = \sup_{}\sigma (f)\leq \inf_{}\Sigma (f) = \overline{\intup_{a}^{b}}f.
\]

Por outro lado, a definição de integrabilidade para f significa que
\[
	\sup_{}\sigma (f) = \underline{\intup_{c}^{b}}f = \overline{\intup_{a}^{b}} f = \inf_{}\Sigma (f),
\]
donde outra aplicação do lema oferece uma condição de integrabilidade da forma ``Dado \(\varepsilon >0\), devem existir \(a = L(f; \mathcal{P})\in\sigma (f)\) e \(b=U(f; \mathcal{Q})\in \Sigma (f)\) tais que
\[
	U(f; \mathcal{Q}) - L(f; \mathcal{P})<\varepsilon \text{''.}
\]

\hypertarget{integrability_conditions}{
	\begin{theorem*}[Condições de Integrabilidade]
		Seja \(f:[a, b]\rightarrow \mathbb{R}\) uma função limitada. As seguintes propriedades são equivalentes:
		\begin{itemize}
			\item[1)] f é integrável;
			\item[2)] Dado \(\varepsilon > 0\), existem partições \(\mathcal{P}, \mathcal{Q}\) tais que
			      \[
				      U(f; \mathcal{P}) - L(f; \mathcal{Q})<\varepsilon ;
			      \]
			\item[3)] Dado \(\varepsilon > 0\), existe uma partição \(\mathcal{P}_{0}\) de \([a, b]\) tal que
			      \[
				      U(f; \mathcal{P}_{0}) - L(f; \mathcal{P}_{0})<\varepsilon;
			      \]
			\item[4)] Dado \(\varepsilon > 0\), existe uma partição
			      \[
				      \mathcal{P}: a = t_{0}<t_1 <\dotsc <t_{n} = b
			      \]
			      tal que, se \(\omega_{i} = M_{i}-m_{i}\) é a oscilação de f dentro de \([t_{i-1}, t_{i}]\), então
			      \[
				      \sum\limits_{i=1}^{n}\omega_{i}(t_{i}-t_{i-1})<\varepsilon .
			      \]
		\end{itemize}
	\end{theorem*}
}
\begin{proof*}
	Para ver que 1 e 2 são equivalentes, foi justificado na observação anterior.

	Agora, quanto à implicação de 3 para a 2, basta tomar
	\[
		\mathcal{Q} = \mathcal{P} = \mathcal{P}_{0}.
	\]

	Com respeito à equivalência entre 3 e 4,
	\begin{align*}
		U(f; \mathcal{P}) - L(f; \mathcal{P}) & = \sum\limits_{i=1}^{n}M_{i}(t_{i}-t_{i-1})-\sum\limits_{i=1}^{n}m_{i}(t_{i}-t_{i-1}) \\
		                                      & = \sum\limits_{i=1}^{n}(M_{i}-m_{i})(t_{i}-t_{i-1}),
	\end{align*}
	mostrando que as duas são equivalentes, mudando apenas a forma de escrever os termos.

	Sendo assim, resta apenas demonstrar que 2 implica 3 para completar o ciclo causal. Com efeito, dado \(\varepsilon \) positivo, sejam duas partições \(\mathcal{P}, \mathcal{Q}\) do intervalo \([a, b]\) tais que
	\[
		U(f; \mathcal{P})-L(f; \mathcal{Q})<\varepsilon .
	\]
	Como
	\[
		U(f; \mathcal{P}\cup \mathcal{Q})\leq U(f; \mathcal{P})\quad\&\quad L(f; \mathcal{Q})\leq L(f; \mathcal{P}\cup \mathcal{Q}),
	\]
	conclui-se que
	\[
		U(f; \mathcal{P}\cup \mathcal{Q})-L(f; \mathcal{P}\cup \mathcal{Q})\leq U(f; \mathcal{P})-L(f; \mathcal{Q})<\varepsilon,
	\]
	que é exatamente o que queríamos mostrar, mas com \(\mathcal{P}_{0} = \mathcal{P}\cup \mathcal{Q}.\) \qedsymbol
\end{proof*}

Tendo em mãos um critério teórico, podemos começar a discutir quais são os tipos de funções cuja integral existe. Após isso, analisaremos um critério muito mais útil na prática, devido ao Lebesgue, mas cujo arcabouço matemático ainda não temos, e iremos trabalhar em seguida. Sendo assim, vejamos alguns exemplos das oscilações antes de vermos as tais classes:
\begin{example}
	Seja \(f:[a, b]\rightarrow \mathbb{R}\) uma função contínua e X um subconjunto compacto do intervalo \([a, b]\) em que f está definida. Pelo \hyperlink{weierstrass_theorem}{\textit{Teorema de Weierstrass}}, existem elementos \(x_{1},\; x_{2}\) em X tais que
	\[
		\sup_{}f(X) = f(x_{2}) \quad\&\quad \inf_{}f(X) = f(x_{1}).
	\]
	Daí,
	\[
		\omega (f; X) = \sup_{}f(X) - \inf_{}f(X) = f(x_{2})-f(x_{1}),
	\]
	o que significa que, quando X for compacto, a oscilação é assumida por valores dentro dele.
\end{example}
\begin{tcolorbox}[
		skin=enhanced,
		title=Lembrete!,
		after title={\hfill Teorema de Weierstrass},
		fonttitle=\bfseries,
		sharp corners=downhill,
		colframe=black,
		colbacktitle=yellow!75!white,
		colback=yellow!30,
		colbacklower=black,
		coltitle=black,
		%drop fuzzy shadow,
		drop large lifted shadow
	]
	O \hypertarget{weierstrass_theorem}{Teorema de Weierstrass} afirma que, dado um compacto K, se uma função \(f:K\rightarrow \mathbb{R}\) for uma função contínua, então ela é limitada e cobre os valores do seu supremo e seu ínfimo: existem c e d dentro de K, tais que
	\[
		f(c)=\sup_{x\in K}f(x)\quad\&\quad f(d)=\inf_{x\in K}f(x).
	\]
\end{tcolorbox}
\begin{example}
	Agora, seja \(f:[a, b]\rightarrow \mathbb{R}\) uma função monótona e X um subconjunto compacto do intervalo \([a, b]\) em que f está definida. Sem perda de generalidade, assuma que f é crescente; então, para cada \(x\) dentro do intervalo \([a, b]\), vale que
	\[
		f(a)\leq f(x) \quad\&\quad f(b)\geq f(x),
	\]
	indicando que a função ``preserva as cotas'' do domínio em sua imagem (se fosse decrescente, ela invereria as cotas!). Além disso, como a e b são, respectivamente, os supremos e ínfimos do intervalo \([a, b]\), vale que, dado \(\varepsilon > 0\), existem \(x_{0}\) e \(y_{0}\) dentro do intervalo satisfazendo
	\[
		a-\varepsilon > x_{0} \geq a \quad\&\quad b-\varepsilon < y_{0} \leq b.
	\]
	Utilizando a monotonicidade novamente, segue que
	\[
		f(a-\varepsilon )> f(x_{0}) > f(a) \quad\&\quad f(b-\varepsilon )< f(y_{0}) < f(b).
	\]
	Como \(a-\varepsilon \) é maior que a, \(f(a-\varepsilon )\) também é maior que \(f(a)\), ou seja,
	\[
		f(a)-f(a-\varepsilon ) > 0,
	\]
	e podemos tomar \(\varepsilon' = f(a)-f(a-\varepsilon ) > 0\), tal que, para algum \(x_{0}\) no domínio da f,
	\[
		f(a)-\varepsilon ' > f(x_{0}) > f(a),
	\]
	mostrando que
	\[
		f(a)=\inf_{x\in [a, b]}f(x)
	\]. Analogamente, mostra-se que \(f(b)=\sup_{x\in [a, b]}f(x)\). Portanto,
	\[
		\omega (f; [a, b])=\sup_{}f([a, b])-\inf_{}f([a, b]) = f(b)-f(a).
	\]
\end{example}

\subsection{Classes de Funções Integráveis}
Estando mais familiarizados com o conceito de oscilação, embarquemos na jornada pelas classes de funções integráveis. O primeiro deles diz respeito à continuidade como condição suficiente, mas não necessário, para integração\footnote{Essa frase vem da forma de ler um ``se, e somente se'' como ``é necessário E suficiente''. Dizer que ela é suficiente siginifica, em termos simples, exatamente isso: ser contínua é suficiente para uma função ser integrável, mas existem funções que não são contínuas e são integráveis. De certa forma, podemos ver esta parte da jornada como a busca pelas condições necessárias e suficientes para uma função ser Riemann-integrável, a fim de caracterizar completamente esta classe de funções.}
\begin{theorem*}
	Toda função \(f:[a, b]\rightarrow \mathbb{R}\) contínua é integrável em \([a, b]\):
	\[
		\mathcal{C}([a, b])\subseteq \mathcal{R}([a, b]).
	\]
\end{theorem*}
\begin{proof*}
	Antes de mais nada, observe que f é limitada pelo \hyperlink{weierstrass_theorem}{\textit{teorema de Weierstrass}}, então faz sentido falar sobre integrabilidade. Além disso, toda função contínua num compacto é \textit{uniformemente contínua} - basta um natural \(N\), dependente apenas de dado \(\varepsilon \) positivo, para determinar como é o comportamento da função em sua imagem conforme os valores em seu domínio aproximam-se uns dos outros.

	Nessas condições, dado \(\varepsilon \) positivo associado à continuidade uniforme de f, existe um \(\delta \) positivo tal que, sempre que \(|x-y|\) for menor que \(\delta \),
	\[
		|f(x)-f(y)|<\frac{\varepsilon }{b-a}.
	\]
	Daí, se fixamos uma partição do intervalo da forma
	\[
		\mathcal{P}_{0}: a = t_{0} < t_{1}<t_2<\dotsc <t_{n}=b,
	\]
	com \(t_{i}-t_{i-1}\) menor que \(\delta \) para todos os índices, resulta
	\[
		x, y\in [t_{i-1}, t_{i}] \Rightarrow |x-y|<\delta  \Rightarrow |f(x)-f(y)|<\frac{\varepsilon }{b-a}.
	\]
	Isto é diretamente traduzido em uma limitação na oscilação dessas funções:
	\[
		\omega_{i}=\sup_{}\{|f(x)-f(y)|:\: x, y\in [t_{i-1}, t_{i}]\} < \frac{\varepsilon }{b-a}.
	\]
	Dessa forma,
	\[
		U(f; \mathcal{P})-L(f; \mathcal{P})=\sum\limits_{i=1}^{n}\omega_{i}(t_{i}-t_{i-1}) < \sum\limits_{i=1}^{n}\frac{\varepsilon }{b-a}(t_{i}-t_{i-1})=\frac{\varepsilon }{b-a}\sum\limits_{i=1}^{n}(t_{i}-t_{i-1})=\varepsilon .
	\]
	Portanto, pelo \hyperlink{integrability_conditions}{\textit{critério de integração 4}}, f é integrável. \qedsymbol
\end{proof*}
\begin{tcolorbox}[
		skin=enhanced,
		title=Observação,
		fonttitle=\bfseries,
		colframe=black,
		colbacktitle=cyan!75!white,
		colback=cyan!15,
		colbacklower=black,
		coltitle=black,
		drop fuzzy shadow,
		%drop large lifted shadow
	]
	Para obter a partição com a propriedade específicada usada no teorema acima, basta escolher um natural n que satisfaça
	\[
		\frac{b-a}{n}<\delta
	\]
	e, dividindo o intervalo \([a, b]\) em partes iguais de comprimento \((b-a)/n\), segue que obtemos tal partição por meio de
	\[
		t_{i}=a + i \frac{b-a}{n},\quad i=1, 2,\dotsc ,n.
	\]
\end{tcolorbox}
\begin{theorem*}
	Se \(f:[a, b]\rightarrow \mathbb{R}\) é uma função monótona, então ela é integrável.
\end{theorem*}
\begin{proof*}
	Sem perda de generalidade, assuma que f é não-decrescente, ou seja,
	\[
		x<y \Rightarrow f(x)\leq f(y),
	\]
	pois, caso fosse não-crescente, bastaria considerarmos -f e aplicar que f é integrável se, e somente se, -f for.

	Feito este adendo, um ponto x entre a e b satisfaz
	\[
		a \leq x \leq b \Rightarrow f(a)\leq f(x)\leq f(b),
	\]
	que garante a limitação de f para podermos começar a discussão sobre integrabilidade. Destaquemos também que, se
	\[
		\mathcal{P}: a = t_{0} < t_{1} < \dotsc < t_{n}=b
	\]
	for qualquer partição, então
	\[
		\omega_{i}=M_{i}-m_{i} = f(t_{i})-f(t_{i-1}) \quad \forall i=1,2,\dotsc ,n.
	\]
	Logo,
	\begin{align*}
		\sum\limits_{i=1}^{n}\omega_{i} & = \sum\limits_{i=1}^{n}[f(t_{i})-f(t_{i-1})]                                   \\
		                                & = [f(t_{1}) - f(t_{0})] + [f(t_{2})-f(t_{1})] + \dotsc + [f(t_{n})-f(t_{n-1})] \\
		                                & = f(t_{n})-f(t_{0}) = f(b)-f(a).
	\end{align*}
	Podemos também supor que \(f(a)\) é estritamente menor que \(f(b)\); caso contrário, pela não crescência, a função seria constante.

	Agora, dado \(\varepsilon \) positivo, fixamos uma partição
	\[
		\mathcal{P}_{0}: a = t_{0} < t_{1} < \dotsc < t_{n}=b
	\]
	tal que
	\[
		t_{i}-t_{i-1}<\frac{\varepsilon }{f(b)-f(a)},\quad i=1,2,\dotsc ,n
	\]
	e seguirá que
	\begin{align*}
		U(f; \mathcal{P}_{0})-L(f; \mathcal{P}_{0}) & = \sum\limits_{i=1}^{n}\omega_{i}(t_{i}-t_{i-1})                 \\
		                                            & < \sum\limits_{i=1}^{n}\omega_{i} \frac{\varepsilon }{f(b)-f(a)} \\
		                                            & = \frac{\varepsilon }{f(b)-f(a)}\sum\limits_{i=1}^{n}\omega_{i}  \\
		                                            & = \frac{\varepsilon }{f(b)-f(a)}(f(b)-f(a))= \varepsilon .
	\end{align*}
	Portanto, novamente pelas \hyperlink{integrability_conditions}{\textit{condições de integrabilidade}}, f é integrável em \([a, b]\). \qedsymbol
\end{proof*}
Note que, no caso da continuidade, tinhamos controlado o tamanho das oscilações locais da função sendo estudada; em contrapartida, na monotonicidade, controlamos o ``tamanho'' da partição.
\end{document}

\documentclass[../analysisII_notes.tex]{subfiles}
\begin{document}
\section{Aula 21 - 02 de Junho, 2025}
\subsection{Motivações}
\begin{itemize}
	\item Familiarizando com Raio de Convergência;
	\item Série de Taylor.
\end{itemize}
\subsection{As Séries de Taylor: Versão Potência.}
Na aula passada, vimos que, como consequência do teste da raiz para séries numéricas, é possível associar um intervalo I a cada série de potências, consistindo dos números reais x tais que ela é convergente, donde saíram os nomes de raio de convergência e intervalo de convergência da série. A partir disso, podemos definir
\begin{def*}
	Se \(R > 0\) for um número positivo, então \(f:(x_{0}-R, x_{0}+R)\rightarrow \mathbb{R}\) dada por
	\[
		f(x)\coloneqq \sum\limits_{n=0}^{\infty}a_{n}(x-x_{0})^{n}
	\]
	é uma função contínua chamada de \textbf{Série de Taylor da função f em torno de }\(x_{0}.\; \square\)
\end{def*}
O que mostraremos a seguir é que uma função f que pode ser escrita como uma série de Taylor é infinitamente diferenciável no seu intervalo de convergência. O ponto chave consistirá em estabelecer que, dada a hipótese da série de potências de f possuir um raio de convergência, então a série das derivadas, que também será uma série de potências, possuirá o mesmo raio de convergência.
\begin{tcolorbox}[
		skin=enhanced,
		title=Observação,
		fonttitle=\bfseries,
		colframe=black,
		colbacktitle=cyan!75!white,
		colback=cyan!15,
		colbacklower=black,
		coltitle=black,
		drop fuzzy shadow,
		%drop large lifted shadow
	]
	Dado um \(x\) real e diferente de zero, segue das propriedades operatórias das séries numéricas que
	\[
		\sum\limits_{n=1}^{\infty}na_{n}x^{n-1}
	\]
	converge se, e somente se,
	\[
		x \sum\limits_{n=1}^{\infty}na_{n}x^{n-1} = \sum\limits_{n=1}^{\infty}na_{n}x^{n}
	\]
	converge.
\end{tcolorbox}
Sem mais delongas, vamos nos restringir ao caso \(x_{0} = 0\) sem perda de generalidade, com o caso geral sendo um exercício.
\begin{lemma*}
	Dado um número positivo \(R > 0\), então as séries
	\[
		\sum\limits_{}^{}a_{n}x^{n}\quad\&\quad \sum\limits_{}^{}na_{n}x^{n-1}
	\]
	possuem o mesmo raio de convergência.
\end{lemma*}
\begin{proof*}
	Antes de mais nada, denote por R o raio de convergência da série \(\sum\limits_{}^{}a_{n}x^{n}\). Feito isso, comecemos calculando o raio de convergência da série
	\[
		\sum\limits_{n=1}^{\infty}na_{n}x^{n} = 0 + a_{1}x + 2a_2x^{2} + 3a_3x^{3} +\dotsc .
	\]
	Para isso, identifiquemos
	\[
		b_{n} = \left\{\begin{array}{ll}
			0,      & \quad n=0            \\
			na_{n}, & \quad n=1, 2, \dotsc
		\end{array}\right.
	\]
	como sendo a sequência de coeficientes da série. Com isso, seu raio de convergência R' será dado por
	\begin{align*}
		\frac{1}{R'} = \limsup_{\to }|b_{n}|^{\frac{1}{n}} & = \limsup_{\to }|na_{n}|^{\frac{1}{n}}                              \\
		                                                   & = \limsup_{\to }[(n^{\frac{1}{n}})|a_{n}|^{\frac{1}{n}}]            \\
		                                                   & = (\lim_{\to }n^{\frac{1}{n}})(\limsup_{\to }|a_{n}|^{\frac{1}{n}}) \\
		                                                   & = \limsup_{\to }|a_{n}|^{\frac{1}{n}}=\frac{1}{R},
	\end{align*}
	pois \(\lim_{\to }n^{\frac{1}{n}} = 1\) e onde usamos a propriedade do limsup de
	\[
		\limsup_{\to }(y_{n}x_{n}) = (\lim_{\to }y_{n})(\limsup_{\to }x_{n}).
	\]
	Logo,
	\[
		\frac{1}{R'} = \frac{1}{R} \Longleftrightarrow R = R',
	\]
	que, junto à observação, permite concluirmos que
	\begin{align*}
		 & |x| < R \Rightarrow \sum\limits_{n=1}^{\infty}na_{n}x^{n-1} \text{ converge absolutamente; e} \\
		 & |x| > R \Rightarrow \sum\limits_{n=1}^{\infty}na_{n}x^{n-1} \text{ diverge.}
	\end{align*}
	Portanto, R deve ser de fato o raio de convergência de \(\sum\limits_{}^{}na_{n}x^{n-1}.\) \qedsymbol
\end{proof*}
\begin{theorem*}
	Se \(R > 0\), então a função \(f:(-R, R)\rightarrow \mathbb{R}\) definida pela série de potências
	\[
		f(x)\coloneqq \sum\limits_{}^{}a_{n}x^{n}
	\]
	é infinitamente derivável, com suas derivadas sucessivas dadas por
	\[
		f^{(i)}(x) = \sum\limits_{n=1}^{\infty}n(n-1)\dotsc (n-(i-1))a_{n}x^{n-i},\quad |x|<R.
	\]
	Além disso,
	\[
		a_{n} = \frac{f^{(n)}(0)}{n!},\quad n=0,1,2,\dotsc .
	\]
\end{theorem*}
\begin{proof*}
	Com efeito, pelas propriedades da convergência uniforme e do fato de
	\[
		f(x)=\sum\limits_{}^{}a_{n}x^{n} \quad\&\quad f'(x) = \sum\limits_{}^{}(a_{n}x^{n})' = \sum\limits_{}^{}na_{n}x^{n-1}
	\]
	terem o mesmo raio de convergência, o resultado é que \(\sum\limits_{}^{}na_{n}x^{n}\) converge uniformemente em \([-r, r]\subseteq (-R, R)\) para todo r estritamente entre 0 e R. Logo, colocando
	\[
		g(x)\coloneqq \sum\limits_{n=1}^{\infty}na_{n}x^{n-1},\quad |x|<R,
	\]
	segue que f é derivável e que \(f'= g\) em \((-R, R)\); em particular, \(f'\) é contínua com
	\[
		f'(x) = \sum\limits_{n=1}^{\infty}na_{n}x^{n-1} = g(x).
	\]

	Num processo análogo ao feito acima, \(\sum\limits_{n=1}^{\infty}na_{n}x^{n-1}\) é uma série de potências com raio de convergência \(R > 0\), resultando na série de suas derivadas
	\[
		\sum\limits_{n=2}^{\infty}n(n-1)a_{n}x^{n-2} = \sum\limits_{n=1}^{\infty}(na_{n}x^{n-1})'
	\]
	tendo raio de convergência R pelo lema provado. Consequentemente, \(f'\) é derivável com derivada
	\[
		f''(x) = \sum\limits_{n=2}^{\infty}n(n-1)a_{n}x^{n-2},\quad |x| < R,
	\]
	ou seja, ela é dada por uma série de potências com o mesmo raio de convergência \textit{também}.

	Repetindo este processo analogamente para todo natural i, conclui-se que f é uma função suave (\(\mathcal{C}^{\infty}\)) em \((-R, R)\), com derivadas dadas pela fórmula
	\[
		f^{(i)}(x) = \sum\limits_{n=1}^{\infty}n(n-1)\dotsc (n-(i-1))a_{n}x^{n-i},\quad |x|<R.
	\]
	Particularmente, ao selecionar a i-ésima derivada de f em \(x=0\), a expressão fica
	\begin{align*}
		f^{(i)}(0) & = i(i-1)(i-2)\dotsc (i-i+1)a_{i}0^{i-i} + \sum\limits_{n=i+1}^{\infty}n(n-1)\dotsc (n-i+1)a_{n}0^{n-i} \\
		           & = i!a_{i} \Rightarrow a_{i} = \frac{f^{(i)}(0)}{i!},
	\end{align*}
	concluindo, portanto, a prova. \qedsymbol
\end{proof*}
\begin{tcolorbox}[
		skin=enhanced,
		title=Observação,
		fonttitle=\bfseries,
		colframe=black,
		colbacktitle=cyan!75!white,
		colback=cyan!15,
		colbacklower=black,
		coltitle=black,
		drop fuzzy shadow,
		%drop large lifted shadow
	]
	Como de costume, para simplifica a notação, convencionamos \(0^{0} = 1\) para não ficar escrevendo
	\[
		\sum\limits_{n=0}^{\infty}a_{n}x^{n} = a_{0} + \sum\limits_{n=1}^{\infty}a_{n}x^{n}.
	\]
\end{tcolorbox}
\end{document}

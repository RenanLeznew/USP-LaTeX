\documentclass[../analysisII_notes.tex]{subfiles}
\begin{document}
\section{Aula 18 - 14 de Maio, 2025}
\subsection{Motivações}
\begin{itemize}
	\item Diferenciação e Convergência Uniforme.
\end{itemize}
\subsection{Diferenciação e Convergência Uniforme.}
\begin{example}
	Seja \(s_{n}:(-1, 1)\rightarrow \mathbb{R}\) dada por
	\[
		s_{n}(x) = \sum\limits_{i=0}^{n}x^{i},
	\]
	em que n é um natural. Vimos que ela converge pontualmente para g em \((-1, 1)\), em que g é a função
	\[
		g(x) = \frac{1}{1-x}.
	\]
	Temos, para um valor r entre 0 e 1, a convergência uniforme de \(s_{n}\) para g
	\[
		s_{n}\overbracket[0pt]{\longrightarrow}^{u}g
	\]
	no intervalo \([-r, r]\). Com efeito, quando x pertence ao intervalo \([-r, r]\) e m é maior que n, então
	\[
		|s_{m}(x)-s_{n}(x)| = \biggl\vert \sum\limits_{i=0}^{m}x^{i}-\sum\limits_{i=0}^{n}x^{i} \biggr\vert = \biggl\vert \sum\limits_{i=n+1}^{m}x^{i} \biggr\vert \leq \sum\limits_{i=n+1}^{m}|x|^{i} \leq \sum\limits_{i=1}^{r}r^{i}.
	\]
	Como o último termo da desigualdade é uma série geométrica cujo termo é menor que 1, segue que, para n suficientemente grande,
	\[
		\sum\limits_{i=1}^{m}r^{i}\leq \sum\limits_{i=n+1}^{\infty}r^{i} = \frac{r^{n+1}}{1-r} < \varepsilon.
	\]
	Consequentemente, dado \(\varepsilon \) positivo, existe um valor \(n_{0}\) natural tal que
	\[
		m>n\geq n_{0}\Rightarrow |s_{m}(x)-s_{n}(x)| < \varepsilon , \quad \forall x\in [-r, r].
	\]
	Portanto, \(s_{n}\) é de Cauchy em \([-r, r]\), o que equivale à convergência uniforme.

	Nessa mesma linha, considere a adaptação do exemplo dada por
	\[
		f_{n}(x) = \sum\limits_{i=0}^{n}(-1)^{i}x^{i},\quad x\in (-1, 1),
	\]
	resultando numa série alternada que coincide com
	\[
		s_{n}(-x) = f_{n}(x).
	\]
	Da mesma forma que fora previamente feito, pode-se provar que \(\{f_{n}\}_{n}\) converge uniformemente para alguma f dentro do intervalo \([-r, r]\), a qual necessariamente deverá ser
	\[
		f(x) = \frac{1}{1+x},\quad x\in (-1, 1).
	\]
	Daí, pelo teorema 2 da aula passada, se t for um valor entre -1 e 1, então podemos calcular
	\begin{align*}
		\int_{0}^{t}\frac{1}{1+x}= \mathrm{dx}\int_{0}^{t}f(x) \mathrm{dx} & = \lim_{n\to \infty}\int_{0}^{t}f_{n}(x) \mathrm{dx}                                                           \\
		                                                                   & = \lim_{n\to \infty}\int_{0}^{t}\biggl[\sum\limits_{i=0}^{n}(-1)^{i}x^{i}\biggr] \mathrm{dx}                   \\
		                                                                   & = \lim_{n\to \infty}\biggl[\sum\limits_{i=0}^{n}(-1)^{i}\int_{0}^{t}x^{i} \mathrm{dx}\biggr]                   \\
		                                                                   & = \lim_{n\to \infty}\sum\limits_{i=0}^{n}(-1)^{i}\frac{t^{i+1}}{i+1}                                           \\
		                                                                   & = \sum\limits_{i=1}^{\infty}(-1)^{i}\frac{t^{i+1}}{i+1} = \sum\limits_{n=1}^{\infty}(-1)^{n+1}\frac{t^{n}}{n}.
	\end{align*}
	Assim, mostramos que, se o t for um valor entre -1 e 1, mostramos que
	\[
		\sum\limits_{n=1}^{\infty}(-1)^{n+1}\frac{t^{n}}{n} = \int_{0}^{t}\frac{1}{1+x} \mathrm{dx} = \log^{}{(1+t)}.
	\]

	Tomando, por exemplo, \(t=-1/2\), temos
	\begin{align*}
		\log^{}{\biggl(\frac{1}{2}\biggr)} & = \sum\limits_{n=1}^{\infty}(-1)^{n+1}\frac{\biggl(\frac{1}{2}\biggr)^{n}}{n} \\
		                                   & = \sum\limits_{n=1}^{\infty}(-1)^{n+1}(-1)^{n}\frac{1}{2^{n}}\frac{1}{n}      \\
		                                   & = -\sum\limits_{n=1}^{\infty}\frac{1}{n2^{n}}.
	\end{align*}
	Logo,
	\[
		\log^{}{(2)} = \sum\limits_{n=1}^{\infty}\frac{1}{n2^{n}}.
	\]
\end{example}
\begin{theorem*}
	Se \(\{f_{n}\}_{n}\) for uma sequência de funções \(f_{n}:[a, b]\rightarrow \mathbb{R}\) de classe \(\mathcal{C}^{1}([a, b])\) tais que
	\begin{itemize}
		\item[i)] Existe um valor c no intervalo \([a, b]\) tal que
		      \[
			      f_{n}(c)\overbracket[0pt]{\longrightarrow}^{n\to \infty}L\in \mathbb{R};\&
		      \]
		\item[ii)] Existe uma função g em \([a, b]\) tal que
		      \[
			      f_{n}'\overbracket[0pt]{\longrightarrow}^{u}g,
		      \]
	\end{itemize}
	então existe uma função f de classe \(\mathcal{C}^{1}([a, b])\) tal que:
	\begin{itemize}
		\item[I)] A sequência \(\{f_{n}\}_{n}\) converge uniformemente para f em \([a, b]\); e
		\item[II)] A derivada da f em \([a, b]\) é igual à g:
		      \[
			      f'=g.
		      \]
	\end{itemize}
\end{theorem*}
Em outras palavras, este teorema fala que
\[
	\frac{\mathrm{d}}{\mathrm{d}x}[\lim_{}f_{n}] = \frac{\mathrm{d}f}{\mathrm{d}x} = g = \lim_{}\frac{\mathrm{d}f_{n}}{\mathrm{d}x},
\]
ou seja, a derivada e o limite, nas condições do teorema, ``comutam''.
\begin{proof*}
	Primeiramente, pelo \hyperlink{ftc}{\textit{Teorema Fundamental do Cálculo}}, se x for um valor no intervalo \([a, b]\), temos
	\[
		f_{n}(x) = f_{n}(c) + \int_{c}^{x}f_{n}'(y) \mathrm{dy}.
	\]
	Além disso, a segunda hipótese imposta sobre as \(f_{n}\)'s, ganhamos a continuidade e integrabilidade da g, tal que
	\[
		\int_{c}^{x}f_{n}'(y) \mathrm{dy}\overbracket[0pt]{\longrightarrow}^{n\to \infty}\int_{c}^{x}g(y) \mathrm{dy}.
	\]
	Por hipótese, sabemos que
	\[
		\lim_{n\to \infty}f_{n}(c) = L,
	\]
	ou seja, juntando tudo, segue que
	\[
		\lim_{n\to \infty}f_{n}(x) = L + \int_{c}^{x}g(y) \mathrm{dy} \eqqcolon f,
	\]
	que define a tal função \(f:[a, b]\rightarrow \mathbb{R}\) como o limite pontual das \(f_{n}\)'s. Temos nosso candidato à função f!

	Ela é \(\mathcal{C}^{1}([a, b])\)? Sim! Por ser definida como uma integral indefinida de uma função contínua, o \hyperlink{ftc}{\textit{Teorema Fundamental do Cálculo}} garante que f é contínua e tem derivada contínua.

	Finalmente, precisamos mostrar que a convergência é uniforme; com efeito, tomando um x qualquer em \([a, b]\), podemos escrever
	\begin{align*}
		|f_{n}(x)-f(x)| & = \biggl\vert (L_{n} - f(c)) + \int_{c}^{x}(f_{n}'(y)-g(y)) \mathrm{dy} \biggr\vert \\
		                & \leq |f_{n}(c)-f(c)| + (b-a)\sup_{y\in[a, b]}|f_{n}'(y) - g(y)|                     \\
		                & \overbracket[0pt]{\longrightarrow}^{n\to \infty}0 + 0 = 0.
	\end{align*}
	Portanto,
	\[
		\sup_{x\in [a, b]}|f_{n}(x)-f(x)|\leq |f_{n}(c)-L|+\sup_{y\in [a, b]}|f_{n}'(y)-g(y)|.\quad \text{\qedsymbol}
	\]
\end{proof*}
\begin{example}
	Seja
	\[
		f_{n}(x) = \sum\limits_{i=0}^{n}\frac{x^{i}}{i!},\quad x\in \mathbb{R},\; n\in \mathbb{N}.
	\]
\end{example}
\end{document}

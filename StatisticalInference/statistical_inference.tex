\documentclass[12pt]{article}
 \usepackage{bookmark}
 \usepackage{amsmath}
 \usepackage{amsthm}
 \usepackage{amssymb}
 \usepackage{tikz}
 \usepackage{tikz-cd}
 \usepackage{pgfplots}
 \usepackage[utf8]{inputenc}
 \usepackage{amsfonts}
 \usepackage{nicematrix}
 \usepackage[x11names]{xcolor}
 \usepackage{geometry}
 \usepackage{graphicx}
 \usepackage{graphics}
 \usepackage[export]{adjustbox}
 \usepackage{fancyhdr}
 \usepackage[portuguese]{babel}
 \usepackage{hyperref}
 \usepackage{multirow}
 \usepackage{lastpage}
 \usepackage{mathtools}
 \usepackage[many]{tcolorbox}
 \usepackage{sansmathfonts}
 \usepackage[T1]{fontenc}
 \usepackage{subfiles}
 \usepackage{flafter}
 \usepackage{float}
 \usepackage{accents}

\usetikzlibrary{positioning, calc, shapes.geometric, shapes.multipart, shapes, arrows.meta, arrows, decorations.markings, external, trees}
 \tikzstyle{Arrow} = [
%Create custom arrow style:
thick,
decoration={
markings,
mark=at position 1 with {
\arrow[thick]{latex}
}
},
shorten >= 3pt, preaction = {decorate}
]

 \pagestyle{fancy}
 \fancyhf{}

 \pgfplotsset{compat = 1.18}

 \hypersetup{
     colorlinks,
     citecolor=black,
     filecolor=black,
     linkcolor=black,
     urlcolor=black
 }
 \newtheorem*{theorem*}{\underline{Teorema}}
 \newtheorem*{lemma*}{\underline{Lema}}
 \newtheorem*{prop*}{\underline{Proposição}}
 \newtheorem*{crl*}{\underline{Corolário}}
 \theoremstyle{definition}
 \newtheorem{example}{\underline{Exemplo}}
 \newtheorem*{def*}{\underline{Definição}}
 \newtheorem*{proof*}{\underline{Prova}}
 \newtheorem{exr}{\underline{Exercício}}
 \renewcommand\qedsymbol{$\blacksquare$}

 \rfoot{Página \thepage \hspace{1pt} de \pageref{LastPage}}

 \geometry{a4paper, left=3cm, top=3cm, right=3cm, bottom=3cm}

\begin{document}
\begin{figure}[ht]
	\minipage{0.76\textwidth}
	\includegraphics[width=4cm]{../icmc.png}
	\hspace{7cm}
	\includegraphics[height=4.9cm,width=4cm]{../brasao_usp_cor.jpg}
	\endminipage
\end{figure}

\begin{center}
	\vspace{1cm}
	\LARGE
	UNIVERSIDADE DE SÃO PAULO

	\vspace{1.3cm}
	\LARGE
	INSTITUTO DE CIÊNCIAS MATEMÁTICAS E COMPUTACIONAIS - ICMC

	\vspace{1.7cm}
	\Large
	\textbf{Indução Estatística, Ontologia e Metafísica}

	\vspace{1.3cm}
	\large
	\textbf{Renan Wenzel - 11169472}

	\vspace{1.3cm}
	\large
	\textbf{Professor(a): Julio Michael Stern}

	\textbf{E-mail: jstern@ime.usp.br}

	\vspace{1.3cm}
	\today
\end{center}

\newpage
\textbf{{\Huge Avisos}}

{\huge Essas notas não possuem relação com professor algum.

	Qualquer erro é responsabilidade solene do autor.

	Caso julgue necessário, contatar:

	renan.wenzel.rw@gmail.com.

	Além disso, alguns textos em itálicos são clicáveis - normalmente, afim de facilitar o encontro de um resultado, definição ou uma continuação.
}

\tableofcontents

\newpage
\subfile{classes/aula01.tex}
\newpage
\subfile{classes/aula02.tex}
\newpage
\subfile{classes/aula03.tex}
\newpage
\subfile{classes/aula04.tex}
\newpage
\subfile{classes/aula05.tex}
\newpage
\subfile{classes/aula06.tex}
\newpage
\subfile{classes/aula07.tex}
\newpage
\subfile{classes/aula08.tex}
\newpage

\newpage
\begin{thebibliography}{99}
	\bibitem{stern2020} Stern, Michael Julio. \textbf{A Sharper Image}: The Quest of Science and Recursive Production of Objective Realities. Principia, 24, 2, 255-297. doi:10.5007/1808-1711.2020v24n2p255, 2020.

	\bibitem{stern2013} Stern, Michael Julio. \textbf{Cognitive Constructivism and the Epistemic Significance of Sharp Statistical Hypotheses in Natural Sciences}. São Paulo: IME-USP. Disponível em: \url{https://www.ime.usp.br/~jmstern/wp-content/uploads/2020/10/evli.pdf}.

	\bibitem{stern2017} J.M. Stern (2017), \textbf{Continuous Version of Haack's Puzzles}: Equilibria, eigen-states and ontologies. Logic Journal of the IGPL, 25, 4, 1, 604-631. Disponível em: \url{http://www.ime.usp.br/~jstern/miscellanea/jmsslide/Princ15A30.pdf}.

\end{thebibliography}

\end{document}

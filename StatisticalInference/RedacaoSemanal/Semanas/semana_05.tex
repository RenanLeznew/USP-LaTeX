\documentclass[../redacoes.tex]{subfiles}
\begin{document}
\section{Redação da Semana 05}
\subsection{Resumo}
Como último tema do curso, foram introduzidos a teoria Teste de Significância Completamente Bayesiano Generalizado (GFBST) como ferramental epistemológico de apoio ao Construtivismo Cognitivo Objetivo e que são naturalmente compatíveis com o \textit{valor epistêmico da hipótese H dado o vetor de observações X} (você pode recuperar o GFBST a partir do valor-e), o e-valor \(\mathrm{ev}(H | X)\), com três avaliações de uma hipótese H pelo procedimento do GFBST -- aceitação, rejeição ou angosticismo.

Aprendemos que o e-valor segue os princípios de inferência estatística mais comuns, especialmente os Bayesianos, incluindo o princípio de verossimilhança, a invariância por coordenada (no ótimo exemplo do professor, seria muito estranho se um resultado que vale em kg deixasse de valer em outra medida como lb), e a consistência assintótica, permite lidar com hipóteses \textit{sharp} e \textit{slack}, é robusta, com formulação simples, entre outras coisas; mas uma das coisas mais importantes é que ele tem a ENORME vantagem de obedecer, também, as propriedades básicas de composicionalidade lógica de um processo de decisão!
Para entender melhor o que isso quer dizer, é mais fácil explicar o que não quer dizer: com o caso do p-valor, considerando hipóteses A e B distintas, pode acontecer da hipótese conjunta \(A \wedge B\) (A E B) ser rejeitada, mas as hipóteses A e B individualmente não serem rejeitadas! Essas coisas só dificultam mais a interpretação de pesquisas, podendo muitas vezes levar a erros grossos de pesquisa.

Elaborando nesse ponto, a forma de compor os modelos e hipóteses no contexto do GFBST é utilizado o formalismo lógico do cálculo de \textit{possibilidades} baseado em probabilidades, que segue a lógica modal presente de forma natural na vida humana; então, ao invés de fazer uma ginástica mental para considerar a conjunção ser rejeitada e as hipóteses individuais não, basta pensar naturalmente no resultado e usar a lógica comum de discussões diárias; para citar a frase ótima do professor, ao seguir a forma normal disjuntiva correspondendo à lógica normal, o estatístico vai poder falar com o lógico sem sentir um complexo de inferioridade.

Vimos algumas outras formas de calcular como brinde ao diferenciar o cálculo de possibilidades do cálculo de probabilidades e da lógica clássica (e outros) a partir da forma que os operadores lógicos de conjunção e disjunção se traduzem -- comparada à probabilidade, onde o OU vira a soma, a possibilidade se caracteriza pelo OU virando o máximo entre os dois valores! Após estudar as diferentes formas dos cálculos das lógicas a partir da tradução dos operadores, aprendemos também que o e-valor e o GFBST têm um comportamento esperado com relação à inversão (hipóteses complementares), à monotonicidade (hipóteses maiores) e à consonância (junção de hipóteses)... Ainda por cima, a forma padronizada do e-valor pode ser lida como um p-valor! Uma maravilha estatística!
Para finalizar, foram apresentados os possíveis futuros desdobramentos para a pesquisa com os e-valores, incluindo a aparição das teorias de categorias mencionadas por aqui :)

\pagebreak
\subsection{A Lógica nas Teorias de Categorias}
\subsubsection{Uma breve contextualização}
No século 20, após discussões relacionadas à axiomatização da matemática por pessoas como Zermelo e Fraenkel, axioma da escolha e toda essa parte relacionada à teoria dos conjuntos, uma reverberação recorrente acabou ocorrendo -- quando alguém pensa ``qual é a fundação da matemática?'', os conjuntos imediatamente vêm à mente.

De fato, quando um estudante passa por uma formação em matemática, ele irá se deparar com vária áreas diferentes: a topologia, teoria de anéis, teoria de grupos, geometria, análise, entre outras, e apesar de toda a polarização que mencionei previamente, existe um elemento comum permeando todas elas, que seria exatamente a noção de conjuntos, ao ponto de muitas vezes vermos conjuntos como ``uma coisa onde é possível fazer matemática'' -- defina uma coleção específica de subconjuntos e você ganha uma topologia, determine alguma operação no conjunto e você para na álgebra, ou então comece a brincar com \(\varepsilon \)'s e \( \delta \)'s e você do nada se vê na análise.

No entanto, a caracterização comum dos conjuntos é algo bem recursivo, pois faz uso dos próprios conjuntos para defini-los! Mais especificamente, você faz uso de conjuntos e de um tipo de relação de pertença a eles, e é disso que surge, por exemplo, o problema do Paradoxo de Russel, então podemos colocar a formulação usual de conjuntos como:
\begin{itemize}
	\item Existem coisas chamadas `conjuntos';
	\item Existe uma relação binária `\(\in \)'; e
	\item Alguns axiomas são aceitos.
\end{itemize}
Porém, a teoria de categorias fornece uma forma diferente de discutir isso! Ao invés de trabalharmos com uma definição recursiva e com a relação de pertencimento, trabalhamos com Objetos e Morfismos/Setas, dando um escape à definição recursiva dos conjuntos, ao ponto deles tornarem-se apenas um exemplo na área, que permite formular a categorias Sets como
\begin{itemize}
	\item Existem alguns objetos chamados `conjuntos';
	\item Existem setas saindo de um conjunto para outro ou para si;
	\item Setas podem ser conectadas umas as outras utilizando a relação binária de composição `\(\circ \)'; e
	\item Alguns axiomas são aceitos.
\end{itemize}
Num curso de teoria de categorias, os axiomas mencionados costumam ser feitos claros, então vou poupar esse espaço para uma discussão mais expositiva, mas vale relatar que eles são diferentes comparados ao ZFC usual da matemática; por si só, isso pode não parecer muita coisa, mas o fato de utilizarmos apenas a linguagem de categorias significa exatamente que temos a liberdade de sair da categoria Sets, permitindo generalizar muitas das boas propriedades que ela tem para outros contextos -- surge a possibilidade de fazer matemática em outros \textit{lugares}.
Isso naturalmente leva à pergunta: a matemática precisa MESMO da categoria de conjuntos para fazer tantas coisas como as mencionadas? Ou será que existem alternativas? Seria possível fazer uso da lógica modal, por exemplo, sem depender de conjuntos e mesmo assim falar de termos similares à conjunção e disjunção?

O uso da palavra italicizada no parágrafo anterior, lugares, serve como um pequeno spoiler para a resposta, pois é sim possível aproveitar as boas características dos conjuntos em outros contextos fazendo uso de um tipo de conjuntos generalizados conhecidos como \textbf{Topos}, da palavra grega para ``lugares'' ou ``linha de argumentação'' (\(\theta o \pi o \varsigma\)), que foi dada a essa estrutura pelo matemático Grothendieck enquanto explorando parte da pergunta postulada acima; mais detalhadamente, ele estava explorando a geometria algébrica e, mais do que determinar se uma coisa era verdade ou não, ele se interessou por descobrir \textit{onde} uma coisa era verdade, que é uma ideia muito tentadora para o que foi conversado em sala de aula nessa matéria.

O tipo específico de Topos que pretendo introduzir muito por cima aqui é o topos elementar, não o de Grothendieck (é uma versão específica para geometria algébrica), e de cara já vale mencionar que eles não precisam da lei do terceiro excluído e muito menos do axioma da escolha ara funcionar, o que efetivamente dá um fim à ideia de verdade como uma questão binária e permite tratar do QUÃO VERDADEIRA uma coisa é, ao invés de um mero SE É VERDADEIRO OU NÃO.

\subsubsection{Topos e Classificadores de Subobjetos}
Jogando direto,
\begin{def*}
	Um \textbf{topos elementar} é uma categoria fechada cartesianamente, com limites finitos e um classificador de subobjeto. \(\square\)
\end{def*}

Foram ditas várias palavras ali em cima, talvez com pouco sentido para todas elas, mas veremos em partes a noção intuitiva de cada coisa conforme foram postas. Uma categoria fechada cartesianamente (closed cartesian category) é uma forma chique de dizer que ela tem um objeto terminal/final (um objeto que está no final de todas as setas, i.e., para qualquer objeto tem uma seta que chega nele), existe uma noção de produto de dois objetos (por exemplo, o produto cartesiano de conjuntos em Sets) e para quaisquer dois objetos, existe um exponencial deles dentro da categoria (pode pensar que é uma forma de concordância entre avaliar várias funções ao mesmo tempo, ou uma só de cada vez), ou seja, pedir que ela seja fechada cartesianamente ``é o mesmo'' que pedir a possibilidade de analisar vários objetos ao mesmo tempo, ou um por vez, e obter o mesmo resultado;
em seguida, temos os limites finitos, que consiste em uma noção de diagramas, normalmente utilizados em demonstrações de categorias, finitos (essencialmente, isso aqui garante que dá para provar coisas ou obter objetos novos com uma quantidade finita ou enumerável de passos). O último é mais interessante, e é a parte que mais interessa aqui, então vai ficar no próximo parágrafo.

Os classificadores de subobjetos consistem em um objeto \(\Omega \) e um mapa t que, a partir do objeto terminal T, determina se o objeto terminal está em um objeto X e também em um dos seus subobjetos A (Um objeto do mesmo tipo: um subconjunto de um conjunto, um subgrupo de um grupo, etc) \textit{ao mesmo tempo}. Apenas por isso, fica difícil de entender a definição, então eu vou me aproveitar do mostrar ao invés de expor, e considerar um exemplo que me ajudou muito a entender: na categoria Sets, um objeto terminal nada mais é que um conjunto consistindo de um único elemento, por exemplo \(\{x\}\); assim, procurar por um classificador de subobjetos é procurar por uma função (as setas em Sets são funções) tal que x esteja incluso em um subobjeto/parte de um conjunto se, e somente se, o valor de t diz que ele é -- se X é um conjunto e A é uma parte/subconjunto (subobjeto!!) de X, a função t que procuramos é aquela que responde a pergunta ``o elemento x do conjunto X é também um elemento de A?'', ou seja, \textit{t classifica os pontos de X que são também de A}, e isso caracteriza o (sub)objeto A completamente!
Uma outra forma de pensar nisso é que x faz parte de A se, e somente se, o valor da função t for `verdade'! É exatamente dizer se a afirmação de pertencer a um conjunto é verdadeira ou falsa, no sentido lógico da coisa!! Por conta disso, generalizar o classificador de subobjetos de Sets, conhecido como função característica, é exatamente generalizar os pontos onde uma afirmação é verdadeira ou falsa, tanto é que o classificador de subobjetos fornece o maior subobjeto de X onde a função característica assume \textit{sempre} o valor ``verdade'', generalizando de forma definitiva a noção de verdadeiro-ou-falso que pode ser dada em termos de um conjunto ser subconjunto de outros!

A primeira conclusão daqui é que, então, um topos é um tipo de categorias na qual é possível resolver as coisas em passos finitos, determinar onde elas são verdades e tratar de várias coisas ao mesmo tempo, que parece tudo muito bom. Porém, quero levar essa discussão um pouquinho só mais a fundo, porque isso já bate no limite do que consigo explicar sem usar matemática. A ideia aqui é que Sets é, por si só, um tipo especial de topos devido a algumas propriedades:
\begin{itemize}
	\item O objeto terminal funciona como um separador: dadas duas funções f e g entre conjuntos X e Y (\(f:X\rightarrow Y\) e \(g:X\rightarrow Y\)), se denotarmos o mapa \(x:\{x\}\rightarrow X\) como o ``elemento'' x de X, então se \(f\circ x = g\circ x\), teremos necessariamente \(f = g\): se morfismos coincidem no mapa que sai do objeto terminal, então eles são iguais;
	\item Sets não é a categoria boba, que é um termo que designei para a categoria com apenas um objeto e uma seta só -- ela é boba, porque tem nada que dá pra fazer nela!;
	\item Sets tem um objeto do tipo ``números naturais'': a gente sabe contar em Sets, com os números de sempre 1, 2, 3, 4, ..., e essa noção pode ser generalizada para qualquer categoria que tenha um objeto terminal!!; e
	\item Todo mapa que leva X a Y com \(f(X)=Y\), i.e., toda função sobrejetora que mapeia todo objeto de Y com um de X, tem um mapa inverso \(m:Y\rightarrow X\) que satisfazem \(e \circ m = \mathrm{Id}_{Y}\) (a composição deles é o mesmo que fazer nada), que é exatamente o axioma da escolha: para cada objeto/conjunto Y, dá pra pegar um elemento y dele tal que y seja também um elemento de \(e^{-1}\{y\}\in X\).
\end{itemize}
Assim, obtemos uma caracterização em Sets completamente em termos categóricos, e a consequência disso é que podemos definir termos comuns de conjuntos para outras categorias parecidas com Sets -- para Topos; a título de exemplo,
\begin{def*}
	Seja \(\mathcal{C}\) uma categoria e um objeto X dela. Um \textbf{elemento generalizado de X} é simplesmente um morfismo em \(\mathcal{C}\) que tem X como destino; mais especificamente, um elemento generalizado \textbf{tem formato S} quando ele é dado por \(x:S\rightarrow X\). \(\square\)
\end{def*}
Consequentemente, ganhamos uma noção generalizada de funções: considere um morfismo \(f:X\rightarrow Y\) em uma categoria qualquer \(\mathcal{C}\); um elemento generalizado x de X com formato S dá origem a um elemento generalizado de Y com formato X por meio do mapa de composição
\[
	S \stackrel{x}\rightarrow X \stackrel{f}\rightarrow Y,
\]
ou seja, a composição de setas \(f\circ x\), também conhecida como \(f(x)\), é um elemento generalizado de Y!! Agora sim, gostaria de chegar às finalizações da minha redação porque esse último raciocínio pode ser aplicado não apenas para a arbitrariedade como uma coisa meramente estética, mas sim porque é uma ferramenta poderosa para provar e definir coisas em categorias distintas, e por conta disso, é conhecido na área como \textbf{linguagem interna de uma categoria}, ou, em termos do curso realizado, é a \textbf{ontologia da categoria}.

Sendo assim, ficam algumas questões: tendo em mente a correspondência feita e a existência de \textit{morfismos lógicos}, mapas entre topos, será que isso seria um caminho para formalizar a tradução entre ontologias diferentes? Ou então, será que Topos seriam a ferramenta que o professor Stern busca para investigar as estruturas abstratas da lógica modal formal dos valores-e? Quem sabe, até mesmo um caminho para traduzir alguns conceitos matemáticos para as áreas das humanidades, como as leis? Eternamente preso à minha ignorância, me resta apenas ponderar e investigar questões dessas aos poucos, torcendo para que mais gente embarque nessa para desenvolver as ideias daqui.

\end{document}

\documentclass[../redacoes.tex]{subfiles}
\begin{document}
\section{Redação 03}
\subsection{Resumo da Semana}
Na aula dessa semana, passamos pela história de vida do Karl Pearson e como ele foi amargurado.
Vimos como ele começou sendo um estudioso do Spinoza e do Spinozismo, uma das filosofias que permeava o estudo estatístico com base em causalidade, na previsão de hipóteses a partir dos dados e que recebeu contribuições de nomes grandes, mencionados posteriormente nesse resumo.
Conforme estudado em aula, Pearson iniciou seus estudos concordando com Spinoza em grandes partes, mas essa perspectiva parou quando publicou sua história dramática de romance \textit{New Werther -- by Locki}, a qual possui tanto uma interpretação de um desejo homoafetivo por parte do Pearson com o seu amigo Raphael, segundo um dos maiores biógrafos do Pearson, quanto a interpretação do professor Stern sobre como o ato de renunciar a própria noiva por parte do personagem que representa Pearson seria, na verdade,
uma metáfora representando ele deixando de lado suas raízes com o Spinoza (representado pela Noiva) e partindo em sua jornada para desenvolver uma nova perspectiva de como a ciência deve ser feita, sem o uso da metafísica.

Após o New Werther, a próxima publicação dele sobre o assunto assume um tom muito mais acadêmico, na qual ele descreve mais a fundo suas ideias, especialmente sobre abandonar a busca por explicações causais, efetivamente iniciando sua pregação da ciência como ele achava que deveria ser.
Para reforçar e provar suas ideias, Pearson vai contra a crença em átomos, moléculas, genes, etc, estudando a física do Éter até que ela foi finalizada por Einstein e seus \textit{papers} no século XX, resultando em Pearson recorrendo à biologia para tentar comprovar, onde ele começa a pregar contra a ideia de genes, desenvolver a eugênica junto ao seu patrocinador e amigo Francis Galton, e estabelecer a estatística frequentista como uma focada apenas em descrever o mundo natural, fugindo de todas as explicações causais em favor das descrições e ajustamentos, conforme ele mesmo acreditava que a ciência deveria ser feita.

Eventualmente, as ideias de Pearson caem por água mais uma vez após acontecer a visualização dos genes na segunda metade do século XX, mas todos os conceitos da estatística frequentistas foram um sucesso até póstumos, permanecendo a forma principal de se realizar estatística até mesmo no século XXI, sem que as pessoas que a usam sequer saberem sobre os pressupostos ideológicos que são comprados junto aos ferramentais que fazem uso quando estabelecem metodologias frequentistas para testagem de hipóteses!

Sendo assim, uma das lições mais importantes que ficou das aulas da semana 03 foi justamente a importância de conhecer os pressupostos filosófico e epistemológicos que estão por trás dos métodos utilizados em situações diversas da vida.

\end{document}

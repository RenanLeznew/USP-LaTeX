\documentclass[../redacoes.tex]{subfiles}
\begin{document}
\section{Redação 03}
\subsection{Resumo da Semana}
Na aula dessa semana, passamos pela história de vida do Karl Pearson e como ele foi amargurado.
Vimos como ele começou sendo um estudioso do Spinoza e do Spinozismo, uma das filosofias que permeava o estudo estatístico com base em causalidade, na previsão de hipóteses a partir dos dados e que recebeu contribuições de nomes grandes, mencionados posteriormente nesse resumo.
Conforme estudado em aula, Pearson iniciou seus estudos concordando com Spinoza em grandes partes, mas essa perspectiva parou quando publicou sua história dramática de romance \textit{New Werther -- by Locki}, a qual possui tanto uma interpretação de um desejo homoafetivo por parte do Pearson com o seu amigo Raphael, segundo um dos maiores biógrafos do Pearson, quanto a interpretação do professor Stern sobre como o ato de renunciar a própria noiva por parte do personagem que representa Pearson seria, na verdade,
uma metáfora representando ele deixando de lado suas raízes com o Spinoza (representado pela Noiva) e partindo em sua jornada para desenvolver uma nova perspectiva de como a ciência deve ser feita, sem o uso da metafísica.

Após o New Werther, a próxima publicação dele sobre o assunto assume um tom muito mais acadêmico, na qual ele descreve mais a fundo suas ideias, especialmente sobre abandonar a busca por explicações causais, efetivamente iniciando sua pregação da ciência como ele achava que deveria ser.
Para reforçar e provar suas ideias, Pearson vai contra a crença em átomos, moléculas, genes, etc, estudando a física do Éter até que ela foi finalizada por Einstein e seus \textit{papers} no século XX, resultando em Pearson recorrendo à biologia para tentar comprovar, onde ele começa a pregar contra a ideia de genes, desenvolver a eugênica junto ao seu patrocinador e amigo Francis Galton, e estabelecer a estatística frequentista como uma focada apenas em descrever o mundo natural, fugindo de todas as explicações causais em favor das descrições e ajustamentos, conforme ele mesmo acreditava que a ciência deveria ser feita.

Eventualmente, as ideias de Pearson caem por água mais uma vez após acontecer a visualização dos genes na segunda metade do século XX, mas todos os conceitos da estatística frequentistas foram um sucesso até póstumos, permanecendo a forma principal de se realizar estatística até mesmo no século XXI, sem que as pessoas que a usam sequer saberem sobre os pressupostos ideológicos que são comprados junto aos ferramentais que fazem uso quando estabelecem metodologias frequentistas para testagem de hipóteses!

Sendo assim, uma das lições mais importantes que ficou das aulas da semana 03 foi justamente a importância de conhecer os pressupostos filosófico e epistemológicos que estão por trás dos métodos utilizados em situações diversas da vida.

\subsection{Um Momento de Reflexão}

Com base na perspectiva super importante abordada pelo professor, que mostrou como as origens filosóficas de uma área podem (e vão) moldar todo o resto do processo epistemológico dela, sinto que é importante trazer uma questão que foge um pouco da perspectiva matemática que estive tentando montar até agora, mas que entra na minha visão sobre a importância da estatística e da epistemologia na ciência moderna.

Durante minha formação em matemática, a cultura predominante foi de renunciar outras áreas, muitas vezes utilizando um argumento de que a matemática seria, em algum sentido, superior a elas, tanto como conceito quanto como ciência (o que é irônico, porque a matemática muitas vezes não tem um processo empírico utilizado para a dedução finita e argumentos circulares da ciência, então ela nem pode ser considerada uma), que resultou numa segmentação enorme das pessoas que cursam matemática e das pessoas de outras áreas no \textit{campus} onde estudo.
Na verdade, a questão vai até um pouco além: tal como existe a escadinha de ordem na qual as pessoas deveriam ser estudar para o Comte, começando pela matemática e passando pela astronomia, física, química, biologia, sociologia e terminando em ética, as pessoas com quem convivi acreditam que essa basicamente é a ordem de importância das ciências (note que a engenharia nem chega a ser considerada na escala de importância).
Para ser sincero, eu mesmo já caí nessa grande falácia e fui encantado pela ideia de estudar um assunto tão nobre, quase divino, e, para muitos, realmente divino.

No entanto, conforme fui amadurecendo e me expondo a outros conhecimentos e, tão importante quanto, a visões de mundo diferentes, percebi o quão infantil o pensamento acima pode ser, pois resulta em fragmentações internas e externas para a matemática: dentro dela, o Analista não consegue conversar com o Algebrista, nem com o Geômetra, os Logicistas, Topólogos, nem mesmo Estatísticos! Muito menos as pessoas da Teoria de Categorias.
Por fora, os matemáticos de lá olham, em grande maioria, com certo desprezo para filósofos e cientistas das humanidades, não reconhecendo a importância dessas áreas e ignorando como as próprias crenças filosóficas, que muitas vezes nem fazem ideia de onde vieram, influenciam nas formações e ideias que produzem.
Continuando com o relato anedótico, já ouvi de colegas e de um dos meus orientadores que é uma perda de tempo se dedicar ao estudo de várias áreas, pois a matemática precisa de pessoas especializadas, não generalistas, e que era melhor eu parar de fazer isso pois iria apenas me afundar.

Sendo assim, a mensagem que o professor passou sobre os perigos de aceitar ferramentas sem estudar as crenças filosóficas e origens epistemológicas teve muita ressonância comigo, pois aparentemente esse problema persiste em várias áreas do conhecimento -- da mesma forma que o matemático prova seus teoremas e usa seus axiomas sem entender as origens deles, os cientistas utilizam a estatística sem buscar conhecer os criadores, as crenças deles e o que eles queriam vender de ideais por trás das ferramentas desenvolvidas!
Consequentemente, as pessoas tentam provar causalidade usando a estatística frequentistas que abominava explicações causais, tentam encontrar teorias de tudo sem buscar as comprovações empíricas dos seus resultados, e tentam desenvolver áreas novas ou velhas sem compreender as motivações que passaram pelas cabeças de quem veio antes.

\pagebreak
\subsection{Um Pouco da História da Teoria de Categorias}
Como uma parte da história e filosofia da epistemologia já foi desenvolvida em sala pelo professor, junto às discussões de cunho filosófico e a uma parte da história da ciência, resolvi trazer um pouco da história e das ideias filosóficas da Teoria de Categorias.
A teoria mencionada foi desenvolvida por uma coleção de matemáticos ao longo do século XX, mas um que foi bem prominente é o William Lawvere, que, em um de seus livros, faz um lembrete sobre as fundações da matemática, que são intrinsecamente ligadas ao pensamento axiomático:
\begin{quotation}
	``\textit{In my own education I was fortunate to have two teachers who used the term `foundations' in a common-sense way (rather than in the speculative way of Bolzano-Frege-Peano-Russell tradition). [...] The orientation of these works seemed to be `concentrate the essence of practice and in turn use the result to guide practice'. I propose to apply the tool of categoric logic to further develop that inspiration.}

	\textit{Foundations is derived from applications by unification and concentration, in other words, by the axiomatic method. Applications are guided by foundations which have been learned through education}.
	''
\end{quotation}

Numa nota meio fora de texto, ao pesquisar sobre as origens epistemológicas da teoria de categorias, acabei entendendo muito melhor o motivo de ter pensado nessa conexão entre a disciplina e essa área; na verdade, ela é quase tautológica, pois o Lawvere estabeleceu a teoria de categorias como tentativa de formalizar e clarificar conceitos da metafísica e da epistemologia! Inclusive como forma de adicionar a ideia de aleatoriedade para teorias já existentes!! a
De fato, a própria frase acima tem um toque do foundherentism da Susana Haack, com a ideia de usar dedução para encontrar hipóteses novas, e a validação das mesmas para determinar novos axiomas.

Logo, ao estudar teoria de categorias, em particular estamos aceitando explicações metafísicas como sendo válidas, e em particular o uso da dialética para estudar os assuntos do mundo, pois uma das bases epistemológicas de Lawvere ao conceitualizar as categorias foi o \textit{Science of Logic} do George Hegel, tanto é que ele acreditava que os avanços resultantes da teoria de categorias seriam de valor alto para a filosofia dialética, sendo ela uma forma muito importante dos matemáticos darem atenção às questões filosóficas necessárias para fazer a matemática, base para diversas outras ciências, mais utilizável ainda.

Em seu livro de 1992, do qual eu parafraseei parte da sentença anterior, Lawvere finalizar um de seus parágrafos com uma frase que conecta muito com o que eu citei haver me incomodado, e talvez isso tenha, no fundo, motivado meu carinho por essa área: para atingir o estado de maior aplicação da matemática mencionado acima, os filósofos terão que aprender mais matemática, e os matemáticos terão que aprender mais filosofia.

No tópico de aleatoriedade que será apresentado na próxima redação, pretendo utilizar algumas das ideias de Lawvere sobre o uso de teoria de categorias em probabilidade, incluindo a metodologia que ele propôs para generalizar processos estocásticos com as chamadas \textit{Categorias de Markov}, e, um tanto surpreendente quando vi, os chamados \textit{mônades de probabilidade}, que permitem adicionar a noção de aleatoriedade a teorias que já existem!

Devo admitir que me senti meio perdido escrevendo esse texto, de várias maneiras, mas com o mesmo sentimento: a primeira foi não sabendo muito bem o que escrever, mas essa sumiu quando pensei em contar sobre a filosofia que percorre a teoria de categorias; a segunda, porém, foi muito mais insidiosa, tanto é que até agora não sei muito bem como lidar com ela -- ao começar meus estudos sobre as bases epistemológicas desse tema, percebi que tem tanta coisa pra falar que eu nem sei por onde começar, mas o rigor por trás é tão grande que eu nem consigo começar, ainda, a realmente falar desse assunto com propriedade.
Por isso, fica aqui o meu agradecimento ao professor Stern e ao monitor Matheus, pois a demanda por redações semanais relacionadas às nossas áreas acabou me fazendo perceber um tema de pesquisa que me interessa muito mais do que eu jamais esperava.

\end{document}

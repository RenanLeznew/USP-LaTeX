\documentclass[../redacoes.tex]{subfiles}
\begin{document}
\section{Redação da Semana 02}
\subsection{Breve Resumo}
Ao longo das aulas dessa semana, as explicações científicas foram exploradas em mais detalhes, e vimos que um bom ponto de partida para explicar um fenômeno de forma que seja compreensível (até mesmo para uma criança brincar) é quebrar o entendimento do fenômeno em três partes: as leis científicas, responsáveis por fornecer uma base teórica a partir de formulações com equações matemáticas; a análise detalhada, correspondente à quebra do sistema analisado em partes pequenas e o eventual estudo aprofundado dessas partes individualmente; e a síntese, correspondendo a determinar as leis de composição para juntar as partes individuais em um sistema complexo que esteja de acordo com a teoria.

Além disso, utilizamos essas ideias para decompor teorias científicas inteiras, utilizando como guias as explicações metafísicas e teleológicas, incluindo as quatro causas de Aristóteles para partir de base, e os diagramas de processos científicos, tal como o ilustrado abaixo:
\begin{center}
	\begin{tikzpicture}[
			observed/.style = {rectangle, thick, text centered, draw, text width = 6em},
			latent/.style = {ellipse, thick, draw, text centered, text width = 6em},
			error/.style ={circle, thick, draw, text centered},
			confounding/.style = {rectangle, thick, text centered, draw, text width = 6em, minimum width = 5.5in},
			outcome/.style = {rectangle, thick, draw, text centered, minimum height = 3.5in, text width = 6em}
		]

		\node(teo) at (-6, 5){Teórico};
		\node(mf) at (-6,  3){\(\substack{\text{Formalização} \\ \text{Matemática}}\)};
		\node(si) at (-6,  0){\(\substack{\text{Interpretação} \\ \text{Especulativa}}\)};
		\node(sm) at (-6, -3){\(\substack{\text{Modelagem} \\ \text{Estatística}}\)};
		\node(ps) at (-6, -5){Teoria Científica};

		\node(mp) at (-2,  5){Metafísico};
		\node(ce) at (-2,  3){\(\substack{\text{Explicação} \\ \text{Causal}}\)};
		\node(ev) at (-2,  0){\(\substack{\text{Leis e Regras} \\ \text{Precisas}}\)};
		\node(da) at (-2, -3){\(\substack{\text{Processamento de} \\ \text{Dados}}\)};
		\node(op) at (-2, -5){Operacional};

		\node(ex) at (2,  5){Experimental};
		\node(hf) at (2,  3){\(\substack{\text{Formulação de} \\ \text{Hipótese}}\)};
		\node(ed) at (2,  0){\(\substack{\text{Desenho da Testagem} \\ \text{Empírica}}\)};
		\node(ti) at (2, -3){\(\substack{\text{Execução do} \\ \text{Experimento}}\)};
		\node(ss) at (2, -5){Implementação};

		\draw[Arrow](sm)--(si);
		\draw[Arrow](si)--(mf);
		\draw[Arrow](mf)--(ce);
		\draw[Arrow](ce)--(hf);
		\draw[Arrow](hf)--(ed);
		\draw[Arrow](ed)--(ti);
		\draw[Arrow](ti)--(da);
		\draw[Arrow](da)--(sm);

	\end{tikzpicture}
\end{center}

Tendo em vista os temas que comecei a apresentar na redação passada, esta busca relacionar os itens que comecei a introduzir com maior profundidade e na tentativa de mesclar com as explicações metafísicas e finais.

\subsection{Metafísica e Teleologia no meu Contexto}
No ponto de vista experimental do diagrama de produção científica, eu não tenho muito lugar de fala ainda, então pretendo manter meu foco nas perspectivas teórica e metafísica.
Mais especificamente, os conceitos que introduzi na redação da aula anterior -- categorias, isomorfismos e a relação deles com ontologias -- servem principalmente para completar a parte da formalização matemática, com algumas utilidades na explicação causal e na interpretação especulativa.

Para solidificar melhor o parágrafo anterior, gostaria de propor que a teoria de categorias serve como fundamentação teórica para as ontologias em si, que por sua vez podem ser aplicadas na prática e usadas para avaliar a causalidade; em outras palavras, quero propor uma construção de diagrama de processo da disciplina de ontologias em si, por mais que eu não tenha muito mais elaborado, acho importante começar de algum lugar.

Primeiramente, a interpretação especulativa está sendo realizada neste exato texto, que seria composta justamente da especulação que ontologias podem ser tratadas parcialmente utilizando os ferramentais matemático que introduzi, inclusive ao ponto de imaginar que seria possível criar um isomorfismo categórico entre a categorias Cat de categorias e a categoria Ont que descrevi previamente.

A formalização matemática seria considerar as ontologias, com seus tokens, alinhamentos e traduções, em termos de objetos e setas das categorias, junto às outras estruturas (como produto, coproduto, produto tensorial, exponenciação, entre outros que não consegui apresentar pois ainda não sei onde exatamente vão se encaixar) pertinentes à área e que permitem operacionalizar matematicamente conceitos mais abstratos;

Uma vez realizada a formalização matemática, o aspecto causal da metafísica seria elaborado por meio das relações possíveis a partir dos ferramentais categóricos, desenvolvidas com o passar do século passado e do atual, que permitem relacionar áreas inicialmente sem conceitos matemáticos preestabelecidos com o que já sabemos, um exemplo disso sendo o uso do produto tensorial para simultaneamente tratar duas ontologias distintas e obter uma nova mesclando elas de forma lógica\footnote{Para referência mais profunda, tensores em categorias podem ser pensados como categorias monoidais ou tensoriais, que consistem em considerar um bifuntor \(\otimes :\mathcal{C}\times \mathcal{C}\rightarrow \mathcal{C}\), ou seja, você trabalha com múltiplos objetos e setas ao mesmo tempo a fim de obter um mesclado ao final.}.

A parte tecnológica me é mais elusiva, mas acredito que, utilizando as relações causais matematicamente formuladas, seria possível estabelecer conexão entre áreas previamente desconexas, as quais poderiam ser testadas tanto a nível teórico (avaliar os resultados de cada uma) quanto a nível prático (a partir dos ferramentais práticas da primeira área e dos da segunda, seria possível obter uma mescla que explique partes novas de ambas as áreas de conhecimento), que levariam a métodos novos para obter resultados tecnológicos novos.

Utilizando as tecnologias que se originaram, seria possível observar quais pontos foram mais soltos dos resultados propostos; às vezes, uma certa tradução de um teorema ou propriedade universal deixa a desejar, ou então olhar para um exemplo específico torne mais clara as explicações propostas com apenas a formulação matemática.

Por fim, já relacionando com as aulas da semana 03, uma análise estatística a nível Bayesiano permitiria usar os dados coletados para testar a hipótese ``junção coesa de áreas distintas usando teorias de categorias e ontologias'', momento no qual seriam aplicadas as probabilidades inversas para testar justamente a validade da hipótese, podendo entrar, também, a questão da medida de evidência que iremos construir ao longo do curso.

Em forma de diagrama, seria algo como
\begin{center}
	\begin{tikzpicture}[
			observed/.style = {rectangle, thick, text centered, draw, text width = 6em},
			latent/.style = {ellipse, thick, draw, text centered, text width = 6em},
			error/.style ={circle, thick, draw, text centered},
			confounding/.style = {rectangle, thick, text centered, draw, text width = 6em, minimum width = 5.5in},
			outcome/.style = {rectangle, thick, draw, text centered, minimum height = 3.5in, text width = 6em}
		]

		\node(teo) at (-6, 5){Teórico};
		\node(mf) at (-6,  3){\(\substack{\text{Teoria de} \\ \text{Categorias}}\)};
		\node(si) at (-6,  0){\(\substack{\text{Interpretação} \\ \text{Especulativa}}\)};
		\node(sm) at (-6, -3){\(\substack{\text{Problemas de} \\ \text{Alinhamentos}}\)};
		\node(ps) at (-6, -5){Ontologias Distintas};

		\node(mp) at (-2,  5){Metafísico};
		\node(ce) at (-2,  3){\(\substack{\text{Explicação} \\ \text{Causal}}\)};
		\node(ev) at (-2,  0){\(\substack{\text{Operacionalização com} \\ \text{Ontologias}}\)};
		\node(da) at (-2, -3){\(\substack{\text{Probabilidade de} \\ \text{Validade da Junção}}\)};
		\node(op) at (-2, -5){Operacional};

		\node(ex) at (2,  5){Experimental};
		\node(hf) at (2,  3){\(\substack{\text{Formulação de} \\ \text{Hipótese}}\)};
		\node(ed) at (2,  0){\(\substack{\text{Coerência e Aplicabilidade} \\ \text{Empírica}}\)};
		\node(ti) at (2, -3){\(\substack{\text{Teste de} \\ \text{Ferramentas Mistas}}\)};
		\node(ss) at (2, -5){Implementação};

		\draw[Arrow](sm)--(si);
		\draw[Arrow](si)--(mf);
		\draw[Arrow](mf)--(ce);
		\draw[Arrow](ce)--(hf);
		\draw[Arrow](hf)--(ed);
		\draw[Arrow](ed)--(ti);
		\draw[Arrow](ti)--(da);
		\draw[Arrow](da)--(sm);

	\end{tikzpicture}
\end{center}

Durante as aulas, vimos que a \textbf{precisão} em uma boa lei científica muitas vezes está codificada nas relações de equivalências, como o sinal de igualdade, e a mescla de teoria de categorias com ontologia permite justamente estabelecer essas relações, que surgem naturalmente no contexto categórico.
Além delas, o fato da teoria de categorias já estar bem desenvolvida como teoria matemática garante a boa estabilidade de seus axiomas, lemas, teoremas (aspecto dedutivo), garantindo a \textbf{invariância}, e eles permitem obter a parte precisa e invariante com \textbf{estabilidade}, afinal são resultados que, conhecendo o axioma e tendo familiaridade com a manipulação deles, qualquer pessoa consegue reproduzir.
Finalmente, fechando a caracterização inicial do ciclo de produção proposto, os produtos, coprodutos e tensores mencionados permitem exatamente a \textbf{composicionalidade} das partes individuais, com regras específicas dadas pelos diagramas e equivalências, enquanto que a \textbf{separabilidade} pode ser obtida por meio das projeções destes objetos conjuntos em suas partes individuais, ou, a nível mais profundo, analisar a categoria como um todo em suas partes individuais (objetos e setas) antes de agregá-los com as leis de composição de suas relações, dadas na definição de categorias.

\subsection{Direções Futuras}
Possivelmente, em algum momento, vou tentar trazer os análogos aos objetos categóricos abstratos que mencionei dessa vez, mas aplicados aos contextos que estou tratando, de forma mais elaborada do que apenas um breve parênteses, como foi o caso da aplicação dos tensores para mesclar objetos distintos.

Para finalizar as ideias dessa redação, um outro tópico abordado nos finalmentes da semana 02 foi a questão das autossoluções. Nesse sentido, uma das coisas que pretendo testar em relação ao referencial teórico que propus aqui é se a categoria de ontologias tem algum objeto dinâmico, pois a teoria de sistemas dinâmicos já possui relações profundas com as categorias, sendo até mesmo retratados alguns resultados que têm em si, de forma natural, os pontos fixos.
Assim, se realmente for possível desenvolver uma dinâmica ou algo parecido, a generalização das autossoluções para esse contexto permitiria a aplicabilidade resultados da área de sistemas dinâmicos, que lida bastante com a questão dos pontos fixos.
Além dela, eu gostaria de explorar, eventualmente, a possibilidade da categoria de ontologias formar um \textit{Topos de Grothendieck}, uma classe específica de categorias onde há um tratamento de lógica formal a um nível muito bem operacionalizado, que fornece as partes essenciais da teoria de conjuntos, mas para contextos onde eles nem precisam aparecer.

\end{document}

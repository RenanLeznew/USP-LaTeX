\documentclass[../redacoes.tex]{subfiles}
\begin{document}
\section{Redação da Semana 01}
\subsection{Introdução do Tema e de Mim}
Sou Renan Wenzel, atualmente curso matemática pura no Instituto de Ciências Matemáticas e Computação (ICMC) da USP de São Carlos, com um toque na formação de estatístico aplicado ofertada virtualmente pla Universidade Federal de São Paulo (UNIFESP) ministradas pelo professor Altay de Souza.
Após completar as três partes do curso, senti que um caminho natural era fazer parte do curso do professor Stern.

Nas minhas redações, planejo relacionar os assuntos vistos no curso a cada semana com alguns aspectos das duas partes distintas da minha formação mencionadas acima (pelo menos, vou tentar); mais especificamente, após ser exposto aos assuntos da primeira semana, senti fortes conexões entre alguns conceitos da matemática, da estatística e do curso, tal como Teoria de Categorias com Ontologias e Alinhamentos Ontológicos, e de isomorfismo de matemática com isomorfismo de epistemologia.

\subsection{Contextualizando as Partes}
Começando pelo aspecto com o qual estou mais familiar, a área de categorias da matemática tem como objetivo oferecer um ferramental que não fique restrito à teoria dos conjuntos, sendo mais geral do que a mesma: ao invés de estudar os conjuntos, que têm uma definição propriamente dita, seguem as regras e problemas conhecidos (como o Paradoxo de Russel), uma categoria é, no mais puro sentido, apenas uma coleção de coisas e setas.
Na verdade, uma categoria é uma coleção de objetos, que podem ser qualquer coisa (inclusive categorias! Vide HCT, ou \textit{Higher Category Theory}) e setas que ligam esses objetos, com três restrições sensatas para operacionalizá-las -- todo objeto deve ter pelo menos uma seta apontando para ele mesmo, conhecida como identidade; é possível juntar duas setas que ligam objetos respectivos em uma seta grandona; e, por último, ligar setas é uma operação associativa.
Já pela definição (não muito formal) dada acima, é possível perceber o potencial de generalizar conceitos que surge com essa área; para alguns exemplos, temos a categoria Sets dos conjuntos, na qual objetos são conjuntos e setas são funções, a categoria Top com objetos sendo espaços topológicos, e setas sendo homeomorfismos, e até mesmo a categoria Cook, com os objetos sendo ingredientes culinários e as setas sendo processos culinários, como cortar, ferver, etc.

Indo mais além, uma possível categoria que podemos formar é (talvez denotando por Ont) a categoria cujos objetos são Ontologias, e os morfismos/setas são os alinhamentos ontológicos, mas é importante argumentar que isso faz sentido. Com efeito, uma ontologia é alinhada com ela mesma, garantindo a identidade; se uma ontologia é alinhada com outra, isso significa que seus termos podem ser relacionados de forma íntima com os termos da outra, e se, por sua vez, esta segunda está alinhada com uma terceira, então basta
relacionar os componentes ontológicos da primeira com os da segunda, e os da segunda com os da terceira, formando um alinhamento direto da primeira ontologia com a terceira; por último, alinhar primeiro os termos da ontologia um com os da dois, seguido da terceira, não é diferente de alinhar primeiro os termos da segunda com a terceira, depois delas com a primeira -- ambos os processos resultam no mesmo final. Consequentemente, a categoria de fato existe!

É do meu interesse, portanto, utilizar essa perspectiva para entender algumas propriedades das ontologias e dos alinhamentos ontológicos, além de utilizar outros exemplos que conheço da área para elaborar mais esse raciocínio. Para tanto, acho indispensável fornecer a definição um pouco mais rigorosa e que sustenta a metodologia altamente comum na área de Categorias conhecida como ``caça de diagramas''.

\begin{def*}
	Uma \textbf{categoria} \(\mathcal{C}\) (cat) consiste de
	\begin{itemize}
		\item[1)] Uma classe de \textbf{objetos}, denotada por \(\mathrm{Obj}(\mathcal{C})\);
		\item[2)] Para cada par de objetos A, B em \(\mathrm{Obj}(\mathcal{C})\), é associado um \textit{conjunto} de \textbf{morfismos/setas}, denotado por \(\mathcal{C}(A, B)\),
	\end{itemize}
	que se relacionam com as seguintes propriedades:
	\begin{itemize}
		\item \textbf{Composicionalidade:} para quaisquer três objetos A, B e C em \(\mathrm{Obj}(\mathcal{C})\), vale que
		      \begin{align*}
			      \mathcal{C} & (A, B)\times \mathcal{C}(B, C) \longrightarrow \mathcal{C}(A, C) \\
			                  & (f, g)\mapsto g\circ f
		      \end{align*}
		\item \textbf{Identidade:} para qualquer objeto A de \(\mathcal{C}\), existe um morfismo \(\mathrm{id}_{A}\in \mathcal{C}(A, A)\)
		\item \textbf{Associatividade:} a composição de setas é associativa:
		      \[
			      h\circ (g\circ f)=(h\circ g)\circ f,
		      \]
		      que significa que o seguinte diagrama é comutativo (tanto faz a ordem que as setas são percorridas):
		      \begin{center}
			      \begin{tikzpicture}[
					      observed/.style = {rectangle, thick, text centered, draw, text width = 6em},
					      latent/.style = {ellipse, thick, draw, text centered, text width = 6em},
					      error/.style ={circle, thick, draw, text centered},
					      confounding/.style = {rectangle, thick, text centered, draw, text width = 6em, minimum width = 5.5in},
					      outcome/.style = {rectangle, thick, draw, text centered, minimum height = 3.5in, text width = 6em},
				      ]
				      \node(1) at (-4,2){\(\mathcal{C}(A, B)\times \mathcal{C}(B, C)\times \mathcal{C}(C, D)\)};
				      \node(2) at (4,2){\(\mathcal{C}(A, C)\times \mathcal{C}(C, D)\)};
				      \node(3) at (-4,-2){\(\mathcal{C}(A, B)\times \mathcal{C}(B, D)\)};
				      \node(4) at (4,-2){\(\mathcal{C}(A, D) \)};

				      \draw[Arrow](1)--(2);
				      \draw[Arrow](1)--(3);
				      \draw[Arrow](2)--(4);
				      \draw[Arrow](3)--(4);


			      \end{tikzpicture},
		      \end{center}
		      \begin{center}
			      \begin{tikzpicture}[
					      observed/.style = {rectangle, thick, text centered, draw, text width = 6em},
					      latent/.style = {ellipse, thick, draw, text centered, text width = 6em},
					      error/.style ={circle, thick, draw, text centered},
					      confounding/.style = {rectangle, thick, text centered, draw, text width = 6em, minimum width = 5.5in},
					      outcome/.style = {rectangle, thick, draw, text centered, minimum height = 3.5in, text width = 6em},
					      <->/.tip =Latex, thick]
				      \node(1) at (-4,2){\((f, g, h)\)};
				      \node(2) at (4,2){\((f\circ g, h)\)};
				      \node(3) at (-4,-2){\((f, g\circ h)\)};
				      \node(4) at (4,-2){\((f\circ g\circ h) \)};

				      \draw[Arrow](1)--(2);
				      \draw[Arrow](1)--(3);
				      \draw[Arrow](2)--(4);
				      \draw[Arrow](3)--(4);


			      \end{tikzpicture}
		      \end{center}
		      traduzindo, de forma mais escrita, como \((h\circ g)\circ f = (g\circ f)\circ h\)
		\item \textbf{Propriedade Universal da Identidade:} dados quaisquer dois objetos A, B em \(\mathcal{C}\) e qualquer seta \(f\in \mathcal{C}(A, B)\), temos a igualdade
		      \[
			      \mathrm{id}_{B}\circ f = f = f \circ \mathrm{id}_{A}.
		      \]
	\end{itemize}
\end{def*}

Com a definição acima, um diagrama que aproximadamente descreveria a categoria de ontologias seria, exemplificando, algo como

\begin{center}
	\begin{tikzpicture}[
			observed/.style = {rectangle, thick, text centered, draw, text width = 6em},
			latent/.style = {ellipse, thick, draw, text centered, text width = 6em},
			error/.style ={circle, thick, draw, text centered},
			confounding/.style = {rectangle, thick, text centered, draw, text width = 6em, minimum width = 5.5in},
			outcome/.style = {rectangle, thick, draw, text centered, minimum height = 3.5in, text width = 6em},
		]
		\node(TL) at (-3,2){Física Newtoniana};
		\node(BL) at (-3,-2){Física Relativística};
		\node(TR) at (3,2){Física Quântica};
		\node(BR) at (3,-2){Língua Comum};

		\draw[Arrow](TL)--node[midway, above] {\(\substack{\text{alinhamento de}  \\ \text{massa}}\)}(TR);
		\draw[Arrow](BL)--node[midway, below] {\(\substack{\text{alinhamento de}  \\ \text{massa}}\)}(BR);
		\draw[Arrow](TL)--node[midway, left]  {\(\substack{\text{alinhamento de}  \\ \text{massa}}\)}(BL);
		\draw[Arrow](TR)--node[midway, right] {\(\substack{\text{alinhamento de}  \\ \text{massa}}\)}(BR);

	\end{tikzpicture}
\end{center}
que ilustra a ideia apresentada da palavra ``massa'' constituir \textit{tokens} diferenciados em cada uma das ontologias acima: a massa inercial da mecânica newtoniana, a massa relativística da relatividade, a massa na física quântica, e o peso/massa comumente utilizado no mundo real (muita gente usa peso no lugar de massa, apesar das distinções)!

Antes, há outro objeto com o qual planejo relacionar os conteúdos citados, que seriam os \textit{isomorfismos} epistemológicos ou da teoria de sistemas, compostos dos tipos ``\textit{isomorfismo por número}'', ``\textit{isomorfismo por método}'' e ``\textit{isomorfismo por matriz}''.
O primeiro deles é o tipo comumente encontrado em pesquisa básica, que consiste em associar uma forma numérica a alguma coisa que tenha um zero bem definido, por exemplo fazer a coleta dos níveis de hormônio de crescimento em uma amostra de pessoas, onde existe um 0 bem definido (o nível sem nenhum hormônio de crescimento, por mais que isso muito provavelmente indicaria a morte da pessoa).

Porém, nem todo fenômeno da natureza pode ser representado com uma estrutura que tenha zero absoluto, e é onde entra o conceito de isomorfismo por método (o segundo mencionado) -- nele, estabelece-se um zero relativo para aquela amostra, e, a partir desse ponto de partida, determinam-se outros valores do que se deseja medir, servindo como âncoras para o restante dos valores; um tipo bem comum de desenho de pesquisa onde aparece esse tipo de isomorfismo é nos estudos da psicofísica, tal qual a medição da percepção de um choque elétrico (no experimento, um choque inicial fraco é dado e chamado de ``0'' e pede-se para os participantes determinarem a intensidade dos choque subsequentes que sentiram com base nele), ou da avaliação de uma obra de arte (referindo-se a uma certa obra como nível de beleza basal, as outras são graduadas a partir dela).

O último é comumente utilizado em teoria de redes sociais, neurais e modelos de rede, que é o chamado \textit{isomorfismo por matriz}, no qual o fenômeno observado emerge como resultado da relação de uma matriz de diversas variáveis, sendo um exemplo bem importante o caso da abordagem matricial para modelos de doenças mentais -- um paciente não \textit{tem} depressão, mas \textit{está} com depressão, e isso é determinado olhando para uma matriz contendo diversas variáveis comportamentais e da vida dele, coletadas ao longo do tempo e testadas a nível de correlação, evolução temporal, etc.

\subsection{Compondo as Ideias}

Olhando para cada conceito separadamente, temos três diferentes ontologias, inclusive quase que uma meta-ontologia, mas com base no que estudei e vi durante as aulas, acredito que hajam algumas relações interessantes a serem exploradas e que criem uma conexão profunda entre os três conceitos.

Primeiramente que a própria palavra isomorfismo tem seu respectivo conceito na matemática como o de uma função um-para-um (todo objeto do conjunto de entrada é mapeado para um da saída e todo objeto de saída tem um respectivo na entrada) que preserve as estruturas algébricas do domínio, tal qual equações, somas, produtos, etc. Por si só, isso já teria relação com o fato do isomorfismo da teoria de sistemas ser justamente um objeto que ``utiliza a mesma forma estrutural para obter valores'',
que em si já forma um alinhamento ontológico desse \textit{token} da teoria de sistemas e o respectivo na álgebra abstrata, área da matemática onde eles mais aparecem.

No entanto, a própria teoria de categorias fornece o conceito de isomorfismo a um nível mais profundo (na verdade, a dois níveis mais profundos: isomorfismos dentro das categorias e isomorfismos entre categorias), que se dá a partir do momento que, entre duas categorias, podem existir mapeamentos de objetos e morfismos de uma nos da outra, os chamados \textbf{funtores}; mais ainda, um tipo especial de funtor são os \textbf{isomorfismos naturais}, que permitem traduzir perfeitamente os objetos e setas de uma categoria na outra, sendo um dos exemplos mais comuns desse tipo de funtor o fato de podermos tratar transformações lineares da álgebra linear como matrizes: isto caracteriza um isomorfismo natural entre a categoria de espaços vetoriais de dimensão n e a de matrizes finitas de tamanho n.

A ideia de isomorfismo natural e de traduzir objetos de uma categoria na de outra é muito similar ao que os alinhamentos ontológicos e os isomorfismos de sistemas se propõem a fazer, que seria traduzir as ideias de um glossário para outro mantendo as relações possíveis entre cada item dele, e a forma de operacionalizá-los.

\subsection{Eu, como Sujeito}

Essa seção é dedicada a um breve pensamento que esteve me ocorrendo -- graças ao curso atual, conheci os conceitos de ontologias e alinhamentos ontológicos, e desde então me peguei pensando na relação entre todas essas coisas, afinal a ideia em si me lembrou alguns conceitos de teoria de categorias pela qual sempre me interessei bastante.

Portanto, aproveitarei algumas das oportunidades das redações semanais para comentar sobre essa possível conexão e, se possível, torná-las mais operacionalizada. Afinal, como aprendemos nas aulas, é importante analisar uma hipótese a partir de várias outras partes do grande quebra-cabeças da ciência, mas sempre mantendo em mente a importância de observar cada parte dela separadamente, encontrar uma regra de composição, leis matemáticas que descrevem a teoria e as autossoluções para a estabilidade.

\end{document}

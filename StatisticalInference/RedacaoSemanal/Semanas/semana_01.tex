\documentclass[../redacoes.tex]{subfiles}
\begin{document}
\section{Redação da Semana 01}
\subsection{Introdução do Tema e de Mim}
Sou Renan Wenzel, atualmente curso matemática pura no Instituto de Ciências Matemáticas e Computação (ICMC) da USP de São Carlos, com um toque na formação de estatístico aplicado ofertada virtualmente pla Universidade Federal de São Paulo (UNIFESP) ministradas pelo professor Altay de Souza.
Após completar as três partes do curso, senti que um caminho natural era fazer parte do curso do professor Stern.

Nas minhas redações, planejo relacionar os assuntos vistos no curso a cada semana com alguns aspectos das duas partes distintas da minha formação mencionadas acima (pelo menos, vou tentar); mais especificamente, após ser exposto aos assuntos da primeira semana, senti fortes conexões entre alguns conceitos da matemática, da estatística e do curso, tal como Teoria de Categorias com Ontologias e Alinhamentos Ontológicos, e de isomorfismo de matemática com isomorfismo de epistemologia.

\subsection{As Ontologias e Alinhamentos}
Começando pelo aspecto com o qual estou mais familiar, a área de categorias da matemática tem como objetivo oferecer um ferramental que não fique restrito à teoria dos conjuntos, sendo mais geral do que a mesma: ao invés de estudar os conjuntos, que têm uma definição propriamente dita, seguem as regras e problemas conhecidos (como o Paradoxo de Russel), uma categoria é, no mais puro sentido, apenas uma coleção de coisas e setas.
Na verdade, uma categoria é uma coleção de objetos, que podem ser qualquer coisa (inclusive categorias! Vide HCT, ou \textit{Higher Category Theory}) e setas que ligam esses objetos, com três restrições sensatas para operacionalizá-las -- todo objeto deve ter pelo menos uma seta apontando para ele mesmo, conhecida como identidade; é possível juntar duas setas que ligam objetos respectivos em uma seta grandona; e, por último, ligar setas é uma operação associativa.
Já pela definição (não muito formal) dada acima, é possível perceber o potencial de generalizar conceitos que surge com essa área; para alguns exemplos, temos a categoria Sets dos conjuntos, na qual objetos são conjuntos e setas são funções, a categoria Top com objetos sendo espaços topológicos, e setas sendo homeomorfismos, e até mesmo a categoria Cook, com os objetos sendo ingredientes culinários e as setas sendo processos culinários, como cortar, ferver, etc.

Indo mais além, uma possível categoria que podemos formar é (talvez denotando por Ont) a categoria cujos objetos são Ontologias, e os morfismos/setas são os alinhamentos ontológicos, mas é importante argumentar que isso faz sentido. Com efeito, uma ontologia é alinhada com ela mesma, garantindo a identidade; se uma ontologia é alinhada com outra, isso significa que seus termos podem ser relacionados de forma íntima com os termos da outra, e se, por sua vez, esta segunda está alinhada com uma terceira, então basta
relacionar os componentes ontológicos da primeira com os da segunda, e os da segunda com os da terceira, formando um alinhamento direto da primeira ontologia com a terceira; por último, alinhar primeiro os termos da ontologia um com os da dois, seguido da terceira, não é diferente de alinhar primeiro os termos da segunda com a terceira, depois delas com a primeira -- ambos os processos resultam no mesmo final. Consequentemente, a categoria de fato existe!

É do meu interesse, portanto, utilizar essa perspectiva para entender algumas propriedades das ontologias e dos alinhamentos ontológicos, além de utilizar outros exemplos que conheço da área para elaborar mais esse raciocínio.

\end{document}

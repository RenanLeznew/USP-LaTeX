\documentclass[12pt]{article}
 \usepackage{bookmark}
 \usepackage{amsmath}
 \usepackage{amsthm}
 \usepackage{amssymb}
 \usepackage{tikz}
 \usepackage{pgfplots}
 \usepackage[utf8]{inputenc}
 \usepackage{amsfonts}
 \usepackage{nicematrix}
 \usepackage[x11names]{xcolor}
 \usepackage{geometry}
 \usepackage{graphicx}
 \usepackage{graphics}
 \usepackage[export]{adjustbox}
 \usepackage{fancyhdr}
 \usepackage[portuguese]{babel}
 \usepackage{hyperref}
 \usepackage{multirow}
 \usepackage{lastpage}
 \usepackage{mathtools}
 \usepackage[many]{tcolorbox}
 \usepackage{newtxsf}
 \usepackage{subfiles}
 \usepackage{flafter}
 \usepackage{float}
 \usepackage{accents}
 \usepackage[T1]{fontenc}

\usetikzlibrary{positioning, calc, shapes.geometric, shapes.multipart, shapes, arrows.meta, arrows, decorations.markings, external, trees}
 \tikzstyle{Arrow} = [
 thick,
 decoration={
 markings,
 mark=at position 1 with {
 \arrow[thick]{latex}
 }
 },
 shorten >= 3pt, preaction = {decorate}
 ]

 \pagestyle{fancy}
 \fancyhf{}

 \pgfplotsset{compat = 1.18}

 \hypersetup{
     colorlinks,
     citecolor=black,
     filecolor=black,
     linkcolor=black,
     urlcolor=black
 }
 \newtheorem*{theorem*}{\underline{Teorema}}
 \newtheorem*{lemma*}{\underline{Lema}}
 \newtheorem*{prop*}{\underline{Proposição}}
 \newtheorem*{crl*}{\underline{Corolário}}
 \theoremstyle{definition}
 \newtheorem{example}{\underline{Exemplo}}
 \newtheorem*{def*}{\underline{Definição}}
 \newtheorem*{proof*}{\underline{Prova}}
 \newtheorem{exr}{\underline{Exercício}}
 \renewcommand\qedsymbol{$\blacksquare$}

 \rfoot{Página \thepage \hspace{1pt} de \pageref{LastPage}}

 \geometry{a4paper, left=3cm, top=3cm, right=3cm, bottom=3cm}

\begin{document}

\begin{center}
	\vspace{1cm}
	\LARGE
	UNIVERSIDADE DE SÃO PAULO

	\vspace{1.3cm}
	\LARGE
	INSTITUTO DE CIÊNCIAS MATEMÁTICAS E COMPUTACIONAIS - ICMC

	\vspace{1.7cm}
	\Large
	\textbf{INFERÊNCIA ESTATÍSTICA E ONTOLOGIAS}

	\textbf{Redações Semanais}

	\vspace{1.3cm}
	\large
	\textbf{Renan Wenzel - 11169472}

	\vspace{1.3cm}
	\large
	\today
\end{center}

\newpage

\tableofcontents

\newpage
\subfile{Semanas/semana_01.tex}
\newpage
\subfile{Semanas/semana_02.tex}
\newpage
\subfile{Semanas/semana_03.tex}
\newpage
\subfile{Semanas/semana_04.tex}
\newpage
\subfile{Semanas/semana_05.tex}
\newpage

\begin{thebibliography}{99}
	\bibitem{lawvere92} \textit{Categories of space and quantity}. in: J. Echeverria et al (eds.), The Space of mathematics , de Gruyter, Berlin, New York (1992)

	\bibitem{lawvere62} Lawvere, William. \textit{The category of probabilistic mappings,} seminar handout with notes by Gian-Carlo Rota, 1962.

	\bibitem{nlab} nLab authors. \textit{William Lawvere}. Diponível em: \url{https://ncatlab.org/nlab/revision/William+Lawvere/111}. Acesso em: 06 de Fevereiro de 2026.

	\bibitem{nLab2} nLab authors. \textit{Category-theoretic approaches to probability theory}. Disponível em: \url{https://ncatlab.org/nlab/show/category-theoretic+approaches+to+probability+theory}. Acesso em: 10 de Fevereiro de 2026.

	\bibitem{fritz20} Fritz, Tobias. \textit{The Category of Probabilistic Mappings with Applications to Stochastic Processes, Statistics, and Pattern Recognition}. Disponível em: \url{lawverearchives.com/wp-content/uploads/2025/07/1962/1962.probmap.pdf}. Acesso em: 12 de Fevereiro de 2026.

	\bibitem{perrone19} Perrone, Paolo. \textbf{Starting Category Theory}. Universidade de Ofxord, UK. 2019.

\end{thebibliography}

\end{document}

\documentclass[../statistical_inference.tex]{subfiles}
\begin{document}
\section{Aula 01 - 12 de Janeiro, 2025}
\subsection{Motivações}
\begin{itemize}
	\item A Metáfora do quebra-cabeça de Haack.
\end{itemize}
\subsection{A Teoria de Hipóteses}
Estudaremos, ao longo do curso, a teoria de hipóteses e a epistemologia, que busca responder ``como sabemos que sabemos de algo?'', uma pergunta que certamente vale a pena ser investigada.
Em filosofia da ciência, essa pergunta ocorre recorrentemente para todas as teorias científicas, pertinente à forma de saber que uma teoria realmente faz sentido com seus métodos e afirmações.
Na atualidade, muita gente acredita que a Terra é plana, que vacinas fazem mal, etc; sendo assim, voltou a ser altamente relevante até mesmo a nível político entendermos o que valida uma crença como uma
teoria científica: quais critérios são válidos para saber se acredito, compro ou sigo uma teoria científica?

Na questão das abordagens, existem ferramentas diversas para estas questões: puramente qualitativos, pelos discursos; utilizando principalmente recursos da lógica formal e teoria de conjuntos; e abordando por meio da
área de estatística, que é a utilizada ao longo destas aulas. Mais especificamente, utilizando as ferramentas estatísticas, buscaremos atacar os problemas apresentados na introdução; afinal, ela é a ferramenta utilizada
na prática da ciência para decidir esse tipo de questão, desde a epidemiologia à classificação de galáxias ou estudos de substâncias! Não à toa, uma das partes indispensáveis da revisão por pares é, justamente, a validação
do modelo estatístico utilizado e a \textit{testagem de hipóteses} relacionada. Em geral, os estatísticos que estudam esses temas se dividem em duas seitas -- os estatísticos frequentistas ou clássicos, e os estatísticos
Bayesianos -- e as consequências para a lógica e filosofia também serão estudadas!

Nesta primeira aula, estaremos introduzindo o tipo de problema que será estudado pela óptica da filosofia, que pode ser entendida como o estudo de \textit{versões contínuas dos quebra-cabeças de Haack} (do inglês \textit{Continuous Versions of Haack's Puzzles}), vindo da metáfora da filósofa Haack, que utilizou a metáfora ``fazer ciência é similar a resolver quebra-cabeças'', que estudaremos
e discutiremos uma versão alternativa dessa metáfora, conhecida, talvez, como fundacionistas. Começamos fazendo uma adaptação -- um jogo de quebra-cabeças é um que utiliza variáveis discretas, pois cada quadradinho nele corresponde a uma variável; por outro lado, a partir do momento em que as variáveis relevantes assumem qualquer valor na reta real, começamos a entrar no universo das variáveis contínuas,
que vai ser a grande adaptação a ser feita na metáfora, eventualmente. Em seguida, utilizaremos a álgebra linear para estudar os problemas da estabilidade, precisão, separação e composição dos sistemas que aparecem. Ao final, buscaremos criar uma \textit{ontologia}, um glossário/dicionário que possui os principais conceitos de uma teoria científica, correspondendo, de alguma forma, aos objetos principais dela.
O processo de ancoragem de termos de uma epistemologia leva o nome de \textit{symbol grounding}, firmando os termos correspondentes à teoria. Caso existam duas teorias científicas diferentes, os termos de uma não têm relação com os da outra (massa na teoria newtoniana não tem relação com a massa na teoria da relatividade restrita, tornando difícil ou impossível a conversa entre as duas teorias) mas é possível
fazer um chamado \textit{alinhamento ontológico}, que busca justamente permitir a conversa entre duas teorias distintas, compondo uma parte essencial da ciência (esse inclusive é o princípio da conversão de diferentes economias, como de Real para Dólar).


\end{document}

\documentclass[../statistical_inference.tex]{subfiles}
\begin{document}
\section{Aula 13 - 02 de Fevereiro, 2025}
\subsection{Motivações}
\begin{itemize}
	\item Aleatorização.
\end{itemize}
\subsection{Aleatorização: quando, e será que devemos, fazer?}
Nas próximas aulas, trataremos o assunto da aleatorização, desde quando ela deve ser feita, até como ela deve ser feita, incluindo diferentes formas de usos nos sorteios judiciais.
No dia-a-dia, algumas formas de aleatorização são bem presentes: dadinhos, cartas, e até selos postais com números de loteria!

As perguntas que serão abordadas são, então, se é necessário mesmo aleatorizar e o que compele uma pessoa a lançar um processo de aleatorização: existe alguma situação onde é melhor utilizar uma abordagem aleatorizada do que determinística?
Como devemos randomizar e quasi são os requerimentos para bons processos de randomização? Acompanhando esses estudo, serão observados alguns processos utilizados pelo STF (Supremo Tribunal Federal) para a aleatorização, por exemplo, na escolha de qual juiz ficará responsável por qual caso, que nunca é um processo determinístico (leva o nome de \textit{distribuição do processo}).
Vamos discutir segurança desses processos, as filosofias por trás, segurança por desenhos e por obscuridade a aplicações a procedimentos legais e ensaios clínicos.

Os dois textos literários mais antigos que continuam a ser lidos frequentemente até a atualidade são a ilíada e a bíblia; tanto na primeira quanto na segunda, processos de aleatorização aparecem neles em formas interessantes! Na Ilíada, um dos casos que é bem conhecido seria a escolha de quem iria receber a tarefa de sair do barco para ver as terras que estavam, e o método foi dez palitos segurados pelo capitão da nave Argos, um deles quebrado e escondido, para que os argonautas escolhessem até que alguém pegasse o que estava quebrado, sendo ele o escolhido para se arriscar.
Será que esse método é bom? Na época dos gregos, sim, afinal ele evita que o capitão escolha alguém de forma voluntária, além de que, se a pessoa fizesse a missão e sobrevivesse, provavelmente surgiria uma mágoa com o capitão.
Claro, uma outra opção seria por votação popular onde os argonautas cada um votariam em uma pessoa para ir fazer a missão, mas aí a mágoa poderia cair sobre o grupo todo, que seria talvez até pior!

Esse exemplo ilustra várias razões para fazer escolhas com aleatorização: para os gregos, alguns dos motivos seriam, além de evitar as mágoas, que a aleatorização pode acabar tendo uma participação do divino caso seja feita com um ``ritual correto'', dando vazão para a manifestação sobrenatural; além disso, há a questão de honrar costume e tradições antigos; e, por fim, caso a escolha seja feita com as regras corretas, há a garantia de que o processo seja menos enviesado, honesto e independente de confundimentos.
De fato, muitas tradições e culturas têm processos parecidos a esses, tal como na cultura Yoruba, na qual um oráculo brasileiro, chamado Ifá, que usavam dispositivos sortidos (jogos de Buzios) para aconselhamento, incluindo explicações para os resultados e respostas que estão profundas nas narrativas dos Orixás (Mitologia Yoruba, presente até o cotidiano).

Em outras situações, a aleatorização e os processos dela são usados frequentemente em países de forte tração militar, como os EUA, com toda a cultura de escolher quem será parte de guerras civis utilizando uma loteria em cada condado (antigamente, o processo era feito com bolinhas para cada dia do ano, para cada ano de nascimento, e quem fazia aniversário nessas datas deveria se alistar), ou então no Brasil, para o tribunal do Júri e a aleatorização de quem cuidará de cada caso\footnote{Nos EUA, tem muito mais casos onde o júri é importante, e o sorteio ainda é feito de forma aleatória, dificilmente aceitando isenção para quem é selecionado. Uma das formas que o sorteio de Júri é feito é por meio de um programa de computador.}.
Contrastando com o caso estadunidense, a jurisdição e democracia em Atenas também era feita pelo tribunal de Júri, mas com no mínimo 501 jurados ao invés de 12, todos sorteados entre os cidadãos atenienses (jogando kleros/dados (\(\kappa \lambda \eta \rho \omega \varsigma \))), que iam ao teatro ateniense, viam o caso que estava sendo apresentado e cada Dikastes (jurados, derivados da palavra \(\delta \iota \kappa \eta\) para a justiça) realizava seu voto de maneira independente e com o melhor do seu conhecimento legal e factual! O juramento dos Dikastes é
\begin{quote}
	``Eu vou dar meu voto em consonância com as leis e com os decretos passados pela assembleia e pelo Conselho, mas, se não houver lei, em consonância com meu próprio senso do que é mais justo, sem beneficiar ou maleficiar. Eu votarei apenas em assuntos levantados para culpabilização, e ouvirei imparcialmente os acusadores e acusados igualmente.'' Hansen (1999, p.182).
\end{quote}

Ambas as sociedades são e foram líderes em suas épocas respectivas, então há sabedoria nessas formas de aplicar a justiça, e portanto deveriam ser estudados não apenas de maneira crítica, mas também buscando compreender o que havia de bom em cada uma dessas partes.
Pela perspectiva matemática, algumas ideias podem justificar o pensamento dos gregos de juntar tanta gente para decidir: a Lei dos Grandes Números, e com bastante variáveis aleatórias que sejam \textit{independentes} e que sigam a mesma distribuição, então a média delas irá convergir para uma distribuição normal; porém, note que isso é recorrer a um erro de análise \textit{nunc pro tunc}, que é o uso dos conhecimentos atuais da distribuição normal para analisar os gregos de outrora, bem antes da distribuição normal.

\end{document}

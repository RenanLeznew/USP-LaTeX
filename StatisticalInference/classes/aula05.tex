\documentclass[../statistical_inference.tex]{subfiles}
\begin{document}
\section{Aula 05 - 19 de Janeiro, 2025}
\subsection{Motivações}
\begin{itemize}
	\item Nitidez de uma teoria;
	\item Leis de Composição.
\end{itemize}
\subsection{Nitidez e Leis de Composição}
Em nossa busca pela produção recursiva de realidades nítidas com a ciência, introduzindo os conceitos de \textit{aberrações} e \textit{artefatos} como as distorções e efeitos espúrios (como pixelação, replicação de pontos, halos, etc), emprestados do campo de resolução de imagens, assim como o termo nitidez.

A qualidade de uma imagem é caracterizada pela \textbf{magnificação} -- o grau de ampliação permitido pela lente -- e a \textbf{resolução} -- o grau de nitidez da imagem -- ou seja, o quão bem um observador consegue distinguir dois objetos próximos; estes dois fatores basicamente determinam o que é a qualidade (e o preço) de câmeras!
Um dos exemplos mais conhecidos disso é que, a olho nu, o planeta Júpiter parece apenas uma bolinha brilhosa; porém, a partir do momento que foram utilizados instrumentos ópticos como os telescópios e lunetas de Galileu, passou-se a acrescentar o fator de magnificação, permitindo discernir as luas que orbitam este planeta, alguns detalhes de sua superfície e outros corpos celestes que não eram facilmente visíveis a olho nú, resultando numa mudança radical na maneira como os corpos celestes eram organizados no Sistema Solar.


A história de Galileu será uma das âncoras desta aula, e aparecerá mais vezes: a aula de hoje seguirá sua história com as qualidades de imagens como alegoria para maior nitidez de teorias que descrevem a realidade, seguindo a metáfora da ciência como olho mental que produz imagens com visões ou teorias dos objetos em nosso ambiente e que têm nitidez cada vez maior.
Quando Galileu mostrou os satélites orbitando outros planetas como resultado de sua maior nitidez, sofreu perseguição e contra argumentos de pessoas que se recusavam a seguir a mudança que a observação trouxera, mas, tirando a contra argumentação, não perseguiremos (talvez em alguma momento possamos sofrer) alguém após estudar estes assuntos.
No fundo, a história da ciência é uma de como obtemos imagens, tanto visuais quanto mentais, cada vez mais nítidas da realidade em que vivemos, e que essa nitidez é caracterizada por quatro propriedades essenciais que já vimos -- \textit{precisão}, \textit{estabilidade}, \textit{separabilidade} e \textit{composicionalidade}.

Em 1610, Galileu apontou sua lunetinha para o céu e observou quatro objetos junto de Júpiter, e, no dia seguinte, viu que elas haviam andado um tanto em torno do planeta; após acompanhar por noites consecutivas o que estava acontecendo com elas e confirmar a observação, aí acabou tudo pra ele, pois caiu por terra a ideia que tinham de que havia uma esfera de cristal com os planetas e astros, mas o fato de ter quebrado isso ia contra o que a Igreja dizia ser certo, e a partir do momento que uma coisa que a Igreja dizia poderia estar errado, então todo o resto também poderia estar!
A Igreja não era infalível e estava sujeita ao questionamento! Como dá para imaginar, isso causou um grande problema tanto a nível teórico quanto prática, pois proferir heresias e negar a Igreja poderia resultar em tortura, fogueira ou prisão.
Não surpreendentemente, muita gente começou a duvidar das ideias de Galileu, seja por realmente haver uma dúvida, seja para salvar a própria pele. Com isso, Galileu escreveu para seu amigo Kepler uma carta pedindo a opinião dele sobre as pessoas com uma persistência de uma cobra, mesmo com todas as observações empíricas: ``[...] Verdadeiramente, tal qual uma serpente possui suas orelhas fechadas, assim fazem estes homens, tornando seu olhos fechados à luz da verdade.''. Como era de se esperar, a inquisição não foi muito amigável com isso, e Galileu passou o resto da sua vida em prisão domiciliar\footnote{Só não queimou porque a filha dele era uma freira com estima alta. No fim, esse processo pegou mal pra Igreja, e ela ficou reconhecida como injusta em especial por conta do julgamento de Galileu.}.

No julgamento de Galileu, veremos que haviam vários argumentos contra que merecem atenção, e olhar para eles levará a várias reflexões não só sobre o fato histórico, como também do que é uma teoria científica; um deles é por Cesare Cremonini, que era um professor de filosofia aristotélica e conhecido de Galileu, mas que se recusou a aceita suas ideias, indicando como argumento que

\begin{quote}
	``Eu não aprovo o que Galileu está dizendo, mesmo porque não tenho conhecimento das coisas que Galileu está dizendo, afinal são coisas que eu não vi: quando eu olhei pela luneta, a única coisa que aconteceu foi que fiquei com a mente embaralhada. Já chega! Não quero saber mais dessa história'' (Galileo, \textit{Opere}, vol.XI, no.564, p.165)
\end{quote}

Dizia ele, então, que ficava com a cabeça embaralhada ao olhar pela luneta, e uma forma de interpretar é que ele estava reclamando que as imagens por ela produzidas eram muito pouco nítidas. Com uma das lentes que foram preservadas da luneta no museu de Galileu, podemos ter uma noção da nitidez que ela teria [Bernieri, 2012] e concluir que a figura que Galileu desenhou teria uma resolução muito pior do que teria sido registrado por ele, tendo uma nitidez bem baixa, ou seja, o Cremonini afirmava não era totalmente sem fundamentos, é como se ele apresentasse um intervalo de confiança enorme.
Conforme Galileu foi fabricando lunetas melhores, pode até ser que esse contra argumento viesse a cair por terra (de fato, perdeu a força), mas já era tarde. Dentro da própria Igreja, alguns criticaram a crítica do Cremonini, tal como o Christopher Clavius, que dizia
\begin{quote}
	``A crítica do Cremonini poderia ser levada muito mais a sério se esse instrumento simplesmente apresentasse um argumento fixo, mas não é bem assim: os pontos andando em volta de Júpiter estão de fato em movimento, e têm uma certa regularidade. O desenho de Galileu pode até não estar correto, mas podemos ver o movimento, não é um efeito fixo criado por essa luneta.'' (Galileo, \textit{Opere}, vol.X, p.442).
\end{quote}
Esse argumento com base em ilusões também se ampara no uso que a óptica tinha na época: a maior parte era produzir certos truques, como lanternas mágicas que produziam figuras na parede, espelhos que queimavam papel, tudo com propósito de entretenimento, a chamada \textit{Magica Naturalis}, que fez muitos pensarem na luneta como ilusionismo, mas o Claviu contrapõe justamente isso.

A resposta do Kepler à carta de Galileu também merece uma análise:
\begin{quote}
	``Na parte Óptica da astronomia, eu já dei uma demonstração geométrica e expliquei o que acontece em lentes simples... Mais ainda, com respeito aos tubos circulares de duas lentes disponíveis ao público, incluindo os próprios instrumentos que você, Galileu, usou para perfurar os céus: estou tentando convencer os incrédulos a terem fé em seus instrumentos.'' (Kepler 1610, p.17-18, \textit{Dissertatio cum Nuncio Sidereo}).
\end{quote}
O que Kepler diz é que a argumentação que ele diz era a melhor forma possível de ganharem confiança no instrumento, afinal entregar uma caixa preta sem explicar abre portas às interpretações mais distintas, então explicar de forma profunda o mecanismo por trás era a melhor forma de convencer.
Galileu, por sua vez, tinha uma fábrica de lunetas, então ele não explicava essa parte para não fabricarem as próprias lunetas e para que comprassem dele, sendo uma atitude meio questionável com a racionalidade de seus argumentos.
Essa última parte, em especial, é muito pertinente às discussões de publicação dos tempos atuais!

A explicação de Kepler sobre o funcionamento da luneta é exemplar, mostrando claramente as quatro partes que compõem uma teoria científica, dissecando ela nos dois tubos, fornecendo uma teoria simples e precisa, com capacidade de cálculo, sobre o comportamento de raios ópticos que passem por uma lente, depois por duas, e mostrando empiricamente que seus cálculos eram precisos e estáveis.
A conclusão que ele chegou foi que o ângulo de incidência de um lado e do outro do plano que separa o ar do vidro é constante, e ela pode ser calculada como a razão de uma constante que caracteriza cada um dos meios pelos quais a luz vai passar:
\[
	\frac{\theta_2}{\theta_1} = \frac{n_1}{n_2},
\]
pelo menos para lentes esféricas. A forma que o Kepler usa é composta de três passos para explicar como e por que um telescópio funciona, que podem ser segmentadas em:
\begin{itemize}
	\item \textbf{Leis Científicas}: é necessário fornecer uma base teórica a partir de leis formuladas como equações matemáticas; no caso de Kepler, era a relação entre ângulo de incidência e índice de refração de um raio de luz passando por dois meios diferentes de propagação;
	\item \textbf{Análise}: é preciso identificar as partes essenciais do sistema e prover um bom entendimento de como e por que cada parte separada funciona; no exemplo, os elementos mais importantes eram as lentes finas, esféricas e o vidro;
	\item \textbf{Síntese}: deve-se providenciar \textit{leis de composição}, mostrando como juntar cada elemento antes observado individualmente, de forma que construam o sistema final que é mais complexo em nível; no caso de um telescópio refratário, isso quer dizer os meios e os métodos de especificar as diferentes lentes e suas localizações no suporte tubular da estrutura, a fim de produzir imagens nítidas.
\end{itemize}
Nas próximas aulas, nosso objetivo é decompor teorias científicas a partir dessas ideias. Uma última conclusão é que confiar na medição da luneta (ou de outros instrumentos) como algo que realmente existe depende de uma teoria científica que explique o que está acontecendo no processo de observação, expressa em termos de leis científicas (normalmente em forma de equações matemáticas).
\end{document}

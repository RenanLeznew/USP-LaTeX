\documentclass[../statistical_inference.tex]{subfiles}
\begin{document}
\section{Aula 04 - 15 de Janeiro, 2025}
\subsection{Motivações}
\begin{itemize}
	\item Alinhamento Ontológico.
\end{itemize}
\subsection{Alinhamento Ontológico}
O fato de existirem inteiros que sejam soluções verdadeiras para o quebra-cabeças, por exemplo, da tabela de afinidades de Morveau, é praticamente um milagre, pois implica que existe uma racionalidade muito grande; entende-se \textit{solução verdadeira} primeiramente por meio do exemplo, sendo ela uma solução tal que, para qualquer uma das potenciais reações, é possível consultar a tabelinha e saber se ela vai acontecer ou não, e ainda por cima dar certo!

Como vimos, apesar disso, as pessoas começaram a perceber que, quando a diferença entre divelente e quiescente era muito pequena, a reação seria lenta ou incompleta, e isso não está previsto dentro dessa concepção sobre as reações...
Passa-se um século, e a abordagem passa a ser do equilíbrio químico e da física estatística, com um sistema descrito pelas forças de reação tal qual o sistema de molinhas, que caracteriza a passagem do quebra-cabeças discreto para o contínuo e uma melhora notável na perspectiva dessa ciência, permitindo não apenas ver os casos onde ocorrem as reações, mas também quando haverá equilíbrio e as concentrações com as frações molares!

Todas essas explicações e sistemas dão certo pois conseguimos encontrar estas constantes invariantes para os sistemas químicos, físicos, musicais e outros exemplos, medidas com alta precisão e eventualmente recebendo seus nomes em cada área (constante elástica na física, entalpia de reação na química). O dicionáriozinho correspondendo a estes nomes, seus conceitos e o que são é justamente o que chamamos de \textbf{ontologia}, incluindo a forma de usá-los e definí-los, mesmo que mudem de área em área.
O processo de produção científica pode, então, ser visto como um diagrama descrevendo um processo contínuo e cíclico.

\begin{center}
	\begin{tikzpicture}[
			observed/.style = {rectangle, thick, text centered, draw, text width = 6em},
			latent/.style = {ellipse, thick, draw, text centered, text width = 6em},
			error/.style ={circle, thick, draw, text centered},
			confounding/.style = {rectangle, thick, text centered, draw, text width = 6em, minimum width = 5.5in},
			outcome/.style = {rectangle, thick, draw, text centered, minimum height = 3.5in, text width = 6em}
		]

		\node(teo) at (-6, 5){Teoretico};
		\node(mf) at (-6,  3){\(\substack{\text{Formalização} \\ \text{Matemática}}\)};
		\node(si) at (-6,  0){\(\substack{\text{Interpretação} \\ \text{Especulativa}}\)};
		\node(sm) at (-6, -3){\(\substack{\text{Modelagem} \\ \text{Estatística}}\)};
		\node(ps) at (-6, -5){\(\substack{\text{Espaço de} \\ \text{Parâmetros}}\)};

		\node(mp) at (-2,  5){Metafísico};
		\node(ce) at (-2,  3){\(\substack{\text{Explicação} \\ \text{Causal}}\)};
		\node(ev) at (-2,  0){\(\substack{\text{Verificação* de} \\ \text{Autossolução}}\)};
		\node(da) at (-2, -3){\(\substack{\text{Obtenção de} \\ \text{Dados}}\)};
		\node(op) at (-2, -5){Operacional};

		\node(ex) at (2,  5){Experimental};
		\node(hf) at (2,  3){\(\substack{\text{Formulação de} \\ \text{Hipótese}}\)};
		\node(ed) at (2,  0){\(\substack{\text{Desenho} \\ \text{Experimental}}\)};
		\node(ti) at (2, -3){\(\substack{\text{Implementação} \\ \text{Tecnológica}}\)};
		\node(ss) at (2, -5){\(\substack{\text{Espaço} \\ \text{Amostral}}\)};

		\draw[Arrow](sm)--(si);
		\draw[Arrow](si)--(mf);
		\draw[Arrow](mf)--(ce);
		\draw[Arrow](ce)--(hf);
		\draw[Arrow](hf)--(ed);
		\draw[Arrow](ed)--(ti);
		\draw[Arrow](ti)--(da);
		\draw[Arrow](da)--(sm);

	\end{tikzpicture}
\end{center}

Uma teoria que se expressa, pelo menos em partes, de forma matemática, auxilia a entender e explicar de forma causal em uma disciplina, entrando no quesito metafísico; a partir dela, formulam-se hipóteses experimentais, com as quais desenhamos experimentos para serem observados ou produzidos para testar hipóteses empíricas, que requerem implementações tecnológicas, utilizados para coletar dados que serão trabalhados e armazenados para análise com modelagem estatística.
O resultado dela, por sua vez, deve ser compatível com o modelo teórico, e levará a uma especulação interpretativa, finalizando o ciclo a partir de uma formalização matemática por conta dela, não só confirmando ou adaptando a teoria, mas também melhorando todo o resto, obtendo um modelo cada vez mais preciso.

Os invariantes que aparecem durante o processo descrito são chamados autossoluções, responsáveis por descrever o mundo e que recebem nomes, podendo ser tanto constantes matemáticas, quanto certos instrumentos ou medidas com regras de usos, e várias outras coisas com suas respectivas regras, compondo a ontologia -- o dicionário de valores invariantes e objetos auxiliares emergentes -- da ciência em questão.
Assim, como os invariantes, os métodos teóricos e os experimentais compõem o ciclo de produção científica, todos eles devem ser representados na ontologia pertinente especializada da área; por exemplo, o medido de pH de Beckman surgiu como consequência da teoria de circuito de Ohm, as medidas na ponte de Wheatstone, bulbos de vidros sensíveis ao pH e amplificadores eletrônicos, cada um compondo uma parte diferente do ciclo.
Ainda nesse exemplo, um químico certamente pode ser um bom químico sem saber as leis de circuitos elétricos, mas alguém no ciclo deve saber, e é essa pessoa que faz a passagem da ontologia da física para a da química, tal como o produtor do equipamento!

Com o gancho acima, devemos abordar a questão de como alinhar diferentes ontologias, e isso é possível, mesmo que alguns filósofos digam que não, basta ver o exemplo acima e pensar em todos os diferentes equipamentos que medem o pH: se o alinhamento ontológico não fosse possível, não teríamos como progredir no ciclo de produção da ciência.

As quantidades empíricas conservadas, ou seja, os invariantes, também podem passar pelo alinhamento ontológico para ter funções distintas em áreas diferentes, mas não é preciso ir para física ou química para encontrar os alinhamentos e os invariantes; existe uma outra área que cumpre bem esse exemplo, que é a economia, a qual foi feita uma ciência exata apenas após haver resposta à pergunta ``Quais são os invariantes que aparecem num sistema econômico?'', e a resposta que Gerard de Breau deu é que os preços cumprem essa função -- num mercado funcionado como uma ciência exata e de forma eficaz, os preços devem ser objetivos, se não com o preço básico, pelo menos acrescido de um custo básica de taxas, e estes preços são denominados em uma moeda, que são chamadas de moedas fortes ou fracas se os preços que elas denominam são bons invariantes ou não! A importância dessa questão é tanta que, caso uma moeda seja muito fraca, a civilização à qual ela corresponde pode até entrar em colapso.

As economias que estão longe desse modelo mais eficaz consiste de moedas fracas, propensas a julgamentos arbitrários com preços altamente variáveis, como a moeda Drachma grega (simbolizada pela corujinha de Atena), que era muito forte e reconhecida no mediterrâneo inteiro em 450 A.C., com o preço de uma comida num dia e no outro praticamente não variando, e tornou-se a melhor alternativa disponível naquele contexto justamente por isso; no entanto, o cruzeiro e o Drachma no sécluo XX eram péssimas moedas, a inflação nelas era enorme e os preços variavam muito. O euro é outro exemplo de moeda forte, tendo toda uma legislação e regras para um país poder utilizá-lo.

A comparação diacrônica (entre épocas) dos preços ao longo do tempo, analisando o preço das coisas com o passar do tempo, consiste um tipo de alinhamento ontológico; é possível, também, fazer a comparação sincrônica (na mesma época) entre moedas em circulação numa mesma época, tal como a conversão de Dólar para Real.
Estes pontos levantam algumas questões para realizar um alinhamento ontológico:

\begin{itemize}
	\item Estamos equalizando conceitos diferentes, mas com o mesmo nome? Por exemplo, a massa, acidez, dinheiro, etc, têm o mesmo significado invariantemente? Se não, existe algum pelo menos algum significado que seja compatível?
	\item Como poamos acessar e medir a compatibilidade?
	\item Como discernir alinhamentos sincrônicos e diacrônicos, horizontais e verticais, etc?
\end{itemize}

Até agora, encontramos alguns exemplos de alinhamentos que poder servir de inspiração para abordarmos estas questões:
\begin{example}[Alinhamentos Ontológicos]
	\begin{itemize}
		\item[i)] Newton e Einstein: \(F = m \frac{\mathrm{d}^{2}x}{\mathrm{d}t^{2}} \cong E = mc^{2}\);
		\item[ii)] Dólar e Real: 1 Dólar \(\cong \) 5.37 Reais (15/01/2026);
		\item[iii)] Medidas Químicas: medidor de pH \(\cong \) Indicador de Litmus (Alquimistas de Idade Média);
		\item[iv)] Afinidade de Misturas: Geffroy (1718) \(\cong \) de Donder (1923);
		\item[v)] Nomenclaturas Químicas: \(BaCO_{3}\cong\) Mefito de Barita;
		\item[vi)] Áreas de Estudos: Termodinâmica \(\cong \) Física Estatística.
	\end{itemize}
\end{example}

\end{document}

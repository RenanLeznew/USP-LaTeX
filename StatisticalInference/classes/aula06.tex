\documentclass[../statistical_inference.tex]{subfiles}
\begin{document}
\section{Aula 06 - 20 de Janeiro, 2025}
\subsection{Motivações}
\begin{itemize}
	\item Critério Objetivos para Avaliar um Processo Científico.
\end{itemize}
\subsection{Ontologia Científica}
Continuaremos nosso avanço pela metáfora da nitidez com as ciências; desta vez, olharemos para alguns brinquedos feitos para ensinar ciências, sendo o primeiro deles uma série que foi comum no Brasil em 1970 nas bancas, a série ``Os Cientistas'', que vinha com uma revista e alguns instrumentos experimentais para testar na prática alguns fenômenos, tal como a Lei da Refração de Déscartes.
Na época, foi um grande sucesso de vendas, e resultou em muita gente empolgada para aprender ciências, ao ponto de ser relançada mais duas vezes para além da inicial.
Outro brinquedo que era comum era conhecido como Polióptico com o objetivo de ensinar para as crianças conceitos de óticas e o efeito de diferentes lentes em tubos, permitindo a construção de lunetas, telescópios, etc, vendo o funcionamento delas e das relações das dimensões desses objetos.
Na Alemanha, tinha a caixa Kosmos-Baukasten com brinquedos em caixas para realizar experimentos e ensinar os princípios de diversos aspectos das ciências.

A estratégia destes brinquedos é exatamente a do Keiler para fazer ciências: existem leis formuladas na forma de equações, que descrevem comportamentos invariantes na disciplina de estudos, tal como a lei de refração da óptica no brinquedo; Além disso, explicar o comportamento de cada componente individual do sistema mais complexo, que é o caso do Polióptico com várias partes separadas; e explicar como montar tudo de volta a partir dos componentes mais simples!
Essa estrutura acaba intimamente relacionada com os princípios do Construtivismo Cognitivo Objetivo apresentado nesta disciplina.
A evolução da ciência dentro desse arcabouço epistemológico se dá pela tentativa de entender e encaixar a realidade dentro de um quebra-cabeças maior, afinal a ciência como feita pela humanidade sempre foi feita dessa forma, com refinamentos, esquecimentos e descobertas ao longo das épocas, tornando a história da ciência uma área fascinante\footnote{Embora a visão de um filósofo e de um historiador sobre ela possam diferir várias vezes, com o filósofo comparando conceitos diacronicamente e o historiador questionando como um conceito se transformaria em outro. Ambos têm seu valor!}.
Como vimos antes, o não entender do funcionamento de alguma coisa pode tornar as pessoas muito mais céticas sobre um possível avanço tecnológico, e isso deixa muito clara a importância de entender a mudança epistemológica ao longo das épocas e todo o contexto que permitiu os avanços contemporâneos da ciência -- tem muita lição que podemos e devemos tirar da História, tanto com a luneta de Galileu, quanto com outros estudiosos e civilizações.

Um objeto ou observado por um instrumento que é englobado da nossa teoria é dito ter \textbf{status ontológico} ou \textbf{realidade ontológica}.
Assim, uma \textbf{Ontologia Científica} é uma linguagem cuidadosamente controlada com palavras (\textit{tokens}) para objetos emergindo no ciclo de produção da disciplina científica, com sua gramática e regras de articulações válidas correspondendo às leis subjacentes e às regras de composicionalidade.
Por exemplo, a ontologia da óptica contém, em seu dicionário de \textit{tokens}, os termos índices refratários, distância focal, tipos de lentes, alinhamento, colimato, etc; a astronomia tem Luas de Júpiter, Anéis de Saturno, Estrelas, ...; e a matemática tem aritmética, funções trigonométricas, comutatividade, associatividade, distributiva, entre outros.

Os processos de inclusão e remoção de itens de uma ontologia científica são ambos semelhantes, afinal uma ontologia pode mudar para a exclusão de palavras, não apenas a adição, como foi o caso da ideia do Éter que perpassava o cosmo e que era considerada existente pel física Newtoniana; porém, ao trocar o Grupo de Galileu pelo Grupo de Lorentz e passou-se da concepção Newtoniana para a relativística, o conceito de Éter passou a ser considerado inválido: para a inclusão de muitas novas coisas, alguma tem que sair.
A ontologia muda para acomoda as mudanças que as teorias científicas trazem e passam por! Acompanhar o diagrama de produção da disciplina é uma boa forma de entender tanto a sua ontologia científica quanto as mudanças em cada etapa que está passando, e um dos tópicos que será coberto nas próximas aulas é justamente a interpretação e leitura corretas destes diagramas, deixando um pequeno spoiler do que descrever a instrumentação óptica:

\begin{center}
	\begin{tikzpicture}[
			observed/.style = {rectangle, thick, text centered, draw, text width = 6em},
			latent/.style = {ellipse, thick, draw, text centered, text width = 6em},
			error/.style ={circle, thick, draw, text centered},
			confounding/.style = {rectangle, thick, text centered, draw, text width = 6em, minimum width = 5.5in},
			outcome/.style = {rectangle, thick, draw, text centered, minimum height = 3.5in, text width = 6em}
		]

		\node(teo) at (-6, 5){Teórico};
		\node(mf) at (-6,  3){\(\substack{\text{Formalização} \\ \text{Matemática}}\)};
		\node(si) at (-6,  0){\(\substack{\text{Interpretação} \\ \text{Especulativa}}\)};
		\node(sm) at (-6, -3){\(\substack{\text{Análise de} \\ \text{Aberrações}}\)};
		\node(ps) at (-6, -5){Óptica};

		\node(mp) at (-2,  5){Metafísico};
		\node(ce) at (-2,  3){\(\substack{\text{Explicação} \\ \text{Causal}}\)};
		\node(ev) at (-2,  0){\(\substack{\text{Imagens} \\ \text{Nítidas*}}\)};
		\node(da) at (-2, -3){\(\substack{\text{Calibrações/} \\ \text{Observações}}\)};
		\node(op) at (-2, -5){Operacional};

		\node(ex) at (2,  5){Experimental};
		\node(hf) at (2,  3){\(\substack{\text{Especificação de} \\ \text{Instrumentos}}\)};
		\node(ed) at (2,  0){\(\substack{\text{Desenho de Projeto e} \\ \text{Cálculos}}\)};
		\node(ti) at (2, -3){\(\substack{\text{Fabricação/} \\ \text{Montagem}}\)};
		\node(ss) at (2, -5){Mão na Massa};

		\draw[Arrow](sm)--(si);
		\draw[Arrow](si)--(mf);
		\draw[Arrow](mf)--(ce);
		\draw[Arrow](ce)--(hf);
		\draw[Arrow](hf)--(ed);
		\draw[Arrow](ed)--(ti);
		\draw[Arrow](ti)--(da);
		\draw[Arrow](da)--(sm);

	\end{tikzpicture}
\end{center}
A figura mostra como percorrer o ciclo num processo de produzir instrumentos ópticos com qualidade ainda melhor, que, como indicado no centro, corresponde a imagens mais nítidas (resolução) e magnitudes maiores!
Não basta ter apenas a teoria, é necessário ter as outras partes do diagrama (do mesmo jeito que não é apenas com observações empíricas que se faz ciência, é necessário ter o arcabouço teórico).

\end{document}

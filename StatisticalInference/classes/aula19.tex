\documentclass[../statistical_inference.tex]{subfiles}
\begin{document}
\section{Aula 19 - 11 de Fevereiro, 2025}
\subsection{Motivações}
\begin{itemize}
	\item Futuros desdobramentos.
\end{itemize}
\subsection{Futuros desdobramentos}
Agora que vimos a definição matemática e uma série de propriedades lógicas do e-valor, podemos ver alguns tópicos de pesquisa que merecem investigação!
As primeiras delas estão na interface entre estatística e lógica, e incluem:
\begin{itemize}
	\item \textbf{\underline{Estatística sentencial}:} ao invés da abordagem de Kolmogorov, utilizando conjuntos finitos ou contáveis e que quebra um tanto no contínuo, foi desenvolvida a teoria da medida, resolvendo o problema com o custo de não podermos falar da probabilidade de qualquer conjunto, restringindo aos conjuntos mensuráveis; porém, fica essa restrição, enquanto os lógicos têm a tendência a trabalhar com sentenças e conjuntos de sentenças, predicando sobre elas numa lógica de primeira ordem num espaço numerável.
	      É preciso, então, ligar essas duas áreas, com algumas soluções incluindo restringir o sistema de predicações a um conjunto de sentenças unitário/do tamanho dos inteiros, ou então gerar um conjunto mais potente ao trabalhar com uma lógica de segunda ordem, ou dotando espaços com topologias especiais sobre eles, entre outras.
	\item \textbf{\underline{Investigação de Estruturas Abstratas}:} da mesma forma que encontramos o arcabouço de traduções e a teoria de confiabilidade, a qual é uma álgebra semelhante à que vimos, é possível fazer generalizações formalmente utilizando uma teoria desenvolvida por matemáticos feita para isso: a teoria de categorias\footnote{Olha só, o tema das minhas redações aparecendo!}, que muito provavelmente é a ferramenta certa para tratar das generalizações necessárias para isso, tornando possível simplificar ou generalizar demonstrações já conhecidas.
	\item \textbf{\underline{Conjuntos Fuzzy e Rough}:} todos os conjuntos sobre os quais predicamos as probabilidades são nítidos, sabendo exatamente quem está neles, mas isso pode não acontecer, com pontos estando dentro ou fora dele sem certeza e que é medido com uma ideia de mensuração dessa certeza, permitindo, para alguns casos, tratar da probabilidade em conjuntos fuzzy/nebulosos.
	      Seria interessante entender a teoria do e-valor e o GFBST para conjuntos Fuzzy!
	\item \textbf{\underline{Leis, Complexidades e Consequências para Sistemas Sociais}:} esse tópico de pesquisa se propões a olhar para leis nas ciências humanas, saindo das exatas, permitindo a validação, por exemplo, no mundo jurídico, respondendo a forma de medir/controlar a complexidade em sistemas legais.
	      Esse caso é especialmente interessante, afinal as leis surgiram de forma metafórica nas ciências com base na palavra no Direito! Daria pra testar as ``leis que pegam'' e as que ``não pegam'' no Brasil, por exemplo :)
	\item \textbf{\underline{Condições de Regularidade}:} ao calcular o e-valor, assumimos que o verdadeiro valor do parâmetro é interior ao espaço paramétrico, mas existem modelos estatísticos onde as hipóteses ou não são contínuas e diferenciáveis, ou estão no bordo do espaço paramétrico, por exemplo ao estudar uma mistura na qual um dos componentes tem valor 0.
	\item \textbf{\underline{Modelos fora dos paramétricos}:} é muito bem possível estender a teoria do e-valor para casos semiparamétricos, nãoparamétricos ou de parâmetros infinitos, por exemplo lidando com espaços de Hilbert, séries de Fourier, polinômios ortogonais, ondaletas, entre outras bases infinitas para espaços funcionais.
	\item \textbf{\underline{Implementação de Modelos Computacionais Eficientes}:} consiste no desenvolvimento de modelos computacionais eficientes, adaptando o método de Monte Carlo, incluindo versões mais sofisticadas, expansões assintóticas dos e-valores, procedimentos integrados de otimização para encontrar bons valores iniciais, e melhorar os procedimentos de convolução ou condensação!

\end{itemize}

E aqui finalizamos, mais uma vez, esse curso (maravilhoso)!
\end{document}

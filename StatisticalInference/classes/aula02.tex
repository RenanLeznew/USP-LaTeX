\documentclass[../statistical_inference.tex]{subfiles}
\begin{document}
\section{Aula 02 - 13 de Janeiro, 2025}
\subsection{Motivações}
\begin{itemize}
	\item Invariantes e equilíbrios em sistemas lineares;
	\item Tabelas de Afinidade Químicas como quebra-cabeças analíticos.
\end{itemize}
\subsection{Fundamentismo de Susan Haack}
Para abordarmos a questão de provar uma hipótese, precisamos diferenciar lemas, dilemas e trilemas:
\begin{itemize}
	\item \textbf{Um Lema} consiste numa única decisão a ser avaliada, como se fosse fugir de um touro com um chifre apenas de um lado, bastando ir para o outro;
	\item \textbf{Um Dilema} tem duas decisões complicadas que devem ser pensadas, tal qual fugir de um touro com um chifre para a esquerda e um para a direita, meio que não tem muito como fugir; e
	\item \textbf{Um Trilema} já é como se o touro tivesse três chifres, é uma situação de três opções e nenhuma muito boa.
\end{itemize}
Aqui, o trilema que aparecerá é o do filósofo Agrippa, o Cético, que dizia que existem três formas de justificar o conhecimento de algo, mas nenhuma delas é muito boa:
\begin{itemize}
	\item[1)] \textbf{Regressão Finita/Dedução}: a partir de afirmações tomadas como fundacionais, como ideais não questionáveis, axiomas, fatos empíricos, ...;
	\item[2)] \textbf{Regressão Infinita}: associar a uma fonte que regride infinitamente: eu sei pois meu pai sabe, pois o pai dele sabia, etc. Dificilmente seria útil;
	\item[3)] \textbf{Argumento Circular}: compõe a coerência geral da teoria.
\end{itemize}

Um bom exemplo da regressão finita é a geometria euclidiana, bastando usar os 5 axiomas de Euclides para deduzir a teoria inteira, mas o Agrippa dizia que isso consiste em parar de duvidar do teorema/hipótese para duvidar dos axiomas -- como sabe-se que são verdadeiros? Tanto é que mudar o quinto axioma resulta em geometrias distintas e válidas no mundo.
A regressão infinita também tinha seu problema para Agrippa, afinal a dúvida foi transferida até um ponto onde não é possível verificar se é verdade ou não. A argumentação circular, também, tem o problema de não levar a dúvida a lugar algum, ficando num ciclo. No fim, não existe maneira de justificar o conhecimento, o mundo é horrível e não tem o que fazer...

A Susan Haack, porém, vem para salvar o dia e afirma que existe um erro fundamental no pensamento do Agrippa, afirmando que tanto a parte 1 quanto a 3 são importantes e permitem validar coisas, pois a dedução é um raciocínio importante por si e o raciocínio circular permite avaliar pelo menos a coerência
Assim, o fundamentismo, tal qual um caça-palavras, é composto de uma mistura da 1 com a 3, pois os dois juntos levam a raciocínios e formas de justificar o conhecimento, resultando na metáfora do caça-palavras:

\begin{quotation}
	``How reasonable a crossword entry is depends on how well it is supported by its clue and any already-completed intersecting entries, how reasonable those other entries are, independent of the entry in question, and how much of the crossword has been completed.
	How warranted an empirical claim is depends, analogously, on how well it is supported by experience and background beliefs, how warranted those background beliefs are, independent of the claim in question, and how much of the relevant evidence the evidence includes.

		[T]he natural sciences, at least, have come up with deep, broad and explanatory theories which are well anchored in experience and interlock surprisingly with each other, and, as plausibly filling in long, much-intersected entries in a crossword puzzle greatly improves one's
	prospects of completing more of the puzzle, these successes have enabled further successes.''.

	Haack (1999, p.198-199).
\end{quotation}
Com os vários modelos e teorias, caso funcionem bem uns com os outros, ganhamos uma descrição de um aspecto do mundo, e quanto mais delas temos, mais partes se encaixam e a confiança no sistema todo aumenta. Não basta ter coerência interna entre teorias, é preciso avaliar se elas concordam com as observações experimentais!\footnote{É aqui que caem por terra as teorias como anti-vacinação e terraplanismo}

Vamos destrinchar a metáfora dela, analisar seus limites e propor uma nova versão dela. Atacaremos este problema pelo papel da argumentação circular nas teorias científicas, como sabemos se elas ou suas hipóteses fundamentais são verdadeiras no contexto da natureza, e como construir uma medida de evidência para apoiar cada uma das hipóteses/palavras do nosso jogo de palavras-cruzadas. Alguns dos limites que aparecem e que devemos conhecer para estudar melhor as questões são:
\begin{itemize}
	\item[i)] Qual é o papel positivo da argumentação circular em construir, provar e corroborar uma teoria científica, além de coerência ser uma condição trivialmente necessária?
	\item[ii)] Qual é a forma mais apropriada de apresentar uma hipótese científica H? Em um modelo estatístico e formalismo lógico?
	\item[iii)] Como construir e interpretar uma medida de evidência, \(\mathrm{ev}(H | X)\), do apoio à hipótese H a partir dos dados observados X?
\end{itemize}
Nessa crítica que buscamos, vale mencionar alguns dos pensamentos de estudiosos dessa questão:
\begin{quotation}
	``A common objection to coherentism is that it cannot account for truth... By stretching Susan Haack's crossword metaphor to its limits, we show that there are circumstances under which this objection is untenable''. Atkinson and Peijnenburg (2010, \textit{abstr}.)

	``It would seem foundherentism is in need of some additional, `objective', virtuous criteria to explicate precisely what Haack means by the evaluation of C-evidence for p''. Lightbody (2006, p.19).

	``As the complexity of the crossword increases, the ambiguity in general decreases: it becomes more and more difficult to come up with different solutions... The number of coherent ways of filling in a finite crossword, with a finite alphabet, irrespective of lexical constraints, is finite.
	In the end, if the crossword puzzle is sufficiently complicated, there might be only one solution.'' Atkinson and Peijnenburg (2010, p.353-354).
\end{quotation}

Com base nessas críticas, algumas coisas que queremos seria uma \textit{operacionalização da verdade} para teorias pequenas, que são limitadas em escopo, não que sejam Teorias de Tudo, mas com precisão \textit{muito} alta, tal como a física Newtoniana, a química de Lavoisier, ou a teoria de circuitos elétricos de Ohm e Kirchhoff.

Para caça-palavras padrões, como são nosso objeto metafórico, é legal entender suas características; dentre elas, temos
\begin{itemize}
	\item Reticulado padrão de tamanho 15 por 15;
	\item Regras comuns de composição: soletragem correta e significados separados para cada palavra, uso do dicionário padrão como base pré-determinada, e letras coincidem onde palavras se intersectam;
	\item Três palavras-respostas especiais de tamanho máximo, percorrendo o comprimento inteiro do caça-palavras;
	\item Existe uma relação de decaimento exponencial da probabilidade de uma palavra ser uma errada (não intencional), com a base sendo a frequência com que a letra mais frequente aparece (F) e o expoente (S) sendo o tamanho da palavra errada:
	      \[
		      \mathbb{P}(\mathrm{erro})\leq F^{S},\quad F < 1,
	      \]
	      tornando palavras acidentais muito improváveis.
\end{itemize}
Sobre o último item, \textit{palavras especiais} grandes ligações que perpassam o quebra-cabeças inteiro, mantendo-o estável. Para o exemplo do \textit{Most Amazing crossword}, que tinha duas respostas certas e foi realizado na eleição do Bill Clinton versus Bob Dole, as palavras corretas eram ``\textit{Prognostication}'', ``Clinton/BobDole - elected'' e ``mister-president''; os valores da improbabilidade eram
\[
	F \cong 0.13 (\text{letra e}),\; S = 15, \quad F^{S}\cong 5.1e-14;
\]
para comparação, a constante de incerteza relativa de Rydberg da física vale aproximadamente \(5.9e-12\). Como \textit{input} inicial, cada
palavra regenera a própria solução, ou seja, o conjunto de soluções genuínas (\textit{eigen}) gerava todas as possibilidades pertinentes --
elas formam uma \textit{base} para os potenciais resultados!

Essa ideia de autossolução motivará nossos estudos: analisaremos alguns sistemas contínuos simples que exibem autossoluções especiais
(equilíbrios ou propriedades invariantes à função), com propriedades especiais parecidas às encontradas em quebra-cabeças discretos:
\textbf{precisão}, \textbf{estabilidade}, \textbf{composição} e \textbf{separação}.
Depois, veremos como fechar o vão entre sistemas discretos e contínuos, com exemplos da química.

\begin{example}[Sistema de Polias]
	Para estudar modelos pequenos, passaremos de quebra-cabeças (discretos) para contínuos (sistema de polias). Para uma balança analítica,
	usada frequentemente pelos químicos (e que fornece felicidade para eles, pois mede peso de agentes com precisão absurda e permite a previsão
	precisa de teorias químicas!). Um sistema de balanças com duas massas e uma polia não tem um ponto de equilíbrio estável, pois uma mudança pequena no peso irá quebrá-lo;
	por outro lado, um sistema com três massas e duas polias possui uma solução estável, descrito pelo valor das massas e pelos ângulos das cordinhas das polias com a horizontal! Para encontrar o equlíbrio, \textit{decompõe-se} as forças em suas componentes e resolve-se o sistema
	\[
		S_{\mu }=\begin{bmatrix}
			0 \\
			0
		\end{bmatrix},\; \mu = \begin{bmatrix}
			\mu_{1} \\
			\mu_{2} \\
			\mu_{3}
		\end{bmatrix},\;\&\; S = \begin{bmatrix}
			\sin^{}{(\theta_{1})} & \sin^{}{(\theta_{2})} & -1 \\
			\cos^{}{(\theta_{1})} & \cos^{}{(\theta_{2})} & 0.
		\end{bmatrix}
	\]
	Aqui, passamos pelas etapas mencionadas pro caso dos quebra-cabeças: a decomposição/recomposição de \(\mu \)-forças na base horizontal-vertical [x, y]; uma estrutura hierárquica da forma de um sistema linear com coeficientes não lineares;
	uma geometria de ângulos \([\theta_{1}, \theta_{2}]\) que, se resolve \([\mu_{1}, \mu_{2}, \mu_{3}]\), então resolverá \(\alpha [\mu_{1}, \mu_{2}, \mu_{3}]\); a geometria é variável que se adapta para restaurar o equilíbrio do sistema caso haja perturbação das forças fixas;
	e uma medida de precisão com pelo menos 5 partes por milhão! Assim, a chance de acertar uma medida por acaso é minúscula, corroborando com a teoria!

	Se a minha teoria consegue acertar previsões com uma dada previsão, podemos dizer que ele está correto, mesmo que ela seja relativamente específica; essa é a chave para extrapolar a metáfora da Haack para sistemas específicos e simples, permitindo generalizar para o caso contínuo.
\end{example}

\begin{example}[Afinidade Química de Étienne François Geoffroy, 1718]
	Outro modelo que é resolvível por sistemas lineares é o de molas, onde há três forças se encontrando e procurando um ponto de equilíbrio, mas os ângulos entre elas estão fixos e elas que ajustarão-se em magnitude para encontrá-lo. Consideramos um triângulo ABC com hastes de aço e três molas \(z_{C},\; z_{B},\; z_{A}\) conectadas às hastes AB, AC e BC respectivamente (sem atrito); cada uma delas tem \(r_{i}\) comprimentos de repouso, constantes elásticas \(k_{i}\) e comprimento atual \(z_{i}\), resultando no sistema
	\[
		z = \begin{bmatrix}
			z_{1} \\
			z_{2} \\
			z_{3}
		\end{bmatrix}, \; \mu (z) = \begin{bmatrix}
			k_{1}(z_{1}-r_{1}) \\
			k_{2}(z_{2}-r_{2}) \\
			k_{3}(z_{3}-r_{3})
		\end{bmatrix},
	\]
	com ângulos fixos
	\[
		S = \begin{bmatrix}
			\sin^{}{(270^{\circ })} & \sin^{}{30^{\circ }} & \sin^{}{(150^{\circ })} \\
			\cos^{}{(270^{\circ })} & \cos^{}{30^{\circ }} & \cos^{}{(150^{\circ })} \\
		\end{bmatrix} = \begin{bmatrix}
			-1 & 1/2          & 1/2            \\
			0  & \sqrt[]{3}/2 & - \sqrt[]{3}/2
		\end{bmatrix},
	\]
	e tal que a soma das componentes se anula e que a soma \(z_{1}+z_{2}+z_{3}\) é a altura do triângulo.

	Uma das coisas que este sistema representa é sistemas químicos, e vamos ver como traduzir do jogo de quebra-cabeças para um jogo contínuo de equilíbrio! Pensamos nas reações químicas como um sistema de uma polia, com a reação indo completamente para a maior afinidade. Uma tabela que contém isso é a \textit{tabula affinitatum}, que explicou de forma algébrica ou mecânica pela primeira vez este fenômeno e permitiu fazer previsões, colocando em ordem desentendente elementos químicos que substituem uns aos outros em um sistema de reações; ao oferecer um agente mais reagente do que o menos, ela irá substituí-lo, resultando na reação química de substituição!
	Assim, para cada classe de substâncias, bastava ordená-las por reatividade para fazer previsões, e foi um grande progresso, pois, antes disso, a química era basicamente um sistema de fábulas tentando codificar as leis de reações com historinhas, a chamada \textit{alquimia}\footnote{Essa ideia foi tão revolucionária que Geoffroy foi conhecido como Euclides da química!}. A frase que associam a esse momento é
	\begin{quote}
		``\textit{Non fingendum aut excogitandum, sed videndum quid natura ferat; aut faciat}'', ou ``Não obtido imaginando ou especulando, mas observando o que a natureza constrói ou efetivamente faz'',
	\end{quote}
	mas será que é esse o caso mesmo? Basta apenas observar a natureza, afinal?
\end{example}
\end{document}

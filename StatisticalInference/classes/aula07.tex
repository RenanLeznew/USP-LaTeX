\documentclass[../statistical_inference.tex]{subfiles}
\begin{document}
\section{Aula 07 - 21 de Janeiro, 2025}
\subsection{Motivações}
\begin{itemize}
	\item Leitura de Diagramas de Produção;
	\item O Diagrama de Produção da Ciência.
\end{itemize}
\subsection{Diagramas de Produção}
Na aula passada, vimos que o processo científico pode ser entendido como desenvolvendo uma estratégia de se entender o que passa no mundo e realizada em três passos:
\begin{itemize}
	\item[Passo 1)] Formular equações que revelam a estrutura de leis naturais e que explicam comportamentos de algumas variáveis que retratem quantidades de interesse; a exemplo, variáveis associadas à trajetória de raios luminosas, especialmente o ângulo de incidência;
	\item[Passo 2)] Explicar, a partir das leis formuladas, os componentes mais simples do sistema sendo analisada, por exemplo explicar os espelhos, os prismas e as lentes;
	\item[Passo 3)] Mostrar como deverá ser feita a composição de sistemas complexos a partir do acoplamento das componentes mais simples, que seria a explicação de Kepler sobre montagem da luneta e por que ela faz sentido.
\end{itemize}

Vimos também alguns brinquedos que se destinavam a ensinar ciência para crianças usando as três etapas acima, como a série \textit{Os Cientistas} e o Polióptico no Brasil.
Ao mesmo tempo que são testadas as etapas acima, vai ocorrendo a formação e atualização do que chamamos de Ontologia Científica, coletando o que chamamos de \textit{tokens} que representam conceitos, instrumentos, teorias, etc, de importância da área.
O processo de melhoramento foi ilustrado pela melhoria diacrônica dos instrumentos ópticos, que levaram a melhores observações, às melhores especulações, e assim adiante; tanto é que, ao final da aula passada, jogamos um diagrama que representa esse processo cíclico:

\begin{center}
	\begin{tikzpicture}[
			observed/.style = {rectangle, thick, text centered, draw, text width = 6em},
			latent/.style = {ellipse, thick, draw, text centered, text width = 6em},
			error/.style ={circle, thick, draw, text centered},
			confounding/.style = {rectangle, thick, text centered, draw, text width = 6em, minimum width = 5.5in},
			outcome/.style = {rectangle, thick, draw, text centered, minimum height = 3.5in, text width = 6em}
		]

		\node(teo) at (-6, 5){Teórico};
		\node(mf) at (-6,  3){\(\substack{\text{Formalização} \\ \text{Matemática}}\)};
		\node(si) at (-6,  0){\(\substack{\text{Interpretação} \\ \text{Especulativa}}\)};
		\node(sm) at (-6, -3){\(\substack{\text{Análise de} \\ \text{Aberrações}}\)};
		\node(ps) at (-6, -5){Óptica};

		\node(mp) at (-2,  5){Metafísico};
		\node(ce) at (-2,  3){\(\substack{\text{Explicação} \\ \text{Causal}}\)};
		\node(ev) at (-2,  0){\(\substack{\text{Imagens} \\ \text{Nítidas*}}\)};
		\node(da) at (-2, -3){\(\substack{\text{Calibrações/} \\ \text{Observações}}\)};
		\node(op) at (-2, -5){Operacional};

		\node(ex) at (2,  5){Experimental};
		\node(hf) at (2,  3){\(\substack{\text{Especificação de} \\ \text{Instrumentos}}\)};
		\node(ed) at (2,  0){\(\substack{\text{Desenho de Projeto e} \\ \text{Cálculos}}\)};
		\node(ti) at (2, -3){\(\substack{\text{Fabricação/} \\ \text{Montagem}}\)};
		\node(ss) at (2, -5){Mão na Massa};

		\draw[Arrow](sm)--(si);
		\draw[Arrow](si)--(mf);
		\draw[Arrow](mf)--(ce);
		\draw[Arrow](ce)--(hf);
		\draw[Arrow](hf)--(ed);
		\draw[Arrow](ed)--(ti);
		\draw[Arrow](ti)--(da);
		\draw[Arrow](da)--(sm);

	\end{tikzpicture}
\end{center}

Foi evoluindo a formalização matemática das leis ópticas, às quais estão associadas explicações causais alternativas para cada uma das leis (também conhecidas como explicações metafísica\footnote{Tem significados como ``algo que não observo'', ou ``gnosiológico'', ou ``para além da físcia'', que surgiu com um sujeito que colecionou as obras de Aristóteles; porém, como ele foi escrevendo sem muita ordem, essa pessoa eventualmente tentou copiar elas de alguma forma que fizesse sentido, mas ao chegar no volume 4, ele nomeou este e os prévios como volumes da física. Assim, o volume 5, com muitos temas variados, ficou conhecido como o volume do que veio depois da física, que, em grego, é a palavra \textit{metafísica}. Nele, também há um artigo que se preocupa com o problema de, quando perguntamos ``por que?'', qual é o tipo de resposta válida para essas perguntas? Ele chegou, a partir disso, às quatro causas de Aristóteles como respostas válidas.}), permitindo a especificações de instrumentos, que levarão ao desenho deles, à eventual fabricação e às calibrações.
Com base nelas, somos capazes de analisar o instrumento óptico, suas qualidades e defeitos, dando origem a novas especulações e às formalizações matemáticas novas. Deste processo, emergem as imagens mais nítidas, representada no centro do diagrama.
É muito importante que emerja esses tipos de critérios, como as imagens nítidas, pois eles servem como \textit{âncora} para a teoria e para analisar suas mudanças ao longo do tempo.

Da mesma forma que as imagens mais nítidas ancoram o desenvolvimento de instrumentos óptiocs, o que ancora a ciência óptica são as leis e regras de composição precisas, obtendo o diagrama
\begin{center}
	\begin{tikzpicture}[
			observed/.style = {rectangle, thick, text centered, draw, text width = 6em},
			latent/.style = {ellipse, thick, draw, text centered, text width = 6em},
			error/.style ={circle, thick, draw, text centered},
			confounding/.style = {rectangle, thick, text centered, draw, text width = 6em, minimum width = 5.5in},
			outcome/.style = {rectangle, thick, draw, text centered, minimum height = 3.5in, text width = 6em}
		]

		\node(teo) at (-6, 5){Teórico};
		\node(mf) at (-6,  3){\(\substack{\text{Formalização} \\ \text{Matemática}}\)};
		\node(si) at (-6,  0){\(\substack{\text{Interpretação} \\ \text{Especulativa}}\)};
		\node(sm) at (-6, -3){\(\substack{\text{Modelagem} \\ \text{Estatística}}\)};
		\node(ps) at (-6, -5){Óptica};

		\node(mp) at (-2,  5){Metafísico};
		\node(ce) at (-2,  3){\(\substack{\text{Explicação} \\ \text{Causal}}\)};
		\node(ev) at (-2,  0){\(\substack{\text{Leis e Regras} \\ \text{Precisas}}\)};
		\node(da) at (-2, -3){\(\substack{\text{Processamento de} \\ \text{Dados}}\)};
		\node(op) at (-2, -5){Operacional};

		\node(ex) at (2,  5){Experimental};
		\node(hf) at (2,  3){\(\substack{\text{Formulação de} \\ \text{Hipótese}}\)};
		\node(ed) at (2,  0){\(\substack{\text{Desenho da Testagem} \\ \text{Empírica}}\)};
		\node(ti) at (2, -3){\(\substack{\text{Execução do} \\ \text{Experimento}}\)};
		\node(ss) at (2, -5){Implementação};

		\draw[Arrow](sm)--(si);
		\draw[Arrow](si)--(mf);
		\draw[Arrow](mf)--(ce);
		\draw[Arrow](ce)--(hf);
		\draw[Arrow](hf)--(ed);
		\draw[Arrow](ed)--(ti);
		\draw[Arrow](ti)--(da);
		\draw[Arrow](da)--(sm);

	\end{tikzpicture}
\end{center}

A precisão e regras de composição, então, ancoram o desenvolvimento desta disciplina!

Antes de finalizar a aula, uma outra analogia que serve bem aparece no cálculo numérico com a resolução de equações de pontos fixos, ou seja, pontos x no domínio de uma função f tais que \(f(x)=x\) (são fixados pela função f).
Normalmente, costuma-se utilizar um processo iterativo aplicando f a ela mesma, e, sob as condições certas\footnote{Por exemplo, o Teorema do Ponto Fixo de Bannach.}, eventualmente essa composição de funções irá convergir para o valor do ponto fixo:
\[
	z_{n} = f^{n}(x) = f\circ \dotsc \circ f(x) \rightarrow x^{*},
\]
onde \(x^{*}\) denota o ponto fixo. O processo ocorre de forma parecida com uma espiral, ``caindo'' para o centro, que é onde ficaria o ponto fixo.
Pontos fixos generalizados são, finalmente, chamados de \textbf{autossoluções}, pois um ponto fixo é um tipo específico de \textit{invariante}, afinal ele não varia sob a ação de f(x); sendo assim, diremos (com mais jargão) que uma autossolução é um invariante bem-comportado em um espaço funcional.

\end{document}

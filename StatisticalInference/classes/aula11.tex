\documentclass[../statistical_inference.tex]{subfiles}
\begin{document}
\section{Aula 11 - 28 de Janeiro, 2025}
\subsection{Motivações}
\begin{itemize}
	\item Positivismo de Auguste Comte;
	\item Positivismo-Lógico.
\end{itemize}
\subsection{Positivismo e seus Tipos}
O objetivo da aula de hoje é discernir entre os diversos tipos de Positivismo, começado pelo filósofos Auguste Comte, mas que possui várias vertentes, em especial o Positivismo-Lógico na área da filosofia, inclusive uma pelo Pearson, cuja ferramenta principal era a estatística.
Acontece, inclusive, um caso contraditório: na área da filosofia de ciência, o Positivismo-Lógico foi altamente influente, mas na prática da ciência em si, essa filosofia é bem escassa; por outro lado, a influência do Karl Pearson na prática da ciência foi extrema, tal qual o p-valor, medidas de validação, e continua sendo o feijão com arroz da prática de ciência, mesmo que poucos saibam quais são as ideias filosóficas que estão comprando junto com isso, ao ponto do diálogo entre as duas vertentes ser praticamente inexistente!

A opinião amargurada de Pearson sobre o positivismo do Comte é que ele começou pelo caminho correto, mas que perdeu a mão na metade do caminho, o que fez Comte ser meia-boca na opinião dele:
\begin{quote}
	``[...] the writings of Comte have at the very least acted as a stimulus -- if only of the irritant kind'' -- Pearson (\textit{The Grammar of Science}, 1911, p.570)
\end{quote}
Para Pearson, o Comte tem um relapso ao cair na metafísica, especialmente em duas instâncias: a chamada \textit{religião da humanidade}, que será brevemente tratada, e o esquema \textit{Scala Intellectus} para a classificação de disciplinas científicas, a qual será tratada mais cuidadosamente.
Já removendo do caminho, a religião da humanidade constitui uma parte tardia da vida de Comte, e Pearson, como revolucionário na França, era contra o conceito de Igrejas até mesmo por conta de toda a questão absolutista, monarquista e que o estado laico francês queria reparar.
Nela, as imagens sacras, figuras que recebem estátuas e ``anjos'' são, na verdade, grandes cientistas e artistas da história, sendo uma religião dedicada à adoração da humanidade em si! Com base na visão de Pearson, dá pra imaginar que ele realmente não gostou dessa ideia.

Um segundo ponto, muito mais importante, é a \textit{Scala Intellectus} mencionada; embora Comte rejeite a apresentação dogmática, ele não elimina a metafísica enquanto tentativa de entendimento dada ciência, e organiza o estudo das ciências numa escada de sete degraus: matemática, astronomia, física, química, biologia, sociologia e ética, seguindo numa ordem que supostamente ficaria melhor de entender o que acontece em cada etapa consecutiva.
O Pearson, sendo violentamente oposto à ciência como fonte de entendimento, nega a escala proposta por Comte, ligando a escada do intelecto a conceitos do Spinoza que ele tanto combateu -- para Spinoza, Deus era identificado à realidade das coisas, como elos numa escada intelectual de causalidade infinita, e a história do mundo seria meramente expressa de forma fenomenológica nos degraus sucessivos da lógica do pensamento puro -- falando que Comte combate o argumento metafísico religioso, mas usa ele na ciência, então teria falhado no meio do caminho (Comte era um positivista \textit{poser} para Pearson).
Outro perigo que surge dessa falta de conhecimento sobre todo esse contexto histórico é que muita gente tenta usar os conceitos frequentistas do Pearson para encontrar relações causais, mas isso não apenas falha, como também faria Pearson mandar as pessoas ajoelharem no milho!

\subsection{Da Filosofia à Inferência Estatística}
Vamos ver, então, tendo toda a contextualização histórica, como o Pearson leva as ideias dele à ciência, começando pela física do Éter.
Uma ideia muito útil para unificar as partes da escada mencionadas é o conceito de átomo, e as pessoas interessadas em eletromagnetismo contemporâneas ao Pearson estavam tentando explicar as coisas com campos magnéticos, por exemplo, e pareceu ao Pearson que estavam tentando usar conceitos metafísicos e causais.
Partindo disso, após ler um artigo de W. Hicks, ele concluiu que
\begin{quote}
	``Duas esferas imersas em um fluído incompressível, e que pulsa (muda de volume) de forma regular, exercem um no outro uma atração (pela mediação do fluído), determinada pela lei dos quadrados inversos, caso as pulsações sejam concordantes, e (similarmente) exercem repulsão se as fases de pulsações diferem por meio período'' K. Pearson, \textit{On the Motion of Spherical and Ellipsoidal Bodies in Fluid Media}, 1884.
\end{quote}
De fato, ele trabalhou bastante na física dos fluídos e na dinâmica do Éter\footnote{O Éter em si poderia ser visto como um conceito metafísico, mas como as pessoas não conseguiam ver o ar muito bem, pensavam que poderiam só tomar ele como existente.}, mas Einstein pouco depois mostrou em seus artigos que o conceito de Éter era ultrapassado e não faria sentido.

Muitos cientistas da época também propunham a ideia de quantização da matéria, utilizando os átomos, moléculas e genes, e os parâmetros estatísticos desses modelos são diretamente relacionados a causas latentes e existentes que fornecem conhecimentos para nós contemporaneamente após obtermos tantas medições dos efeitos, por exemplo o movimento Browniano de partículas; porém, como de costume, o Pearson falava que essas eram apenas explicações causais, e que estes conceitos deveriam cair em desuso, evitando os parâmetros correspondentes dos modelos matemáticos.
Novamente, o Pearson caiu por água DE NOVO pela mão do Einstein, pois Perrin, em torno de 1911, fez uma série de experimentos que comprovaram as hipóteses do Einstein sobre o movimento Browniano que comprovariam a existência de átomos.

Mais uma vez em uma cruzada contra elementos de ligação na escada do intelecto, Pearson voltou suas revoltas à biologia, entrando contra a ideia de genes, afinal os cientistas da hereditariedade estavam propondo, tal qual físicos e químicos, partículas que pudessem compartilhar informação entre gerações, e novamente ninguém havia observado um deles\footnote{Esse demorou mais pra ser observado, apenas em 1950 para frente que foram, tanto é que Pearson morreu achando que estava certo sobre não existirem genes e sua função meramente casual. Mendel até já havia mostrado algo parecido em seus experimentos, mas o monastério tomou conta de seu tempo total e deixou de lado os trabalhos que estava fazendo.}.
Nisso, ele encontrou o grande parceiro intelectual e eugenista da vida dele, o Francis Galton, patrocinador de grande parte da jornada intelectual do Pearson; por sua vez, ele começou tentando mostrar com um modelo matemático a existência dos traços parecidos entre populações, dando origem à ideia de \textit{regressão}, e é daqui que vem o nome da ferramente no núcleo da estatística frequentista, com a \textit{regressão das populações à média}!

Com os modelos prontos, ambos chegaram à conclusão que todas as características que serão transmitidas estavam relacionados a uma matriz, a \textit{matriz de correlação}, e chegaram a uma conclusão muito esperta de que \textbf{correlação não é causalidade!} O modelo de regressão era muito bom para explicar os fenômenos de hereditariedade que eles estavam estudando, então eles passam adiante com as ideias, afirmando que o estudo da genética seria não causal e baseado em correlações, sem explicações causais.
Uma característica desses modelos é justamente a capacidade de trabalhar com associações sem ideias de causalidade, e chegaram a levar para a área da política, pois se conseguia explicar a relação entre populações, eles deveriam testar quem deveria abrir mão de se reproduzir, inicialmente como um conselho, mas tornando-se uma imposição -- a eugenia! É uma péssima ideia definir um tipo racial para um país com base em reprodução, bastando olhar para o caso dos nazistas, nos quais haviam campos de extermínio e de reprodução para reproduzir a nova geração mais pura da raça, só ideia ruim.


Note como Pearson mesmo tenta tanto fugir da causalidade, começa falando que não quer ela, cai na eugenia e começa a dar uma explicação causal para motivar uma política pública, enquanto que, em outras áreas da ciência, ele opera com um jogo de mostrar que uma relação supostamente causal na verdade é apenas uma correlação, por exemplo (apenas exemplo), aceitando que exista apenas uma correlação entre QI baixo e formato do crânio na ciência biológica, mas usar isso de justificativa para permitir certas pessoas se reproduzirem ou não, que é justamente uma justificativa que seria causal. Ele mesmo se contradiz!
Um alerta: a linguagem da estatística frequentista foi moldada para dar base a essa filosofia sem metafísica -- nenhuma ferramenta matemática é neutra, todas vêm com pressupostos epistemológicas que são compradas junto delas.

\end{document}

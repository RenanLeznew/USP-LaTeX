\documentclass[../statistical_inference.tex]{subfiles}
\begin{document}
\section{Aula 09 - 26 de Janeiro, 2025}
\subsection{Motivações}
\begin{itemize}
	\item Spinozismo invertido, idealismo;
	\item Ciência Empírica;
	\item Gramática da Ciência
\end{itemize}
\subsection{O Renuncio da Noiva}
Em misticismo, muitas vezes, o conhecimento metafísico humano normalmente é chamado de ``A Noiva'', e veremos as perspectivas sobre causalidade e metafísica de alguns filósofos, até chegar à renúncia da noiva por Karl Pearson.
Passaremos de Maimônides até Spinoza e adiante, e ambos acreditavam que Deus age no mundo através de leis naturais, entrando na caixinha da figura dele como grande arquiteto do universo e impessoal (ao invés da imagem dele como pai), cujo intelecto se manifestaria justamente com as leis como fora mencionado; além disso, para entender este Deus e suas vontades, é necessário estudar as leis naturezas que regem esse universo, aprendendo também a se comportar de acordo.

Essa ideia deles foi mal recebida por alguns, como todas as pessoas que trabalham como intermediários ou despachantes de Deus, mas muitos abraçaram ela, espacialmente os filósofos e as pessoas da igreja protestante\footnote{Afinal, era uma forma direta de ir contra a venda de terrenos no céu que estava acontecendo na Igreja católica.}, obtendo a atenção do pastor Bayes, que decidiu levar as questões à matemática em busca de calcular e encontrar uma maneira matemática de testar causalidade, oferecendo um tratamento de inferência estatística sobre causas que não são diretamente observadas!
A partir dele, a estatística passou a ser desenvolvida dentro dessa ideia de variáveis que podemos observar (espaço das variáveis observáveis/amostral) e o das que não observo, mas estão ligadas ao modelo (espaço das variáveis latentes/parâmetros); em seguida, Laplace continuou a desenvolver essas ideias para usar a estatística como forma de obter conhecimento das causas a partir dos efeitos observados, e pastaram o bastão para o George Boole, dedicado à lógica formal e ao avanço dessa parte matemática.
Por sua vez, Boole passou as ideias ao Pearson, mas ele achava que a busca pelas causas era uma bobagem, dando enfoque apenas ao pensamento descritivo, efetivamente renunciando A Noiva! Pearson passou a procurar, na estatística, formas de descrever e retratar o mundo com modelos matemáticos, considerando todo o resto como vanidade.

Nas próximas aulas, vamos explorar a jornada do Pearson, começando com seus estudos do Baruch Spinoza e sua obra de linguagem pseudomatemática da ética pela perspectiva geométrica (``\textit{Ethica - Ordine Geometrico Demonstrata}'' (1677)), o momento em que ele se desiludiu do pensamento de Spinoza (Com sua obra New Werther) e publica o ``Grammar of Science'', onde abandona de vez e expressa esse abandono sobre a causalidade, pregando a ciência com a forma que ele propões e desenvolveu a física do Éter para mostrar a ciência dele\footnote{Curiosamente, a física do Éter foi assassinada logo em seguida por um sujeito infame chamado Albert Einstein.}, mas ele tinha motivo para se dedicar tanto à física do Éter, como veremos.

Quando ela morre, Pearson vai para a área de genética e eugênica, mostrando como supostamente controlar a reprodução para melhorar raças\footnote{Sempre tem alguém pensando nisso :p.}, mas essa área também morre (ainda bem) não muito tempo depois, no século XX.
Finalmente, ele desenvolve a estatística com a visão filosófica que tem, pelo menos sucedendo nisso, influenciando fortemente a forma como as pessoas faziam estatística, mas essas ideias são ruins para a forma que as pessoas desenvolveram a estatística, como veremos, tendo ramificado até a modernidade (e sendo parte do motivo de buscarmos quebrar essas ideias).

Aprofundando nos temas mencionados, o New Werther marcou a quebra de Pearson com Spinoza, apesar de ter publicado com um pseudônimo Locki, referente ao deus traiçoeiro da mitologia nórdica, que retrata um triângulo amoroso entre Arthur, Raphael e a Ethel, respectivamente representando Pearson, Spinoza e Sophia/Shekinah.
Na obra, é retratado um drama homoafetivo, pois Arthur (Pearson) tinha uma noiva (representando um amigo de Pearson por quem ele supostamente teria sentimentos) e Raphael (outro rapaz, Wertheimer, que ele conhece depois), perspectiva que surge por entrevistas com mais velhos e cartas que o Pearson havia escrito.
Outra interpretação (não são mutualmente exclusivas) possível seria que essa obra, na verdade, Raphael representaria o Spinoza, Ethel seria a personificação do conhecimento (Sophia na grécia, Shekinah no judaísmo) e Arthur seria o Pearson mesmo, que se mata por conta de um surto de ciúmes ao ver Raphael e Ethel em um romance após retornar de um tempo longe.
Cada uma das duas interpretações pode ter pedaços de verdade, fica a cargo do leitor pensar nisso!

O núcleo epistemológico da filosofia de Spinoza pode ser resumido em três princípios:
\begin{itemize}
	\item[1)] \textit{Deus sive natura} (Deus como a natureza);
	\item[2)] \textit{Cognitione causae} e \textit{Leges naturae universale} (Conheça as causas e leis da natureza universais); e
	\item[3)] \textit{Amor Dei intellectuais} (O amor intelectual de Deus).
\end{itemize}

Vamos entender estes princípios mais a fundo e mostrar como, no romance do Pearson, cada um deles aparece!
A primeira dica é que, no romance, o Arthur (que foi afirmado como representando Pearson por ele mesmo) vai à biblioteca procurar um livro do Maimônides para entender as ideias do Spinoza, e tem uma cena mística quando ele vai à estante e encontra Raphael magicamente com o livro na mão, como se Spinoza entregasse o antedimento de Maimônides ao Pearson.
Aqui, vale contextualizar: em sua obra filosófica, Maimônides ensina sua concepção não antropomórfica de Deus (``no mundo onde vivemos, as ações de Deus são ações da natureza'' (ha-pe\(^{(}\)ulot ha\(^{)}\)elohiyoth, ha-pe\(^{(}\)ulot ha-teb\(^{(}\)ayoth))), conforme a argumentação siléptica de Gikatilla (1248-1310) -- \textit{teba} (substantivo) corresponde à natureza ou substância, \textit{tab}\(^{(}\)\textit{a} (verbo) seria estampado, atribuído, ou formulado, \textit{matbe\(^{(}\)a} é tipo ou fórmula, \textit{mutb\(^{(}\)a} é impresso e \textit{tebiyoth-\(^{(}\)ayn} é intuição (lit. impressão-no-olho).
Esse processo também se aplica à relação das palavras latinas \textit{causare, causa}, que são etimologicamente associadas a \textit{cudere, cusum} de martelar, forjar, estampar ou atribuir, sugerindo paralelos entre a formulação de Maimônides e os seus intérpretes, até chegar no Spinoza\footnote{A mesma analogia se repete entre causa e cousa no Português de Portugal, a lingua materna de Spinoza!}, ou seja, faz sentindo entender que Deus seria a causa do universo na filosofia Spinoziana.

O segundo princípio é quase um corolário: se Deus é a causa, devemos tentar entender as relações causais, manifestadas como as leis da natureza! O Ser eterno e infinito chamado Deus age pela mesma necessidade com a qual ele existe, e o conhecimento das leis universais é o método que temos para conhecer a natureza e Deus, sendo este o trabalho dos cientistas para tentar obter um conhecimento racional do mundo criado por Deus.

O terceiro princípio tem a ver com uma nota positiva, respondendo a pergunta ``Será que este conhecimento sequer é alcançável?'', ou seja, ``Será que Deus é bom?'' e permitiu acesso às leis universais? O terceiro princípio afirma que sim, e que dado esforço suficiente na procura do conhecimento, é possível obter entendimento do universo; a possibilidade de obter conhecimento das causas e das leis que regem a natureza é a maior manifestação do amor de Deus conosco, representado graficamente pela famosa Escada de Jacó pela qual os anjos sobem e descem.
Até por conta dessa justificativa, o reverendo Bayes estava dedicado fortemente a esse projeto, levando à sua formulação estatística mencionada.

Em Hebraico, a palavra \textit{kalah} que designa noiva também significa todo (como se a noiva completasse o noivo), permitindo entender ainda mais a função da noiva no conto de Pearson -- quando Arthur estava comemorando o Solstício de Verão nas montanhas e com fogueiras, e ele se encontra com Rapahael, ele diz que conhecer as causas é bobagem, não existindo causas externas para as coisas e que isso não passa de nós tentarmos achar justificativas para o que vimos no mundo, e a racionalidade estaria apenas dentro de nós, não no mundo.
A forma descritiva seria, então, tudo que podemos querer, não uma abordagem causal; Raphael, por sua vez, discorda veementemente dessa perspectiva, frustrando Arthur e seguindo seu caminho solitariamente após se despedir dele e de sua noiva, deixando Raphael passando um tempo com ela enquanto ele passeava sozinho com seu cachorro Gaspar. Infelizmente, quando ele volta, Raphael e Ethel estavam apaixonados, levando ao seu suicídio e o fim da história contada por Locki, prenunciando toda a carreira do Pearson e como ele refaz toda a estatística como era conhecida.

O ferramental que Pearson desenvolve é uma maravilha técnica, incluindo a função de verossimilhança, valores-p, goodness-of-fit, etc, mas trouxeram os vários problemas associados a cada uma delas, como veremos, e a popularização destes métodos levou à popularização da sua filosofia positivista também.
\end{document}

\documentclass{article}
\usepackage{amsmath}
\usepackage{amsthm}
\usepackage{amssymb}
\usepackage{amsfonts}
\usepackage[margin=2.5cm]{geometry}
\usepackage{graphicx}
\usepackage[export]{adjustbox}
\usepackage{fancyhdr}
\usepackage[portuguese]{babel}
\usepackage{hyperref}
\usepackage{lastpage}
\usepackage{physics}

\pagestyle{fancy}
\fancyhf{}

\rfoot{P\'agina \thepage \hspace{1pt} de \pageref{LastPage}}

\hypersetup{
    colorlinks,
    citecolor=black,
    filecolor=black,
    linkcolor=black,
    urlcolor=black
}

\newtheorem*{def*}{\underline{Defini\c c\~ao}}
\newtheorem*{prop*}{\underline{Propriedades:}}
\newtheorem*{proof*}{\underline{Prova:}}
\newtheorem*{thm*}{\underline{Teorema:}}
\renewcommand\qedsymbol{$\blacksquare$}

\title{Resumo Te\'orico - Derivadas e Integrais}
\author{Renan Wenzel}
\date{\today}

\begin{document}
    \maketitle
    \newpage

    \tableofcontents

    \newpage
    \section{Derivadas}
    \subsection{Defini\c c\~oes e Propriedades}
    \begin{def*}
        Dada uma fun\c c\~ao $f:[a, b]\rightarrow\mathbb{R}$ cont\'inua em $x_0$, dizemos que ela \'e deriv\'avel em $x_0$ se existe o limite:
        $$
        \lim_{x\to{x_0}}\frac{f(x) - f(x_0)}{x - x_0} = f^{'}(x_0) = \frac{d}{dx}f(x_0).
        $$
    \end{def*}
    Note que escrevendo $h = x - x_0$, a defini\c c\~ao acima equivale ao limite 
    $$
    \lim_{h\to{0}}\frac{f(x_0 + h) - f(x_0)}{h} = f^{'}(x_0) = \frac{d}{dx}f(x_0).
    $$
    \begin{prop*}
        A derivada satisfaz as seguintes propriedades:
        \begin{enumerate}
            \item[Propriedade I)] $$(f + g)'(x_0) = f'(x_0) + g'(x_0).$$
            \item[Propriedade II)] $$(f \cdot g)'(x_0) = f'(x_0)g(x_0) + g'(x_0)f(x_0).$$
            \item[Propriedade III)] $$(cf)'(x_0) = cf'(x_0),\quad c\in\mathbb{R}.$$
            \item[Propriedade IV)] $$\biggl(\frac{f}{g}\biggr)'(x_0) = \frac{f'(x_0)g(x_0) - g'(x_0)f(x_0)}{(g(x))^2}.$$
            \item[Propriedade V)] $$f(g(x_0))' = f'(g(x_0))g'(x_0).$$
            \item[Propriedade VI)] $$\frac{d}{dx} e^x = e^x, \quad\frac{d}{dx} ln(x) = \frac{1}{x}.$$  
            \item[Propriedade VII)] $$\frac{d}{dx} x^n = nx^{n-1}.$$
            \item[Propriedade VIII)] $$(\sin)'(x_0) = (\cos)(x_0), \quad (\cos)'(x_0) = -\sin(x_0).$$ 
            \item[Propriedade IX)] $$(f^{-1})'(x_0) = \frac{1}{f'(f^{-1}(x_0))}.$$ 
        \end{enumerate}
    \end{prop*}
    Resuminho: A derivada pode ser vista como a taxa de mudan\c ca de uma fun\c c\~ao, al\'em de ser super \'util no estudo do gr\'afico das fun\c c\~oes, como veremos 
    posteriormente. Por agora, familiarize-se com as propriedades e tente prov\'a-las, \'e um bom treino.

    \begin{def*}
        Se a derivada de uma fun\c c\~ao for cont\'inua num ponto $x_0$ e o limite 
        $$
        \lim_{x\to{x_0}}\frac{f^{'}(x) - f^{'}(x_0)}{x - x_0} = f^{''}(x_0) = \frac{d^2}{dx^2}f(x_0)
        $$
        existir, chamamos este valor de segunda derivada de f em $x_0$.
    \end{def*}
    Este processo pode ser repetido "infinitamente", contanto que os limites continuem existindo. 

    \begin{def*}
        Seja I um intervalo e $f: I\rightarrow\mathbb{R}$ uma fun\c c\~ao. Diremos que $x_0\in{I}$ \'e um ponto de m\'aximo local de f, se existir $\delta > 0$
        tal que $f(x) \leq f(x_0)$ para todo $x\in(x_0 - \delta, x_0 + \delta)\cap{I}.$ Neste caso, diremos que $f(x_0)$ \'e um m\'aximo local. Se o que ocorrer for 
        $f(x) \geq f(x_0)$, ent\~ao diremos que $f(x_0)$ \'e um m\'inimo local. Em qualquer dos casos, $x_0\in{I}$ ser\'a chamado de ponto extremo local.
    \end{def*}

    \begin{def*}
        Seja I um intervalo e $f: I\rightarrow\mathbb{R}$ uma fun\c c\~ao. Diremos que $x_0\in{I}$ \'e um ponto de m\'aximo global de f, se $f(x) \leq f(x_0)$ para todo 
        $x\in{I}.$ Neste caso, diremos que $f(x_0)$ \'e um m\'aximo global. Se o que ocorrer for $f(x) \geq f(x_0)$, ent\~ao diremos que $f(x_0)$ \'e um m\'inimo local.
        Em qualquer dos casos, $x_0\in{I}$ ser\'a chamado de ponto extremo global.
    \end{def*}

    \begin{def*}
        Um ponto cr\'itico de uma fun\c c\~ao f \'e um ponto c em que $f'(x) = 0$ ou $f'(c)$ n\~ao existe.
    \end{def*}

    \subsubsection{Resultados Importantes}
    Come\c camos com o teorema de Rolle, que afirma que se uma fun\c c\~ao for cont\'inua e diferenci\'avel num intervalo em que os valores do ponto inicial e final
    coincidem, ent\~ao essa fun\c c\~ao assume seu m\'aximo ou m\'inimo em um ponto deste intervalo. 
    \begin{thm*}
        Seja $f:[a, b]\rightarrow\mathbb{R}$ uma fun\c c\~ao cont\'inua em $[a, b]$ e diferenci\'avel em $(a, b)$. Se f(a) = f(b), ent\~ao existir\'a $c\in(a, b)$ 
        tal que $f'(c) = 0.$ 
    \end{thm*}

    Com o teorema de Rolle como base, vamos ao Teorema do Valor M\'edio, um dos, se n\~ao o mais importante resultado do curso:
    \begin{thm*}
        Seja $f:[a, b]\rightarrow\mathbb{R}$ uma fun\c c\~ao cont\'inua em $[a, b]$ e diferenci\'avel em $(a, b)$. Ent\~ao, existe $c\in(a, b)$ tal que
        $$
            f(b) - f(a) = f'(c)(b - a),
        $$
        equivalente a 
        $$
            f'(c) = \frac{f(b) - f(a)}{b - a}.
        $$
    \end{thm*}

    Tendo estas duas ferramentas, \'e poss\'ivel estudar a fundo pontos de m\'aximo, m\'inimo e comportamento de fun\c c\~oes quanto ao seu crescimento ou decrescimento,
    al\'em da concavidade delas.

    \begin{thm*}
        Sejam $f:D_f\subset\mathbb{R}\rightarrow\mathbb{R}$ uma fun\c c\~ao cont\'inua e c um ponto cr\'itico de f.

        \quad Se o sinal de f' mudar de positivo para negativo em c, ent\~ao f ter\'a um m\'aximo local em c.

        \quad Se o sinal de f' mudar de negativo para positivo em c, ent\~ao f ter\'a um m\'inimo local em c.
    \end{thm*}

    \begin{thm*}
        Seja f uma fun\c c\~ao cont\'inua num intervalo $[a, b]$ e diferenci\'avel em $(a, b)$. 
        
        \quad Se $f'(x) > 0$ para todo $x\in(a, b)$, ent\~ao f ser\'a estritamente crescente em $[a, b].$

        \quad Se $f'(x) < 0$ para todo $x\in(a, b)$, ent\~ao f ser\'a estritamente decrescente em $[a, b].$

        \quad Se $f'(x) = 0$ para todo $x\in(a, b)$, ent\~ao f ser\'a constante em $[a, b].$
    \end{thm*}

    \begin{thm*}
        Seja f uma fun\c c\~ao diferenci\'avel em $(a, b)$.

        \quad Se $f{''}(x) > 0$ para todo $x\in(a, b)$, ent\~ao f ter\'a concavidade para cima em $(a, b).$

        \quad Se $f{''}(x) < 0$ para todo $x\in(a, b)$, ent\~ao f ter\'a concavidade para baixo em $(a, b).$
    \end{thm*}
    
    A seguir, veremos a Regra de L'Hopital, que permite calcular um limite a partir dos limites das derivadas das fun\c c\~oes, o que normalmente simplifica a conta.
    \begin{thm*}
        Sejam f e g fun\c c\~oes deriv\'aveis num intervalo com $g'(x)\neq0$ para todo $x\in{I}$. Ent\~ao, se $\lim_{x\to{p}}f(x) = \lim_{x\to{p}}g(x) = 0$ ou 
        $\lim_{x\to{p}}f(x) = \lim_{x\to{p}}g(x) = \infty$ e se 
        $$
            \lim_{x\to{p}}\biggl(\frac{f'(x)}{g'(x)}\biggr) = \lambda, \quad\lambda\in\mathbb{R},
        $$
        segue que 
        $$ 
            \lim_{x\to{p}}\biggl(\frac{f(x)}{g(x)}\biggr) = \lambda.
        $$
    \end{thm*}
    
    Por fim, as expans\~oes de Taylor permitem estudar fun\c c\~oes cont\'inuas e diferenci\'aveis como polin\^omios, que s\~ao muito mais simples:
    \begin{thm*}
        A s\'erie de Taylor de uma fun\c c\~ao $f:[a, b]\rightarrow\mathbb{R}$ infinitamente diferenci\'avel (ou seja, todas as derivadas existem) no ponto $x_0$ \'e
        dada por:
        $$
            f(x) = \sum_{i=0}^{\infty}\frac{f^{(i)(x_0)}}{i!}(x-x_0)^i.
        $$
        Tamb\'em pode-se definir o polin\^omio de Taylor de grau n como
        $$
            p(x) = \sum_{i=0}^{n}\frac{f^{(i)(x_0)}}{i!}(x-x_0)^i.
        $$
    \end{thm*}
    Da defini\c c\~ao acima, segue que a s\'erie de Taylor da fun\c c\~ao \'e o limite de n indo pra infinito do polin\^omio de Taylor.\vspace{0.5cm}

    Segue abaixo um algor\'itmo para analisar o gr\'afico de uma fun\c c\~ao:

    Determine, se poss\'ivel, os pontos em que f se anula e os intervalos em que ela \'e positiva ou negativo. 
    
    Em seguida, encontre as ass\'intotas verticais e horizontais
    de f e os pontos cr\'iticos de f. 
    
    A seguir, estude o sinal de f' para determinar o crescimento de f. 
    
    Calcular f'' para dizer a concavidade da fun\c c\~ao em cadaintervalo.

    \newpage
    \section{Integra\c c\~ao}
    \subsection{Defini\c c\~oes e Propriedades}
    \begin{def*}
        Uma antiderivada de uma fun\c c\~ao $f:[a, b]\rightarrow\mathbb{R}$ \'e uma fun\c c\~ao deriv\'avel F definida em $[a, b]$ tal que
        $$  
            \frac{d}{dx}(F(x) + k) = f(x), \quad k\in\mathbb{R}
        $$
        Denotamos a fam\'ilia de primitivas de f por:
        $$
            \int f(x)dx = F(x) + k, \quad k\in\mathbb{R},
        $$
        tamb\'em chamada integral indefinida de f com respeito a x.
    \end{def*}
    \begin{prop*}    
        Assim como a derivada, a integral satisfaz algumas propriedades:
        \begin{enumerate}
            \item[Propriedade I)] $$\int f(x) + g(x)dx = \int f(x)dx + \int g(x)dx.$$
            \item[Propriedade II)] $$\int cf(x)dx = c\int f(x)dx.$$
            \item[Propriedade IV)] $$|\int f(x)dx| \leq \int |f(x)dx.$$
        \end{enumerate}
    \end{prop*}
    A ideia da integral indefinida por si n\~ao apresenta muito significado e, para isso, \'e preciso introduzir as integrais definidas, respons\'aveis por grande
    parte das aplica\c c\~oes das integrais. Antes, no entanto, introduz-se a base das integrais definidas, as somas de Riemann e as parti\c c\~oes.
    \begin{def*}
        Uma parti\c c\~ao de um intervalo fechado $[a, b]$ \'e um subconjunto $P: a = t_0 < t_1 < t_2 < \cdots < t_n = b.$ Dada uma parti\c c\~ao e uma fun\c c\~ao 
        $f:[a, b]\rightarrow\mathbb{R}$, a soma inferior e a soma superior s\~ao:
        $$
            s(f, P):=\sum_{i=1}^{n}m_i\Delta{t_i}, \quad S(f, P):= \sum_{i=1}^{n}M_i\Delta{t_i},
        $$
        em que $m_i = \inf\{f(x): x\in(t_{i-1}, t_{i})\}$ e $M_i = \sup\{f(x): x\in(t_{i-1}, t_{i})\}$ e $\Delta{t_i} = t_i - t_{i-1}.$
    \end{def*}
    \begin{prop*}
        As somas superiores e inferiores satisfazem as seguintes propriedades: Sejam $f:[a, b]\rightarrow\mathbb{R}$ uma fun\c c\~ao limitada, 
        $P: a = t_0 < t_1 < \cdots < t_n = b$ e $Q: a = q_0 < q_1 < \cdots < q_n = b$ parti\c c\~oes de $[a, b].$
        \begin{enumerate}
            \item[Propriedade I)] $$m(b-a)\leq{s(f, P)}\leq{S(f, P)}\leq{M(b-a)}, \quad m=\inf\{f(x): x\in[a, b]\}, \quad M=\sup\{f(x): x\in[a, b]\}.$$
            \item[Propriedade II)] $$s(f, P) \leq s(f, P\cup{Q}) \leq S(f, P\cup{Q}) \leq S(f, Q).$$ 
        \end{enumerate}
    \end{prop*}
    \begin{def*} 
        Seja $f:[a, b]\rightarrow\mathbb{R}$ uma fun\c c\~ao limitada. Escrevemos:
        $$
            \int_{\underline{a}}^{b} f(x)dx := \lim_{\Delta{t}\to0} s(f, P) = \sup_{P}\{s(f, P)\}
        $$
        e
        $$
            \int_{a}^{\overline{b}} f(x)dx := \lim_{\Delta{t}\to0} S(f, P) = \inf_{P}\{S(f, P)\}
        $$
        para a integral definida inferior e superior de f com respeito a x, respectivamente. No caso em que 
        $\int_{\underline{a}}^{b} f(x)dx = \int_{a}^{\overline{b}} f(x)dx$, dizemos que f \'e integr\'avel em $[a, b]$ e escrevemos simplesmente
        $$
            \int_{\underline{a}}^{b} f(x)dx = \int_{a}^{\overline{b}} f(x)dx = \int_{a}^{b} f(x)dx.
        $$
    \end{def*}
    \begin{prop*}
        As integrais definidas satisfazem:
        \begin{enumerate}
            \item[Propriedade I)] $$m(b-a)\leq{s(f, P)}\leq{S(f, P)}\leq{M(b-a)}, \quad m=\inf\{f(x): x\in[a, b]\}, \quad M=\sup\{f(x): x\in[a, b]\}.$$
            \item[Propriedade II)] $$\int_{a}^{b} f(x)dx = \int_{a}^{c} f(x)dx + \int_{c}^{b} f(x)dx, \quad c\in(a, b) \text{(v\'alido para todas)}$$
            \item[Propriedade III)] $$\int_{a}^{\overline{b}} f(x) + g(x)dx \leq \int_{a}^{\overline{b}} f(x)dx + \int_{a}^{\overline{b}} g(x)dx, \quad\text{g limitada}.$$
            \item[Propriedade IV)] $$\int_{a}^{b} f(x)dx \leq \int_{a}^{b} g(x)dx, \quad f(x)\leq{g(x)} \text{(v\'alido para todas)}.$$ 
            \item[Propriedade V)] $$\int_{a}^{b} f(x)dx = -\int_{b}^{a} f(x)dx, \quad \int_{a}^{a} f(x)dx = 0.$$
        \end{enumerate}
    \end{prop*}
    \subsection{Resultados}

    Para encontrar as integrais indefinidas, existem t\'ecnicas que podemos utilizar para facilitar o processo. A primeira delas \'e a substitui\c c\~ao de vari\'avel,
    baseada em alterar "o dx para outra vari\'avel mais simples du", o equivalente \`a regra da cadeia para integrais:
    \begin{prop*}
        Sejam f e g tais que $\Im(g)\subset{D_f}$ e suponha que F \'e uma primitiva de f. Ent\~ao, $f(g(x))g'(x)$ tem como primitiva F(g(x)). Escrevendo F' como f
        e g(x) = u, segue a Regra da Substitui\c c\~ao:
        $$
            \int F'(g(x))g'(x)dx = \int f(u) du.
        $$
    \end{prop*}
    A outra \'e an\'aloga \`a regra da derivada do produto, mas para as integrais, tamb\'em conhecida como Integra\c c\~ao por Partes:
    \begin{prop*}
        Sejam $f, g:[a, b]\rightarrow\mathbb{R}$ deriv\'aveis em $(a, b).$ Ent\~ao, vale:
        $$
            \int f(x)g'(x)dx = f(x)g(x) - \int f'(x)g(x)dx.
        $$
    \end{prop*}
    \underline{Observa\c c\~ao:}(Tamb\'em tem a substitui\c c\~ao trigonom\'etrica, mas eu n\~ao sei comentar muito sobre ela, pe\c co desculpas.)

    Com rela\c c\~ao a uma fun\c c\~ao ser integr\'avel, temos os seguintes resultados:
    \begin{thm*}
        Toda fun\c c\~ao $f:[a, b]\rightarrow\mathbb{R}$ cont\'inua \'e integr\'avel.
    \end{thm*}
    \begin{thm*}
        Toda fun\c c\~ao $f:[a, b]\rightarrow\mathbb{R}$ cont\'inua possui primitiva.
    \end{thm*}
    Com base nesses dois, chegamos no Teorema Fundamental do C\'alculo, que disputa com o Teorema do Valor M\'edio como o mais importante do curso.
    \begin{thm*}
        Seja $f:[a, b]\rightarrow\mathbb{R}$ integr\'avel. Se F \'e uma primitiva de f, ent\~ao
        $$
            \int_{a}^{b} f(x)dx = \eval{F(x)}_{a}^{b} = F(b) - F(a).
        $$
    \end{thm*}

    \section{Aplica\c c\~oes e Usos da Integral}
    \subsection{\'Areas de Superf\'icies de Revolu\c c\~ao}
    \begin{def*}
        Seja $f:[a, b]\rightarrow\mathbb{R}$ positiva. Uma superf\'icie de revolu\c c\~ao gerada por f \'e obtida rotacionando 
        o gr\'afico de f em torno de x.
    \end{def*}
    Seja, agora, $f:[a, b]\rightarrow\mathbb{R}$ positiva e diferenci\'avel e considere a superf\'icie de revolu\c c\~ao gerada por ela.
    A sua \'area \'e:
    $$
        A_f = 2\pi\int_{a}^{b} f(x)\sqrt{1+(f'(x))^2}dx.
    $$
    \subsection{Comprimento de Gr\'afico de Fun\c c\~ao}
    Seja $f:[a, b]\rightarrow\mathbb{R}$ diferenci\'avel. O comprimento do gr\'afico da f \'e dado por:
    $$
        C_f = \int_{a}^{b} \sqrt{1+f'(x)^2}dx 
    $$
\end{document}

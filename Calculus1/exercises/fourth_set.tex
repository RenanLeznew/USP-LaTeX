\documentclass[exercícios_de_cálculo.tex]{subfiles}
\begin{document}
\section{Integrais e Primitivas}
\subsection{Exercício 1} Utilize a ideia de sólido de rotação para encontrar os volumes das
seguintes construções:

a.) Um cilindro de altura h e raio da base r;

b.) Um cone de altura h e raio da base r;

c.) de um dirigível simétrico, cuja parte inferior quando observada lateralmente é modelada
em termos do gráfico da função dada implicitamente por uma elipse.


\begin{sol*}

	a.) Pensando no formato do cilindro, concluimos que a função responsável por gerá-lo é dada
	por f(x) = r. Com isso, basta aplicarmos a fórmula para o volume de sólidos de rotação:
	$$
		V = \pi\int_{0}^{h}f(x)^2dx = \pi\int_{0}^{h}r^2dx = \pi{r^2}x\biggr\rvert_{0}^{h} = \pi{r^2}h.
	$$

	b.) Agora, usando a função $f(x) = \frac{r}{h}x$ como geratriz do cone, segue que
	$$
		V = \pi\int_{0}^{h}f(x)^2dx = \pi\int_{0}^{h}\frac{r^2}{h^2}x^2dx = \pi\frac{r^2}{h^2}\int_0^h x^2dx =
	$$
	$$
		= \pi\frac{r^2}{h^2}\frac{x^3}{3}\biggr\rvert_{0}^{h} = \pi\frac{r^2h^3}{3h^2} = \frac{\pi{r^2}h}{3}.
	$$

	c.) Observe que um dirigível simétrico é, essencialmente, uma elipse com eixo maior no x. Com isso,
	levando como base a fórmula geral da elipse:
	$$
		\frac{x^2}{a^2} + \frac{y^2}{b^2} = 1,
	$$
	isolamos o y (como f(x)) a fim de obter
	$$
		f(x) = \sqrt{(1-\frac{x^2}{a^2})b^2} = b\sqrt{1-\frac{x^2}{a^2}}.
	$$
	Agora, repetimos os processos anteriores para obtermos o volume do dirigível:
	$$
		V = \pi{b^2}\int_{-a}^{a} 1-\frac{x^2}{a^2}dx = \pi{b^2}\biggl(\int_{-a}^{a}1 dx - \int_{a}^{a}\frac{x^2}{a^2}\biggr) =
	$$
	$$
		= \pi{b^2}(2a - \frac{2}{3}a) = \frac{4\pi{b^2}a}{3}
	$$
	\qedsymbol
\end{sol*}
\subsection{Exercício 2}
Agora, encontre a área de superfície de cada um dos itens anteriores.

\begin{sol*}
	Começamos lembrando a fórmula para a área de superfície de um sólido de revolução:
	$$
		A = \int_{0}^{h}2\pi f(x)\sqrt{1 + f'(x)^2}dx.
	$$
	Tendo isso em mente, damos continuidade ao exercício:

	a.) Segue da fórmula acima que a área de superfície do cilindro é:
	$$
		A = 2\pi \int_{0}^{h} r\sqrt{1 + 0}dx = 2\pi rx\biggr\rvert_{0}^{h} = 2\pi rh.
	$$

	b.) Para o cone fornecido, uma aplicação simples da fórmula dada e um pouco de álgebra resulta em:
	$$
		A = 2\pi \int_{0}^{h} \frac{rx}{h}\sqrt{1 + \frac{r^2}{h^2}}dx =
		\frac{2\pi{r}}{h}\sqrt{\frac{h^2 + r^2}{h^2}}\int_{0}^{h}xdx =
	$$
	$$
		= \frac{2\pi{r}}{h}\sqrt{\frac{h^2 + r^2}{h^2}} \frac{h^2}{2} = \pi{r}\sqrt{r^2 + h^2}.
	$$
	\qedsymbol
\end{sol*}
\end{document}

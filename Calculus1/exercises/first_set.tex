\documentclass[exercícios_de_cálculo.tex]{subfiles}
\begin{document}
\section{Números Reais, Funções e Introdução a Limites}
\subsection{Exercícios de Funções e Panorama Geral}
\subsubsection{Exercício 1}
\paragraph{}Parte 1 - Se considerarmos
$$
	f(x) = \frac{x^3 + x^2 - x - 1}{x-1}, x\neq{1},
$$
então a que classe de funções ela pertence? Note que se efetuarmos a divisão polinomial, concluiremos que
$$
	\frac{x^3 + x^2 - x - 1}{x-1} = x^2 + 2x + 1.
$$
Isto significa que f é uma função polinomial de grau 2? Justifique e faça o passo-a-passo.
\begin{proof*}
	Considere a função
	$$
		f(x) = \frac{x^3 + x^2 - x - 1}{x-1}, x\neq{1},
	$$
	e $p(x) = x^3 + x^2 - x - 1, q(x) = x - 1,$ em que $p:\mathbb{R}\rightarrow\mathbb{R}$ e $q(x)\mathbb{R}/\{1\}\rightarrow\mathbb{R}$ são polinômios. Segue que:
	$$
		f(x) = \frac{x^3 + x^2 - x - 1}{x-1} = \frac{p(x)}{q(x)}, x\neq{1}.
	$$
	Por definição, uma função na forma de quociente de polinômios, em que o denominador é um polinômio com o domínio tal que ele nunca é nulo, é conhecida como uma funç\~ao racional.

	No entanto, é melhor lidar com frações de polinômios simplificados, ou seja, é preciso encontrar o fator comum entre ambos. Observe que, em x = 1,
	$$
		p(x) = 1^3 + 1^2 - 1 - 1 = 1 + 1 - 1 - 1 = 0
	$$
	e
	$$
		q(x) = 1 - 1 = 0,
	$$
	então (x-1) é um fator comum entre ambos, isto é, ele pode ser fatorado após manipular o polinômio p(x) . Assim, note que, somando e subtraindo fatores para que possamos fatorar (x-1) de p(x), chegamos em:
	$$
		p(x) = x^3 + x^2 - x - 1 = x^3 + (2x^2 - x^2) +(x - 2x) -1 = (x-1)(x^2 + 2x^2 + 1) = q(x)(x^2 + 2x^2 + 1),
	$$
	de forma que:
	$$
		f(x) = \frac{x^3 + x^2 - x - 1}{x-1} = \frac{p(x)}{q(x)} = \frac{q(x)(x^2 + 2x^2 + 1)}{q(x)} = x^2 + 2x^2 + 1, x\neq{1}.
	$$
	Logo, após efetivada a divisão, obtemos que f é uma função polinomial de grau 2.
	\qedsymbol
\end{proof*}

\newpage

\paragraph{}Parte 2 - Verifique as seguintes identidades:
\begin{itemize}
	\item[a)] \(x^{2} - a^{2} = (x-a)(x+a)\)
	\item[b)] \(x^{3} - a^{3} = (x-a)(x^{2}+ax+a^{2})\)
	\item[c)] \(x^{4} - a^{4} = (x-a)(x^{3}+ax^{2}+a^{2}x+a^{3})\)
	\item[d)] \(x^{5} - a^{5} = (x-a)(x^{4} + ax^{3} + a^{2}x^{2} + a^{3}x + a^{4})\)
	\item[e)] \(x^{n}-a^{n} = (x-a)(x^{n-1} + ax^{n-2} + a^{2}x^{n-3} + \dotsc + a^{n-2}x + a^{n-1})\), em que \(n\neq0\) é um natural.
\end{itemize}

\begin{proof*}
	Antes da formalização da prova, um bom começo é analisar a imagem cuidadosamente. De fato, ao fazer isso, observe a repetição do termo (x - a) à direita de cada igualdade. Outro ponto notável é que, para cada n, ocorre uma expans\~ao de $\sum_{i=0}^{n}a^ix^{n-i-1}$ ao lado de (x - a), em que $n\in\mathbb{N}$. Em outras palavras, isso está indicando fortemente a presença de uma hipótese indutiva para demonstrar o resultado.

	Com  efeito, provemos o caso base do item a, ou seja, n = 2. Considere o produto (x-a)(x+a):
	$$
		(x - a)(x + a) = x^2 + xa - ax - a^2 = x^2 + ax - ax - a^2 = x^2 - a ^2.
	$$
	Destarte, obtivemos o caso base como verdadeiro. Nessa linha de raciocínio, a hipótese indutiva afirma que, dado uma base verdadeira, o resultado será provado se, assumindo o caso n-1 como verdade, o caso n também será (pois assim, o caso 1 sendo verdadeiro implica que o 2 também é, consequentemente o 3, o 4, etc.). Suponha que o resultado vale para n - 1, isto é, para $n\neq{0}$
	$$
		x^{n - 1} - a^{n - 1} = (x - a)\biggl(\sum_{i=0}^{n-2}a^ix^{n-i-2}\biggr).
	$$
	Em partícular, somando $2a^{n-1}, $segue que:
	$$
		x^{n - 1} + a^{n - 1} = (x - a)\biggl(\sum_{i=0}^{n-2}a^ix^{n-i-2}\biggr) + 2a^{n-1}.
	$$
	Multiplicando ambos os lados por (x-a), chegamos em:
	$$
		(x - a)\biggl(\sum_{i=0}^{n-1}a^ix^{n-i-1}\biggr)  = (x - a)(x^{n-1} + \biggl(\sum_{i=1}^{n-2}a^ix^{n-i-1}\biggr) + a^{n-1})  = (x - a)(x^{n-1} + a^{n-1} + \biggl(\sum_{i=1}^{n-2}a^ix^{n-i-1}\biggr))  =
	$$
	$$
		= (x - a)(2a^{n-1} + (x - a)\biggl(\sum_{i=0}^{n-2}a^ix^{n-i-2}\biggr) + \biggl(\sum_{i=1}^{n-2}a^ix^{n-i-1}\biggr))  = (x - a)(2a^{n-1} + x^{n-1} - a^{n-1} + \biggl(\sum_{i=1}^{n-2}a^ix^{n-i-1}\biggr)) =
	$$
	$$
		= 2xa^{n-1} - 2a^n + (x-a)\biggl(x^{n-1} - a^{n-1} + \biggl(\sum_{i=1}^{n-2}a^ix^{n-i-1}\biggr)\biggr) = 2xa^{n-1} - 2a^n + x^{n} - xa^{n-1} -a^{n-1}x + a^{n} + (x-a)\biggl(\sum_{i=1}^{n-2}a^ix^{n-i-1}\biggr)\biggr) =
	$$
	$$
		= xa^{n-1} - a^n + x^{n} - xa^{n-1} + (x-a)\sum_{i=1}^{n-2}a^ix^{n-i-1} = (x^n - a^n) + (xa^{n-1} - ax^{n-1} + (x-a)\sum_{i=1}^{n-2}a^ix^{n-i-1}) .
	$$
	Analisemos o termo $(x-a)\sum_{i=1}^{n-2}a^ix^{n-i-1}$ antes de prosseguir. Temos:
	$$
		(x-a)\sum_{i=1}^{n-2}a^ix^{n-i-1} = x\sum_{i=1}^{n-2}a^ix^{n-i-1} - a\sum_{i=1}^{n-2}a^ix^{n-i-1} = \sum_{i=1}^{n-2}a^ix^{n-i} - \sum_{i=1}^{n-2}a^{i+1}x^{n-i-1} =
	$$
	$$
		= ax^{n-1} + a^2x^{n-2} + \cdots + a^{n-2}x^2 - a^2x^{n-2} - ... -a^{n-2}x^2 - a^{n-1}x = ax^{n-1} - a^{n-1}x.
	$$
	Juntando isso com a conta anterior, chegamos, finalmente, em:
	$$
		(x^n - a^n) + (xa^{n-1} - ax^{n-1} + (x-a)\sum_{i=1}^{n-2}a^ix^{n-i-1}) = (x^n - a^n) + (xa^{n-1} - ax^{n-1} + ax^{n-1} - a^{n-1}x ) =
	$$
	$$
		= x^n - a^n + 0 = x^n - a^n.
	$$
	Portanto,
	$$
		(x - a)\biggl(\sum_{i=0}^{n-1}a^ix^{n-i-1}\biggr) = x^n - a^n.
	$$
	Agora que provamos isso, utilizando n = 3, 4, 5, mostramos as identidades que faltam:
	\begin{itemize}
		\item[n=3:] $$ x^3 - a^3 = (x-a)\sum_{i=0}^{2}a^ix^{2-i} = (x-a)(a^2 + ax + x^2). $$
		\item[n=4:] $$ x^4 - a^4 = (x-a)\sum_{i=0}^{3}a^ix^{3-i} = (x-a)(a^3 + a^2x + ax^2 + x^3). $$
		\item[n=5:] $$ x^5 - a^5 = (x-a)\sum_{i=0}^{4}a^ix^{4-i} = (x-a)(a^4 + a^3x + a^2x^2 + ax^3 + x^4).$$
	\end{itemize}
	\qedsymbol
\end{proof*}

\subsubsection{Exercício 2}
\paragraph{}Parte 1 - Fazendo todos os detalhes e explicando todos os passos, explicite o domínio de cada umas das funções abaixo e calcule os produtos $f\cdot{}g, g\cdot{}h \text{ e } h\cdot{}i$, em que:
$$
	f(x) = 2x^3 - 5x^2 + 3, \hspace{0.5cm}
	g(x) = 3x^2 - x + 2, \hspace{0.5cm}
	h(x) = \frac{x^2 - 1}{x - 3} \hspace{0.2cm} \text{\&} \hspace{0.2cm}
	i(x) = \frac{x^3 - 1}{x^2 + 1}
$$
\begin{sol*}
	A priori, analisemos os domínios de cada uma das funções. Começando por f, levando em conta que, quando não explicitado, o dom\'inio de uma função é o maior subconjunto de $\mathbb{R}$ em que faz sentido defin\'i-la, temos $D_f = \mathbb{R}$, pois a funç\~ao n\~ao possui pontos problemáticos (com isso, queremos dizer um ponto em que, por exemplo, ter\'iamos $\frac{1}{0}$ ou $\sqrt{-x}, x > 0$ e $x\in\mathbb{R}.$) Analogamente, segue que o dom\'inio de g também é $D_g = \mathbb{R}.$

	Contudo, ao lidarmos com os domínios de h e i, é necessário ter cautela, já que são definidas por frações. No caso de h, seu dom\'inio é o conjunto dos reais tais que x - 3 não é nulo, ou seja,
	$$
		D_h = \{x\in\mathbb{R}: x - 3 \neq 0\} = \mathbb{R}/\{3\}.
	$$
	Em primeira vista, o caso da função i pode parecer o mesmo, ou seja, que vai ser definido como o conjunto dos reais a menos de um conjunto finito de pontos. No entanto, note que, para isso, seria preciso que $x^2 + 1 = 0, x\in\mathbb{R},$ o que nunca acontece (nos reais!). Portanto, i está definido em $D_{i} = \mathbb{R}$.

	Ademais, a forma de realizar produtos entre funções deve ser esclarecida: O produto entre duas funções f e g, definido ponto-a-ponto, é dado por
	$$
		(f\cdot{g})(x) = f(x) \cdot{g(x)}.
	$$
	Com isso em mente, vamos aos cálculos:
	\begin{itemize}
		\item[i.)] $f\cdot{g}$ (produto de f com g)
		      $$
			      (f \cdot{g})(x) = f(x)\cdot{g(x)} = (2x^3 - 5x^2 + 3)\cdot(3x^2 - x + 2) = 2x^3(3x^2 - x + 2) - 5x^2 (3x^2 - x + 2) + 3(3x^2 - x + 2) =
		      $$
		      $$
			      = 6x^5 - 2x^4 + 4x^3 - 15x^4 + 5x^3 - 10x^2 + 9x^2 - 3x + 6 = 6x^5 - 17x^4 + 9x^3 - x^2 + 6
		      $$
		\item[ii.)]$g\cdot{h}$  (produto de g com h)
		      $$
			      (g\cdot{h})(x) = g(x)\cdot{h(x)} = (3x^2 - x + 2)\cdot\biggl(\frac{x^2 - 1}{x - 3}\biggr) = \biggl(\frac{(3x^2 - x + 2)(x^2 - 1)}{x - 3}\biggr) =
		      $$
		      $$
			      = \biggl(\frac{3x^4 - 3x^2 - x^3 + x + 2x^2 - 2}{x - 3}\biggr) = \biggl(\frac{3x^4 - x^2 - x^3 + x - 2}{x - 3}\biggr)
		      $$
		\item[iii.)]$h\cdot{i}$ (produto de h com i)
		      $$
			      (h\cdot{i})(x) = h(x)\cdot{i(x)} = \biggl(\frac{x^2 - 1}{x - 3}\biggr)\biggl(\frac{x^3 - 1}{x^2 + 1}\biggr) = \biggl(\frac{(x^2 - 1)(x^3 - 1)}{(x - 3)(x^2 + 1)}\biggr)
		      $$
		      $$
			      = \biggl(\frac{x^5 - x^2 -x^3 + 1}{x^3 + x - 3x^2 - 3}\biggr)
		      $$
	\end{itemize}
	\qedsymbol
\end{sol*}

\paragraph{}Parte 2 - Sabendo que $\sin{x}$ não é uma função racional, mostre que a funç\~ao $\tan{x}$ n\~ao pode ser uma funç\~ao racional.
\begin{proof*}
	A priori, sabemos que, para uma função ser racional, ela deve ser o quociente de dois polinômios. Analogamente, se uma função n\~ao é racional, ela n\~ao pode ser escrita como o quociente de dois polinômios.

	A posteriori, suponha que $\sin{x}$ não é uma função racional. Defina
	$$
		\tan(x) = \frac{\sin(x)}{\cos(x)}.
	$$
	Desta forma, segue de cara que $\tan(x)$ não é uma função racional, pois um de seus componentes, no caso, $\sin(x)$, n\~ao pode ser escrito como o quociente de dois polinômios, de forma que, mesmo se $\cos(x)$ fosse racional, ainda assim seria impossível escrevê-la como o quociente desejado. Portanto, a tangente $\tan(x)$ n\~ao pode ser uma funç\~ao racional.
	\qedsymbol
\end{proof*}

\subsubsection{Exercício 3}
\paragraph{}Parte 1  - Defina os conceitos de injetividade e sobrejetividade.
\begin{sol*}
	Antes de definí-los explicitamente, é importante conhecer um pouco de suas utilidades. O primeiro deles, a injetividade, lida com a questão da unicidade na imagem da função, tanto é que também é conhecido como funç\~ao 1-1, enquanto a sobrejetividade lida com o ``alcance'' da funç\~ao. Se ambos os casos ocorrem, chamamos a funç\~ao de bijeç\~ao, uma classe muito importante pois ela relaciona cada elemento de cada um dos conjuntos (o dom\'inio e o contra-dom\'inio) unicamente, de forma que há um inverso pra funç\~ao, mas isso é outro tópico.

	Destarte, definamos ambas matematicamente. Dados dois conjuntos A e B não-vazios, seja $f:A\rightarrow{B}$ uma função entre os dois conjuntos. Dizemos que:
	\begin{itemize}
		\item[a)] f é uma função injetora se, para $a_1, a_2\in{A}, f(a_1) = f(a_2)$ implica que $a_1 =  a_2.$
		\item[b)] f é uma função sobrejetora se, dado $b\in{B}$, existe (pelo menos) um elemento $a\in{A}$ tal que $f(a) = b.$
	\end{itemize}

	Com essas definições em mente, retomemos o primeiro parágrafo. A unicidade mencionada segue pois, para uma aplicação qualquer de A em B ser uma função, ela precisa que, dados $a_1, a_2\in{A}$, caso $a_1 = a_2, f(a_1) = f(a_2)$. A injetividade diz o oposto, ou seja, se $f(a_1) = f(a_2), a_1 = a_2$. Juntando os dois, uma funç~ao injetora obedece $f(a_1) = f(a_2) \text{ se, e somente se, } a_1 = a_2,$ dados $a_1, a_2\in{A},$ tal que cada elemento de um conjunto define unicamente um elemento no outro. Quanto à sobrejetividade, ela define quando uma funç\~ao tem alcance máximo, pois como cada $b\in{B}$ pode ser escrito como a funç\~ao aplicada a algum elemento de A, segue que $B \subset f(A)$, tal que, como por definiç\~ao $f(A) \subset B$, temos $f(A) = B$, ou seja, a imagem da funç\~ao é o contra-domínio inteiro.
	\qedsymbol
\end{sol*}

\paragraph{}Parte 2 - Mostre que a função $f(x) = \sin(x), x\in[0, \pi]$ não é injetora, mas para $x\in[0, \frac{\pi}{2}]$ ela é.
\begin{proof*}
	Vamos mostrar uma contradição engraçada. Suponha que, de fato, $f(x) = \sin(x)$ é injetora no intervalo $[0, \pi]$. Em particular, temos:
	$$
		\sin(0) = 0 = \sin(\pi) \Rightarrow 0 = \pi.
	$$
	Se isso fosse verdade, alguns desastres aconteceriam. Dentre eles, não existiriam círculos, pois todos eles poderiam ser vistos como pontos, já que sua área, $\pi\cdot{r^2} = 0$ para todo r, ou seja, também não existiria engenharia e, quem sabe, nem mesmo o universo. Isso está obviamente errado. Logo, $\sin(x)$ n\~ao pode ser injetora em $[0, \pi].$

	De lado com os cataclismas e fins do mundo, considere, agora, o intervalo $[0, \frac{\pi}{2}].$ Sabemos que a função seno é estritamente crescente nesse intervalo, que é o primeiro quadrante. Assim, temos, para $x, y\in[0, \frac{\pi}{2}],$
	$$
		\sin(x) < \sin(y), x < y \text{ ou } \sin(x) > \sin(y), x > y.
	$$
	Assim, a única forma de $\sin(x) = \sin(y)$ é quando x = y, que é a exata definição de uma função injetora.
	\qedsymbol
\end{proof*}

\paragraph{}Parte 3 - Faça as seguintes composições: $f\circ{g}, g\circ{f}, f\circ{h} \text{ e } h\circ{f}$, em que:
$$
	f(x) = -3x + 2, \hspace{0.5cm} g(x) = 3x^2 - x  + 2, \hspace{0.2cm} \text{ e } \hspace{0.2cm} h(x) = \frac{x^2 - 1}{x - 3}.
$$
\begin{sol*}
	Antes de dar início às contas propriamente ditas, note que, ao compor h com f, ou f com h, o dom\'inio de f mudará de $D_f = \mathbb{R}$ para $D_f = \mathbb{R}/\{3\}.$ Feita essa observação, sigamos em frente:
	\begin{itemize}
		\item[i)]$$(f\circ{g})(x) = f(g(x)) = f(3x^2 - x + 2) = -3(3x^2 - x + 2) + 2 = -9x^2 + 3x - 6 + 2 = -9x^2 + 3x - 4.$$
		\item[ii)]$(g\circ{f})(x)$
		      $$
			      (g\circ{f})(x) = g(f(x)) = g(-3x + 2) = 3(-3x + 2)^2 +3x - 2 + 2 =
		      $$
		      $$
			      = 3(9x^2 - 12x + 4) + 3x = 27x^2 - 36x + 12 + 3x = 27x^2 - 33x + 12.
		      $$
		\item[iii)]$(f\circ{h})(x)$
		      $$
			      (f\circ{h})(x) = f(h(x)) = f(\frac{x^2 - 1}{x - 3}) = -3\biggl(\frac{x^2 - 1}{x - 3}\biggr) + 2 =
		      $$
		      $$
			      = \frac{-3x^2 + 3}{x-3} + 2 = \frac{-3x^2 + 3 + 2x - 6}{x-3} = \frac{-3x^2 + 2x -3}{x-3}.
		      $$
		\item[iv)]$(h\circ{f})(x) $
		      $$
			      (h\circ{f})(x) = h(f(x)) = h(-3x + 2) = \frac{(-3x + 2)^2 - 1}{-3x+2 - 3} = \frac{9x^2 -12x + 4 - 1}{-3x -1} = -\frac{3(3x^2 - 4x + 1)}{3x+1}
		      $$
	\end{itemize}
	\qedsymbol
\end{sol*}
\subsection{Um Panorama Geral}
\subsubsection{Exercício 4}
\paragraph{} Parte 1 - Quais são os dois principais problemas a que se refere o Cálculo diferencial e integral?
\begin{sol*}
	O cálculo pode ser visto como o estudo de ``infinitos'', então, partindo desse princípio, é poss\'ivel iluminar a mente com relação à resposta para essa pergunta. Nessa linha de racioc\'inio, o cálculo diferencial é apresentado, normalmente, com o problema de motivaç\~ao da reta tangente e do passo de uma funç\~ao. Explicitamente falando, a busca do ponto único para cada reta tangente e o que acontece com a fraç\~ao $\frac{\Delta{f(x)}}{\Delta{x}}$ quando $\Delta{x}$ fica cada vez menor, ou seja, como uma mudança até a vizinhanç\a imediata de x afeta sua funç\~ao f(x). Em essência, ambos os problemas s\~ao os mesmos, pois lidam com a taxa de variaç\~ao de x em n\'iveis infinitesimais, da\'i que vem a parte do cálculo diferencial que estuda infinitos, mas s\~ao as coisas infinitamente pequenas.

	Tratando-se do cálculo integral, diferente da busca pela taxa de variação, ele lida com as áreas de gráficos de funções, isto é, como calcular a área de um gráfico arbitrário. Para isso, a noção de infinito como mencionada previamenta aparece em forma de aproximações cada vez mais finas por meio de retângulos que possuem tamanhos maiores ou menores para uma dada secç\~ao do gráfico. Quanto mais retângulos forem usados, ou seja, quando menor forem seus tamanhos, mas maiores forem seus números, melhor será a aproximaç\~ao da área de dada funç\~ao, tal que, quando alcançados infinitos retângulos, a área da figura formada pelo gráfico é exata. Essa formaç\~ao de retângulos cada vez menores também pode ser compreendida como uma divis\~ao do intervalo da reta que representa o domínio da funç\~ao em partes iguais cada vez menores, tal que quanto maior o n\'umero de partes, melhor a aproximaç\~ao, até que, novamente, com infinitas partes, chega-se no valor exato da área da funç\~ao.
	\qedsymbol
\end{sol*}

\paragraph{} Parte 2 - Utilize a construção da secante ao gráfico para obter a tangente, em que $f(x) = x^5$, explicitando a reta tangente no ponto (a, f(a)) e deixando claro como obteve o coeficiente angular desta reta.
\begin{sol*}
	Antes de tudo, lembre-se que uma reta secante a um gráfico é tal que ela passa por exatos dois pontos dele. Considerando que o que liga dois pontos é uma reta, a secante pode ser interpretada como a taxa de variação da função em dois pontos $x, x_0\in{D_f}$ dados. Em outras palavras, se S for a secante, temos
	$$
		S = \frac{f(x) - f(x_0)}{x - x_0} = \frac{\Delta{f(x)}}{\Delta{x}}.
	$$
	Retome, agora, o exerçício 4.1. A palavra "taxa de variação" também aparece lá, então, consequentemente, a secante apareceu. Vamos seguir nessa linha de racioc\'inio, junto com a noç\~ao de infinitesimalidade, para obter a taxa de variaç\~ao imediata (Nome chique para derivada) de f no ponto $x_0$. Com efeito, a variaç\~ao imediata é o valor de S quando $x$ se torna $a$ e, para isso, utilizemos a noç\~ao de limite:
	$$
		\lim_{x\to{a}} S = \lim_{x\to{a}}\biggl(\frac{f(x) - f(a)}{x - a}\biggr) = \lim_{x\to{a}}\biggl(\frac{x^5 - a^5}{x - a}\biggr) = \lim_{x\to{a}}\biggl(\sum_{n=0}^{4}a^nx^{4-n}\biggr) = \sum_{n=0}^{4}a^n\lim_{x\to{a}}x^{4-n}=
	$$
	$$
		= \sum_{n=0}^{4}a^na^{4-n} = \sum_{n=0}^{4}a^{n+4-n} = \sum_{n=0}^{4}a^4 = 5a^4 .
	$$
	Como esse valor é exato e, consequentemente uma reta passando em um ponto só, ele representa a tangente em (a, f(a)) com coeficiente angular igual a 5, pois a soma possui 5 termos, independentes do índice, $a^4$ iguais (apesar de n = 4, a soma começa do 0, então são n+1 = 5 termos). Esse processo pode, de fato, ser generalizado para um monômio de grau n, $f(x) = x^n$, para obtermos que a tangente ao ponto $(x_0, f(x_0))$ é dada por $nx^{n-1}.$ Essa regra de obtenç\~ao da tangente de um monômio é mais conhecida como regra do tombo em cursos iniciantes, e é uma das bases para fazer a maioria das derivações do Cálculo diferencial. Uma observaç\~ao interessante, para finalizar, é o uso do exerc\'icio 1.1 parte 2 para chegar no resultado que quer\'iamos, sendo este o caso em que n = 4 dividido por (x - a).
	\qedsymbol
\end{sol*}

\paragraph{} Parte 3 - Calcule a área de $f(x) = x^3$ dividindo o intervalo $[0, 1]$ em 7 parte iguais. Qual o valor aproximado da área a que se chega? Dividindo-se o intervalo em mais partes, digamos $\lfloor{\pi}\rfloor\cdot{10^{36}}$, espera-se que esta aproximação do valor real da área melhore ou piore?
\begin{sol*}
	A priori, dividiremos o intervalo em 7 partes iguais, ou seja, I = [0, 1] se quebra nos seguintes pedaços:
	$$
		I_1 = \biggl[0, \frac{1}{7}\biggr],     I_2 = \biggl[\frac{1}{7}, \frac{2}{7}\biggr],       I_3 = \biggl[\frac{2}{7}, \frac{3}{7}\biggr],    I_4 = \biggl[\frac{3}{7}, \frac{4}{7}\biggr],    I_5 = \biggl[\frac{4}{7}, \frac{5}{7}\biggr], I_6 = \biggl[\frac{5}{7}, \frac{6}{7}\biggr], I_7 = \biggl[\frac{6}{7}, 1\biggr].
	$$
	Agora, vejamos como a função f(x) se comporta neles, ou melhor, como sua área é influenciada por deslocamentos ao longo de cada pedaço de I. O princípio por trás desse processo é aproximar área por retângulos menores que a total e depois por retângulos maiores que ela. Comecemos, então, por esses menores, ou seja, analisemos o valor de f nos pontos da esquerda dos intervalos. Em seguida, somaremos eles, dividindo pelo número de divisões, no caso, 7, e repetiremos para os pontos à direita dos intervalos. Segue que
	$$
		L_7 = \frac{1}{7}\biggl(f(0) + f(\frac{1}{7}) + f(\frac{2}{7}) + f(\frac{3}{7}) + f(\frac{4}{7}) + f(\frac{5}{7}) + f(\frac{6}{7})\biggr) =
	$$
	$$
		\frac{1}{7}(0 + \frac{1}{7^3}(1^3 + 2^3 + 3^3 + 4^3 + 5^3 + 6^3)) = \frac{1}{7^4}\biggl(1^3 + 2^3 + 3^3 + 4^3 + 5^3 + 6^3\biggr) = \frac{441}{7^4}
		= 0.183.$$
	Repetindo o processo para os números das pontas direitas dos intervalos, temos:
	$$
		R_7 = \frac{1}{7}\biggl(f(\frac{1}{7}) + f(\frac{2}{7}) + f(\frac{3}{7}) + f(\frac{4}{7}) + f(\frac{5}{7}) + f(\frac{6}{7}) + f(1)\biggr) =
	$$
	$$
		\frac{1}{7}(\frac{1}{7^3}(1^3 + 2^3 + 3^3 + 4^3 + 5^3 + 6^3) + 1) = \frac{1}{7^4}\biggl(1^3 + 2^3 + 3^3 + 4^3 + 5^3 + 6^3\biggr) + \frac{1}{7} = \frac{441}{7^4} + \frac{1}{7}= 0.183 + 0.142 = 0.325.
	$$
	Assim, obtemos que a área da função, A, é tal que:
	$$
		L_7 < A < R_7.
	$$
	\par{}Ademais, dividindo-se o intervalo em $3\cdot{10^{36}} = \lfloor{\pi}\rfloor\cdot{10^{36}}$ partes iguais, note que os retângulos se tornam mais numerosos e com bases menores. Assim, a área individual de cada um deles será menor, tal que, ao somá-los, chegaremos em uma área mais aproximada, uma aproximação mais refinada do valor da área da função. Isso esconde o principal mecanismo da soma de Riemann, ou seja, da definiç\~ao da Integral Definida, no sentido que, ao tomar a soma de Riemann, busca-se dividir os intervalos em um quantidade infinitamente pequena, tal que "ao chegar em infinito", a área, antes aproximada, torna-se exata. Este exemplo ilustra isso através da comparaç\~ao de intervalos antes n\~ao t\~ao pequenos (7 divisões) com intervalos minúsculos ($\lfloor{\pi}\rfloor\cdot{10}^{36}$ divisões),
	\qedsymbol
\end{sol*}
\end{document}

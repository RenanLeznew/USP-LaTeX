   \documentclass{article}
 \usepackage{amsmath}
 \usepackage{amsthm}
 \usepackage{amssymb}
 \usepackage{pgfplots}
 \usepackage{amsfonts}
 \usepackage[margin=2.5cm]{geometry}
 \usepackage{graphicx}
 \usepackage[export]{adjustbox}
 \usepackage{fancyhdr}
 \usepackage[portuguese]{babel}
 \usepackage{hyperref}
 \usepackage{lastpage}
 \usepackage{mathtools}

 \pagestyle{fancy}
 \fancyhf{}

 \pgfplotsset{compat = 1.18}

 \hypersetup{
     colorlinks,
     citecolor=black,
     filecolor=black,
     linkcolor=black,
     urlcolor=black
 }
 \newtheorem*{def*}{\underline{Definition}}
 \newtheorem*{theorem*}{\underline{Theorem:}}
 \newtheorem{example}{\underline{Example:}}[section]
 \newtheorem*{proof*}{\underline{Proof:}}
 \renewcommand\qedsymbol{$\blacksquare$}
 \newcommand{\Lin}[1]{Lin_{\mathbb{K}}({#1})}

 \rfoot{P\'agina \thepage \hspace{1pt} de \pageref{LastPage}}

 \title{Explanations}
 \author{Renan Wenzel}
 \date{\today}

 \begin{document}
 \maketitle
    Mostre que $\lim\limits_{x\to0}x^2 = 0.$ Dado $\epsilon > 0$, queremos encontrar um $\delta > 0$ satisfazendo 
  ``$0 < |x-0| < \delta$, ent\~ao $|x^2 - 0| < \epsilon$".  Observe que
 $$
   |x^{2} - 0| < \epsilon\Longleftrightarrow |x - 0| < \sqrt{\epsilon}
 $$
 Com isso, colocando $\delta = \sqrt{\epsilon}$, segue que, se $0 < |x - 0| < \delta$, ent\~ao
 $$
  |x - 0| < \delta\Rightarrow |x - 0| < \sqrt{\epsilon}\Rightarrow |x^2 - 0| < \epsilon
 $$
 Portanto, pela defini\c c\~ao de limite, $\lim\limits_{x\to0}x^{2}=0.$ \qedsymbol
 \end{document}

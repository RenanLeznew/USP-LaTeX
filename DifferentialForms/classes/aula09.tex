\documentclass[../differential_forms.tex]{subfiles}
\begin{document}
\section{Aula 09 - 17 de Setembro, 2025}
\subsection{Motivações}
\begin{itemize}
	\item Produto exterior e propriedades.
\end{itemize}
\subsection{Produto Exterior}
Seja \(\mathcal{E}=(e_1,\dotsc , e_{m})\) uma base de E e \(\mathcal{E}^{*} = (e_{1}^{*}, \dotsc , e_{m}^{*})\) uma base dual de \(E^{*}.\)
\begin{def*}
	Para uma sequência de índices \(s=(i_1,\dotsc ,i_r)\), definimos o \textbf{produto tensorial das formas lineares} correspondentes por
	\[
		e_{(s)}^{*} = e_{i_1}^{*}\otimes e_{i_2}^{*}\otimes \dotsc \otimes e_{i_r}^{*}\in (E^{*})^{\otimes_r}.
	\]
	A forma que ele atua em vetores \(v_1,\dotsc ,v_r\in E\) é
	\[
		e_{(s)}^{*} (v_1,\dotsc ,v_r) = \prod\limits_{k=1}^{r}e_{i_k}^{*}(v_{k}). \quad \square
	\]
\end{def*}
\begin{def*}
	Para um subconjunto \(J=\{j_1<\dotsc <j_r\}\subseteq \{1,\dotsc ,m\}\), definimos a \textbf{forma alternada} associada como
	\[
		e_{[J]}^{*} \coloneqq \sum\limits_{\sigma \in S_{r}}^{} \mathrm{sgn}(\sigma )e_{(j_{\sigma (1)}, \dotsc , j_{\sigma (r)})}^{*},
	\]
	onde \(S_r\) é o grupo simétrico com r elementos. \(\square\)
\end{def*}
\begin{def*}
	Seja \(\alpha = (\alpha_{j}^{i})\) uma matriz \(m\times r\). Para um subconjunto \(J=\{j_1<\dotsc <j_r\},\) denotamos por \(\alpha^{J}\) a submatriz \(r\times r\) obtida escolhendo as linhas de \(\alpha \) com índices em \(J.\; \square\)
\end{def*}
\begin{prop*}
	Se \(\mathcal{E}=(e_1,\dotsc , e_{m})\) é uma base de E e
	\[
		v_{j} = \sum\limits_{i}^{}\alpha_{j}^{i}e_{i},\quad j=1,\dotsc ,r.
	\]
	Então, para cada subconjunto J,
	\[
		e_{[J]}^{*}(v_1,\dotsc , v_r) = \det{(\alpha^{J})}.
	\]
\end{prop*}
\begin{tcolorbox}[
		skin=enhanced,
		title=Observação,
		fonttitle=\bfseries,
		colframe=black,
		colbacktitle=cyan!75!white,
		colback=cyan!15,
		colbacklower=black,
		coltitle=black,
		drop fuzzy shadow,
		%drop large lifted shadow
	]
	A introdução destas notações ajuda a distinguir o produto tensorial puro da anti simetrização:  \(e_{(s)}^{*}\) indica o \textbf{produto tensorial}, enquanto \(e_{[J]}^{*}\) indica a \textbf{forma alternada}. Esta distinção eixa explícita a passagem entre cálculo concreto de produtos tensoriais e determinantes de submatrizes.
\end{tcolorbox}

Nas próximas partes da aula, definiremos o produto exterior e mostraremos a relação entre o que conhecemos como volumes de paralelepípedos e
r-formas, mostrando que calcular volumes realmente é, em sua essência, calcular determinantes. Essa parte não tem exatamente uma construção uniforme,
mas elas são as mesmas ``a menos de isomorfismo''.

Definir o produto exterior de um espaço vetorial consiste em considerar a base do dual, olhar a ordem do vetor que deseja e analisar o espaço das formas
utilizando os funcionais.

Uma vez introduzidos os vetores e o seu produto, podemos construir uma nova operação fundamental, chamada \textit{produto exterior}, que surge da necessidade de considerar
apenas as combinações assimétricas de vetores.

Conforme vimos anteriormente, os tensores podem ser multiplicados de forma natural: se T é um p-tensor e S um q-tensor, definimos o tensor (p+q)-linear \(T\otimes S\) por
\[
	(T\otimes S)(v_1, \dotsc , v_{p+q}) = T(v_1, \dotsc , v_p)S(v_{p+1}, \dotsc , v_{p+q}),
\]
e o objeto \(T\otimes S\) leva o nome de \textit{produto tensorial de T com S}, que é associativo e distributivo com relação à soma, mas não é comutativo.

O primeiro passo para explorar o produto exterior é, assim, definirmos o \textit{espaço exterior}; mais precisamente,
\begin{def*}
	Seja E um espaço vetorial de dimensão m. O \textbf{espaço exterior de ordem r}, denotado por \(\bigwedge^{r}E\), é definido como a imagem do
	operador de antissimetrização
	\begin{align*}
		A: & \mathcal{L}^{r}(E; \mathbb{R})\rightarrow\mathcal{L}^{r}(E; \mathbb{R})                                                                                                                                                        \\
		   & (v_{1}^{*}\otimes \dotsc \otimes v_{r}^{*})\longmapsto A(v_{1}^{*}\otimes \dotsc \otimes v_{r}^{*}) \coloneqq \sum\limits_{\sigma \in S_r}^{}\mathrm{sgn}(\sigma )v_{\sigma (1)}^{*}\otimes \dotsc \otimes v_{\sigma (r)}^{*},
	\end{align*}
	ou seja,
	\[
		\bigwedge^{r}E\coloneqq A(\mathcal{L}^{r}(E; \mathbb{R})).
	\]
	Os elementos deste espaço levam o nome de \textbf{r-covetores}.

	Equivalentemente, podemos ver este espaço como o espaço vetorial gerado pelas expressões formais
	\[
		v_1 \wedge \dotsc \wedge v_r \coloneqq A(v_{1}^{*}\otimes \dotsc \otimes v_{r}^{*}),
	\]
	com \(v_{i}\in E\). \(\square\)
\end{def*}

Olhando para as duas definições equivalentes, um elemento
\[
	v_1\wedge\dotsc \wedge v_r\in \bigwedge^{r}E
\]
pode ser interpretado tanto como uma classe de tensores, quanto como um funcional no dual \(\mathcal{A}^{r}(E; \mathbb{R})\); assim, outra descrição
possível do espaço exterior de orem r é como o dual do espaço das formas alternadas:
\[
	\bigwedge^{r} E \coloneqq (\mathcal{A}^{r}(E; \mathbb{R}))^{*}.
\]
Aqui, para cada r-upla \((v_1, \dotsc , v_r)\) de vetores em \(E^{r},\) temos um funcional que naturalmente define um elemento de \(\bigwedge^{r}E\), sendo ele
\begin{align*}
	\wedge : & \mathcal{A}^{r}(E; \mathbb{R})\rightarrow \bigwedge^{r}E                              \\
	         & \varphi \longmapsto \varphi (v_1, \dotsc , v_r)\coloneqq v_1\wedge \dotsc \wedge v_r.
\end{align*}
\begin{example}
	Considere \(E = \mathbb{R}^{3}\) como espaço que usaremos de exemplo inicial. Para o caso do espaço externo de ordem 1, estamos fazendo nada mais que considerando
	os funcionais alternados que atuam em apenas 1 vetor, ou seja,
	\[
		\bigwedge^{1}\mathbb{R}^{3} = (\mathcal{A}^{1}(\mathbb{R}^{3}; \mathbb{R}))^* = \mathrm{span}\{e_1, e_2, e_3\} = \mathbb{R}^{3}.
	\]

	Quando consideramos a ordem 2, mexemos com os índices em pares, que, no caso de \(\mathbb{R}^{3},\) consiste em
	\[
		e_1 \wedge e_2,\; e_1 \wedge e_3,\;\&\; e_{2}\wedge e_3.
	\]
	Logo,
	\[
		\bigwedge^{2}\mathbb{R}^{3} = \mathrm{span}\{e_1\wedge e_2, e_1\wedge e_3, e_2\wedge e_3\}.
	\]

	Por fim, com para a álgebra exterior de ordem 3, a única combinação dos 3 vetores da base de \(\mathbb{R}^{3}\) é
	\[
		e_1 \wedge e_2 \wedge e_3,
	\]
	tal que
	\[
		\bigwedge^{3}\mathbb{R}^{3} = \mathrm{span}\{e_1 \wedge e_2\wedge e_3\}.
	\]

	Além disso, o produto exterior entre \(v=(v_1, v_2, v_3), w=(w_1, w_2, w_3)\in \mathbb{R}^{3}\) é
	\[
		v\wedge w = (v_1w_2 - v_2w_1)(e_1\wedge e_2) + (v_1w_3 - v_3w_1)(e_1\wedge e_3) + (v_2w_3-v_3w_2)(e_2\wedge e_3).
	\]
\end{example}

Nesses exemplo, podemos confirmar quais são as bases tendo em vista que há uma fórmula para a dimensão da álgebra exterior de ordem r para um espaço
vetorial de dimensão m, dada por
\[
	\mathrm{dim}\biggl(\bigwedge^{r}E\biggr) = \binom{n}{r}.
\]

\begin{example}
	Para \(E=\mathbb{R}^{2}\), temos também
	\[
		\bigwedge^{1}\mathbb{R}^{2} = \mathbb{R}^{2},\quad \mathrm{dim}\biggl(\bigwedge^{1}\mathbb{R}^{2} \biggr) = \binom{2}{1} = 2,
	\]
	e de fato
	\[
		\bigwedge^{1}\mathbb{R}^{2} = \mathrm{span}\{e_1, e_2\},
	\]
	enquanto que
	\[
		\bigwedge^{2}\mathbb{R}^{2} = \mathrm{span}\{e_1\wedge e_2\}, \quad \mathrm{dim}\biggl(\bigwedge^{2}\mathbb{R}^{2}\biggr) = \binom{2}{2}=1,
	\]
	sendo que um elemento desse espaço terá forma
	\[
		v\wedge w = \det{\begin{pmatrix}
				v_1 & w_1 \\
				v_2 & w_2
			\end{pmatrix}} e_1 \wedge e_2.
	\]
\end{example}

\begin{lemma*}
	Sejam \(f\in \mathcal{A}^{p}(E; \mathbb{R})\) e \(g\in \mathcal{A}^{q}(E; \mathbb{R})\). Se \(A(F) = 0 \), então
	\[
		f \wedge g = 0 = g \wedge f.
	\]
\end{lemma*}
\begin{proof*}
	Sejam \(S_{p}\) o grupo das permutações em \(I_{p}\) e \(S_{p+q}\) o das permutações em \(I_{p+q}\); além disso, considere \(G\subseteq S_{p+q}\) o subgrupo formado
	pelas permutações que fixam \(\{p+1, \dotsc , p+q\}\). Podemos, assim, relacionar de forma natural um elemento \(\sigma \in G\) com um de \(S_{p}\) que também denotaremos
	por \(\sigma\).

	Consideremos a aplicação
	\begin{align*}
		A^{G}(f \otimes g) = \sum\limits_{\sigma \in S_{p}}^{}\mathrm{sgn}(\sigma) \underbrace{\sigma}_{\mathclap{\in S_{p+q}}} \overbrace{(f\otimes g)}^{\in \mathcal{A}^{p+q}} & = \sum\limits_{\sigma \in S_{p}}^{}\mathrm{sgn}(\sigma) \sigma (f)g                       \\
		                                                                                                                                                                         & = \biggl(\sum\limits_{\sigma \in S_{p}}^{}\mathrm{sgn}(\sigma) \sigma (f)\biggr)\otimes g \\
		                                                                                                                                                                         & = 0.
	\end{align*}

	Porém, isso foi feito para apenas um subgrupo; precisamos repetir a conta para o seguinte conjunto: dado \(\rho \in S_{p+q}\),
	\[
		\{\rho \sigma :\; \sigma \in G\} = \rho G.
	\]
	Seja
	\begin{align*}
		A^{\rho G}(f\otimes g) = \sum\limits_{\sigma \in G}^{}\mathrm{sgn}(\rho \sigma )(\rho \sigma )(f\otimes G) & = \sum\limits_{\sigma \in G}^{}\mathrm{sgn}(\rho) \mathrm{sgn}(\sigma )\rho (\sigma (f\otimes g))                                          \\
		                                                                                                           & = \mathrm{sgn}(\rho )\rho \biggl(\underbrace{\bigl(\sum\limits_{\sigma \in G}^{}\mathrm{sgn}(\sigma)\sigma f\bigr)}_{A(f)}\otimes g\biggr) \\
		                                                                                                           & = 0.
	\end{align*}
	Como os conjuntos dessa forma são classes laterais de \(S_{p+q}\), temos uma partição deste grupo e, como em cada parte a soma é zero, o resultado segue. \qedsymbol
\end{proof*}
\subsection{Produto Interno em \(\bigwedge^{r}E\)}
\begin{def*}
	Um \textbf{produto interno} em um espaço vetorial E é uma forma bilinear
	\[
		\langle \cdot , \cdot  \rangle:E\times E\rightarrow \mathbb{R}
	\]
	satisfazendo:
	\begin{itemize}
		\item[\textbf{Simetria})] para quaisquer \(u, v\) em E,
		      \[
			      \langle u, v \rangle = \langle v, u \rangle
		      \]
		\item[\textbf{Positividade})] sempre que \(u\neq 0\),
		      \[
			      \langle u, u \rangle > 0. \;\square
		      \]
	\end{itemize}
\end{def*}
O isomorfismo
\[
	\mathcal{L}_2(E; \mathbb{R})\cong  \mathcal{L}(E, E^{*})
\]
associa, a cada produto interno \(\langle \cdot , \cdot  \rangle\) em E, uma transformação linear
\[
	\xi :E\rightarrow E^{*}
\]
definido de forma explícita para um funcional \(\xi (u)\in E^{*}\) como
\[
	\xi (u)v = \langle u, v \rangle,\quad \forall v\in E,
\]
chamado \textbf{isomorfismo canônico associado ao produto interno}.

Agora, seja E um espaço munido de produto interno; queremos caracterizar a estrutura para os espaços de r-covetores, ou seja, para cada \(\bigwedge^{r}E\).

Para tanto, fixemos \(r>0\) e consideremos a forma 2r-linear
\[
	f:E\times \dotsc \times E\rightarrow \mathbb{R}
\]
definida por
\[
	f(u_1, \dotsc , u_r, v_1, \dotsc , v_r) = \det{(\langle u_{i}, v_{j} \rangle)}.
\]
Como o determinante de uma matriz é uma função multilinear alternada das colunas e das linhas, a forma pertence a
\[
	\mathcal{A}^{r, r}(E; \mathbb{R}),
\]
ou seja, f é alternada nos u's e nos v's separadamente, que nos induz naturalmente uma forma bilinear
\[
	\langle \cdot , \cdot  \rangle:\bigwedge^{r}E \times \bigwedge^{r}E\rightarrow \mathbb{R}
\]
dada por
\[
	\langle u_1\wedge \dotsc \wedge u_r, v_1\wedge \dotsc \wedge v_r \rangle = \det{(\langle u_{i}, v_{j} \rangle)}
\]
para quaisquer \(u_1, \dotsc , u_r, v_1, \dotsc , v_r\in E\).

\end{document}

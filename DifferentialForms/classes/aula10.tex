\documentclass[../differential_forms.tex]{subfiles}
\begin{document}
\section{Aula 10 - 22 de Setembro, 2025}
\subsection{Motivações}
\begin{itemize}
	\item Produto interno;
	\item (Co)Vetores Decomponíveis.
\end{itemize}
\subsection{(Co)Vetores Decomponíveis}
Para relembrar, na aula passada definimos
\begin{def*}
	Um \textbf{produto interno} em um espaço vetorial E é uma forma bilinear
	\[
		\langle \cdot , \cdot  \rangle:E\times E\rightarrow \mathbb{R}
	\]
	satisfazendo:
	\begin{itemize}
		\item[\textbf{Simetria})] para quaisquer \(u, v\) em E,
		      \[
			      \langle u, v \rangle = \langle v, u \rangle
		      \]
		\item[\textbf{Positividade})] sempre que \(u\neq 0\),
		      \[
			      \langle u, u \rangle > 0. \;\square
		      \]
	\end{itemize}
\end{def*}

E propusemos um candidato a produto interno no produto exterior de um espaço vetorial E. Hoje, mostraremos que ele é de fato um produto interno, por meio do

\begin{theorem*}
	Sejam E um espaço vetorial com produto interno e \(\bigwedge^{r}E\) o r-produto exterior de E. A forma bilinear
	\[
		\langle \cdot , \cdot  \rangle:\bigwedge^{r}E \times \bigwedge^{r}E\rightarrow \mathbb{R}
	\]
	dada por
	\[
		\langle u_1\wedge \dotsc \wedge u_r, v_1\wedge \dotsc \wedge v_r \rangle = \det{(\langle u_{i}, v_{j} \rangle)}
	\]
	para quaisquer \(u_{i}, v_{j}\) em E, é um produto interno.
\end{theorem*}
\begin{proof*}
	Para verificar que a 2-forma é simétrica, note que
	\begin{align*}
		\langle u_1\wedge \dotsc\wedge u_r, v_1\wedge \dotsc \wedge v_r \rangle & = \det{(\langle u_{i}, v_{j} \rangle)}                                      \\
		                                                                        & = \det{(\langle v_{j}, u_{i} \rangle)}                                      \\
		                                                                        & =\langle v_1\wedge \dotsc \wedge v_r, u_1 \wedge \dotsc \wedge u_r \rangle.
	\end{align*}
	Vamos mostrar agora que
	\[
		\langle u_1\wedge \dotsc \wedge u_r, u_1\wedge \dotsc \wedge u_r \rangle = \det{(\langle u_{i}, u_{j} \rangle)} \geq 0,
	\]
	ou seja, que o produto interno do vetor com ele mesmo é positivo; aproveitando, mostraremos que ele se anula se, e somente se, \(u_1\wedge \dotsc \wedge u_r\) é nulo, ou seja, os vetores \(u_1, u_2, \dotsc , u_r\) são linearmente
	dependentes.

	Para isso, começaremos mostrando o resultado por meio de uma base ortonormal: seja \((e_1, \dotsc , e_m)\) uma base ortonormal de E; ela induz uma base natural em \(\bigwedge^{r}E\) formada pelos vetores
	\[
		e_{J} = e_{j_1}\wedge \dotsc \wedge e_{j_r}, \quad J = \{j_1 < \dotsc < j_r\} \subseteq I_{m}.
	\]
	Para \(K = \{k_1 < \dotsc < k_r\}\), temos
	\[
		e_{k} = e_{k_1} \wedge \dotsc \wedge e_{k_r}
	\]
	e, pela definição,
	\[
		\langle e_{J}, e_{K} \rangle = \det{(\langle e_{j}, e_{k} \rangle)},\quad j\in J,\; k\in K.
	\]

	\textbf{\underline{Afirmação}:} Se \(K\neq J\), existe \(k_{0}\not\in J\) tal que \(\langle e_{j}, e_{k_{0}} \rangle = 0\) para todo j em J. Com efeito,
	\[
		\begin{pmatrix}
			\langle e_1, e_1 \rangle & \langle e_1, e_2 \rangle & \langle e_1, e_3 \rangle & \dotsc & \langle e_1, e_{r-1} \rangle & \langle e_1, e_r \rangle \\
			\langle e_2, e_1 \rangle & \langle e_2, e_2 \rangle & \langle e_2, e_3 \rangle & \dotsc & \langle e_2, e_{r-1} \rangle & \langle e_2, e_r \rangle \\
			\vdots                   & \vdots                   & \vdots                   & \ddots & \vdots                       & \vdots                   \\
			\langle e_r, e_1 \rangle & \langle e_r, e_2 \rangle & \langle e_r, e_3 \rangle & \dotsc & \langle e_r, e_{r-1} \rangle & \langle e_r, e_r \rangle \\
		\end{pmatrix}.\quad  \blacktriangle
	\]
	Decorrente da afirmação, o determinante correspondente à definição de \(\langle e_{J}, e_{K} \rangle\) possui uma coluna nula.

	Por outro lado, temos
	\[
		\langle e_{J}, e_{J} \rangle = 1,
	\]
	tendo em vista que
	\[
		\det{(\langle e_{i}, e_{j} \rangle)} = \det{(\mathrm{Id}_{r})} = 1,\quad i, j\in J,
	\]
	ou seja, \(\{e_{J}\}_{J\subseteq I_{m}}\) é uma base ortonormal de \(\bigwedge^{r}E\) relativa à forma bilinear \(\langle \cdot , \cdot  \rangle\).

	Armados com estas ferramentas, vamos mostrar que
	\[
		\langle u_1\wedge \dotsc \wedge u_r, u_1\wedge \dotsc \wedge u_r \rangle \geq 0.
	\]
	Sejam
	\[
		x = \sum\limits_{J}^{}\xi^{J}e_{J} \quad\&\quad y = \sum\limits_{K}^{}\eta^{K}e_{K},\quad | J | = | K | = r;
	\]
	pela ortogonalidade da base \((e_{J})\), temos
	\begin{align*}
		\langle x, y \rangle & = \biggl\langle \sum\limits_{J}^{}\xi^{J}e_{J}, \sum\limits_{K}^{}\eta^{K}e_{K} \biggr\rangle \\
		                     & = \sum\limits_{J}^{}\sum\limits_{K}^{}\xi^{J}\eta^{K}\langle e_{J}, e_{K} \rangle             \\
		                     & = \left\{\begin{array}{ll}
			                                \sum\limits_{J}^{}\xi^{J}\eta^{J}, & \quad J = K   \\
			                                0,                                 & \quad J\neq K
		                                \end{array}\right..
	\end{align*}
	Em particular,
	\[
		\langle x, x \rangle = \sum\limits_{J}^{}(\xi^{J})^{2} \geq 0,
	\]
	com igualdade se, e somente se, \(x=0\).

	Portanto,
	\[
		\langle u_1\wedge \dotsc \wedge u_r, v_1\wedge \dotsc \wedge v_r \rangle = \det{(\langle u_{i}, v_{j} \rangle)}
	\]
	define de fato um produto interno. \qedsymbol

\end{proof*}
\begin{tcolorbox}[
		skin=enhanced,
		title=Observação,
		fonttitle=\bfseries,
		colframe=black,
		colbacktitle=cyan!75!white,
		colback=cyan!15,
		colbacklower=black,
		coltitle=black,
		drop fuzzy shadow,
		%drop large lifted shadow
	]
	O valor \(\det{(\langle u_{i}, u_{j} \rangle)}\) é chamado de \textbf{Gramiano dos vetores} \(u_1, \dotsc , u_r\).
\end{tcolorbox}

Tendo um produto interno, podemos definir uma norma, que leva a um comprimento e mostraremos como isso relaciona-se ao volume de um paralelepípedo.
\end{document}

\documentclass[../differential_forms.tex]{subfiles}
\begin{document}
\section{Aula 13 - 22 de Outubro, 2025}
\subsection{Motivações}
\begin{itemize}
	\item Derivada Exterior de Aplicações.
\end{itemize}
\subsection{Um Pouco Sobre a Estrutura do Módulo das k-Formas}
Com tudo que fora visto até o momento, deduzimos que o módulo das formas diferenciais de grau k possui uma base natural, formada pelas formas
\[
	dx_{i_1}\wedge \dotsc \wedge dx_{i_{k}},\quad 1\leq i_{1}<\dotsc <i_{k}\leq n,
\]
ou seja, qualquer k-forma diferencial \(\omega \) pode ser escrita unicamente como
\[
	\omega  = \sum\limits_{1\leq i_{1}<\dotsc <i_{k}\leq n}^{}a_{i_1, \dotsc , i_{k}}(x)dx_{i_1}\wedge \dotsc \wedge dx_{i_{k}},\; a_{i_1,\dotsc , i_{k}}\in C^{\infty}(\mathbb{R}^{n}),
\]
e a independência linear delas sobre \(\mathcal{C}^{\infty}(\mathbb{R}^{n})\) faz com que não existam combinações não triviais das formas da base que resultem na forma zero, ou seja, o módulo é um \textit{módulo livre}. Assim, podemos
finalmente dar a estrutura explícita do conjunto de k-formas diferenciais:
\begin{prop*}
	O conjunto das formas diferenciais de grau k em \(\mathbb{R}^{n}\), com a soma e o produto externo conforme definidos na aula passada, forma um módulo livre sobre \(\mathcal{C}^{\infty}(\mathbb{R}^{n})\), tendo como base as formas
	\[
		dx_{i_1}\wedge \dotsc \wedge dx_{i_{k}}, \quad 1\leq i_1 < \dotsc < i_{k}\leq n.
	\]
\end{prop*}
Lembremos, também, que a definição de produto exterior é feita de forma tal que, se \(\varphi_1, \dotsc , \varphi_{k}\) são 1-formas, então o produto exterior \(\varphi_1 \wedge \dotsc \wedge \varphi_{k}\) coincide com a k-forma definida por
\[
	(\varphi_1 \wedge \dotsc \wedge \varphi_{k})(v_1, \dotsc , v_{k}) = \det{(\varphi_{i}(v_{j}))}.
\]

\begin{tcolorbox}[
		skin=enhanced,
		title=Observação,
		fonttitle=\bfseries,
		colframe=black,
		colbacktitle=cyan!75!white,
		colback=cyan!15,
		colbacklower=black,
		coltitle=black,
		drop fuzzy shadow,
		%drop large lifted shadow
	]
	Com relação às propriedades das k-formas, note que, embora \(dx_{i}\wedge dx_{i} = 0\), não é verdade que \(\omega\wedge \omega = 0\) para toda forma \(\omega \). Para um contra exemplo, considere
	\[
		\omega = x_1 \mathrm{d}x_1\wedge \mathrm{d}x_2 + x_2 \mathrm{d}x_3 \wedge \mathrm{d}x_4;
	\]
	então,
	\[
		\omega \wedge \omega = 2x_1x_2 \mathrm{d}x_1 \wedge \mathrm{d}x_2 \wedge \mathrm{d}x_3 \wedge \mathrm{d}x_4
	\]
\end{tcolorbox}
\subsection{A Derivada Exterior}
Uma das propriedades mais importantes das formas diferenciais é a maneira pela qual elas se comportam sob aplicações diferenciáveis: seja \(f:\mathbb{R}^{n}\rightarrow \mathbb{R}^{m}\) uma aplicação diferenciável. Com isso, f induz uma aplicação \(f^{*}\) que leva k-formas em \(\mathbb{R}^{m}\) em k-formas em \(\mathbb{R}^{n}\), definida por
\[
	(f^{*}\omega )(p)(v_1, \dotsc , v_{k}) = \omega(f(p))(df_p (v_1), \dotsc , df_p(v_k)).
\]
Em particular, se g é uma 0-forma, então \(f^{*}(g) = g\circ f\), ou seja, a \(f^{*}\) equivale a um tipo generalizado de substituição de variáveis.
\begin{prop*}
	Sejam \(f:\mathbb{R}^{n}\rightarrow \mathbb{R}^{m}\) diferenciável, \(\omega \) e \(\varphi \) k-formas em \(\mathbb{R}^{m}\), e \(g:\mathbb{R}^{m}\rightarrow \mathbb{R}\) uma 0-forma. Então:
	\begin{align*}
		 & \text{a) } f^{*}(\omega +\varphi ) = f^{*}\omega + f^{*}\varphi;                                              \\
		 & \text{b) } f^{*}(g\omega) = f^{*}(g)f^{*}(\omega );\text{ e}                                                  \\
		 & \text{c) } f^{*}(\varphi_1 \wedge \dotsc \wedge \varphi_{k}) = f^{*}\varphi_1 \wedge \dotsc f^{*}\varphi_{k}.
	\end{align*}
\end{prop*}
\begin{example}
	Considere a 2-forma em \(\mathbb{R}^{2}\setminus{\{(0, 0)\}}\) dada por
	\[
		\omega  = -\frac{y}{x^{2}+y^{2}}\mathrm{d}x + \frac{x}{x^{2}+y^{2}}\mathrm{d}y.
	\]
	Em coordenadas polares, obtidas pela aplicação \(f(r, \theta ) = (r\cos^{}{(\theta )}, r \sin^{}{(\theta )})\), temos
	\[
		\mathrm{d}x = \cos^{}{(\theta )}\mathrm{d}r - r \sin^{}{(\theta )}\mathrm{d}\theta , \quad \mathrm{d}y = \sin^{}{(\theta )}\mathrm{d}r + r \cos^{}{(\theta )}\mathrm{d}\theta.
	\]
	Portanto, \(f^{*}\omega  = \mathrm{d}\theta \).
\end{example}
\begin{prop*}
	Seja \(f:\mathbb{R}^{n}\rightarrow \mathbb{R}^{m}\) diferenciável. Então:
	\begin{align*}
		 & \text{a) }f^{*}(\omega \wedge \varphi ) = (f^{*}\tau )\wedge (f^{*}\varphi ); \text{ e} \\
		 & \text{b) }(f\circ g)^{*}\omega = g^{*}(f^{*}(\omega )).
	\end{align*}
\end{prop*}
Equipados com estas ferramentas e propriedades, finalmente introduzimos a generalização do diferencial de uma função:
\begin{def*}
	Seja \(\omega = \sum\limits_{I}^{}a_{I}dx_{I}\) uma k-forma em \(\mathbb{R}^{n}\). Definimos a \textbf{derivada exterior de }\(\omega \) por
	\[
		d\omega  = \sum\limits_{I}^{}da_{I}\wedge dx_{I}.\; \square
	\]
\end{def*}
\begin{example}
	Seja \(\omega  = xyz \mathrm{d}x + yz \mathrm{d}y + (x+z)\mathrm{d}z\). Então,
	\begin{align*}
		\mathrm{d}\omega & = \mathrm{d}(xyz) \wedge \mathrm{d}x + \mathrm{d}(yz) \wedge \mathrm{d}y + \mathrm{d}(x+z)\wedge \mathrm{d}z                              \\
		                 & = xy \mathrm{d}z \wedge \mathrm{d}x + xz \mathrm{d}y\wedge \mathrm{d}x + y \mathrm{d}z \wedge \mathrm{d}y + \mathrm{d}x\wedge \mathrm{d}z \\
		                 & = -xz \mathrm{d}x \wedge \mathrm{d}y + (1-xy)\mathrm{d}x \wedge \mathrm{d}z - y \mathrm{d}y\wedge \mathrm{d}z.
	\end{align*}
\end{example}
\begin{prop*}
	Para formas \(\omega_{1}, \omega_{2}, \varphi \) e \(f:\mathbb{R}^{n}\rightarrow \mathbb{R}^{m}\) diferenciável, temos
	\begin{align*}
		 & \text{a) } d(\omega_1 + \omega_2) = d\omega_1 + d\omega_2                                       \\
		 & \text{b) } d(\omega \wedge \varphi ) = d\omega \wedge \varphi  + (-1)^{k}\omega \wedge d\varphi \\
		 & \text{c) } d^{2}\omega  = 0                                                                     \\
		 & \text{d) } d(f^{*}\omega ) = f^{*}(d\omega ).
	\end{align*}
\end{prop*}
\end{document}

\documentclass[../differential_forms.tex]{subfiles}
\begin{document}
\section{Aula 04 - 29 de Agosto, 2025}
\subsection{Motivações}
\begin{itemize}
	\item Propriedades de Formas e Aplicações Multilineares.
\end{itemize}
\subsection{Propriedades de Formas e Aplicações Multilineares}
A partir deste ponto, vamos nos restringir ao estudo de aplicações r-lineares da forma \(f:E\times \dotsc \times E\rightarrow F\), com todas as variáveis pertencentes a um mesmo espaço E. As generalizações para os outros casos são imediatas.
\begin{prop*}
	Sejam E, F espaços vetoriais e S um conjunto gerador de E. Se duas aplicações r-lineares \(f, g:E\times \dotsc \times E\rightarrow F\) coincidem para todo \((v_1,\dotsc ,v_{r})\in S^{r}\), então elas coincidem em todo ponto, ou seja, \(f=g\).
\end{prop*}
\begin{proof*}
	A prova segue por indução sobre r. No caso em que \(r=1\), o resultado é automático; assim, suponha que ele é válido para algum \(r\geq 1\) e considere aplicações \(f, g\in \mathcal{L}^{r+1}(E; F)\) tais que
	\[
		f(v_1,\dotsc ,v_{r+1})=g(v_{1}, \dotsc , v_{r+1}),\quad \forall v_1,\dotsc ,v_{r+1}\in S.
	\]
	Fixemos um vetor v de E arbitrário e definamos as aplicações r-lineares
	\[
		\tilde{f}(v_1,\dotsc ,v_{r})=f(v_{1},\dotsc , v _{r}, v) \quad\&\quad \tilde{g}(v_1,\dotsc ,v_{r})=g(v_{1},\dotsc , v_{r}, v).
	\]
	Por indução, \(\tilde{f}=\tilde{g}\), donde segue que, para todos \(v_1,\dotsc , v_r, v\in E\),
	\[
		f(v_{1},\dotsc , v_{r}, v)=g(v_{1},\dotsc , v_{r}, v).
	\]
	Portanto, \(f=g\). \qedsymbol
\end{proof*}
\begin{prop*}
	Sejam \(E=(e_1,\dotsc ,e_{m})\) uma base ordenada do espaço vetorial E e \(E^{*}=(e_{1}^{*}, \dotsc , e_{m}^{*})\) sua base dual. Para cada sequência \(s=(i_1,\dotsc ,i_{r})\) de índices em \(I_{m}\), consideremos o produto tensorial
	\[
		e_{(s)}^{*}=e_{i_1}^{*}\otimes e_{i_2}^{*}\otimes \dotsc \otimes e_{i_r}^{*}.
	\]
	Então:
	\begin{itemize}
		\item[i)] As formams r-lineares \(e_{(s)}^{*}\) constituem uma base do espaço \(L^{r}(E; \mathbb{R})\); e
		\item[ii)] As coordenadas de uma forma r-linear \(f:E^{r}\rightarrow \mathbb{R}\) relativamente a esta base são dadas por
		      \[
			      \zeta_{(s)}^{*}= f(e_{i_1}, \dotsc , e_{i_r}), \quad s=(i_1,\dotsc ,i_{r}).
		      \]
	\end{itemize}
\end{prop*}
\begin{proof*}
	Pela definição do produto tensorial de formas lineares, para cada sequência s temos:
	\[
		e_{(s)}^{*}(e_{i_1},\dotsc , e_{i_{r}})=1 \;\&\; e_{(s)}^{*}(e_{j_1},\dotsc e_{j_r})=0
	\]
	sempre que \((i_1,\dotsc ,i_{r})\neq (j_1,\dotsc j _{r})\). Assim, dado f uma aplicação \(L^{r}(E; \mathbb{R})\), definimos
	\[
		g = \sum\limits_{s}^{}\zeta_{(s)}^{*}e_{(s)}^{*},
	\]
	sendo esta a soma estendida a todas as sequências s compostas por r índices. Por construção,
	\[
		g(e_{i_1},\dotsc , e_{i_r}) = \zeta_{(s)}^{*}=f(e_{i_1},\dotsc , e_{i_r}),
	\]
	donde basta aplicarmos a proposição anterior e obter \(f=g\), mostrando que toda f é combinação linear das \(e_{(s)}^{*}.\) Finalmente, como existem exatamente \(m^{r}\) tais formas e \(\mathrm{dim}(\mathcal{L}^{r}(E; \mathbb{R}))=m^{r}\), concluímos portanto que \((e_{(s)}^{*})\) é uma base de \(\mathcal{L}^{r}(E; \mathbb{R})\). \qedsymbol
\end{proof*}
\begin{prop*}
	Sejam E e F espaços vetoriais sobre um corpo e suponha que E possui uma base ordenada \(E=(e_1,\dotsc , e_{m})\). Consideremos uma correspondência associando, a cada sequência \(s=(i_1,\dotsc ,i_r)\) de índices em \(I_{m}\), um vetor arbitrário \(w^{(s)}\) de F. Então, existe uma única aplicação r-linear
	\[
		f:E\times \dotsc \times E\rightarrow F
	\]
	satisfazendo, para toda sequência s como acima,
	\[
		f(e_{i_1}, \dotsc , e_{i_r})=w^{(s)}.
	\]
\end{prop*}
\begin{proof*}
	A unicidade decorre exatamente da proposição anterior, tendo em vista que uma aplicação r-linear é determinada pelos seus vetores da base. Agora, com relação à existência, definimos f a partir de
	\[
		f(v_1,\dotsc ,v_r)=\sum\limits_{s}^{}e^{(s)}(v_1,\dotsc , v_r)w^{(s)},
	\]
	onde esta soma percorre todas as sequências de r índices em \(I_{m}\), e \(e^{(s)}\) representa as formas coordenadas associadas à base ordenada de E. \qedsymbol
\end{proof*}
\begin{prop*}
	Consideremos novamente os espaços vetoriais E e F, dotados de bases ordenadas \(E=(e_1,\dotsc , e_r)\) e \(F=(w_1,\dotsc ,w_n)\). Para cada par \(((s), k)\), onde \((s)\) é uma sequência de r índices em \(I_{m}\) e \(1\leq k\leq n\), definimos a aplicação r-linear única
	\[
		f_{w}^{(s)}:E\times \dotsc \times E\rightarrow F
	\]
	tal que
	\[
		f_{k}^{(s)}(e_{i_1},\dotsc ,e_{i_r}) = \left\{\begin{array}{ll}
			w_k, & \quad (s)=(i_1,\dotsc ,i_r), \\
			0,   & \text{ caso contrário}
		\end{array}\right.
	\]
	Desta forma, as aplicações \(f_{k}^{(s)}\) formam uma base do espaço vetorial \(L^{r}(E; F)\).

	Em particular, toda aplicação r-linear arbitrária \(f:E\times \dotsc \times E\rightarrow F\) pode ser expressa em termos desta base, com coordenadas \(\zeta_{k}^{(s)}\) dadas por
	\[
		f(e_{i_1},\dotsc e_{i_r})=\sum\limits_{k}^{}\zeta_{k}^{(s)}w_{k}.
	\]
\end{prop*}
\begin{proof*}
	Tendo em vista que \(\mathrm{dim}(\mathcal{L}^{r}(E; F))=m^{r}n\) e que existem exatamente \(m^{r}n\) pares \(((s), k)\), basta mostrar que as aplicações \(f_{k}^{(s)}\) geram \(\mathcal{L}^{r}(E; F)\), e, para tanto, consideraremos uma aplicação \(f\in \mathcal{L}^{r}(E; F)\) qualquer. Para cada sequência \(s=(i_{1}, \dotsc , i_{r})\), podemos decompor
	\[
		f(e_{i_1},\dotsc , e_{i_r})=\sum\limits_{k}^{}\zeta_{k}^{(s)}w_{k}.
	\]
	Definimos, então,
	\[
		g = \sum\limits_{(s), k}^{}\zeta_{k}^{(s)}f_{k}^{(s)},
	\]
	onde esta soma é tomada sobre todos os índices k e todas as sequências \((s)\). Com isso, para toda sequência \(s=(i_1,\dotsc ,i_{r})\),
	\[
		g(e_{i_1},\dotsc , e_{i_k})=f(e_{i_1},\dotsc , e_{i_r}),
	\]
	e, pela proposição provada, \(f=g\). Portanto, f é a combinação linear das aplicações \(f_{k}^{(s)}.\) \qedsymbol
\end{proof*}
\end{document}

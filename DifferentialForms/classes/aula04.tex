\documentclass[../differential_forms.tex]{subfiles}
\begin{document}
\section{Aula 04 - 29 de Agosto, 2025}
\subsection{Motivações}
\begin{itemize}
	\item Propriedades de Formas e Aplicações Multilineares;
	\item Aplicações Alternadas;
	\item Álgebra Exterior.
\end{itemize}
\subsection{Propriedades de Formas e Aplicações Multilineares}
A partir deste ponto, vamos nos restringir ao estudo de aplicações r-lineares da forma \(f:E\times \dotsc \times E\rightarrow F\), com todas as variáveis pertencentes a um mesmo espaço E. As generalizações para os outros casos são imediatas.
\begin{prop*}
	Sejam E, F espaços vetoriais e S um conjunto gerador de E. Se duas aplicações r-lineares \(f, g:E\times \dotsc \times E\rightarrow F\) coincidem para todo \((v_1,\dotsc ,v_{r})\in S^{r}\), então elas coincidem em todo ponto, ou seja, \(f=g\).
\end{prop*}
\begin{proof*}
	A prova segue por indução sobre r. No caso em que \(r=1\), o resultado é automático; assim, suponha que ele é válido para algum \(r\geq 1\) e considere aplicações \(f, g\in \mathcal{L}^{r+1}(E; F)\) tais que
	\[
		f(v_1,\dotsc ,v_{r+1})=g(v_{1}, \dotsc , v_{r+1}),\quad \forall v_1,\dotsc ,v_{r+1}\in S.
	\]
	Fixemos um vetor v de E arbitrário e definamos as aplicações r-lineares
	\[
		\tilde{f}(v_1,\dotsc ,v_{r})=f(v_{1},\dotsc , v _{r}, v) \quad\&\quad \tilde{g}(v_1,\dotsc ,v_{r})=g(v_{1},\dotsc , v_{r}, v).
	\]
	Por indução, \(\tilde{f}=\tilde{g}\), donde segue que, para todos \(v_1,\dotsc , v_r, v\in E\),
	\[
		f(v_{1},\dotsc , v_{r}, v)=g(v_{1},\dotsc , v_{r}, v).
	\]
	Portanto, \(f=g\). \qedsymbol
\end{proof*}
\begin{prop*}
	Sejam \(E=(e_1,\dotsc ,e_{m})\) uma base ordenada do espaço vetorial E e \(E^{*}=(e_{1}^{*}, \dotsc , e_{m}^{*})\) sua base dual. Para cada sequência \(s=(i_1,\dotsc ,i_{r})\) de índices em \(I_{m}\), consideremos o produto tensorial
	\[
		e_{(s)}^{*}=e_{i_1}^{*}\otimes e_{i_2}^{*}\otimes \dotsc \otimes e_{i_r}^{*}.
	\]
	Então:
	\begin{itemize}
		\item[i)] As formams r-lineares \(e_{(s)}^{*}\) constituem uma base do espaço \(L^{r}(E; \mathbb{R})\); e
		\item[ii)] As coordenadas de uma forma r-linear \(f:E^{r}\rightarrow \mathbb{R}\) relativamente a esta base são dadas por
		      \[
			      \zeta_{(s)}^{*}= f(e_{i_1}, \dotsc , e_{i_r}), \quad s=(i_1,\dotsc ,i_{r}).
		      \]
	\end{itemize}
\end{prop*}
\begin{proof*}
	Pela definição do produto tensorial de formas lineares, para cada sequência s temos:
	\[
		e_{(s)}^{*}(e_{i_1},\dotsc , e_{i_{r}})=1 \;\&\; e_{(s)}^{*}(e_{j_1},\dotsc e_{j_r})=0
	\]
	sempre que \((i_1,\dotsc ,i_{r})\neq (j_1,\dotsc j _{r})\). Assim, dado f uma aplicação \(L^{r}(E; \mathbb{R})\), definimos
	\[
		g = \sum\limits_{s}^{}\zeta_{(s)}^{*}e_{(s)}^{*},
	\]
	sendo esta a soma estendida a todas as sequências s compostas por r índices. Por construção,
	\[
		g(e_{i_1},\dotsc , e_{i_r}) = \zeta_{(s)}^{*}=f(e_{i_1},\dotsc , e_{i_r}),
	\]
	donde basta aplicarmos a proposição anterior e obter \(f=g\), mostrando que toda f é combinação linear das \(e_{(s)}^{*}.\) Finalmente, como existem exatamente \(m^{r}\) tais formas e \(\mathrm{dim}(\mathcal{L}^{r}(E; \mathbb{R}))=m^{r}\), concluímos portanto que \((e_{(s)}^{*})\) é uma base de \(\mathcal{L}^{r}(E; \mathbb{R})\). \qedsymbol
\end{proof*}
\begin{prop*}
	Sejam E e F espaços vetoriais sobre um corpo e suponha que E possui uma base ordenada \(E=(e_1,\dotsc , e_{m})\). Consideremos uma correspondência associando, a cada sequência \(s=(i_1,\dotsc ,i_r)\) de índices em \(I_{m}\), um vetor arbitrário \(w^{(s)}\) de F. Então, existe uma única aplicação r-linear
	\[
		f:E\times \dotsc \times E\rightarrow F
	\]
	satisfazendo, para toda sequência s como acima,
	\[
		f(e_{i_1}, \dotsc , e_{i_r})=w^{(s)}.
	\]
\end{prop*}
\begin{proof*}
	A unicidade decorre exatamente da proposição anterior, tendo em vista que uma aplicação r-linear é determinada pelos seus vetores da base. Agora, com relação à existência, definimos f a partir de
	\[
		f(v_1,\dotsc ,v_r)=\sum\limits_{s}^{}e^{(s)}(v_1,\dotsc , v_r)w^{(s)},
	\]
	onde esta soma percorre todas as sequências de r índices em \(I_{m}\), e \(e^{(s)}\) representa as formas coordenadas associadas à base ordenada de E. \qedsymbol
\end{proof*}
\begin{prop*}
	Consideremos novamente os espaços vetoriais E e F, dotados de bases ordenadas \(E=(e_1,\dotsc , e_r)\) e \(F=(w_1,\dotsc ,w_n)\). Para cada par \(((s), k)\), onde \((s)\) é uma sequência de r índices em \(I_{m}\) e \(1\leq k\leq n\), definimos a aplicação r-linear única
	\[
		f_{w}^{(s)}:E\times \dotsc \times E\rightarrow F
	\]
	tal que
	\[
		f_{k}^{(s)}(e_{i_1},\dotsc ,e_{i_r}) = \left\{\begin{array}{ll}
			w_k, & \quad (s)=(i_1,\dotsc ,i_r), \\
			0,   & \text{ caso contrário}
		\end{array}\right.
	\]
	Desta forma, as aplicações \(f_{k}^{(s)}\) formam uma base do espaço vetorial \(L^{r}(E; F)\).

	Em particular, toda aplicação r-linear arbitrária \(f:E\times \dotsc \times E\rightarrow F\) pode ser expressa em termos desta base, com coordenadas \(\zeta_{k}^{(s)}\) dadas por
	\[
		f(e_{i_1},\dotsc e_{i_r})=\sum\limits_{k}^{}\zeta_{k}^{(s)}w_{k}.
	\]
\end{prop*}
\begin{proof*}
	Tendo em vista que \(\mathrm{dim}(\mathcal{L}^{r}(E; F))=m^{r}n\) e que existem exatamente \(m^{r}n\) pares \(((s), k)\), basta mostrar que as aplicações \(f_{k}^{(s)}\) geram \(\mathcal{L}^{r}(E; F)\), e, para tanto, consideraremos uma aplicação \(f\in \mathcal{L}^{r}(E; F)\) qualquer. Para cada sequência \(s=(i_{1}, \dotsc , i_{r})\), podemos decompor
	\[
		f(e_{i_1},\dotsc , e_{i_r})=\sum\limits_{k}^{}\zeta_{k}^{(s)}w_{k}.
	\]
	Definimos, então,
	\[
		g = \sum\limits_{(s), k}^{}\zeta_{k}^{(s)}f_{k}^{(s)},
	\]
	onde esta soma é tomada sobre todos os índices k e todas as sequências \((s)\). Com isso, para toda sequência \(s=(i_1,\dotsc ,i_{r})\),
	\[
		g(e_{i_1},\dotsc , e_{i_k})=f(e_{i_1},\dotsc , e_{i_r}),
	\]
	e, pela proposição provada, \(f=g\). Portanto, f é a combinação linear das aplicações \(f_{k}^{(s)}.\) \qedsymbol
\end{proof*}
\subsection{Álgebra Exterior}
O estudo de aplicações multilineares constitui um pilar muito importante no estudo de estruturas algébricas e geométricas nos espaços vetoriais, e um dos conceitos que surgem naturalmente ao estudá-las é o de \textit{alternância}, desempenhando papel central na construção da álgebra exterior e na definição de formas diferenciais. Essencialmente, a ideia é entender o que ocorre com uma aplicação r-linear quando os índices dos vetores nos quais ela está sendo aplicada são alternados de posição. Formalmente,
\begin{def*}
	Sejam E e F espaços vetoriais. Uma aplicação r-linear \(f:E\times \dotsc \times E\rightarrow F\) é dita \textbf{alternada} se ela é anulada sempre que dois de seus argumentos são iguais, \textit{i.e.}, para quaisquer \(v_1,\dotsc v_r, v\) em E,
	\[
		f(v_1,\dotsc , v_{i-1}, \underbrace{v}_{\mathclap{\text{i-ésima}}}, v_{i+1},\dotsc , v_{j-1}, \underbrace{v}_{\mathclap{\text{j-ésima}}}, v_{j+1},\dotsc , v_{r})=0, \quad i\neq j. \; \square
	\]
\end{def*}
Uma caracterização mais parecida à dada de intuição pode ser vista pela
\begin{prop*}
	Uma aplicação r-linear f é alternada se, e somente se, para quaisquer vetores \(v_1,\dotsc , v_{r}\) em E e quaisquer índices \(i<j\), tem-se
	\[
		f(v_1,\dotsc ,v_{i},\dotsc ,v_{j},\dotsc ,v_r)=-f(v_1,\dotsc ,v_{j},\dotsc ,v_{i},\dotsc ,v_r).
	\]
\end{prop*}
\begin{proof*}
	Suponha, primeiramente, que f é alternada e considere
	\[
		\varphi (u, v)=f(v_1,\dotsc , u,\dotsc ,v,\dotsc v_r).
	\]
	Pela alternância, segue que
	\[
		0=\varphi (u+v, u+v) = \varphi(u, u)+\varphi (u, v)+\varphi (v, u)+\varphi (v, v)
	\]
	e, como \(\varphi (u, u)=\varphi (v, v)=0\), temos
	\[
		\varphi (u,v) = -\varphi (v, u),
	\]
	ou seja, f é anti simétrica e a igualdade acima pode ser explicitamente escrita como
	\[
		f(v_1,\dotsc , u,\dotsc ,v,\dotsc v_r) = -f(v_1,\dotsc , v,\dotsc ,u,\dotsc v_r).
	\]

	Por outro lado, se f é anti simétrica, então
	\[
		\varphi (v, v)=-\varphi (v, v) \Rightarrow 2\varphi (v, v)=0.
	\]
	Logo,
	\[
		\varphi (v, v)=f(v_1,\dotsc , v,\dotsc ,v,\dotsc v_r) = 0
	\]
	e, portanto, f é alternada. \qedsymbol
\end{proof*}
\begin{def*}
	Dizemos que
	\[
		\mathcal{A}^{r}(E; F)=\{f\in \mathcal{L}^{r}(E; F):\; f \text{ é alternada}\}
	\]
	é o \textbf{espaço das aplicações r-lineares alternadas}, também chamado de \textbf{álgebra exterior.} \(\square\)
\end{def*}
\begin{example}
	Para \(r=1\), toda aplicação \(f:E\rightarrow F\) é alternada. Assim,
	\[
		\mathcal{A}^{1}(E; F)=\mathcal{L}(E; F).
	\]
\end{example}
\begin{example}
	Se \(f:\mathbb{R}\times \dotsc \times \mathbb{R}\rightarrow F\) é r-linear, então
	\[
		f(t_1,\dotsc , t_r)=t_1 \dotsc t_r \cdot v,\quad v=f(1,\dotsc ,1).
	\]
	Para r maior que 1, a f só é alternada se \(v=0\), que equivale a dizer que f é a aplicação nula. Portanto,
	\[
		\mathcal{A}^{r}(\mathbb{R}; F)=0,\quad r>1.
	\]
\end{example}
\begin{example}
	A aplicação bilinear
	\[
		f(u, v)=u_1v_2-u_2v_1,\quad u=(u_1,u_2),\; v=(v_1, v_2),
	\]
	é alternada em \(\mathbb{R}^{2}.\) Com efeito, note que
	\[
		f(u, v)=u_1v_2-u_2v_1 = -(u_2v_1 - u_1v_2) = - f(v, u),
	\]
	e já vimos que isto equivale à alternância.
\end{example}
\begin{example}
	O produto vetorial em \(\mathbb{R}^{3}\), dado por
	\[
		f(u, v)=u \times v,\quad u, v\in \mathbb{R}^{3},
	\]
	é um exemplo de aplicação bilinear alternada, afinal
	\[
		f(u, u) = u \times u = 0, \quad u\in \mathbb{R}^{3}
	\]
	que e é exatamente a definição de aplicação alternada.
\end{example}
\begin{prop*}
	Seja \(f:E^{r}\rightarrow F\) uma aplicação r-linear alternada. Se \(v_1,\dotsc ,v_r\) forem vetores linearmente dependentes de E, então
	\[
		f(v_1,\dotsc , v_r)=0.
	\]
\end{prop*}
\begin{proof*}
	Suponha que
	\[
		v_{i} = \sum\limits_{j<i}^{}\alpha_{j}v_{j}.
	\]
	Pela linearidade de f, junto à sua alternância,
	\[
		f(v_1,\dotsc ,v_r)=\sum\limits_{j<i}^{}\alpha_{j}f(v_1,\dotsc , v_{j},\dotsc , v_{j}, \dotsc , v_r)=0. \text{ \qedsymbol}
	\]
\end{proof*}
\begin{crl*}
	Se \(r>\mathrm{dim}(E)\), então
	\[
		\mathcal{A}^{r}(E; F)=\{0\}.
	\]
\end{crl*}

\end{document}

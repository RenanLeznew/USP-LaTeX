\documentclass[../differential_forms.tex]{subfiles}
\begin{document}
\section{Aula 06 - 10 de Setembro, 2025}
\subsection{Motivações}
\begin{itemize}
	\item Endomorfismo Induzido por Permutação;
	\item Operador de Anti-Simétrização.
\end{itemize}
\subsection{Endomorfismo Induzido por Permutação}
Antes de entra no tópico de determinantes, mostraremos um exemplo de uso do produto tensorial:
\begin{example}[Construindo uma Base para \(\mathcal{L}^{2}(\mathbb{R}^{3}; \mathbb{R}^{2})\)]
	O objetivo deste exemple é mostrar como podemos utilizar o produto tensorial para construir uma base para o espaço \(\mathcal{L}^{2}(\mathbb{R}^{3}; \mathbb{R}^{2})\); para isto, fixemos a notação \(\{v_1, \dotsc , v_r\}\) quando nos referirmos a uma base e \((v_1, \dotsc , v_r)\) para referir a uma base ordenada.

	Consideremos a base ordenada canônica do \(\mathbb{R}^{3}\) dada por \((e_1, e_2, e_3)\), onde
	\[
		e_1 = (1, 0, 0),\quad e_2 = (0, 1, 0) \quad\&\quad e_3 = (0, 0, 1)
	\]

	e a base ordenada de \(\mathbb{R}^{2}\), dada por \((w_1, w_2)\) com
	\[
		{\color{Chartreuse3}w_1 = (1, 0)} \quad\&\quad {\color{DodgerBlue2}w_2 = (0, 1)}.
	\]

	Com isso, temos a base de \((\mathbb{R}^{3})^{*}\) associada a \((e_1^{*}, e_2^{*}, e_3^{*})\); se \(f\in \mathcal{L}^{2}(\mathbb{R}^{3}; \mathbb{R}^{2})\), a f tem como resultado \(f(u, v)\in \mathbb{R}^{2}\) dados \(u, v\in \mathbb{R}^{3}\).

	Como próximo passo, consideremos todas as sequências de dois elementos em \(I_3\):
	\begin{align*}
		(S):\quad & (1, 1),\; (1, 2),\; (1, 3),\; (2, 1),\; (2, 2),\; (2, 3) \\
		          & (3, 1),\; (3, 2),\; (3, 3),
	\end{align*}
	e, por termos dois elementos na saída (\(w_1\) e \(w_2\)), teremos um total de 9 combinações de índices vezes 2 elementos, totalizando 18 e bastando trocar os números por letras para generalizar este mesmo raciocínio aos casos mais abstratos.

	Vamos, então, explicitar esse caso inteiro para melhorar o entendimento do caso geral. Utilizemos a notação
	\[
		f_{{\color{Firebrick3}(1, 1)}}^{{\color{Chartreuse3}1}}(u, v) = e_{{\color{Firebrick3}1}}^{*}(u)e_{{\color{Firebrick3}1}}^{*}(v){\color{Chartreuse3}(1, 0)}
	\]
	para indicar que estamos escolhendo o {\color{Chartreuse3}primeiro elemento da lista correspondente à base de \(\mathbb{R}^{3}\)}, ou seja, \({\color{Chartreuse3}e_1}\) e o {\color{Firebrick3}elemento 1 da lista de índices}, correspondente a \({\color{Firebrick3}(1, 1)}\) -- o domínio me dá o número de funcionais que preciso, e o k é o elemento da base que estou associando o elemento da base, correspondendo, na notação geral, a
	\[
		f_{{\color{Firebrick3}(s)}}^{{\color{Chartreuse3}k}} = e_{{\color{Firebrick3}(s)}}^{*}(u, v){\color{Chartreuse3}w_k}.
	\]

	Logo, repetindo este processo,
	\begin{align*}
		 & f_{{\color{Firebrick3}(1, 1)}}^{{\color{Chartreuse3}1}}(u, v) = e_{{\color{Firebrick3}1}}^{*}(u)e_{{\color{Firebrick3}1}}^{*}(v){\color{Chartreuse3}(1, 0)},\; f_{{\color{Firebrick3}(1, 1)}}^{{\color{DodgerBlue2}2}}(u, v) = e_{{\color{Firebrick3}1}}^{*}(u)e_{{\color{Firebrick3}1}}^{*}(v){\color{DodgerBlue2}(0, 1)}, \\
		 & f_{{\color{Firebrick3}(1, 2)}}^{{\color{Chartreuse3}1}}(u, v) = e_{{\color{Firebrick3}1}}^{*}(u)e_{{\color{Firebrick3}2}}^{*}(v){\color{Chartreuse3}(1, 0)},\; f_{{\color{Firebrick3}(1, 2)}}^{{\color{DodgerBlue2}2}}(u, v) = e_{{\color{Firebrick3}1}}^{*}(u)e_{{\color{Firebrick3}2}}^{*}(v){\color{DodgerBlue2}(0, 1)}, \\
		 & f_{{\color{Firebrick3}(1, 3)}}^{{\color{Chartreuse3}1}}(u, v) = e_{{\color{Firebrick3}1}}^{*}(u)e_{{\color{Firebrick3}3}}^{*}(v){\color{Chartreuse3}(1, 0)},\; f_{{\color{Firebrick3}(1, 3)}}^{{\color{DodgerBlue2}2}}(u, v) = e_{{\color{Firebrick3}1}}^{*}(u)e_{{\color{Firebrick3}3}}^{*}(v){\color{DodgerBlue2}(0, 1)}, \\
		 & f_{{\color{Firebrick3}(2, 1)}}^{{\color{DodgerBlue2}1}}(u, v) = e_{{\color{Firebrick3}2}}^{*}(u)e_{{\color{Firebrick3}1}}^{*}(v){\color{Chartreuse3}(1, 0)},\; f_{{\color{Firebrick3}(2, 1)}}^{{\color{DodgerBlue2}2}}(u, v) = e_{{\color{Firebrick3}2}}^{*}(u)e_{{\color{Firebrick3}1}}^{*}(v){\color{DodgerBlue2}(0, 1)}, \\
		 & f_{{\color{Firebrick3}(2, 2)}}^{{\color{DodgerBlue2}1}}(u, v) = e_{{\color{Firebrick3}2}}^{*}(u)e_{{\color{Firebrick3}2}}^{*}(v){\color{Chartreuse3}(1, 0)},\; f_{{\color{Firebrick3}(2, 2)}}^{{\color{DodgerBlue2}2}}(u, v) = e_{{\color{Firebrick3}2}}^{*}(u)e_{{\color{Firebrick3}2}}^{*}(v){\color{DodgerBlue2}(0, 1)}, \\
		 & f_{{\color{Firebrick3}(2, 3)}}^{{\color{DodgerBlue2}1}}(u, v) = e_{{\color{Firebrick3}2}}^{*}(u)e_{{\color{Firebrick3}3}}^{*}(v){\color{Chartreuse3}(1, 0)},\; f_{{\color{Firebrick3}(2, 3)}}^{{\color{DodgerBlue2}2}}(u, v) = e_{{\color{Firebrick3}2}}^{*}(u)e_{{\color{Firebrick3}3}}^{*}(v){\color{DodgerBlue2}(0, 1)}, \\
		 & f_{{\color{Firebrick3}(3, 1)}}^{{\color{DodgerBlue2}1}}(u, v) = e_{{\color{Firebrick3}3}}^{*}(u)e_{{\color{Firebrick3}1}}^{*}(v){\color{Chartreuse3}(1, 0)},\; f_{{\color{Firebrick3}(3, 1)}}^{{\color{DodgerBlue2}2}}(u, v) = e_{{\color{Firebrick3}3}}^{*}(u)e_{{\color{Firebrick3}1}}^{*}(v){\color{DodgerBlue2}(0, 1)}, \\
		 & f_{{\color{Firebrick3}(3, 2)}}^{{\color{DodgerBlue2}1}}(u, v) = e_{{\color{Firebrick3}3}}^{*}(u)e_{{\color{Firebrick3}2}}^{*}(v){\color{Chartreuse3}(1, 0)},\; f_{{\color{Firebrick3}(3, 2)}}^{{\color{DodgerBlue2}2}}(u, v) = e_{{\color{Firebrick3}3}}^{*}(u)e_{{\color{Firebrick3}2}}^{*}(v){\color{DodgerBlue2}(0, 1)}, \\
		 & f_{{\color{Firebrick3}(3, 3)}}^{{\color{DodgerBlue2}1}}(u, v) = e_{{\color{Firebrick3}3}}^{*}(u)e_{{\color{Firebrick3}3}}^{*}(v){\color{Chartreuse3}(1, 0)},\; f_{{\color{Firebrick3}(3, 3)}}^{{\color{DodgerBlue2}2}}(u, v) = e_{{\color{Firebrick3}3}}^{*}(u)e_{{\color{Firebrick3}3}}^{*}(v){\color{DodgerBlue2}(0, 1)}.
	\end{align*}
	Mostraremos, então, que este conjunto é linearmente independente.

	Considere a combinação linear nula
	\[
		\sum\limits_{j,\; (s)}^{}\alpha_{(s)}^{j}f_{(s)}^{j} = 0, \quad j\in \{1, 2\},\; (i_1, i_2)\in I_3.
	\]
	Para concluir que os escalares devem ser nulos, lembre-se que estamos construindo usando \textit{produto tensorial}, e agora que ele será útil: ao escolhermos os elementos que, quando respeitam a ordem dos índices, resultam meramente na unidade multiplicada pelo respectivo \(w_{i}\) OU zero, ou seja, sobram apenas os coeficientes
	com coeficientes iguais, que é uma combinação linear de \(w_1\) e \(w_2\):
	\[
		\sum\limits_{j,\; (s)}^{}\alpha_{(s)}^{j}f_{(s)}^{j}(e_{i_1}, e_{i_2}) = \alpha_{(s)}^{1}w_1 + \alpha_{(s)}^{2}w_2 = 0,
	\]
	que em si já são linearmente independentes. Logo, \(\alpha_{(s)}^{1} = \alpha_{(s)}^{2} = 0\) e \(\{f_{(s)}^{j}\}_{(s),\; j}\) forma um conjunto de vetores linearmente independentes com um número de elementos igual à dimensão -- é uma base para \(\mathcal{L}^{2}(\mathbb{R}^{3}; \mathbb{R}^{2})\).
\end{example}

A seguir, tentaremos associar estes assuntos ao tópico das permutações, afinal deu para perceber até pelo exemplo acima que a ordem dos índices pode afetar bastante as questões que estamos estudando: sejam E e F espaços vetoriais; uma permutação \(\sigma \in S_r\) induz um endomorfismo linear no espaço vetorial \(\mathcal{L}^{r}(E; F)\) ao trocar a ordem dos índices -- mais explicitamente, para \(\sigma \in S_r\), definimos
\begin{align*}
	\sigma : & \mathcal{L}^{r}(E; F)\rightarrow\mathcal{L}^{r}(E; F)                                    \\
	         & f\longmapsto \sigma f(v_1, \dotsc , v_r) = f(v_{\sigma (1)}, \dotsc , f_{v_\sigma (r)}).
\end{align*}
Formalmente, dizemos
\begin{def*}
	Uma permutação \(\sigma \in S_r\) \textbf{atua em f \(L^{r}(E; F)\)} definindo
	\[
		(\sigma f)(v_1,\dotsc ,v_r) = f(v_{\sigma (1)}, \dotsc , v_{\sigma (r)}).
	\]
	Diremos que f é \textbf{alternada} se
	\[
		f(v_{\sigma (1)},\dotsc , v_{\sigma (r)})= \mathrm{sgn}(\sigma )f(v_1,\dotsc ,v_r),
	\]
	onde \(\mathrm{sgn}(\sigma )\) é o sinal da permutação. \(\square\)
\end{def*}
Com isso, ganhamos a definição de uma transformação r-linear alternada com base nas permutações e seus sinais!
\begin{exr}
	Prove as seguintes propriedades da atuação de permutações:
	\begin{align*}
		 & \circ \;\sigma (f+g) = \sigma f + \sigma g;                               \\
		 & \circ \; \sigma ( k f) =  k (\sigma f), \quad k\in \mathbb{R};            \\
		 & \circ \; (\sigma \circ \rho )f = \sigma (\rho f), \quad \sigma ,r\in S_r.
	\end{align*}
	Em particular, mostre que \(\sigma \) é um isomorfismo.
	%\[
	%  f = (\sigma^{-1}\circ \sigma )(f),
	%\]
\end{exr}

\begin{def*}
	O \textbf{Operador de Anti-simetrização} é o operador \(A:L^{r}(E; F)\rightarrow L^{r}(E; F)\) definido por
	\[
		A(f)= \sum\limits_{\sigma \in S_r}^{} \mathrm{sgn}(\sigma )(\sigma f) = \sum\limits_{\sigma \in S_r}^{}\varepsilon_{\sigma }(\sigma f). \; \square
	\]
\end{def*}
\begin{prop*}
	Para qualquer \(f\in L^{r}(E; F)\), valem:
	\begin{itemize}
		\item[i)] O operador de anti-simetrização é uma transformação r-linear alternada entre E e F, isto é, \(Af\in A^{r}(E; F)\);
		\item[ii)] f é alternada se, e somente se, \(Af=r!f\); e
		\item[iii)] Se existir \(\rho \in S_r\) ímpar tal que \(\rho f = f\), então \(Af=0\).
	\end{itemize}
\end{prop*}
\begin{crl*}
	O operador A projeta \(L^{r}(E; F)\) sobre \(A^{r}(E; F)\).
\end{crl*}
\begin{prop*}
	Dada uma base ordenada \(E^{*}=(e_{1}^{*}, \dotsc , e_{m}^{*})\) de \(E^{*}\), as formas
	\[
		e_{(J)}^{*}=\varphi (e_{j_1}^{*}, \dotsc , e_{j_r}^{*}), \quad J = \{j_1<\dotsc <j_r\}
	\]
	formam uma base de \(A^{r}(E; \mathbb{R})\) e, em particular,
	\[
		\mathrm{dim}(A^{r}(E; \mathbb{R}))= \binom{m}{r}.
	\]
\end{prop*}
\begin{crl*}
	Se \(\mathrm{dim}(E) = m\), então
	\[
		\mathrm{dim}(A^{m}(E; \mathbb{R}))=1 \quad\&\quad \mathrm{dim}(A^{r}(E; \mathbb{R}^{n}))= \binom{m}{r}n.
	\]
\end{crl*}
\begin{prop*}
	Sejam \(E=(e_1,\dotsc ,e_{m})\) e \(F=(w_1,\dotsc ,w_{n})\) bases ordenadas; para cada \(J=\{j_1<\dotsc <j_r\}\subseteq \{1,\dotsc ,m\}\) e cada k, defina
	\[
		f_{k}^{J}(e_{j_1}, \dotsc , e_{j_r}) = w_{k}, \quad f_{k}^{J}(e_{p_1}, \dotsc , e_{p_r})=0 \text{ se }\{p_1,\dotsc , p_r\}\neq J.
	\]
	Então, as aplicações \(f_{k}^{J}\) constituem uma base de \(A^{r}(E; F)\) e, consequentemente,
	\[
		\mathrm{dim}(A^{r}(E; F))=\binom{m}{r}n.
	\]
\end{prop*}
\end{document}

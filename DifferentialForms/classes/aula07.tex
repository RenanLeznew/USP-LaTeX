\documentclass[../differential_forms.tex]{subfiles}
\begin{document}
\section{Aula 07- 12 de Setembro, 2025}
\subsection{Motivações}
\begin{itemize}
	\item A Construção Intrínseca dos Determinantes.
\end{itemize}
\subsection{Construção do Determinante}
Começaremos o tópico dos determinantes apresentando uma definição intrínseca dos determinantes pela perspectiva de um escalar único para cada transformação \(T:E\rightarrow E\) que satisfaz uma propriedade específica. Antes dela, porém, vale mencionar que denotaremos o \textbf{espaço das formas m-lineares alternadas}, ou seja, o dual de \(A^{m}(E; F)\), por \(\Lambda^{m}(E; \mathbb{R})\).
\begin{def*}
	Seja \(T:E\rightarrow E\) um endomorfismo. O \textbf{determinante de T} é o \textit{único} escalar \(\det{T}\in \mathbb{R}\) tal que, para toda forma m-linear alternada \(f\in \Lambda^{m}(E; \mathbb{R})\) e quaisquer \(v_1,\dotsc , v_{m}\in E\),
	\[
		f(Tv_1,\dotsc , Tv_{m}) = \det{(T)} f(v_1,\dotsc ,v_{m}). \quad \square
	\]
\end{def*}
\begin{prop*}
	Uma aplicação linear \(T:E\rightarrow F\) induz, para cada r positivo, uma aplicação linear
	\[
		T^{\#}: \Lambda^{r}(E; \mathbb{R})\rightarrow \Lambda^{r}(E; \mathbb{R})
	\]
	definida como
	\[
		(T^{\#}f)(v_1,\dotsc ,v_r)= f(Tv_1,\dotsc ,TV_r),\quad \forall f\in \Lambda ^{r}(F; \mathbb{R})\;\&\; v_1,\dotsc ,v_r\in E.
	\]
\end{prop*}
\begin{proof*}
	O fato de \(T^{\#}f\) ser uma forma m-linear alternada é imediato, assim como a linearidade de \(T^{\#}.\) Além disso, se T corresponde à identidade de E, então \(T^{\#}=\mathrm{id}.\) e, se \(S:E\rightarrow F\) e \(T:F\rightarrow G\) são lineares, então
	\[
		(T \circ S)^{\#}=S^{\#}\circ T^{\#}.\text{ \qedsymbol}
	\]
\end{proof*}
\begin{crl*}
	Se \(S:E\rightarrow F\) for um isomorfismo, então \(S^{\#}\) também o é, com inversa dada por \((S^{\#})^{-1}=(S^{-1})^{\#}\).
\end{crl*}
\end{document}

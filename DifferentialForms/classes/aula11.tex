\documentclass[../differential_forms.tex]{subfiles}
\begin{document}
\section{Aula 11 - 29 de Setembro, 2025}
\subsection{Motivações}
\begin{itemize}
	\item Formas Diferenciais em \(\mathbb{R}^{n}\);
	\item Derivadas e Integrais em Termos de Formas.
\end{itemize}
\subsection{Derivadas e Integrais como Formas}
O espaço \(\mathbb{R}^{n}\) é um no qual matemáticos costumam forças um monte de estruturas diferentes, e aqui estaremos fazendo o mesmo,
mas com formas; normalmente, a abordagem é utilizando um sistema de coordenadas para representar seus elementos, tornando muito natural
denotarmos um \(p\in \mathbb{R}^{n}\) por
\[
	p = (x_1, \dotsc , x_{n}).
\]
Um exemplo de transformação linear (inclusive de uma 1-forma) que já temos é a projeção
\begin{align*}
	\pi_{i}: & \mathbb{R}^{n}\rightarrow\mathbb{R}     \\
	         & (x_1, \dotsc , x_{n})\longmapsto x_{i},
\end{align*}
e que inclusive pode ser representada com a notação \(\pi_{i} = e_{i}^{*}\).

Porém, mudaremos a notação mais uma vez -- para representar a i-ésima forma da base ortonormal \(\{e_{1}^{*}, \dotsc , e_{n}^{*}\}\) correspondendo à dual de
\(\{e_{1}, e_{2}, \dotsc , e_{n}\}\), escreveremos \(dx_1, dx_2, \dotsc , dx_{n}\), ou seja,
\[
	\mathrm{span}\{dx_1, \dotsc , dx_{n}\} = (\mathbb{R}^{n})^{*}.
\]
\begin{def*}
	Considere uma função \(f:U\subseteq \mathbb{R}^{n}\rightarrow \mathbb{R}\) de classe \(\mathcal{C}^{1}\), sendo U um aberto de
	\(\mathbb{R}^{n}\). A \textbf{diferencial de f em x} é a transformação linear \(\mathcal{L}(\mathbb{R}^{n}, \mathbb{R})\) dada por:
	\begin{align*}
		df_x: & \mathbb{R}^{n}\rightarrow\mathbb{R}                                                                                                                                                         \\
		      & h\longmapsto \frac{\partial^{}f}{\partial x_1^{}}(x)\cdot h_1 + \frac{\partial^{}f}{\partial x_2^{}}(x)\cdot h_2 + \dotsc + \frac{\partial^{}f}{\partial x_{n}^{}}(x)\cdot h_{n}.\; \square
	\end{align*}
\end{def*}
Em particular, a definição acima basicamente expande a expressão
\[
	df_p = \langle \nabla f(x), h \rangle,\quad x\in U\subseteq \mathbb{R}^{n}\;\&\; h = (h_1, \dotsc , h_{n})
\]
\begin{example}
	Em \(\mathbb{R}^{3}\), considere um exemplo da função
	\begin{align*}
		f: & \mathbb{R}^{3}\rightarrow\mathbb{R}         \\
		   & (x, y, z)\longmapsto x^{2} + y^{2} + z^{2}.
	\end{align*}
	Dado o ponto \(p = (1, 1, 1)\), a diferencial de f em p seria dada por
	\begin{align*}
		df_p: & \mathbb{R}^{3}\rightarrow\mathbb{R}                              \\
		      & h\longmapsto \langle \nabla f(p), h \rangle = 2h_1 + 2h_2 + 2h_3
	\end{align*}
\end{example}

Definiremos os campos de formas alternadas, posteriormente utilizados para obter resultados geométricos. Começamos com o
\begin{def*}
	O \textbf{espaço tangente a }\(\mathbb{R}^{n}\) no ponto \(p\in \mathbb{R}^{n}\), denotado por \(\mathbb{R}_{p}^{n}\), é o conjunto dos vetores \(q-p\), com \(q\in \mathbb{R}^{n}\). \(\square\)
\end{def*}
Para o espaço tangente ao espaço n-euclidiano, podemos já obter uma base canônica observando que \(\mathbb{R}^{n} = \mathbb{R}_{0}^{n}\), então basta considerar seus vetores transladados
\[
	(e_1)_p, \dotsc , (e_{n})_p
\]
para obter uma base para \(\mathbb{R}_{p}^{n}\).

Essa notação não costuma ser tão usada em contextos mais gerais -- normalmente, para denotar o plano tangente a \(\mathbb{R}^{n}\) em p, utiliza-se a notação \(T_{p}\mathbb{R}^{n}\), que será mais comumente
utilizada aqui também. Com estes conceitos, podemos finalmente definir o campo vetorial como
\begin{def*}
	Um \textbf{campo vetorial em }\(\mathbb{R}^{n}\) é uma aplicação \textit{suave}\footnote{De classe \(\mathcal{C}^{\infty},\; \mathcal{C}^{k},\; \mathcal{C}^{r)}\) a depender do contexto; acho que, aqui, é \(\mathcal{C}^{\infty}\)}
	\[
		v:\mathbb{R}^{n}\rightarrow \bigsqcup_{p\in \mathbb{R}^{n}}^{}T_p\mathbb{R}^{n}=T\mathbb{R}^{n},
	\]
	onde o símbolo \(\sqcup\) denota a união disjunta de conjuntos, e \(T\mathbb{R}^{n}\coloneqq \bigsqcup_{p}^{}T_{p}\mathbb{R}^{n}\) é o chamado \textbf{fibrado tangente de }\(\mathbb{R}^{n}\). \(\square\)
\end{def*}
A cada ponto \(p\in \mathbb{R}^{n}\), associa-se um vetor \(v(p)\) no plano tangente no ponto; assim, com relação à base canônica transladada \(\{(e_{i})_p\}_{i=1}^{n}\) de \(T_p\mathbb{R}^{n}\), o campo vetorial v pode ser escrito como
\[
	v(p) = \sum\limits_{i=1}^{n}a_{i}(p)(e_{i})_p.
\]
\begin{def*}
	Para um campo vetorial v escrito com respeito à base transladada, diremos que v é \textbf{diferenciável} se todas as funções \(a_{i}:\mathbb{R}^{n}\rightarrow \mathbb{R}\) são diferenciáveis; analogamente, v será suave se todas as \(a_{i}\)'s forem suaves. \(\square\)
\end{def*}
\begin{def*}
	O \textbf{fibrado cotangente de }\(\mathbb{R}^{n}\) é o espaço
	\[
		T^{*}\mathbb{R}^{n}\coloneqq \bigsqcup_{p\in \mathbb{R}^{n}}^{}T_{p}^{*}\mathbb{R}^{n},
	\]
	onde \(T_{p}^{*}\mathbb{R}^{n}\) é o espaço dual de \(T_{p}\mathbb{R}^{n}.\; \square\)
\end{def*}
\begin{def*}
	Um \textbf{campo de 1-formas em }\(\mathbb{R}^{n}\) é uma aplicação suave
	\[
		\omega :\mathbb{R}^{n}\rightarrow T_{p}^{*}\mathbb{R}^{n}
	\]
	associando, a cada ponto p de \(\mathbb{R}^{n}\), uma forma linear \(\omega (p)\in T_{p}^{*}(\mathbb{R}^{n})\). \(\square\)
\end{def*}
Com relação à base dual \(\{dx_{i}|_{p}\}_{i=1}^{n}\) de \(T_{p}^{*}\mathbb{R}^{n}\), uma 1-forma tem expressão
\[
	\omega (p)=\sum\limits_{i=1}^{n}f_{i}(p)dx_{i}|_{p},
\]
onde \(f_{i}:\mathbb{R}^{n}\rightarrow \mathbb{R}\) são diferenciáveis quando \(\omega \) o é.

Resumindo, a cada espaço tangente \(\mathbb{R}_{p}^{n}\), associa-se seu espaço dual \((\mathbb{R}_{p}^{n})^{*}\) consistindo das aplicações lineares \(\varphi :\mathbb{R}^{n}\rightarrow \mathbb{R}\). Ademais,
obtivemos uma base para \((\mathbb{R}_{p}^{n})^{*}\) dada por \(\{(dx_{i})_{p}\}_{i=1}^{n}\), onde \(x_{i}:\mathbb{R}^{n}\rightarrow \mathbb{R}\) é a função coordenada
\[
	x_{i}(x_1,\dotsc ,x_{n})=x_{i},
\]
e essa base é dual à base \(\{(e_{i})_{p}\}_{i=1}^{n}\) pois
\[
	(dx_{i})_{p}((e_{j})_{p})=\frac{\partial^{}x_{i}}{\partial x_{j}^{}}  = \left\{\begin{array}{ll}
		0, & \quad i\neq j \\
		1, & \quad i=j
	\end{array}\right..
\]

\end{document}

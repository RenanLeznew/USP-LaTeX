\documentclass[../differential_forms.tex]{subfiles}
\begin{document}
\section{Aula 11 - 29 de Setembro, 2025}
\subsection{Motivações}
\begin{itemize}
	\item Formas Diferenciais em \(\mathbb{R}^{n}\);
	\item Derivadas e Integrais em Termos de Formas.
\end{itemize}
\subsection{Derivadas e Integrais como Formas}
O espaço \(\mathbb{R}^{n}\) é um no qual matemáticos costumam forças um monte de estruturas diferentes, e aqui estaremos fazendo o mesmo,
mas com formas; normalmente, a abordagem é utilizando um sistema de coordenadas para representar seus elementos, tornando muito natural
denotarmos um \(p\in \mathbb{R}^{n}\) por
\[
	p = (x_1, \dotsc , x_{n}).
\]
Um exemplo de transformação linear (inclusive de uma 1-forma) que já temos é a projeção
\begin{align*}
	\pi_{i}: & \mathbb{R}^{n}\rightarrow\mathbb{R}     \\
	         & (x_1, \dotsc , x_{n})\longmapsto x_{i},
\end{align*}
e que inclusive pode ser representada com a notação \(\pi_{i} = e_{i}^{*}\).

Porém, mudaremos a notação mais uma vez -- para representar a i-ésima forma da base ortonormal \(\{e_{1}^{*}, \dotsc , e_{n}^{*}\}\) correspondendo à dual de
\(\{e_{1}, e_{2}, \dotsc , e_{n}\}\), escreveremos \(dx_1, dx_2, \dotsc , dx_{n}\), ou seja,
\[
	\mathrm{span}\{dx_1, \dotsc , dx_{n}\} = (\mathbb{R}^{n})^{*}.
\]
\begin{def*}
	Considere uma função \(f:U\subseteq \mathbb{R}^{n}\rightarrow \mathbb{R}\) de classe \(\mathcal{C}^{1}\), sendo U um aberto de
	\(\mathbb{R}^{n}\). A \textbf{diferencial de f em x} é a transformação linear \(\mathcal{L}(\mathbb{R}^{n}, \mathbb{R})\) dada por:
	\begin{align*}
		df_x: & \mathbb{R}^{n}\rightarrow\mathbb{R}                                                                                                                                                         \\
		      & h\longmapsto \frac{\partial^{}f}{\partial x_1^{}}(x)\cdot h_1 + \frac{\partial^{}f}{\partial x_2^{}}(x)\cdot h_2 + \dotsc + \frac{\partial^{}f}{\partial x_{n}^{}}(x)\cdot h_{n}.\; \square
	\end{align*}
\end{def*}
Em particular, a definição acima basicamente expande a expressão
\[
	df_p = \langle \nabla f(x), h \rangle,\quad x\in U\subseteq \mathbb{R}^{n}\;\&\; h = (h_1, \dotsc , h_{n})
\]
\begin{example}
	Em \(\mathbb{R}^{3}\), considere um exemplo da função
	\begin{align*}
		f: & \mathbb{R}^{3}\rightarrow\mathbb{R}         \\
		   & (x, y, z)\longmapsto x^{2} + y^{2} + z^{2}.
	\end{align*}
	Dado o ponto \(p = (1, 1, 1)\), a diferencial de f em p seria dada por
	\[
		\begin{align*}
			df_p: & \mathbb{R}^{3}\rightarrow\mathbb{R}                              \\
			      & h\longmapsto \langle \nabla f(p), h \rangle = 2h_1 + 2h_2 + 2h_3
		\end{align*}
	\]
\end{example}

\end{document}

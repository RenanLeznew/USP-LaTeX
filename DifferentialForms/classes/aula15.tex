\documentclass[../differential_forms.tex]{subfiles}
\begin{document}
\section{Aula 15 - 05 de Novembro, 2025}
\subsection{Motivações}
\begin{itemize}
	\item Variedades com bordo.
\end{itemize}
\subsection{Variedades com Bordo -- Parte 1}
Começamos a última aula introduzindo a noção do modelo local e da variedade local, então vamos relembrar ambos os conceitos e elaborar seus usos na aula de hoje.
\begin{def*}[Modelo local]
	Para $m\ge 1$, denote o produto cartesiano de m cópias de H por
	\[
		H^m \;=\; \{(x_1,\dots,x_m)\in\mathbb{R}^m: x_m\ge 0\},
	\]
	e o \textbf{bordo de \(H^{m}\)} por
	\[
		\partial H^m \;=\; \{(x_1,\dots,x_m)\in\mathbb{R}^m: x_m=0\}\cong \mathbb{R}^{m-1}.
	\]
	Ambos são munidos da topologia induzida de $\mathbb{R}^m$. \(\square\)
\end{def*}
\begin{def*}[Variedade suave com bordo]
	Sejam $m\le n$ e $M\subset\mathbb{R}^n$. Dizemos que $M$ é uma \textbf{$m$-variedade suave com bordo} se para todo $p\in M$ existe um tríplice $(U,V,\varphi)$ onde
	\begin{itemize}
		\item $U\subset H^m$ é aberto (na topologia induzida de $\mathbb{R}^m$);
		\item $V\subset M$ é aberto (na topologia subespaço de $\mathbb{R}^n$);
		\item $\varphi:U\to V$ é um \emph{difeomorfismo} (com inversa suave) entre $U$ e $V$;
		\item $\varphi(x)=p$ para algum $x\in U$.
	\end{itemize}
	Chamamos $\varphi$ de \textbf{parametrização local} de $M$ sobre $p$. O \textbf{bordo} de $M$ é
	\[
		\partial M \;=\; \bigl\{\,p\in M \;\bigm|\; \exists\,(U,V,\varphi),\, x\in U\cap \partial H^m \text{ com } \varphi(x)=p\,\bigr\}.
	\]
	O \textbf{interior} de $M$ é $\,\mathrm{Int}(M)=M\setminus \partial M$. Se $\partial M=\emptyset$, dizemos que $M$ é \textbf{sem bordo}. \(\square\)
\end{def*}


\begin{example}
	Com base na definição acima, temos os seguintes exemplos:
	\begin{enumerate}
		\item O intervalo fechado $[0,1]\subset\mathbb{R}$ é uma $1$-variedade com bordo; $\partial[0,1]=\{0,1\}$.
		\item O disco fechado $\overline{D}^2=\{(x,y)\in\mathbb{R}^2:\,x^2+y^2\le 1\}$ é uma $2$-variedade com bordo; $\partial\overline{D}^2=S^1$.
		\item A bola fechada $\overline{B}^m\subset\mathbb{R}^m$ é uma $m$-variedade com bordo; $\partial\overline{B}^m=S^{m-1}$.
		\item O cilindro sólido $D^2\times[0,1]\subset\mathbb{R}^3$ é uma $3$-variedade com bordo;
		      \[
			      \partial(D^2\times[0,1])=(S^1\times[0,1])\cup(D^2\times\{0\})\cup(D^2\times\{1\}).
		      \]
	\end{enumerate}
\end{example}

\begin{tcolorbox}[
		skin=enhanced,
		title=Observação,
		fonttitle=\bfseries,
		colframe=black,
		colbacktitle=cyan!75!white,
		colback=cyan!15,
		colbacklower=black,
		coltitle=black,
		drop fuzzy shadow,
		%drop large lifted shadow
	]
	Não confunda \emph{bordo de variedade} com \emph{fronteira topológica} como subconjunto de $\mathbb{R}^n$. Por exemplo, o círculo $S^1\subset\mathbb{R}^2$ tem fronteira topológica (no sentido de subconjuntos de $\mathbb{R}^2$) igual a ele mesmo, mas é uma variedade \emph{sem bordo}. Já o disco fechado $\overline{D}^2$ tem bordo (da variedade) igual a $S^1$.
\end{tcolorbox}

\begin{prop*}[Independência de parametrização do bordo]
	Se $p\in M$ é imagem de um ponto de $\partial H^m$ por alguma parametrização local $(U,V,\varphi)$, então $p\in\partial M$ para \emph{qualquer} parametrização local $(U',V',\psi)$ centrada em $p$; isto é, o fato de $p$ estar no bordo é independente da escolha de carta.
\end{prop*}

\begin{proof*}
	Sejam $(U,V,\varphi)$ e $(U',V',\psi)$ parametrizações locais com $\varphi(x)=\psi(y)=p$. A \emph{mudança de coordenadas}
	\[
		\Phi \;=\; \varphi^{-1}\circ \psi : U' \longrightarrow U
	\]
	é um difeomorfismo entre abertos de $H^m$. Tal difeomorfismo estende-se (localmente) a abertos de $\mathbb{R}^m$ e, pelo Teorema da Função Inversa, envia interior em interior: $\Phi\bigl(U'\cap\{x_m>0\}\bigr)\subset U\cap\{x_m>0\}$, e o mesmo vale para $\Phi^{-1}$. Logo, $\Phi$ envia $\partial H^m$ em $\partial H^m$. Em particular, se $x\in\partial H^m$ então $y=\Phi^{-1}(x)\in\partial H^m$. Portanto, $p=\psi(y)$ é ponto de bordo em \emph{qualquer} carta centrada em $p$. \qedsymbol
\end{proof*}

\begin{prop*}[O bordo é subvariedade]
	Se $M\subset\mathbb{R}^n$ é uma $m$-variedade suave com bordo, então $\partial M$ é uma subvariedade suave de $M$ de dimensão $m-1$ (e, portanto, uma $(m-1)$-variedade suave \emph{sem} bordo).
\end{prop*}

\begin{proof*}
	Seja $p\in\partial M$ e escolha uma carta $(U,V,\varphi)$ com $\varphi(0)=p$ e $U\subset H^m$ aberto. Defina
	\[
		U' \;=\; U\cap \partial H^m \;\cong\; \text{aberto de }\mathbb{R}^{m-1},
		\qquad
		\psi \;=\; \varphi\big|_{U'} : U'\longrightarrow V\cap \partial M .
	\]
	Então $\psi$ é um homeomorfismo suave sobre sua imagem (restrição de uma difeomorfismo) e fornece cartas para $\partial M$. As mudanças de coordenadas entre cartas de $\partial M$ são restrições das mudanças de coordenadas de $M$ e, pelo argumento da proposição anterior, preservam $\partial H^m$, logo são difeomorfismos entre abertos de $\mathbb{R}^{m-1}$. Assim, $\partial M$ herda uma estrutura de variedade suave de dimensão $m-1$. Como localmente $\partial M$ é modelado em abertos de $\mathbb{R}^{m-1}$ (não do semiespaço), ele não possui bordo. \qedsymbol
\end{proof*}

\begin{exr}
	Mostre as seguintes coisas:
	\begin{itemize}
		\item Se $M$ é uma $m$-variedade com $\partial M\neq\emptyset$, então $\partial M$ é uma $(m-1)$-variedade suave \emph{sem} bordo.
		\item Para $M$ variedade sem bordo e $I=[0,1]$, vale
		      \[
			      \partial(M\times I) \;=\; (M\times\{0\})\cup(M\times\{1\}).
		      \]
		\item Se $M$ e $N$ são variedades, com $\partial N=\emptyset$, então
		      \[
			      \partial(N\times M) \;=\; N\times \partial M.
		      \]
	\end{itemize}
\end{exr}
\end{document}

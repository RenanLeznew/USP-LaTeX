\documentclass[../differential_forms.tex]{subfiles}
\begin{document}
\section{Aula 08 - 15 de Setembro, 2025}
\subsection{Motivações}
\begin{itemize}
	\item Outras construções do determinante.
\end{itemize}
\subsection{Construção Inrínseca dos Determinantes}
Na aula passada, mais curta para reposição, dado um espaço vetorial E com dimensão m, vimos que \(\mathrm{dim}(\mathcal{A}^{m}(E; \mathbb{R}))=1\),
e pudemos definir o determinante de um endomorfismo linear intrinsecamente, ou seja, sem depender de qualquer escolha de base ou elemento,
da seguinte forma:
\begin{def*}
	Seja \(T:E\rightarrow E\) um endomorfismo linear; o \textbf{determinante} de T é definido como sendo o escalar \(\det{(T)}\in \mathbb{R}\) tal que
	\[
		f(Tv_1, \dotsc , Tv_{m}) = \det{(T)} f(v_1, \dotsc , v_{m})
	\]
	para qualquer \(f\in \mathcal{A}^{m}(E; \mathbb{R})\) e quaisquer \(v_1, \dotsc , v_m\). \(\square\)
\end{def*}
\begin{tcolorbox}[
		skin=enhanced,
		title=Observação,
		fonttitle=\bfseries,
		colframe=black,
		colbacktitle=cyan!75!white,
		colback=cyan!15,
		colbacklower=black,
		coltitle=black,
		drop fuzzy shadow,
		%drop large lifted shadow
	]
	Vale comentar um pouco sobre o nome determinante: no contexto de \(E = \mathbb{R}\), conforme sabemos de outros conteúdos, os endomorfismos da reta são sempre da forma
	\(x\mapsto \lambda x\), então a única coisa que os \textit{determina} individualmente é o escalar -- e, no fim, a definição intrínseca do determinante
	é exatamente isso; daí o nome \textit{determinante}: é ele quem vai dizer como é o endomorfismo da reta na reta.

	Além disso, observe que este valor \(\det{(T)}\) é único.
\end{tcolorbox}
Seguindo em nossa breve revisão, também vimos algumas ferramentas usadas nessa construção; uma delas foi que toda aplicação linear \(T:E\rightarrow F\) induz, para cada r positivo, uma aplicação linear
\[
	T^{\#}: \Lambda^{r}(E; \mathbb{R})\rightarrow \Lambda^{r}(E; \mathbb{R})
\]
definida como
\[
	(T^{\#}f)(v_1,\dotsc ,v_r)= f(Tv_1,\dotsc ,TV_r),\quad \forall f\in \Lambda ^{r}(F; \mathbb{R})\;\&\; v_1,\dotsc ,v_r\in E,
\]
satisfazendo
\begin{itemize}
	\item T é linear;
	\item Se \(T = \mathrm{Id}_E\), então \(T^{\#} = \mathrm{Id}_{\mathcal{A}^{r}(E; \mathbb{R})}\);
	\item Se \(S:E\rightarrow W\) e \(T:W\rightarrow F\) são lineares, então
	      \[
		      (T\circ S)^{ \#} = S^{\#}\circ T^{\#}; \text{ e }
	      \]
	\item Se \(T:E\rightarrow F\) é um isomorfismo, então \(T^{\#}\) também é, com inversa \((T^{\#})^{-1}=(T^{-1})^{\#}\).
\end{itemize}
Observe que, disso, temos a explicitação do determinante como
\begin{align*}
	\det{}: & \mathcal{L}(E)\rightarrow\mathbb{R} \\
	        & T\longmapsto \det{(T)}.
\end{align*}

Dando continuidade, veremos algumas outras propriedades dos determinantes e veremos as outras construções dessa transformação.
\begin{prop*}
	Seja \(T:E\rightarrow E\) um endomorfismo com \(\mathrm{dim}(E)=m\). Então, o determinante satisfaz:
	\begin{itemize}
		\item[i)] O determinante da identidade é a unidade: \(\det{(\mathrm{id})}=1\);
		\item[ii)] O determinante da composição é o produto dos determinantes: para todo \(S:E\rightarrow E\),
		      \[
			      \det{(S\circ T)}=\det{(S)}\det{(T)}; \text{ e}
		      \]
		\item[iii)] T é invertível se, e somente se, \(\det{(T)})\neq 0\).
	\end{itemize}
\end{prop*}
%\begin{proof*}
% (i) A primeira propriedade é automática;
%
% (ii) Para a segunda, usaremos
% \[
% 	(S\circ T)^{\#} = T^{\#}\circ S^{\#},
% \]
% tal que, para toda \(f\in \Lambda^{r}(E; \mathbb{R})\) e quaisquer \(v_1,\dotsc , v_r\in E\)
% \begin{align*}
% 	 & \circ \quad (S\circ T)^{\#}(f)(v_1,\dotsc ,v_r)=f((S\circ O)v_1,\dotsc ,(S\circ T)v_r) = \det{(S\circ T)}f(v_1,\dotsc ,v_r)  \\
% 	 & \circ \quad  (S\circ T)^{\#}(f)(v_1,\dotsc ,v_r) = (T^{\#}\circ S^{\#})(f)(v_1,\dotsc ,v_r) = T^{\#}(f(Sv_1,\dotsc , Sv_r))= \\
% 	 & \quad \quad T^{\#}(\det{S}f(v_1,\dotsc ,v_r)) = \det{(S)} T^{\#}(f(v_1,\dotsc ,v_r)) = \det{(S)}f(Tv_1,\dotsc ,Tv_r)         \\
% 	 & \quad \quad = \det{(S)}\det{(T)}f(v_1,\dotsc ,v_r).
% \end{align*}
% Assim, comparando as duas últimas igualdades, segue que, para toda \(f\in \Lambda^{r}(E; \mathbb{R})\) e quaisquer \(v_1,\dotsc , v_r\in E\),
% \[
% 	\det{(S)}\det{(T)}f(v_1,\dotsc ,v_r) = \det{(S\circ T)}f(v_1,\dotsc ,v_r) \Rightarrow \det{(S)}\det{(T)}=\det{(S\circ T)}.
% \]
%
% (iii) Finalmente, por um lado, se T é invertível, então
% \[
% 	1=\det{(T\circ T^{-1})} = \det{(T)}\det{(T^{-1})},
% \]
% donde segue que \(\det{(T)}\neq 0\); por outro, se \(\det{(T)}\neq 0\), significa que as imagens \(Te_1,\dotsc , Te_{m}\) de uma base formam outra base e, portanto, T é um isomorfismo. \qedsymbol
%\end{proof*}
\begin{exr}
	Prove a proposição acima.
\end{exr}

\subsection{Definição por Matrizes}
Para a definição por matrizes, que é a usualmente conhecida para o determinante, faremos uso dos resultados que obtivemos para ele como
escalar para chegar no que já sabemos. O jeito mais natural para isso seria utilizando bases como base\footnote{Frase estranha, mas ok}
e usando as matrizes de mudança de base; o problema com isso é justamente o fato de fixar uma base, e cada pessoa pode acabar escolhendo
uma diferente.

No fim das contas, sabemos que isso acaba dando certo lá da álgebra linear, mas essa definição tem exatamente o problema de não ser
intrínseca -- ela sempre tem, no fundo, essa dependência em uma \textit{escolha para além da própria matriz}. Ainda assim, vamos mostrar
que as duas construções coincidem; para isso, definimos como de costume:
\begin{def*}
	Seja \(\alpha \in \mathbb{M}_{m\times m}(\mathbb{R})\) uma matriz. Definimos o \textbf{determinante da matriz \(\alpha \)} como
	\begin{align*}
		\mathrm{det}: & \mathbb{M}_{m\times m}(\mathbb{R})\rightarrow \mathbb{R}             \\
		              & \alpha \longmapsto \det{(\alpha )}\coloneqq \det{(\tilde{\alpha })},
	\end{align*}
	onde \(\tilde{\alpha }:\mathbb{R}^{m}\rightarrow \mathbb{R}^{m}\) é o endomorfismo dado por
	\[
		\tilde{\alpha }(e_{j}) =\sum\limits_{i=1}^{m}\alpha_{ij}e_{i},
	\]
	sendo \((e_1, \dotsc , e_{m})\) a base canônica de \(\mathbb{R}^{m}\). \(\square\)
\end{def*}
Como de costume, denotamos simplesmente por \( \mathbb{M}_{m}(\cdot ) \coloneqq \mathbb{M}_{m \times m}(\cdot )\)
\begin{prop*}
	O determinante é a única forma m-linear alternada
	\[
		\det: \mathbb{M}_{m}(\mathbb{R})\rightarrow \mathbb{R}
	\]
	dos vetores-coluna de uma matriz que vale 1 na identidade, ou seja, \(\det{(\mathrm{Id}_{m})}=1\).
\end{prop*}
Vale notar que a unicidade pode ser vista como uma consequência justamente da forma com que observamos como o determinante é o escalar que \textit{determina}
o endomorfismo entre os espaços vetoriais, então apenas ele pode ser o escalar que fixa o endomorfismo como o escalar 1, exatamente como na
observação, pois o determinante tem a reta como imagem; então, podemos fazer sua prova apenas para a base canônica e o resto segue do que estudamos
até agora.

%\begin{proof*}
% Seja \(f_{0}\in \Lambda^{m}(\mathbb{R}^{m}; \mathbb{R})\) tal que \(f_{0}(e_1,\dotsc , e_{m})=1\). Se \(\alpha =(\alpha_1,\dotsc , \alpha_{m})\) são as colunas, então
% \[
% 	\det{(\alpha )} = f_{0}(\alpha_1,\dotsc , \alpha_{m}).
% \]
% Portanto, \(\mathrm{det}\) é m-linear, alternada e, por unicidade de \(f_{0}\), única. \qedsymbol
%\end{proof*}
\begin{exr}
	Prove esta proposição.
\end{exr}

Agora, como de costume na álgebra linear, vamos exibir os resultados necessários para compreender que, no fim, o determinante vai dar
o mesmo valor independente da base (que foi algo que ganhamos por definição com a outra construção).

\begin{def*}
	Sejam \(S\in \mathcal{L}(E)\) e \(T\in \mathcal{L}(F)\). Dizemos que S e T são \textbf{conjugadas} se existe um isomorfismo \(\varphi :E\rightarrow F\) tal que
	\[
		T= \varphi \circ S\circ \varphi^{-1}.
	\]
	Em outras palavras, o seguinte diagrama comuta:
	\begin{center}
		\begin{tikzpicture}[
				observed/.style = {rectangle, thick, text centered, draw, text width = 6em},
				latent/.style = {ellipse, thick, draw, text centered, text width = 6em},
				error/.style ={circle, thick, draw, text centered},
				confounding/.style = {rectangle, thick, text centered, draw, text width = 6em, minimum width = 5.5in},
				outcome/.style = {rectangle, thick, draw, text centered, minimum height = 3.5in, text width = 6em},
			]
			\node(TL) at (-1,1){E};
			\node(BL) at (-1,-1){E};
			\node(TR) at (1,1){F};
			\node(BR) at (1,-1){F};

			\draw[Arrow](TL)--node[midway, above] {\(\varphi \)}(TR);
			\draw[Arrow](BL)--node[midway, below] {\(\varphi \)}(BR);
			\draw[Arrow](TL)--node[midway, left] {S}(BL);
			\draw[Arrow](TR)--node[midway, right] {T}(BR);

		\end{tikzpicture}
	\end{center}
\end{def*}
\begin{lemma*}
	Se S e T são conjugados, então \(\det{(S)} = \det{(T)}\).
\end{lemma*}
%\begin{proof*}
% Utilizando o fato de que \(T^{\#}= (\varphi^{\#})^{-1}\circ S^{\#}\circ \varphi^{\#}\), obtemos
% \[
% 	T^{\#}f = \det{(S)}f,\quad \forall f\in \Lambda^{m}(F; \mathbb{R}).
% \]
% Portanto, por unicidade,
% \[
% 	\det{(T)}=\det{(S)}.\text{ \qedsymbol}
% \]
%\end{proof*}
\begin{exr}
	Prove o lema acima.
\end{exr}

Para amarrar tudo bonitinho, vamos mostrar que os dois determinantes que vimos de fato coincidem, a menos da correspondência natural entre matrizes e transformações lineares:
\begin{prop*}
	Seja \(A:E\rightarrow E\) um endomorfismo. Para qualquer matriz \(\alpha \) que representa A em uma base de E, vale
	\[
		\det{(A)} = \det{(\alpha )}.
	\]
\end{prop*}
\begin{crl*}
	Se \(\mathrm{dim}(E)=m\), f é uma forma m-linear alternada, \(\alpha \) é uma matriz m por m e \(\mathcal{E}=(e_1,\dotsc , e_{m})\) é uma base alternada de E, então
	\[
		f\biggl(\sum\limits_{i=1}^{m}\alpha_{1i}e_{i}, \dotsc , \sum\limits_{i=1}^{m}\alpha_{mi}e_{i}\biggr) = \det{(\alpha )}f(e_1,\dotsc , e_{m}).
	\]
\end{crl*}

\subsection{Apêndice: determinante de vetores}
Até o momento, definimos o determinante de duas formas diferentes, que coincidem:
\begin{itemize}
	\item como um escalar associado a um endomorfismo linear;
	\item como uma função definida sobre as colunas de uma matriz quadrada.
\end{itemize}
Agora, veremos mais uma perspectiva -- a do determinante de um conjunto de vetores em relação a uma base fixa.
\begin{def*}
	Seja \(\mathcal{E}=(e_1,\dotsc , e_{m})\) uma base ordenada de E. Dado um conjunto de vetores \(u_1, \dotsc , u_{m}\in E\), escrevemos cada vetor em coordenadas na base \(\mathcal{E}:\)
	\[
		u_{j}=\sum\limits_{i=1}^{m} \alpha_{j}^{i}e_{i},\quad j=1,\dotsc ,m,
	\]
	a partir do qual formamos a matriz \(\alpha =(\alpha_{j}^{i})\) de ordem m, com os vetores \(u_1,\dotsc , u_{m}\) como coordenadas. Com isso, definimos o \textbf{determinante dos vetores \(u_1,\dotsc , u_{m}\) em relação à base \(\mathcal{E}\)} como
	\[
		\det{[u_{1},\dotsc , u_{m}]}_{\mathcal{E}}\coloneqq \det{(\alpha )}. \quad \square
	\]
\end{def*}
\begin{tcolorbox}[
		skin=enhanced,
		title=Observação,
		fonttitle=\bfseries,
		colframe=black,
		colbacktitle=cyan!75!white,
		colback=cyan!15,
		colbacklower=black,
		coltitle=black,
		drop fuzzy shadow,
		%drop large lifted shadow
	]
	Diferente do determinante do endomorfismo, endomorfismo, que é uma propriedade intrínseca, ou seja, independente da base, o determinante de vetores \textit{depende} da escolha da base \(\mathcal{E}.\)

	Geometricamente, esse determinante mede o \textit{\textbf{volume orientado}} do paralelepípedo gerado pelos vetores \(u_1,\dotsc , u_{m}\) em relação ao sistema de coordenadas dado por \(\mathcal{E}.\)
\end{tcolorbox}

\end{document}

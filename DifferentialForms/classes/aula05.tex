\documentclass[../differential_forms.tex]{subfiles}
\begin{document}
\section{Aula 05 - 08 de Setembro, 2025}
\subsection{Motivações}
\begin{itemize}
	\item Operador de Anti-Simetrização;
	\item Determinantes.
\end{itemize}
\subsection{Operador de Anti-Simetrização e Determinantes}
\begin{def*}
	Uma permutação \(\sigma \in S_r\) \textbf{atua em f \(L^{r}(E; F)\)} definindo
	\[
		(\sigma f)(v_1,\dotsc ,v_r) = f(v_{\sigma (1)}, \dotsc , v_{\sigma (r)}).
	\]
	Diremos que f é \textbf{alternada} se
	\[
		f(v_{\sigma (1),\dotsc , f_{\sigma (r)}})= \mathrm{sgn}(\sigma )f(v_1,\dotsc ,v_r),
	\]
	onde \(\mathrm{sgn}(\sigma )\) é o sinal da permutação. \(\square\)
\end{def*}
\begin{def*}
	O \textbf{Operador de Anti-simetrização} é o operador \(A:L^{r}(E; F)\rightarrow L^{r}(E; F)\) definido por
	\[
		A(f)= \sum\limits_{\sigma \in S_r}^{} \mathrm{sgn}(\sigma )(\sigma f). \; \square
	\]
\end{def*}
\begin{prop*}
	Para qualquer \(f\in L^{r}(E; F)\), valem:
	\begin{itemize}
		\item[i)] O operador de anti-simetrização é uma transformação r-linear alternada entre E e F, isto é, \(Af\in A^{r}(E; F)\);
		\item[ii)] f é alternada se, e somente se, \(Af=r!f\); e
		\item[iii)] Se existir \(\rho \in S_r\) ímpar tal que \(\rho f = f\), então \(Af=0\).
	\end{itemize}
\end{prop*}
\begin{crl*}
	O operador A projeta \(L^{r}(E; F)\) sobre \(A^{r}(E; F)\).
\end{crl*}
\begin{prop*}
	Dada uma base ordenada \(E^{*}=(e_{1}^{*}, \dotsc , e_{m}^{*})\) de \(E^{*}\), as formas
	\[
		e_{(J)}^{*}=\varphi (e_{j_1}^{*}, \dotsc , e_{j_r}^{*}), \quad J = \{j_1<\dotsc <j_r\}
	\]
	formam uma base de \(A^{r}(E; \mathbb{R})\) e, em particular,
	\[
		\mathrm{dim}(A^{r}(E; \mathbb{R}))= \binom{m}{r}.
	\]
\end{prop*}
\begin{crl*}
	Se \(\mathrm{dim}(E) = m\), então
	\[
		\mathrm{dim}(A^{m}(E; \mathbb{R}))=1 \quad\&\quad \mathrm{dim}(A^{r}(E; \mathbb{R}^{n}))= \binom{m}{r}n.
	\]
\end{crl*}
\begin{prop*}
	Sejam \(E=(e_1,\dotsc ,e_{m})\) e \(F=(w_1,\dotsc ,w_{n})\) bases ordenadas; para cada \(J=\{j_1<\dotsc <j_r\}\subseteq \{1,\dotsc ,m\}\) e cada k, defina
	\[
		f_{k}^{J}(e_{j_1}, \dotsc , e_{j_r}) = w_{k}, \quad f_{k}^{J}(e_{p_1}, \dotsc , e_{p_r})=0 \text{ se }\{p_1,\dotsc , p_r\}\neq J.
	\]
	Então, as aplicações \(f_{k}^{J}\) constituem uma base de \(A^{r}(E; F)\) e, consequentemente,
	\[
		\mathrm{dim}(A^{r}(E; F))=\binom{m}{r}n.
	\]
\end{prop*}
\subsection{Determinantes}
Começaremos este tópico apresentando uma definição intrínseca dos determinantes pela perspectiva de um escalar único para cada transformação \(T:E\rightarrow E\) que satisfaz uma propriedade específica. Antes dela, porém, vale mencionar que denotaremos o \textbf{espaço das formas m-lineares alternadas}, ou seja, o dual de \(A^{m}(E; F)\), por \(\Lambda^{m}(E; \mathbb{R})\).
\begin{def*}
	Seja \(T:E\rightarrow E\) um endomorfismo. O \textbf{determinante de T} é o \textit{único} escalar \(\det{T}\in \mathbb{R}\) tal que, para toda forma m-linear alternada \(f\in \Lambda^{m}(E; \mathbb{R})\) e quaisquer \(v_1,\dotsc , v_{m}\in E\),
	\[
		f(Tv_1,\dotsc , Tv_{m}) = \det{(T)} f(v_1,\dotsc ,v_{m}). \quad \square
	\]
\end{def*}
\begin{prop*}
	Uma aplicação linear \(T:E\rightarrow F\) induz, para cada r positivo, uma aplicação linear
	\[
		T^{\#}: \Lambda^{r}(E; \mathbb{R})\rightarrow \Lambda^{r}(E; \mathbb{R})
	\]
	definida como
	\[
		(T^{\#}f)(v_1,\dotsc ,v_r)= f(Tv_1,\dotsc ,TV_r),\quad \forall f\in \Lambda ^{r}(F; \mathbb{R})\;\&\; v_1,\dotsc ,v_r\in E.
	\]
\end{prop*}
\begin{proof*}
	O fato de \(T^{\#}f\) ser uma forma m-linear alternada é imediato, assim como a linearidade de \(T^{\#}.\) Além disso, se T corresponde à identidade de E, então \(T^{\#}=\mathrm{id}.\) e, se \(S:E\rightarrow F\) e \(T:F\rightarrow G\) são lineares, então
	\[
		(T \circ S)^{\#}=S^{\#}\circ T^{\#}.\text{ \qedsymbol}
	\]
\end{proof*}
\begin{crl*}
	Se \(S:E\rightarrow F\) for um isomorfismo, então \(S^{\#}\) também o é, com inversa dada por \((S^{\#})^{-1}=(S^{-1})^{\#}\).
\end{crl*}
\begin{prop*}
	Seja \(T:E\rightarrow E\) um endomorfismo com \(\mathrm{dim}(E)=m\). Então, o determinante satisfaz:
	\begin{itemize}
		\item[i)] O determinante da identidade é a unidade: \(\det{(\mathrm{id})}=1\);
		\item[ii)] O determinante da composição é o produto dos determinantes: para todo \(S:E\rightarrow E\),
		      \[
			      \det{(S\circ T)}=\det{(S)}\det{(T)}; \text{ e}
		      \]
		\item[iii)] T é invertível se, e somente se, \(\det{(T)})\neq 0\).
	\end{itemize}
\end{prop*}
\begin{proof*}
	(i) A primeira propriedade é automática;

	(ii) Para a segunda, usaremos
	\[
		(S\circ T)^{\#} = T^{\#}\circ S^{\#},
	\]
	tal que, para toda \(f\in \Lambda^{r}(E; \mathbb{R})\) e quaisquer \(v_1,\dotsc , v_r\in E\)
	\begin{align*}
		 & \circ \quad (S\circ T)^{\#}(f)(v_1,\dotsc ,v_r)=f((S\circ O)v_1,\dotsc ,(S\circ T)v_r) = \det{(S\circ T)}f(v_1,\dotsc ,v_r)                                     \\
		 & \circ \quad  (S\circ T)^{\#}(f)(v_1,\dotsc ,v_r) = (T^{\#}\circ S^{\#})(f)(v_1,\dotsc ,v_r) = T^{\#}(f(Sv_1,\dotsc , Sv_r)) = T^{\#}(\det{S}f(v_1,\dotsc ,v_r)) \\
		 & \quad \quad = \det{(S)} T^{\#}(f(v_1,\dotsc ,v_r)) = \det{(S)}f(Tv_1,\dotsc ,Tv_r) = \det{(S)}\det{(T)}f(v_1,\dotsc ,v_r).
	\end{align*}
	Assim, comparando as duas últimas igualdades, segue que, para toda \(f\in \Lambda^{r}(E; \mathbb{R})\) e quaisquer \(v_1,\dotsc , v_r\in E\),
	\[
		\det{(S)}\det{(T)}f(v_1,\dotsc ,v_r) = \det{(S\circ T)}f(v_1,\dotsc ,v_r) \Rightarrow \det{(S)}\det{(T)}=\det{(S\circ T)}.
	\]

	(iii) Finalmente, por um lado, se T é invertível, então
	\[
		1=\det{(T\circ T^{-1})} = \det{(T)}\det{(T^{-1})},
	\]
	donde segue que \(\det{(T)}\neq 0\); por outro, se \(\det{(T)}\neq 0\), significa que as imagens \(Te_1,\dotsc , Te_{m}\) de uma base formam outra base e, portanto, T é um isomorfismo. \qedsymbol
\end{proof*}

Para a definição por matrizes, que é a usualmente conhecida para o determinante, faremos uso dos resultados que obtivemos para ele como escalar para chegar no que já sabemos. Para isso,
\begin{def*}
	Seja \(\alpha \in \mathbb{M}_{m\times m}(\mathbb{R})\) uma matriz. Definimos o \textbf{determinante da matriz \(\alpha \)} como
	\begin{align*}
		\mathrm{det}: & \mathbb{M}_{m\times m}(\mathbb{R})\rightarrow \mathbb{R}             \\
		              & \alpha \longmapsto \det{(\alpha )}\coloneqq \det{(\tilde{\alpha })},
	\end{align*}
	onde \(\tilde{\alpha }:\mathbb{R}^{m}\rightarrow \mathbb{R}^{m}\) é o endomorfismo dado por
	\[
		\tilde{\alpha }(e_{j}) =0\sum\limits_{i=1}^{m}\alpha_{ij}e_{i},
	\]
	sendo \((e_1, \dotsc , e_{m})\) a base canônica. \(\square\)
\end{def*}
\begin{prop*}
	O determinante é a única forma m-linear alternada dos vetores-coluna de uma matriz que vale 1 na identidade.
\end{prop*}

\begin{proof*}
	Seja \(f_{0}\in \Lambda^{m}(\mathbb{R^{m}}; \mathbb{R})\) tal que \(f_{0}(e_1,\dotsc , e_{m})=1\). Se \(\alpha =(\alpha_1,\dotsc , \alpha_{m})\) são as colunas, então
	\[
		\det{(\alpha )} = f_{0}(\alpha_1,\dotsc , \alpha_{m}).
	\]
	Portanto, \(\mathrm{det}\) é m-linear, alternada e, por unicidade de \(f_{0}\), única. \qedsymbol
\end{proof*}

\begin{def*}
	Sejam \(S\in \mathcal{L}(E)\) e \(T\in \mathcal{L}(F)\). Dizemos que S e T são \textbf{conjugadas} se existe um isomorfismo \(\varphi :E\rightarrow F\) tal que
	\[
		T= \varphi \circ S\circ \varphi^{-1}. \quad \square
	\]
\end{def*}
\begin{lemma*}
	Se S e T são conjugados, então \(\det{(S)} = \det{(T)}\).
\end{lemma*}
\begin{proof*}
	Utilizando o fato de que \(T^{\#}= (\varphi^{\#})^{-1}\circ S^{\#}\circ \varphi^{\#}\), obtemos
	\[
		T^{\#}f = \det{(S)}f,\quad \forall f\in \Lambda^{m}(F; \mathbb{R}).
	\]
	Portanto, por unicidade,
	\[
		\det{(T)}=\det{(S)}.\text{ \qedsymbol}
	\]
\end{proof*}
Para amarrar tudo bonitinho, vamos mostrar que os dois determinantes que vimos de fato coincidem, a menos da correspondência natural entre matrizes e transformações lineares:
\begin{prop*}
	Seja \(A:E\rightarrow E\) um endomorfismo. Para qualquer matriz \(\alpha \) que representa A em uma base de E, vale
	\[
		\det{(A)} = \det{(\alpha )}.
	\]
\end{prop*}
\begin{crl*}
	Se \(\mathrm{dim}(E)=m\), f é uma forma m-linear alternada, \(\alpha \) é uma matriz m por m e \(\mathcal{E}=(e_1,\dotsc , e_{m})\) é uma base alternada de E, então
	\[
		f\biggl(\sum\limits_{i=1}^{m}\alpha_{1i}e_{i}, \dotsc , \sum\limits_{i=1}^{m}\alpha_{mi}e_{i}\biggr) = \det{(\alpha )}f(e_1,\dotsc , e_{m}).
	\]
\end{crl*}
\end{document}

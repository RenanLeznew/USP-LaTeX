\documentclass[../differential_forms.tex]{subfiles}
\begin{document}
\section{Aula 05 - 08 de Setembro, 2025}
\subsection{Motivações}
\begin{itemize}
	\item Álgebra Exterior;
	\item Operador de Anti-Simetrização.
\end{itemize}
\subsection{Álgebra Exterior}
O estudo de aplicações multilineares constitui um pilar muito importante no estudo de estruturas algébricas e geométricas nos espaços vetoriais,
e um dos conceitos que surgem naturalmente ao estudá-las é o de \textit{alternância}, desempenhando papel central na construção da álgebra
exterior e na definição de formas diferenciais. Essencialmente, a ideia é entender o que ocorre com uma aplicação r-linear quando os índices
dos vetores nos quais ela está sendo aplicada são alternados de posição. Formalmente,
\begin{def*}
	Uma aplicação r-linear \(f:E^{r}\rightarrow F\) é dita \textbf{alternada}, ou \textbf{anti-simétrica}, se para quaisquer permutações de dois argumentos, a aplicação muda de sinal; ou seja,
	\begin{align*}
		  & f(v_1, \dotsc , v_{i-1}, v_{i}, v_{i+1}, \dotsc , v_{j-1}, v_{j}, v_{j+1}, \dotsc , v_r) =          \\
		= & -f(v_1,\dotsc , v_{i-1}, v_{j}, v_{i+1}, \dotsc , v_{j-1}, v_{i}, v_{j+1}, \dotsc , v_r).\; \square
	\end{align*}
\end{def*}
\begin{def*}
	Dizemos que
	\[
		\mathcal{A}^{r}(E; F)=\{f\in \mathcal{L}^{r}(E; F):\; f \text{ é alternada}\}
	\]
	é o \textbf{espaço das aplicações r-lineares alternadas}, também chamado de \textbf{álgebra exterior.} \(\square\)
\end{def*}
\begin{exr}
	Mostre que, com as mesmas operações de \(\mathcal{L}^{r}(E; F)\), temos uma estrutura de espaço vetorial para \(\mathcal{A}^{r}(E; F).\)
\end{exr}
\begin{prop*}
	Uma aplicação r-linear \(f:E^{r}\rightarrow F\) é atlernada se, e somente se, sempre que a lista de argumentos contiver vetores
	\textit{repetidos}, a aplicação se anula, isto é,
	\[
		f(v_1, \dotsc , v_{i-1}, \underbrace{v}_{\mathclap{\text{i-ésima}}}, v_{i+1}, \dotsc , v_{j-1}, \underbrace{v}_{\mathclap{\text{j-ésima}}}, v_{j+1}, \dotsc , v_r) = 0
	\]
	para quaisquer vetores \(v, v_1, \dotsc , v_r\in E.\)
\end{prop*}
\begin{proof*}
	A ideia nesta prova é que basta provarmos isso para formas bilineares e, conforme for aumentando a linearidade por entrada, podemos
	apenas considerá-las dois a dois.
	Suponha primeiramente, então, que f se anula quando tem valores iguais índices diferentes e considere
	\[
		\varphi (u, v)=f(v_1,\dotsc , u,\dotsc ,v,\dotsc v_r).
	\]
	Pela alternância, segue que
	\[
		0=\varphi (u+v, u+v) = \varphi(u, u)+\varphi (u, v)+\varphi (v, u)+\varphi (v, v)
	\]
	e, como \(\varphi (u, u)=\varphi (v, v)=0\), temos
	\[
		\varphi (u,v) = -\varphi (v, u),
	\]
	ou seja, f é alternada e a igualdade acima pode ser explicitamente escrita como
	\[
		f(v_1,\dotsc , u,\dotsc ,v,\dotsc v_r) = -f(v_1,\dotsc , v,\dotsc ,u,\dotsc v_r).
	\]

	Por outro lado, se f é alternada, então
	\[
		\varphi (v, v)=-\varphi (v, v) \Rightarrow 2\varphi (v, v)=0.
	\]
	Logo,
	\[
		\varphi (v, v)=f(v_1,\dotsc , v,\dotsc ,v,\dotsc v_r) = 0
	\]
	e, portanto, f se anula quando tem entradas iguais em índices diferentes. \qedsymbol
\end{proof*}

\begin{example}
	Para \(r=1\), ou seja, \(f:E\rightarrow F\), toda transformação é alternada, pois não é possível violar a condição imposta para
	a alternabilidade. Assim,
	\[
		\mathcal{A}^{1}(E; F)=\mathcal{L}(E; F).
	\]
\end{example}
Para melhorar nosso entendimento, vamos olhar alguns exemplos e observar detalhes deles a fim de encontrar um padrão que contribua
para nossa compreensão sobre o assunto.

\begin{example}
	Se \(f:\mathbb{R}\times \dotsc \times \mathbb{R}\rightarrow F\) é r-linear, então
	\[
		f(t_1,\dotsc , t_r)=t_1 \dotsc t_r \cdot v,\quad v=f(1,\dotsc ,1).
	\]
	Para r maior que 1, a f só é alternada se \(v=0\), que equivale a dizer que f é a aplicação nula. Portanto,
	\[
		\mathcal{A}^{r}(\mathbb{R}; F)=0,\quad r>1.
	\]
\end{example}
\begin{example}
	Vamos considerar \(f\in \mathcal{L}^{r}(\mathbb{R}; \mathbb{R})\) e observe que
	\[
		\mathrm{dim}(\mathcal{L}^{r}(\mathbb{R}; \mathbb{R})) = 1^{r}1 = 1.
	\]
	Vimos que
	\[
		f_{(s)}^{1} = \left\{\begin{array}{ll}
			1, & \quad (s) = (i_1,\dotsc , i_r)    \\
			0, & \quad (s) \neq (i_1,\dotsc , i_r)
		\end{array}\right.
	\]
	é um elemento gerador de \(\mathcal{L}^{r}(\mathbb{R}; \mathbb{R})\). A grosso modo, o que um funcional linear permite que façamos é
	escolher uma variável a menos de uma multiplicação por um número; então, se tivemos uma quantidade enorme de variáveis, ele escolhe
	uma variável, mas ele faz o mesmo com apenas uma.

	Em outras palavras, se temos apenas um número (espaço de dimensão 1), o funcional irá ler este número, permitindo ver o mesmo elemento
	de duas formas diferentes -- uma por ele mesmo, e outra pelo funcional linear que o ``escolhe''.

	Como a dimensão aqui do exemplo é 1, a variedade de funcionais que aparece para nós aqui é \(f\in \mathbb{R}^{*},\; f:\mathbb{R}\rightarrow \mathbb{R}\), que pode
	ser apenas do tipo multiplicação por escalar \(x\mapsto \alpha x\). Consequentemente, este espaço é apenas uma dilatação dos vetores, com
	a única diferença entre os elementos sendo qual o fator de dilatação (qual \(\alpha \) está multiplicando o x).

	Assim, apesar de termos	infinitos membros que fazem \(f(x) = \alpha x\), eles todos compõem um único tipo, tal que, na hora de fazer a conta,
	estaremos fazendo o produto tensorial de um monte de escalar diferente sendo multiplicado -- r cópias tornam-se um produto
	de r funcionais do tipo \(\alpha x\) -- que, no fim, continua sendo só um escalar (formado por todos os outros) multiplicando o produto
	das variáveis de cada cópia: se denotarmos por \(f_{i}(t_{i}) = \alpha_{i}t_{i},\; 1\leq i\leq r\), então
	\[
		f_1(t_1)\otimes \dotsc \otimes f_r(t_r) = at_1 \dotsc t_r, \quad a =\alpha_1 \dotsc \alpha_r,
	\]
	que é um gerador do espaço unidimensional \(\mathcal{L}^{r}(\mathbb{R}; \mathbb{R})\). Consequentemente, dada f uma aplicação r-linear, podemos escrever
	\[
		f(t_1, \dotsc , t_r) = \alpha f_{(s)}^{1}(t_1, \dotsc , t_r) = \alpha t_1, \dotsc , t_r, \quad \alpha \in \mathbb{R}.
	\]
\end{example}
\begin{example}
	A aplicação bilinear
	\[
		f(u, v)=u_1v_2-u_2v_1,\quad u=(u_1,u_2),\; v=(v_1, v_2),
	\]
	é alternada em \(\mathbb{R}^{2}.\) Com efeito, note que
	\[
		f(u, v)=u_1v_2-u_2v_1 = -(u_2v_1 - u_1v_2) = - f(v, u),
	\]
	e já vimos que isto equivale à alternância.
\end{example}
\begin{example}
	O produto vetorial em \(\mathbb{R}^{3}\), dado por
	\[
		f(u, v)=u \times v,\quad u, v\in \mathbb{R}^{3},
	\]
	é um exemplo de aplicação bilinear alternada, afinal
	\[
		f(u, u) = u \times u = 0, \quad u\in \mathbb{R}^{3}
	\]
	que e é exatamente a definição de aplicação alternada.
\end{example}
\begin{prop*}
	Seja \(f:E^{r}\rightarrow F\) uma aplicação r-linear alternada. Se \(v_1,\dotsc ,v_r\) forem vetores linearmente dependentes de E, então
	\[
		f(v_1,\dotsc , v_r)=0.
	\]
\end{prop*}
\begin{proof*}
	Suponha que
	\[
		v_{i} = \sum\limits_{j<i}^{}\alpha_{j}v_{j}.
	\]
	Pela linearidade de f, junto à sua alternância,
	\[
		f(v_1,\dotsc ,v_r)=\sum\limits_{j<i}^{}\alpha_{j}f(v_1,\dotsc , v_{j},\dotsc , v_{j}, \dotsc , v_r)=0. \text{ \qedsymbol}
	\]
\end{proof*}
\begin{crl*}
	Se \(r>\mathrm{dim}(E)\), então
	\[
		\mathcal{A}^{r}(E; F)=\{0\}.
	\]
\end{crl*}
\end{document}

\documentclass[../differential_forms.tex]{subfiles}
\begin{document}
\section{Aula 16 - 10 de Novembro, 2025}
\subsection{Motivações}
\begin{itemize}
	\item Atlas em Variedades com Bordo;
\end{itemize}
\subsection{Variedades com Bordo -- Parte 2}
Nosso próximo passo no estudo de variedades com bordo, que serão as bases para podermos generalizar as integrais, será encontrar uma forma de definir os Atlas nelas.

Por sorte, podemos nos basear fortemente na construção utilizada para o caso das variedades sem bordo para definí-los, destacando apenas a questão que, diferentemente do caso anterior,
precisamos encontrar a função definida num aberto maior utilizada para as variedades com bordo.
\begin{def*}
	Para uma m-variedade com bordo M, o par \((U, \varphi )\) (associado a um ponto p de M) do aberto pareado com o difeomorfismo é chamado \textbf{parametrização de M em torno de p}, sendo p um ponto da imagem de U sob \(\varphi \). A família
	de parametrizações de M cuja união resulta em M é chamado \textbf{Atlas da Variedade}, ou seja, o atlas \(\mathcal{A} = \{(U, \varphi )\}\) é a coleção de parametrizações de M tal que
	\[
		\bigcup_{p\in M}^{}V = M. \quad \square
	\]
\end{def*}

Outra noção indispensável no estudo e cálculo das integrais, quando eram feitas em \(\mathbb{R}^{n}\), era a \textit{orientação}; como já vimos até mesmo no início desse curso, sabemos que é possível definir essa ideia para espaços vetoriais -- tanto é que temos a diferenciação entre uma base \(\{v_1, \dotsc , v_{n}\}\) e uma base com orientação \((v_1, \dotsc , v_{n})\).
Sendo assim, precisamos encontrar uma estrutura parecida para o caso das nossas variedades, e assim como a maior parte das coisas que estamos fazendo, iremos copiar o caso dos espaços vetoriais ao olhar para o exemplo do espaço tangente, e definiremos a orientação na variedade com base na orientação que temos nele. Uma questão ``natural'' que surge é: quantas orientações possíveis distintas somos capazes
de colocar numa única variedade?

A resposta para essa pergunta encontra-se no determinante, e a resposta é apenas duas: esquerda e direita para 1-espaços vetoriais, horário e anti-horário para 2-espaços vetoriais, as regras das mãos esquerda e direita para 3-espaços vetoriais, e assim por diante. Esse permanece sendo o caso para as variedades, e leva à noção das homotopias, que conservam o sinal do determinante; porém, as opções de sinais do determinante
são apenas duas, sendo ele positivo ou negativo. Mais formalmente, as bases podem ser levadas num determinante positivo ou num determinante negativo, e é isso que determina a orientação do espaço.
\begin{tcolorbox}[
		skin=enhanced,
		title=Observação,
		fonttitle=\bfseries,
		colframe=black,
		colbacktitle=cyan!75!white,
		colback=cyan!15,
		colbacklower=black,
		coltitle=black,
		drop fuzzy shadow,
		%drop large lifted shadow
	]
	Esse é o caso para variedades conexas por caminho, pelo menos, mas as variedades desconexas podem ter mais -- para uma variedade suave com n \textit{componentes conexas maximais}, o número de orientações possíveis será \(2^{n}\); este é, na verdade, o caso mais geral do que estamos estudando, pois as variedade conexas por caminho
	nada mais são do que aquelas com uma única componente conexa maximal, ou seja, que tem 2 orientações possíveis.
\end{tcolorbox}

Antes de definição, vale o lembrete formal:
\begin{tcolorbox}[
		skin=enhanced,
		title=Lembrete!,
		after title={\hfill Orientação de um Espaço Vetorial},
		fonttitle=\bfseries,
		sharp corners=downhill,
		colframe=black,
		colbacktitle=yellow!75!white,
		colback=yellow!30,
		colbacklower=black,
		coltitle=black,
		%drop fuzzy shadow,
		drop large lifted shadow
	]
	A \textbf{orientação de um espaço vetorial} E de dimensão finita n e base \(\mathcal{B} = \{v_1, \dotsc , v_{n}\}\) é a escolha de uma ordenação específica \(\mathcal{B}\), denotada por \((v_1, \dotsc , v_{n})\).

	Apenas duas orientações são possíveis para E, e dizemos que orientação é \textbf{positiva} se \(\det[v_1, \dotsc , v_{n}] > 0\) e \textbf{negativa} se \(det[v_1, \dotsc , v_{n}] < 0.\)
\end{tcolorbox}

\begin{def*}
	Dizemos que uma variedade (com bordo ou sem) diferenciável M é \textbf{orientável} se ela admite um Atlas \(\mathcal{A}\) tal que, para quaisquer parametrizações \((U_{\alpha }, \varphi_{\alpha }),\; (U_{\beta }, \varphi_{\beta })\) com \(\varphi_{\alpha }(U_{\alpha }) \cap \varphi _{\beta }(U_{\beta }) = \emptyset \), temos
	\[
		\det\biggl[\mathrm{d}(\varphi_{\beta }^{-1}\circ \varphi_{\alpha })_{q}\biggr] > 0,
	\]
	com o d de diferencial, para todo
	\[
		q\in \varphi_{\alpha }^{-1}(\varphi_{\alpha }(U_{\alpha })\cap \varphi_{\beta }(U_{\beta })).\; \square
	\]
\end{def*}
\subsection{Integrais de Formas}
Finalmente, armados com as definições e resultados construídos até aqui, podemos entrar no tópico das integrais em formas diferenciais. Para conseguirmos generalizar o que já sabemos, vamos relembrar o caso da integral de uma função f com imagem em \(\mathbb{R}\). Para encontrar a integral, costumamos utilizar a \textit{soma de Riemann}
\[
	S(f, P, \xi ) = \sum\limits_{\beta \in S(P)}^{} \omega (\xi (\beta ))(v_1(\beta ), \dotsc , v_n(\beta )).
\]
Sabendo que f é integrável e entendendo o significado do diferencial conforme estudamos aqui, isso significa que o limite
\[
	\lim_{| P |\to 0}S(f, P, \xi )= \int_{\mathbb{R}}^{}f \mathrm{dv}
\]
existe, e que, vendo ``dv'' pela ótica das formas e diferenciais, o limite acima é o mesmo que dizer que existe
\[
	\lim_{| P |\to 0}\sum\limits_{\beta \in S(P)}^{}\omega (\xi (\beta ))(v_1(\beta ), \dotsc , v_{n}(\beta )).
\]
Juntando tudo, introduzimos a notação do limite acima em termos de
\[
	\lim_{| P |\to 0}\sum\limits_{\beta \in S(P)}^{}\omega (\xi (\beta ))(v_1(\beta ), \dotsc , v_{n}(\beta )) = \int_{\mathbb{R}}^{}\omega
\]
\end{document}

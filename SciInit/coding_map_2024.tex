\documentclass{article}
 \usepackage{bookmark}
 \usepackage{amsmath}
 \usepackage{amsthm}
 \usepackage{amssymb}
 \usepackage{tikz}
 \usepackage{pgfplots}
 \usepackage[utf8]{inputenc}
 \usepackage{amsfonts}
 \usepackage[margin=2.5cm]{geometry}
 \usepackage{graphicx}
 \usepackage[export]{adjustbox}
 \usepackage{fancyhdr}
 \usepackage[portuguese]{babel}
 \usepackage{hyperref}
 \usepackage{multirow}
 \usepackage{lastpage}
 \usepackage{mathtools}
 \usepackage{newtxsf}
 \usepackage{subfiles}
 \usepackage[T1]{fontenc}
 \setcounter{section}{-1}

 \pagestyle{fancy}
 \fancyhf{}

 \pgfplotsset{compat = 1.18}

 \hypersetup{
     colorlinks,
     citecolor=black,
     filecolor=black,
     linkcolor=black,
     urlcolor=black
 }
 \newtheorem*{def*}{\underline{Definition}}
 \newtheorem*{theorem*}{\underline{Theorem}}
 \newtheorem*{lemma*}{\underline{Lemma}}
 \newtheorem*{prop*}{\underline{Proposition}}
 \newtheorem{example}{\underline{Example}}
 \newtheorem*{proof*}{\underline{Proof}}
 \newtheorem*{crl*}{\underline{Corollary}}
 \newtheorem{exr}{\underline{Exercise}}
 \renewcommand\qedsymbol{$\blacksquare$}

 \rfoot{Página \thepage \hspace{1pt} de \pageref{LastPage}}

\begin{document}
\begin{figure}[ht]
	\minipage{0.76\textwidth}
	\includegraphics[width=4cm]{../icmc.png}
	\hspace{7cm}
	\includegraphics[height=4.9cm,width=4cm]{../brasao_usp_cor.jpg}
	\endminipage
\end{figure}

\begin{center}
	\vspace{1cm}
	\LARGE
	UNIVERSIDADE DE SÃO PAULO

	\vspace{1.3cm}
	\LARGE
	INSTITUTO DE CIÊNCIAS MATEMÁTICAS E COMPUTACIONAIS - ICMC

	\vspace{1.7cm}
	\Large
	\textbf{Iterated Function Systems}
	\large\text{A probabilistic approach to fractals}

	\vspace{1.3cm}
	\large
	\textbf{Renan Wenzel e Bella Rocxane}

	\vspace{1.3cm}
	\large
	\textbf{Professor(a): Tiago Pereira da Silva}

	\vspace{1.3cm}
	\today
\end{center}

\newpage

\tableofcontents

\newpage
\section{What You Should Know}
Here, I'll briefly talk about cylinder sets, Hausdorff metric and such

\hypertarget{fixed_point_theorem}{Fixed Point Theorem Statement}
\newpage
\section{The Coding Map}
Our context here is to take consider the workspace as the probability space \((\Sigma_{k} , \mathcal{B}(\Sigma ), \mathbb{P})\). More explicitly,
the elements we'll work with are in \(\Sigma_{k} = \{1,\dotsc , k\}^{\mathbb{N}}\) - sequences of digits whose values are natural numbers between 1 and k -
as well as looking at a Borel \(\sigma-\)algebra \(\mathcal{B}(\Sigma )\) on \(\Sigma_{k} \), \textit{i.e.}, the \(\sigma-\)algebra generated by open sets,
and since the topology is generated by cylinder sets, they are themselves these open sets. The probability here will be taken as
\[
	\mathbb{P}([a_{0}, a_{1}, \dotsc , a_{l}]) = \prod\limits_{i=0}^{l}p_{a_{i}},\quad \sum\limits_{i=0}^{k}p_{i}=1.
\]
For instance, if \([a_{0}, a_{1}, \dotsc , a_{l}] = (1,2,0,3,4),\) then
\[
	\mathbb{P}([a_{0}, a_{1}, \dotsc , a_{l}]) = p_{1}\cdot p_{2}\cdot p_{0}\cdot p_{3}\cdot p_{4},
\]
where \(\sum\limits_{i=0}^{4}p_{i} = 1\). It should be noted that there is the possibility of changing \(\mathbb{P}\) to a finite measure on \(\mathcal{B}(\Sigma )\),
but one does not gain much from this, since it can be done by simply multiplying the values of \(\mu \) by their inverse.

It is also important to review some things about how functions and sequences interplay, such as i.i.d products, orbits and the shift operator:
\begin{def*}
	Let \(\Sigma_{k}=\{1,\dotsc ,k\}^{\mathbb{N}}\) and \(f_{1},\dotsc , f_{k}:M\rightarrow M\) be functions. The \textbf{independent and identically distributed product of functions }\(f_{1,\dotsc , f_{k}}\), or
	i.i.d product of functions in short, is the function
	\begin{align*}
		\varphi: & \mathbb{N}\times \Sigma_{k}\times M\rightarrow M                          \\
		         & (n, \omega, x)\mapsto f_{\omega_{n-1}}\circ\cdots\circ f_{\omega_{0}}(x),
	\end{align*}
	or equivalently,
	\[
		f_{\omega }^{n}(x)\coloneqq \varphi(n, \omega, x)=f_{\omega_{n-1}}\circ\cdots\circ f_{\omega_{0}}(x). \quad \square
	\]
\end{def*}
\begin{def*}
	Let X be a topological space and \(f:X\rightarrow X\) be a continuous function. We say that \textbf{a point x in X has a dense orbit} if, for every non-empty open set \(U\subseteq X\), there exists a natural number
	n such that \(f^{n}(x)\) belongs to U. \(\square\)
\end{def*}
In the case of our topological space, the open sets are cylinders, hence for sequence \(\omega\) in \(\Sigma_{k}\) to have a dense orbit means that for all cylinders \([a_{0},\dotsc ,a_{l}]\), there exists a natural number
n such that \(f^{n}(\omega)\) ends up inside that cylinder, where f is a function \(f:\Sigma_{k}\rightarrow \Sigma_{k}\).
\begin{def*}
	Consider the set \(\Sigma_{k}=\{1,\dotsc ,k\}^{\mathbb{N}}\). The \textbf{shift operator} \(\sigma:\Sigma_{k}\rightarrow \Sigma_{k}\) is the operator that takes a sequence and removes its first coordinate: for
	\(\omega = (\omega_{0}, \omega_{1}, \omega_{2}, \dotsc ) \), its action is to shift the sequence so that it starts on \(\omega_{1}\) instead of \(\omega_{0}\):
	\[
		\sigma(\omega)=\sigma((\omega_{0}, \omega_{1}, \omega_{2}, \dotsc ))=(\omega_{1}, \omega_{2}, \dotsc ).\quad \square
	\]
\end{def*}

We can now go to the first main theorem, where the \textit{coding map} will be defined:
\begin{theorem*}[The Coding Map]
	Let \(f_{1},\dotsc , f_{k}:M\rightarrow M\) be a family of contractions indexed according to the set \(\Sigma_{k}\), and \(\mathcal{K}\) be the Hutchinson Attractor. For all sequences
	\(\omega  \) in \(\mathcal{C}\), where \(\mathcal{C}\) is a cylinder set, the limit
	\[
		\pi(\omega )=\lim_{n\to \infty} f_{\omega_{0}}\circ f_{\omega_{1}}\circ\cdots\circ f_{\omega_{n}}
	\]
	exists and is independent of p. Moreover, \(\pi(\Sigma_{k})\) is exactly \(\mathcal{K}\).
\end{theorem*}
One way to think of this theorem is that no matter how a person may shuffle the indices of a sequence \(\omega \), composing \(f_{i}\)'s according to the numbers in it
will yield \(\mathcal{K}\) as a result, and, pointwise, no matter the point p of M where they are applied, the final composition will land inside \(\mathcal{K}\).

\begin{proof*}
	For starters, let \(p\in \mathcal{K}\). One thing to realise is: for each additional time some of the \(f_{i}\)'s is applied to \(\mathcal{K}\), it will contract ever more, so that
	\[
		\mathcal{K}\supseteq f_{\omega_{0}}\supseteq \dotsc \supseteq f_{\omega_{0}}\circ f_{\omega_{1}}\circ\cdots\circ f_{\omega_{n}}(\mathcal{K}) \supseteq f_{\omega_{0}}\circ f_{\omega_{1}}\circ\cdots\circ f_{\omega_{n+1}}(\mathcal{K})
	\]
	furthermore, the diameter keeps getting smaller, until it reaches value 0, meaning the set \(\mathcal{K}\) will contract so much that it becomes a single point
	when infinitely many iterations of f are made, and such point must be inside each of the previous iterations (you can't contract something away from the set because
	of the chains of subsets mentioned above), i.e.,
	\[
		\lim_{n\to \infty}f_{\omega_{0}}\circ \dotsc \circ f_{\omega_{n}}(\mathcal{K}) \subseteq \bigcap_{n\in \mathbb{N}}^{}f_{\omega_{0}}\circ f_{\omega_{1}}\circ\cdots\circ f_{\omega_{n}} (\mathcal{K})
	\]
	Using the fact that \(f_{i}(\mathcal{K})\) is closed for \(1\leq i\leq k\), the following is also true:
	\[
		\lim_{n\to \infty} f_{\omega_{0}}\circ f_{\omega_{1}}\circ\cdots\circ f_{\omega_{n}}(p)\in \bigcap_{n\in \mathbb{N}}^{}f_{\omega_{0}}\circ f_{\omega_{1}}\circ\cdots\circ f_{\omega_{n}} (\mathcal{K}),
	\]
	which proves the existence of the limit and its independence of p.

	However, this is does not suffice - the theorem is stated for points that are also outside \(\mathcal{K}\), and that will be the next step; take \(q\not\in \mathcal{K}\). Using the \hyperlink{fixed_point_theorem}{\textit{Fixed Point Theorem}},
	we know that applying the family of contractions \(\{f_{i}\}_{i=1}^{k}\) infinitely many times will inevitably lead to the same point, so that \(\{f_{\omega_{0}}\circ \dotsc \circ f_{\omega_{n}}(q)\}_{n}\) and \(\{f_{\omega_{0}}\circ \dotsc \circ
	f_{\omega_{n}}(p)\}_{n}\) converge to the same point, and by the argument made for p, that point is in \(\mathcal{K}\).

	Now, what remains to prove is that \(\pi(\Sigma_{k})=\mathcal{K} \), which is the same as proving that when you collect all points resulted from applying \(\pi\) to individual sequences \(\omega\in \Sigma_{k}\), and put them in a set,
	that set coincides exactly with \(\mathcal{K}\). This also allows us to think of \(\mathcal{K}\) as ``the set of all shuffles of \(f_{\omega_{0}}\circ f_{\omega_{1}}\circ\cdots\circ f_{\omega_{n}}(p)\)''. To reach this goal, the first step
	will be to prove the continuity of \(\pi \).

	\textbf{\underline{Claim}:} \(\pi \) is a continuous mapping. In other words, we mean to show that for all \(\varepsilon >0\), there exists a cylinder (the open sets in the preimage) such that
	\[
		\nu\in \mathcal{C} \Rightarrow d(\pi(\nu), \pi(\omega ))<\varepsilon .
	\]
	We already know that \(\pi (\omega )\) in in \(\mathcal{K}\) for all sequences in \(\Sigma_{k}\). Moreover, because of \(\mathcal{K}\) being a compact set, it is bounded, and thus its diameter is finite, let's say
	\[
		\mathrm{diam}(\mathcal{K})=\alpha < \infty.
	\]
	Let \(\lambda \) be the greatest factor of the family of contractions, so that
	\[
		\lambda =\max_{1\leq i\leq k}\{\lambda_{i}\} < 1 \quad\&\quad \lim_{n\to \infty}\lambda^{n}=0.
	\]
	Hence, there exists a number \(N\coloneqq N(\varepsilon )\), positive, such that
	\[
		\lambda^{N}\cdot \alpha < \varepsilon .
	\]
	That will be the foundation for a candidate as our cylinder, \textit{i.e.}, \(\mathcal{C}(\varepsilon ) = [\omega_{0}, \dotsc , \omega_{N}]\). For any sequence \(\omega \) inside \(\mathcal{C}(\varepsilon )\), then,
	\begin{align*}
		\pi (\omega ) & = f_{\omega_{0}}\circ f_{\omega_{1}}\circ\cdots\circ f_{\omega_{N}}\circ f_{\omega_{N+1}}\circ \dotsc   \\
		              & = f_{\omega_{0}}\circ f_{\omega_{1}}\circ\cdots\circ f_{\omega_{N}}(\circ f_{\omega_{N+1}}\circ \dotsc) \\
		              & = f_{\omega_{0}}\circ f_{\omega_{1}}\circ\cdots\circ f_{\omega_{N}}(\pi(\sigma^{N}(\omega ))).
	\end{align*}
	Thus,
	\begin{align*}
		d(\pi (\nu), \pi (\omega )) & = d(\lim_{n\to \infty}f_{\nu_{0}}\circ f_{\nu_{1}}\circ\cdots\circ f_{\nu_{n}}(p), \lim_{n\to \infty}f_{\omega_{0}}\circ f_{\omega_{1}}\circ\cdots\circ f_{\omega_{n}}(p))     \\
		                            & = d(f_{\nu{0}}\circ f_{\nu{1}}\circ\cdots\circ f_{\nu{N}}(\pi(\sigma^{N}(\nu ))), f_{\omega_{0}}\circ f_{\omega_{1}}\circ\cdots\circ f_{\omega_{N}}(\pi(\sigma^{N}(\omega )))) \\
		                            & \leq \lambda^{N+1}d(\pi(\sigma^{N}(\nu)), \pi(\sigma^{N}(\omega)))                                                                                                             \\
		                            & <\lambda ^{N+1}\alpha<\varepsilon ,\quad \forall \nu\in \mathcal{C}. \blacktriangle
	\end{align*}

	Having proven continuity, we still have to show that \(\pi \) is a fixed point of the Hutchinson Operator, so that by unicity, it will be exactly \(\mathcal{K}\). First, we prove that \(\{f_{i}\}\) acts on \(\pi (\Sigma_{k})\)
	by also contracting it: using the fact that \(\pi \) is continuous, we can just append \(f_{i}\) to the limit which defines \(\pi \), such that
	\begin{align*}
		f_{i}(\pi(\Sigma_{k})) & = f_{i}(\lim_{n\to \infty}f_{\omega_{0}}\circ f_{\omega_{1}}\circ\cdots\circ f_{\omega_{n}}(p))       \\
		                       & = \lim_{n\to \infty}(f_{i}\circ f_{\omega_{0}}\circ f_{\omega_{1}}\circ\cdots\circ f_{\omega_{n}}(p)) \\
		                       & = \pi(i\omega ),\quad i\omega = (i, \omega_{0}, \dotsc , \omega_{n}, \dotsc )
	\end{align*}
	which shows that \(i\omega \) is in \(\Sigma_{k} \), and thus \(f_{i}(\pi(\Sigma_{k}))\subseteq \pi(\Sigma_{k})\).

	Finally, by continuity of \(\pi \) and compactness of \(\Sigma_{k} , \; \pi(\Sigma_{k})\) is compact, hence
	\[
		\mathcal{B}(\pi(\Sigma_{k}))=\bigcup_{i=1}^{k}f_{i}(\pi(\Sigma_{k}))=\bigcup_{i=1}^{k}\{\pi(i\omega ):\;\omega \in \Sigma_{k}\}
	\]
	defines an open cover of \(\pi(\Sigma_{k})\) while simultaneously being a subset of it. Therefore,
	\[
		\mathcal{B}(\pi(\Sigma_{k}))=\pi(\Sigma ) \Rightarrow \pi(\Sigma_{k})=\mathcal{K}.\quad \text{\qedsymbol}
	\]
\end{proof*}
\section{Usual-order Iterations}
However, there is a problem: as it is seen, the theorem above reverses the order of composition - usually, one would start from the 0-th term of te sequence and work upwards from that, for instance \(f_{\omega_{n}}\circ \dotsc \circ f_{\omega_{1}}\circ f_{\omega_{0}}(p)\),
while the theorem is stated with \(f_{\omega_{0}}\circ f_{\omega_{1}}\circ\cdots\circ f_{\omega_{n}}(p)\). What happens with the first case? Is there a version for it?

As a matter of facts, there is another version, stated for i.i.d products of functions and slightly different, but which still allows for the usual order of composition of functions. We will show the proof now, but for it to be truly complete,
we'll need to make a brief argument of authority, namely that the set of sequences with dense orbit under the shift operator has measure 1. It will be proven later, but for now, trust us! As for the theorem,

\begin{theorem*}
	Let \(\varphi \) be an i.i.d product of contractions \(f_{1},\dotsc , f_{k}:M\rightarrow M\), and \(\mathcal{K}\) be the corresponding Hutchinson Attractor. Given a point p in M, we have
	\[
		\mathcal{K}=\lim_{l\to \infty}\overline{\{f_{\omega }^{n}(x)\}: n\geq l}
	\]
	for \(\mathbb{P}\)-almost every \(\omega \) (meaning for all \(\omega \) outside of a set with probability 0).
\end{theorem*}
In other words, the tail of an i.i.d product of contractions accumulates on the Hutchinson Attractor for essentially all possible sequences, except for a few.

\section*{Appendix A - \(\sigma \)-Algebras, Measures, and Borel \(\sigma \)-Algebras}

\section*{Appendix B - Cylinder Sets}

\end{document}

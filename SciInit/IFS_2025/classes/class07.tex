\documentclass[../stationary_ifs.tex]{subfiles}
\begin{document}
\section{Class 07 - November 28th, 2025}
\subsection{Motivações}
\begin{itemize}
	\item a
\end{itemize}
\subsection{a}
Given the quadruplet \((\Omega , \mathcal{F}, \mathbb{P}, \theta )\), let us fix a measure-preserving transformation \(\theta :\Omega \rightarrow \Omega  \) and consider a contraction
\[
	f_{\omega }(x) = \lambda x + b(\omega ),\quad f_{\omega }^{n}(x) = f_{\theta^{n-1}(\omega )}\circ \dotsc \circ f_{\omega }(x), \quad \lambda \in (0, 1)
\]
where b is just a coefficient depending on \(\omega \) known as the \textbf{displacement}; what can we say about stationary measures? If we have a control on the displacement, for instance \(\log^{+}{b(\omega )}\in L^{1}(\mathbb{P})\) which is a control on the tail of the variable (i.e. in how it spreads out), then
the limit
\[
	\pi (\omega ) = \lim_{n\to \infty}f_{\theta^{-1}(\omega )}\circ \cdots\circ f_{\theta^{-n}(\omega )}(x)
\]
exists and does not depend on x. Furthermore, we know the distribution of the map, since \(\pi_{*}\mathbb{P}\) is an attracting measure in the sense
\[
	f_{\omega_{n-1}}\circ \cdots\circ f_{\omega_{0}}(x)\stackrel{*}\rightarrow \pi_{*}\mathbb{P},
\]
which is stationary. What we want to do is to propose a way of discussing something similar to that as follow:
\begin{theorem*}
	Let \((\Omega , \mathcal{F}, \mathbb{P}, \theta )\) be a quadruplet where \(\theta \) is a measure-preserving transformation; consider a family of contractions \(\{f_{\omega }:M\rightarrow M\}\), so that
	\[
		d(f_{\omega }(x), f_{\omega }(y))\leq \lambda d(x, y),\quad \forall x, y\in M,\; \omega \in \Omega .
	\]
	Assume that there is an \(x_{0}\) for which
	\[
		\int_{}d(x_{0}, f_{\omega }(x_{0})) d \mathbb{P}(\omega ) < +\infty,
	\]
	characterizing bounded displacement. Then, the map \(F:\Omega \times M\rightarrow \Omega \times M\) defined by
	\[
		F(\omega , x) = (\theta (\omega ), f_{\omega }(x))
	\]
	has an invariant graph \(\pi :\Omega \rightarrow M\) such that
	\[
		F_{*}^{n}\mu  \stackrel{*}\rightarrow \varphi_{*}\mathbb{P}
	\]
	where
	\[
		\varphi (\omega ) = (\omega , \pi (\omega )).
	\]
\end{theorem*}
The interest in studying these maps arise in the fact that while we were previously studying random composition, now we have a single map \(F(\omega , x)\) capturing those random compositions but with extra information, and by projecting \(F\) onto its second coordinate, we can study them once again.

On \(\Omega \times M\), we have three collections with respect to the measures that impose conditions on the moments of their statistics:
\begin{align*}
	 & \tikz[baseline=-0.5ex]\draw[black, fill=black, radius=1.5pt](0, 0)circle; \; \mathcal{P}(\Omega \times M) = \text{ probabilities on }\Omega \times M;                                                                                \\
	 & \tikz[baseline=-0.5ex]\draw[black, fill=black, radius=1.5pt](0, 0)circle; \; \mathcal{P}_{\mathbb{P}}(\Omega \times M) = \{\mu \in \mathcal{P}(\Omega \times M),\; \pi_{1*}\mu  = \mathbb{P}:\; \pi_{1}(\omega , x) = \omega \};\&   \\
	 & \tikz[baseline=-0.5ex]\draw[black, fill=black, radius=1.5pt](0, 0)circle; \; \mathcal{P}_{\mathbb{P}}^{1}(\Omega \times M) = \{\mu \in \mathcal{P}(\Omega \times M),\; \int_{}^{}d(x, x_{0}) \mathrm{d}\mu (\omega , x) < +\infty\};
\end{align*}
The wersseinstein metric
\end{document}




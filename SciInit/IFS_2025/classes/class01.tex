\documentclass[../stationary_ifs.tex]{subfiles}
\begin{document}
\section{Class 01 - September 26th, 2025 (Edgar Matias)}
\subsection{Motivations}
\begin{itemize}
	\item Random Orbits;
	\item Perron-Frobenius and Koopman Operators;
	\item Stationary Measures.
\end{itemize}
\subsection{Random Orbits and The Chaos Game}
This talk started using the chaos game iterated system as example; it is the following: imagine three points in the plane, all of them fixed,
and take any orbit point \(E_{0}\) in the plane. We are going to construct a random sequence by imagining a dice of three faces (A, B, C) and if, for instance, you get
C, then the point \(E_1\) will be the middle point of the line connecting points C and \(E_{0}\):
\[
	K =f_1(K)\cup f_2(K)\cup f_3(K),\quad f_{i} = A_{i}+\frac{1}{2}(x-A_{i}),
\]
where f is a contraction and K is compact.

The result from this construction, after some time has passed, \textit{i.e.}, after some iterations, the image formed will be the Sierpinski Triangle, built from Chaos Game.
One could start at any point, anad the result would still be the same.

By defining the function
\[
	\mathcal{B}(A) = f_1(A)\cup f_2(A)\cup f_3(A),
\]
or, more generally,
\[
	\mathcal{B}(A) = \bigcup_{k=1}^{n}f_{k}(A)
\]
acting on the space of nonempty compacts subsets of the plane, it follows that the example above is as example of a fixed point -- the unique fixed point -- of such operator,
which receives the name of \textbf{Hutchinson Attractor}.

We'll approach this problem from a probabilistic point of view, and what will follow is that even if the dice is weighted, the idea is that the contractions will still converge
to the attractor. For that, our main theme will be \textit{stationary measures}.

\begin{def*}
	Given a random sequence \(\{E_{n}\}\) constructed from the chaos game, we define a \textbf{probability measure} by
	\[
		\mu_{n}(A) = \mathbb{P}(E_{n}\in A).
	\]
\end{def*}
It can be show that \(\mu_{n}\) in fact converges to a probability measure \(\mu \), and what we had seen on the previous images are ``visualizations'' of the limit of this measure
for each given weight. Moreover, if \(E_{0}\) has a certain distribution \(\mu \), then \(E_{n}\) also has distribution \(\mu \) for all n and, for that reason, \(\mu \) is called stationary.

\begin{example}
	A very famous example of the chaos game is the Bansley Fern.
	\begin{align*}
		 & f_1(x) = \begin{pmatrix}
			            0 & 0    \\
			            0 & 0.16
		            \end{pmatrix}\begin{pmatrix}
			                         x \\
			                         y
		                         \end{pmatrix}                 \\
		 & f_2(x) = \begin{pmatrix}
			            0.85  & 0.04 \\
			            -0.04 & 0.85
		            \end{pmatrix}\begin{pmatrix}
			                         x \\
			                         y
		                         \end{pmatrix} + \begin{pmatrix}
			                                         0 \\
			                                         1.6
		                                         \end{pmatrix} \\
		 & f_3(x) = \begin{pmatrix}
			            0.20 & -0.26 \\
			            0.23 & 0.22
		            \end{pmatrix}\begin{pmatrix}
			                         x \\
			                         y
		                         \end{pmatrix} + \begin{pmatrix}
			                                         0 \\
			                                         1.6
		                                         \end{pmatrix} \\
		 & f_4(x) = \begin{pmatrix}
			            -0.15 & 0.28 \\
			            0.26  & 0.24
		            \end{pmatrix}\begin{pmatrix}
			                         x \\
			                         y
		                         \end{pmatrix} + \begin{pmatrix}
			                                         0 \\
			                                         0.44
		                                         \end{pmatrix} \\
	\end{align*}
	The way Barnsley found out which functions would work is by trial and error, specifically by taking the fern shape and start looking for a linear transformation that will map a part of the fern
	onto another untill you eventually find a family that covers the whole set.

	It is also a nice example to notice how one can pick the weights of the four functions to be either the same or not, and it will still coverge to the fern after some number of iterations.
\end{example}

\subsection{Iterated Random Functions}

Fix a probabilty space \((\Omega , \mathcal{O}, \mathbb{P})\) and let \(\{X_{n}\}\) be a sequence of random iid variables with values on another probabilty space \((E, \mathcal{E}, \nu)\), say
\(E = \{1, \dotsc , k\}\) or \(E = [0, 1]\).

\begin{def*}
	Given a family of functions \(f_{i}:M\rightarrow M,\; i\in E\) and a random variable \(Z_{0}\in \Omega \) with values in M independent of \(\{X_{n}\}\), we define the \textbf{random i.i.d iteration of functions} by
	\[
		Z_{n} = f_{X_{n-1}}\circ \dotsc \circ f_{X_{0}}(Z_{0}), \quad n\geq 1
	\]
\end{def*}
An important property of these kinds of iterations is that they form a \textbf{Markov Chain}, hence it only depends on the immediately before step:
\[
	\mathbb{P}(X_{n+1}\in A| X_{0}, \dotsc , X_{n}) = \mathbb{P}(X_{n+1}\in A| X_{n})
\]

For the case where \(Z_{0} = x\) and \(E = \{1, \dotsc , k\}\), we want to describe if, after going through our process we started at some point x, what is the probability that x will end at set A? Well, that only happens if for some index in the family of functions,
\(x\in f_{i}^{-1}(A)\), and since each function \(f_{i}\) is chosen with a certain probability \(p_{i}\), we only add the weights for which x is in the preimage of A under i:
\[
	\mathbb{P}(Z_{0}\in A) = \sum\limits_{i=1}^{k}p_{i}\chi_{A}(f_{i}(x)), \quad p_{i} = \mathbb{P}(X_{0} = i).
\]
This expression defines exactly the probability of a point x transitioning to the set A after the experiment.

In the more general case where E is some space and \(\nu \) is the common distribution of the i.i.d sequence \(\{X_{n}\}\), the transition probability is given by
\[
	p(x, A) = \int_{}\chi_{A}(f_{i}(x)) d\nu_{i},
\]
and if \(\mu_{n}\) is the distribution of \(Z_{n}\), one can show that
\[
	\mu_{n+1}(A) = \int_{}p(x, A) d\mu_{n}.
\]
In other words, \(\mu_{n+1} = T \mu_{n}\), where T denotes the \textbf{Perron-Frobenius Operator} acting on the probability measures space \(\mathcal{M}_1(X)\) and defined by
\[
	T\mu (A) = \int_{}p(x, A) d\mu_{}.
\]
Because we have a Markov Chain here, thus, it suffices to look only at the action of the Perron-Frobenius Operator. Hence, we define a few ideas based on that:
\begin{def*}
	A fixed point of the Perron-Frobenius Operator is called a \textbf{stationary measure}. \(\square\)
\end{def*}
\begin{def*}
	We say T is \textbf{assintotically stable} if there is some stationary measure \(\mu \) such taht, por any initial probability measure \(\eta \), one has
	\[
		T^{n}\nu \substack{ \\ \longrightarrow \\ n\to \infty} \mu . \square
	\]
\end{def*}
The above convergence can be understood differently based on the topology being adopted, such as the weak-* topology or the convergence with respect to the Wasserstein Metric.

On the topic of operators, another one that shall appear with some frequence is the \hypertarget{koopman_operator}{\textbf{Koopman Operator}} acting on the space of bounded measurable functions or
on bounded continuous functions, given by
\[
	P\varphi (x)=\int_{}\varphi(f_{i}(x)) d\nu_{}.
\]
\begin{tcolorbox}[
		skin=enhanced,
		title=Observação,
		fonttitle=\bfseries,
		colframe=black,
		colbacktitle=cyan!75!white,
		colback=cyan!15,
		colbacklower=black,
		coltitle=black,
		drop fuzzy shadow,
		%drop large lifted shadow
	]
	Notice how if \(Z_{n}^{x} = f_{X_{n-1}}\circ \cdots\circ f_{X_{0}}(x)\), then the action of repeated iterations of the \hyperlink{koopman_operator}{\textit{Koopman Operator}} is exactly the expected value of \(\varphi \) applied on the
	i.i.d iteration of \(\varphi \), i.e.,
	\[
		P^{n}(\varphi (x)) = \mathbb{E}(\varphi(Z_{n}^{x})),
	\]
	such that one can prove the following relation between the Koopman Operator and the Perron-Frobenius Operator T:
	\[
		\int_{}^{}\varphi  \mathrm{d(T^{n}\mu )} = \int_{}^{}P^{n}\varphi  \mathrm{d}\mu
	\]
	for any bounded measurable functions and any \(n\geq 1\).

	The proof of the \hyperlink{letac_principle}{\textit{Letac Principle}} is a direct consequence of this relationship together with the fact that both \((X_{0}, \dotsc , X_{n-1})\) and \((X_{n-1}, \dotsc , X_{0})\) have the same distribution
	as long as \(\{X_{n}\}\) is an i.i.d. sequence.
\end{tcolorbox}

\end{document}

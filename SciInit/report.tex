\documentclass[12pt]{article}
 \usepackage{bookmark}
 \usepackage{amsmath}
 \usepackage{amsthm}
 \usepackage{amssymb}
 \usepackage{tikz}
 \usepackage{pgfplots}
 \usepackage[utf8]{inputenc}
 \usepackage{amsfonts}
 \usepackage{geometry}
 \usepackage{graphicx}
 \usepackage{graphics}
 \usepackage[export]{adjustbox}
 \usepackage{fancyhdr}
 \usepackage[portuguese]{babel}
 \usepackage{hyperref}
 \usepackage{multirow}
 \usepackage{lastpage}
 \usepackage{mathtools}
 \usepackage{newtxsf}
 \usepackage{subfiles}
 \usepackage{flafter}
 \usepackage{float}
 \usepackage{accents}
 \usepackage[T1]{fontenc}
 \setcounter{section}{-1}

 \pagestyle{fancy}
 \fancyhf{}

 \pgfplotsset{compat = 1.18}

 \hypersetup{
     colorlinks,
     citecolor=black,
     filecolor=black,
     linkcolor=black,
     urlcolor=black
 }
 \newtheorem*{def*}{\underline{Definição}}
 \newtheorem*{theorem*}{\underline{Teorema}}
 \newtheorem*{lemma*}{\underline{Lema}}
 \newtheorem*{prop*}{\underline{Proposição}}
 \newtheorem{example}{\underline{Exemplo}}
 \theoremstyle{definition}
 \newtheorem*{proof*}{\underline{Prova}}
 \newtheorem*{crl*}{\underline{Corolário}}
 \newtheorem{exr}{\underline{Exercício}}
 \renewcommand\qedsymbol{$\blacksquare$}

 \rfoot{Página \thepage \hspace{1pt} de \pageref{LastPage}}

 \geometry{a4paper, left=3cm, top=3cm, right=3cm, bottom=3cm}

\begin{document}
\begin{figure}[ht]
	\minipage{0.76\textwidth}
	\includegraphics[width=4cm]{../icmc.png}
	\hspace{7cm}
	\includegraphics[height=4.9cm,width=4cm]{../brasao_usp_cor.jpg}
	\endminipage
\end{figure}

\begin{center}
	\vspace{1cm}
	\LARGE
	UNIVERSIDADE DE SÃO PAULO

	\vspace{1.3cm}
	\LARGE
	INSTITUTO DE CIÊNCIAS MATEMÁTICAS E COMPUTACIONAIS - ICMC

	\vspace{1.7cm}
	\Large
	\textbf{Construção de Conjuntos Fractais}

	Relatório Parcial de Bolsa PUB - Setembro a Fevereiro.

	\vspace{1.3cm}
	\large
	\textbf{Renan Wenzel - 11169472}

	\textbf{E-mail: 11169472@icmc.usp.br}

	\vspace{1.3cm}
	\large
	\textbf{Professor: Tiago Pereira}

	\textbf{E-mail: tiago@icmc.usp.br}

	\vspace{1.3cm}
	\today
\end{center}

\hrule
\vspace{1.3cm}

\section*{ATIVIDADES REALIZADAS}

\paragraph{}Tendo em vista o desejo original de abordar a construção de espaços auto-similares, uma formalização do conceito de fractais,
através da perspectiva dos sistemas dinâmicos, foram apresentadas de início os conceitos matemátios necessárias para trabalhar com o assunto,
tais como o espaço em que serão feitos os estudos, sua estrutura métrica e completude, a topologia dos conjuntos-cilindros e noções de probabilidade que aparecerão.
O estudo começou provando a existência do Atrator de Barnsley-Hutchinson, construído a partir da união das imagens de contrações no espaço dos conjuntos compactos não-vazios de um espaço métrico M, denotado por \(H(M)\), e do operador de Barnsley-Hutchinson, denotado por \(\mathcal{B}:H(M)\rightarrow H(M)\), é dado por
\[
	\mathcal{B}(A)=\bigcup_{i=1}^{k}f_{i}(A), \quad f_{i}:M\rightarrow M,\: 1\leq i\leq k,
\]
em que cada \(\{f_{i}\}\) é uma contração. Utilizando o fato delas serem contrações num espaço métrico completo, aplicamos o Teorema do Ponto-Fixo de Banach para provar a existência
de um ponto fixo para \(\mathcal{B}\), o qual denotamos por \(\mathcal{K}\), que é justamente o Atrator de Barnsley-Hutchinson.

Após determinada a existência desse ponto fixo, passamos a analisar de quais formas ele pode ser obtido e computado. A primeira delas foi por meio do teorema a seguir, conhecido como \textit{\textbf{The Coding Map}}, o primeiro
grande resultado a ser aprofundado:

\begin{theorem*}[The Coding Map]
	Considere uma família de contrações \(\{f_{i}\}_{i=1}^{k}\) em M e seja \(\mathcal{K}\) o Atrator de Barnsley-Hutchinson correspondente. Dado um ponto p
	em M, vale que, para qualquer sequência \(\omega \) em um conjunto-cilindro de \(\Sigma_{k}=\{1,\dotsc ,k\}^{\mathbb{N}}\),
	\[
		\pi(\omega )\coloneqq \lim_{n\to \infty}f_{\omega_{0}}\circ \dotsc \circ f_{\omega_{n}}(p)
	\]
	existe e não depende de p. Além disso, \(\pi \bigl(\Sigma_{k}\bigr)\) é exatamente \(\mathcal{K}\).
\end{theorem*}
\begin{proof*}
	Comece tomando \(p\in \mathcal{K}\) e perceba que, toda vez que uma das \(f_{i}\) é aplicada a \(\mathcal{K}\), ela contrai \(\mathcal{K}\), diminuindo seu diâmetro, até que ele
	chegue em 0, ou seja, \(\mathcal{K}\) torna-se contraído a um único ponto, ponto este que deve estar dentro de cada um dos \(\mathcal{K}\)'s contraídos.
	Além disso, como \(\mathcal{K}\) é fechado,
	\[
		\lim_{n\to \infty} f_{\omega_{0}}\circ f_{\omega_{1}}\circ\cdots\circ f_{\omega_{n}}(p)\in \bigcap_{n\in \mathbb{N}}^{}f_{\omega_{0}}\circ f_{\omega_{1}}\circ\cdots\circ f_{\omega_{n}} (\mathcal{K}),
	\]
	provando que o limite existe e não depende de p.

	Ainda devemos provar para pontos \(q\) for de \(\mathcal{K}\), e, para isso, basta aplicar infinitamente a família de contrações
	\(\{f_{i}\}_{i=1}^{k}\) tal que \(\{f_{\omega_{0}}\circ \dotsc \circ f_{\omega_{n}}(q)\}_{n}\) e \(\{f_{\omega_{0}}\circ \dotsc \circ
	f_{\omega_{n}}(p)\}_{n}\) convergem para o mesmo ponto, que sabemos, para o caso de p, estar dentro de \(\mathcal{K}\). A ideia aqui era de que as contrações vão aproximando as imagens desse ponto.
	Pela unicidade do limite, o limite em q também deve estar.

	Precisamos, por fim, provar que \(\pi(\Sigma_{k})=\mathcal{K} \), que pode ser entendido como uma forma de visualizar \(\mathcal{K}\) como
	``O conjunto de todos os embaralhamentos de \(f_{\omega_{0}}\circ f_{\omega_{1}}\circ\cdots\circ f_{\omega_{n}}(p)\)''. Começamos isso provando a continuidade de \(\pi \),
	e é importante entender o que é contínuo neste contexto: provamos que, para cada \(\varepsilon >0\), deve existir um conjunto-cilindro tal que
	\[
		\nu\in \mathcal{C} \Rightarrow d(\pi(\nu), \pi(\omega ))<\varepsilon.
	\]
	Sabendo que \(\pi (\omega )\) pertence a \(\mathcal{K}\) para todas as sequências em \(\Sigma_{k}\) e por conta da compacidade, segue que ele tem diâmetro finito, que podemos denotar por \(\alpha = \mathrm{diam}(\mathcal{K})\).

	Tomando \(\lambda \) como o maior fator da família de contrações,
	\[
		\lambda =\max_{1\leq i\leq k}\{\lambda_{i}\} < 1 \quad\&\quad \lim_{n\to \infty}\lambda^{n}=0.
	\]
	o que significa que existe um \(N\coloneqq N(\varepsilon )\) positivo para o qual		\(\lambda^{N}\cdot \alpha < \varepsilon\). Aqui, \(\sigma \) é o chamado \textit{\textbf{Shift Operator}},
	presente em um dos estudos ao longo do semestre.

	Com base nisso, construiremos nosso conjunto-cilindro \(\mathcal{C}(\varepsilon ) = [\omega_{0}, \dotsc , \omega_{N}]\).
	Para quaisquer sequências \(\omega \) dentro do conjunto-cilindro \(\mathcal{C}(\varepsilon )\), podemos chegar em
	\[
		\pi (\omega ) = f_{\omega_{0}}\circ f_{\omega_{1}}\circ\cdots\circ f_{\omega_{N}}(\pi(\sigma^{N}(\omega ))).
	\]
	e, consequentemente,
	\begin{align*}
		d(\pi (\nu), \pi (\omega )) & = d(f_{\nu{0}}\circ f_{\nu{1}}\circ\cdots\circ f_{\nu{N}}(\pi(\sigma^{N}(\nu ))), f_{\omega_{0}}\circ f_{\omega_{1}}\circ\cdots\circ f_{\omega_{N}}(\pi(\sigma^{N}(\omega )))) \\
		                            & \leq \lambda^{N+1}d(\pi(\sigma^{N}(\nu)), \pi(\sigma^{N}(\omega)))<\varepsilon ,\quad \forall \nu\in \mathcal{C}.
	\end{align*}

	Podemos, agora, provar que \(\pi \) é um ponto-fixo do Operador de Barnsley-Hutchinson, que, por unicidade, deverá coincidir com \(\mathcal{K}\).
	Esta etapa começa provando que a contração também contrai \(\pi \bigl(\Sigma_{k} \bigr)\), que pode ser feito usando a continuidade de \(\pi \)
	para realizar uma troca de ordem de limites:
	\begin{align*}
		f_{i}(\pi(p)) & = f_{i}(\lim_{n\to \infty}f_{\omega_{0}}\circ f_{\omega_{1}}\circ\cdots\circ f_{\omega_{n}}(p))       \\
		              & = \lim_{n\to \infty}(f_{i}\circ f_{\omega_{0}}\circ f_{\omega_{1}}\circ\cdots\circ f_{\omega_{n}}(p)) \\
		              & = \pi(i\omega ),\quad i\omega = (i, \omega_{0}, \dotsc , \omega_{n}, \dotsc ),
	\end{align*}
	ou seja, \(i\omega \) é um elemento de \(\Sigma_{k} \), e, logo, \(f_{i}(\pi(\Sigma_{k}))\subseteq \pi(\Sigma_{k})\).

	Por fim, usando a continuidade de \(\pi \) e a compacidade de \(\Sigma_{k} , \; \pi(\Sigma_{k})\) deve ser compacto, o que significa que
	\[
		\mathcal{B}(\pi(\Sigma_{k}))=\bigcup_{i=1}^{k}f_{i}(\pi(\Sigma_{k}))=\bigcup_{i=1}^{k}\{\pi(i\omega ):\;\omega \in \Sigma_{k}\}
	\]
	é uma cobertura aberta de \(\pi(\Sigma_{k})\) ao mesmo tempo que é um subconjunto dele. Portanto,
	\[
		\mathcal{B}(\pi(\Sigma_{k}))=\pi(\Sigma ) \Rightarrow \pi(\Sigma_{k})=\mathcal{K}.\quad \text{\qedsymbol}
	\]
\end{proof*}

O passo seguinte, que ainda está sendo desenvolvido, tem relação com um detalhe na ordem em que as funções são compostas no teorema - se fosse para seguir a ordem
usual de composição de funções, teríamos
\[
	f_{\omega_{0}}\rightarrow f_{\omega_{1}}\circ f_{\omega_{0}}\rightarrow \dotsc \rightarrow f_{\omega_{n}}\circ \dotsc \circ f_{\omega_{0}},
\]
mas o resultado faz ao contrário. Por conta disso, a etapa seguinte será explorar sob quais condições é possível obter o Atrator de Barnsley-Hutchinson percorrendo a ordem normal
de composição de funções. Já temos alguns planejamentos para isso, é esperado que estudemos teoria da medida, derivada de Radon-Nikodym para a continuidade de medidas, uma quantidade de
Teoria Ergódica e, eventualmente, tentaremos atacar o teorema
\begin{theorem*}
	Seja \(\varphi \) um produto independente e identicamente distribuído de contrações \(f_{1},\dotsc f_{k}:M\rightarrow M\) e \(\mathcal{K}\) o Atrator de Barnsley-Hutchinson correspondente. Dado um
	ponto p em M, temos
	\[
		\mathcal{K}=\lim_{l\to \infty}\overline{\{f_{\omega }^{n}(x): n\geq l\}}
	\]
	para \(\mathbb{P}\)-quase todo \(\omega \).
\end{theorem*}
o que já sabemos sobre a probabilidade \(\mathbb{P}\) será parte de entender quais são esses \(\omega \) que funcionam no teorema e, talvez tão importante quanto isso,
quais são os \(\omega \)'s que quebram ele.

Com base nisso, vamos ver quando pode acontecer desse teorema quebrar: 
\begin{example}
  No teorema acima, considere \(M=[0,1]\) com a métrica induzida de \(\mathbb{R}\) e as funções \(f_{i}\) como as funções de Cantor, dadas por 
    \[
      f_{i}(x) = \left\{\begin{array}{ll}
          \frac{x}{3},&\quad i=0\\ 
          \frac{(2+x)}{3},&\quad i=1
        \end{array}\right.,
    \]
    com nosso espaço \(\{0,1\}^{\mathbb{N}}\) como lugar onde \(\omega \) está.

    Em particular, suponha que \(\omega =(0,0,0,\dotsc )\); desta forma, dado l natural, 
   \begin{align*}
     f_{\omega }^{n}(x)=f_{\omega_{n}}\circ \dotsc \circ f_{\omega_{l+1}}(x)\circ f_{\omega_{l}}(x)&=f_{0}\circ \dotsc \circ f_{0}\biggl(\frac{x}{3}\biggr)^{l}\\ 
                                                                                                   &\vdots
                                                                                                   &=f_{0}\biggl(\frac{x}{3}\biggr)^{n}\\ 
                                                                                                   &= \frac{x^{n}}{3^{n}}.
   \end{align*}
   Conforme l tende a infinito -- e, por conta de tomarmos \(n\geq l\), o próprio n -- a composição acima tenderá a 0: 
     \[
       \lim_{n\to \infty}f_{\omega}^{n} = \lim_{n\to \infty}\frac{x^{n}}{3^{n}}= 0.
     \]
     Sendo assim, o único ponto de acumulação do conjunto 
       \[
         \{f_{\omega }^{n}(x):\; n\geq l\}
       \]
       é o 0. Logo, 
         \[
           \lim_{l\to \infty}\overline{\{f_{\omega }^{n}(x):\; n\geq l\}}=\{0\},
         \]
         que não é o conjunto de Cantor. 
\end{example}
Uma forma de entender este exemplo é que, se o único caminho tomado ao caminhar pelos pontos do conjunto de Cantor for o da esquerda (analogamente para o da direita), não tem como formá-lo -- chegará a um conjunto unitário composto pelo 0 (ou pelo 1 pela direita). Com isso, um tipo de \(\omega \) que não permite afirmar o teorema acima é quando eles têm a cauda constante.

Com o entendimento extra fornecido pelo exemplo acima, o próximo passo da pesquisa foi buscar uma prova do teorema. 

\begin{thebibliography}{99}
	\bibitem{Barnsley1988} BARNSLEY, M. F. Fractals Everywhere. \textbf{Academic Press, Inc. (London) LTD.} 1988.
\end{thebibliography}

\end{document}

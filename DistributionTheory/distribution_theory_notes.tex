\documentclass[12pt]{article}
 \usepackage{bookmark}
 \usepackage{amsmath}
 \usepackage{amsthm}
 \usepackage{amssymb}
 \usepackage{mathabx}
 \usepackage{tikz}
 \usepackage{pgfplots}
 \usepackage[utf8]{inputenc}
 \usepackage{amsfonts}
 \usepackage{nicematrix}
 \usepackage[x11names]{xcolor}
 \usepackage{geometry}
 \usepackage{graphicx}
 \usepackage{graphics}
 \usepackage[export]{adjustbox}
 \usepackage{fancyhdr}
 \usepackage[portuguese]{babel}
 \usepackage{hyperref}
 \usepackage{multirow}
 \usepackage{lastpage}
 \usepackage{mathtools}
 \usepackage[many]{tcolorbox}
 \usepackage{newtxsf}
 \usepackage{subfiles}
 \usepackage{flafter}
 \usepackage{float}
 \usepackage{accents}
 \usepackage[T1]{fontenc}

 \pagestyle{fancy}
 \fancyhf{}

 \pgfplotsset{compat = 1.18}

 \hypersetup{
     colorlinks,
     citecolor=black,
     filecolor=black,
     linkcolor=black,
     urlcolor=black
 }

 \newtheorem*{theorem*}{\underline{Teorema}}
 \newtheorem*{lemma*}{\underline{Lema}}
 \newtheorem*{prop*}{\underline{Proposição}}
 \newtheorem*{crl*}{\underline{Corolário}}

 \theoremstyle{definition} % remove o itálico desses abaixo
 \newtheorem{example}{\underline{Exemplo}}
 \newtheorem*{def*}{\underline{Definição}}
 \newtheorem*{proof*}{\underline{Prova}}
 \newtheorem{exr}{\underline{Exercício}}

 \renewcommand\qedsymbol{$\blacksquare$}

 \rfoot{Página \thepage \hspace{1pt} de \pageref{LastPage}}

 \geometry{a4paper, left=3cm, top=3cm, right=3cm, bottom=3cm}

\begin{document}
\begin{figure}[ht]
	\minipage{0.76\textwidth}
	\includegraphics[width=4cm]{../icmc.png}
	\hspace{7cm}
	\includegraphics[height=4.9cm,width=4cm]{../brasao_usp_cor.jpg}
	\endminipage
\end{figure}

\begin{center}
	\vspace{1cm}
	\LARGE
	UNIVERSIDADE DE SÃO PAULO

	\vspace{1.3cm}
	\LARGE
	INSTITUTO DE CIÊNCIAS MATEMÁTICAS E COMPUTACIONAIS - ICMC

	\vspace{1.7cm}
	\Large
	\textbf{Notas de Teoria da Distribuições}

	\vspace{1.3cm}
	\large
	\textbf{Renan Wenzel - 11169472}

	\vspace{1.3cm}
	\large
	\textbf{Professor(a): Éder Ritis Aragão Costa}

	\textbf{E-mail: ritis@icmc.usp.br}

	\vspace{1.3cm}
	\today
\end{center}

\newpage
\textbf{{\Huge Disclaimer}}

{\huge Essas notas foram feitos a pedido do estimado e famoso professor Édinho.

	Ainda assim, qualquer erro é responsabilidade solene do autor.

	Caso julgue necessário, contatar:

	renan.wenzel.rw@gmail.com}
\tableofcontents

\newpage

\subfile{./classes/aula01}
\newpage
\subfile{./classes/aula02}
\newpage
\subfile{./classes/aula03}
\newpage
\subfile{./classes/aula04}
\newpage
\subfile{./classes/aula05}
\newpage
\subfile{./classes/aula06}
\newpage
\subfile{./classes/aula07}
\newpage
\subfile{./classes/aula08}
\newpage
\subfile{./classes/aula09}
\newpage
\subfile{./classes/aula10}
\newpage
\subfile{./classes/aula11}
\newpage
\subfile{./classes/aula12}
\newpage
\subfile{./classes/aula13}
\newpage
\subfile{./classes/aula14}
\newpage
\subfile{./classes/aula15}
\newpage

\begin{thebibliography}{99}
	\bibitem{1} L. Hörmander, "Linear Partial Differential Operators" (Cap. I, pags 1-28: Corpo principal da teoria das distribuições - os resultados mais básicos da teoria)
	\bibitem{2} G. Friedlander, M. Joshi, "Introduction to the Theory of Distributions" (Produto Tensorial,  Núcleos de Schwartz e exemplos)
	\bibitem{3} J. Hounie, "Teoria Elementar das Distribuições", CBM - IMPA (1976) (Introdução com ênfase no cálculo, exemplos, resultados básicos sobre soluções fundamentais e a Transformada Parcial de Fourier).
	\bibitem{4} F. Treves, "Topological Vector Spaces, Distributions and Kernels" (estrutura topológica que sustenta à Teoria das Distribuições começando com filtros, espaços de Fréchet e Limite Indutivo de Espaços de Fréchet - converte resultados gerais a resultados sobre distribuições).
	\bibitem{5} W. Rudin, "Functional Analysis" (introdução sólida aos TVSs, usando abertos no lugar de filtros, e o Teorema do Limite Indutivo para funções teste - mais do que suficiente para o que pretendo)
	\bibitem{6} G.B. Folland, "Real Analysis, Modern Techniques and their Applications" (introdução rápida aos Espaços de Fréchet e nets, excelente abordagem da Transformada de Fourier em \(L^1\), \(L^2\) e S e operadores diferenciais com coeficientes constantes).
	\bibitem{7} G.B. Folland, "Fourier Analysis and Its Applications" (alguns exemplos interessantes e simples sobre convolução e a Transformada de Fourier e os resultados básicos de aproximação num nível elementar)
\end{thebibliography}

\end{document}

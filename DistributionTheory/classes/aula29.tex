\documentclass[../distribution_theory_notes.tex]{subfiles}
\begin{document}
\section*{Apêndices}
\subsection*{Aula 01}
\subsubsection*{Aplicações Contínuas e SFVs}
Neste primeiro apêndice, começamos por relembrar o que é uma base e o que é um SFV.

\begin{def*}
	Seja \((E, \tau )\) um espaço topológico. Uma \textbf{base para} \(\tau \) é uma coleção de abertos \(\mathcal{B}\subseteq \tau \) tal que todo aberto de \(\tau \) pode ser expresso como
	a união de membros de \(\mathcal{B}\). Em outras palavras, dados um aberto U de \(\tau \) contendo um elemento x, existe B na coleção \(\mathcal{B}\) tal que x pertence a B, que está contido em U (\(x\in B\subseteq U\)). \(\square\)
\end{def*}

\hypertarget{sistema_fundamental_vizinhas}{\begin{def*}
		Um \textbf{Sistema Fundamental de Vizinhanças} (SFV) ou \textbf{base local de um ponto x} é uma coleção \(\mathcal{B}_{x}\) de abertos tal que, dados um aberto U e x seu elemento, existe B em \(\mathcal{B}_{x}\) tal que x pertence a B, que está contido em U. \(\square\)
	\end{def*}}

\begin{def*}
	Diz-se que a topologia de E é \textbf{priemiro-enumerável}, ou \(E_{1}\), quando todo elemento x de E possui um SFV enumerável \(\mathcal{B}\). \(\square\)
\end{def*}

Segue que todo espaço métrico (M, d) é priemiro-enumerável, pois
\[
	\mathcal{B}_{x}=\biggl\{B \biggl(x; \frac{1}{n}\biggr):\: n\in \mathbb{N}\biggr\}
\]
para cada x em M forma uma SFV enumerável.

Além disso, veremos as noções equivalentes de continuidade:
Sejam \((E, \tau_{E})\) e \((F, \tau_{F})\) espaços topológicos e \(f:E\rightarrow F\) uma aplicação. São equivalentes:
\begin{itemize}
	\item[I)] f é contínua, ou seja, \(f^{-1}(U)\in \tau_E\) para todo \(U\in \tau_F\);
	\item[II)] Para todo x em E e \(U\in \tau_{F}\) com \(f(x)\in U\), existe \(V\in \tau_{E}\) contendo x tal que \(V\subseteq f^{-1}(U)\);
	\item[III)] Para todo x em E e \(U\in \tau_{F}\) com \(f(x)\in U\), existe \(V\in \tau_{E}\) contendo x tal que \(f(V)\subseteq U\);
	\item[IV)] Se \(\{x_{\alpha }\}_{\alpha \in A}\) é uma sequência de elementos em E tal que
	      \[
		      x_{\alpha }\overbracket[0pt]{\longrightarrow}^{E}x,
	      \]
	      então
	      \[
		      f(x_{\alpha })\overbracket[0pt]{\longrightarrow}^{F}f(x)
	      \]
	      para qualquer x em E;
	\item[V)] Quando E é priemiro-enumerável e, para todo \(x_{n}\) convergindo para x em E quando n tende a infinito,
	      \[
		      \lim_{n\to \infty}f_{n}(x)\overbracket[0pt]{\longrightarrow}^{F}f(x);
	      \]
	\item[VI)] Se \(\mathcal{B}_{F}\) é uma base para a topologia de F, então \(f^{-1}(B)\) é um aberto em E para todo B pertencente a esta base de F;
	\item[VII)] Se \(\mathcal{B}_{E}\) é uma base para a topologia de E, e \(\mathcal{B}_{F} \) uma para a de F, então para todo x em E e B da base de F com \(f(x)\) sendo elemento de B, existe um C na base de E conténdo x tal que \(f(c)\) é um subconjunto de B.
\end{itemize}


\end{document}

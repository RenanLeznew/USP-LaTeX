\documentclass[../distribution_theory_notes.tex]{subfiles}
\begin{document}
\section{Aula 16 - 18 de Outubro, 2024}
\subsection{Motivações}
\begin{itemize}
	\item Consequências da Existência de \(u*\varphi \);
	\item Propriedades de \(u*\varphi \);
	\item Soluções Fundamentais e Densidade das Funções Testes em \(\mathcal{D}'\).
\end{itemize}
\subsection{Consequências e Propriedades da Existência de \(u*\varphi \)}
Estabelecidas as primeiras propriedades de convolução ``\(u*\varphi \)'', sendo \(u\in \mathcal{D}'(\mathbb{R}^{n})\) e \(\varphi\in \mathcal{C}_{c}^{\infty}(\mathbb{R}^{n}) \); agora, nosso objetivo é descrever algumas das consequências, sendo a principal ligada ao conceito de \textit{solução fundamental}, que foi sendo mencionado em quantidades diluídas ao longo do curso.

\begin{prop*}[Invariância por Translação]
	Para todo \(h\in \mathbb{R}^{n},\; u\in \mathcal{D}'\) e \(\varphi \in \mathcal{C}_{c}^{\infty}\), tem-se:
	\[
		\tau_{h}(u*\varphi )=(\tau_{h}u)*\varphi = u*(\tau_{h}\varphi ).
	\]
\end{prop*}
\begin{proof*}
	A verificação é imediata de definição:
	\begin{align*}
		[\tau_{h}(u*\varphi )](x) = u * \varphi (x-h) & = \langle u, \tau_{(x-h)}\check{\varphi } \rangle              \\
		                                              & = \langle u, \tau_{-h}\circ (\tau_{x}\check{\varphi }) \rangle \\
		                                              & = \langle \tau_{h}u, \tau_{x}\check{\varphi } \rangle          \\
		                                              & = (\tau_{h}u)*\varphi (x).
	\end{align*}
	Da mesma forma,
	\begin{align*}
		\langle u, \tau_{(x-h)}\check{\varphi } \rangle = \langle u, \tau_{x}(\tau_{-h}\check{\varphi }) \rangle & = \langle u, \tau_{x}(\tau_{h}\varphi \check{)} \rangle \\
		                                                                                                         & = [u*(\tau_{h}\varphi )](x),
	\end{align*}
	pois
	\[
		(\tau_{h}\varphi \check{)}(y)=\tau_{h}\varphi (-y)=\varphi (-y-h)=\check{f}(y+h)=(\tau_{-h}\check{\varphi })(y). \text{ \qedsymbol}
	\]
\end{proof*}
Esta propriedade pode até parecer ``basal'', mas é uma questão da Análise de Fourier caracterizar, em diversos espaços, os operadores lineares que são invariantes por translações; de fato, nós mesmos enfrentaremos, nas aulas seguintes, a necessidade em saber dizer quando um aplicação linear \(T:\mathcal{C}_{c}^{\infty}(\mathbb{R}^{n})\rightarrow \mathcal{C}^{\infty}(\mathbb{R}^{n})\) é invariante por translação (ou melhor, que forma assume uma aplicação linear com uma característica)

\begin{prop*}[Convolução de \(\varphi \) com uma Medida \(\mu \)]
	Para o caso em que \(u = \mu \in \mathcal{D}'(\mathbb{R}^{n})\) provém de uma medida boreliana localmente finita,
	\[
		(\mu *\varphi )(x) = \langle \mu , \tau_{x}\check{\varphi } \rangle = \int_{\mathbb{R}^{n}}^{}\varphi (x-y) \mathrm{d}\mu(y)
	\]
	e, derivando sob o sinal da integral, obtemos:
	\[
		\partial^{\alpha }(\mu *\varphi )(x)=\int_{\mathbb{R}^{n}}^{}(\partial^{\alpha }\varphi )(x-y) \mathrm{d}\mu (y) = [\mu *(\partial^{\alpha }\varphi )](x).
	\]
	Além disso,
	\begin{align*}
		\int_{\mathbb{R}^{n}}^{}(\partial^{\alpha }\varphi )(x-y) \mathrm{d}\mu (y) = (-1)^{| \alpha  |}\int_{\mathbb{R}^{n}}^{}\partial_{y}^{\alpha }\phi(x, y) \mathrm{d}\mu (y) & = (-1)^{| \alpha  |}\langle u, \partial_{y}^{\alpha }\phi(x, \cdot ) \rangle \\
		                                                                                                                                                                           & = \langle \partial^{\alpha }\mu , \phi(x, \cdot ) \rangle                    \\
		                                                                                                                                                                           & = \langle \partial^{\alpha }\mu , \tau_{x}\check{\varphi } \rangle           \\
		                                                                                                                                                                           & = [(\partial^{\alpha }\mu )*\varphi ](x),
	\end{align*}
	onde \(\phi(x, y)=\varphi (x-y)\).

	Em conclusão,
	\[
		\partial^{\alpha }(u*\varphi ) = (\partial^{\alpha }\mu )*\varphi  = \mu *(\partial^{\alpha }\varphi ), \quad \forall \alpha \in \mathbb{Z}_{+}^{n}.
	\]
\end{prop*}

Com isso, obtemos uma prova mais direta da propriedade (ii) nesse caso particular, que é bastante útil e comum.

Dentre outras consequências, podemos ver as derivadas parciais como operador do tipo ``convolução com um núcleo'': fixada \(u\in \mathcal{D}'(\mathbb{R}^{n})\), fica bem definido o operador linear contínuo
\begin{align*}
	T: & \mathcal{C}_{c}^{\infty}(\mathbb{R}^{n})\rightarrow \mathcal{C}^{\infty}(\mathbb{R}^{n}) \\
	   & \varphi \longmapsto T\varphi = u*\varphi .
\end{align*}
Inclusive, é surpreendente o fato de que \(\partial^{\alpha }\) é dessa forma! Com efeito, dados \(\varphi \in \mathcal{C}_{c}^{\infty}\) e \(\alpha \in \mathbb{Z}_{+}^{n}\), temos
\[
	\partial^{\alpha }\varphi  = \partial^{\alpha }(\delta *\varphi ) = (\partial^{\alpha }\delta )*\varphi.
\]
Logo, pondo \(u\coloneqq \partial^{\alpha }\delta \), a afirmação segue (em particular, \(u\in \mathcal{E}'(\mathbb{R}^{n})\)); consequentemente, se \(P(D)=\sum\limits_{| \alpha  |\leq m}^{}a_{\alpha }\partial^{\alpha }\) é um ODPCC, então, em \(\mathcal{C}_{c}^{\infty}\), ele também assume a forma
\[
	P(D)\varphi  = \sum\limits_{| \alpha  |\leq m}^{}a_{\alpha }(\partial^{\alpha }\delta )*\varphi = \biggl(\sum\limits_{| \alpha  |\leq m}^{}a_{\alpha }\partial^{\alpha }\delta \biggr)*\varphi = (P(D)\delta )*\varphi .
\]
Com isso, a ação de \(P(D)\) em \(\mathcal{C}^{\infty}\) se reduz ao cálculo de \(P(D)\delta !\)

\subsection{Soluções Fundamentais e Densidade das Funções Teste em \(\mathcal{D}'\)}
\begin{def*}
	Seja \(P(D)=\sum\limits_{| \alpha  |\leq m}^{}a_{\alpha }\partial^{\alpha }\) um ODPCC. Uma \textbf{solução fundamental} (S.F.) de \(P(D)\), ou uma \textbf{Função de Green}, é uma distribuição E em \(\mathbb{R}^{n}\) tal que \(P(D)E = \delta .\; \square\)
\end{def*}
O conhecimento de uma solução fundamental pode, em muitos casos, fornecer toda a informação que se deseja extrair sobre o operador, tal como a resolução de certas equações não-homogêneas:
\begin{example}
	Dada qualquer função \(f\in \mathcal{C}_{c}^{\infty}(\mathbb{R}^{n})\), pondo \(u = E * f\), obtemos
	\begin{align*}
		P(D)u = P(D)(E*f)= (P(D)E)*f = = \delta * f = f,
	\end{align*}
	ou seja, \(P(D)u = f\) em \(\mathcal{D}'.\) Mais do que isso, segue que
	\[
		P(D)(E*f)=\sum\limits_{}^{}a_{\alpha }\partial^{\alpha }(E*f) = \sum\limits_{}^{}a_{\alpha }(\partial^{\alpha }E)*f = \biggl(\sum\limits_{}^{}a_{\alpha }\partial^{\alpha }E\biggr)*f.
	\]
\end{example}
\begin{tcolorbox}[
		skin=enhanced,
		title=Observação,
		fonttitle=\bfseries,
		colframe=black,
		colbacktitle=cyan!75!white,
		colback=cyan!15,
		colbacklower=black,
		coltitle=black,
		drop fuzzy shadow,
		%drop large lifted shadow
	]
	A conclusão acima se traduz dizendo que, caso \(P(D)\) possui uma solução fundamental, então ele é \textit{globalmente} resolvível.
\end{tcolorbox}

Em vista do exemplo e das discussões, a existência de uma solução fundamental para \(P(D)\) garante uma recíproca para o fato observado acima! Se \(u\in \mathcal{D}'(\mathbb{R}^{n})\) (em \(\mathcal{E}'(\mathbb{R}^{n})\)) e \(P(D)u = f\), então \(u=E*f\), mas ele precisa da parte 3 do produto da convolução, a qual trata do caso
em que ambos os fatores são distribuições; quando isto estiver estabelecido, poderemos escrever
\[
	u = \delta *u = (P(D)E)*u = E*(P(D)u)=E*f.
\]

Um resultado bastante importante para caracterizarmos as distribuições em \(\mathcal{D}'(\mathbb{R}^{n})\) é a densidade das funções testes nele:
\begin{theorem*}
	As funções teste são densas em \(\mathcal{D}'(\mathbb{R}^{n})\), ou seja, \(\mathcal{C}_{c}^{\infty}(\mathbb{R}^{n})\) é denso em \(\mathcal{D}'(\mathbb{R}^{n})\) na topologia fraca-*.
\end{theorem*}
\begin{proof*}
	Primeiro, provamos o seguinte fato intermediário:

	\textbf{\underline{Afirmação}:} Se \(\{\psi_{j}\}_{j}\) é uma sequência de funções de Urysohn associadas a um esgotamento \((K_{j})_{j}\) (isto é, \(\psi_{j}\equiv1\) em \(K_{j}\) com \(\psi_{j}\in \mathcal{C}_{c}^{\infty}\)), então dada \(u\in \mathcal{D}'\), segue que
	\[
		\psi_{j}\stackrel{j\to \infty}\longrightarrow u
	\]
	em \(\mathcal{D}'\).

	Com efeito, se \(\mathrm{supp}(\varphi )\subseteq K_{j_{0}}\), então \(j\geq j_{0}\) implica que \(\psi_{j}\varphi = \varphi \), tendo em vista que \(\psi_{j}\equiv 1\) em \(\mathrm{supp}(\varphi )\), se \(j\geq j_{0}\) e \(\psi_{j}\varphi \equiv 0\) em \(\mathbb{R}^{n}\setminus{\mathrm{supp}(\varphi )}\); logo, as duas coincidem mesmo. Com isso, \(\mathrm{supp}(\psi_{j}\varphi )\subseteq \mathrm{supp}(\varphi ),\; j\geq j_{0}\), e como \(j\geq j_{0}\),
	\[
		\partial^{\alpha }(\psi_{j}\varphi )=\partial^{\alpha }\varphi,
	\]
	donde segue a convergência imediatamente.

	Disto, temos
	\[
		\langle \psi_{j}u, \varphi  \rangle = \langle u, \psi_{j}\varphi  \rangle\stackrel{j\to \infty}\longrightarrow \langle u, \varphi  \rangle,
	\]
	concluindo a afirmação. \blacktriangle

	Tendo provado-a, é suficiente mostrar que \((\psi u)*\varphi_{\varepsilon }\) converge para \(\psi u\) em \(\mathcal{D}'\) conforme \(\varepsilon \) tende a zero pela direita, sendo \(\{\varphi_{\varepsilon }\}_{\varepsilon>0}\) é a família regularizante padrão, pois
	\(\mathrm{supp}((\psi u)*\varphi_{\varepsilon })\) é compacto para todo \(\varepsilon \in (0,1]\), ou seja, \((\psi u)*\varphi_{\varepsilon }\in \mathcal{C}_{c}^{\infty}(\mathbb{R}^{n})\).

	Dada \(\varphi \in \mathcal{C}_{c}^{\infty}\) para \(j\in \mathbb{N}\) e \(\varepsilon >0\), podemos escrever
	\begin{align*}
		\langle (\psi_{j}u)*\varphi_{\varepsilon }, \phi \rangle & = \langle (\psi_{j}u)*\varphi_{\varepsilon } - \psi_{j}u, \phi \rangle + \langle \psi_{j}u, \phi \rangle                                    \\
		                                                         & = \langle (\psi_{j}u)*\varphi_{\varepsilon }-\psi_{j}u, \phi \rangle + \langle \psi_{j}u-u, \varphi  \rangle + \langle u, \varphi  \rangle.
	\end{align*}
	Assim, dado \(\eta >0\) e fixando \(j_{0}\) tal que a segunda parcela é menor que \(\eta /2\), com esse \(j_{0}\) fixamos \(\varepsilon >0\) suficientemente pequeno para que a primeira seja também menor que \(\eta/2\), e aí o teorema estará provado.

	Dito isso, para \(\varphi \in \mathcal{C}_{c}^{\infty}\), calculamos
	\begin{align*}
		\langle (\psi u)*\varphi_{\varepsilon }, \phi \rangle = \langle (\psi u)*\varphi_{\varepsilon }, \tau_{0}\phi \rangle & = \langle (\psi u)*\varphi_{\varepsilon }, \tau_{0}[\check{\phi}\check{]} \rangle     \\
		                                                                                                                      & = [(\psi u)*\varphi_{\varepsilon }]*(\check{\phi})(0)                                 \\
		                                                                                                                      & = (\psi u)*[\varphi_{\varepsilon }*(\check{\varphi })](0)                             \\
		                                                                                                                      & = \langle \psi u, \tau_{0}(\varphi_{\varepsilon }*(\check{\varphi })\check{)} \rangle \\
		                                                                                                                      & = \langle \psi u, [\varphi_{\varepsilon }*(\check{\varphi })\check{]} \rangle.
	\end{align*}
	Como \(\varphi_{\varepsilon }*(\check{\varphi })\) converge para \(\check{\varphi }\) em \(\mathcal{C}_{c}^{\infty}(\mathbb{R}^{n})\) e \(\varepsilon \) tende ao 0 pela direita e \(R:\mathcal{C}_{c}^{\infty}\rightarrow \mathcal{C}_{c}^{\infty}\) definida por \(R\phi = \check{\phi}\) é contínua, vem da última igualdade que
	\[
		\langle (\psi u)*\varphi_{\varepsilon }, \phi \rangle = \langle \psi u, [\varphi_{\varepsilon }*(\check{\varphi })\check{]} \rangle\stackrel{\varepsilon \to 0^{+}} \langle \psi u, (\check{\varphi }\check{)}  \rangle = \langle \psi u, \varphi  \rangle. \text{ \qedsymbol}
	\]
\end{proof*}
\begin{exr}
	Se \(d(x, \mathrm{supp}(\psi ))>1\), então \(B_{\varepsilon }(x)\cap \mathrm{supp}(\psi ) = \phi\), onde \(0<\varepsilon \leq 1\); consequentemente, \(\psi (y)\varphi_{\varepsilon }(x-y)=0\) para todo \(y\in \mathbb{R}^{n}\), donde
	\[
		\langle u, \psi \tau_{X}\check{\varphi }_{\varepsilon } \rangle =0.
	\]
	Conclusão:
	\[
		\mathrm{supp}((\psi u)*\varphi_{\varepsilon })\subseteq \overline{\varphi }_{1}(\mathrm{supp}(\psi )) = \{x\in \mathbb{R}^{n}:\; d(x, \mathrm{supp}(\psi ))\leq 1\},
	\]
	comprovando a afirmação da prova acima.
\end{exr}

\end{document}

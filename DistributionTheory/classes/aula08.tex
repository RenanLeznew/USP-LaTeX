\documentclass[../distribution_theory_notes.tex]{subfiles}
\begin{document}
\section{Aula 08 - 16 de Setembro, 2024}
\subsection{Motivações}
\begin{itemize}
	\item Convolução;
	\item O Melhor Parágrafo do Édinho;
	\item Teorema de Fubini-Tonelli.
\end{itemize}
\subsection{Convoluções e Fubini-Tonelli.}
Em continuidade ao nosso estudo do espaço das funções teste, \(\mathcal{C}_{c}^{\infty}(\Omega )\), estudaremos agora um método de construir funções deste espaço a partir do único exemplo que demos, ou seja,
\[
	\varphi (x) = \left\{\begin{array}{ll}
		e^{-\frac{1}{1-|x|^{2}}}, & \quad |x|<1 \\
		0,                        & |x|\geq 1
	\end{array}\right.,
\]
utilizando a operação chamada \textit{convolução}.
\end{document}

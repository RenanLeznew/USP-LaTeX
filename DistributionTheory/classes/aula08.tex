\documentclass[../distribution_theory_notes.tex]{subfiles}
\begin{document}
\section{Aula 08 - 16 de Setembro, 2024}
\subsection{Motivações}
\begin{itemize}
	\item Convolução;
	\item O Melhor Parágrafo do Édinho;
	\item Teorema de Fubini-Tonelli.
\end{itemize}
\subsection{Convoluções e Fubini-Tonelli.}
Em continuidade ao nosso estudo do espaço das funções teste, \(\mathcal{C}_{c}^{\infty}(\Omega )\), estudaremos agora um método de construir funções deste espaço a partir do único exemplo que demos, ou seja,
\[
	\varphi (x) = \left\{\begin{array}{ll}
		e^{-\frac{1}{1-|x|^{2}}}, & \quad |x|<1 \\
		0,                        & |x|\geq 1
	\end{array}\right.,
\]
utilizando a operação chamada \textit{convolução}.
  \begin{tcolorbox}[
  skin=enhanced,
  title=Observação,
  fonttitle=\bfseries,
colframe=black,
  colbacktitle=cyan!75!white, 
  colback=cyan!15,
  colbacklower=black,
coltitle=black,
  drop fuzzy shadow,
  %drop large lifted shadow
  ]
  A análise de Fourier, ou análise harmônica, é um conjunto de técnicas que visam representar ``funções'' de uma determinada classe por meio de somas (série de Fourier) ou integrais (integral de Fouriar) de ``funções especiais'' dentro daquela classe. Essas técnicas giram em torno de duas operações fundamentais: a \textit{convolução} e a \textit{transformada de Fourier}.
  \end{tcolorbox}

  A fim de introduzir as ideias e uma intuição para elas, consideremos o seguinte exemplo simplificado: sejam \(f, p:[a, b]\rightarrow \mathbb{R}\) funções integráveis e, para cada n natural, dividimos \([a, b]\) em n partes iguais, obtendo a partição 
    \[
      P_{n}: a=x_{0}<x_1<\dotsc <x_{n}=b,
    \]
    onde \(x_{i}=a + i (b-a)/n,\; i = 0,1,2,\dotsc ,n,\) ou \(x_{i}=x_{i-1}+(b-a)/n\) a partir do índice 1. Com isso, 
      \[
        \Delta x_{i}=\frac{b-a}{n},\quad i=1,2,\dotsc ,n.
      \]
      Escolhendo arbitrariamente pontos  \(\xi_{i}, \zeta_{i}\) em \([x_{i-1}, x_{i}]\), a \textbf{média ponderada} dos n valores \(f(\xi_1), f(\xi_2),\dotsc , f(\xi_{n})\) com ``pesos'' \(p(\zeta_1), p(\zeta_2), \dotsc , p(\zeta_{n})\) é o número: 
        \[
          M_{n}(f; p) \coloneqq \frac{1}{\sum\limits_{i=1}^{n}p(\zeta_{i})}\sum\limits_{j=1}^{n}f(\xi_{j})p(\zeta_{j}).
        \]
        Vale mencionar que, a rigor, os pesos deveriam ser 
          \[
            \frac{p(\zeta_1)}{\sum\limits_{}^{}p(\zeta_{i})}, \dotsc , \frac{p(\zeta_{n})}{\sum\limits_{}^{}p(\zeta_{i})},
          \]
          pois a soma deles deve ser igual a 1. 
         \begin{figure}[H]
         \begin{center}
         \includegraphics[height=0.5\textheight, width=0.5\textwidth, keepaspectratio]{./Images/weighted_avg_08.png}
         \end{center}
         \caption{os pesos ``puxam a média'' para cima.}
         \end{figure}

         Note que, multiplicando e dividindo \(M_{n}(f; p)\) por \((b-a)/n\), temos: 
           \[
             M_{n}(f; p)=\frac{1}{\sum\limits_{i=1}^{n}p(\zeta_{i})\frac{(b-a)}{n}}\sum\limits_{j=1}^{n}f(\xi_{j})p(\zeta_{j})\frac{(b-a)}{n} = \frac{1}{\sum\limits_{i=1}^{n}\Delta x_{i}}\sum\limits_{j=1}^{n}f(\xi_{j})p(\zeta_{j})\Delta x_{j},
           \]
           tal que, fazendo n tender a infinito e usando a continuidade de f e p, concluímos que 
             \[
               \lim_{n\to \infty}M_{n}(f; p)= \frac{1}{\int_{a}^{b}p(y) \mathrm{dy}}\int_{a}^{b}f(y)p(y) \mathrm{dy} \eqqcolon M(f; p).
             \]
             A média é um valor representativo de uma amostra, no sentido de representar alguma característica que se deseja analisar, então é razoável se definir a média ponderada dos valores de f com pesos p, também chamados \textbf{densidade}, em \([a, b]\) como esse limite. Em particular, se 
               \[
                 \int_{a}^{b}p(y) \mathrm{dy}=1,
               \]
               então 
               \[
                 M(f; p)=\int_{a}^{b}f(y)p(y) \mathrm{dy},
               \]
               sendo que, quando tanto f quanto p são não negativos, isto representa a área sob o gráfico de f. É claro que os maiores valores \(p(y)\) de p puxam a média para a média dos valores \(f(y)\) -- pode-se pensar que eles \textit{aumentam a influência}. 

               Para fixar estas ideias, consideramos o caso em que \([a, b]=[-1, 1]\), no qual p assume seu valor máximo em \(y=0\), p é não negativo no intervalo, \(\int_{-1}^{1}p \mathrm{dy} = 1\) e que seus valores sejam muito pequenos ``fora'' da origem; nestas condições, nosso objetivo é ver como isto ``puxa'' a média a fim de obtermos o valor \(f(0).\) Com efeito: 
              \begin{align*}
                M(f; p)-f(0) = \int_{-1}^{1}f(y)p(y) \mathrm{dy}-f(0) \cdot 1 &= \int_{-1}^{1}f(y)p(y) \mathrm{dy}-\int_{-1}^{1}f(0)p(y) \mathrm{dy}\\ 
                                                                              &= \int_{-1}^{1}f(y)p(y)-f(0)p(y) \mathrm{dy}\\ 
                                                                              &= \int_{-1}^{1}[f(y)-f(0)]p(y) \mathrm{dy}\\ 
                                                                              &= \int_{-\delta }^{\delta }[f(y)-f(0)]p(y) \mathrm{dy} + \int_{|y|\geq \delta }^{}[f(y)-f(0)]p(y) \mathrm{dy}.
              \end{align*}
              Daí, vemos que, sendo f contínua em \(y_{0}=0,\; p\geq 0\) e \(\int_{}^{}p \mathrm{dy}=1\), a primeira parcela da igualdade acima tende a zero conforme \(\delta \) em si tende a zero; a segunda, como f é limitada e os valores de p são pequenos fora da origem, pode ser desprezada, o que sugere que nossa afirmação é correta: \(M(f; p)\) está próxima do valor \(f(0)\). Finalmente, se supormos que p é, além de tudo isso, uma função par com \(\mathrm{supp}(p)=[-1, 1]\) e que f esteja definida em toda a reta, podemos considerar os pesos ``móveis'' ponto a ponto, isto é, mover os pesos p ao longo da reta por meio das rotações \(r(y)=-y\) e translações 
              \[
              p_{x}(y)\coloneqq p(x-y) = p(y-x),
            \]
            sendo que 
              \[
                \mathrm{supp}(p_x)=[x-1, x+1],
              \]
              pois \(|x-y|\geq 1\) implica que \(p(x-y)=0\), tal que 
                \[
                  p(x-y)\neq 0 \Rightarrow |y-x|\leq 1 \Rightarrow y\in [x-1, x+1].
                \]
 \begin{figure}[H]
 \begin{center}
 \includegraphics[height=0.5\textheight, width=0.5\textwidth, keepaspectratio]{./Images/density_08.png}
 \end{center}
 \caption{por meio de translações, os suportes de cada \(p_x\) são deslocados, mas mantêm a mesma forma.}
 \end{figure}

            Logo, a média 
              \[
                M(f; p_{x})= \int_{x-1}^{x+1}f(y)p_{x}(y) \mathrm{dy}=\int_{\mathbb{R}}^{}f(y)p(x-y) \mathrm{dy}
              \]
              é um valor aproximado do valor \(f(x).\) Observe que \(M(f; p_x)\) é uma função de x em \(\mathbb{R}\), no sentido de 
                \[
                  x\mapsto M(f; p_x)
                \]
                e, se p for uma função \(\mathcal{C}^{\infty}\) com f apenas contínua (ou só integrável), o \textit{teorema da convergência dominada de Lebesgue} pode ser usado para provar que essa média é \(\mathcal{C}^{\infty}\) por meio da derivação sob o sinal de integração: 
                  \[
                    \frac{\mathrm{d}}{\mathrm{d}x} \int_{\mathbb{R}}^{}f(y)p(x-y) \mathrm{dy}=\int_{\mathbb{R}}^{}f(y) \frac{\mathrm{d}}{\mathrm{d}x}p(x-y) \mathrm{dy},
                  \]
                  ou seja, ela fornece um mecanismo para aproximar a função f por meio de funções \(\mathcal{C}^{\infty}\), nos fornecendo, assim, \textit{teoremas de densidade} em diversos espaços de funções integráveis. Os detalhes técnicos deste procedimento serão apresentados a seguir em algumas situações concretas, primeiro ``entre funções'' e segundo ``entre distribuições''.

                  Essas ideias também explicam que, ao calcular 
                    \[
                      p\mapsto \int_{}^{}f \varphi
                    \]
                    para uma função \(f\in L_{\mathrm{loc}}^{1}\) e \(\varphi \in \mathcal{C}_{c}^{\infty}\), estamos calculando uma média ponderada dos valores de f (multiplicada por \(\int_{}^{}\varphi \)) com peso \(\varphi \) ou, mais geralmente, se \(u\in \mathcal{D}',\) então 
                      \[
                        \varphi \mapsto \left< u, \varphi  \right>
                      \]
                      teria o mesmo significado, que é coerente com a física, a qual trabalha com médias dos valores obtidos de suas medições.

  Para tudo o que se segue, a menos que explicitamente dito o contrário, a palavra ``mensurabilidade'' irá se referir à mensurabilidade Borel (conjuntos e funções) e dx indicará a medida de Lebesgue (no espaço euclidiano correspondente) definida na \(\sigma \)-álgebra de Borel. Além disso, a integral em \(\mathbb{R}^{n}\), ``\(\int_{\mathbb{R}^{n}}^{}\)'' será indicada por apenas \(\int\).
  O caso mais geral, no qual consideramos a classe \(\mathcal{L}\) das funções Lebesgue mensuráveis e m a medida de Lebesgue nel definida, só se faz necessário quando precisarmos argumentar com medidas completas.
   \begin{tcolorbox}[
   skin=enhanced,
   title=Observação,
   fonttitle=\bfseries,
 colframe=black,
   colbacktitle=cyan!75!white, 
   colback=cyan!15,
   colbacklower=black,
 coltitle=black,
   drop fuzzy shadow,
   %drop large lifted shadow
   ]
   Se f e g são Borel mensuráveis, então a função 
  \begin{align*}
    F:&\mathbb{R}^{n}\times \mathbb{R}^{n}\rightarrow \mathbb{C}\\ 
      &(x, y)\longmapsto F(x,y)=f(x-y)g(y)
  \end{align*}
  é mensurável, pois 
    \[
      G(z, w)=f(z)g(w) = [M \circ (f, g)](z, w),\quad M(s, t)=st
    \]
    é contínua, \(S(x, y)=x-y\) é contínua e 
   \begin{align*}
     j:&\mathbb{R}^{n}\times \mathbb{R}^{n}\rightarrow \mathbb{C}\times \mathbb{C}\\ 
       & (z, w)\longmapsto j(z, w)= (f(z), g(w))
   \end{align*}
   é (\(\mathcal{B}_{\mathbb{R}^{n}}\times \mathcal{B}_{\mathbb{R}^{n}}, \mathcal{B}_{\mathbb{C}\times \mathbb{C}}\))-mensurável.
   \end{tcolorbox}

  \begin{def*}
    Sejam f, g funções mensuráveis em \(\mathbb{R}^{n}\) e a valores complexos. Para cada x em \(\mathbb{R}^{n}\), o \textbf{produto de convolução de f por g em x} é definido pela integral 
      \[
        (f*g)(x)=(f*g)(x) = \int_{}^{}f(x-y)g(y) \mathrm{dy} = M(g; f_x)\int_{}^{}f(y) \mathrm{dy}. \quad \square
      \]
  \end{def*}
  Sempre que a mesma existir (\(p=\infty\) implica que \(f*g(x)\) existe para todo x em \(\mathbb{R}^{n}\)) e for um número. Nesse caso, pelo teorema de mudança de variáveis com difeomorfismos \(h(z)=x-z\), onde z está em \(\mathbb{R}^{n}\), resulta que 
    \[
      f*g(x)=\int_{}^{}f(x-y)g(y) \mathrm{dy} = \int_{}^{}f(z)g(x-z) \mathrm{dz},
    \]
    ou seja, * é uma multiplicação comutativa:
      \[
        f*g(x)=g*f(x).
      \]
\begin{tcolorbox}[
       skin=enhanced,
       title=Lembrete!,
       after title={\hfill Fubini-Tonelli},
       fonttitle=\bfseries,
       sharp corners=downhill,
     colframe=black,
       colbacktitle=yellow!75!white, 
       colback=yellow!30,
       colbacklower=black,
     coltitle=black,
       %drop fuzzy shadow,
       drop large lifted shadow
       ]
       \hypertarget{fubini_tonelli}{ 
         \begin{theorem*}[Fubini_Tonelli]
          Sejam \((X, \mathcal{M}, \mu )\) e \((Y, \mathcal{N}, \nu)\) dois espaços de medidas \(\sigma \)-finitas e f uma função \( \mathcal{M} \otimes \mathcal{N}\)-mensurável (\(f:X\times Y\rightarrow \mathbb{C}\) mensurável no produto \(\mathcal{M}\otimes \mathcal{N}\) ou segundo ele). Então, temos: 

          \textbf{\underline{Tonelli}:} se \(f\geq 0\), então as funções \(g(x)=\int_{}^{}f(x, y) \mathrm{d}\nu(y)\) e \(h(y)=\int_{}^{}f(x, y) \mathrm{d}\mu(x)\) são mensuráveis em \(\mathcal{M}\) e \(\mathcal{N}\) respectivamente, e vale 
         \begin{align*}
           \int_{X\times Y}^{}f \mathrm{d}(\mu \times \nu ) = \int_{X}^{}\biggl[\int_{Y}^{}f(x, y) \mathrm{d}\nu (y)\biggr] \mathrm{d}\mu(x)&= \int_{X}^{}g(x) \mathrm{d}\mu (x)\\ 
                                                                                                                                            &= \int_{Y}^{}\biggl[\int_{X}^{}f(x, y) \mathrm{d}\mu (x)\biggr] \mathrm{d}\nu(y)\\ 
                                                                                                                                            &= \int_{Y}^{}h(y) \mathrm{d}\nu (y).
                                                                                                                                                                  
         \end{align*} 
         Equivalentemente, isto é o mesmo que dizer que \(f_x:Y\rightarrow \mathbb{C},\; f_x(y)=f(x, y)\) e \(f^{y}:X\rightarrow \mathbb{C},\; f(x, y)\) são \(L^{1}(\nu )\) e \(L^{1}(\mu )\), respectivamente.

         \textbf{\underline{Fubini}:} se \(f\in L^{1}(\mu \times \nu )\), então \(g(x)\) é finito \(\mu\)-q.t.p., \(h(y)\) é finito \(\nu\)-q.t.p. Além disso, \(g\in L^{1}(\mu ),\; h\in L^{1}(\nu )\) e a igualdade acima se verifica.
     \end{theorem*} }
       \end{tcolorbox}

        \begin{tcolorbox}[
        skin=enhanced,
        title=Lembrete!,
        after title={\hfill Desigualdade de Hölder},
        fonttitle=\bfseries,
        sharp corners=downhill,
      colframe=black,
        colbacktitle=yellow!75!white, 
        colback=yellow!30,
        colbacklower=black,
      coltitle=black,
        %drop fuzzy shadow,
        drop large lifted shadow
        ]
        \hypertarget{holder_inequality}{
          \begin{theorem*}[Desigualdade de Hölder]
           Dado \(1\leq p\leq \infty\), seja \(p'\) tal que 
             \[
               \frac{1}{p}+\frac{1}{p'}=1. 
             \]
             Se \(f\in L^{p}(\mu )\) e \(g\in L^{p}(\mu )\), então \(fg\in L^{1}(\mu )\) com 
               \[
                 \Vert f \cdot g \Vert_{1}\leq \Vert f \Vert_p \cdot \Vert g \Vert_{p'}.
               \]
         \end{theorem*}
        }
        \end{tcolorbox}

      A seguir, destacaremos algumas condições que asseguram a existência de \(f*g\) e demonstraremos alguns resultados:
      \hypertarget{young_inequality}{
        \begin{theorem*}[Uma das Desigualdades de Young]
         Se \(f\in L^{1}(\mathbb{R}^{n})\) e \(g\in L^{p}(\mathbb{R}^{n}),\; 1\leq p\leq \infty\), então \(f*g(x)\) existe para quase todo ponto x em \(\mathbb{R}^{n}\). Além disso, \(f*g\in L^{p}(\mathbb{R}^{n})\) com 
           \[
             \Vert f*p \Vert_{p}\leq \Vert f \Vert_{1} \cdot \Vert g \Vert_{p}.
           \]
       \end{theorem*}
      }

    \begin{proof*}
         Separaremos em casos: 

         \textbf{\underline{p = \(\infty\)}:} como \(y\mapsto f(x-y)\) é \(L^{1}(y)\) e \(y\mapsto g(y)\) é \(L^{\infty}(y)\), para cada vetor x em \(\mathbb{R}^{n},\) o mapa 
           \[
             y\mapsto f(x-y)\cdot g(y)
           \]
           é \(L^{1}(y)\) pela \hyperlink{holder_inequality}{\textit{Desigualdade de Hölder}} e vale que, para todo \(x\in \mathbb{R}^{n}\), 
          \begin{align*}
            |f*g(x)|\leq \int_{}^{}|f(x-y)g(y)| \mathrm{dy} & \leq \biggl(\int_{}^{}|f(x-y)| \mathrm{dy}\biggr) \Vert g \Vert_{\infty}\\ 
                                                            &=\Vert f \Vert_1 \cdot \Vert g \Vert_{\infty}\\ 
                                                            &\Rightarrow \Vert f*g \Vert_{\infty}\leq \Vert f \Vert_1 \cdot \Vert g \Vert_{\infty}.
          \end{align*}

          \textbf{\underline{p = 1}:} Neste caso, o mapa 
            \[
              (x, y)\mapsto f(x-y)g(y),\quad (x, y)\in \mathbb{R}^{n}\times \mathbb{R}^{n} 
            \]
            é mensurável; consequentemente, 
              \[
                (x, y)\mapsto |f(x-y)g(y)| \geq 0.
              \]
              Do \hyperlink{fubini_tonelli}{\textit{Teorema de Tonelli}}, 
                \[
                  \int_{}^{}\biggl[\int_{}^{}f(x-y)g(y) \mathrm{dy}\biggr] \mathrm{dx}=\int_{}^{}\biggl[\int_{}^{}|f(x-y)| \mathrm{dx} \biggr]|g(y)| \mathrm{dy}= \int_{}^{}\Vert f \Vert_1 |g(y)| \mathrm{dy} = \Vert f \Vert_1 \cdot \Vert g \Vert_1 <\infty,
                \]
                mostrando que \((x, y)\mapsto f(x-y)g(y)\) é \(L^{1}(\mathbb{R}^{n}\times \mathbb{R}^{n})\), donde, do \hyperlink{fubini_tonelli}{\textit{Teorema de Fubini}}, 
                  \[
                    f*g(x)=\int_{}^{}f(x-y)g(y) \mathrm{dy}
                  \]
                  é finito em quase todo ponto \(x\in \mathbb{R}^{n}.\) Logo, 
                 \begin{align*}
                   |f*g(x)| = |\int_{}^{}f(x-y)g(y) \mathrm{dy}| &\leq \int_{}^{}|f(x-y)||g(y)| \mathrm{dy}\\ 
                                                                 &\Rightarrow \int_{}^{}|f*g(x)| \mathrm{dx}\\ 
                                                                 &\leq \int_{}^{}\biggl[\int_{}^{}|f(x-y)||g(y)| \mathrm{dy}\biggr] \mathrm{dx}\\ 
                                                                 &= \Vert f \Vert_1 \cdot \Vert g \Vert_1
                 \end{align*}
                 pelo que foi provado acima.

                 \textbf{\underline{\(1<p<\infty\)}:}

       \end{proof*}
\end{document}

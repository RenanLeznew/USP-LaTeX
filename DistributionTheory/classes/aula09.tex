\documentclass[../distribution_theory_notes.tex]{subfiles}
\begin{document}
\section{Aula 09 - 20 de Setembro, 2024}
\subsection{Motivações}
\begin{itemize}
	\item Existência e Regularidade das Convoluções;
	\item Resultados de Densidade;
	\item Lema de Uryhson \(\mathcal{C}_{c}^{\infty}\).
\end{itemize}
\subsection{Propriedades das Convoluções I}
Para contextualizar, tudo o que disser respeito à ``mensurabilidade'' será no sentido de Borel para conjuntos e funções, e ``dx, dy, dz'' todos indicarão a medida de Lebesgue na \(\sigma \)-álgebra \(\mathcal{B}_{\mathbb{R}^{n}}\) correspondente.
\begin{tcolorbox}[
		skin=enhanced,
		title=Observação,
		fonttitle=\bfseries,
		colframe=black,
		colbacktitle=cyan!75!white,
		colback=cyan!15,
		colbacklower=black,
		coltitle=black,
		drop fuzzy shadow,
		%drop large lifted shadow
	]
	Se f é mensurável em \(\mathbb{R}^{n}\), então o mapa
	\[
		(x, y)\mapsto F(x, y)=f(x-y)
	\]
	é mensurável em \(\mathbb{R}^{n}\times \mathbb{R}^{n}\) (equipado com a \(\sigma \)-álgebra produto). O argumento para isso é notar que, definindo a função contínua
	\begin{align*}
		S: & \mathbb{R}^{n}\times \mathbb{R}^{n}\rightarrow \mathbb{R}^{n} \\
		   & (x, y)\mapsto S(x, y)=x-y,
	\end{align*}
	podemos então escrever
	\[
		F = f\circ s,
	\]
	donde segue que, se U é um aberto de \(\mathbb{R}^{n}\),
	\[
		F^{-1}(U)=S^{-1}(\underbrace{f^{-1}(U)}_{\in \mathcal{B}_{\mathbb{R}^{n}}})
	\]
	e S é \((\mathcal{B}_{\mathbb{R}^{n}\times \mathbb{R}^{n}}, \mathcal{B}_{\mathbb{R}^{n}})\)-mensurável.

	Em particular, fixado um vetor x de \(\mathbb{R}^{n}\), o mapa \(y\mapsto f(x-y)\) ser mensurável resulta em
	\[
		y\mapsto f(x-y)g(y)
	\]
	também sendo sempre que \(g\) o for, justificando a integração na definição seguinte.
\end{tcolorbox}
\begin{def*}
	Sejam f e g mensuráveis em \(\mathbb{R}^{n}\). Dado um vetor \(x\in \mathbb{R}^{n}\), o \textbf{produto de convolução de f por g em x} é dado por
	\[
		f*g(x)=\int_{}^{}f(x-y)g(y) \mathrm{dy},
	\]
	sempre que a integral existir. \(\square\)
\end{def*}
\begin{lemma*}
	Supondo que todas as integrais a seguir existam, temos:
	\begin{align*}
		 & (i)\;f*g = g*f; \text{ (comutatividade)}                               \\
		 & (ii)\;f*(g*h) = (f*g)*h; \text{ (associatividade)}                     \\
		 & (iii)\;f*(g+h) = f*g + f*h; \text{ (distributividade)}                 \\
		 & (iv)\; T_hf(x)\coloneqq f(x-h) \Rightarrow T_h(f*g)=(T_hf)*g=f*(T_hg).
	\end{align*}
	Além disso, (v) se F e G são fechados de \(\mathbb{R}^{n}\) tais que \(f=0\) em quase todo ponto de \(\mathbb{R}^{n}\setminus{F}\) e \(g=0\) em quase todo ponto de \(\mathbb{R}^{n}\setminus{G},\) então \(f*g=0\) em quase todo ponto de \(\mathbb{R}^{n}\setminus{(\overline{F+G})}\); em outras palavras,
	\[
		\mathrm{supp}(f*g)\subseteq \overline{\mathrm{supp}(f)+\mathrm{supp}(g)}.
	\]
\end{lemma*}
\begin{proof*}
	(i) É uma consequência direta do Teorema de Mudanças de Variáveis: fazendo \(h(z)=x-z\) em
	\[
		\int_{h(A)}^{}F(y) \mathrm{dy}=\int_{A}^{}(F\circ h)(z) |\det{h'(z)}|\mathrm{dz},
	\]
	temos
	\[
		f*g(x)=\int_{}^{}f(x-y)g(y) \mathrm{dy} = \int_{}^{}f(z)g(x-z) \mathrm{dz}=g*f(x).
	\]

	(ii) Segue do \hyperlink{fubini_tonelli}{\textit{Teorema de Fubini}} da seguinte forma:
	\begin{align*}
		[f*(g*h)](x) = \int_{}^{}f(x-y)[g*h(y)] \mathrm{dy} & = \int_{}^{}f(x-y)\biggl[\int_{}^{}g(y-z)h(z) \mathrm{dz}\biggr] \mathrm{dy} \\
		                                                    & = \int_{}^{}\biggl[\int_{}^{}f(x-y)g(y-z) \mathrm{dy}\biggr] h(z)\mathrm{dz} \\
		                                                    & = \int_{}^{}\biggl[\int_{}^{}f(w)g(x-w-z) \mathrm{dw}\biggr]h(z) \mathrm{dz} \\
		                                                    & = \int_{}^{}(f*g)(x-z)h(z) \mathrm{dz}  = [(f*g)*h](x),
	\end{align*}
	onde foi utilizada a mudança de variáveis de z para w pela forma
	\[
		H(w)=x-w,\quad w\in \mathbb{R}^{n}.
	\]

	(iii) Exercício.

	(iv) Com efeito, pela própria definição de convolução,
	\begin{align*}
		[T_h(f*g)](x)=f*g(x-h) & = \int_{}^{}f(x-h-y)g(y) \mathrm{dy}    \\
		                       & = \int_{}^{}(T_hf)(x-y)g(y) \mathrm{dy} \\
		                       & = [(T_hf)*g](x),
	\end{align*}
	e a outra igualdade segue aplicando o item (i), ou seja, por comutação.

	(v) Devemos mostrar que, se x não pertence a \(F+G\), então \(F\cap (x-G)=\emptyset \) e, consequentemente, \(f*g(x)=0\). Com efeito, caso contrário, existiria z em \(F\cap (x-G)\), ou seja, z é tanto um membro de F quanto pode ser escrito como \(z=x-w\), onde \(w\in G\). Assim, \(x=z+w\), com z em F e w em G, uma contradição à escolha de x. Logo, \(x\in \mathbb{R}^{n}\setminus{(F+G)}\), resultando em
	\[
		f*g(x)=0
	\]
	e, portanto, se \(f*g\) for contínua, teremos \(\mathrm{supp}(f*g)\subseteq \overline{F+G}.\) \qedsymbol
\end{proof*}

Para a existência e regularidade da convolução, comecemos com o seguinte lema:
\begin{lemma*}
	Se \(\varphi \in \mathcal{C}_{c}^{\infty}(\mathbb{R}^{n})\) e \(f\in L_{\mathrm{loc}}^{1}(\mathbb{R}^{n})\), então:
	\begin{itemize}
		\item[i)] A convolução \(\varphi * f(x)\) existe para todo x em \(\mathbb{R}^{n}\), com
		      \[
			      |\varphi *f(x)| \leq \Vert \varphi  \Vert_{\infty} \int_{x-K}^{}|f(y)| \mathrm{dy},\quad \forall x\in \mathbb{R}^{n}; \text{ e}
		      \]
		\item[ii)] A convolução \(\varphi *f(x)\) pertence a \(\mathcal{C}^{\infty}(\mathbb{R}^{n})\), com
		      \[
			      \partial^{\alpha }(\varphi *f)=(\partial^{\alpha }\varphi )*f,\quad \forall \alpha \in \mathbb{Z}_{+}^{n}.
		      \]
	\end{itemize}
\end{lemma*}
\begin{proof*}
	(i) Com efeito, dado x em \(\mathbb{R}^{n}\), se \(x-y\not\in K\coloneqq \mathrm{supp}(\varphi )\) (ou, equivalentemente, \(y\not\in x-K\)), então
	\[
		\varphi (x-y)=0 \Rightarrow \varphi (x-y)f(y)=0.
	\]
	Por outro lado, se y pertence a \(x-K\), então
	\[
		|\varphi(x-y) f(y) | \leq \Vert \varphi  \Vert_{\infty} |f(y)|,
	\]
	donde segue que
	\[
		y\mapsto \varphi (x-y)f(y)
	\]
	é integrável sobre \(x-K\), pois \(f\in L^{1}(x-K)\); consequentemente,
	\[
		\varphi *f(x)=\int_{x-K}^{}\varphi (x-y)f(y) \mathrm{dy}=\int_{\mathbb{R}^{n}}^{}\varphi (x-y)f(y) \mathrm{dy}
	\]
	existe, com
	\[
		|\varphi *f(x)|\leq \Vert \varphi  \Vert_{\infty} \int_{x-K}^{}|f(y)| \mathrm{dy},\quad \forall x\in \mathbb{R}^{n},
	\]
	terminando a prova da existência.

	(ii) Este item é uma consequência do teorema da convergência dominada de Lebesgue. Começando pela continuidade, seja
	\[
		x_{j}\substack{\mathbb{R}^{n} \\ \longrightarrow \\ x_{0}}
	\]
	e definamos:
	\[
		L\coloneqq \{x_{j}:\; j=0,1,2,\dotsc \}-K,
	\]
	que é compacto com \(\varphi (x_{j}-y)=0\) se y não pertence ao L, para todo j, tendo em vita que
	\[
		L=\bigcup_{j=0}^{\infty}(x_{j}-K)\Rightarrow \mathbb{R}^{n}\setminus{L}\subseteq \bigcap_{j=0}^{\infty}\mathbb{R}^{n}\setminus{(x_{j}-K)}\Rightarrow \varphi (x_{j}-y)=0
	\]
	sempre que y não pertence a L. Disto, obtemos
	\[
		\varphi *f(x_{j})=\int_{L}^{}\varphi (x_{j}-y)f(y) \mathrm{dy},\quad \forall j=0,1,2,\dotsc .
	\]
	Como, para todo y em L,
	\[
		\varphi (x_{j}-y)f(y)\substack{ \\ \longrightarrow \\ j\to\infty}\varphi (x_{0}-y)f(y)
	\]
	por conta da continuidade de \(\varphi \) e como
	\[
		|\varphi (x_{j}-y)f(y)|\leq \Vert \varphi  \Vert_{\infty}|f(y)|,\quad \forall y\in L,\; f\in L^{1}(L),
	\]
	o teorema da convergência dominada permite concluirmos que
	\[
		\lim_{j\to \infty}\varphi *f(x_{j})=\lim_{j\to \infty}\int_{L}^{}\varphi (x_{j}-y)f(y) \mathrm{dy}=\int_{L}^{}\varphi (x_{0}-y)f(y) \mathrm{dy}=\varphi *f(x_{0})
	\]
	e, logo, \(\varphi *f\) é contínua em qualquer \(x_{0}\) de \(\mathbb{R}^{n}\), mostrando que \(\varphi *f\in \mathcal{C}(\mathbb{R}^{n})\).

	A diferenciabilidade é provada de maneira análoga e usando indução, observando que, para \(x_{0}\in \mathbb{R}^{n}\), fixando \(j=1,2,\dotsc ,n\) e um \(\delta >0\) qualquer, temos:

	(I) Quando \(y\not\in \overline{B}(x_{0}; \delta  )-K\),
	\[
		\frac{\partial^{}\varphi }{\partial x_{j}^{}}(x-y)f(y)=0,\quad \forall x\in \overline{B}(x_{0}; \delta ).
	\]
	Logo, se \(R\coloneqq \overline{B}(x_{0}; \delta )-K\), então
	\[
		\biggl\vert \frac{\partial^{\varphi }}{\partial x_{j}^{}}(x-y)f(y) \biggr\vert \leq \biggl\Vert \frac{\partial^{}\varphi }{\partial x_{j}^{}} \biggr\Vert_{\infty}|f(y)|,\quad \forall y\in R,
	\]
	com f em \(L^{1}(R)\), mostrando que o segundo termo é uma função g em \(L^{1}(R)\).

	(II) Além do que foi dito em (I),
	\[
		\varphi *f(x)=\int_{R}^{}\varphi (x-y) \mathrm{dy}
	\]
	e, para todo x em \(\overline{B}(x_{0}; \delta )\),
	\[
		\biggl(\frac{\partial^{}\varphi }{\partial x_{j}^{}}\biggr)* f(x) = \int_{R}^{}\biggl(\frac{\partial^{}\varphi }{\partial x_{j}^{}}(x-y)\biggr)f(y) \mathrm{dy}.
	\]

	Assim, juntando (I) e (II), temos as condições para aplicar a derivada sob o sinal de integração via convergência dominada, nos dando
	\[
		\frac{\partial^{}}{\partial x_{j}^{}}(\varphi *f)(x_{0})=\int_{R}^{}\biggl(\frac{\partial^{}\varphi }{\partial x_{j}^{}}(x_{0}-y)\biggr)f(y) \mathrm{dy},
	\]
	que, pela segunda parte do (II), é exatamente \(\biggl(\frac{\partial^{}\varphi }{\partial x_{j}^{}}\biggr)* f(x)\). Pela arbitrariedade de \(x_{0}\in \mathbb{R}^{n}\), resulta que existe a derivada \(\frac{\partial^{}}{\partial x_{j}^{}}(\varphi *f)\) para todo j, e, como \(\mathrm{supp}\biggl(\frac{\partial^{}\varphi }{\partial x_{j}^{}}\biggr)\) está contido em \(\mathrm{supp}(\varphi )\), a continuidade d derivada também está garantida, ou seja, \(\varphi *f\in \mathcal{C}^{1}\). Para as próximas derivada, basta reiterar este processo, portanto provando o lema. \qedsymbol

\end{proof*}
\begin{tcolorbox}[
		skin=enhanced,
		title=Observação,
		fonttitle=\bfseries,
		colframe=black,
		colbacktitle=cyan!75!white,
		colback=cyan!15,
		colbacklower=black,
		coltitle=black,
		drop fuzzy shadow,
		%drop large lifted shadow
	]
	Se \(\varphi \in \mathcal{C}_{c}^{\infty}(\mathbb{R}^{n})\) e \(f\in \mathcal{C}^{m}(\mathbb{R}^{n})\), desde que \(|\alpha |\leq m,\) então
	\[
		\partial^{\alpha }(\varphi*f)=\varphi *(\partial^{\alpha }f).
	\]
\end{tcolorbox}

Este lema será usado para provar a densidade de \(\mathcal{C}_{c}^{\infty}(\Omega )\) em diversos espaços de funções, tal como no
\begin{theorem*}
	O espaço \(\mathcal{C}_{c}^{\infty}(\Omega )\) é denso em \(\mathcal{C}(\Omega )\) segundo a convergência uniforme em compactos.
\end{theorem*}

Antes de demonstrá-lo, precisamos do \hyperlink{uryhson_lemma}{\textit{Lema de Uryhson \(\mathcal{C}_{c}^{\infty}\)}}:
\hypertarget{uryhson_lemma}{
	\begin{theorem*}[Lema de Uryhson \(\mathcal{C}_{c}^{\infty}\)]
		Seja K um compacto contido em \(\Omega \). Existe \(\psi \in \mathcal{C}_{c}^{\infty}(\Omega )\) com \(\psi \equiv 1\) uma vizinhança de K e \(0\leq \psi \leq 1\).
	\end{theorem*}
}
\begin{proof*}
	Começamos fixando \(\varphi \in \mathcal{C}_{c}^{\infty}(\mathbb{R}^{n})\) com \(\varphi \geq 0,\; \mathrm{supp}(\varphi )=\overline{B}(0;1)\) e
	\[
		\int_{}^{}\varphi = 1.
	\]
	Como exemplo de função que satisfaça tudo isso, temos a \(\varphi_{0}\in \mathcal{C}_{c}^{\infty}(\overline{B}(0;1))\) que apresentamos antes; aí, para garantir a integral unitária, fazemos
	\[
		\varphi \coloneqq \frac{1}{c_{0}}\varphi_{0},\quad c_{0}= \int_{}^{}\varphi_{0}.
	\]

	Para cada \(\varepsilon >0\), seja \(\varphi_{\varepsilon }(x)\coloneqq \varepsilon^{-n}\varphi(x/\varepsilon )\), onde x é um vetor de \(\mathbb{R}^{n}\).
	\begin{figure}[H]
		\begin{center}
			\includegraphics[height=0.5\textheight, width=0.5\textwidth, keepaspectratio]{./Images/phi_epsilon_09.png}
		\end{center}
		\caption{conforme \(\varepsilon \) tende ao 0 pela direita, o valor da \(\varphi \) se aproxima de infinito: \(\lim_{\varepsilon \to 0^{+}}\varphi_{\varepsilon }(0)=\lim_{\varepsilon \to 0^{+}}\frac{1}{\varepsilon^{n}}=\infty.\)}
	\end{figure}

	Observe que:
	\begin{align*}
		 & (a)\; \varphi_{\varepsilon }\in \mathcal{C}_{c}^{\infty}(\mathbb{R}^{n});                                                                                                                                                                               \\
		 & (b)\; \mathrm{supp}(\varphi_{\varepsilon })=\overline{B}(0; \varepsilon );                                                                                                                                                                              \\
		 & (c)\; \varphi_{\varepsilon }\geq 0; \text{ e}                                                                                                                                                                                                           \\
		 & (d)\; \int_{}^{}\varphi_\varepsilon(x) \mathrm{dx}=\int_{}^{}\varepsilon^{-n}\varphi \biggl(\frac{x}{ \varepsilon }\biggr) \mathrm{dx} \underbrace{=}_{\mathclap{h(y)=\varepsilon y;\; \det{h'(y)}=\varepsilon^{n}}} \int_{}^{}\varphi (y) \mathrm{dy}.
	\end{align*}

	Além disso, fixemos \(\delta_{0}>0\) com \(\delta_{0}< d(K, \partial \Omega )\), porque \(K\cap \partial \Omega =\emptyset \) com K compacto e \(\partial \Omega \) fechado, e consideremos o conjunto
	\[
		K_{0}=\{x\in \Omega :\; d(x, K)\leq \delta_{0}\} = \bigcup_{a\in K}^{}\overline{B}(a; \delta_{0}),
	\]
	o qual está contido em \(\Omega \) e contém o K.
	\begin{figure}[H]
		\begin{center}
			\includegraphics[height=0.5\textheight, width=0.5\textwidth, keepaspectratio]{./Images/delta_from_k_09.png}
		\end{center}
		\caption{o K é o conjunto dos pontos de \(\Omega \) a uma distância não maior que \(\delta_{0}\) do conjunto K.}
	\end{figure}

	Agora, consideremos também \(f=\chi_{K_{0}}\); note que
	\[
		f\in L^{1}(\mathbb{R}^{n})\hookrightarrow L_{\mathrm{loc}}^{1}(\mathbb{R}^{n}),
	\]
	donde, para todo \(\varepsilon >0\),
	\[
		\varphi_{\varepsilon }*f\in \mathcal{C}^{\infty}(\mathbb{R}^{n}).
	\]
	Expandindo a conta da convolução de \(\varphi_\varepsilon \) com f, segue o seguinte:
	\begin{align*}
		\varphi_\varepsilon *f(x)=\int_{}^{}\varphi_\varepsilon (x-y)f(y) \mathrm{dy} & = \int_{K_{0}}^{}\varphi_\varepsilon (x-y)f(y) \mathrm{dy}                        \\
		                                                                              & = \int_{K_{0}}^{}\varphi_\varepsilon (x-y) \mathrm{dy}                            \\
		                                                                              & = \int_{K_{0}\cap (x-B_\varepsilon (0))}^{}\varphi_\varepsilon (x-y) \mathrm{dy}.
	\end{align*}
	pois \(\mathrm{supp}(\varphi_\varepsilon )=\overline{B}(0;\varepsilon =\overline{B}(0; \varepsilon ))\). Logo, \(x-y\) não pertence à bola fechada \(\overline{B}_\varepsilon (0)\), o que resulta em \(\varphi_\varepsilon (x-y)=0\), ou seja,
	\[
		y\not\in \overline{B}(0; \varepsilon )\Rightarrow \varphi_\varepsilon (x-y)=0.
	\]
	Agora, basta diminuir \(\varepsilon \) o suficiente! Para começar, como uma ``etapa zero'', repare que podemos dar continuidade às formas equivalentes à convolução acima de onde paramos:
	\[
		\varphi_\varepsilon *f(x)=\int_{K_{0}\cap (x-)\overline{B}(0; \varepsilon )}^{}\varphi_\varepsilon (x-y) \mathrm{dy} = \int_{K_{0}\cap \overline{B}(x; \varepsilon )}^{} \varphi_\varepsilon (x-y)\mathrm{dy},
	\]
	tal que, se \(B_\varepsilon (x)\) for um subconjunto de \(K_{0}\), teremos
	\[
		\varphi_\varepsilon *f(x)=\int_{\overline{B}(x; \varepsilon )}^{}\varphi_\varepsilon (x-y) \mathrm{dy}=1.
	\]
	Com respeito ao primeiro passo, diminuiremos \(\varepsilon \) a fim de garantir a compacidade de \(\mathrm{supp}(\varphi_\varepsilon *f)\) e a continência dele em \(\Omega \). Note que, se \(\delta_0<\delta_1<d(x, \mathbb{R}^{n}\setminus{\Omega }) \), então
	\[
		K_1\coloneqq \bigcup_{a\in K}^{}\overline{B}(a;\delta_1)
	\]
	é tal que \(K_{0}\subseteq K_1\subseteq \Omega \) pelo mesmo raciocínio usado para \(K_{0}\).
	\begin{figure}[H]
		\begin{center}
			\includegraphics[height=0.5\textheight, width=0.5\textwidth, keepaspectratio]{./Images/covering_k_09.png}
		\end{center}
		\caption{o \(K_1\) agora se situa entre \(K_0\) e \(\Omega\).}
	\end{figure}

	Também, x não pertence a \(K_1\), tendo em vista que \(d(\mathbb{R}^{n}\setminus{K}, K)\geq \delta_1\), do que segue que, quando \(0<\varepsilon <\delta_1-\delta_0,\)
	\[
		\varphi_\varepsilon *f(x)=0,
	\]
	pois
	\[
		K_{0}\cap (x-\overline{B}(0; \varepsilon ))=\emptyset;
	\]
	caso contrário, se \(y\in K_{0}\cap (x-\overline{B}(0; \varepsilon ))\), então \(y\in K_{0}\) e \(y=x-z\) com \(|z|\leq \varepsilon \), que resultaria em
	\[
		|y-a|\leq \delta_{0}
	\]
	para algum \(a\in K\). Logo,
	\[
		|x-a|\leq |x-y|+|y-a|\leq |z|+\delta_{0}\leq \varepsilon +\delta_{0}<(\delta_1-\delta_0)+\delta_0=\delta_1,
	\]
	que contradiz o fato de x não pertencer a \(K_1\) a partir do momento que isto requer que \(d(x, a)>\delta_1\) para todo a em K. Isto mostra duas coisas: que \(0<\varepsilon <\delta_1-\delta_0\), tal que \(\mathrm{supp}(\varphi_\varepsilon *f)\) é um compacto contido em \(K_1\), e que, consequentemente,
	\[
		\varphi_\varepsilon *f\in \mathcal{C}_{c}^{\infty}(\Omega ).
	\]

	Quanto ao segundo passo, queremos encontrar o aberto \(U\) de \(\Omega \) contendo K tal que
	\[
		\varphi_\varepsilon *f\equiv 1
	\]
	em U. Para isso, precisamos escolher \(0<\varepsilon_0\) para o qual teremos
	\[
		x-\overline{B}_{\varepsilon_{0}}(0)\subseteq K_{0}
	\]
	com x em U. Ora, se a é um membro de K e \(y=x-z\in x-\overline{B}_{\varepsilon_{0} }(0)\), onde \(|z|\leq \varepsilon_{0}\), então
	\[
		|y-a|\leq |x-a|+|z|\leq |x-a|+\varepsilon_{0} \leq \delta_{0},
	\]
	o qual pode ser obtido tomando
	\[
		U\coloneqq \bigcup_{a\in K}^{}B(a; \varepsilon_{0})
	\]
	se \(2\varepsilon_{0}=\delta_{0}\). Logo, com este \(\varepsilon_{0}\), caso x pertença a U, seguirá que
	\[
		x-\overline{B}_{\varepsilon_{0}}(0)\subseteq K_{0}\Rightarrow (\varphi_{\varepsilon_{0}}*f)(x)=\int_{x-\overline{B}_{\varepsilon }(0)}^{}\varphi_{\varepsilon }(x-y) \mathrm{dy}=\int_{|z|\leq \varepsilon }^{}\varphi_{\varepsilon }(z) \mathrm{dz}=1.
	\]

	Portanto,
	\[
		0<\varepsilon <\min\limits_{}\biggl\{\frac{\delta_{0}}{2}, \delta_1-\delta_0\biggr\} \Rightarrow \psi \coloneqq \varphi_\varepsilon *f\in \mathcal{C}_{c}^{\infty}(\Omega )
	\]
	e \(\psi\equiv 1\) em \(U=\bigcup_{a\in K}^{}B(a; \varepsilon )\subseteq \Omega \), com
	\[
		0\leq \psi (x)=\int_{}^{}\varphi_\varepsilon (x-y)f(y) \mathrm{dy}\leq \int_{}^{}\varphi_\varepsilon (x-y) \mathrm{dy}=1,
	\]
	pois, sendo f a característica de \(K_{0}\), temos \(0\leq f\leq 1\), finalizando a prova. \qedsymbol

\end{proof*}

\begin{tcolorbox}[
		skin=enhanced,
		title=Observação,
		fonttitle=\bfseries,
		colframe=black,
		colbacktitle=cyan!75!white,
		colback=cyan!15,
		colbacklower=black,
		coltitle=black,
		drop fuzzy shadow,
		%drop large lifted shadow
	]
	A topologia limite indutivo em \(\mathcal{C}_{c}^{\infty}(\Omega )\) \textit{não é metrizável}! Com efeito, seja d uma métrica em \(\mathcal{C}_{c}^{\infty}(\Omega )\) tal que
	\[
		\varphi_{j}\substack{\mathcal{C}_{c}^{\infty}(\Omega ) \\ \longrightarrow \\ }0 \Longleftrightarrow d(\varphi_{j}, 0)\substack{ \\ \longrightarrow \\ j\to \infty}0.
	\]
	Dado um esgotamento \(\{K_{j}\}_{j}\) de \(\Omega \), podemos escolher, para cada j, \(\varphi_{j}\in \mathcal{C}_{c}^{\infty}(\Omega )\) com \(\varphi_{j}\equiv 1\) em \(K_{j}.\) Assim, fixado j,
	\[
		\lim_{\varepsilon \to 0^{+}}\varepsilon \varphi_{j}=0
	\]
	em \(\mathcal{C}_{c}^{\infty}(\Omega )\). Portanto, escolhemos \(\varepsilon_{j}\) tal que
	\[
		d(\varepsilon_{j}\varphi_{j}, 0)< \frac{1}{j},
	\]
	donde teremos
	\[
		\varepsilon_{j}\varphi_{j}\to 0
	\]
	segundo d, mas não em \(\mathcal{C}_{c}^{\infty}(\Omega )\), pois
	\[
		\bigcup_{j}^{}\mathrm{supp}(\varepsilon_{j}\varphi_{j})=\Omega
	\]
	não é compacto.

\end{tcolorbox}
\subsection{Apêndices}
\subsubsection{Consequências da Convergência Dominada}
Seja \((X, \mathcal{M}, \mu )\) um espaço de medida e \(f:X\times [a, b]\rightarrow \mathbb{C}\) tal que, para cada t em \([a, b],\; f(\cdot , t)\in L^{1}(\mu )\). Definimos
\begin{align*}
	F: & [a, b]\rightarrow \mathbb{C}                                      \\
	   & t \longmapsto F(t)\coloneqq \int_{X}^{}f(x, t) \mathrm{d\mu (x)}.
\end{align*}
Nessas condições, temos:

(a) Se \(t_{n}\to t_{0}\) em \([a, b]\), \(f(x, t_{n})\rightarrow f(x, t_{0})\) para todo x e existe g em \(L^{1}(\mathcal{N})\) tal que
\[
	| f(x, t_{n}) |\leq g(x),\quad \forall x\in X,\; \forall n\in \mathbb{N},
\]
então \(F(t_{n})\to F(t_{0})\). Em particular, \(f(x, \cdot )\) é contínua em \(t_{0}\) para todo x e
\[
	| f(x, t) |\leq g(x),\quad \forall x, t.
\]
Consequentemente, F é contínua em \(t_{0}.\)

(b) Se existir a derivada parcial de f com respeito a t para todo x de X e t de \([a, b]\), e existir uma g em \(L^{1}(\mu )\) com
\[
	\biggl\vert \frac{\partial^{}F}{\partial t^{}}(x, t) \biggr\vert\leq g(x),\quad \forall x, t,
\]
então F é derivável com
\[
	F'(t) = \int_{X}^{}\frac{\partial^{}F}{\partial t^{}}(x \Theta ) \mathrm{d}\mu (x).
\]

\begin{crl*}
	Quando \(f:X \times U\rightarrow \mathbb{C},\; U\subseteq \mathbb{R}^{n}\) aberto, para concluir que
	\[
		\frac{\partial^{}F}{\partial t_{j}^{}}(t_{0}) = \int_{}^{}\frac{\partial^{}f}{\partial t_{j}^{}}(x, t_{0}) \mathrm{d}\mu (x)
	\]
	nem determinado \(t_{0}\), basta tomar \(f_{0}(x, s),\; | s |\leq \delta \), definido por
	\[
		f_{0}(x, s)\coloneqq f(x, t_{0}+se_{j})
	\]
	e observar que
	\[
		\frac{\partial^{}f_{0}}{\partial s^{}}(x, 0) = \frac{\partial^{}f}{\partial t_{j}^{}}(x, t_{0}),\quad \forall x\in X.
	\]
\end{crl*}

\subsubsection{Operadores Integrais}
Seja \(K\in \mathcal{M}\otimes \mathcal{N}\)-mensurável e \((X, \mathcal{M})\) e \((Y, \mathcal{N})\) \(\sigma \)-finitos tais que existe c positivo com
\[
	\int_{X}^{}| K(x, y) | \mathrm{d}\mu (x) \leq c \quad\&\quad \int_{Y}^{}| K(x, y) | \mathrm{d}\nu (y)\leq c.
\]
Pondo \(Tf(x) = \int_{Y}^{}K(x, y)f(y) \mathrm{d}\nu (y),\; f\in L^{p}(\nu )\), tem-se
\[
	Tf\in L^{p}(\mu )
\]
com
\[
	\Vert Tf \Vert_{p}\leq c \Vert f \Vert_{p}.
\]

No caso da convolução, isto significa que, se \(g\in L^{1}\) e \(K(x, y) = g(x-y)\), então
\[
	c = \Vert g \Vert_{1}
\]
acima. O caso particular frequentemente usado é quando se tem \(X = \Omega \) aberto com \(\partial \Omega \in \mathcal{C}^{\infty}\) e \(Y = \partial \Omega \); daí, dada \(f\in L^{p}(\partial \Omega )\), segue que \(Tf\in L^{p}(\Omega )\) e, mais ainda, \(u=Tf\) resolve uma EDP com \(u=f\) em \(\partial \Omega \) -- noutras palavras, o Problema de Dirichlet:
\[
	\left\{\begin{array}{ll}
		P(D)u = 0 \text{ em }\Omega      \\
		u = f \text{ em }\partial \Omega \\
	\end{array}\right.
\]
\begin{example}
	Dada \(f\in \mathcal{C}(\{\gamma \})\), onde \(\gamma :[a, b]\rightarrow \mathbb{C}\) é uma curva fechada (\(\gamma (a) = \gamma (b)\)) com \(\{\gamma \}=\partial \Omega \) e
	\[
		u(z) = Tf(z) = \frac{1}{2\pi i}\int_{\gamma }^{}\frac{f(w)}{w-z} \mathrm{dw} = g*f(z),\quad g(\zeta ) = \frac{1}{2\pi i \zeta },
	\]
	tem-se u holomorfa em \(\Omega \) com \(u = f\) em \(\{\gamma \},\) onde ser holomorfa significa resolver um sistema de EDP's, especificamente as Equações de Cauchy-Riemann
	\begin{figure}[H]
		\begin{center}
			\includegraphics[height=0.5\textheight, width=0.5\textwidth, keepaspectratio]{./Images/closed_path_09.png}
		\end{center}
		\caption{ilustração de \(\gamma \) com seu contorno sendo o bordo de \(\Omega \).}
	\end{figure}
\end{example}

\end{document}

\documentclass[../distribution_theory_notes.tex]{subfiles}
\begin{document}
\section{Aula 12 - 04 de Outubro, 2024}
\subsection{Motivações}
\begin{itemize}
 \item Exemplos do Transposto Formal.
\end{itemize}
\subsection{O Transposto Formal: recapitulação e exemplos}
O objetivo da última aula foi apresentar um conceito que permitisse estender as operações realizadas com funções para distribuições, baseado no chamado \textit{transposto formal}: partindo de uma 
aplicação linear contínua \(T:\mathcal{C}_{c}^{\infty}(\Omega )\rightarrow \mathcal{C}_{c}^{\infty}(\Omega ')\), associada à qual existe um operador linear \(T':\mathcal{C}_{c}^{\infty}(\Omega' )\rightarrow \mathcal{C}_{c}^{\infty}(\Omega )\), também contínuo, tal que 
  \[
    \int_{\Omega '}^{} (T\varphi )(y)\psi (y) \mathrm{dy} = \int_{\Omega }^{}\varphi (x)(T'\psi )(x) \mathrm{dx},\quad \forall \varphi \in \mathcal{C}_{c}^{\infty}(\Omega )\;\&\; \psi \in \mathcal{C}_{c}^{\infty}(\Omega '),
  \]
  podemos definir a ``extensão'' \(\overline{T}:\mathcal{D}'(\Omega )\rightarrow \mathcal{D}'(\Omega ') \) pondo: 
    \[
      \langle \overline{T}u, \psi  \rangle\coloneqq \langle u, T\psi  \rangle, \quad \psi \in \mathcal{C}_{c}^{\infty}(\Omega '),\; u\in \mathcal{D}'(\Omega ),
    \]
    que, a rigor, estende a T levando em conta as identificações 
      \[
        T_{\Omega }: L_{\mathrm{loc}}^{1}(\Omega )\rightarrow \mathcal{D}'(\Omega) \;\&\; T_{\Omega'}:L_{\mathrm{loc}}^{1}(\Omega')\rightarrow \mathcal{D}'(\Omega '),
      \]
      dando para nós 
        \[
          \overline{T}(T_{\Omega }\varphi )= T_{\Omega'}(T\varphi ),\quad \forall \varphi \in \mathcal{C}_{c}^{\infty}(\Omega ),
        \]
        mas ficaria muito exaustivo constantemente aplicar este rigor na prática, então escrevemos apenas \(\overline{T}\varphi = T\varphi \). Sendo assim, usamos o mesmo símbolo T para representar a extensão \(\overline{T}.\)

        Queremos, a seguir, continuar a desenvolver os exemplos de uso da transposta formal, como estávamos fazendo ao fim da última aula. O primeiro que vimos foi a multiplicação por uma função f de classe \(\mathcal{C}^{\infty}(\Omega )\): para \(\varphi \in \mathcal{C}_{c}^{\infty}(\Omega )\), obtivemos 
          \[
            \langle f u, \varphi  \rangle=\langle u, f \varphi  \rangle,
          \]
          e, em particular, \(f\delta =f(0)\delta \); assim, se \(f(0)=1\), então \(f\delta =\delta \) e, se \(f(0)=0,\) \(f\delta =0.\) No fim, não especulamos os casos em que \(f\in \mathcal{C}^{m}(\Omega )\) para m finito, ou o caso em que f em si era uma distribuição, seguindo em direção à derivação de distribuições.

          O provável mais importante exemplo que vimos foi com respeito à derivação de distribuições, tanto pelo aspecto técnico quanto teórico, onde observamos que 
            \[
              \biggl\langle \frac{\partial^{}u}{\partial x_{j}^{}}, \varphi  \biggr\rangle =-\biggl\langle u, \frac{\partial^{}\varphi }{\partial x_{j}^{}} \biggr\rangle,\quad u\in \mathcal{D}'(\Omega ),\; \varphi \in \mathcal{C}_{c}^{\infty}(\Omega ),\; j=1,2,\dotsc ,n.
            \]

            Assim, derivar uma distribuição cria uma nova distribuição, tal que obtivemos uma maneira de gerar distribuições. Além disso, esta definição pode ser aplicada recorrentemente, resultando na expressão geral para derivadas parciais de qualquer ordem: 
              \[
                \langle \partial^{\alpha }u, \varphi  \rangle= (-1)^{| \alpha  |}\langle u, \partial^{\alpha }\varphi  \rangle,\quad u\in \mathcal{D}'(\Omega ),\; \varphi\in \mathcal{C}_{c}^{\infty}(\Omega ),\; \alpha \in \mathbb{Z}_{+}^{n}.
              \]
              Por definição, \(\frac{\partial^{}}{\partial x_{j}^{}} \) é uma extensão da derivada parcial do cálculo, no sentido que, se \(u\in \mathcal{C}^{1}(\Omega )\), então \(\frac{\partial^{}u}{\partial x_{j}^{}} \) no sentido das distribuições e no sentido clássico coincidem (tente pensar numa recíproca para isso).
               \begin{tcolorbox}[
               skin=enhanced,
               title=Observação,
               fonttitle=\bfseries,
             colframe=black,
               colbacktitle=cyan!75!white, 
               colback=cyan!15,
               colbacklower=black,
             coltitle=black,
               drop fuzzy shadow,
               %drop large lifted shadow
               ]
               Quando u é uma medida boreliana localmente finita em \(\Omega \), absolutamente contínua com respeito a m ( \(\mu << m\)) e f é sua derivada de Radon-Nikodym, então 
                 \[
                   \biggl\langle \frac{\partial^{}}{\partial ^{}},  \biggr\rangle,
                 \]
                 isto é, 
                   \[
                     \frac{\partial^{}\mu }{\partial x_{j}^{}} = \frac{\partial^{}f}{\partial x_{j}^{}},
                   \]
                   que, apesar de parecer contra-intuitivo, podemos perceber que não, pois é como se f e \(\mu \) fossem a mesma coisa. Medite sobre isso!

                   Com efeito, se \(\mathrm{d}\mu  = f \mathrm{d}m\), então para g de classe \(\mathcal{L}^{1}(\mu )\), tem-se 
                  \begin{align*}
                    \int_{}g d\mu_{} &= \int_{}gf \mathrm{d}x \\ 
                                     &\Rightarrow \biggl\langle \frac{\partial^{}\mu }{\partial x_{j}^{}}, \varphi  \biggr\rangle = - \biggl\langle \mu , \frac{\partial^{}\varphi }{\partial x_{j}^{}} \biggr\rangle\\ 
                                     &= - \int_{}^{}\frac{\partial^{}\varphi }{\partial x_{j}^{}} \mathrm{d}\mu \\ 
                                     &= - \int_{}\frac{\partial^{}\varphi }{\partial x_{j}^{}} f\mathrm{d}\mu_{}\\ 
                                     &= - \biggl\langle f, \frac{\partial^{}\varphi }{\partial x_{j}^{}} \biggr\rangle\\ 
                                     &= \biggl\langle \frac{\partial^{}\varphi }{\partial x_{j}^{}},  \varphi \biggr\rangle,
                  \end{align*}
                  sendo \(\varphi \) uma função teste. Portanto, a derivada da medida é a derivada de sua derivada Radon-Nikodym. 
               \end{tcolorbox}

               Pela observação acima, percebemos que a derivada de uma função resulta numa medida, revelando que enfraquecermos a maneria necessária de medir a taxa de variação. Integrando por partes, vimos que 
                 \[
                   \int_{\varepsilon }^{b}\log^{}{x}\varphi '(x) \mathrm{d}x = \varphi (x)\log^{}{x}\biggl|_{\varepsilon }^{b}\biggr. - \int_{\varepsilon }^{b}\frac{\varphi (x) }{x} \mathrm{d}x = -\varphi (\varepsilon )\log^{}{\varepsilon }- \int_{\varepsilon }^{b}\frac{\varphi (x)}{x} \mathrm{d}x
                 \]
                 e, mais ainda, 
                   \[
                     \int_{-b}^{-\varepsilon }\log^{}{(-x)}\varphi'(x) \mathrm{d}x = \log^{}{(-x)}\varphi (x)\biggl|_{-b}^{-\varepsilon }\biggr. - \int_{-b}^{-\varepsilon }\frac{\varphi (x)}{x} \mathrm{d}x = \log^{}{(\varepsilon )} \varphi (-\varepsilon )- \int_{-b}^{-\varepsilon }\frac{\varphi (x)}{x} \mathrm{d}x,
                   \]
                   donde segue 
                     \[
                       \biggl\langle \frac{\mathrm{d}}{\mathrm{d}x}\log^{}{| x |}, \varphi  \biggr\rangle = - \lim_{\varepsilon \to 0^{+}}\biggl\{\biggl[\log^{}{\varepsilon }(\varphi (-\varepsilon )-\varphi(\varepsilon ))\biggr]-\biggl[\int_{-b}^{-\varepsilon }\frac{\varphi(x)}{x} \mathrm{d}x + \int_{\varepsilon }^{b}\frac{\varphi (x)}{x} \mathrm{d}x\biggr]\biggr\}
                     \]
                     e, portanto, o limite existe. Como, pelo TVM, 
                       \[
                         | \varphi (\varepsilon )-\varphi (-\varepsilon ) | \leq \Vert \varphi  \Vert_{\infty} 2\varepsilon \;\&\; \lim_{\varepsilon \to 0^{+}} \varepsilon \log^{}{\varepsilon }=0,
                       \]
                       tal que o limite da primeira parcela acima é zero. Consequentemente, o limite da segunda deve existir, isto é, 
                         \[
                           \biggl\langle \frac{\mathrm{d}}{\mathrm{d}x} \log^{}{| x |}, \varphi  \biggr\rangle = \lim_{\varepsilon \to 0^{+}} \int_{| x |\geq \varepsilon }^{}\frac{\varphi (x)}{x} \mathrm{d}x,\; \varphi \in \mathcal{C}_{c}^{\infty}(\mathbb{R}).
                         \]
                          distribuição resultante disso é a chamada \textbf{Valor Principal da Função Inversa}, indicada por \(\mathrm{pv}\bigl(\frac{1}{x}\bigr)\) (ou \(\mathrm{vp}\) em português). Assim, 
                            \[
                              \biggl\langle \mathrm{pv}\bigl(\frac{1}{x}\bigr), \varphi  \biggr\rangle = \lim_{\varepsilon \to 0^{+}} \int_{| x |\geq \varepsilon }^{}\frac{\varphi (x)}{x} \mathrm{d}x = \biggl\langle \frac{\mathrm{d}}{\mathrm{d}x}\log^{}{|x|}, \varphi \biggr\rangle.
                            \]

                            Muito bonito, mas o que isso tem de importante ou de não trivial? O primeiro ponto é que, para uma função \(f:\mathbb{R}\rightarrow \mathbb{R}\) contínua qualquer, não se sabe se \(\int_{-b}^{b}\frac{f(x)}{x} \mathrm{d}x\) existe no sentido de Lebesgue (o módulo do argumento). Com isso, o sentido de Lebesgue não é uma alternativa válida; porém, como em \([-b, -\delta ]\) e em \([\varepsilon , b]\), onde \(\delta \) e \(\varepsilon \) são positivos, as integrais existem em ambos os sentidos e coincidem... Neste sentido, a tentativa mais viável é tentar uma integral imprópria de Riemann: 
                              \[
                                \int_{-b}^{b}\frac{f(x)}{x} \mathrm{d}x = \lim_{\varepsilon , \delta \to 0^{+}} \biggl[\int_{-b}^{-\delta }\frac{f(x)}{x} \mathrm{d}x + \int_{\varepsilon }^{b} \frac{f(x)}{x} \mathrm{d}x\biggr],
                              \]
                              que seria um critério equivalente à convergência condicional de séries, na qual a convergência depende de uma escolha na ordem da somatória. A conclusão, portanto, é que, quando \(\varphi \in \mathcal{C}_{c}^{\infty}(\mathbb{R})\) e \(\delta =\varepsilon \) no limite acima, ele existe. Uma outra parte muito interessante é que, com isso, ganhamos uma base para escolher uma maneira de somar a série \(\sum\limits_{\substack{n=-\infty \\ n\neq 0}}^{\infty}\frac{1}{n}\) a fim de fazê-la convergir.

                             \begin{figure}[H]
                             \begin{center}
                             \includegraphics[height=0.4\textheight, width=0.4\textwidth, keepaspectratio]{./Images/modulus_smaller_epsilon_12.png}
                             \end{center}
                             \caption{Quando \(| x |\leq \varepsilon \), a integral é \(\int_{| x |\leq \varepsilon }^{}\varphi(x)/x \mathrm{d}x=0\).}
                             \end{figure}

                             \begin{figure}[H]
                             \begin{center}
                             \includegraphics[height=0.4\textheight, width=0.4\textwidth, keepaspectratio]{./Images/modulus_greater_epsilon_12.png}
                             \end{center}
                             \caption{O modo como \(\varphi (x)/x\) interage à direita e à esquerda da origem faz o limite existir (há cancelamento de termos).}
                             \end{figure}
\end{document}

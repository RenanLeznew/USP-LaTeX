\documentclass[../distribution_theory_notes.tex]{subfiles}
\begin{document}
\section{Aula 15 - 14 de Outubro, 2025}
\subsection{Motivações}
\begin{itemize}
	\item Continuando a Convergência em \(\mathcal{D}'\);
	\item Produto de Convoluções.
\end{itemize}
\subsection{Continuando a Convergência em \(\mathcal{D}'\)}
Na aula de sexta, estudamos a segunda parte do problema relativo a determinar a ``estrutura'' das distribuições com derivadas nulas.

Vimos que, quando \(n=1\) e \(u\in \mathcal{D}'(I)\) tem \(u'=0\), sendo I um intervalo; então, existe \(c\in \mathbb{C}\) tal que
\[
	\langle u, \varphi  \rangle = c \int_{I}^{}\varphi  \mathrm{d}, \quad \forall \varphi \in \mathcal{C}_{c}^{\infty}(I),
\]
isto é, u provém da função \(f=c\) constante \(L_{\mathrm{loc}}^{1}(I)\). No caso geral, considera-se \(U\subseteq \mathbb{R}^{n}\)
aberto com \(x\in U\) e \(I\subseteq \mathbb{R}\) um intervalo com t e, supondo u em \(\mathcal{D}'(U\times I)\) com \(\partial_t u = 0\),
concluímos que existe uma distribuição \(u_{0}\) em \(\mathcal{D}'(U)\) tal que
\[
	\langle u, \varphi  \rangle = \biggl\langle u_{0}, \int_{I}^{}\varphi (\cdot , t) \mathrm{d}t \biggr\rangle,\quad \varphi \in \mathcal{C}_{c}^{\infty}(U\times I).
\]
Nesse sentido, u depende apenas das variáveis x em U e é independente de t.

Em seguida, começaremos a tratar da convergência em \(\mathcal{D}'\), sendo ele o dual topológico do TVS \(\mathcal{C}_{c}^{\infty}(\Omega )\) com o limite indutivo; ele estaŕa sempre munido da topologia fraca-*, segundo a qual uma sequência \(\{u_{j}\}_{j}\) converge
para uma \(u\in \mathcal{D}'(\Omega )\) se, e somente se, para todo \(\varphi \in \mathcal{C}_{c}^{\infty}(\Omega )\),
\[
	\langle u_{j}, \varphi  \rangle \substack{ \\ \longrightarrow \\ j\to \infty}\langle u, \varphi  \rangle.
\]
Vimos que, por meio dessa topologia, dois resultados merecem destque: o primeiro é que podemos definir a transformação \(T:L_{\mathrm{loc}}^{1}(\Omega )\rightarrow \mathcal{D}'(\Omega )\) que identifica uma \(f\in L_{\mathrm{loc}}^{1}\) com a distribuição
\[
	\langle Tf, \varphi  \rangle = \int_{\Omega }^{}f\varphi,
\]
e é contínua, permitindo que praticamente todas as convergências usuaris no espaço de funções conhecidas ocorram também em \(\mathcal{D}'.\)

\begin{tcolorbox}[
		skin=enhanced,
		title=Observação,
		fonttitle=\bfseries,
		colframe=black,
		colbacktitle=cyan!75!white,
		colback=cyan!15,
		colbacklower=black,
		coltitle=black,
		drop fuzzy shadow,
		%drop large lifted shadow
	]
	Por outro lado, sempre que \(\mathcal{C}_{c}^{\infty}(E)\) for mergulhado em E, teremos também \(E'\hookrightarrow \mathcal{D}'(\Omega )\) com a topologia fraca-*, o que permite vermos em \(\mathcal{D}'\) as converências fracas-* de várias duais de TVS, por exemplo as \(L^{p'}(\Omega )\) como dual de \(L^{p}(\Omega )\), onde \(1\leq p <\infty\), ou, ainda,
	a de \(\mathcal{E}'(\Omega )\) como o dual de \(\mathcal{C}^{\infty}(\Omega )\hookleftarrow \mathcal{C}_{c}^{\infty}(\Omega )\).
\end{tcolorbox}
A segunda coisa é que a continuidade de \(\overline{T}:\mathcal{D}'(\Omega )\rightarrow \mathcal{D}'(\Omega ')\), a extensão de uma aplicação linear \(T:\mathcal{C}_{c}^{\infty}(\Omega )\rightarrow \mathcal{C}_{c}^{\infty}(\Omega ')\) que admite um transposto formal \(T'.\)
Um caso particular extremamente importante dessa situação é o caso das ODP, denotadas por \(P(x, D):\mathcal{D}'(\Omega )\rightarrow \mathcal{D}'(\Omega )\), que são automaticamente contínuas sempre que
\[
	u_{j}\substack{ \mathcal{D}'(\Omega )\\ \longrightarrow \\ j\to \infty}u \Rightarrow P(x, D)u_{j}\substack{\mathcal{D}'(\Omega ) \\ \longrightarrow \\ j\to \infty}P(x, D)u,
\]
por exemplo, sempre que \(u_{j}\) converge fraco-* para u,
\[
	\frac{\partial^{}u_{j}}{\partial x_{k}^{}}\substack{ \mathcal{D}'\\ \longrightarrow \\ j\to \infty}\frac{\partial^{}u}{\partial x_{k}^{}}.
\]

A aula terminou quando estávamos dando exemplos de sequências de funções que sabemos não convergir nos espaços usuais que habitam, mas que
trivialmente convergem em \(\mathcal{D}'(\Omega )\). Além do mais, convergem para limites que podem contradizer nossa intuição.

Hoje, completaremos a lista de exemplos sobre a convergência em \(\mathcal{D}'(\Omega )\), destacando:
\begin{itemize}
	\item[i)] A família regularizante do começo, \(\varphi_{\varepsilon }\substack{\varepsilon \to 0^{+} \\ \longrightarrow \\ }\delta,\; (\varphi_\varepsilon )_{\varepsilon > 0} \);
	\item[ii)] Condições mais gerais para que isso ocorra para uma sequência \(\{f_{j}\}_{j}\) de funções integráveis; e
	\item[iii)] Teorema: a série de Fourier de qualquer \(f\in L(-\pi , \pi )\) converge para f em \(\mathcal{D}'(-\pi , \pi )\).
\end{itemize}

Um dos resultados mais importantes sobre a convergênia em \(\mathcal{D}'\) é a densidade das funções teste em \(\mathcal{D}'\) e isso passa pelo caso especial da densidade de \(\mathcal{C}_{c}^{\infty}(\mathbb{R}^{n})\) em \(\mathcal{D}'(\mathbb{R}^{n})\),
assim como foi com \(\mathcal{C}(\Omega )\) e \(L^{1}(\Omega )\), primeiro prova-se a densidade em \(\mathcal{C}(\mathbb{R}^{n})\) e em \(L^{1}(\mathbb{R}^{n})\) das convoluções \(\varphi_{\varepsilon }*f\), que precisa de uma extensão do conceito de convolução
sendo um dos fatores uma distribuição e o outro uma função teste.

\subsection{Produto de Convolução - Segunda Parte}

Dados \(\varphi , \psi \in \mathcal{C}_{c}^{\infty}(\mathbb{R}^{n})\), partindo do modelo
\[
	\varphi *\psi (x) = \int_{\mathbb{R}^{n}}^{}\varphi(x-y)\psi (y) \mathrm{d}y = \int_{\mathbb{R}^{n}}^{}\varphi(y)\psi (x-y) \mathrm{dy}
\]
com x em \(\mathbb{R}^{n}\), e, adotando a notação \(\check{\varphi }\coloneqq \varphi \circ R\), onde \(R(y) = -y\), podemos escrever
\[
	\varphi * \psi (x) = \int_{\mathbb{R}^{n}}^{}\varphi (y)\tau_{x}(\psi )^{\check}(y) \mathrm{d}y,
\]
seguindo uma maneira de estender \(\varphi * \psi (x)\) com \(\varphi \in \mathcal{C}_{c}^{\infty}(\mathbb{R}^{n})\) para uma \(u\in \mathcal{D}'(\mathbb{R}^{n})\), pondo:
\[
	u*\psi (x)\coloneqq \langle u, \tau_{x}(\psi )^{\check} \rangle,\quad x\in \mathbb{R}^{n}\;\&\; \psi \in \mathcal{C}_{c}^{\infty}(\mathbb{R}^{n}),
\]
onde assumimos que u atua na variável y, \textit{i.e.}, \(\tau_{X}(\psi )^{\check}\) é função da variável y; quando for necessário deixar bem explícito e evitar confusões, escreveremos
\[
	\langle u(y), \tau_{x}(\psi )^{\check}(y) \rangle.
\]
\begin{tcolorbox}[
		skin=enhanced,
		title=Observação,
		fonttitle=\bfseries,
		colframe=black,
		colbacktitle=cyan!75!white,
		colback=cyan!15,
		colbacklower=black,
		coltitle=black,
		drop fuzzy shadow,
		%drop large lifted shadow
	]
	Em \(\tau_{x}(\psi )^{\check},\) primeiro refletimos e depois transladamos:
	\[
		[\tau_{x}(\psi )^{\check}](y) = \psi^{\check}(y-x) = \psi(x-y) = (\tau_{-x}\psi )(-y) = (\tau_{-x}\psi )^{\check}(y).
	\]
\end{tcolorbox}
Ao definirmo \(u*\psi (x)\), poderíamos escrever também
\[
	u * \psi (x) = \langle u, \psi (x - \cdot ) \rangle,
\]
tal que, de maneira explícita,
\[
	u*\psi (x) = \langle u(y), \psi (x-y) \rangle.
\]
Esta definição associa, a cada distribuição \(u\in \mathcal{D}'(\mathbb{R}^{n})\) e função teste \(\varphi \in \mathcal{C}_{c}^{\infty}(\mathbb{R}^{n})\), uma função
\[
	u*\varphi :\mathbb{R}^{n}\rightarrow \mathbb{C}
\]
definida em todo vetor \(x\in \mathbb{R}^{n}\).

A seguir, resumimos as propriedades dessa operação:
\begin{prop*}
	A convolução, conforme definida acima, satisfaz:
	\begin{itemize}
		\item[0)] Para toda \(\varphi \in \mathcal{C}_{c}^{\infty}(\mathbb{R}^{n})\), \(\delta * \varphi = \varphi \);
		\item[1)] Quando \(u\in L_{\mathrm{loc}}^{1}(\mathbb{R}^{n})\), as definições atuais e a antiga coincidem;
		\item[2)] A convolução \(u*\varphi \) é de classe \(\mathcal{C}^{\infty}\) com
		      \[
			      \partial^{\alpha }(u*\varphi ) = u * (\partial^{\alpha }\varphi ) = (\partial^{\alpha }u)*\varphi ,\; \forall \alpha \in \mathbb{Z}_{+}^{n}; \text{ e}
		      \]
		\item[3)] Para todas as distribuições \(u\in \mathcal{D}'(\mathbb{R}^{n})\) e \(\varphi , \psi \in \mathcal{C}_{c}^{\infty}(\mathbb{R}^{n})\), tem-se
		      \[
			      \overbrace{u}^{\mathcal{\in \mathcal{D}'}}*(\overbrace{\varphi* \psi}^{\mathclap{\in \mathcal{C}_{c}^{\infty}}} )= \underbrace{(u*\varphi )}_{\mathclap{\in \mathcal{C}^{\infty}}}*\underbrace{\psi}_{\mathclap{\in \mathcal{C}_{c}^{\infty}}},
		      \]
		      com a regularidade sendo distribuída entre os fatores.
	\end{itemize}
\end{prop*}
\begin{proof*}
	(0): Para este item, basta notar que
	\[
		\delta *\varphi (x) = \langle \delta , \tau_{x}\check{\varphi } \rangle = \tau_{x}\check{\varphi }(0) = \check{\varphi }(0-x) = \varphi (x).
	\]

	(1): este item segue da maneira como se vê uma função de classe \(u\in L_{\mathrm{loc}}^{1}\) no contexto de \(\mathcal{D}'\):
	\[
		u*\varphi (x) = \langle u, \tau_{x}(\varphi)^{\check} \rangle = \int_{\mathbb{R}^{n}}^{}u(y)\varphi (x-y) \mathrm{dy},\quad \forall x\in \mathbb{R}^{n}.
	\]
	Se quiser, no lugar de u acima, pode escrever \((Tu)*\varphi (x)\) para aumentar a consciência do que se está fazendo com o objeto, sendo Tu a distribuição provida de \(u\in L_{\mathrm{loc}}^{1}\).

	(2): antes de mais nada, devemos provar que \(u*\varphi \) é de contínua. Com efeito, se \(\{x_{\ell}\}_{\ell}\) converge para \(x_{0}\) em \(\mathbb{R}^{n}\), então
	\[
		\tau_{x_{\ell}}\check{\varphi }\substack{ \mathcal{C}_{c}^{\infty}(\mathbb{R}^{n})\\ \longrightarrow \\ \ell \to \infty}\tau_{x_{0}}\check{\varphi },
	\]
	pois, sendo \(K\coloneqq \{x_{\ell}:\; \ell =0,1,2,\dotsc \}.\)
\end{proof*}

\end{document}

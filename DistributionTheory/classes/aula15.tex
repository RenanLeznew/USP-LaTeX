\documentclass[../distribution_theory_notes.tex]{subfiles}
\begin{document}
\section{Aula 15 - 14 de Outubro, 2025}
\subsection{Motivações}
\begin{itemize}
	\item Continuando a Convergência em \(\mathcal{D}'\);
	\item Produto de Convoluções.
\end{itemize}
\subsection{Continuando a Convergência em \(\mathcal{D}'\)}
Na aula de sexta, estudamos a segunda parte do problema relativo a determinar a ``estrutura'' das distribuições com derivadas nulas.

Vimos que, quando \(n=1\) e \(u\in \mathcal{D}'(I)\) tem \(u'=0\), sendo I um intervalo; então, existe \(c\in \mathbb{C}\) tal que
\[
	\langle u, \varphi  \rangle = c \int_{I}^{}\varphi  \mathrm{d}, \quad \forall \varphi \in \mathcal{C}_{c}^{\infty}(I),
\]
isto é, u provém da função \(f=c\) constante \(L_{\mathrm{loc}}^{1}(I)\). No caso geral, considera-se \(U\subseteq \mathbb{R}^{n}\)
aberto com \(x\in U\) e \(I\subseteq \mathbb{R}\) um intervalo com t e, supondo u em \(\mathcal{D}'(U\times I)\) com \(\partial_t u = 0\),
concluímos que existe uma distribuição \(u_{0}\) em \(\mathcal{D}'(U)\) tal que
\[
	\langle u, \varphi  \rangle = \biggl\langle u_{0}, \int_{I}^{}\varphi (\cdot , t) \mathrm{d}t \biggr\rangle,\quad \varphi \in \mathcal{C}_{c}^{\infty}(U\times I).
\]
Nesse sentido, u depende apenas das variáveis x em U e é independente de t.

Em seguida, começaremos a tratar da convergência em \(\mathcal{D}'\), sendo ele o dual topológico do TVS \(\mathcal{C}_{c}^{\infty}(\Omega )\) com o limite indutivo; ele estaŕa sempre munido da topologia fraca-*, segundo a qual uma sequência \(\{u_{j}\}_{j}\) converge
para uma \(u\in \mathcal{D}'(\Omega )\) se, e somente se, para todo \(\varphi \in \mathcal{C}_{c}^{\infty}(\Omega )\),
\[
	\langle u_{j}, \varphi  \rangle \substack{ \\ \longrightarrow \\ j\to \infty}\langle u, \varphi  \rangle.
\]
Vimos que, por meio dessa topologia, dois resultados merecem destque: o primeiro é que podemos definir a transformação \(T:L_{\mathrm{loc}}^{1}(\Omega )\rightarrow \mathcal{D}'(\Omega )\) que identifica uma \(f\in L_{\mathrm{loc}}^{1}\) com a distribuição
\[
	\langle Tf, \varphi  \rangle = \int_{\Omega }^{}f\varphi,
\]
e é contínua, permitindo que praticamente todas as convergências usuaris no espaço de funções conhecidas ocorram também em \(\mathcal{D}'.\)

\begin{tcolorbox}[
		skin=enhanced,
		title=Observação,
		fonttitle=\bfseries,
		colframe=black,
		colbacktitle=cyan!75!white,
		colback=cyan!15,
		colbacklower=black,
		coltitle=black,
		drop fuzzy shadow,
		%drop large lifted shadow
	]
	Por outro lado, sempre que \(\mathcal{C}_{c}^{\infty}(E)\) for mergulhado em E, teremos também \(E'\hookrightarrow \mathcal{D}'(\Omega )\) com a topologia fraca-*, o que permite vermos em \(\mathcal{D}'\) as converências fracas-* de várias duais de TVS, por exemplo as \(L^{p'}(\Omega )\) como dual de \(L^{p}(\Omega )\), onde \(1\leq p <\infty\), ou, ainda,
	a de \(\mathcal{E}'(\Omega )\) como o dual de \(\mathcal{C}^{\infty}(\Omega )\hookleftarrow \mathcal{C}_{c}^{\infty}(\Omega )\).
\end{tcolorbox}
A segunda coisa é que a continuidade de \(\overline{T}:\mathcal{D}'(\Omega )\rightarrow \mathcal{D}'(\Omega ')\), a extensão de uma aplicação linear \(T:\mathcal{C}_{c}^{\infty}(\Omega )\rightarrow \mathcal{C}_{c}^{\infty}(\Omega ')\) que admite um transposto formal \(T'.\)
Um caso particular extremamente importante dessa situação é o caso das ODP, denotadas por \(P(x, D):\mathcal{D}'(\Omega )\rightarrow \mathcal{D}'(\Omega )\), que são automaticamente contínuas sempre que
\[
	u_{j}\substack{ \mathcal{D}'(\Omega )\\ \longrightarrow \\ j\to \infty}u \Rightarrow P(x, D)u_{j}\substack{\mathcal{D}'(\Omega ) \\ \longrightarrow \\ j\to \infty}P(x, D)u,
\]
por exemplo, sempre que \(u_{j}\) converge fraco-* para u,
\[
	\frac{\partial^{}u_{j}}{\partial x_{k}^{}}\substack{ \mathcal{D}'\\ \longrightarrow \\ j\to \infty}\frac{\partial^{}u}{\partial x_{k}^{}}.
\]

A aula terminou quando estávamos dando exemplos de sequências de funções que sabemos não convergir nos espaços usuais que habitam, mas que
trivialmente convergem em \(\mathcal{D}'(\Omega )\). Além do mais, convergem para limites que podem contradizer nossa intuição.

Hoje, completaremos a lista de exemplos sobre a convergência em \(\mathcal{D}'(\Omega )\), destacando:
\begin{itemize}
	\item[i)] A família regularizante do começo, \(\varphi_{\varepsilon }\substack{\varepsilon \to 0^{+} \\ \longrightarrow \\ }\delta,\; (\varphi_\varepsilon )_{\varepsilon > 0} \);
	\item[ii)] Condições mais gerais para que isso ocorra para uma sequência \(\{f_{j}\}_{j}\) de funções integráveis; e
	\item[iii)] Teorema: a série de Fourier de qualquer \(f\in L(-\pi , \pi )\) converge para f em \(\mathcal{D}'(-\pi , \pi )\).
\end{itemize}

Um dos resultados mais importantes sobre a convergênia em \(\mathcal{D}'\) é a densidade das funções teste em \(\mathcal{D}'\) e isso passa pelo caso especial da densidade de \(\mathcal{C}_{c}^{\infty}(\mathbb{R}^{n})\) em \(\mathcal{D}'(\mathbb{R}^{n})\),
assim como foi com \(\mathcal{C}(\Omega )\) e \(L^{1}(\Omega )\), primeiro prova-se a densidade em \(\mathcal{C}(\mathbb{R}^{n})\) e em \(L^{1}(\mathbb{R}^{n})\) das convoluções \(\varphi_{\varepsilon }*f\), que precisa de uma extensão do conceito de convolução
sendo um dos fatores uma distribuição e o outro uma função teste.

\subsection{Produto de Convolução - Segunda Parte}

Dados \(\varphi , \psi \in \mathcal{C}_{c}^{\infty}(\mathbb{R}^{n})\), partindo do modelo
\[
	\varphi *\psi (x) = \int_{\mathbb{R}^{n}}^{}\varphi(x-y)\psi (y) \mathrm{d}y = \int_{\mathbb{R}^{n}}^{}\varphi(y)\psi (x-y) \mathrm{dy}
\]
com x em \(\mathbb{R}^{n}\), e, adotando a notação \(\check{\varphi} \coloneqq \varphi \circ R\), onde \(R(y) = -y\), podemos escrever
\[
	\varphi * \psi (x) = \int_{\mathbb{R}^{n}}^{}\varphi (y)\tau_{x}{\psi} )(y) \mathrm{d}y,
\]
seguindo uma maneira de estender \(\varphi * \psi (x)\) com \(\varphi \in \mathcal{C}_{c}^{\infty}(\mathbb{R}^{n})\) para uma \(u\in \mathcal{D}'(\mathbb{R}^{n})\), pondo:
\[
	u*\psi (x)\coloneqq \langle u, \tau_{x}{\psi} ) \rangle,\quad x\in \mathbb{R}^{n}\;\&\; \psi \in \mathcal{C}_{c}^{\infty}(\mathbb{R}^{n}),
\]
onde assumimos que u atua na variável y, \textit{i.e.}, \(\tau_{X}\check{(\psi )}\) é função da variável y; quando for necessário deixar bem explícito e evitar confusões, escreveremos
\[
	\langle u(y), \tau_{x}{\psi} )(y) \rangle.
\]
\begin{tcolorbox}[
		skin=enhanced,
		title=Observação,
		fonttitle=\bfseries,
		colframe=black,
		colbacktitle=cyan!75!white,
		colback=cyan!15,
		colbacklower=black,
		coltitle=black,
		drop fuzzy shadow,
		%drop large lifted shadow
	]
	Em \(\tau_{x}\check{(\psi )},\) primeiro refletimos e depois transladamos:
	\[
		[\tau_{x}\check{(\psi )}](y) = \check{\psi}(y-x) = \psi(x-y) = (\tau_{-x}\psi )(-y) = \widecheck{(\tau_{-x}\psi )}(y).
	\]
\end{tcolorbox}
Ao definirmo \(u*\psi (x)\), poderíamos escrever também
\[
	u * \psi (x) = \langle u, \psi (x - \cdot ) \rangle,
\]
tal que, de maneira explícita,
\[
	u*\psi (x) = \langle u(y), \psi (x-y) \rangle.
\]
Esta definição associa, a cada distribuição \(u\in \mathcal{D}'(\mathbb{R}^{n})\) e função teste \(\varphi \in \mathcal{C}_{c}^{\infty}(\mathbb{R}^{n})\), uma função
\[
	u*\varphi :\mathbb{R}^{n}\rightarrow \mathbb{C}
\]
definida em todo vetor \(x\in \mathbb{R}^{n}\).

A seguir, resumimos as propriedades dessa operação:
\begin{prop*}
	A convolução, conforme definida acima, satisfaz:
	\begin{itemize}
		\item[0)] Para toda \(\varphi \in \mathcal{C}_{c}^{\infty}(\mathbb{R}^{n})\), \(\delta * \varphi = \varphi \);
		\item[1)] Quando \(u\in L_{\mathrm{loc}}^{1}(\mathbb{R}^{n})\), as definições atuais e a antiga coincidem;
		\item[2)] A convolução \(u*\varphi \) é de classe \(\mathcal{C}^{\infty}\) com
		      \[
			      \partial^{\alpha }(u*\varphi ) = u * (\partial^{\alpha }\varphi ) = (\partial^{\alpha }u)*\varphi ,\; \forall \alpha \in \mathbb{Z}_{+}^{n}; \text{ e}
		      \]
		\item[3)] Para todas as distribuições \(u\in \mathcal{D}'(\mathbb{R}^{n})\) e \(\varphi , \psi \in \mathcal{C}_{c}^{\infty}(\mathbb{R}^{n})\), tem-se
		      \[
			      \overbrace{u}^{\mathcal{\in \mathcal{D}'}}*(\overbrace{\varphi* \psi}^{\mathclap{\in \mathcal{C}_{c}^{\infty}}} )= \underbrace{(u*\varphi )}_{\mathclap{\in \mathcal{C}^{\infty}}}*\underbrace{\psi}_{\mathclap{\in \mathcal{C}_{c}^{\infty}}},
		      \]
		      com a regularidade sendo distribuída entre os fatores.
	\end{itemize}
\end{prop*}
\begin{proof*}
	(0): Para este item, basta notar que
	\[
		\delta *\varphi (x) = \langle \delta , \tau_{x}\check{\varphi } \rangle = \tau_{x}\check{\varphi }(0) = \check{\varphi }(0-x) = \varphi (x).
	\]

	(1): este item segue da maneira como se vê uma função de classe \(u\in L_{\mathrm{loc}}^{1}\) no contexto de \(\mathcal{D}'\):
	\[
		u*\varphi (x) = \langle u, \tau_{x}{\varphi}) \rangle = \int_{\mathbb{R}^{n}}^{}u(y)\varphi (x-y) \mathrm{dy},\quad \forall x\in \mathbb{R}^{n}.
	\]
	Se quiser, no lugar de u acima, pode escrever \((Tu)*\varphi (x)\) para aumentar a consciência do que se está fazendo com o objeto, sendo Tu a distribuição provida de \(u\in L_{\mathrm{loc}}^{1}\).

	(2): antes de mais nada, devemos provar que \(u*\varphi \) é de fato contínua. Com efeito, se \(\{x_{\ell}\}_{\ell}\) converge para \(x_{0}\) em \(\mathbb{R}^{n}\), então
	\[
		\tau_{x_{\ell}}\check{\varphi }\substack{ \mathcal{C}_{c}^{\infty}(\mathbb{R}^{n})\\ \longrightarrow \\ \ell \to \infty}\tau_{x_{0}}\check{\varphi },
	\]
	pois, sendo \(K\coloneqq \{x_{\ell}:\; \ell =0,1,2,\dotsc \} - \mathrm{supp}(\varphi )\) um compacto fixo,
	\[
		\mathrm{supp} \tau_{x_{\ell}}{\varphi }) = \mathrm{supp} \varphi(x_{\ell } - \cdot ) = x_{\ell}-\mathrm{supp}(\varphi )\subseteq K, \quad \forall \ell = 0, 1, 2, \dotsc
	\]
	Além disso, para \(x\in \mathbb{R}^{n}\) fixo,
	\[
		\partial_{y}^{\alpha }[\tau_{x}\check{\varphi }] = \partial_{y}^{\alpha }(\varphi \circ g),\; g(y) = x-y.
	\]
	Daí, pela regra da cadeia e pela indução, conclui-se que
	\[
		\partial_{y}^{\alpha }[\tau_{x}\check{\varphi }] = (-1)^{| \alpha  |}(\partial^{\alpha }\varphi )\circ g = (-1)^{|\alpha |}\tau_{x}(\partial^{\alpha }\varphi )^{\check{}},
	\]
	isto é, \(\partial_{y}^{\alpha }[\tau_{x}\check{\varphi }]\) é do tipo \(\tau_{X}{\psi} )\) para uma certa \(\psi\in \mathcal{C}_{c}^{\infty}(\mathbb{R}^{n}).\) Finalmente, se \(\psi \in \mathcal{C}_{c}^{\infty}(\mathbb{R}^{n})\), então
	\[
		| \tau_{x}\check{\psi }(y) - \tau_{x'} | = | \psi(x-y) - \psi(x'-y) | \leq \Vert \nabla \psi  \Vert_{\infty}| x-x' |,\quad \forall x, x', y\in \mathbb{R}^{n},
	\]
	pois \(\nabla \psi \) é limitado em \(\mathbb{R}^{n}\), contínuo e com suporte compacto, onde
	\[
		\biggl\Vert \nabla \psi  \biggr\Vert_{\infty}\coloneqq \sum\limits_{j=1}^{n}\sup_{y}| \partial_{j}\psi(y) |.
	\]

	Com essa estimativa, é suficiente para garantir que, quando x converge para \(x'\) em \(\mathbb{R}^{n}\),
	\[
		\sup_{y\in \mathbb{R}^{n}}| \tau_{x}\check{\psi }(u) - \tau_{x'}\check{\psi }(y) |\rightarrow 0
	\]
	para qualquer que seja \(\psi \in \mathcal{C}_{c}^{\infty}(\mathbb{R}^{n})\), resultando em
	\[
		\sup_{y\in \mathbb{R}^{n}}\biggl\vert \partial_{y}^{\alpha }[\tau_{x_{\ell}}\check{\varphi }](y) - \partial_{y}^{\alpha }[\tau_{x_{0}}\check{\varphi }](y) \biggr\vert\stackrel{\ell \to \infty}0
	\]
	e obtemos, em \(\mathcal{C}_{c}^{\infty}(\mathbb{R}^{n})\),
	\[
		\phi_{\ell} \coloneqq \tau_{x_{\ell}}\check{\varphi }\stackrel{\ell \to \infty}\tau_{x_{0}}\circ \check{\varphi }\eqqcolon \phi_{0}.
	\]

	Como \(u\in \mathcal{D}'(\mathbb{R}^{n})\), o resultado obtido é
	\[
		u*\varphi (x_{\ell}) = \langle u, \phi_{\ell} \rangle\stackrel{\ell \to \infty}\rightarrow \langle u, \phi_0 \rangle = u * \varphi(x_{0}),
	\]
	e da continuidade de \(u*\varphi \) em \(x_{0}\), segue que, por ser arbitrário, concluímos que \(u*\varphi \in \mathcal{C}(\mathbb{R}^{n})\).

	Com respeito às derivadas parciais e a igualdade em (ii), basta prová-la para uma derivada de primeira ordem, pois as outras resultarão, também, por indução. Calculemos, então, \(\partial_{j}(u*\varphi )(x_{0})\) num \(x_{0}\in \mathbb{R}^{n}\) e \(j= 1, 2, \dotsc , n\) arbitrários:

	\(1):\; u*\varphi (x_{0}+te_{j}) - u*\varphi (x_{0}) = \langle u, \tau_{(x_{0}+te_{j})}\check{\varphi } - \tau_{x_{0}}\check{\varphi }\rangle\).

	Em seguida, dado \(y\in \mathbb{R}^{n}\), escrevemos
	\[
		2):\; \tau_{(x_{0}+be_{j})}\check{\varphi }(y) - \tau_{x_{0}}\check{\varphi }(y) = \varphi(x_{0}+te_{j}-y) - \varphi(x_{0}-y) = \int_{0}^{t}\frac{\partial^{}\varphi }{\partial w_{j}^{}}(x_{0}+se_{j}-y) \mathrm{d}s,
	\]
	donde
	\[
		\frac{\tau_{(x_{0}+te_{j})}\check{\varphi}(y) - \tau_{x_{0}}\check{\varphi }(y)}{t} - \frac{\partial^{}\varphi }{\partial w_{j}^{}}(x_{0}-y) = \frac{1}{t}\int_{0}^{t}\biggl[\frac{\partial^{}\varphi }{\partial w_{j}^{}}(x_{0}+se_{j}-y) - \frac{\partial^{}\varphi }{\partial w_{j}^{}}(x_{0}-y)\biggr] \mathrm{d}s.
	\]

	O quociente de Newton acima é uma sequência \(q_{t}(y)\), de funções teste da variável y, e o que se deseja provar é que ela converge para \(\frac{\partial^{}\varphi }{\partial w_{j}^{}}(x_{0}-y)\) em \(\mathcal{C}_{c}^{\infty}(\mathbb{R}^{n})\) quando \(t\to 0\), concluindo a prova graças ao fato de \(u\in \mathcal{D}'(\mathbb{R}^{n})\) de funções teste, e o que se deseja provar é que ela converge para
	\(\frac{\partial^{}\varphi }{\partial w_{j}^{}}(x_{0}-y)\) em \(\mathcal{C}_{c}^{\infty}(\mathbb{R}^{n})\) quando \(t\to 0\), concluindo a prova, graças ao fato de que \(u\in \mathcal{D}'(\mathbb{R}^{n})\). Com efeito, derivando em y sob o sinal da integral e usando um argumento (que já se tornou bem comum),
	\[
		\partial_{y}^{\alpha }q_{t}(y) - \partial_{y}^{\alpha }\biggl[\frac{\partial^{}\varphi }{\partial w_{j}^{}}(x_{0}-y)\biggr] = \frac{1}{t}\int_{0}^{t}[(-1)^{| \alpha  |}\partial^{\alpha +e_{j}}\varphi(x_{0}+se_{j}-y)]-(-1)^{| \alpha  |}\partial^{\alpha +e_{j}}\varphi(x_{0}-y) \mathrm{d}s
	\]
	e, pela desigualdade do valor médio, o resultado é que, supondo \(t > 0\) (sem perda de generalidade),
	\[
		| \partial_{y}^{\alpha }q_{t}(y) - \partial_{y}^{\alpha }\biggl[\frac{\partial^{}\varphi }{\partial w_{j}^{}}(x_{0}-y)\biggr] | \leq \biggl\vert \frac{1}{t}\int_{0}^{t}\Vert \nabla (\partial^{\alpha +e_{j}}\varphi ) \Vert_{\infty}| s | \mathrm{d}s \biggr\vert = \frac{1}{t}\frac{t^{2}}{2} = \frac{t}{2}.
	\]
	Daí,
	\[
		\sup_{y\in \mathbb{R}^{n}}\biggl\vert \partial_{y}^{\alpha }q_{t}(y) - \partial_{y}^{\alpha }\biggl[\frac{\partial^{}\varphi }{\partial w_{j}^{}}(x_{0}-y)\biggr] \biggr\vert\leq \frac{| t |}{2}\stackrel{t\to 0}\rightarrow 0.
	\]

	Resta encontrar um compacto K fixo que contém \(\mathrm{supp}(q_{t})\) e \(\mathrm{supp}\biggl[\frac{\partial^{}\varphi }{\partial w_{j}^{}}(x_{0}-y)\biggr]\) para \(| t | > 0\) pequeno, mas isso é imediato quando \(| t |\leq 1\), porque o suporte em questão está contido em \(x_{0} - \mathrm{supp}(\varphi )\) e
	\begin{align*}
		\mathrm{supp}(q_{t}) & \subseteq \mathrm{supp}(\varphi (x_{0}+te_{j}-\cdot ) - \varphi(x_{0}-\cdot ))                               \\
		                     & \subseteq [x_{0}+te_{j}-\mathrm{supp}(\varphi )]\cup [x_{0} - \mathrm{supp}(\varphi )]                       \\
		                     & \subseteq [\overline{B}(x_{0}, 1) - \mathrm{supp}(\varphi )]\cup [x_{0}-\mathrm{supp}(\varphi )]\eqqcolon K.
	\end{align*}

	Por meio disso tudo,
	\[
		\frac{u*\varphi (x_{0}+te_{j}) - u*\varphi(x_{0})}{t} = \langle u, q_{t} \rangle\stackrel{t\to0}\rightarrow \biggl\langle u, \frac{\partial^{}\varphi }{\partial w_{j}^{}}(x_{0} - \cdot ) \biggr\rangle = \biggl\langle u, \tau_{x_{0}}\biggl(\frac{\partial^{}\varphi }{\partial w_{j}^{}}^{\check{}}\biggr) \biggr\rangle,
	\]
	donde \(u*\varphi \) é derivável em \(x_{0}\) com relação a \(x_{j}\), e vale:

	\[
		\frac{\partial^{}}{\partial x_{j}^{}}(u*\varphi )(x_{0}) = \biggl\langle u, \tau_{x_{0}}\biggl(\frac{\partial^{}\varphi }{\partial w_{j}^{}}^{\check{}}\biggr) \biggr\rangle = u * \biggl(\frac{\partial^{}\varphi }{\partial x_{j}^{}}\biggr)(x_{0}).
	\]

	A outra igualdade decorre desta, por meio da observação de que:
	\begin{align*}
		\phi: & \mathbb{R}^{n}\times \mathbb{R}^{n}\rightarrow \mathbb{C} \\
		      & (x, y)\longmapsto \phi (x, y) = \varphi(x-y),
	\end{align*}
	já vimos que
	\[
		\frac{\partial^{}\phi}{\partial y_{j}^{}}(x, y) = - \frac{\partial^{}\phi}{\partial x_{j}^{}}(x, y),
	\]
	pois
	\begin{align*}
		\frac{\partial^{}\phi}{\partial x_{j}^{}}(x, y) = \frac{\partial^{}}{\partial x_{j}^{}}(\varphi *D)(x, y) & = \partial_{j}\varphi(D(x, y)) \\
		                                                                                                          & = \partial_{j}\varphi (x-y),
	\end{align*}
	e
	\begin{align*}
		\frac{\partial^{}\phi}{\partial y_{j}^{}}(x, y) & = \partial_{j}\varphi(x-y)\frac{\partial^{}}{\partial y_{j}^{}}(x_{j}-y_{j}) \\
		                                                & = -\partial_{j}\varphi (x-y).
	\end{align*}
	Além disso,
	\begin{align*}
		u*\varphi (x)=\langle u(y), \phi(x, y) \rangle & \Rightarrow u*\biggl(\frac{\partial^{}\varphi }{\partial x_{j}^{}}\biggr)(x)=         \\
		                                               & = \biggl\langle u(y), \frac{\partial^{}\phi}{\partial x_{j}^{}}(x, y) \biggr\rangle   \\
		                                               & = - \biggl\langle u(y), \frac{\partial^{}\phi}{\partial y_{j}^{}}(x, y) \biggr\rangle \\
		                                               & = \biggl\langle \frac{\partial^{}u}{\partial y_{j}^{}}(y), \phi(x, y) \biggr\rangle   \\
		                                               & = \biggl\langle \frac{\partial^{}u}{\partial y_{j}^{}}, \phi(x-\cdot ) \biggr\rangle  \\
		                                               & = \biggl(\frac{\partial^{}u}{\partial y_{j}^{}}\biggr)*\varphi (x).
	\end{align*}
	A parte (iii), porém, exige mais cuidado, e por isso precisaremos fazer com calma; a prova se assemelha à de (ii), substituindo o quociente de newton por somas de Riemann, a partir das seguintes considerações: fixado \(x\in \mathbb{R}^{n}\), escrevamos
	\[
		[u*(\varphi *\psi )](x) = \langle u, \tau_{x}(\varphi*\psi \check{)} \rangle.
	\]
	A função \(\tau_{x}(\varphi *\psi \check{)}:\mathbb{R}^{n}\rightarrow \mathbb{C}\) é dada por
	\[
		\tau_{x}(\varphi*\psi \check{)}(y) = \varphi*\psi (x-y) = \int_{\mathbb{R}^{n}}^{}\varphi ((x-y)-z)\psi (z) \mathrm{d}z,\; y\in \mathbb{R}^{n}.
	\]

	A integral acima ocorre em \(\mathrm{supp}(\psi )\subseteq R = \prod\limits_{i=1}^{n}[a_{i}, b_{i}]\), e é uma integral de Riemann! Nessa situação, a integral sobre \(\mathbb{R}^{n}\) se reduz à integral num retângulo qualquer que contenha o suporte do integrando, conforme aprendemos em análise.
	Como, em 2, x está fixo e só y varia, para simplificar as notações, indiquemos o integrando por
	\[
		F(y, z) \coloneqq \varphi ((x-y)-z)\psi (z),
	\]
	de modo que
	\[
		\varphi *\psi (x-y) = \int_{\mathbb{R}^{n}}^{}F(y ,z) \mathrm{d}z = \int_{R}^{}F(y, z) \mathrm{d}z.
	\]

	Assim, se \(p_{\zeta } = (p, \zeta )\) indica uma partição partilhada de R, que podemos supor ser um intervalo só para fixar ideias: neste caso, \(R = [a, b]\), a partição é \(P\coloneqq a = z_{0} < z_1 < \dotsc < z_{n} = b\) com \(\zeta_{j}\in [z_{j-1}, z_{j}], j=1, \dotsc , m\), então
	\[
		\int_{R}^{}F(y, z) \mathrm{d}z = \lim_{| P |\to 0}S(F(y, \cdot ); P_{\zeta }),
	\]
	onde
	\[
		S(F(y, \cdot ); P_{\zeta })\coloneqq \sum\limits_{j=1}^{m}F(y, \zeta_{j})(z_{j}-z_{j-1})
	\]
	para qualquer que seja \(y\in \mathbb{R}^{n}\). As somas de Riemann acima definem, desta forma, famílias de funções teste da variável y:
	\[
		S(F(y, \cdot ), P_{\zeta }) = \sum\limits_{j=1}^{m}\underbrace{\varphi ((x-y)\cdot \zeta_{j})}_{\mathclap{\substack{\text{funções teste da variável y,} \\ \text{pois x e }\zeta_{j}\text{ estão fixados.}}}}\psi (\zeta_{j})(z_{j}-z_{j-1})
	\]

	Observe que
	\begin{align*}
		\langle u(y), S(F(y, \cdot ), P_{\zeta }) \rangle & = \sum\limits_{j=1}^{m} \langle u(y), F(y, \zeta_{j}) \rangle(z_{j}-z_{j-1})                        \\
		                                                  & = \sum\limits_{j=1}^{m}\langle u(y), \varphi(x-y-\zeta_{j})\psi(\zeta_{j}) \rangle(z_{j}-z_{j-1})   \\
		                                                  & = \sum\limits_{j=1}^{m}\langle u(y), \varphi(x-y-\zeta_{j}) \rangle \psi(\zeta_{j})(z_{j}-z_{j-1}),
	\end{align*}
	que é a soma de Riemann de uma função de z,
	\[
		f(z)\coloneqq \langle u(y), \varphi(x-y-z) \rangle \cdot \psi (z) = \langle u(y), \tau_{(x-z)}\check{\varphi }(y) \rangle\psi (z) = (u*\varphi )(x-z)\psi(z),
	\]
	onde \(\varphi(x-y-z) = \tau_{(x-z)}\check{\varphi }(y)\), e que \(\mathcal{C}^{\infty}\) é contínua em R, donde
	\[
		(u*\varphi )*\psi (x) = \int_{R}^{}(u*\varphi )(x-z)\psi (z) \mathrm{d}z = \lim_{| P |\to 0}s(f; P_{\zeta }) = \lim_{| P |\to 0}\langle u(y), S(F(y, \cdot ), P_{\zeta }) \rangle.
	\]
	A demonstração é concluída após mostrarmos que
	\[
		S(F(y, \cdot ), P_{\zeta })\stackrel{| P |\to 0}\longrightarrow \varphi *\psi (x-y)
	\]
	em \(\mathcal{C}_{c}^{\infty}(\mathbb{R}^{n})\); de fato,
	\begin{align*}
		\varphi *\psi (x-y) = \tau_{x}(\varphi *\psi \check{)}(y) & \Rightarrow \langle u, S(F(y, \cdot ), P_{\zeta }) \rangle\rightarrow \langle u, \tau_{x}(\varphi*\psi \check{)} \rangle= u * (\varphi *\psi )(x).
	\end{align*}

\end{proof*}

\end{document}

\documentclass[../distribution_theory_notes.tex]{subfiles}
\begin{document}
\section{Aula 11 - 27 de Setembro, 2025}
\subsection{Motivações}
\begin{itemize}
	\item Operações com Distribuições;
	\item Cálculo com Distribuições.
\end{itemize}
\subsection{Mexendo nas Distribuições}

Agora que já temos uma noção melhor tanto da definição de distribuições quanto do fato de que, essencialmente, qualquer função em um aberto \(\Omega \) do espaço euclidiano possui um ``representante''/``cópia'' dentro de \(\mathcal{D}'(\Omega )\), vamos dar mais uma amostra do quão amplo esta classe de funções é, e faremos isto mostrando que também podemos incluir medidas borelianas localmente finitas nele.

Além disso, mostraremos como operacionalizar os objetos deste espaço com o objetivo de transportar o cálculo das funções para as distribuições. Em particular, as operações que previamente definimos em \(E'\) de adição e produto escalar podem ser aplicadas para os elementos de \(\mathcal{D}'(\Omega )\). Sem mais delongas,
\begin{example}
	Seja \(\mu \) uma medida boreliana localmente finita num aberto \(\Omega \subseteq \mathbb{R}^{n}.\) Definimos
	\begin{align*}
		T_{\mu }: & \mathcal{C}_{c}^{\infty}(\Omega u)\rightarrow \mathbb{C}                                        \\
		          & \varphi \longmapsto \left< T_{\mu }, \varphi  \right> \coloneqq \int_{\Omega }\varphi  d\mu_{},
	\end{align*}
	de tal forma a termos
	\[
		\int_{\Omega }^{}x \varphi  \mathrm{d}| \mu  | = \int_{\mathrm{supp}(\varphi )}| \varphi  | d| \mu_{} | \leq | \mu   |(\mathrm{supp}(\varphi ))\sup_{x\in \Omega }| \varphi (x) |,
	\]
	onde \(| \mu  |\) representa a variação total de \(\mu .\) Assim, dado um compacto K de \(\Omega \), tomando \(C\coloneqq | \mu  |(K)\) e \(m=0\), vemos que
	\[
		|\left< T_{\mu }, \varphi  \right>  | \leq c\sup_{x\in K}| \varphi (x) |,\quad \varphi \in \mathcal{C}_{c}^{\infty}(K),
	\]
	mostrando que \(T_{\mu }\) define de fato uma distribuição em \(\Omega.\)
\end{example}
\begin{tcolorbox}[
		skin=enhanced,
		title=Observação,
		fonttitle=\bfseries,
		colframe=black,
		colbacktitle=cyan!75!white,
		colback=cyan!15,
		colbacklower=black,
		coltitle=black,
		drop fuzzy shadow,
		%drop large lifted shadow
	]
	Diferentemente do que ocorre com o exemplo de \(L_{\mathrm{loc}}^{1}\), a prova de T definir um injeção contínua de algum espaço de medidas borelianas em \(\mathcal{D}'(\Omega )\) decorre de fatos mais profundos de Teoria da Medida -- os teoremas de Radon em espaços HLC e os teoremas de Representação de Riesz em \(\mathcal{C}_{c}^{}(\Omega )\) e em \(\mathcal{C}_{0}(\Omega )\) -- e identificamos, através deles,
	\[
		\left< T_{\mu }, \varphi  \right> = \left< \mu , \varphi  \right> = \int_{}\varphi  d\mu_{}
	\]
\end{tcolorbox}
\begin{example}
	Quando \(M\subseteq \Omega \) é uma superfície orientável e \(\mu =dM\) é a medida de superfície (construída a partir do elemento de volume de M, teoremas de extensão de medidas de álgebras para \(\sigma-\)álgebras geradas, etc), então a integral de uma \(\varphi \in \mathcal{C}_{c}^{\infty}(\Omega )\) ao longo de uma superfície M define uma distribuição:
	\[
		\left< \mu , \varphi  \right>=\int_{M}\varphi  dM.
	\]
	Particularmente, se \(\Omega \subseteq \mathbb{R}^{2}\) é um aberto e M é a diagonal do plano, então
	\[
		\left< u, \varphi  \right>= \int_{}^{}\varphi (x, x) \mathrm{dx}
	\]
	é uma distribuição em \(\Omega \).
	\begin{figure}[H]
		\begin{center}
			\includegraphics[height=0.5\textheight, width=0.5\textwidth, keepaspectratio]{./Images/diag_distri_11.png}
		\end{center}
	\end{figure}

	Ademais, a integral de linha ao longo de um caminho, digamos \(\mathcal{C}^{1}\) por partes e descrita arbitrariamente como
	\begin{align*}
		\gamma: & [a, b]\rightarrow \Omega \subseteq \mathbb{R}^{2}  \\
		        & t\longmapsto \gamma (t)=(x(t), y(t)) = x(t)+iy(t),
	\end{align*}
	com as devidas adaptações pode também ser vista como uma distribuição \(u\in \mathcal{D}'(\Omega )\) pondo
	\[
		\left< u, \varphi  \right> = \int_{a}^{b}\varphi (\gamma (t)) \mathrm{d}\gamma (t) = \int_{a}^{b}\varphi (\gamma (t))\cdot \varphi'(t)\mathrm{d}t
	\]
	para \(\varphi :\Omega  \subseteq \mathbb{R}^{2}\rightarrow \mathbb{C}\) sendo a função teste.
\end{example}

\subsection{Operações com Distribuições}
A maneira mais comum de se definir operações com distribuições é procurar estender operações que podem ser efetuadas com funções para as distribuições, que será feito por meio de aplicações lineares de \(\mathcal{C}_{c}^{\infty}(\Omega )\) em \(\mathcal{C}_{c}^{\infty}(\Omega ')\), que sejam também contínuas com relação ao limite indutivo, com \(\Omega \) e \(\Omega '\) abertos e que possuam o chamado ``transposto formal''.
\begin{def*}
	Sejam \(\Omega \) e \(\Omega '\) abertos de espaços euclidianos e \(T:\mathcal{C}_{c}^{\infty}(\Omega )\rightarrow \mathcal{C}_{c}^{\infty}(\Omega )'\) e \(T':\mathcal{C}_{c}^{\infty}(\Omega') \rightarrow \mathcal{C}_{c}^{\infty}(\Omega )\) aplicações lineares contínuas tais que:
	\[
		\int_{\Omega '}^{}(T\varphi )(y)\psi (y) \mathrm{dy} = \int_{\Omega }^{}\varphi (x)(T'\psi )(x) \mathrm{dx},\quad \varphi \in \mathcal{C}_{c}^{\infty}(\Omega )\;\&\; \psi\in \mathcal{C}_{c}^{\infty}(\Omega ').
	\]
	Quando isto ocorre, dizemos que \(T'\) é o \textbf{transposto formal de T}. \(\square\)
\end{def*}
Esta nomenclatura vem do problema de Stern-Liouville, onde, na definição do adjunto, ignora-se as condições de fronteira.
\begin{lemma*}
	Se T e T' são transpostos um do outro, então T possui uma extensão contínua \(\tilde{T}:\mathcal{C}_{c}^{\infty}(\Omega )\rightarrow \mathcal{D}'(\Omega )\) tal que
	\[
		\tilde{T}\varphi = T\varphi
	\]
	quando \(\varphi \in \mathcal{C}_{c}^{\infty}(\Omega )\).
\end{lemma*}
\begin{proof*}
	Com efeito, definamos \(T:\mathcal{D}'(\Omega )\rightarrow \mathcal{D}'(\Omega ')\) pondo
	\[
		\left< \tilde{T}u, \psi  \right> = \left< u, T'\psi  \right>,\quad u\in \mathcal{D}'(\Omega ),\;\&\;\psi \in \mathcal{C}_{c}^{\infty}(\Omega ')
	\]
	e observamos que, se \(\varphi \in \mathcal{C}_{c}^{\infty}(\Omega )\), então
	\begin{align*}
		\left< \tilde{T}\varphi , \psi  \right> = \left< \varphi , T'\psi  \right> & = \int_{\Omega }^{}\varphi (x)(T'\psi )(x) \mathrm{dx}                                       \\
		                                                                           & = \int_{\Omega '}^{}(T\varphi )(y)\psi (y) \mathrm{dy}                                       \\
		                                                                           & = \left< T\varphi , \psi  \right>,\quad \forall \psi \in \mathcal{C}_{c}^{\infty}(\Omega '),
	\end{align*}
	que basta para provar o fato de \(\tilde{T}\) estender T; além disto, \(\tilde{T}\) é linear (verifique, mas segue da linearidade de T e de u). Consequentemente, \(\tilde{T}u\in \mathcal{D}'(\Omega ')\) quando u é um elemento de \(\mathcal{D}'(\Omega )\). Com efeito, basta provar a continuidade sequencial: suponha que \(\psi_{j}\rightarrow 0\) em \(\mathcal{C}_{c}^{\infty}(\Omega ')\); então,
	\[
		T'\psi_{j}\substack{\mathcal{C}_{c}^{\infty}(\Omega ) \\ \longrightarrow \\ }0 \Rightarrow \left< u, T'\psi_{j} \right>\substack{ \\ \longrightarrow \\ j\to \infty}0,
	\]
	ou seja,
	\[
		\left< \tilde{T}u, \psi_{j} \right>\substack{ \\ \longrightarrow \\ j\to \infty}0.
	\]
	Portanto, \(\tilde{T}u\in \mathcal{D}'(\Omega ')\), provando o lema. \qedsymbol
\end{proof*}
\begin{tcolorbox}[
		skin=enhanced,
		title=Observação,
		fonttitle=\bfseries,
		colframe=black,
		colbacktitle=cyan!75!white,
		colback=cyan!15,
		colbacklower=black,
		coltitle=black,
		drop fuzzy shadow,
		%drop large lifted shadow
	]
	Na prática, para designar a extensão de T construída acima, costumamos usar apenas a letra T.
\end{tcolorbox}
\begin{lemma*}
	Um funcional linear \(u:\mathcal{C}_{c}^{\infty}(\Omega )\rightarrow \mathbb{C}\) é uma distribuição se, e somente se,  convergência de \(\varphi_{j}\) ao 0 em \(\mathcal{C}_{c}^{\infty}(\Omega )\) resultar na convergência do produto interno \(\left< u, \varphi_{j} \right>\) ao 0 também, isto é,
	\[
		\varphi_{j}\substack{\mathcal{C}_{c}^{\infty}(\Omega ) \\ \longrightarrow \\ }0 \Rightarrow \left< u, \varphi_{j} \right>\substack{ \\ \longrightarrow \\ j\to\infty}0.
	\]
\end{lemma*}
Noutras palavras, para provar a continuidade de um funcional linear em \(\mathcal{C}_{c}^{\infty}(\Omega )\), basta provar a continuidade sequencial, e sabemos que isso não precisa ser verdade num espaço topológico qualquer, evidenciando que a topologia de \(\mathcal{C}_{c}^{\infty}(\Omega )\) é especial, já que ele requer \(E_1\) com sua topologia que provém da métrica.
\begin{proof*}
	O fato de \(u\in \mathcal{D}'(\Omega )\) implicar na condição da sequência segue do fato de que, se \(\varphi_{j}\) converge para 0, existe um compacto \(K\subseteq \Omega \) tal que
	\[
		\mathrm{supp}(\varphi_{j})\subseteq K,\quad \forall j,
	\]
	isto é,
	\[\varphi_{j}\in \mathcal{C}_{c}^{\infty}(K),\; \forall j\;\&\; \sup_{K}| \partial^{\alpha }\varphi_{j} |\substack{ \\ \longrightarrow \\ j\to \infty}0,\; \forall \alpha .\]
	Como \(u\in \mathcal{D}'(\Omega )\), correspondem a este compacto K os números \(c>0\) e m em \(\mathbb{Z}_{+}\) tais que
	\[
		| \left< u, \varphi  \right> | \leq c \sum\limits_{| \alpha  |\leq m}^{}\sup_{K}| \partial^{\alpha }\varphi (x) |,\quad \varphi \in \mathcal{C}_{c}^{\infty}(K).
	\]
	Logo, sendo
	\[
		\sum\limits_{| \alpha  |\leq m}^{} \sup_{}| \partial^{\alpha }\varphi_{j}(x) |\to 0,
	\]
	o resultado que obtemos é que
	\[
		\left< u, \varphi_{j} \right>\to 0.
	\]

	A recíproca, por outro lado, é muito mais interessante, e provaremos ela por contradição. Com efeito, supondo a condição das sequências e que \(u\not\in \mathcal{D}'(\Omega )\), por definição deve existir um compacto \(K\) de \(\Omega \) associado ao qual não existem \(c > 0\) e \(m\in \mathbb{Z}_{+}\) tais que, para todo \(\varphi \in \mathcal{C}_{c}^{\infty}(K)\),
	\[
		| \left< u, \varphi  \right> | \leq cp_{m}(\varphi );
	\]
	com isto, tomando sucessivamente \(c=m=j\in \mathbb{N},\) podemos encontrar uma \(\varphi_{j}\in \mathcal{C}_{c}^{\infty}(K)\) tal que
	\[
		| \left< u, \varphi_{j} \right> | > j \sum\limits_{| \alpha  |\leq j}^{}\sup_{x\in K}| \partial^{\alpha }\varphi_{j}(x) |,\quad j\in \mathbb{N},
	\]
	mas isso não pode ser verdade, pois se pusermos
	\[
		\psi_{j}\coloneqq \frac{\varphi_{j}}{| \left< u, \varphi_{j} \right> |},
	\]
	por um lado a desigualdade que encontramos nos daria
	\[
		\sup_{x\in K} | \partial^{\alpha }\psi_{j}(x) | < \frac{1}{j},\quad \forall j\;\&\; | \alpha  |\leq j,
	\]
	donde teríamos \(\psi_{j}\to 0\) em \(\mathcal{C}_{c}^{\infty}(K)\), mas \(| \left< u, \psi_{j} \right> | = 1\) para todo j, e não é verdade que
	\[
		\left< u, \psi_{j} \right>\substack{ \\ \longrightarrow \\ j\to \infty}0,
	\]
	provando o lema. \qedsymbol
\end{proof*}
\begin{example}[Multiplicação por Funções f de classe \(\mathcal{C}^{\infty}(\Omega )\)]
	Dada f de classe \(\mathcal{C}^{\infty}(\Omega )\), vimos que a transformação
	\begin{align*}
		T: & \mathcal{C}^{\infty}(\Omega )\rightarrow \mathcal{C}^{\infty}(\Omega ) \\
		   & \varphi \longmapsto T\varphi \coloneqq f\varphi
	\end{align*}
	é linear, contínua e, em particular, para cada compacto K de \(\Omega \),
	\[
		T(\mathcal{C}_{c}^{\infty}(K))\subseteq \mathcal{C}_{c}^{\infty}(K)
	\]
	e \(T|_{\mathcal{C}_{c}^{\infty}(K)}\) é contínua. Logo, como sabemos, \(T:\mathcal{C}_{c}^{\infty}(K)\rightarrow \mathcal{C}_{c}^{\infty}(\Omega )\) se torna contínua e, além disso, dados \(\varphi \) e \(\psi \) de classe \(\mathcal{C}_{c}^{\infty}(\Omega )\), teremos
	\begin{align*}
		\int_{}^{}(T\varphi )\psi (y) \mathrm{dy} = \int_{}^{}(f(y)\varphi (y))\psi (y) \mathrm{dy} & = \int_{}^{}\varphi (y)(f(y)\psi (y)) \mathrm{dy} \\
		                                                                                            & = \int_{}^{}\varphi (x)(T\psi )(x) \mathrm{dx},
	\end{align*}
	donde T é seu próprio transposto formal, o que nos permite definir Tu quando \(u\in \mathcal{D}'(\Omega )\), ou seja, \(fu\) por
	\[
		\left< f \cdot u, \varphi  \right> = \left< Tu, \varphi  \right> = \left< u, T'\varphi  \right> = \left< u, f \cdot \varphi  \right>,\quad \varphi \in \mathcal{C}_{c}^{\infty}(\Omega ).
	\]
\end{example}
\begin{example}[Derivação de Distribuições]
	Como estudamos anteriormente, se \(\Omega \subseteq \mathbb{R}^{n}\) é um aberto, a aplicação
	\begin{align*}
		S: & \mathcal{C}^{\infty}(\Omega )\rightarrow \mathcal{C}^{\infty}(\Omega )               \\
		   & \varphi \longmapsto S\varphi \coloneqq \frac{\partial^{}\varphi }{\partial ^{}x_{j}}
	\end{align*}
	é linear e contínua com
	\[
		\mathrm{supp} \biggl(\frac{\partial^{}\varphi }{\partial x_{j}^{}}\biggr) \subseteq \mathrm{supp}(\varphi ),
	\]
	logo, para todo K contido em \(\Omega \), a transformação \(S:\mathcal{C}_{c}^{\infty}(K)\rightarrow \mathcal{C}_{c}^{\infty}(K)\) está bem definida e é contínua, garantindo a continuidade de \(S:\mathcal{C}_{c}^{\infty}(\Omega )\rightarrow \mathcal{C}_{c}^{\infty}(\Omega )\).

	Agora, dados \(\varphi , \psi \in \mathcal{C}_{c}^{\infty}(\Omega )\), integrando por partes, podemos escrever (verifique)
	\[
		\int_{\Omega }^{}(S\varphi )(x)\psi (x) \mathrm{dx} = \int_{\Omega }^{}\biggr(\frac{\partial^{}\varphi }{\partial x_{j}^{}}(x)\biggl)\psi (x) \mathrm{dx} = - \int_{\Omega }^{}\varphi (x)\frac{\partial^{}\psi }{\partial x_{j}^{}} \mathrm{dx},
	\]
	mostrando que \(S'=-\frac{\partial^{}}{\partial x_{j}^{}}\) é o transposto formal de S, permitindo definir as \textit{derivadas parciais de distribuições} pondo, para todo \(u\in \mathcal{D}'(\Omega )\) e \(\varphi \in \mathcal{C}_{c}^{\infty}(\Omega )\),
	\[
		\biggr\langle \frac{\partial^{}u}{\partial x_{j}^{}}, \varphi  \biggl\rangle = \left< Su, \varphi  \right> = \left< u, S'\varphi  \right> = \biggr\langle u, - \frac{\partial^{}\varphi }{\partial x_{j}^{}}\biggl\rangle = -\biggr\langle u, \frac{\partial^{}\varphi }{\partial x_{j}^{}} \biggl\rangle
	\]
	\begin{figure}[H]
		\begin{center}
			\includegraphics[height=0.5\textheight, width=0.5\textwidth, keepaspectratio]{./Images/support_k_11.png}
		\end{center}
		\caption{K está contido num retângulo, donde é só aplicar Fubini.}
	\end{figure}

	Naturalmente, pode-se aplicar sucessivamente esta definição acima para definir derivadas parciais qualquer de \(u\in \mathcal{D}'(\Omega )\):
	\[
		\biggr\langle \frac{\partial^{2}u}{\partial x_{j}\partial x_{k}}, \varphi  \biggl\rangle = \biggl\langle \frac{\partial^{}u}{\partial x_{j}^{}}, \frac{\partial^{}\varphi }{\partial x_{k}^{}} \biggr\rangle = -\biggr(-\biggr\langle u, \frac{\partial^{}\varphi }{\partial x_{k} x_{j}^{}} \biggl\rangle \biggl) = \biggl\langle u, \frac{\partial^{2}\varphi }{\partial x_{j}\partial x_{k}} \biggr\rangle
	\]
	Em particular, vale um teorema de Schwarz: a ordem na qual se toma as derivadas sucessivas é indiferente, mesmo o objeto não possuindo o número de derivadas exigidas no cálculo. Assim, sucessivamente para todo \(\alpha \in \mathbb{Z}_{+}^{n},\) tem-se
	\[
		\langle \partial^{\alpha }u, \varphi  \rangle = (-1)^{| \alpha  |}\langle u, \partial^{\alpha}\varphi  \rangle, \quad u\in \mathcal{D}'(\Omega ),\; \varphi \in \mathcal{C}_{c}^{\infty}(\Omega ).
	\]

\end{example}

\end{document}

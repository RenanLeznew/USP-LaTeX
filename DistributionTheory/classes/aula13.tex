\documentclass[../distribution_theory_notes.tex]{subfiles}
\begin{document}
\section{Aula 13 - 07 de Outubro, 2025}
\subsection{Motivações}
\begin{itemize}
	\item Mudança de Variáveis para Distribuições;
	\item Aprofundamento na Convergência Fraca-*.
\end{itemize}
\subsection{Operadores Diferenciais e Mudança de Variáveis para Distribuições}
Tendo analisado operações com distribuições e destacado alguns exemplos muito interessantes, daremos continuidade aos estudos desta parte introduzindo os chamados \textit{operadores diferenciais lineares} em \(\mathcal{D}'(\Omega ),\) junto à versão neste espaço da mudança de variáveis. Além disso, veremos distribuições com derivadas
parciais nulas e examinaremos a convergência em \(\mathcal{D}'\) mais a fundo, apresentando funções que não convergiriam de modo natural em seus espaços originais, mas convergem em \(\mathcal{D}'.\)

\begin{example}[Operadores Diferenciais]
	Sejam \(a_{\alpha }\in \mathcal{C}^{\infty}(\Omega ),\; | \alpha  |\leq m,\; \alpha \in \mathbb{Z}_{+}^{n}\), onde \(\Omega \) é um aberto de \(\mathbb{R}^{n}\). Considerando
	\[
		P(x, D)=\sum\limits_{| \alpha  |\leq m}^{}a_{\alpha }(x)\partial^{\alpha },
	\]
	resulta, do que já estudamos, que \(P(x, D):\mathcal{C}_{c}^{\infty}(\Omega )\rightarrow \mathcal{C}_{c}^{\infty}(\Omega )\) é um operador linear contínuo, chamado \textbf{operador diferencial parcial linear com coeficientes} \(\mathcal{C}^{\infty},\) que chamaremos apenas de \textbf{operador diferencial parcial} ou \textbf{operador diferencial ordinário}, caso n seja 1. Cada uma das parcelas da somatória que o define, ou seja, cada
	\[
		a_{\alpha }(x)\partial^{\alpha }=a_{\alpha }(x)\partial_{1}^{\alpha_{1}}\circ \partial_{2}^{\alpha_{2}}\circ\cdots\circ \partial_{n}^{\alpha_{n}},
	\]
	é uma composição de m derivadas parciais sucessivas seguidas de uma multiplicação por função \(\mathcal{C}^{\infty}(\Omega )\), justificando a classe do operador. Caso ocorra de cada \(a_{\alpha }\) ser uma função constante, enfatizaremos este fato dizendo que se trata de um \textbf{operador diferencial parcial com coeficientes constantes}, indicando-o por apenas
	\[
		P(D), = \sum\limits_{| \alpha  |\leq m}^{}a_{\alpha }\partial^{\alpha };
	\]
	assim, \(P(x, D)\) possui uma extensão a \(\mathcal{D}'(\Omega )\), \(P(x, D):\mathcal{D}'(\Omega )\rightarrow \mathcal{D}'(\Omega )\), daad por
	\begin{align*}
		\langle P(x, D)u, \varphi  \rangle = \sum\limits_{| \alpha  |\leq m}^{}\langle a_{\alpha }D^{\alpha }u, \varphi  \rangle & = \sum\limits_{| \alpha  |\leq m}^{}\langle \partial^{\alpha }u, a_{\alpha }\varphi  \rangle                                  \\
		                                                                                                                         & = \sum\limits_{| \alpha  |\leq m}^{}(-1)^{| \alpha  |}\langle u, \partial^{\alpha }(a_{\alpha }\varphi ) \rangle              \\
		                                                                                                                         & = \biggl\langle u, \sum\limits_{| \alpha  |\leq m}^{}(-1)^{| \alpha  |}\partial^{\alpha }(a_{\alpha })\varphi  \biggr\rangle,
	\end{align*}
	onde \(\varphi \) é de classe \(\mathcal{C}_{c}^{\infty}(\Omega )\) e, como visto na segunda linha da equação, faz-se necessário derivar os coeficientes.

	Desta forma,
	\[
		P'(x, D)\psi = \sum\limits_{| \alpha  |\leq m}^{}(-1)^{| \alpha  |}\partial^{\alpha }(a_{\alpha }\psi )
	\]
	é o transposto formal de \(P(x, D)\); em particular, se os coeficientes forem constantes, teremos
	\[
		\langle P(D)u, \varphi  \rangle = \biggl\langle u, \sum\limits_{| \alpha  |\leq m}^{}(-1)^{| \alpha  |}a_{\alpha }\partial^{\alpha }\varphi  \biggr\rangle.
	\]

\end{example}
Vamos considerar alguns subexemplos do operador derivada parcial.

\begin{example}[Operadores Comuns de EDP]
	Temos uma versão do \textbf{operador de Laplace} para as distribuições ao colocarmos
	\[
		P(D)=\Delta = \sum\limits_{j=1}^{m}\frac{\partial^{2}}{\partial x_{j}^{2}},
	\]
	que se exprime como
	\[
		\frac{\partial^{2}}{\partial x_{j}^{2}} = \partial_{j}^{2e_{j}},\quad j=1,2,\dotsc ,n.
	\]
	A forma que ele age em uma distribuição \(u\in \mathcal{D}'(\Omega )\) é
	\[
		\langle \Delta u, \varphi  \rangle=\biggl\langle u, \sum\limits_{}^{}(-1)^{2}\frac{\partial^{2}}{\partial x_{j}^{2}}\varphi  \biggr\rangle = \langle u, \Delta \varphi  \rangle,
	\]
	isto é,
	\[
		\langle \Delta u, \varphi  \rangle = \langle u, \Delta \varphi  \rangle,\quad u\in \mathcal{D}'(\Omega ),\; \varphi \in \mathcal{C}_{c}^{\infty}(\Omega ).
	\]

	Seguindo neste caminho de EDPs, podemos considerar um \textbf{operador do calor} dado por
	\[
		P(D)=\frac{\partial^{}}{\partial t^{}} - \Delta x
	\]
	com \((t, x)\in (0, \infty)\times \mathbb{R}^{n}\), um \textbf{operador das ondas}
	\[
		P(D)= \frac{\partial^{2}}{\partial t^{2}} - \Delta x
	\]
	com \((t,x)\in (0, \infty)\times \mathbb{R}^{n}\), e até mesmo um \textbf{operador de Cauchy-Riemann}
	\[
		P(D)=\frac{1}{2}\biggl(\frac{\partial^{}}{\partial x^{}}+i \frac{\partial^{}}{\partial y^{}}\biggr),
	\]
	com \((x, y)\in \mathbb{R}^{2}.\)
\end{example}
Finalmente, vejamos a versão da mudança de variáveis para distribuições:
\begin{example}[Mudança de Variáveis]
	Sejam U e V abertos de \(\mathbb{R}^{2}\), \(h:U\rightarrow V\) um difeomorfismo e definamos
	\begin{align*}
		H: & \mathcal{C}_{c}^{\infty}(V)\rightarrow \mathcal{C}_{c}^{\infty}(U) \\
		   & \varphi \longmapsto H\varphi \coloneqq \varphi \circ h
	\end{align*}
	Como exercício, mostre que H está bem definido, é linear e contínuo. Além disso, se \(\varphi \in \mathcal{C}_{c}^{\infty}(V)\) e \(\psi \in \mathcal{C}_{c}^{\infty}(U)\), pela mudança de variáveis na integral de Riemann, podemos escrever:
	\[
		\int_{U}^{}(H\varphi )(x) \mathrm{dx} = \int_{U}^{}(\varphi \circ h)\psi (x) \mathrm{dx} = \int_{V}^{}\varphi (y)\psi (h^{-1}(y)) | \det{h^{-1}'}(y) | \mathrm{d}y = \int_{V}^{}\varphi (y)(H'\psi )(y) \mathrm{d}y,
	\]
	onde
	\[
		H'\psi \coloneqq (\psi \circ h^{-1})| \det{(h^{-1})'} |
	\]
	é, portanto, o transposto formal de H (a linearidade e continuidade dele também ficam como exercício!). Logo, podemos definir \(u\circ h\) para uma distribuição \(u\in \mathcal{D}'(V)\) pondo
	\[
		\langle u\circ h, \psi  \rangle=\langle u, (\psi \circ h^{-1})| \det{(h^{-1})'} | \rangle,\quad \psi \in \mathcal{C}_{c}^{\infty}(U),\; u\in \mathcal{D}'(V).
	\]
	Este exemplo é indispensável para que se possa definir um tipo de distribuições em variedades ou superfícies.
\end{example}
\begin{example}[Rotações, Translações e Dilatações]
	Sejam \(\Omega \) um aberto de \(\mathbb{R}^{n}\) e h um vetor fixo. Consideremos
	\begin{align*}
		\tau_h: & \mathcal{C}_{c}^{\infty}(\Omega +h)\rightarrow \mathcal{C}_{c}^{\infty}(\Omega ) \\
		        & \varphi \longmapsto \tau_h\varphi (x)\coloneqq \varphi (x-h),
	\end{align*}
	a transação por h. Com isso, note que podemos escrever, para \(g:\Omega +h\rightarrow \Omega \),
	\[
		\tau_h\varphi = \varphi \circ g, \quad g(x)=x-h,
	\]
	que está nas condições da mudança de variável do exemplo acima; então, faz sentido definirmos, para uma distribuição \(u\in \mathcal{D}'(\Omega )\), a translação dela por h pondo:
	\[
		\langle \tau_h, u \rangle = \langle u, \varphi \circ \tau_{-h} \rangle,
	\]
	que às vezes pode ser escrita também como \(\varphi \circ \tau_{-h}=\varphi (\cdot +h)\); logo,
	\[
		\langle \tau_hu, \varphi  \rangle = \langle u, \varphi (\cdot +h) \rangle.
	\]

	Nos casos em que temos uma rotação \(T:\mathbb{R}^{n}\rightarrow \mathbb{R}^{n}\), com T linear satisfazendo \(TT^{*}= I\) e \(| \det{(T)} = 1 |\), e uma dilatação \(d:\mathbb{R}^{n}\rightarrow \mathbb{R}^{n}\) com \(d(x)=tx\), colocamos
	\[
		\langle u \circ T, \varphi  \rangle= \langle u, \varphi \circ T^{*} \rangle \quad\&\quad \langle u\circ d, \varphi  \rangle = \langle u, (\varphi \circ d^{-1})t^{-n} \rangle
	\]
	para uma distribuição \(u\in \mathcal{D}'(\mathbb{R}^{n})\) e função teste \(\varphi \in \mathcal{C}_{c}^{\infty}(\mathbb{R}^{n})\); ou seja,
	\[
		\langle u\circ d, \varphi  \rangle = \langle u, \varphi_{t} \rangle,\quad \varphi_{t}(x)=t^{-n}\varphi \biggl(\frac{x}{t}\biggr).
	\]

	Estas situações nos permitem introduzir:
	\begin{itemize}
		\item[i)] as funções testes \(\hat{\varphi }=\varphi (-x)\) e \(\varphi_{t}(x)=\varphi (tx)\);
		\item[ii)] distribuições pares e ímpares: se \(T=-I\), então \(u\circ T = u\), a qual chamaremos de \textbf{par}, e, se \(u\circ T=-u\), chamaremos de \textbf{ímpar}; e
		\item[ii)] distribuições homogêneas de grau \(\theta >0\) quando \(u_{t}=t^{\theta }u.\)
	\end{itemize}

\end{example}
\begin{tcolorbox}[
		skin=enhanced,
		title=Observação,
		fonttitle=\bfseries,
		colframe=black,
		colbacktitle=cyan!75!white,
		colback=cyan!15,
		colbacklower=black,
		coltitle=black,
		drop fuzzy shadow,
		%drop large lifted shadow
	]
	Uma distribuição \(u\in \mathcal{D}'(\mathbb{R}^{n})\) se diz \textbf{periódica de período \(h\in \mathbb{R}^{n}\)} se \(\tau_h u = u\), e um caso particularmente útil é o caso em que \(h\in \mathbb{Z}^{n}\), ou seja, é um período inteiro.
\end{tcolorbox}
\begin{tcolorbox}[
		skin=enhanced,
		title=Observação,
		fonttitle=\bfseries,
		colframe=black,
		colbacktitle=cyan!75!white,
		colback=cyan!15,
		colbacklower=black,
		coltitle=black,
		drop fuzzy shadow,
		%drop large lifted shadow
	]
	Além disso, a translação fornece outra maneira de definirmos a derivada parcial \(\frac{\partial^{}u}{\partial x_{j}^{}}\): dada \(u\in \mathcal{D}'(\mathbb{R}^{n})\) e \(\tau_{-te_{j}},\) segue que
	\[
		\biggl\langle \frac{\tau_{-te_{j}}u - u}{t}, \varphi  \biggr\rangle = \biggl\langle u, \frac{\varphi (\cdot -te_{j}) - \varphi (\cdot )}{t} \biggr\rangle,
	\]
	donde pode-se observar que
	\[
		\frac{\varphi (\cdot -te_{j}) - \varphi (\cdot )}{t} \substack{\mathcal{C}_{c}^{\infty}(\mathbb{R}^{n}) \\ \longrightarrow \\ t\to 0} - \frac{\partial^{}\varphi }{\partial x_{j}^{}}.
	\]
\end{tcolorbox}

\subsection{Distribuições com Derivadas Nulas}
Para começar, vamos nos restringir às distribuições num intervalo aberto \(I \subseteq \mathbb{R}\) e tentemos caracterizar as distribuições \(u\in \mathcal{D}'(I)\) tais que \(u'=0\) em \(\mathcal{D}'\). Conforme já sabemos, se \(u\in L_{\mathrm{loc}}^{1}(I)\) é constante em todo ponto deste intervalo, então
\[
	\langle u', \varphi  \rangle = - \int_{I}^{}c\varphi ' = -c \biggl[\int_{a}^{b}\varphi '(x) \mathrm{dx}\biggr] = - c[\varphi (b) - \varphi (+a)] = 0,
\]
tendo em vista que \(\mathrm{supp}(\varphi )\subseteq [a, b]\subseteq I\). Nosso objetivo será estabelecer algum tipo de recíproca, e, para  isso, precisaremos do
\begin{lemma*}
	Uma função teste \(\varphi \in \mathcal{C}_{c}^{\infty}(I)\) possui uma primitiva \(\psi \in \mathcal{C}_{c}^{\infty}(I)\) se, e somente se, \(\int_{I}^{}\varphi = 0\)
\end{lemma*}
\begin{proof*}
	Seja \(\psi \in \mathcal{C}_{c}^{\infty}(I)\) com \(\psi ' = \varphi \)  e \(\mathrm{supp}(\psi )\subseteq [a, b]\subseteq I\); então,
	\[
		\int_{I}^{}\varphi = \int_{a}^{b}\psi '(x) \mathrm{d}x = \psi (b) - \psi (a) = 0-0=0.
	\]
	\begin{figure}[H]
		\begin{center}
			\includegraphics[height=0.5\textheight, width=0.5\textwidth, keepaspectratio]{./Images/phi_psi_13.png}
		\end{center}
		\caption{As áreas positivas e negativas devem tero mesmo valor absoluto.}
	\end{figure}


	Reciprocamente, definindo
	\[
		\psi (x) = \int_{-\infty}^{x}\varphi (t) \mathrm{d}t
	\]
	e admitindo \(\mathrm{supp}(\varphi )\subseteq [a, b] \subseteq I\), teremos
	\[
		x\leq a \Rightarrow \psi (x) = \int_{-\infty}^{x}\varphi (t) \mathrm{d}t = \int_{-\infty}^{x}0 \mathrm{d}t = 0
	\]
	e, em outra via,
	\[
		x\geq b \Rightarrow \psi (x) = \int_{-\infty}^{x}\varphi (t) \mathrm{d}t = \int_{a}^{b}\varphi (t) \mathrm{d}t = 0,
	\]
	tal que concluímos que \(\mathrm{supp}(\varphi )\subseteq [a, b]\). Por fim, o fato de \(\psi \in \mathcal{C}^{\infty}\) decorre de \(\psi \) a ser e, por definição, \(\psi ' = \varphi .\). \qedsymbol
\end{proof*}
\begin{theorem*}
	Se u for uma distribuição satisfazendo \(u'= 0\) em \(\mathcal{D}'\), então existe um número complexo \(c\in \mathbb{C}\) tal que
	\[
		\langle u, \varphi  \rangle = c \int_{I}^{}\varphi, \quad \varphi \in \mathcal{C}_{c}^{\infty}(I),
	\]
	isto é,
	\[
		\langle u, \varphi  \rangle = \int_{I}^{}c\varphi
	\]
	e u provém da função \(L_{\mathrm{loc}}^{1}(I)\) constante em quase todo ponto, ou seja, \(f=c\) q.t.p.
\end{theorem*}
\begin{proof*}
  Antes de tudo, note que a hipótese significa que, para toda função teste \(\psi \in \mathcal{C}_{c}^{\infty}(I)\), 
    \[
      0=\langle u', \psi  \rangle = - \langle u, \psi ' \rangle.
    \]
    Em particular, se \(\varphi \in \mathcal{C}_{c}^{\infty}\) é tal que \(\varphi = \psi '\) com \(\psi \in \mathcal{C}_{c}^{\infty}(I)\), então \(\langle u, \varphi  \rangle=0\). Dito isso, fixemos \(\varphi_{0}\in \mathcal{C}_{c}^{\infty}(I)\) com \(\int_{}^{}\varphi_{0}=1\), construída ao tomar \(\varphi_1\in \mathcal{C}_{c}^{\infty}\) com \(i=\int_{}^{}\varphi_1\) e definir 
      \[
        \varphi_{0}\coloneqq \frac{1}{c}\varphi_1.
      \]
     Desta forma, dada uma \(\varphi \in \mathcal{C}_{c}^{\infty}(I)\), a função \(\psi =\psi_{\varphi }\) inspirada no Riesz e na projeção ortogonal definida por 
       \[
         \psi (x)\coloneqq \varphi_{0}(x) \int_{I}^{}\varphi - \varphi (x),\quad x\in I,
       \]
       cumpre: 
      \begin{align*}
          &(i)\quad \psi \in \mathcal{C}_{c}^{\infty}(I); \text{ e} 
          &(ii)\quad \int_{}^{}\psi = \int_{}^{}\varphi_{0} \int_{}^{}\varphi - \int_{}^{}\varphi = \int_{}^{}\varphi - \int_{}^{}\varphi =0.
      \end{align*}
  Agora, do lema anterior, \(\psi =\phi^{'}\) com \(\phi\in \mathcal{C}_{c}^{\infty}(I)\) e, da hipótese, devemos ter 
    \[
      0=\langle u, \psi  \rangle = \int_{I}^{}\varphi \langle u, \varphi_{0} \rangle - \langle u, \varphi  \rangle,
    \]
    isto é, 
      \[
        \langle u, \varphi  \rangle = c \int_{I}^{}\varphi,
      \]
      onde \(c = \langle u, \varphi_{0} \rangle\), provando o teorema. \qedsymbol
\end{proof*}
  \begin{tcolorbox}[
  skin=enhanced,
  title=Observação,
  fonttitle=\bfseries,
colframe=black,
  colbacktitle=cyan!75!white, 
  colback=cyan!15,
  colbacklower=black,
coltitle=black,
  drop fuzzy shadow,
  %drop large lifted shadow
  ]
  Note que c não depende da \(\varphi_{0}\) fixada, pois se \(\int_{}^{}\varphi_1\), então 
    \[
      \langle u, \varphi_1 \rangle = c \int_{I}^{}\varphi_1 = c. 
    \]
    A prova do caso n-dimensional revela fatos que nos obrigarão a saber o que deve ser o futuro \textit{produto tensorial} de distribuições.
  \end{tcolorbox}

 \begin{crl*}
   Se u for uma distribuição em \(\mathcal{D}'(\mathbb{R})\) ou \(\mathcal{D}'(I)\), com \(I\subseteq \mathbb{R}\) qualquer, e for tal que \(u'= au\) para algum \(a\in \mathbb{C},\) então existe \(c\in \mathbb{C}\) satisfazendo 
     \[
       u(x)=ce^{ax},\quad x\in \mathbb{R}.
     \]
     Noutras pacavaras, a EDO \(u'= au\) não admite soluções novas além das clássicas que já possuía. 
 \end{crl*}
 \begin{proof*}
   Com efeito, como \(v(x)=e^{-ax}\) satisfaz \(v'= - av\), o resultado que obtemos pelo fator integrante é 
     \[
       e^{-ax}u' - \underbrace{a e^{-ax}}_{-av}u = 0 \Rightarrow (e^{-ax}u)'=0 \Rightarrow \exists\; c\in \mathbb{C}:\; e^{-ax}u = c.
     \]
     Portanto, 
       \[
         u = c e^{ax},
       \]
       tendo em vista que \(e^{z}\) não possui zeros. \qedsymbol
 \end{proof*}
 
 \textbf{\underline{Escólio 1:}}\footnotetext{Definido como um comentário que às margens de um texto, pode ser para crítica, explicação, ou outra coisa.} se uma distribuição u for tal que 
   \[
     u'-au= f\in \mathcal{C}^{\infty}(\mathbb{R}),
   \]
   então a distribuição em si é de uma função de classe \(\mathcal{C}^{\infty}(\mathbb{R})\). Com efeito, basta lembrar que 
     \[
       v(x)=ce^{ax} + \int_{0}^{x}e^{a(x-t)}f(t) \mathrm{d}t 
     \]
     resolve a equação e considerar \(w\coloneqq u-v\); daí, segue que 
    \begin{align*}
      w'= u'-v' &= au + f - av - f \\ 
                &= a(u-v)\\ 
                &= aw \\ 
                &\Rightarrow w=\alpha e^{ax}\\ 
                &\Rightarrow u = (c+\alpha )e^{ax} + \int_{0}^{x}e^{a(x-t)}f(t) \mathrm{d}t.   
    \end{align*}
    Logo, u é de classe \(\mathcal{C}^{\infty}.\)
   \begin{exr}
     Melhore o resultado acima colocando \(f\in \mathcal{C}^{m}(\mathbb{R}).\)
   \end{exr}

     \begin{tcolorbox}[
     skin=enhanced,
     title=Observação,
     fonttitle=\bfseries,
   colframe=black,
     colbacktitle=cyan!75!white, 
     colback=cyan!15,
     colbacklower=black,
   coltitle=black,
     drop fuzzy shadow,
     %drop large lifted shadow
     ]
     Se \(F'= \delta ,\) onde \(F\in \mathcal{D}'(\mathbb{R}),\) então \(F=c+H\) para algum c complexo.
     \end{tcolorbox}

     \begin{crl*}[Escólio 2]
      Se \(E\in \mathcal{D}'(\mathbb{R})\) satisfaz \(E' - a E = \delta \), onde \(\delta \) é o delta de Dirac, então existe c complexo tal que  
        \[
          E = (H+c)e^{ax},
        \]
        onde H é a heaviside, e a equação acima representa as soluções fundamentais do operador \(P(D)u = \frac{\mathrm{d}u}{\mathrm{d}x}-au.\)
    \end{crl*}
   \begin{proof*}
     Analogamente à prova do corolário 1, escrevemos:
     \begin{align*}
      & e^{-ax}E'-ae^{-ax}=e^{-ax}\delta =\delta \\ 
      \Longleftrightarrow & (e^{-ax}E)' = \delta \\ 
                          \Rightarrow &\exists\; c\in \mathbb{C}: e^{-ax}E = H+c\\ 
                          \Rightarrow & E = (H+c)e^{ax}. \text{ \qedsymbol}
     \end{align*}
   \end{proof*}
   \begin{exr}
      Mostre que se a n-éisma derivada de u é nula, então ele é um polinômio de grau, no máximo, m-1, isto é. 
      \[
        \frac{\mathrm{d}^{n}u}{\mathrm{d}x^{n}}=0 \Rightarrow u = \sum\limits_{j=0}^{m-1}a_{j}x^{j}
      \]
   \end{exr}

   Não é verdade, em geral, que a derivada parcial de f com respeito a \(x_{j}\) ser nula significa que f não depende da variável. A hipótese comumente adotada para que isso ocorra é supor que o domínio U de f é ``j-covexo'', o que quer dizer que, dados 
     \[
       x=x_{0}+t_{0}e_{j}\;\&\; y=x_{0}+s_{0}e_{j}\in U,
     \]
     então 
       \[
         z(t)=x_{0}+te_{j}\in U,\quad \forall s_{0}\leq t\leq t_{0}.
       \]
       Um exemplo disto é tomar \(U = V \times I\), onde \(V\subseteq \mathbb{R}^{n-1}\) e I é um intervalo, resultando num conjunto U convexo na ``última variável''.
      \begin{figure}[H]
      \begin{center}
      \includegraphics[height=0.4\textheight, width=0.4\textwidth, keepaspectratio]{./Images/two_convex_13.png}
      \end{center}
      \caption{exemplo do U 2-convexo conforme ilustrado no exemplo.}
      \end{figure}

     \begin{figure}[H]
     \begin{center}
     \includegraphics[height=0.6\textheight, width=0.6\textwidth, keepaspectratio]{./Images/rectangle_13.png}
     \end{center}
     \caption{Neste exemplo, U é um retângulo aberto menos um segmento fechado na direção \(e_2\). Aqui, \(f:U\rightarrow \mathbb{R}\), onde \(U\subseteq \mathbb{R}^{2}\), e \(\partial_{x_1} f = 0\) em U, mas f não é constante na primeira variável.}
     \end{figure}

     Assim, a geometria do domínio do problema deve, fortemente, ser levado em conta na \textbf{análise de qualquer equação!}
    \begin{theorem*}
      Sejam \(U\) um aberto de \(\mathbb{R}^{n}\) com variáveis \(x=(x_1,\dotsc , x_{n})\) e \(I\subseteq \mathbb{R}\) um intervalo aberto com variáveis t. Se \(u\in \mathcal{D}'(U\times I)\) e \(\partial_t u=0\), então existe \(u_{0}\in \mathcal{D}'(U)\) tal que, para toda função teste \(\varphi \in \mathcal{C}_{c}^{\infty}(U\times I)\), tem-se 
        \[
          \langle u, \varphi  \rangle = \biggl\langle u_{0}, \int_{I}^{}\varphi (\cdot , t) \mathrm{dt} \biggr\rangle = \int_{I}^{}\langle u_{0}, \varphi (\cdot , t) \rangle \mathrm{dt}.
        \]
    \end{theorem*}
    Antes de prová-lo, formalizemos a notação do produto tensorial: dadas \(f\in \mathcal{C}(U)\) e \(g\in \mathcal{C}(I)\), escrevmos 
      \[
        (f\otimes g)(x, t)=f(x)g(t),
      \]
      donde temos também \(f\otimes g\in \mathcal{C}(U\times I)\).
   \begin{proof*}
     Tentando repetir a prova do primeiro Teorema, observemos que, para todo \(\varphi \in \mathcal{C}_{c}^{\infty}(U\times I)\), temos 
       \[
         \biggl\langle u, \frac{\partial^{}\varphi }{\partial t^{}} \biggr\rangle = - \biggl\langle \frac{\partial^{}u}{\partial t^{}}, \varphi \biggr\rangle = 0.
       \]
       Além disso, existe \(\psi \in \mathcal{C}_{c}^{\infty}(U\times I)\) com 
         \[
           \frac{\partial^{}\psi }{\partial t^{}} \Longleftrightarrow \int_{I}^{}\varphi (x, t) \mathrm{d}t =0, \quad \forall x\in U.
         \]
         Dito isso, para cada \(\varphi \in \mathcal{C}_{c}^{\infty}(U\times I)\), teremos 
           \[
             \Phi (x, t)= \overbrace{\varphi_{0}(t) \underbrace{\int_{I}^{}\varphi (x, s) \mathrm{d}s}_{=\psi (x)}}^{(\varphi_{0}\otimes \psi )(x, t)} - \varphi (x, t),
           \]
           onde \(\varphi_{0}\) é de classe \(\mathcal{C}_{c}^{\infty}(I)\) e satisfaz \(\int_{I}^{}\varphi_{0} =1\). Assim, a \(\Phi \) cumpre, para todo x em U, 
             \[
               \int_{I}^{}\Phi (x, t) \mathrm{d}t = 0, \;\&\; \Phi \in \mathcal{C}_{c}^{\infty}(U\times I),
             \]
             tendo em vista que podemos pôr 
               \[
                 \psi (x)=\int_{I}^{}\varphi (x, t) \mathrm{d}t
               \]
               para obter \(\mathrm{supp}(\psi )\subseteq \Pi_{\mathbb{R}^{n}}(\mathrm{supp}(\varphi ))\subseteq U\) e 
                 \[
                 \partial^{\alpha }\psi (x) = \int_{I}^{}\partial^{\alpha }\varphi (x, t) \mathrm{d}t.
                 \]

             Consequentemente, 
               \[
                 0=\langle u, \Phi  \rangle = \biggl\langle u, \varphi_{0}\otimes \int_{I}^{}\varphi (\cdot , s) \mathrm{d}s \biggr\rangle - \langle u, \varphi  \rangle \Longleftrightarrow \langle u, \varphi  \rangle = \biggl\langle u, \varphi_{0} \otimes \int_{I}^{}\varphi (\cdot , s) \mathrm{d}s \biggr\rangle
               \]

               Resta provar que ``\(\langle u_{0}, \psi  \rangle\coloneqq \langle u, \varphi_{0}\otimes \psi  \rangle\)'' define de fato uma distribuição em U. Com efeito, se \(\psi_{j}\to 0\) em \(\mathcal{C}_{c}^{\infty}(U)\) e \(\mathrm{supp}(\psi_{j})\subseteq K\subseteq U\) para todo j, segue que 
                 \[
                   \mathrm{supp}(\psi_{j}\otimes \varphi_{0})\subseteq \mathrm{supp}(\psi_{j}) \times \mathrm{supp}(\varphi_{0})\subseteq K\times \mathrm{supp}(\varphi_{0}) \subseteq U\times I
                 \]
    e 
      \[
        \partial_{x}^{\alpha }\partial_{t}^{\beta }(\psi_{j}\otimes \varphi_{0})=(\partial_{x}^{\alpha }\psi_{j})\otimes (\partial_{t}^{\beta }\varphi_{0})\to (\partial_{x}^{\alpha }0) \otimes (\partial_{t}^{\beta }\varphi_{0})
      \]
      em  \(\mathcal{C}_{c}^{\infty}(U\times I)\), tendo em vista que 
        \[
          \sup_{}| \partial_{x}^{\alpha }\psi_{j}(x)\partial_{t}^{\beta }\varphi_{0}(t) | = \sup_{K}| \partial_x^{\alpha } \psi_j(x)| \sup_{I}| \partial_{t}^{\beta }\varphi_{0}(t) |\substack{ \\ \longrightarrow \\ j\to0}0.
        \]
        Portanto, \(u_{0}\) é uma distribuição em U. \qedsymbol
   \end{proof*}
  \begin{exr}
    prove a outra igualdade.
  \end{exr}
\end{document}

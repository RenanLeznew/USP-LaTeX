\documentclass{article}
\usepackage{amsmath}
\usepackage{amsthm}
\usepackage{amssymb}
\usepackage{pgfplots}
\usepackage{amsfonts}
\usepackage[margin=2.5cm]{geometry}
\usepackage{graphicx}
\usepackage[export]{adjustbox}
\usepackage{fancyhdr}
\usepackage[portuguese]{babel}
\usepackage{hyperref}
\usepackage{lastpage}
\usepackage{mathtools}
\usepackage{subfiles}
\usepackage[T1]{fontenc}

\usepackage[tracking]{microtype}

\usepackage[sc,osf]{mathpazo}   % With old-style figures and real smallcaps.
\linespread{1.025}              % Palatino leads a little more leading

% Euler for math and numbers
\usepackage[euler-digits,small]{eulervm}
\AtBeginDocument{\renewcommand{\hbar}{\hslash}}

\pagestyle{fancy}
\fancyhf{}

\pgfplotsset{compat = 1.18}

\hypersetup{
  colorlinks,
  citecolor=black,
  filecolor=black,
  linkcolor=black,
  urlcolor=black
}
\newtheorem*{def*}{\underline{Definição}}
\newtheorem*{theorem*}{\underline{Teorema}}
\newtheorem{example}{\underline{Exemplo}}[section]
\newtheorem*{proof*}{\underline{Prova}}
\newtheorem*{prop*}{\underline{Proposição}}
\newtheorem*{crl*}{\underline{Corolário}}
\newtheorem*{lmm*}{\underline{Lema}}
\newtheorem*{exer*}{\underline{Exercícios}}
\renewcommand\qedsymbol{$\blacksquare$}

\rfoot{Página \thepage \hspace{1pt} de \pageref{LastPage}}

\title{Funções de Variáveis Complexas}
\vspace{1.5cm}
\author{Prof. Thaís Jordão\\
  \vspace{2cm}\\
  Notas por:\\
  Lucas Giraldi Almeida Coimbra\\
  Renan Wenzel\\
  \vspace{4cm}\\
  \href{https://github.com/RenanLeznew/USP-Math-LaTeX/tree/master/ComplexAnalysis}{GitHub com o Arquivo das Notas (Link clicável)}\\
  \url{https://github.com/RenanLeznew/USP-Math-LaTeX/tree/master/ComplexAnalysis}\\
  \vspace{4cm}\\
}
\date{\today}

\begin{document}
\maketitle
\newpage
\tableofcontents
\newpage

\subfile{ComplexAnalysis/classes/aula01}
\newpage
\subfile{ComplexAnalysis/classes/aula02}
\newpage
\subfile{ComplexAnalysis/classes/aula03}
\newpage
\subfile{ComplexAnalysis/classes/aula04}
\newpage
\subfile{ComplexAnalysis/classes/aula05}
\newpage
\subfile{ComplexAnalysis/classes/aula06}
\newpage
\subfile{ComplexAnalysis/classes/aula07}
\newpage
\subfile{ComplexAnalysis/classes/aula08}
\newpage
\subfile{ComplexAnalysis/classes/aula09}
\newpage
\subfile{ComplexAnalysis/classes/aula10}
\newpage
\subfile{ComplexAnalysis/classes/aula11}
\newpage
\subfile{ComplexAnalysis/classes/aula12}
\newpage
\subfile{ComplexAnalysis/classes/aula13}
\newpage
\subfile{ComplexAnalysis/classes/aula14}
\newpage
\section*{COMENTÁRIO À PARTE}
\paragraph{}  A partir desse momento, as aulas são dadas pelos alunos através da forma de Seminários. Neste prisma, o crédito será
devidamente dado a cada grupo e, caso não ocorra isso, por favor, me contate para que possa ser corrigido. Ademais, os créditos
também devem ser dados à professora Thaís pela organização e orientação.
\newpage
\subfile{ComplexAnalysis/classes/aula15}
\newpage
\subfile{ComplexAnalysis/classes/aula16}
\newpage
\subfile{ComplexAnalysis/classes/aula17}
\newpage
\subfile{ComplexAnalysis/classes/aula18}
\newpage
\subfile{ComplexAnalysis/classes/aula19}
\newpage
\end{document}

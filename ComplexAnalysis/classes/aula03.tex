\documentclass[complex.tex]{subfiles}
\begin{document}
\section{Aula 03 - 06/01/2023}
\subsection{Motivações}
\begin{itemize}
	\item[i)] Introdução às séries de potência e raio de convergência;
	\item[ii)] Funções analíticas e diferenciáveis em $\mathbb{C}$;
	\item[iii)] Definição da exponencial complexa;
	\item[iv)] Ramos de funções inversas.
\end{itemize}
\subsection{Séries de Potências}
\begin{def*}
	Considere $\{a_{n}\}$ uma sequência em $\mathbb{C}$. A série de potência em $\{a_n\}$, denotada
	por $\sum\limits_{n=0}^{\infty}$, é dita \textbf{convergente} se para todo $\epsilon > 0$, existe $n_0\in\mathbb{N}$
	tal que $|\sum\limits_{n=0}^{k} - a|, k\geq{n_0}$, para algum $a\in\mathbb{C}$. Denotamos isso por
	$$
		a = \sum_{n=0}^{\infty} a_n < \infty,
	$$
	A série $\sum\limits_{n=0}^{\infty}a_n$ é \textbf{absolutamente convergente} se $\sum\limits_{n=0}^{\infty}|a_n|<\infty$.
\end{def*}
\begin{exer*}
	Mostre que se uma soma converge absolutamente, ela também converge normalmente.
\end{exer*}
\begin{def*}
	Uma \textbf{série de potências} é uma série da forma
	$$
		\sum_{n=0}^{\infty}a_n(z-a)^n, \quad z\in\mathbb{C},
	$$
	em que $\{a_n\}$ é uma sequência de $\mathbb{C}$ e a é um número complexo.
\end{def*}
\begin{example}
	No caso da série geométrica $\sum\limits_{n=0}^{\infty}z^n, z\in\mathbb{C}$, considere
	a soma parcial $s_n = \sum\limits_{k=0}^{n} = \frac{1 - z^{n+1}}{1-z}, z\neq{1}.$ Se
	$|z| < 1,$ então $z^{n+1}\to{0}$ e $\sum\limits_{n=0}^{\infty}z^n = \frac{1}{1-z}, |z| < 1.$
	Caso $|z|\geq{1},$ a série geométrica diverge.
\end{example}
Denotamos por $\limsup_{n\to\infty}\{b_n\}$ a expressão $\lim_{n\to\infty}\sup_{k\geq{n}}\{b_k\}$.
\begin{theorem*}
	Considere a série de potências $\sum\limits_{n=0}^{\infty}(z-a)^n$ e $\frac{1}{R}\coloneqq \limsup_{n\to\infty}\{\sqrt[n]{|a_n|}\}$.
	Então,
	\begin{itemize}
		\item[1)] A série converge absolutamente em B(a, R)
		\item[2)] A série diverge se $|z-a| > R$
		\item[3)] A série converge uniformemente em B(a, r) para $0 < r < R.$
	\end{itemize}
\end{theorem*}
\begin{proof*}
	Sem perda de generalidade, suponha a = 0.
	1.) Seja $z\in{B(0, R)}.$ Existe $|z| < r < R$ e $n_0\in\mathbb{N}$ tal que $|a_n^{\frac{1}{n}}| < \frac{1}{r},
		n\geq{n_0}.$ Daí, temos
	$$
		\sum_{k=n_0}^{\infty}|a_n||z^n|\leq \sum_{k=n_0}^{\infty}\frac{|z^n|}{r^n} < \infty.
	$$
	Como essa fração é menor que um, o resultado está provado.

	2.) Seja $|z| > R$ e r tal que $|z|> r > R$. Existe $\{a_{n_k}\}_k$ tal que $|a_{n_k}|^{\frac{1}{n_k}} > \frac{1}{r},
		k = 0, 1, \cdots.$ Assim, temos
	$$
		|a_{n_k}||z|^{n_k} > \biggl(\frac{|z|}{r}\biggr)^{n_k}\to\infty
	$$
	Conforme k tende a infinito.

	3.) Seja $0 < r < R \text{ e } r < \rho < R.$ Se z pertence a uma bola B(0, r), então
	$$
		|a_n||z|^n < \biggl(\frac{r}{\rho}\biggr)^n, \quad n\geq{n_0}, n_0\in\mathbb{N}.
	$$
	Como consequência do teste M de Weierstrass, já que $\frac{r}{\rho}$ é um número, segue
	o resultado.
	\qedsymbol
\end{proof*}
\begin{exer*}
	Mostre que o R do teorema acima é único.
\end{exer*}
\begin{example}
	Considere a série que define a exponencial de z:
	$$
		\sum_{n=0}^{\infty}\frac{z^n}{n!}, R = \infty.\quad e^z\coloneqq \sum_{n=0}^{\infty}\frac{z^n}{n!}, z\in\mathbb{C}.
	$$
	Este série é convergente pelo teste da razão. Com efeito,
	$$
		R = \lim_{n\to\infty}\biggl|\frac{a_n}{a_{n+1}}\biggr| = \lim_{n\to\infty}\biggl(\frac{(n+1)!}{n!}\biggr) = \infty.
	$$
	Com isso, a série converge para todos os valores possíveis, pois seu raio de convergência
	é infinito.
\end{example}
\begin{prop*}
	Nas notações da proposição anterior, se $R < \infty$, então
	$$
		R = \lim_{n\to\infty}\biggl|\frac{a_n}{a_{n+1}}\biggr|.
	$$
\end{prop*}

\subsection{Funções Analíticas}
\begin{def*}
	Seja G um aberto de $\mathbb{C}$ e $f:G\rightarrow\mathbb{C}$ uma função. Dizemos que ela é \textbf{diferenciável}
	em $z\in{G}$ se
	$$
		\lim_{h\to{0}}\frac{f(z+h) - f(z)}{h} = \lim_{w\to{z}}\frac{f(z)-f(w)}{z-w}
	$$
	existe. Neste caso, o denotamos por f'(z). Diremos que f é diferenciável se f'(z) existe para todo z de G.
\end{def*}
\begin{def*}
	Se $f:G\rightarrow\mathbb{C}$ é diferenciável e $f':G\rightarrow\mathbb{C}(z\mapsto{f'(z)})$ é contínua,
	então dizemos que f é \textbf{continuamente diferenciável}.

	Analogamente, se $f':G\rightarrow\mathbb{C}$ é diferenciável e $f'':G\rightarrow\mathbb{C}$ (f'' = (f')') é
	contínua, então f é \textbf{duas vezes continuamente diferenciável}. Nesta linha, diremos que uma função é
	\textbf{analítica} se ela é continuamente diferenciável em G.
\end{def*}

\begin{prop*}
	Seja G um aberto de $\mathbb{C}$. Então,
	\begin{itemize}
		\item[i)] Se $f:G\rightarrow\mathbb{C}$ é diferenciável em $a\in{G}$, então f é contínua em a;
		\item[ii)] Se f e g são analíticas em G, então f+g e f.g são analíticas em G. Se $G' = G - \{0\}$,
		      então f/g é analítica em G'. Valem as regras clássicas de derivação.
		\item[iii)] Sejam f e g analíticos em $G_f, G_g$, respectivamente, com $f(G_f)\subseteq{f(G_g)}$. Então,
		      $g\circ f$ é analítica em $G_f$ e
		      $$
			      (g\circ f)'(z) = g'(f(z))f'(z), \quad z\in{G}.
		      $$
	\end{itemize}
\end{prop*}
\begin{proof*}
	Exercício.
	\qedsymbol
\end{proof*}

\begin{prop*}
	Seja $f(z) = \sum\limits_{n=0}^{\infty}a_n(z-a)^n$ com raio de convergência R. Então,
	f é infinitamente diferenciável em B(a, R). Além disso, a derivada de ordem k é
	$$
		f^{(k)}(z) = \sum_{n=k}^{\infty}a_n\frac{n!}{(n-k)!}(z-a)^{n-k}, k\in\mathbb{N}
	$$
	com mesmo raio de convergência de f.
\end{prop*}
\begin{proof*}
	A última afirmação fica como exercício.

	Consideremos
	\begin{align*}
		 & s_n(z) = \sum_{k=0}^{n}a_k(z-a)^k, \quad R_n(z)= f(z) - s_n(z),                 \\
		 & g(z) = \sum_{n=1}^{\infty}a_nn(z-a)^{n-1}, \quad z\in{B(a, R)}, n\in\mathbb{N}.
	\end{align*}
	Seja $\delta > 0$ tal que $B(z, \delta)\subseteq{B(a, r)}$ com $|z| < r < R.$ Assim, para
	w em $B(z, \delta)$
	$$
		\frac{f(z)-f(w)}{z-w} - g(z) = \frac{s_n(z) - s_n(w)}{z-w} + \frac{R_n(z) - R_n(w)}{z-w} - g(z) =
		= \biggl[\frac{s_n(z) - s_n(w)}{z-w} - s_n'(z)\biggr] + \biggl[\frac{R_n(z) - R_n(w)}{z-w}\biggr] - (g(z) - s_n'(z)).
	$$
	Note que
	$$
		\biggl|\frac{R_n(z) - R_n(w)}{z-w}\biggr| = \biggl|\frac{1}{z-w}\sum_{k=n+1}^{\infty}a_k\frac{[(z-a)^k - (w-a)^k]}{(z-a)-(w-a)}\biggr|
		= \biggl|\sum_{k=n+1}^{\infty}a_k\biggl((z-a)^{k-1} + \cdots + (w-a)^{k-1}\biggr)\biggr|
		\leq \sum_{k=n+1}^{\infty}|a_k|kr^{k-1}\to{0},
	$$
	pois $g(r) < \infty,$ em que n tende a infinito.
	Como as duas expressões em chaves tendem a 0 quando w tende a z, concluímos que
	$$
		\lim_{z\to{w}}\frac{f(z)-f(w)}{z-w} = g(z)
	$$
	e a afirmação segue.
	\qedsymbol
\end{proof*}
\begin{crl*}
	Nas notações e condições da proposição anterior, f é analítica em
	B(a, R) e
	$$
		a_n = \frac{f^{(n)}(a)}{n!}, n\in\mathbb{N}
	$$
\end{crl*}
\begin{proof*}
	Exercício.
\end{proof*}
\begin{prop*}
	Seja G aberto e conexo. Se $f:G\rightarrow\mathbb{C}$ é tal que $f'(z) = 0, z\in{G},$
	então f é constante.
\end{prop*}
\begin{proof*}
	Seja $z_0\in{G}$ e considere $C = f^{-1}(\{f(z_0)\}),$ tal que C é não-vazio e fechado.
	Mostremos que C é, também, aberto.
	Seja z um elemento de C e $r > 0$ tal que $B(z, r)\subseteq{G}$. Para todo $w\in{B(z, w)}$,
	definimos $g:[0, 1]\rightarrow\mathbb{C}$ por g(t) = f(tz + (1-t)w). Neste caso,
	$$
		g'(t) = f'(tz + (1-t)w)(z-w) = 0.
	$$
	Como g é real, segue que ela é constante. Com isso, note que $f(w) = g(0) = g(1) =
		f(z) = f(z_0)$, tal que $w\in{C}$.
	\qedsymbol
\end{proof*}
\begin{example}
	$e^z = \sum_{n=0}^{\infty}\frac{z^n}{n!}, R = \infty$
\end{example}
Coloque $g(z) = e^ze^{z-w}, w\in\mathbb{C}$ fixo. Temos $g'(z) = (e^z)'e^{w-z} + e^z(e^{w-z})' - 0.$
Assim, g é constante e, como $g(0) = e^w$, concluímos que $e^w = e^ze^{w-z}$ para todo
$z, w\in\mathbb{C}$.
\begin{exer*}
	Prove que, para $z, w\in\mathbb{C},$:
	\begin{itemize}
		\item[1)] $e^{z+w} = e^ze^w;$
		\item[2)] $e^ze^{-z} = 1;$
		\item[3)] $e^{\overline{z}} = \overline{(e^z)}$;
		\item[4)] $|e^z| = e^{Re(z)}.$
	\end{itemize}
\end{exer*}
\begin{example}
	Defina, para z complexo,
	\begin{align*}
		 & \cos{(z)} = \sum_{n=0}^{\infty}(-1)^n\frac{z^{2n}}{(2n)!}      \\
		 & \sin{(z)} = \sum_{n=0}^{\infty}(-1)^n\frac{z^{2n+1}}{(2n+1)!}.
	\end{align*}
\end{example}
\begin{exer*}
	Dado z complexo, mostre que
	\begin{itemize}
		\item[i)] $(\sin{(z)}' = \cos{(z)}, \quad (\cos{(z)})' = -\sin{(z)};$
		\item[ii)] $\cos{(z)} = \frac{1}{2}\biggl(e^{iz} + e^{-iz}\biggr), \quad \sin{(z)} = \frac{1}{2}
			      \biggl(e^{iz} - e^{-iz}\biggr)$;
		\item[iii)] $\cos^2{(z)} + \sin^2{(z)} = 1$;
		\item[iv)] $e^{iz} = \cos{(z)} + i\sin{(z)}.$
	\end{itemize}
\end{exer*}

\subsection{Ramos de Funções Inversas}
Seja $z\in \mathbb{C}.$ Buscamos $w\in \mathbb{C}$ tal que $e^{w} = z, z\neq0.$ Logo, w deve satisfazer $|e^{w}|
	= e^{Re(w)} = |z|\Rightarrow Re(w) = \ln{|z|}.$ Se w = x + iy, então
$$
	e^{w} = e^{x}e^{iy} = e^{x}\biggl(\cos{(y)} + i\sin{(y)}\biggr) = z = |z|\biggl(\cos{(\theta)} + i\sin{(\theta)}\biggr)
$$
com $\theta = \arg{z}.$ Assim, $y = \theta + 2k\pi$ para algum k. Portanto, $w = \ln{|z|} + i(\arg{z} + 2k\pi), k\in \mathbb{Z}.$
\begin{def*}
	Seja G um aberto conexo de $\mathbb{C} e f:G\rightarrow \mathbb{C}$ contínua. Diremos que \textbf{f é um ramo de logarítmo}
	em G se $e^{f(z)} = z, z\in{G}.$
\end{def*}
\begin{prop*}
	Se G é um aberto conexo e f, g são ramos de logarítmos em G, então $f(z) = g(z) + 2k\pi i$ para algum
	$k\in \mathbb{Z}$.
\end{prop*}
\begin{proof*}
	Seja z em G. Mostraremos que
	$$
		\frac{f(z) - g(z)}{2\pi i}\in \mathbb{Z}.
	$$
	Observe que $e^{f(z) - g(z)} = \frac{e^{f(z)}}{e^{g(z)}} = \frac{z}{z} = 1.$ Daí, $f(z) = g(z) + 2k\pi i$ para algum
	inteiro k, pois
	$$
		f(z) - g(z) = \ln{|1|} + i(\arg{1} + 2k\pi)
	$$
	Definimos $h:G\rightarrow \mathbb{C}$ por
	$$
		h(w) = \frac{f(w) - g(w)}{2 \pi i}, \quad \in{G}.
	$$
	De forma análoga ao anterior, concluímos $Im(h)\subseteq{\mathbb{Z}}$ deve ser conexo, pois h é contínua. Assim,
	h é constante, pois os únicos conexos de $\mathbb{Z}$ são o vazio e conjuntos unitários, provando o resultado. \qedsymbol
\end{proof*}
\begin{prop*}
	Sejam $G, \Omega$ abertos e $f:G\rightarrow \mathbb{C}\text{ e }g:\Omega\rightarrow \mathbb{C}$ contínuas com $f(G)\subseteq{\Omega}$
	e satisfazendo $f(g(z)) = z, z\in{G}.$ Se g é diferenciável em z e $g'(f(z))\neq0,$ então
	$$
		f'(z) = \frac{1}{g'(f(z))}.
	$$
	Caso g seja analítica, f também o é.
\end{prop*}
\begin{proof*}
	Exercício.
\end{proof*}
Considere G um aberto conexo. Chamamos a função $f:G\rightarrow \mathbb{C} $ dada por
$$
	f(z) = \ln{|z|} + i \theta, \quad \theta=\arg{(z)}\in{(-\pi, \pi)}
$$
de ramo principal do logarítimo.
\end{document}

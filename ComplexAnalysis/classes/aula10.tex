\documentclass[ComplexAnalysis/complex.tex]{subfiles}
\begin{document}
\section{Aula 10 - 19/01/2023}
\subsection{Motivações}
\begin{itemize}
	\item Teorema do Máximo Módulo;
	\item índice de Curvas Fechadas e Dando Voltas em Círculos;
\end{itemize}

\subsection{Continuação de Zeros de Funções Analíticas}
\begin{prop*}
	Suponha $f:G\rightarrow \mathbb{C}$ analítica na região G. São equivalentes:
	\begin{itemize}
		\item[i)] $f\equiv0$;
		\item[ii)] Existe um a de G tal que $f^{(n)}(a) = 0, n = 0, 1, \cdots;$
		\item[iii)] O conjunto $\{z\in{G}: f(z) = 0\} = F_{0} $ possui ponto de acumulação.
	\end{itemize}
\end{prop*}
\begin{proof*}
	$iii)\Rightarrow ii)$ Seja a ponto de acumulação de $F_{0}$. Considere $\{a_{n}\} $ em $F_{0}$ tal que $a_{n}\to{a}$, f contínua em a.

	Afirmamos que $f^{(n)}(a) = 0.$  Seja m tal que $f^{(m)}(a)\neq 0 $ e $f^{(n)}(a) = 0, n = 0, 1, \cdots, m-1 $. Seja r positivo
	e, para z em B(a, r), temos
	$$
		f(z) = \sum\limits_{n=m}^{\infty}a_{n}(z-a)^{n} = (z-a)^{m}\underbrace{\sum\limits_{n=m}^{\infty}a_{n}(z-a)^{n-m}}_{g(z)\text{ analítica.}}.
	$$
	Temos g analítica e g(a) não-nulo. Logo, existe $0 < \delta < r$ tal que g(z) é não-nulo para z em $B(a, \delta)$. Portanto,
	$f(z)\neq0$ para z em $B(a, r) - \{a\} $, uma contradição, donde segue que a terceira afirmação deve implicar na segunda. \qedsymbol

	$ii)\Rightarrow i)$ Considere a tal que $f^{(n)}(a) = 0, n = 0, 1, \cdots$. Seja $A=\{z\in{G}: f^{(n)}(z) = 0, n=0, 1, \cdots\} $. Note que
	a pertence a A e, além disso,
	$$
		A = \bigcap_{n=0}^{\infty}\{z\in{G}: f^{(n)}(z) = 0\}
	$$
	é fechado. Mostremos que A é aberto. Tome b em A e r positivo, tal que $B(b, r)\subseteq{G}.$ Observe que
	$$
		f(z) = \sum\limits_{n=0}^{\infty}b_{n}(z-b)^{n}, \quad z\in{B(b, r)}.
	$$
	Portanto, A = G e segue a prova. \qedsymbol
\end{proof*}
\begin{crl*}
	Com as hipóteses do Teorema, se f(a) = 0, entõ existe $n_a\in \mathbb{Z}_{+}$ tal que $f(z) = (z-a)^{n_a}g(z), z\in{G},$
	com g analítica em g e $g(a)\neq0$ Em particular, os zeros de f são isolados.
\end{crl*}
\begin{theorem*}
	Seja f analítica numa região G. Se existe a em G tal que
	$$
		|f(z)|\leq|f(a)|, \quad z\in{G},
	$$
	então f é constante.
\end{theorem*}
\begin{proof*}
	Seja $r > 0$ tal que $B(a, r)\subseteq{G}.$ Mostraremos que $|f|:B(a, r)\rightarrow \mathbb{R}$ é constante na bola, mas $|f|\equiv|f(a)|\neq0.$
	Mostrado isso, se $f = u + iv$, então, para z em B(a, r), vale
	$$
		f(a) = \frac{1}{2\pi i}\int_{\gamma_{v}}^{}\frac{f(w)}{w-a}dw,
	$$
	$\gamma_{v}(t) = a + re^{it}, t\in[0, 2\pi].$
	$$
		|f(a)|\leq \frac{1}{2\pi}\biggl|\int_{0}^{2\pi}\frac{f(a + re^{it})}{re^{it}}rie^{it}dt\biggr|\leq \frac{1}{2\pi}\int_{0}^{2\pi}|f(a + re^{it})|dt
		\leq |f(a)|.
	$$
	Assim, $|f(a+re^{it})| = |f(a)|.$ Como r é qualquer tal que $B(a, r)\subseteq{G},$ a afirmação segue.

	Agora, se f = u + iv, z em B(a, r),
	$$
		|f(z)| = u^{2} + v^{2} = |f(a)|\neq 0.
	$$
	\begin{exer*}
		Mostre que
		$$
			\frac{\partial{u}}{\partial{x}} = \frac{\partial{v}}{\partial{y}} = 0,
		$$
		o que implicará f'(z) = 0. \qedsymbol.
	\end{exer*}
\end{proof*}
\subsection{índice de Curvas Fechadas}
\begin{example}
	Tome $\gamma(t) = a + re^{nit}, t\in[0, 2\pi], n=1, 2, \cdots$. Então,
	$$
		\frac{1}{2\pi i}\int_{\gamma_{n}}^{}\frac{1}{w-a}dw = \frac{1}{2\pi  i}\int_{0}^{2\pi}\frac{1}{re^{nit}}rine^{rint}dt = \frac{1}{2\pi i}\int_{0}^{2\pi} indt = n.
	$$
\end{example}
\begin{prop*}
	Considere $\gamma$ retificável e fechada. Então,
	$$
		\frac{1}{2\pi i}\int_{\gamma}^{}\frac{1}{w-a}dw\in \mathbb{Z}, \quad a\not\in \{\gamma\}.
	$$
\end{prop*}
\begin{proof*}
	Seja $\gamma$ curva $C^{1}$ por partes definida em [0, 1]. Considere $g(t) = \int_{0}^{t}\frac{\gamma'(s)}{\gamma(s) - a}ds, t\in[0, 1].$
	Note que g(0) = 0 e $g(1) = \int_{\gamma}^{}\frac{1}{w-a}dw.$ Defina $h(t) = e^{-g(t)}(\gamma(t) - a), t\in[0, 1]$ e observe que
	$h(0) = (\gamma(0) - a)\neq0.$ Derivando h:
	$$
		h'(t) = -g'(t)e^{-g(t)}(\gamma(t) - a) + e^{-g(t)}\gamma'(t) = 0, t\in[0, 1].
	$$
	Assim, $h\equiv{(\gamma(0) - a)}$. Neste caso, $h(1) = e^{-g(1)}(\gamma(0)-a) = \gamma(0) - a.$ Com isso, $e^{-g(1)} = 1.$ Portanto,
	$g(1) = 2\pi ki,$ para algum $k\in \mathbb{Z}.$ Portanto, a fórmula está provada. \qedsymbol
\end{proof*}
\end{document}

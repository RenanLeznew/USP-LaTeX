\documentclass[complex.tex]{subfiles}
\begin{document}
\section{Aula 06 - 12/01/2023}
\subsection{Motivações}
\begin{itemize}
	\item Transformações de M\"{o}bius e Harm\^onicos Conjugados;
	\item Simetrias e Orientação no Plano $\mathbb{C};$
	\item Integração Complexa.
\end{itemize}
\subsection{Exercícios de Hoje }
\subsubsection{Jéssica}
\begin{itemize}
	\item[a)] $(7+i, 1, 0, \infty)$
	\item[b)] $(2, 1-i, 1, 1+i)$
	\item[c)] $(0, 1, i, -1)$
	\item[d)] $(i-1, \infty, 1+i, 0)$
\end{itemize}
Utilizaremos os seguintes casos: Se $z_{2}, z_{3}, z_{4}\in \mathbb{C},$ então $S(z) = \displaystyle \frac{\frac{z-z_{3}}{z-z_{4}}}{\frac{z_{2} - z_{3}}{z_{2} - z_{4}}}.$
Caso $z_{3} = \infty, \displaystyle S(z) = \frac{z - z_{3}}{z - z_{4}}.$ Por fim, se $z_{4} = \infty, S(z) = \displaystyle \frac{z - z_{3}}{z_{2} - z_{3}}.$

Assim, vamos às contas.
\begin{itemize}
	\item[a)] $$S(7 + i) = \frac{7 + i}{1} = 7 + i;$$
	\item[b)] $$S(2) = \frac{\frac{2 -1}{2 - 1 - i}}{\frac{1 - i - 1}{1 - i - 1 - i}} = \frac{\frac{1}{1-i}}{\frac{i}{2i}} = \frac{2}{1-i}\frac{1+i}{1+i} = \frac{2}{2}(1 + i) = 1 + i$$
	\item[c)] $$S(0) = \frac{\frac{0-i}{0+1}}{\frac{1-i}{2}} = \frac{-2i}{1-i}\frac{1+i}{1+i} = \frac{-2i}{2}(1+i) = 1-i;$$
	\item[d)] $$S(1i) = \frac{i-1-1-i}{i-1-0} = \frac{-2}{i-1}\frac{1+i}{1+i} = \frac{-2}{-2}1+i = 1+i.$$
\end{itemize}

\subsubsection{Tiago}
Vamos mostrar que $T(\mathbb{R}_\infty) = \mathbb{R}_{\infty}\Longleftrightarrow a, b, c, d\in \mathbb{R}$.
$\Rightarrow)$ Suponha que $T(\mathbb{R}_\infty) = \mathbb{R}_\infty, T(z_{0}) = 0, z_{0}\in \mathbb{R}_\infty.$ Então,
$$
	\frac{az_{0} + b}{cz_{0} + d} = 0\Rightarrow az_{0} + b = 0\Rightarrow z_{0} = \frac{-b}{a}\in \mathbb{R}_\infty.
$$
No caso de $z_\infty \in \mathbb{R}_\infty = \infty, $ então
$$
	\frac{az_{\infty} + b}{cz_{\infty} + d} = \infty\Rightarrow \frac{cz_{\infty} + d}{az_{\infty} + b} = 0\Rightarrow cz_\infty + d = 0
	\Rightarrow z_\infty = \frac{-d}{c}\in \mathbb{R}_\infty.
$$
Agora, para $z_{1}\in \mathbb{R}_{\infty}, T(z_{1}) = 1,$ tal que
$$
	\frac{az_{1} + b}{cz_{1} + d} = 1\Rightarrow az_{1} + b = cz_{1} + d\Rightarrow z_{1}(a - c) = db\Rightarrow az_{1}\biggl(1 - \frac{c}{a}\biggr) = db
	\Rightarrow z_{1}\biggl(1 - \frac{c}{a}\biggr) = \frac{db}{a}
$$
$$
	\Rightarrow \frac{z_{1}}{c} - \frac{z_1}{a} = \frac{\frac{d}{a}}{c} - \frac{\frac{b}{a}}{c}\Rightarrow \frac{z_1}{c} - \frac{z_1}{a} =
	= \frac{r_2}{a} - \frac{r_1}{c}\Rightarrow \frac{z_1 + r_1}{c} = \frac{r_2 + z_1}{a}\Rightarrow \frac{z_1 + r}{z_1 + r_2} = \frac{c}{a}
$$
$$
	\Rightarrow \frac{d}{a} = \frac{d}{c}\frac{c}{a}\in \mathbb{R}_{\infty} = r_2r_3
$$
Logo, colocando $r_{1} = \frac{b}{a}, r_2 = \frac{d}{c}, r_3 = \frac{c}{a},$ encontramos os coeficientes
$$
	Tz = \frac{az + b}{cz + d} = \frac{a}{a}\frac{z + \frac{b}{a}}{z \frac{c}{a} + \frac{d}{a}} = \frac{z + r_1}{r_3z + r_2r_3}.
$$

$\Leftarrow)$ Para provar esse lado, considere $z\in \mathbb{R}_\infty.$ Então, $T(z) = \frac{az + b}{cz + d}\in \mathbb{R}_\infty.$ Portanto,
$T(\mathbb{R}_\infty) = \mathbb{R}_\infty.$  \qedsymbol

\subsection{Final de Transformações de M\"{o}bius.}
A continuação da prova da proposição é exercício.
\begin{theorem*}
	Transformações de M\"{o}bius levam círculos em círculos.
\end{theorem*}
\begin{proof*}
	Exercício.
\end{proof*}

\subsection{Simetria e Orientação.}
Dada uma circunferência $\Omega\text{ e }z_1, z_2, z_3\in \Omega$ distintos, diremos que z e z* são simétricos se
$[z^*, z_1, z_2, z_3] = \overline{[z, z_1, z_2, z_3]}$
\begin{example}
	Um ponto é simétrico a si mesmo se $z\in{C}$ com C o círculo determinado por $z_1, z_2, z_3.$
\end{example}
\begin{exer*}
	Mostre que a definição de simetria não depende da escolha dos $z_{i}'s.$
\end{exer*}
A ideia geométrica por trás desse conceito é a seguinte: Considere $\gamma$ uma reta e $z_1, z_2\in \mathbb{C}$
e coloque $z_3 = \infty.$ Dizer que z e z* são simétricos equivale a
$$
	[z^*, z_1, z_2, \infty] = \overline{[z, z_1, z_2, z_3]}\Longleftrightarrow \frac{z^* - z_2}{z_1 - z_2} = \overline{\biggl(\frac{z - z_2}{z_1 - z_2}\biggr)} =
	= \biggl(\frac{\overline{z} - \overline{z_2}}{\overline{z_1} - \overline{z_2}}\biggr).
$$
Assim, obtemos
$$
	\frac{z^* - z_2}{z_1 - z_2} = \frac{\overline{z} - \overline{z_2}}{\overline{z_1} - \overline{z_2}}
$$
o que implica
$$
	\frac{z^* - z_2}{|z_1 - z_2|^2} = \overline{z} - \overline{z_2} \quad \text{(Exercício: } |z^* - z_2| = |z - z_2|)
$$
para qualquer $z_2$ em r. Logo, d(z*, r) = d(z, r). Além disso, $[z^*, z]\perp{r}.$

A seguir, vamos lidar com o conceito de simetria com relação à um círculo de $\mathbb{C}.$ De fato, tome
\begin{align*}
	 & \Omega = \{z: |z - a| = r\}, \quad r > 0.                                                                                                                                                     \\
	 & [z^*, z_1, z_2, z_3] = \overline{[z, z_1, z_2, z_3]} = \quad \text{(Aplicando translação, inversão, homotetia:)}                                                                              \\
	 & = [\bar{z}, \bar{z_1}, \bar{z_2}, \bar{z_3}] = \biggl[\frac{r^2}{\bar{z} - \bar{a}}, \frac{r^2}{\bar{z_1} - \bar{a}}, \frac{r^2}{\bar{z_2} - \bar{a}}, \frac{r^2}{\bar{z_3} - \bar{a}}\biggr] \\
	 & = \biggl[\frac{r^2 + a}{\bar{z} - \bar{a}}, z_1 - a, z_2 - a, z_3 - a\biggr].
\end{align*}
Decorre que
$$
	z^* = a + \frac{r^2}{\bar{z} - \bar{a}}
$$
\begin{exer*}
	$z^*\in{l}\coloneqq  \{a + t(z-a): 0 < t < \infty\} $
\end{exer*}
\end{document}

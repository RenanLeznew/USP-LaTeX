\documentclass[complex.tex]{subfiles}
\begin{document}
\section{Aula 17 - 02/02/2023}
\subsection{Motivações}
\begin{itemize}
	\item Teorema dos Resíduos.
\end{itemize}
\subsection{Resíduos}
Seja f uma função com singularidade isolada a e \(\gamma \) uma curva fechada homotópica a 0 tal que \(a\not\in\{\gamma \}.\)
Se a é uma singularidade removível, então existe uma função analítica \(g:B(a, \delta )\rightarrow \mathbb{C}\) tal que
\(\{\gamma \}\subseteq B(a, \delta )\) e \(g(z) = f(z)\) para \(0 < |z-a| < \delta \), de modo que
\[
	\int_{\gamma }^{}f(z)dz = \int_{\gamma }^{}g(z)dz = 0.
\]
Mas e quando a é um polo ou uma singularidade esssencial? O que acontece com \(\int_{\gamma }^{}f\)?

Antes de responder esta pergunta, definiremos o que é um resíduo.
\begin{def*}
	Seja f uma função que admite singularidade isolada em z=a e
	\[
		f(z) = \sum\limits_{n=-\infty}^{\infty}b_{n}(z-a)^{n}
	\]
	a sua expansão de Laurente sobre a. Dizemos que o coeficiente \(b_{-1}\) é chamado \textbf{resíduo de f em a},
	denotado por \(\mathrm{Res}(f; a).\quad \square\)
\end{def*}
\begin{example}
	Consideramos a função \(f(z) = e^{\frac{1}{z}}.\) Sabe-se que \(e^{z}\) é analítica e admite expansão em série
	de Taylor, tal que
	\[
		e^{z} = 1 + z + \frac{1}{2}z^{2} + \frac{1}{3!}z^{3} + \dotsc
	\]
	Dessa forma, temos a expansão de Laurent de f em torno de 0, dada por
	\[
		e^{\frac{1}{z}} = 1 + \frac{1}{z} + \frac{1}{2}\frac{1}{z^{2}}+\dotsc  = \sum\limits_{n=-\infty}^{-1}\underbrace{\frac{1}{n!}}_{b_{n}}(z-0)^{n}
	\]
	Logo, \(\mathrm{Res}(f; 0) = b_{-1} = 1.\)
\end{example}
\hypertarget{residue}{ \begin{theorem*}
		Seja f uma função analítica na região G com exceção das singularidades isoladas \(a_{n},\dotsc ,a_{n}.\) Se \(\gamma \) é uma curva fechada e retificável
		em G com \(a_{k}\not\in\{\gamma \}\) para \(k=1, \dotsc , n\) e tal que \(\gamma \sim 0,\) então
		\[
			\frac{1}{2\pi i}\int_{\gamma }^{}f(z)dz = \sum\limits_{k=1}^{n}n(\gamma; a_{k})\mathrm{Res}(f; a_{k}).
		\]
	\end{theorem*}}
\begin{proof*}
	Seja \(m_{k} = n(\gamma ; a_{k}).\) Podemos tomar \(r_{k}>0\) de modo que \(\overline{B(a_{k}, r_{k})}\subseteq G, B(a_{k}, r_{k})\cap\{\gamma \}=\emptyset \) e
	\(B(a_{k}, r_{k})\cap B(a_{j}, r_{j}) = \emptyset \) se \(k\neq j\).

	Considere a curva \(\gamma_{k}(t) = a_{k} + r_{k}e^{-2\pi i m_{k}t},t\in[0, 1].\) Com isso,
	\[
		n(\gamma ; a_{j}) + \sum\limits_{k=1}^{n}n(\gamma_{k}, a_{j}) = 0.
	\]
	Como \(\gamma \sim 0\) e \(\overline{B(a_{k}, r_{k})}\subseteq G\), se \(a\not\in G\setminus{\{a_{1},\dotsc ,a_{n}\}}\),
	\[
		n(\gamma ; a) + \sum\limits_{k=1}^{n}n(\gamma_{k}, a) = 0
	\]
	e segue, do Teorema de Cauchy, que
	\[
		\int_{\gamma }^{}f(z)dz + \sum\limits_{k=1}^{n}\int_{\gamma_{k}}^{}f(z)dz = 0.
	\]
	Caso \(f(z) = \sum\limits_{n=-\infty}^{\infty}b_{n}(z-a_{k})^{n}\) seja a expansão de Laurent de f sobre \(a_{k}\), então a série converge uniformemente
	em \(\overline{B(a_{k}, r_{k})}.\) Logo,
	\begin{align*}
		\int_{\gamma_{k}}^{}f(z)dz & = \sum\limits_{n=-\infty}^{\infty}b_{n}\int_{\gamma_{k}}^{}(z-a_{k})^{n}dz \\
		                           & = b_{-1}\int_{\gamma_{k}}^{}(z-a_{k})^{-1}dz                               \\
		                           & = \mathrm{Res}(f; a_{k})\cdot 2\pi i \cdot n(\gamma_{k}; a_{k})            \\
		                           & = -2\pi i n(\gamma ; a_{k})\cdot \mathrm{Res}(f; a_{k}).
	\end{align*}
	Portanto,
	\[
		\int_{\gamma }^{}f(z)dz + \sum\limits_{k=1}^{n}\int_{\gamma_{k}}^{}f(z)dz = \int_{\gamma }^{}f(z)dz - 2\pi i \sum\limits_{k=1}^{n}\mathrm{Res}(f; a_{k})n(\gamma ; a_{k}) = 0,
	\]
	o que equivale a
	\[
		\frac{1}{2\pi i}\int_{\gamma }^{}f(z)dz = \sum\limits_{k=1}^{n}n(\gamma ; a_{k})\mathrm{Res}(f; a_{k}).\quad \text{\qedsymbol}
	\]
\end{proof*}
\begin{prop*}
	Suponha que \(f:G\rightarrow \mathbb{C}\) tem um polo de ordem \(m \geq 1\) em a e seja \(g(z) = (z-a)^{m}f(z).\) Então,
	\[
		\mathrm{Res}(f; a) = \frac{1}{(m-1)!}g^{(m-1)}(a).
	\]
\end{prop*}
\begin{proof*}
	Sabemos que g(z) é analítico em torno de a e satisfaz \(g(a)\neq 0.\) Considere a expansão em Taylor de g(z) em torno de a:
	\[
		g(z) = \sum\limits_{k=0}^{\infty}b_{k}(z-a)^{k}.
	\]
	Pela definição de g(z), temos
	\[
		f(z) = \frac{g(z)}{(z-a)^{m}} = \frac{b_{0}}{(z-a)^{m}} + \dotsc + \frac{b_{m-1}}{z-a} + \sum\limits_{k=0}^{\infty}b_{m+k}(z-a)^{k}.
	\]
	Portanto,
	\[
		\mathrm{Res}(f; a) = b_{m-1} = \frac{g^{(m-1)(a)}}{(n-1)!}.\quad \text{\qedsymbol}
	\]
\end{proof*}
\begin{exer*}
	Seja \(f(z) = \tan^{}{(z)}.\) Calcule \(\mathrm{Res}(f; \frac{\pi }{2})\) utilizando a proposição anterior.
\end{exer*}
\begin{example}
	Consderando f uma função com um polo simples em a, sabemos que a expansão de f em sua série de Laurent é
	\[
		f(z) = \frac{b_{-1}}{z-a} + \sum\limits_{n=0}^{\infty}b_{n}(z-a).
	\]
	Dessa forma,
	\begin{align*}
		\lim_{z\to a}(z-a)f(z) = \lim_{z\to a}(z-a)\sum\limits_{n=-1}^{\infty}b_{n}(z-a)^{n} & = \lim_{z\to a}\sum\limits_{n=-1}^{\infty}b_{n}(z-a)^{n+1}    \\
		                                                                                     & = \lim_{z\to a}b_{-1} + \sum\limits_{n=0}^{\infty}(z-a)^{n+1} \\
		                                                                                     & = \mathrm{Res}(f; a).
	\end{align*}
	Assim, dado que g é analítica em um aberto contendo a, segue que \(f \cdot g\) tem uma singularidade isolada em a do tipo polo simples
	ou removível \textbf{[Exercício]} e, consequentemente,
	\[
		\mathrm{Res}(fg; a) = \lim_{z\to a}(z-a)f(z)g(z) = \lim_{z\to a}(z-a)f(z)\lim_{z\to a}g(z).
	\]
	Portanto,
	\[
		\mathrm{Res}(fg; a) = \mathrm{Res}(f; a)\cdot g(a).
	\]
\end{example}
\begin{example}
	Seja f uma função analítica na região G com exceção das singularidades isoladas \(a_{1}, a_{2}, \dotsc , a_{n}\), que são polos simples.
	Se \(\gamma \) é uma curva fechada retificável em G tal que \(a_{k}\not\in \{\gamma \}\) com \(k=1,\dotsc ,n \) e \(\gamma \) homotópica a 0,
	segue do \hyperlink{residue}{Teorema do Resíduo} q pelo exemplo anterior que, se g é analítica em G, teremos
	\[
		\frac{1}{2\pi i}\int_{\gamma }^{}f(z)g(z)dz = \sum\limits_{k=1}^{n}n(\gamma ; a_{k})\mathrm{Res}(fg; a_{k}) = \sum\limits_{k=1}^{n}n(\gamma ; a_{k})g(a_{k})\mathrm{Res}(f; a_{k}).
	\]
\end{example}
\end{document}

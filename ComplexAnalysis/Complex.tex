  \documentclass{article}
  \usepackage{amsmath}
  \usepackage{amsthm}
  \usepackage{amssymb}
  \usepackage{pgfplots}
  \usepackage{amsfonts}
  \usepackage[margin=2.5cm]{geometry}
  \usepackage{graphicx}
  \usepackage[export]{adjustbox}
  \usepackage{fancyhdr}
  \usepackage[portuguese]{babel}
  \usepackage{hyperref}
  \usepackage{lastpage}
  \usepackage{mathtools}

  \pagestyle{fancy}
  \fancyhf{}

  \pgfplotsset{compat = 1.18}

  \hypersetup{
      colorlinks,
      citecolor=black,
      filecolor=black,
      linkcolor=black,
      urlcolor=black
  }
  \newtheorem*{def*}{\underline{Defini\c c\~ao}}
  \newtheorem*{theorem*}{\underline{Teorema}}
  \newtheorem{example}{\underline{Exemplo}}[section]
  \newtheorem*{proof*}{\underline{Prova}}
  \newtheorem*{prop*}{\underline{Proposi\c c\~ao}}
  \newtheorem*{crl*}{\underline{Corol\'ario}}
  \newtheorem*{lmm*}{\underline{Lema}}
  \newtheorem*{exer*}{\underline{Exerc\'icios}}
  \renewcommand\qedsymbol{$\blacksquare$}

  \rfoot{P\'agina \thepage \hspace{1pt} de \pageref{LastPage}}

  \title{Fun\c c\~oes de Vari\'aveis Complexas}
  \vspace{1.5cm}
  \author{Prof. Tha\'is Jord\~ao\\
  \vspace{2cm}\\
  Notas por:\\
  Lucas Giraldi Almeida Coimbra\\
  Renan Wenzel\\
  \vspace{4cm}\\
  \href{https://github.com/RenanLeznew/USP-Math-LaTeX/tree/master/ComplexAnalysis}{GitHub com o Arquivo das Notas (Link clic\'avel)}\\
  \url{https://github.com/RenanLeznew/USP-Math-LaTeX/tree/master/ComplexAnalysis}\\
  \vspace{4cm}\\
}
  \date{\today}

  \begin{document}
  \maketitle
  \newpage
  \tableofcontents
  \newpage

  \section{Aula 01 - 03/01/2023}
  \subsection{Motiva\c c\~oes}
  \begin{itemize}
   \item Definir o corpos dos complexos
   \item Definir a topologia no corpo dos complexos
   \item Esfera de Riemann
  \end{itemize}

  \subsection{Defini\c c\~oes B\'asicas}
  \begin{def*}
    Um corpo f \'e um conjunto n\~ao vazio em que definem-se duas opera\c c\~oes $+:F\times{F}\rightarrow F, \cdot:F\times{F}\rightarrow F$ satisfazendo:
  \begin{itemize}
    \item[i)] w + z = z + w
    \item[ii)] w + (z+ u) = (w + z) + u
    \item[iii)] Existe 0 em F tal que w + 0 = w
    \item[iv)] Para cada $w\in F$, existe $-w \in F$ tal que w + (-w) = 0
    \item[v)] $w\cdot z = z\cdot w$
    \item[vi)] $w\cdot(z\cdot u) = (w\cdot z)\cdot u$
    \item[vii)] Existe $e\in \mathbb{F}$ tal que $w\cdot{e} = w$
    \item[viii)] Para cada $w\in{F-\{0\}}$, existe $w ^{-1}\in{F}$ tal que $w\cdot w ^{-1} = e$
    \item[ix)] $(w+z)\cdot{u} = w\cdot u + z\cdot u,$
  \end{itemize}
  em que w, z, u pertencem a F.
  \end{def*}
  Considere F um corpo contendo $\mathbb{R}$ e tal que 
  $$
  x ^{2} + 1 = 0
  $$
tenha solu\c c\~ao. Seja i esta solu\c c\~ao. Segue que -i \'e solu\c c\~ao dela tamb\'em, -1.z = z e 0.z = 0 para z em F. Definimos
  $$
  \mathbb{C}:= \{a + bi: a, b\in \mathbb{R}\,
  $$
de maneira que os elementos de $\mathbb{C}$ s\~ao unicmaente determinados, $\mathbb{C}$ \'e subcorpo de F e a estrutura 
alg\'ebrica de $\mathbb{C}$ n\~ao depende de F. Al\'em disso, este corpo existe.

  Com efeito, 
  \subsubsection{Unicidade}
  Sejam a, b, c, d $\in \mathbb{R}$ tais que 
  $$
  a + bi = c + di.
  $$
  Assim, $a-c = i(d-b)\Rightarrow (a-c)^{2} = (d-b)^{2}$, donde segue a unicidade a = c e d = b
  \subsubsection{Subcorpo}
  Exerc\'icio.
  \subsubsection{Estrutura Alg\'ebrica Independe de F}
  Seja F\ outro corpo contendo $\mathbb{R}$ em que $x ^{2} + 1 = 0$ possui solu\c c\~ao. Considere $\mathbb{C}' = \mathbb{R} + j \mathbb{R},$
  em que j \'e a solu\c c\~ao da equa\c c\~ao em F'. Definimos $T:\mathbb{C}\rightarrow \mathbb{C}'$ por 
  $$
  T(a + bi) = a + bj
  $$
e, neste caso, T(z + w) = T(z) + T(w), T(zw) = T(z)T(w) para todos $z, w\in \mathbb{C}.$ (Exerc\'icio.)
  \subsubsection{Exist\^encia}
  Seja $F = \{(a, b):a, b\in \mathbb{R}\}$ munido das opera\c c\~oes $+:F\times{F}\rightarrow F, \cdot:F\times{F}\rightarrow F$ dadas por
 \begin{align*}
  +((a, b), (c,d)) = (a + c, b + d) \\
  \cdot((a, b), (c, d)) = (ac - bd, ad + bc).
 \end{align*}
  Note que $(0, 1)^{2} = (-1, 0)$. Assim, (F, +, .) \'e um corpo contendo $\mathbb{R}.$ (Exerc\'icio).

  Algumas propriedades(Exerc\'icios):
 \begin{itemize}
   \item[a)] $Re(z)\leq{|z|}$ e $Im(z)\leq{|z|}$
   \item[b)] $\overline{z+w} = \overline{z}+\overline{w}$ e $\overline{z\cdot w} = \overline{z}\cdot\overline{w}$
   \item[c)] $\overline{\frac{1}{z}} = \frac{1}{\overline{z}}$
   \item[d)] $|z| = |\overline{z}|$ e $|z|^{2} = z\cdot\overline{z}$
   \item[e)] $z + \overline{z} = 2Re(z), z - \overline{z} = 2iIm(z)$ e $\frac{1}{z} = \frac{\overline{z}}{|z|^{2}}$
 \end{itemize}
  \subsection{Representa\c c\~ao Polar de $\mathbb{C}$}
  Dado $z\in \mathbb{C}$, temos 
  $$
  z = |z|(\cos{\theta} + i\sin{\theta}), \quad \theta = \arg z.
  $$
  Neste caso, temos, para z n\~ao-nulo,
  $$
  z ^{-1} = |z|^{-1}(\cos{-\theta} + i\sin{-\theta}) = |z|^{-1}(\cos{\theta}-i\sin{-\theta})
  $$
  Para $z _{1}, z _{2}\in \mathbb{C},$ temos
  $$
  z _{1}\cdot z _{2} = |z _{1}||z _{2}|(\cos(\theta _{1} + \theta _{2}) + i\sin(\theta _{1} + \theta _{2}))
  $$
com $\theta _{1} = \arg z _{1}, \theta _{2} = z _{2}.$ Mais geralmente, 
  $$
  \prod_{k=1}^{n}z _{k} = \prod_{k=1}^{n} |z _{k}|(\cos(\sum_{k=1}^{n}\theta _{k}) + i\sin(\sum_{k=1}^{n}\theta _{k})),
  $$
com $\theta _{k} = \arg z _{k}$. Em particular, 
  $$
  z ^{n} = |z|^{n}(\cos(n\theta)+i\sin(n \theta)), \quad n\in \mathbb{N}.
  $$
  Buscando w tal que $w ^{n} = z$ para dado z n\~ao-nulo,
  $$
  w = |z|^{\frac{1}{n}}(\cos(\frac{\theta + 2k\pi}{n}) + i\sin(\frac{\theta + 2k\pi}{n})), \quad k = 0, 1, \cdots, n.
  $$

  \subsection{A Esfera de Riemann}
  Considere $\mathbb{S}^{2}\subset{\mathbb{R}^{3}}$ a esfera 
  $$
  \mathbb{S}^{2}:= \{(x, y, z): x ^{2} + y ^{2} + z ^{2}\leq{1}\}.
  $$
  Chame N=\{0, 0, 1\} de polo norte. Fazemos uma associa\c c\~ao entre $\mathbb{S}^{2}-\{N\}$ e o plano z=0 de $\mathbb{R}^{3}$,
chamada de proje\c c\~ao estereogr\'afica. Nessa associa\c c\~ao, o ponto $z(=x + iy)\in\mathbb{C}$ \'e associado a (x, y, 0), e
definimos uma reta por N e z como r: N + t(x, y, -1), $t\in \mathbb{R}$. Assim,
  $$
  r\cap{\mathbb{S}^{2}} \Rightarrow S _{z} = \biggl(\frac{2x}{|z|^{2}+1}, \frac{2y}{|z|^{2}+1}, \frac{|z|^{2}-1}{|z|^{2}+1}\biggr)\in \mathbb{S}^{2}
  $$
  Reciprocramente, o ponto (x, y, z) de $\mathbb{S}^{2}$ pode ser associado ao considerar a reta r: N + t(x, y, s-1), em que
s \'e um n\'umero real. Com isso, a intersec\c c\~ao $r\cap \{(x, y, 0): x, y \in \mathbb{R}\}\Rightarrow t=\frac{1}{1-s}$ mostra que 
$z = \biggl(\frac{x}{1-s}, \frac{y}{1-s}, 0\biggr)$ corresponde ao ponto z de $\mathbb{C}$.
  
  Associando N ao infinito, obtemos o plano estendido $\mathbb{C}_{\infty} = \mathbb{C}\cup \{\infty\}$, chamado de Esfera de Riemann. Se $\phi:\mathbb{C}_{\infty}\rightarrow \mathbb{S}^{2}$
  \'e dada por $\phi(\infty) = N$ e, para $z\neq{\infty},$
    $$
    \phi(z) = (\frac{z + \overline{z}}{|z|^{2} + 1}, \frac{z - \overline{z}}{|z|^{2}+1}, \frac{|z|^{2}-1}{|z|^{2}+1}),
    $$
  ent\~ao dados $z, w\in \mathbb{C}_{\infty},$ definimos a m\'etrica
    $$
    d(z, w)=\left\{
      \begin{array}{ll}
        ||\phi(z) - \phi(w)||, & \quad z, w\neq{\infty} \\
        0, & \quad z = w = \infty \\
        \infty, & \quad \text{ caso contr\'ario}
      \end{array}\right.
    $$
   \begin{example}
     Se $z, w\neq{\infty}$, ent\~ao
     $$
     d(z, w) = d(\phi(z), \phi(w)) = \frac{2|z - w|}{[(1+|z|^{2})(1+|w|^{2})]^{\frac{1}{2}}}
     $$
    e, se $z\neq\infty$,
     $$
     d(z, \infty) = ||\phi(z) - N|| = \frac{2}{(1+|z|^{2})^{\frac{1}{2}}}
     $$
   \end{example}

   \subsection{Topologia de $\mathbb{C}$}
  \begin{def*}
    Sejam X um conjunto e $d:X\times{X}\rightarrow X$ uma fun\c c\~ao. Dizemos que d \'e uma m\'etrica se
   \begin{itemize}
     \item[i)] $d(x, y) \geq{0}, d(x, y) = 0\Longleftrightarrow x=y$
     \item[ii)] $d(x, y) = d(y, x)$
     \item[iii)] $d(x, z) \leq d(x, y) + d(y, z)$,
   \end{itemize}
   em que x, y, z pertencem a X. Neste caso, chamamos a terna (X, d) de espa\c co m\'etrico.
  \end{def*}
   Considere (X, d) um espa\c co m\'etrico. Dado x em X e $r > 0$,
   $$
   B(x, r):= \{y\in{X}: d(x, y)< r\}
   $$
   \'e a bola aberta, seu fecho \'e
   $$
   B_c(x, r):= \{y\in{X}: d(x, y) = r\}
   $$
   e a bola fechada \'e a uni\~ao deles, ou seja, 
   $$
   \overline{B(x, r)}:= \{y\in{X}: d(x, y) \leq{r}\}
   $$
  \begin{example}
    Considere X n\~ao-nulo e d(x, y) = $\delta _{x,y}.$ (X, d) \'e metrico e 
    $$
    B\biggl(x, \frac{1}{2}\biggr) = \{x\} = \overline{B\biggl(x, \frac{1}{2}\biggr)} = B\biggl(x, \frac{1}{333}\biggr), x\in{X}
    $$
    $$
    B(x, 2) = X = B(x, 1001), x\in{X}
    $$
  \end{example}
  Utilizando bolas, definimos que um conjunto $A\subset{X}$ \'e aberto se para todo x em A, existe $r > 0$ tal que 
B(x, r) est\'a contido em A. Por outro lado, um conjunto \'e fechado se seu complemenetar \'e aberto. A uni\~ao infinita
de abertos \'e aberta e, pelas Leis de DeMorgan, a intersec\c c\~ao infinita de fechados \'e fechada. Al\'em disso, intersec\c c\~oes
finitas de abertos \'e aberta e uni\~ao finita de fechados \'e fechado.

  Definimos, tamb\'em, o interior de A como $A^{\circ}:= \cup_{B} \{B\subset{A}: B \text{ aberto}\}$, o fecho de A
  como $\overline{A} = \cap_{F} \{A\subseteq{F}: F \text{ fechado}\}$ e o bordo de A como $\partial{A} = \overline{A}\cap\overline{A^c}.$
  Diremos que A \'e denso quando $\overline{A} = X.$
\begin{prop*}
  Seja (X, d) um espa\c co m\'etrico e A um subconjunto. Ent\~ao,
 \begin{itemize}
   \item[i)] A \'e aberto se, e s\'o se, $A = A^{\circ}$
   \item[ii)] A \'e fechado se, e s\'o se, $A = \overline{A}$
   \item[iii)] Se x pertence a $A^{\circ}$, ent\~ao existe $\epsilon>0$ tal que $B(x, \epsilon)\subseteq{A}$.
   \item[iv)] Se x pertence a $\overline{A}$, ent\~ao para todo $\epsilon > 0$ tal que $B(x, \epsilon)\cap{A}\neq\emptyset$.
 \end{itemize}
\end{prop*}

  Um espa\c co m\'etrico (X, d) \'e conexo se os \'unicos subconjuntos abertos e fechados de X s\~ao X e vazio. Caso contr\'ario,
X \'e dito ser desconexo, ou seja, existem abertos disjunto n\~ao-vazios cuja uni\~ao d\'a o espa\c co todo. Um exerc\'icio \'e mostrar
mostrar que um conjunto \'e conexo se, e s\'o se, ele \'e um intervalo.

  Dados z, w em $\mathbb{C}$, o segmento [z, w] \'e o conjunto
  $$
  [z, w]:= \{tw + (1-t)z: t\in[0, 1]\}
  $$
  Al\'em disso, dados $z _{1}, \cdots, z _{n}$, a poligonal com esses v\'ertices \'e
  $$
  [z _{1}, \cdots, z _{n}] = \bigcup _{k=1}^{n-1}[z _{k}, z _{k+1}]
  $$
 \begin{prop*}
   Seja G um subconjunto de $\mathbb{C}$ aberto. Ent\~ao, G \'e conexo se, e s\'o se para todo z, w em G, existe uma poligonal
  $[z, z _{1}, \cdots, z _{n}, w]\subseteq{G}.$
 \end{prop*}
\begin{proof*}
  $\Leftarrow)$ Assumindo que G satisfaz a propriedade da poligonal, suponha tamb\'em que G n\~ao \'e conexo. Assim,
podemos escrever $G = B\cup{C}$ com $B\cap{C}=\emptyset$ e B, C n\~ao-vazios. Pela propriedade de G, existe 
  $[b, z _{1}, \cdots, z _{n}, c]\subseteq{G}$. Neste caso, existe k tal que $z _{k}\in{B}$ e $z _{k+1}\in{C}.$ Agora,
  considere os conjuntos 
 \begin{align*}
   &B' = \{t\in[0, 1]: tz _{k} + (1 - t)z _{k+1}\in{B}\}\\
   &C' = \{t\in[0, 1]: tz _{k} + (1 - t)z _{k+1}\in{C}\}
 \end{align*}
e note que $B'\neq{\emptyset}$ pois $z _{k}\in{B}$ e $1\in{B'}$. Analogamente, C' \'e n\~ao-vazio. No entanto, isso \'e um absurdo,
pois [0, 1] seria conexo e $B'\cup{C'}$ seria uma cis\~ao n\~ao trivial

 $\Rightarrow)$ Suponha, agora, que G \'e conexo e seja z um elemento dele. Defina 
 $$
 C = \{w\in{G}: \text{Existe } [z, z _{1}, \cdots, z _{n}, w]\subseteq{G}\}
 $$
 Observe que C \'e n\~ao-vazio, z pertence a G e [z] \'e subconjunto de G. Mostremos que C \'e aberto e fechado (pois implicar\'a em C = G).
Com efeito, se $w\in{C}\subseteq{G}$, existe $r>0$ tal que B(w, r) est\'a contigo em G, pois G \'e aberto.
 Assim, para todo $s\in B(w, r)$, temos $[s, w]\subseteq{B(w, r)}$ e, com isso, existe uma poligonal ligando s a z com $s\in{C},$ mostrando 
que C \'e aberto.

  Mostrar que o complementar de C \'e aberto \'e an\'alogo. Com efeito, se $C ^{c} = \emptyset,$ o resultado est\'a provado. Por outro
lado, se $C ^{c}\neq\emptyset$, seja $w\in{C ^{c}}$ = G - C. Logo, existe $r > 0$ tal que $B(w, r)\subseteq{G}$. Afirmamos
que $B(w, r)\subseteq{G-C}$. Caso contr\'ario, existe s em B(w, r) contido, tamb\'em, em C. Neste caso, existe uma poligonal
ligando s a z e s a w, uma contradi\c c\~ao, pois isso conectaria w a z, mesmo com w no complementar de z. Portanto, o complementar
\'e aberto e C \'e aberto e fechado. \qedsymbol
\end{proof*}
\newpage

\section{Aula 02 - 05/01/2023}
\subsection{Motiva\c c\~oes}
\begin{itemize}
  \item Sequ\^encias e suas converg\^encias;
  \item Teorema de Cantor para espa\c cos completos;
  \item Compacidade e Heine-Borel;
  \item Continuidade e converg\^encia de fun\c c\~oes.
\end{itemize}

\subsection{Fim de Conexos}
\begin{theorem*}
  Seja $G\subseteq{\mathbb{C}}$ um aberto e conexo, ent\~ao existe uma poligonal ligando qualquer z, w em G cujos segmentos
  sejam paralelos ao eixo real ou imagin\'ario.
\end{theorem*}

\begin{def*}
  Um subconjunto de um espa\c co m\'etrico (M, d) \'e uma componente conexa se \'e um conexo maximal
\end{def*}

\begin{example}
Coloque $A = \{1, 2, 3\}. \{1\}$ \'e componente conexa de A, mas $\{1, 2\}$ n\~ao \'e. 
 \end{example}

\begin{theorem*}
  Seja (M, d) um espa\c co m\'etrico. Ent\~ao,
 \begin{itemize}
   \item[1)] Para x em M, existe $C _{x}$ uma componente conexa de M com $x\text{ em }C _{x}; $
   \item[2)] As componentes s\~ao disjuntas.
 \end{itemize}
\end{theorem*}
\begin{proof*}
  1) \par Seja x em M e tomemos
  $$
  C _{x} = \bigcup _{D\subseteq{M}} \{D: D \text{ conexos com } x\in{D}\}
  $$
  Mostremos que $C _{x}$ \'e conexo, pois a maximalidade segue da defini\c c\~ao dada a ele. Note que $C _{x}\neq\emptyset$, visto que
qualquer conjunto unit\'ario \'e conexo. Seja $A\subseteq{C _{x}}$ aberto, fechado e n\~ao-nulo. Existe $D _{x}\in C _{x}$ tal
que $D _{x}\cap{A}\neq\emptyset$, o que implica que $D _{x}\subseteq{A}.$

Finalmente, considere $D\in C _{x}$, de modo que $D _{x}\cup{D}$ \'e conexo e $(D _{x}\cup{D})\cap{A}\neq\emptyset$ o que garante
que $D\subseteq{A}. \text{ Assim, } A = C _{x}.$
\qedsymbol.
\end{proof*}
\begin{exer*}
 \begin{itemize}
  \item[1)] Prove a segunda afirma\c c\~ao do teorema;
  \item[2)] Se D e conexo e $D\subseteq{A}\subseteq{\overline{D}}$, ent\~ao A \'e conexo.
 \end{itemize}
\end{exer*}

\begin{theorem*}
  Seja G um subconjunto aberto de $\mathbb{C}.$ As componentes conexas s\~ao abertas e h\'a no m\'aximo uma quantidade enumer\'avel
delas.
\end{theorem*}
\begin{proof*}
  Seja D uma componente conexa de G. Tome $x\in{D}$, tal que existe $r > 0 \text{ com } B(x, r)\subseteq{G}$, j\'a que G \'e aberto.
Suponha que $B(x, r)\not\subseteq{D}.$ Neste caso, $B(x, r)\cup{D}$ seria um conexo contendo D propriamente. Logo, 
$B(x, r)\subseteq{D}$ e D \'e aberto. 

  Para a segunda afirma\c c\~ao, considere 
  $$
  \Omega = \mathbb{Q} + i\mathbb{Q} (\overline{\Omega} = \mathbb{C})
  $$  
  Para cada componente conexa C de G, como G \'e aberto, existe $z\in{\Omega\cap{C}}$, o que \'e suficiente para garantir a enumerabilidade
das componentes de G.
\qedsymbol.
\end{proof*}

\subsection{Sequ\^encias e Completude}
\begin{def*}
  Seja (M, d) um espa\c co m\'etrico. Uma sequ\^encia $\{x _{n}\}$ de M \'e convergente se existe x em M tal que 
para todo $\epsilon > 0$, existe $n_{0}$ natural tal que 
  $$
  d(x _{n}, x) < \epsilon, \quad n\geq n_{0}.
  $$
Escrevemos, neste caso, $x _{n}\to x$. Dizemos que uma sequ\^encia \'e de Cauchy se para todo $\epsilon > 0$, existe
 $n_{0}$ natural satisfazendo 
  $$
  d(x _{n}, x _{m}) < \epsilon, \quad n, m \geq n_{0}.
  $$
\end{def*}

 \begin{exer*}
  \begin{itemize}
    \item[i)] Se $\{x _{n}\}$ \'e convergente, ent\~ao $\{x_n\}$ \'e de Cauchy, mas a rec\'iproca \'e s\'o v\'alida
  quando a sequ\^encia possui uma subsequ\^encia convergente.
    \item[ii)] Se $\{x_{n}\}$ \'e de Cauchy, ent\~ao $x_{n}$ \'e limitada.
    \item[iii)] $F\subseteq{M}$ \'e fechado se e s\'o se toda $x_{n}$ de F com $x_{n}\to x$ \'e tal que x pertence a F.
  \end{itemize}
 \end{exer*}

  Dizemos que um espa\c co m\'etrico \'e completo se toda sequ\^encia de Cauchy for convergente.  
 \begin{exer*}
  \begin{itemize}
    \item[i)] Mostre que $\mathbb{R}, \mathbb{C}$ s\~ao espa\c cos m\'etricos completos;
    \item[ii)] Se (M,d) \'e um espa\c co m\'etrico e $S\subseteq{M}$, mostre que se (S,d) for completo, ele \'e fechado em M. 
Mostre e rec\'iproca no caso em que (M, d) \'e completo.
  \end{itemize}
 \end{exer*}
O resultado a seguir \'e conhecido como Teorema de Cantor.
\begin{theorem*}
  Um espa\c co m\'etrico \'e completo se e s\'o se toda cadeia descendente de fechado $\{F_{n}\} $ satisfazendo
  $$
  diam F_{n}\to{0}, \quad n\to\infty
  $$
\'e tal que $\bigcap_{n\in \mathbb{N}}F_{n}$ \'e unit\'ario. Aqui, $diam A:=sup\{d(x, y): x, y\in{A}\}.$
\end{theorem*}
\begin{proof*}
  Suponha que M \'e um espa\c co m\'etrico completo. Se $\bigcap\limits _{n\in \mathbb{N}}F\neq\emptyset,$ ent\~ao ele \'e unit\'ario. 
De fato, se $x, y\in{\cap_n{F}}$, 
  $$
  d(x, y)\leq diam F _{n} (diam F _{n+1}\leq diam F _{n}),
  $$
mas $diam F _{n}\to{0}$ e d(x, y) = 0, de modo que x = y.

  Agora, seja $x _{n}\in F _{n}, n\in \mathbb{N}$ e observe que 
  $$
  d(x _{n}, x _{n+1})\leq diam F _{n},
  $$
pois $F _{n+1}\subseteq{F _{n}}$. Isto garante que $\{x_{n}\}$ \'e de Cauchy e, como M \'e completo, existe x com $x_{n}\to{x}$.
Neste caso, $x\in{F_{n}}$ para todo n e $\bigcap _{n\in \mathbb{N}}F_{n}=\{x\}.$
  
  Reciprocramente, seja $\{a_{n}\}$ de Cauchy em M. Constru\'imos 
  $$
  F_{n} = \overline{\{a_{k}: k\geq{n}\}}
  $$
que s\~ao fechados satisfazendo $F_{n+1}\subseteq{F_{n}}.$ Assim, $\bigcap\limits _{n\in \mathbb{N}}F_{n} = \{x\}$ para algum
x de M. Como
  $$
  d(x, a_{n})\leq diamF_{n}\to{0},
  $$
temos, portanto, $a_{n}\to{x}.$
\qedsymbol
\end{proof*}
Um exerc\'icio que fica \'e mostrar que se $\{a_{n}\}$ \'e de Cauchy, ent\~ao $diamF_{n}\to{0}$
\subsection{Compactos}
\begin{def*}
Seja (M, d) um espa\c co m\'etrico. Um subconjunto $S\subseteq{M}$ \'e compacto se para toda cole\c c\~ao $\mathcal{A}$
de abertos de M cobrindo S existe $A_1, \cdots, A_{n}\in \mathcal{A}$ tal que 
  $$
  S\subseteq\bigcup_{k=1}^{n}A_{k}
  $$
\end{def*}
  Dado um espa\c co m\'etrico (M, d), M \'e dito sequencialmente completo se todas as sequ\^encias de M possuem subse
qu\^encia convergente. Tamb\'em diremos que ele \'e totalmente limitado se para todo $\epsilon > 0$, existe $n\in \mathbb{N},
x_{1}, \cdots, x_{n}\in{M}$ com 
  $$
  M = \bigcup_{i=1}^{n}B(x_{i}, \epsilon).
  $$
Um conjunto A \'e dito limitado se seu diametro \'e finito.
 \begin{exer*}
 \begin{itemize}
   \item[i)] Se A \'e totalmente limitado, ent\~ao A \'e limitado, mas a rec\'iproca n\~ao \'e necessariamente verdade.
    \item[ii)] Se A \'e compacto, ent\~ao A \'e limitado, mas a rec\'iproca n\~ao \'e necessariamente verdade.
 \end{itemize} 
 \end{exer*}
\begin{prop*}
  Seja (M, d) um espa\c co m\'etrico e K um subconjutno de M. Ent\~ao, K \'e compacto se es\'o se toda fam\'ilia de fechados com PIF tem 
interse\c c\~ao n\~ao-vazia.
\end{prop*}
A PIF \'e a Propriedade da Intersec\c c\~ao Finita, que afirma que dados conjuntos $F _{1}, \cdots, F_{n}\Rightarrow \bigcap\limits_{k=1}^{n}F_{k}\neq\emptyset$
\begin{theorem*}
  Seja (M, d) um espa\c co m\'etrico. As seugintes afirma\c c\~oes s\~ao equivalentes:
 \begin{itemize}
   \item[i)]M \'e compacto;
   \item[ii)] Para todo conjunto ininito S de M, existe x em S tal que para todo $\epsilon > 0, B(x, \epsilon)\cap{S-\{x\}}\neq\emptyset$;
   \item[iii)] M \'e sequencialmente compacto;
   \item[iv)] M \'e completo e totalmente limitado.
 \end{itemize}
\end{theorem*}
\begin{theorem*}
  Um conjunto K de $\mathbb{R}^{n}$ \'e compacto se e s\'o se ele \'e fechado e limitado.
\end{theorem*}
Segue um esbo\c co da prova.
\begin{proof*}
  Se K \'e compacto, ele \'e completo (logo, fechado) e totalmente limitado (logo, limitado). Por outro lado, se K \'e fechado
e limitado, ent\~ao K \'e completo porque $\mathbb{R}^{n}$ \'e completo. Al\'em disso, pela propriedade Arquimediana da reta,
para todo $\epsilon > 0$, existem $x_1, \cdots, n_{n}\in{K}$ com 
  $$
  K\subseteq{\bigcup_{i=1}^{n}B(x_{i}, \epsilon)}
  $$
\end{proof*}

\subsection{Continuidade}
\begin{def*}
  Sejam (X, d), (Y, d') espa\c cos m\'etricos. $f:X\rightarrow Y$ \'e cont\'inua em x de X se para todo $\epsilon > 0$, existir
  $\delta > 0$ tal que 
  $$
  d(x, y) < \delta\Rightarrow d'(f(x), f(y)) < \epsilon 
  $$
f \'e dita cont\'inua se isso ocorre para todos os pontos de M.
\end{def*}
\begin{exer*}
  Mostre que equivalem \`a defini\c c\~ao de cont\'inua:
 \begin{itemize}
   \item[i)] $f^{-1}(B(x, \epsilon))$ cont\'em uma bola aberta centrada em x, para todo $\epsilon > 0$;
   \item[ii)] $x_{n}\to{x}$ implica $f(x_{n})\to{f(x)}$
   \item[iii)] $F ^{-1}(A)$ \'e aberta em $X$ para todo aberto A com $x\in{A}$
 \end{itemize}
\end{exer*}
\begin{prop*}
  Sejam $f, g:X\rightarrow \mathbb{C}$ fun\c c\~oes cont\'inuas. Ent\~ao,
 \begin{itemize}
   \item[1)] $\alpha f + \beta g$ \'e cont\'inua, $\alpha, \beta\in \mathbb{C};$
   \item[2)] fg \'e con\'inua;
   \item[3)] Se $x\neq{0},$ ent\~ao f/g \'e cont\'inua em x;
   \item[4)] Se $h:Y\rightarrow X$ \'e con\'tinua, ent\~ao $f\circ{h}:Y\rightarrow \mathbb{C}$ \'e cont\'inua.
 \end{itemize}
\end{prop*}
\begin{def*}
  Uma fun\c c\~ao $f:(X, d)\rightarrow (Y, d')$ \'e uniformemente cont\'inua se para todo $\epsilon > 0$, existe $\delta > 0$
tal que 
 $$
 d(x, y) < \delta\Rightarrow d'(f(x), f(y)) < \epsilon.
 $$
 Uma fun\c c\~ao $f:(X, d)\rightarrow (Y, d')$ \'e Lipschitz se existe $c > 0$ tal que 
 $$
 d'(f(x), f(y)) \leq cd(x, y)
 $$
\end{def*} 
\begin{theorem*}
  Seja $f:(X, d)\rightarrow (Y, d')$ uma fun\c c\~ao. Ent\~ao, 
 \begin{itemize}
   \item[i)] Se X \'e compacto, ent\~ao f(X) \'e compacto;
     \item[ii)] Se X \'e conexo, ent\~ao f(X) \'e conexo. Adicionalmente, se Y = $\mathbb{R}$, ent\~ao f(X) \'e um intervalo.
 \end{itemize}
\end{theorem*}
\begin{crl*}
  Se $f:X\rightarrow \mathbb{R}$ \'e cont\'inua, ent\~ao para todo $K \subseteq{X}$ compacto, existem $x _{m}, x _{M}\in{K}$
tais que 
  $$
  f(x _{m}) = \inf _{x\in{K}} \{f(x)\}, \quad f(x _{M}) = \sup _{x\in{K}} \{f(x)\}
  $$
\end{crl*}
\begin{crl*}
  Nas mesmas condi\c c\~oes, mas f uma fun\c c\~ao complexa, temos 
  $$
  |f(x _{m})| = \inf _{x\in{K}} \{|f(x)|\}, \quad |f(x _{M})| = \sup _{x\in{K}} \{|f(x)|\}
  $$
\end{crl*}
\begin{theorem*}
  Seja $f:X\rightarrow Y$ con\'tinua. Se X \'e compacto, ent\~ao f \'e uniformemente cont\'inua.
\end{theorem*}

\subsection{Converg\^encia Uniforme}
\begin{def*}
  Uma sequ\^encia de fun\c c\~oes $\{f_{n}\}$ de X em Y converge pontualmente para $f:X\rightarrow Y$ se 
  $$
  f_{n}(x)\to f(x), \quad n\to\infty, \forall{x\in{X}}
  $$
  $\{f_{n}\}$ converge uniformemente para f se para todo $\epsilon > 0$, existe $n_{0}\in \mathbb{N}$ tal que
  $$
  \sup _{x\in{X}} \{d'(f_{n}(x), f(x))\} < \epsilon, n\geq{n_{0}}
  $$  
\end{def*}
\begin{theorem*}
  Se $\{f_{n}\}$ \'e uma sequ\^encia de fun\c c\~oes con\'tinuas e $f_{n}\to{f}$ uniformemente, ent\~ao f \'e cont\'inua.
\end{theorem*}
\begin{theorem*}
  Seja $u_{n}:X\rightarrow \mathbb{C}$ uma sequ\^encia de fun\c c\~oes satisfazendo
  $$
  |u_{n}(x)|\leq c_{n}, n\in \mathbb{N}.
  $$
  Se $\sum\limits_{n=0}^{\infty}c_{n} < \infty,$ ent\~ao $\sum\limits_{k=1}^{n}u_{k}\to \sum\limits_{n=0}^{\infty}u_{n}$ uniformemente.
\end{theorem*}
\newpage

\section{Aula 03 - 06/01/2023}
\subsection{Motiva\c c\~oes}
\begin{itemize}
  \item[i)] Introdu\c c\~ao \`as s\'eries de pot\^encia e raio de converg\^encia;
  \item[ii)] Fun\c c\~oes anal\'iticas e diferenci\'aveis em $\mathbb{C}$;
  \item[iii)] Defini\c c\~ao da exponencial complexa;
  \item[iv)] Ramos de fun\c c\~oes inversas.
\end{itemize}
\subsection{S\'eries de Pot\^encias}
\begin{def*}
  Considere $\{a_{n}\}$ uma sequ\^encia em $\mathbb{C}$. A s\'erie de pot\^encia em $\{a_n\}$, denotada
por $\sum\limits_{n=0}^{\infty}$, \'e dita convergente se para todo $\epsilon > 0$, existe $n_0\in\mathbb{N}$
tal que $|\sum\limits_{n=0}^{k} - a|, k\geq{n_0}$, para algum $a\in\mathbb{C}$. Denotamos isso por 
  $$
  a = \sum_{n=0}^{\infty} a_n < \infty,
  $$
A s\'erie $\sum\limits_{n=0}^{\infty}a_n$ \'e absolutamente convergente se $\sum\limits_{n=0}^{\infty}|a_n|<\infty$.
\end{def*}
\begin{exer*}
  Mostre que se uma soma converge absolutamente, ela tamb\'em converge normalmente.
\end{exer*}
\begin{def*}
  Uma s\'erie de pot\^encias \'e uma s\'erie da forma
  $$
  \sum_{n=0}^{\infty}a_n(z-a)^n, \quad z\in\mathbb{C},
  $$
em que $\{a_n\}$ \'e uma sequ\^encia de $\mathbb{C}$ e a \'e um n\'umero complexo.
\end{def*}
\begin{example}
  No caso da s\'erie geom\'etrica $\sum\limits_{n=0}^{\infty}z^n, z\in\mathbb{C}$, considere
a soma parcial $s_n = \sum\limits_{k=0}^{n} = \frac{1 - z^{n+1}}{1-z}, z\neq{1}.$ Se
$|z| < 1,$ ent\~ao $z^{n+1}\to{0}$ e $\sum\limits_{n=0}^{\infty}z^n = \frac{1}{1-z}, |z| < 1.$
Caso $|z|\geq{1},$ a s\'erie geom\'etrica diverge.
\end{example}
Denotamos por $\limsup_{n\to\infty}\{b_n\}$ a express\~ao $\lim_{n\to\infty}\sup_{k\geq{n}}\{b_k\}$.
\begin{theorem*}
  Considere a s\'erie de pot\^encias $\sum\limits_{n=0}^{\infty}(z-a)^n$ e $\frac{1}{R}:=\limsup_{n\to\infty}\{\sqrt[n]{|a_n|}\}$.
Ent\~ao, 
\begin{itemize}
  \item[1)] A s\'erie converge absolutamente em B(a, R)
  \item[2)] A s\'erie diverge se $|z-a| > R$
  \item[3)] A s\'erie converge uniformemente em B(a, r) para $0 < r < R.$
\end{itemize}
\end{theorem*}
\begin{proof*}
  Sem perda de generalidade, suponha a = 0.
  1.) Seja $z\in{B(0, R)}.$ Existe $|z| < r < R$ e $n_0\in\mathbb{N}$ tal que $|a_n^{\frac{1}{n}}| < \frac{1}{r},
n\geq{n_0}.$ Da\'i, temos 
  $$
  \sum_{k=n_0}^{\infty}|a_n||z^n|\leq \sum_{k=n_0}^{\infty}\frac{|z^n|}{r^n} < \infty.
  $$
Como essa fra\c c\~ao \'e menor que um, o resultado est\'a provado.

  2.) Seja $|z| > R$ e r tal que $|z|> r > R$. Existe $\{a_{n_k}\}_k$ tal que $|a_{n_k}|^{\frac{1}{n_k}} > \frac{1}{r},
k = 0, 1, \cdots.$ Assim, temos
  $$
    |a_{n_k}||z|^{n_k} > \biggl(\frac{|z|}{r}\biggr)^{n_k}\to\infty
  $$
Conforme k tende a infinito.

  3.) Seja $0 < r < R \text{ e } r < \rho < R.$ Se z pertence a uma bola B(0, r), ent\~ao
  $$
    |a_n||z|^n < \biggl(\frac{r}{\rho}\biggr)^n, \quad n\geq{n_0}, n_0\in\mathbb{N}.
  $$
Como consequ\^encia do teste M de Weierstrass, j\'a que $\frac{r}{\rho}$ \'e um n\'umero, segue
o resultado.
\qedsymbol
\end{proof*}
\begin{exer*}
  Mostre que o R do teorema acima \'e \'unico.
\end{exer*}
\begin{example}
  Considere a s\'erie que define a exponencial de z:
  $$
  \sum_{n=0}^{\infty}\frac{z^n}{n!}, R = \infty.\quad e^z:=\sum_{n=0}^{\infty}\frac{z^n}{n!}, z\in\mathbb{C}.
  $$
  Este s\'erie \'e convergente pelo teste da raz\~ao. Com efeito, 
  $$
  R = \lim_{n\to\infty}\biggl|\frac{a_n}{a_{n+1}}\biggr| = \lim_{n\to\infty}\biggl(\frac{(n+1)!}{n!}\biggr) = \infty.
  $$
  Com isso, a s\'erie converge para todos os valores poss\'iveis, pois seu raio de converg\^encia
\'e infinito.
\end{example}
\begin{prop*}
  Nas nota\c c\~oes da proposi\c c\~ao anterior, se $R < \infty$, ent\~ao
  $$
    R = \lim_{n\to\infty}\biggl|\frac{a_n}{a_{n+1}}\biggr|.
  $$
\end{prop*}

\subsection{Fun\c c\~oes Anal\'iticas}
\begin{def*}
  Seja G um aberto de $\mathbb{C}$ e $f:G\rightarrow\mathbb{C}$ uma fun\c c\~ao. Dizemos que ela \'e diferenci\'avel
em $z\in{G}$ se
  $$
  \lim_{h\to{0}}\frac{f(z+h) - f(z)}{h} = \lim_{w\to{z}}\frac{f(z)-f(w)}{z-w}
  $$
existe. Neste caso, o denotamos por f'(z). Diremos que f \'e diferenci\'avel se f'(z) existe para todo z de G.
\end{def*}
\begin{def*}
  Se $f:G\rightarrow\mathbb{C}$ \'e diferenci\'avel e $f':G\rightarrow\mathbb{C}(z\mapsto{f'(z)})$ \'e cont\'inua,
ent\~ao dizemos que f \'e continuamente diferenci\'avel.
  
  Analogamente, se $f':G\rightarrow\mathbb{C}$ \'e diferenci\'avel e $f'':G\rightarrow\mathbb{C}$ (f'' = (f')') \'e
cont\'inua, ent\~ao f \'e duas vezes continuamente diferenci\'avel. Nesta linha, diremos que uma fun\c c\~ao \'e
anal\'itica se ela \'e continuamente diferenci\'avel em G.
\end{def*}

\begin{prop*}
  Seja G um aberto de $\mathbb{C}$. Ent\~ao, 
  \begin{itemize}
    \item[i)] Se $f:G\rightarrow\mathbb{C}$ \'e diferenci\'avel em $a\in{G}$, ent\~ao f \'e cont\'inua em a;
    \item[ii)] Se f e g s\~ao anal\'iticas em G, ent\~ao f+g e f.g s\~ao anal\'iticas em G. Se $G' = G - \{0\}$, 
  ent\~ao f/g \'e anal\'itica em G'. Valem as regras cl\'assicas de deriva\c c\~ao.
    \item[iii)] Sejam f e g anal\'iticos em $G_f, G_g$, respectivamente, com $f(G_f)\subseteq{f(G_g)}$. Ent\~ao, 
  $g\circ f$ \'e anal\'itica em $G_f$ e 
    $$
    (g\circ f)'(z) = g'(f(z))f'(z), \quad z\in{G}.
    $$
  \end{itemize}
\end{prop*}
\begin{proof*}
  Exerc\'icio.
\qedsymbol
\end{proof*}

\begin{prop*}
  Seja $f(z) = \sum\limits_{n=0}^{\infty}a_n(z-a)^n$ com raio de converg\^encia R. Ent\~ao,
f \'e infinitamente diferenci\'avel em B(a, R). Al\'em disso, a derivada de ordem k \'e
  $$
    f^{(k)}(z) = \sum_{n=k}^{\infty}a_n\frac{n!}{(n-k)!}(z-a)^{n-k}, k\in\mathbb{N}
  $$
com mesmo raio de converg\^encia de f.
\end{prop*}
\begin{proof*}
  A \'ultima afirma\c c\~ao fica como exerc\'icio. 

  Consideremos 
\begin{align*} 
  &s_n(z) = \sum_{k=0}^{n}a_k(z-a)^k, \quad R_n(z)= f(z) - s_n(z), \\
  &g(z) = \sum_{n=1}^{\infty}a_nn(z-a)^{n-1}, \quad z\in{B(a, R)}, n\in\mathbb{N}.
\end{align*} 
  Seja $\delta > 0$ tal que $B(z, \delta)\subseteq{B(a, r)}$ com $|z| < r < R.$ Assim, para 
w em $B(z, \delta)$
  $$
    \frac{f(z)-f(w)}{z-w} - g(z) = \frac{s_n(z) - s_n(w)}{z-w} + \frac{R_n(z) - R_n(w)}{z-w} - g(z) =
= \biggl[\frac{s_n(z) - s_n(w)}{z-w} - s_n'(z)\biggr] + \biggl[\frac{R_n(z) - R_n(w)}{z-w}\biggr] - (g(z) - s_n'(z)).
  $$
  Note que
  $$
    \biggl|\frac{R_n(z) - R_n(w)}{z-w}\biggr| = \biggl|\frac{1}{z-w}\sum_{k=n+1}^{\infty}a_k\frac{[(z-a)^k - (w-a)^k]}{(z-a)-(w-a)}\biggr|
= \biggl|\sum_{k=n+1}^{\infty}a_k\biggl((z-a)^{k-1} + \cdots + (w-a)^{k-1}\biggr)\biggr|
\leq \sum_{k=n+1}^{\infty}|a_k|kr^{k-1}\to{0},
  $$
pois $g(r) < \infty,$ em que n tende a infinito.
  Como as duas express\~oes em chaves tendem a 0 quando w tende a z, conclu\'imos que
  $$
    \lim_{z\to{w}}\frac{f(z)-f(w)}{z-w} = g(z)
  $$
e a afirma\c c\~ao segue.
\qedsymbol
\end{proof*}
\begin{crl*}
  Nas nota\c c\~oes e condi\c c\~oes da proposi\c c\~ao anterior, f \'e anal\'itica em
B(a, R) e 
  $$
    a_n = \frac{f^{(n)}(a)}{n!}, n\in\mathbb{N}
  $$
\end{crl*}
\begin{proof*}
  Exerc\'icio.
\end{proof*}
\begin{prop*}
  Seja G aberto e conexo. Se $f:G\rightarrow\mathbb{C}$ \'e tal que $f'(z) = 0, z\in{G},$ 
ent\~ao f \'e constante.
\end{prop*}
\begin{proof*}
  Seja $z_0\in{G}$ e considere $C = f^{-1}(\{f(z_0)\}),$ tal que C \'e n\~ao-vazio e fechado. 
Mostremos que C \'e, tamb\'em, aberto.
  Seja z um elemento de C e $r > 0$ tal que $B(z, r)\subseteq{G}$. Para todo $w\in{B(z, w)}$,
definimos $g:[0, 1]\rightarrow\mathbb{C}$ por g(t) = f(tz + (1-t)w). Neste caso, 
  $$
    g'(t) = f'(tz + (1-t)w)(z-w) = 0.
  $$
  Como g \'e real, segue que ela \'e constante. Com isso, note que $f(w) = g(0) = g(1) =
f(z) = f(z_0)$, tal que $w\in{C}$.
\qedsymbol
\end{proof*}
\begin{example}
  $e^z = \sum_{n=0}^{\infty}\frac{z^n}{n!}, R = \infty$
\end{example}
  Coloque $g(z) = e^ze^{z-w}, w\in\mathbb{C}$ fixo. Temos $g'(z) = (e^z)'e^{w-z} + e^z(e^{w-z})' - 0.$
Assim, g \'e constante e, como $g(0) = e^w$, conclu\'imos que $e^w = e^ze^{w-z}$ para todo
$z, w\in\mathbb{C}$.
\begin{exer*}
  Prove que, para $z, w\in\mathbb{C},$:
  \begin{itemize}
    \item[1)] $e^{z+w} = e^ze^w;$
    \item[2)] $e^ze^{-z} = 1;$
    \item[3)] $e^{\overline{z}} = \overline{(e^z)}$;
    \item[4)] $|e^z| = e^{Re(z)}.$
  \end{itemize}
\end{exer*}
\begin{example}
  Defina, para z complexo,
  \begin{align*}
    &\cos{(z)} = \sum_{n=0}^{\infty}(-1)^n\frac{z^{2n}}{(2n)!} \\
    &\sin{(z)} = \sum_{n=0}^{\infty}(-1)^n\frac{z^{2n+1}}{(2n+1)!}.
  \end{align*}
\end{example}
\begin{exer*}
  Dado z complexo, mostre que
  \begin{itemize}
    \item[i)] $(\sin{(z)}' = \cos{(z)}, \quad (\cos{(z)})' = -\sin{(z)};$
    \item[ii)] $\cos{(z)} = \frac{1}{2}\biggl(e^{iz} + e^{-iz}\biggr), \quad \sin{(z)} = \frac{1}{2}
    \biggl(e^{iz} - e^{-iz}\biggr)$; 
    \item[iii)] $\cos^2{(z)} + \sin^2{(z)} = 1$;
  \item[iv)] $e^{iz} = \cos{(z)} + i\sin{(z)}.$ 
  \end{itemize}
\end{exer*}

\subsection{Ramos de Fun\c c\~oes Inversas}
  Seja $z\in \mathbb{C}.$ Buscamos $w\in \mathbb{C}$ tal que $e^{w} = z, z\neq0.$ Logo, w deve satisfazer $|e^{w}|
= e^{Re(w)} = |z|\Rightarrow Re(w) = \ln{|z|}.$ Se w = x + iy, ent\~ao
  $$
  e^{w} = e^{x}e^{iy} = e^{x}\biggl(\cos{(y)} + i\sin{(y)}\biggr) = z = |z|\biggl(\cos{(\theta)} + i\sin{(\theta)}\biggr)
  $$
  com $\theta = \arg{z}.$ Assim, $y = \theta + 2k\pi$ para algum k. Portanto, $w = \ln{|z|} + i(\arg{z} + 2k\pi), k\in \mathbb{Z}.$
 \begin{def*}
   Seja G um aberto conexo de $\mathbb{C} e f:G\rightarrow \mathbb{C}$ cont\'inua. Diremos que f \'e um ramo de logar\'itmo
  em G se $e^{f(z)} = z, z\in{G}.$
 \end{def*}
\begin{prop*}
  Se G \'e um aberto conexo e f, g s\~ao ramos de logar\'itmos em G, ent\~ao $f(z) = g(z) + 2k\pi i$ para algum
$k\in \mathbb{Z}$.
\end{prop*}
\begin{proof*}
  Seja z em G. Mostraremos que 
  $$
    \frac{f(z) - g(z)}{2\pi i}\in \mathbb{Z}.
  $$
  Observe que $e^{f(z) - g(z)} = \frac{e^{f(z)}}{e^{g(z)}} = \frac{z}{z} = 1.$ Da\'i, $f(z) = g(z) + 2k\pi i$ para algum
inteiro k, pois
  $$
  f(z) - g(z) = \ln{|1|} + i(\arg{1} + 2k\pi)
  $$
Definimos $h:G\rightarrow \mathbb{C}$ por 
  $$
  h(w) = \frac{f(w) - g(w)}{2 \pi i}, \quad \in{G}.
  $$
  De forma an\'aloga ao anterior, conclu\'imos $Im(h)\subseteq{\mathbb{Z}}$ deve ser conexo, pois h \'e cont\'inua. Assim,
h \'e constante, pois os \'unicos conexos de $\mathbb{Z}$ s\~ao o vazio e conjuntos unit\'arios, provando o resultado. \qedsymbol
\end{proof*}
\begin{prop*}
  Sejam $G, \Omega$ abertos e $f:G\rightarrow \mathbb{C}\text{ e }g:\Omega\rightarrow \mathbb{C}$ cont\'inuas com $f(G)\subseteq{\Omega}$
e satisfazendo $f(g(z)) = z, z\in{G}.$ Se g \'e diferenci\'avel em z e $g'(f(z))\neq0,$ ent\~ao
  $$
    f'(z) = \frac{1}{g'(f(z))}.
  $$
Caso g seja anal\'itica, f tamb\'em o \'e.
\end{prop*}
\begin{proof*}
  Exerc\'icio.
\end{proof*}
  Considere G um aberto conexo. Chamamos a fun\c c\~ao $f:G\rightarrow \mathbb{C} $ dada por 
  $$
  f(z) = \ln{|z|} + i \theta, \quad \theta=\arg{(z)}\in{(-\pi, \pi)}
  $$
de ramo principal do logar\'itmo.
\newpage

\section{Aula 04 - 09/01/2023}
\subsection{Motiva\c c\~oes}
\begin{itemize}
  \item Equa\c c\~oes de Cauchy-Riemann;
  \item Fun\c c\~oes Harm\^onicas e suas Rela\c c\~oes com as Anal\'iticas.
  \item Fun\c c\~oes Conformes e Transforma\c c\~oes de M\"{o}bius
\end{itemize}
\subsection{Equa\c c\~oes de Cauchy-Riemann}
\begin{def*}
  Uma regi\~ao G do plano complexo \'e um aberto conexo dele.
\end{def*}
  Considere uma fun\c c\~ao $f:G\rightarrow \mathbb{C}$ anal\'itica sobre a regi\~ao G e defina
  $$
  u(x, y) = Re(f(z)), \quad v(x, y) = Im(f(z)), \quad z=x+iy, x, y\in \mathbb{R}
  $$
Assim, $f(z) = u(x, y) + iv(x, y), z=x+iy\in \mathbb{C}.$ Observe que 
  \begin{align*} 
    f'(z) &= \lim _{h\to{0}}\frac{f(z+h) - f(z)}{h} = \lim _{ih\to{0}}\frac{f(z+ih) - f(z)}{ih} \\
          &= \lim _{h\to{0}}\biggl(\frac{u(x+h, y) - u(x, y)}{h} + i\frac{v(x + h, y) - v(x, y)}{h}\biggr)
  \end{align*} 
 \begin{equation}
   = \frac{du}{dx}(x, y) + i \frac{dv}{dx}(x, y), \quad z = x + iy  
 \end{equation}
\begin{align}
   &= \lim _{ih\to{0}}\biggl(\frac{u(x, y+h) - u(x, y)}{ih} + i\biggl(\frac{v(x, y+h) - v(x, y)}{ih}\biggr)\biggr) \nonumber\\
   &= \frac{1}{i}\frac{du}{dy}(x, y) + \frac{dv}{dy}(x, y) = \frac{dv}{dy}(x, y) - i \frac{du}{dy}(x, y).
\end{align}
  A partir de (1) e (2), derivamos as equa\c c\~oes de Cauchy-Riemann:
  $$
  \boxed{\frac{du}{dx}=\frac{dv}{dy} \quad \text{ e } \frac{dv}{dx} = -\frac{du}{dy}}
  $$
\subsection{Fun\c c\~oes Harm\^onicas}
  Al\'em disso, se u e v possuem derivadas de segunda ordem, temos
  $$
  \frac{d}{dy}\biggl(\frac{du}{dx}\biggr) = \frac{d^2v}{dy^2}, \quad \frac{d}{dy}\biggl(\frac{dv}{dx}\biggr), \quad \frac{d^2v}{dx^2} = -\frac{dy}{dxdy}
  $$
de onde segue que 
  $$
  \frac{d^2v}{dx^2} + \frac{d^2v}{dy^2} = 0
  $$
e, de forma an\'aloga, u \'e harm\^onica. Nesta l\'ogica, diremos que f \'e harm\^onica
se $\Delta f = \frac{d^2f}{dx^2} + \frac{d^2f}{dy^2} = 0.$
  
  Seja $u:G\rightarrow \mathbb{R}$ harm\^onica, a busca por $v:G\rightarrow \mathbb{R}$ harm\^onica 
satisfazendo Cauchy-Riemman \'e um quest\~ao. Um exerc\'icio \'e mostrar que a exist\^encia de v
depende de G e que, em geral, n\~ao encontra-se v harm\^onica satisfazendo Cauchy-Riemann. 
(Por exemplo, $G = G - \{0\}, \quad u(x, y) = \ln{(x ^{2} + y ^{2})}^{\frac{1}{2}}$)
 \begin{theorem*}
   Sejam $u, v:G\rightarrow \mathbb{R}$ harm\^onicas de classe $C^1$. Ent\~ao, $f = u + iv$
\'e anal\'itica se e s\'o se u e v satisfazem Cauchy-Riemann.
 \end{theorem*}
\begin{proof*}
  Exerc\'icio.
\end{proof*}
  Dada $u:G\rightarrow \mathbb{R}$ harm\^onica, uma fun\c c\~ao $v:G\rightarrow \mathbb{R}$
tal que f = u + iv seja anal\'itica \'e dita ser a fun\c c\~ao harm\^onica conjugada de u.
\begin{exer*}
\item[1)] Seja $f:G\rightarrow \mathbb{C}$  um ramo e n um natural. Ent\~ao, $z ^{n} = e ^{nf(z)}, z\in{G}.$
  \item[2)] Mostre que $Re(z ^{\frac{1}{2}}) > 0;$
  \item[3)] tome $G = \mathbb{C} - \{z: z\leq{0}\}.$ Ache todos as fun\c c\~oes anal\'iticas
tais que $z = (f(z))^{n}.$
  \item[4)] Seja $f:G\rightarrow \mathbb{C}$, G conexo e f ana\'itica. Se, para todo
z de G, f(z) \'e real, ent\~ao f \'e constante.
\end{exer*}
\begin{theorem*}
  Considere $G = \mathbb{C} \text{ ou } G = B(0, r), r > 0.$ Se $u:G\rightarrow \mathbb{R}$, 
ent\~ao u admite harm\^onico conjugado.
\end{theorem*}
\begin{proof*}
  Buscamos $v:G\rightarrow \mathbb{R}$ satisfazendo Cauchy-Riemann. Coloque 
  $$
    v(x, y) = \int_{0}^{y}\frac{du}{dx}(x, t)dt + \phi(x)
  $$
em que $\phi(x) = -\int\limits_{0}^{x}\frac{du}{dy}(t, 0)dt.$

  Portanto, 
  $$
  f = u(x, y) + i\biggl(\int_{0}^{y}\frac{du}{dx}(x, t)dt - \int_{0}^{x}\frac{du}{dy}(t, 0)dt.\biggr).\quad\text{\qedsymbol}
  $$
\end{proof*}

\subsection{Transforma\c c\~oes Conformes}
 \begin{exer*}
   Mostre que $e^{z}$ leva retas ortogonais em curvas ortogonais.
 \end{exer*}
\begin{def*}
 Uma $\gamma$ \'e uma curva numa regi\~ao G se $\gamma:[a, b]\rightarrow G$ \'e cont\'inua.
\end{def*}
Sejam $\gamma _{1}, \gamma_2$ curvas em G tais que $\gamma_1'(t_1)\neq{0}, \gamma_2'(t_2)\neq{0}, \gamma_1(t_1) = \gamma_2(t_2) = z_{0}\in{G}.$
O \^angulo entre $\gamma _{1}\text{ e }\gamma_2$ em $z_{0}$ \'e dado por
  $$
  \arg(\gamma_1'(t_1)) - \arg(\gamma_2'(t_2)).
  $$
  Observe que se $\gamma$ \'e uma curva em G e $f:G\rightarrow \mathbb{C}$ \'e anal\'itica,
$\sigma = f\circ\gamma$ \'e uma curva em $\mathbb{C}.$ Assumimos $\gamma\in{C^1}.$ Neste
caso, $[a, b] = Dom(\gamma),$ ou seja, temos
  $$
  \gamma'(t) = f'(\gamma(t))\gamma'(t), \quad t\in{[a, b]},
  $$
donde segue que
  $$
  \arg(\gamma'(t)) = \arg(f'(\gamma(t))) + \arg(\gamma'(t))
  $$
\begin{theorem*}
  Seja $f:G\rightarrow \mathbb{C}$ anal\'itica. Ent\~ao, f preserva \^angulos para todo
z em G tal que $f'(z)\neq{0}$.
\end{theorem*}
\begin{proof*}
  Seja $z_{0}\in{G}$ tal que $f'(z_{0})\neq{0}$. Considere curvas $\gamma_1, \gamma_2$
tais que $\gamma_1(t_1) = \gamma_2(t_2) = z_{0}.$ Se $\theta$ \'e \^angulo entre $\gamma_1\text{ e }\gamma_2\text{ em }z_{0},$
ent\~ao
  $$
  \theta = \arg(\gamma_1'(t_1)) - \arg(\gamma_2'(t_2))
  $$
  Agora, note que o \^angulo entre $\sigma_1 = f\circ{\gamma_1}$ e $\sigma_2 = f\circ{\gamma_2}$ em
$f(z_{0})$ \'e
  $$
  \arg \sigma_1'(t_1) - \arg \sigma_2'(t_2) = \theta.
  $$
Portanto, f preserva \^angulos. \qedsymbol.
\end{proof*}
  Seja $f:G\rightarrow \mathbb{C}$ que preserva \^angulo e 
  $$
  \lim_{w\to{z}} \frac{|f(z) - f(w)|}{|z-w|}
  $$
existe. Ent\~ao, f \'e dita aplica\c c\~ao conforme. Por exemplo, $f(z) = e^z$ \'e injetora
em qualquer faixa horizontal de largura menor que $2\pi.$
\begin{crl*}
  $e ^{G} = \mathbb{C} - \{z: z\leq{0}\}.$
\end{crl*}
  Se G \'e uma faixa aberta de comprimento $2\pi$, o ramo de log faz o caminho inverso. Adicionalmente,
$\frac{1}{z}$ \'e a sua derivada.
\newpage

\section{Aula 05 - 10/01/2023}
\subsection{Motiva\c c\~oes}
\begin{itemize}
  \item Transforma\c c\~oes de M\"{o}bius elementares;
  \item Consequ\^encias Geom\'etricas da Transforma\c c\~ao de M\"{o}bius;
\end{itemize}
\subsection{Transforma\c c\~oes de M\"{o}bius}
 \begin{def*}
   Uma fra\c c\~ao linear \'e $\frac{az + b}{cz + d}, z\in \mathbb{C}, a, b, c, d\in \mathbb{C}$ fixos.
 \end{def*}
\begin{def*}
  Uma fra\c c\~ao linear tal que $ad-bc\neq0$ define uma transforma\c c\~ao
  $$
  T(z) = \frac{az + b}{cz + d}, \quad z\in \mathbb{C},
  $$
chamada tranforma\c c\~ao de M\"{o}bius.
\end{def*}
  Consideraremos a tranforma\c c\~ao como sendo $T:\mathbb{C}_{\infty}\rightarrow \mathbb{C}_{\infty}$ da seguinte
maneira:
\begin{align*}
  &T(z) = \frac{az + b}{cz + d}, \quad z\neq -\frac{d}{c} \\
  &T\biggl(-\frac{d}{c}\biggr) = \infty \quad\text{ e }\quad T(\infty) = \frac{a}{c}.
\end{align*}
Neste caso, $T ^{-1}(z) = \displaystyle\frac{dz - b}{-cz + a}, \quad z\in \mathbb{C}_{\infty}.$ Note, tamb\'em, que os coeficientes
de uma Transforma\c c\~ao de M\"{o}bius s\~ao unicamente determinados, pois 
  $$
  \frac{az + b}{cz + d} = \frac{(\lambda a)z + (\lambda b)}{(\lambda c)z + (\lambda d)}, \quad \lambda\neq0.
  $$
  Denotaremos por TM a cole\c c\~ao de transforma\c c\~oes de M\"{o}buis.
 \begin{example}
  As TM's elementares, dado $a\in \mathbb{C}$, s\~ao
 \begin{itemize}
   \item[-] Transla\c c\~ao: $T(z) = z + a, z\in \mathbb{C}_{\infty},$
   \item[-] Rota\c c\~ao: $R(z) = e^{i \theta}z, \theta\in \mathbb{R},$
   \item[-] Invers\~ao: $I(z) = \frac{1}{z},$
   \item[-] Homotetia: $H(z) = az.$
 \end{itemize}
 \end{example}
 \begin{prop*}
   Toda TM \'e composi\c c\~ao de TM's elementares.
 \end{prop*}
 \begin{proof*}
   Seja $T\in{TM}$ dada por $T(z) = \displaystyle\frac{az + b}{cz + d}.$

   Caso 1) Se c = 0, ent\~ao $T(z) = \frac{az}{d} + \frac{b}{d}.$ Neste caso, $H(z) = \frac{a}{d}z \text{ e } S(z) = z + \frac{b}{d},$
   tal que $T(z) = S\circ{H(z)}$

  Caso 2) Se $c\neq0$, ent\~ao tome 
  $$
   T_1(z) = z + \frac{d}{c}, I(z) = \frac{1}{z}, H(z) = \frac{(bc - ad)z}{c^2}, \text{ e } T_2(z) = z + \frac{a}{c}.
  $$
Com isso, temos 
  $$
  t_2\circ{H}\circ{I}\circ{T_1} = t. \quad \text{\qedsymbol}
  $$
 \end{proof*}
\begin{exer*}
 \begin{itemize}
   \item[1)]Mostre que $(TM, \circ)$ \'e um grupo.
   \item[2)] Se $T\in{TM}$ \'e tal que $T(z_{i}) = z_{i}, i = 1, 2, 3, z_{i}\neq z_{j}, i\neq{j},$ ent\~ao $T = Id_{\mathbb{C}_{\infty}}.$
 \end{itemize}
\end{exer*} 
 \begin{prop*}
   Sejam $z_1, z_2, z_3\in \mathbb{C}_{\infty}, $ distintos. Existe uma \'unica $T\in{TM}$ tal que
   $$
    T(z_1) = 1, T(z_2) = 0, T(z_3) = \infty.
   $$
 \end{prop*}
\begin{proof*}
  \underline{Unicidade}: 

  Se existem $T, S\in{TM}$ satisfazendo a hip\'otese, ent\~ao $S^{-1}(T(z_i)) = z_{i}, i=1, 2, 3$. Logo, 
  $S^{-1}\circ{T} = Id_{\mathbb{C}_{\infty}} \text{ e } S = T.$

  \underline{Exist\^encia}: Defina $T:\mathbb{C}_{\infty}\rightarrow \mathbb{C}_{\infty}$ por 
  $$
  T(z) = \left\{\begin{array}{ll}
      \displaystyle\frac{\frac{z-z_2}{z-z_3}}{\frac{z_1-z_2}{z_1-z_3}}, \quad z_{i}\in \mathbb{C}, i=1, 2, 3; \\\
      \displaystyle\frac{z-z_2}{z-z_3}, \quad z_1 = \infty; \\\
      \displaystyle\frac{z_1 - z_3}{z - z_3}, \quad z_2 = \infty; \\\
      \displaystyle\frac{z - z_2}{z_1 - z_2}, \quad z_3 = \infty.
    \end{array}\right.,
  $$
  tal que $T\in{TM}$ satisfazendo a hip\'otese. \qedsymbol
\end{proof*}
 \begin{crl*}
   Dados $z_1, z_2, z_3, w_1, w_2, w_3$ distintos em $\mathbb{C}_{\infty}$, existe uma \'unica $T\in{TM}$ tal que
  $$
  T(z_{i}) = w_{i}, \quad i=1, 2, 3.
  $$
 \end{crl*}
\begin{proof*}
  Exerc\'icio. \qedsymbol
\end{proof*}
  Observe que se $z_{i}\in \mathbb{C}_{\infty}, i = 1, 2, 3,$ distintos e $T\in{TM}$ \'e tal que a proposi\c c\~ao
seja satisfeita, denotaremos T(z) por  $T(z) := [z, z_1, z_2, z_3].$
\begin{example}
  Se $[z, 1, 0, \infty] = z, z\in \mathbb{C}_{\infty}, z_1, z_2, z_3\in \mathbb{C}_{\infty}$ distintos, ent\~ao
  \begin{align*}
    &[z_1, z_1, z_2, z_3] = 1; \\
    &[z_2, z_1, z_2, z_3] = 0; \\
    &[z_3, z_1, z_2, z_3] = \infty.
  \end{align*}
\end{example}
\begin{prop*}
  Sejam $z_1, z_2, z_3\in \mathbb{C}_{\infty}$ distintos e $S\in{TM}.$ Ent\~ao, 
  $$
  [z, z_1, z_2, z_3] = [S(z), S(z_1), S(z_2), S(z_3)], \quad z\in \mathbb{C}_{\infty}.
  $$
\end{prop*}
\begin{proof*}
  Seja $T(z) = [z, z_1, z_2, z_3]$ e tome $M = T\circ{S^{-1}}.$ Note que 
 \begin{align*}
   &M(S(z_1)) = 1, \\
   &M(S(z_2)) = 0, \\
   M(S(z_3)) = \infty.
 \end{align*} 
 Assim, 
 $$
 M(z) = [S(z), S(z_1), S(z_2), S(z_3)]
 $$
 e $T(z) = M(S(z)) = [S(z), S(z_1), S(z_2), S(z_3)]$. \qedsymbol
\end{proof*}
\begin{prop*}
  Sejam $z_1, z_2, z_3, z_4\in \mathbb{C}_{\infty}$ distintos. Ent\~ao, $[z_1, z_2, z_3, z_4]\in \mathbb{R}$ se e s\'o se 
$z_{i}\in{C}$ para algum c\'irculo.
\end{prop*}
\begin{proof*}
  $\Rightarrow)$ Se $z_{i}\in{C}, i=1, 2, 3, 4,$ ent\~ao $z_1\in{D},$ em que D \'e o \'unico c\'irculo determinado por $z_2, z_3, z_4.$
 \begin{exer*}
   Mostre que $[z_1, z_2, z_3, z_4]\in \mathbb{R}$
 \end{exer*}

 $\Leftarrow)$ Definimos $S(z) = [z, z_2, z_3, z_4], z\in \mathbb{C}_{\infty}$. Mostraremos que $S^{-1}(\mathbb{R})\subseteq{\mathbb{R}}$ e $S^{-1}(\mathbb{R}_{\infty})$ \'e um c\'irculo.
 \underline{Caso 1}: Seja $w\in{S^{-1}(\mathbb{R})}$ e sejam a, b, c, d n\'umeros complexos tais que
   $$
   S(z) = \frac{az + b}{cz + d}.
   $$
   como S(w) pertence a $\mathbb{R}$, temos $S(w) = \overline{S(w)},$ donde segue que 
   $$
   \frac{aw + b}{cw + d} = \frac{\bar{a}\bar{w} + \bar{b}}{\bar{c}\bar{w} + \bar{d}},
   $$
   o que implica em $(cw + d)(\bar{a}\bar{w} + \bar{b}) = (aw + b)(\bar{c}\bar{w} + \bar{d}).$ Logo, 
   \begin{equation}\label{MTSF}
     (c\bar{a} - a\bar{c})|w|^2 + (c\bar{b} - a\bar{d})w + (d\bar{a} - b\bar{c})\bar{w} + (d\bar{b} - b\bar{d}) = 
     2iIm(a\bar{c}) + 2i(Im(w(-\bar{b}c + a\bar{d})) + 2iIm(b\bar{d})) = 0.
  \end{equation}

  \underline{Caso 1.1}: $Im(\bar{a}c) = 0$, seja $\alpha = bc - ad.$ Segue de \ref{MTSF} que 
  $$
    2i(Im(w \alpha) + Im(d\bar{b})) = 0.
  $$
Logo, $Im(\alpha w + \beta) = 0, \beta = Im(d\bar{b})$. Assim, $\alpha w + \beta\in r,$ em que $r: \frac{-\beta t}{\alpha}, t\in \mathbb{R}.$  

\underline{Caso 1.2}: $\rho = Im(\bar{a}c)\neq{0}$. Seja $\gamma = c\bar{b} - a\bar{d}.$ Ent\~ao, dividindo \ref{MTSF} por $2i\rho$, temos
 \begin{align*}
   &|w|^{2} + Im(\frac{\gamma}{\rho})w + Im(\frac{d\bar{b}}{\rho}) = 0 \\
   &|w - \gamma|^2 = (|\gamma|^2 - \beta)^{\frac{1}{2}} = r > 0.
  \end{align*}
\end{proof*}
\newpage

\section{Aula 06 - 12/01/2023}
\subsection{Motiva\c c\~oes}
 \begin{itemize}
   \item Transforma\c c\~oes de M\"{o}bius e Harm\^onicos Conjugados;
   \item Simetrias e Orienta\c c\~ao no Plano $\mathbb{C};$
   \item Integra\c c\~ao Complexa.
 \end{itemize}
\subsection{Exerc\'icios de Hoje }
\subsubsection{J\'essica}
\begin{itemize}
  \item[a)] $(7+i, 1, 0, \infty)$
  \item[b)] $(2, 1-i, 1, 1+i)$
  \item[c)] $(0, 1, i, -1)$
  \item[d)] $(i-1, \infty, 1+i, 0)$
\end{itemize}
  Utilizaremos os seguintes casos: Se $z_{2}, z_{3}, z_{4}\in \mathbb{C},$ ent\~ao $S(z) = \displaystyle \frac{\frac{z-z_{3}}{z-z_{4}}}{\frac{z_{2} - z_{3}}{z_{2} - z_{4}}}.$
Caso $z_{3} = \infty, \displaystyle S(z) = \frac{z - z_{3}}{z - z_{4}}.$ Por fim, se $z_{4} = \infty, S(z) = \displaystyle \frac{z - z_{3}}{z_{2} - z_{3}}.$ 
  
  Assim, vamos \`as contas.
 \begin{itemize}
   \item[a)] $$S(7 + i) = \frac{7 + i}{1} = 7 + i;$$
   \item[b)] $$S(2) = \frac{\frac{2 -1}{2 - 1 - i}}{\frac{1 - i - 1}{1 - i - 1 - i}} = \frac{\frac{1}{1-i}}{\frac{i}{2i}} = \frac{2}{1-i}\frac{1+i}{1+i} = \frac{2}{2}(1 + i) = 1 + i$$
   \item[c)] $$S(0) = \frac{\frac{0-i}{0+1}}{\frac{1-i}{2}} = \frac{-2i}{1-i}\frac{1+i}{1+i} = \frac{-2i}{2}(1+i) = 1-i;$$
   \item[d)] $$S(1i) = \frac{i-1-1-i}{i-1-0} = \frac{-2}{i-1}\frac{1+i}{1+i} = \frac{-2}{-2}1+i = 1+i.$$
 \end{itemize}

\subsubsection{Tiago}
  Vamos mostrar que $T(\mathbb{R}_\infty) = \mathbb{R}_{\infty}\Longleftrightarrow a, b, c, d\in \mathbb{R}$.
 $\Rightarrow)$ Suponha que $T(\mathbb{R}_\infty) = \mathbb{R}_\infty, T(z_{0}) = 0, z_{0}\in \mathbb{R}_\infty.$ Ent\~ao,
  $$
  \frac{az_{0} + b}{cz_{0} + d} = 0\Rightarrow az_{0} + b = 0\Rightarrow z_{0} = \frac{-b}{a}\in \mathbb{R}_\infty. 
  $$
  No caso de $z_\infty \in \mathbb{R}_\infty = \infty, $ ent\~ao
  $$
  \frac{az_{\infty} + b}{cz_{\infty} + d} = \infty\Rightarrow \frac{cz_{\infty} + d}{az_{\infty} + b} = 0\Rightarrow cz_\infty + d = 0
\Rightarrow z_\infty = \frac{-d}{c}\in \mathbb{R}_\infty.
  $$
  Agora, para $z_{1}\in \mathbb{R}_{\infty}, T(z_{1}) = 1,$ tal que 
  $$
    \frac{az_{1} + b}{cz_{1} + d} = 1\Rightarrow az_{1} + b = cz_{1} + d\Rightarrow z_{1}(a - c) = db\Rightarrow az_{1}\biggl(1 - \frac{c}{a}\biggr) = db
 \Rightarrow z_{1}\biggl(1 - \frac{c}{a}\biggr) = \frac{db}{a}
  $$
  $$
  \Rightarrow \frac{z_{1}}{c} - \frac{z_1}{a} = \frac{\frac{d}{a}}{c} - \frac{\frac{b}{a}}{c}\Rightarrow \frac{z_1}{c} - \frac{z_1}{a} =
= \frac{r_2}{a} - \frac{r_1}{c}\Rightarrow \frac{z_1 + r_1}{c} = \frac{r_2 + z_1}{a}\Rightarrow \frac{z_1 + r}{z_1 + r_2} = \frac{c}{a}
  $$
  $$
 \Rightarrow \frac{d}{a} = \frac{d}{c}\frac{c}{a}\in \mathbb{R}_{\infty} = r_2r_3
  $$
  Logo, colocando $r_{1} = \frac{b}{a}, r_2 = \frac{d}{c}, r_3 = \frac{c}{a},$ encontramos os coeficientes
  $$
    Tz = \frac{az + b}{cz + d} = \frac{a}{a}\frac{z + \frac{b}{a}}{z \frac{c}{a} + \frac{d}{a}} = \frac{z + r_1}{r_3z + r_2r_3}.
  $$

  $\Leftarrow)$ Para provar esse lado, considere $z\in \mathbb{R}_\infty.$ Ent\~ao, $T(z) = \frac{az + b}{cz + d}\in \mathbb{R}_\infty.$ Portanto,
  $T(\mathbb{R}_\infty) = \mathbb{R}_\infty.$  \qedsymbol

  \subsection{Final de Transforma\c c\~oes de M\"{o}bius.}
  A continua\c c\~ao da prova da proposi\c c\~ao \'e exerc\'icio.
 \begin{theorem*}
   Transforma\c c\~oes de M\"{o}bius levam c\'irculos em c\'irculos.
 \end{theorem*}
\begin{proof*}
  Exerc\'icio.
\end{proof*}

  \subsection{Simetria e Orienta\c c\~ao.}
  Dada uma circunfer\^encia $\Omega\text{ e }z_1, z_2, z_3\in \Omega$ distintos, diremos que z e z* s\~ao sim\'etricos se
  $[z^*, z_1, z_2, z_3] = \overline{[z, z_1, z_2, z_3]}$ 
 \begin{example}
   Um ponto \'e sim\'etrico a si mesmo se $z\in{C}$ com C o c\'irculo determinado por $z_1, z_2, z_3.$
 \end{example}
\begin{exer*}
  Mostre que a defini\c c\~ao de simetria n\~ao depende da escolha dos $z_{i}'s.$
\end{exer*}
  A ideia geom\'etrica por tr\'as desse conceito \'e a seguinte: Considere $\gamma$ uma reta e $z_1, z_2\in \mathbb{C}$
e coloque $z_3 = \infty.$ Dizer que z e z* s\~ao sim\'etricos equivale a
  $$
  [z^*, z_1, z_2, \infty] = \overline{[z, z_1, z_2, z_3]}\Longleftrightarrow \frac{z^* - z_2}{z_1 - z_2} = \overline{\biggl(\frac{z - z_2}{z_1 - z_2}\biggr)} =
  = \biggl(\frac{\overline{z} - \overline{z_2}}{\overline{z_1} - \overline{z_2}}\biggr).
  $$
  Assim, obtemos
  $$
  \frac{z^* - z_2}{z_1 - z_2} = \frac{\overline{z} - \overline{z_2}}{\overline{z_1} - \overline{z_2}} 
  $$
  o que implica
  $$
  \frac{z^* - z_2}{|z_1 - z_2|^2} = \overline{z} - \overline{z_2} \quad \text{(Exerc\'icio: } |z^* - z_2| = |z - z_2|)
  $$
  para qualquer $z_2$ em r. Logo, d(z*, r) = d(z, r). Al\'em disso, $[z^*, z]\perp{r}.$

  A seguir, vamos lidar com o conceito de simetria com rela\c c\~ao \`a um c\'irculo de $\mathbb{C}.$ De fato, tome
  \begin{align*}
  &\Omega = \{z: |z - a| = r\}, \quad r > 0. \\
  &[z^*, z_1, z_2, z_3] = \overline{[z, z_1, z_2, z_3]} = \quad \text{(Aplicando transla\c c\~ao, invers\~ao, homotetia:)} \\
  & = [\bar{z}, \bar{z_1}, \bar{z_2}, \bar{z_3}] = \biggl[\frac{r^2}{\bar{z} - \bar{a}}, \frac{r^2}{\bar{z_1} - \bar{a}}, \frac{r^2}{\bar{z_2} - \bar{a}}, \frac{r^2}{\bar{z_3} - \bar{a}}\biggr] \\
  & = \biggl[\frac{r^2 + a}{\bar{z} - \bar{a}}, z_1 - a, z_2 - a, z_3 - a\biggr].
  \end{align*}
  Decorre que 
  $$
  z^* = a + \frac{r^2}{\bar{z} - \bar{a}}
  $$
 \begin{exer*}
   $z^*\in{l}:= \{a + t(z-a): 0 < t < \infty\} $
 \end{exer*}
 \newpage

 \section{Aula 07 - 13/01/2023}
 \subsection{Motiva\c c\~oes}
  \begin{itemize}
    \item Simetrias e Transforma\c c\~oes d M\"{o}bius;
    \item Orienta\c c\~ao;
    \item Fun\c c\~oes de Varia\c c\~ao Limitada.
  \end{itemize}
  \subsection{Exerc\'icios de Hoje}
  \subsubsection{Ana L\'idia}
  Dado z em C, temos $\bar{z}$ em C tamb\'em, tal que $|z|^2 = z\bar{z} = 1$. Como queremos T(C) = C, $|T(z)| = 1$ e 
  $T(z)\overline{T(z)} = 1\forall z\in{C}$. Note que 
 \begin{align*}
   T(z)\overline{T(z)} &= 1\Longleftrightarrow \frac{az + b}{cz + d}\frac{\overline{az}+\overline{b}}{\overline{cz} + \overline{d}} = 1\Longleftrightarrow \\ 
                       &0 = z\overline{z}\biggl(a\overline{a} - c\overline{c}\biggr) + \overline{z}\biggl(b\overline{a} - d\overline{c}\biggr) + z(\overline{b}a - \overline{d}c) + b\overline{b} - d\overline{d}.
 \end{align*}
 Como $z\bar{z} - 1 = 0$, temos
 $$
   \left\{\begin{array}{ll}
      a\bar{a} - c\bar{c} = 1 \\
      b\overline{a} - d\overline{c} = 0 \\
      a\overline{b} - c\overline{d} = 0 \\
      b\overline{b} - d\overline{d} = -1
    \end{array}\right.
 $$
 Da\'i, 
\begin{align*}
  |a|^2 - |c|^2 = 1, |b|^2 - |d|^2 = -1\Rightarrow|a|^2 - |c|^2 = -|b|^2 + |d|^2\Rightarrow |a|^2 + |b|^2 = |c|^2 + |d|^2.
\end{align*}
  Assim, as condi\c c\~oes suficientes para o sistema s\~ao
  $$
  a\overline{b} - c\overline{d} = 0, \quad \text{ e } |a|^2 + |b|^2 = |c|^2 + |d|^2.
  $$
  Para a condi\c c\~ao necess\'aria, suponha $c = \lambda\overline{b}.$ Ent\~ao, 
  $$
  a\overline{b} - \lambda\overline{b}\overline{d} = 0\Rightarrow (a - \lambda\overline{d})\overline{b} = 0\Rightarrow a = \lambda\overline{d}\Longleftrightarrow d = \frac{\overline{a}}{\lambda}
  $$
  \subsubsection{Jo\~ao Vitor Occhiucci}
  Do Teorema 3.14, sabemos transformações de M\"{o}bius levam c\'irculos em c\'irculos, portanto
  $T(\mathbb{R}_\infty) = \mathbb{R}_\infty$ é equivalente a $Tz = (z, z_2, z_3, z_4)$ com $z_2, z_3, z_4 \in \mathbb{R}_\infty$ distintos. Ademais, do
exercício 3.7, temos
 \begin{align*}
  z_2 = \frac{d - b}{a - c}\\
  z_3 = -\frac{b}{a} \\
  z_4 = -\frac{d}{c}
 \end{align*}
  Portanto, \'e imediato que se existirem a, b, c e d reais para T, então $z_2, z_3, z_4$ estarão em $\mathbb{R}_\infty$ 
e, consequentemente, $T(\mathbb{R}_\infty) = \mathbb{R}_\infty.$
Por outro lado, se tivermos uma transformação de M\"{o}bius T, tal que $T(\mathbb{R}_\infty) = \mathbb{R}_\infty$, então
 $T(z) = (z, z_2, z_3, z_4)$ com $z_2, z_3, z_4 \in \mathbb{R}_\infty$ distintos. Daí, tome
 $$
   a = \frac{1}{z_2 - \_3}, b = \frac{z_3}{z_2 - z_3}, c = \frac{1}{z_2 - z_4}\text{ e }d = \frac{z_4}{z_2 - \_4}
 $$
  e
  $$
  Uz = \frac{az + b}{cz + d}
  $$
  Veja que $Uz_2 = 1, Uz_3 = 0 \text{ e } Uz_4 = \infty$, portanto, pela proposição 3.9, U = T, ou seja, podemos
escolher a, b, c e d reais tais que $Tz =\displaystyle \frac{az + b}{cz + d}$ .
  \subsection{Continuando Simetrias}
 \begin{prop*}
   Transforma\c c\~oes de M\"{o}bius levam pontos sim\'etricos em pontos sim\'etricos.
 \end{prop*}
 \begin{proof*}
  Seja l uma circunfer\^encia e z e z* sim\'etricos com rela\c c\~ao \`a $\Omega.$  Devemos mostrar que T(z), T(z*) s\~ao sim\'etricos
com rela\c c\~ao a $T(\Omega)$. Em outras palavras, queremos 
  $$
  [T(z), T(z_{1}), T(z_2), T(z_3)] = \overline{[T(z^*), T(z_1), T(z_2), T(z_3)]}.
  $$
  (Fica como exerc\'icio mostrar que $[T(z), T(z_{1}), T(z_2), T(z_3)] = [\bar{z}, \bar{z_1}, \bar{z_2}, \bar{z_3}]$).
\end{proof*}
\begin{def*}
  Dada uma circunfer\^encia $\Omega$ e uma tripla $z_{i}\in \Omega, i = 1, 2, 3,$ dizemos que esta tripla \'e uma orienta\c c\~ao.
Definimos o conjunto
  $$
    D_{l} = \{z: Im[z, z_1, z_2, z_3] > 0\}
  $$
  como o lado direito de l. O lado esquerdo, por outro lado, \'e 
  $$
    E_{l} = \{z: Im[z, z_1, z_2, z_3] < 0\}.
  $$
\end{def*}
 \begin{example}
   Um circuito passando por $\infty < z_1, z_2, z_3\text{ em }\mathbb{R}_\infty.$ Seja $T(z) = [z, z_1, z_2, z_3], z\in \mathbb{R}_\infty$. 
Neste caso, $T(z) = \frac{az + b}{cz + d}, a, b, c, d\in \mathbb{R}_\infty$. Assim, 
  $$
  \frac{az + b}{cz + d}\frac{c\bar{z} + d}{c\bar{z} + d} = \frac{ac|z|^2 + adz + cb\bar{z} + bd}{|cz+d|^2}.
  $$
  Logo, $Im(T(z)) > 0\Longleftrightarrow Im\biggl[(ad-bc)z\biggr] > 0.$ Portanto, se $\Omega$ \'e o c\'irculo,
  $$ 
  D_{\Omega} = \{z: (ad - bc)Im(z) > 0.\}
  $$
 \end{example}
 \begin{prop*}
   Sejam $\Omega_1, \Omega_2$ circunfer\^encias em $\mathbb{C}_\infty$ e T uma TM com $T(\Omega_1) = \Omega_2.$ Ent\~ao, T preserva
  orienta\c c\~ao.
 \end{prop*}
\begin{proof*}
  Exerc\'icio.
\end{proof*}
 \begin{example}
   Seja $D = \{z: Re z > 0\}, D^U = \{z: |z| < 1\} $. Seja $\Omega_1$ o c\'irculo e $\Omega_2$ dados por 
  \begin{align*}
    \Omega_1 = \{z: z = iy, y\in \mathbb{R}\} \\
    \Omega_2 = \{e^{iy}, y\in \mathbb{R}.\}
  \end{align*}
  Assim,
 \begin{align*}
   D_{\Omega_1} = \{z: Im[z, -i, 0, i] > 0\} = \{z: Im(iz) > 0\} = \{z: Re(z) > 0\}. \\
   D_{\Omega_2} = \{z: Im[z, -i, -1, i] > 0\} = \{z: |z| < 1\}.
 \end{align*}
 A TM que leva $\Omega_1\text{ em }\Omega_2$ \'e dada por
 $$
 T(z) = \frac{z-1}{z+1}, 
 $$
 e $M(z) = \frac{e^{z} - 1}{e^{z} + 1}$ \'e tal que $M(D) = D^U.$
\end{example}

 \subsection{Integra\c c\~ao Complexa}
 \subsubsection{Fun\c c\~oes de Varia\c c\~ao Limitada (BV - Bounded Variation)}
\begin{def*}
 Seja $\gamma:[a, b]\rightarrow \mathbb{C}$ uma fun\c c\~ao. Diremos que $\gamma$ tem varia\c c\~ao limitada se 
 $$
  v(\gamma, P) = \sum\limits_{k=1}^{n}|\gamma(t_{k}) - \gamma(t_{k+1})| < M, \quad M > 0,
 $$
 com $P = \{a=t_{0}, t_1, \cdots, t_n = b\} $ parti\c c\~ao de [a, b]. Se $\gamma$ \'e BV, a quantia
 $$
 V(\gamma) = \sup_p v(\gamma, P)
 $$
 \'e chamada varia\c c\~ao de $\gamma$.
\end{def*}
\begin{exer*}
  Se $P\subseteq{Q}, $ ent\~ao $V(\gamma, P)\leq{V(\gamma, Q)}$. Se $\gamma_1, \gamma_2$ s\~ao BV, ent\~ao $\alpha \gamma_1 + \beta \gamma_2$
\'e BV para $\alpha, \beta\in \mathbb{C}.$ Al\'em disso, 
  $$
  V(\alpha \gamma_1 + \beta \gamma_2) \leq |\alpha|V(\gamma_1) + |\beta|V(\gamma_2)
  $$
\end{exer*}
\begin{exer*}
  Se $\gamma$ \'e BV, ent\~ao ela \'e limitada, mas a rec\'iproca n\~ao vale.
\end{exer*}
\begin{example}
  Tome $\gamma:[a, b]\rightarrow \mathbb{C}, Im(\gamma) = 0$ e $\gamma$ crescente. Neste caso, $\gamma$ \'e BV. Com efeito, para
toda parti\c c\~ao P de [a, b], temos 
  $$
  v(\gamma, P) = \sum\limits_{k=1}^{n}|\gamma(t_{k}) - \gamma(t_{k+1})| = \gamma(b) - \gamma(a)
  $$
  De fato, dada uma $\gamma$ com as duas caracter\'isticas acima, ela \'e BV se, e s\'o se, $\gamma = \gamma_1 - \gamma_2$, com
 $\gamma_1, \gamma_2$ mon\'otonas crescentes.
 \begin{exer*}
   $$
   \gamma(t) = \left\{\begin{array}{ll}
       t\sin{(\frac{1}{t})}, \quad t\in[0, 2\pi] \\
       0, t= 0
  \end{array}\right.
  $$
  n\~ao \'e bv, apesar de ser con\'tinua.

  \underline{Dica: } Tome $t_{n} =\displaystyle \frac{1}{n\pi + \frac{\pi}{2}}\Rightarrow \gamma(t_{n}) =\displaystyle \frac{(-1)^n}{n\pi + \frac{\pi}{2}}\Rightarrow
v(\gamma, P) \geq c \sum\limits_{k=1}^{n}\frac{1}{k}.$
 \end{exer*}
\end{example}
Dada $\gamma$ BV em [a, b], considere 
  $$
  \gamma_t:[a, t]\rightarrow \mathbb{R}
  $$
  a restri\c c\~ao de $\gamma.$ Ent\~ao, considerando a aplica\c c\~ao $v(\gamma_t), t\in[a, b]$, crescente e BV, defina $
g(t) = \gamma(t) + v(\gamma_t)$, de modo que
  $$
  \gamma(t) = -g(t) _ v(\gamma_t).
  $$
  \newpage

\section{Aula 08 - 16/01/2023}
\subsection{Motiva\c c\~oes}
 \begin{itemize}
   \item Integra\c c\~ao Complexa em Curvas;
   \item Propriedades das Integrais de Linha;
   \item Primeira F\'ormula Integral de Cauchy.
 \end{itemize}
  \subsection{Finalizando Fun\c c\~oes BV's}
 \begin{prop*}
   Se $\gamma$ \'e suave, ent\~ao $\gamma$ \'e BV e $V(\gamma) = \int_{a}^{b}|\gamma'(t)|dt.$
 \end{prop*}
\begin{proof*}
  Seja $P=\{t_{0}, \cdots, t_n\} $ parti\c c\~ao de [a, b], 
  \begin{align*}
    v(\gamma, P) &= \sum\limits_{k=1}^{n}|\gamma(t_{k}) - \gamma(t_{k-1})|\\
                 &= \sum\limits_{k=1}^{n}\biggl|\int_{t_{k-1}}^{t_{k}}\gamma'(t)dt\biggr|\\
                 &\leq \sum\limits_{k=1}^{n}\int_{t_{k-1}}^{t_{k}}|\gamma'(t)|dt = \int_{a}^{b}|\gamma'(t)|dt
  \end{align*}
  Agora, dado $\epsilon > 0$, buscamos $\delta > 0$ tal que $|P| < \delta$ resulta em 
  $$
  v(\gamma) - \epsilon < v(\gamma, P)
  $$
  Seja $\delta > 0$ tal que $|\gamma'(s) - \gamma'(t)| < \epsilon$ para $|s - t| < \delta$. Agora, se $|P| < \delta,$ ent\~ao
  $$
  \biggl|\int_{a}^{b}|\gamma'(t)|dt - \sum\limits_{k=1}^{n}|\gamma'(s_{k})||t_{k}-t_{k-1}|\biggr| < \epsilon
  $$
  Com isso, o que quer\'iamos est\'a satisfeito e $s_{k}\in{[t_{k}, t_{k-1}]}$, o que implica em 
    $$
    0 < \sum\limits_{k=1}^{n}\int_{t_{k-1}}^{t_{k}}|\gamma'(s_{k}) - \gamma'(s_{k-1}) + |\gamma'(t)| dt + \epsilon < (b-a)\epsilon + 
    \int_{a}^{b}|\gamma'(t)|dt + \epsilon. \text{ \qedsymbol}
    $$
\end{proof*}
\subsection{Integrais de Linha}
\begin{def*}
  Seja $\gamma$ BV e $f:[a, b]\rightarrow \mathbb{C}$ limitada. Se existir I complexo tal que para todo $\epsilon > 0$ existe $\delta > 0$
de forma que $|P| < \delta$ implica
  $$
    \biggl|I - \sum\limits_{k=1}^{n}f(s_{k})|\gamma(t_{k}) - \gamma(t_{k-1})\biggr| < \epsilon,
  $$
  para qualquer escolha de $s_{k}\in{[t_{k}, t_{k-1}]},$ dizemos que I \'e a integral de f sobre a curva $\gamma$, denotado por
  $\int_{\gamma}^{}f\text{ ou } \int_{a}^{b}f(t)d\gamma(t)$
\end{def*}
 \begin{theorem*}
   Se $\gamma$ \'e BV e f \'e cont\'inua, ent\~ao f \'e integr\'avel sobre $\gamma.$
 \end{theorem*}
\begin{proof*}
  Usaremos o Teorema de Cantor.
 \begin{itemize}
   \item[1)] Dado $\epsilon > 0,$ seja $\delta > 0$ tal que $|s-t| < \delta.$ Ent\~ao, $|f(s) - f(t) < \epsilon$
   \item[2)] Para cada n natural, seja $\delta_{n} > 0$ tal que $|f(s) - f(t)| < \frac{1}{n}$, em que $\{\delta_{n}\}$ pode ser 
  considerado decrescente. 
 \end{itemize}

  Definimos 
    $$
    \mathcal{F}_{n} = \{s(f, p): |P| < \delta_{n}\}, \quad n\in \mathbb{N}
    $$
    Observamos que  $\overline{\mathcal{F}_{n}}\supseteq\overline{\mathcal{F}_{n+1}}$. Se $diam \mathcal{F}_{n}\to{0},$ ent\~ao
  o Teorema de Cantor garante que $\bigcap_{n=0}^{\infty} = \{I\}.$ Neste caso, $\int_{\gamma}^{}f.$ Fica de exerc\'icio mostrar que
  $diam \mathcal{F}_{n}\leq \frac{c}{n}.$
\end{proof*}
 \begin{prop*}
   Sejam $f, g:[a, b]\rightarrow \mathbb{C}$ cont\'inuas, $\gamma_{1}, \gamma_2$ BV em [a, b] e $\alpha, \beta\in{\mathbb{C}}.$
Temos: 
 \begin{itemize}
   \item[i)] $\int_{}^{}\alpha f + \beta g d;g = \alpha \int_{}^{}fd \gamma + \beta \int_{}^{}gd \gamma$;
   \item[ii)] $\int_{}^{}f d(\alpha \gamma_1 + \beta \gamma_2) = \alpha\int_{}^{}f d \gamma_1 + \beta \int_{}^{}f d \gamma_2;$
   \item[iii)] $\int_{a}^{b}f(t)d \gamma(t) - \int_{a}^{c}f(t)d \gamma(t) + \int_{c}^{b}f(t) d \gamma(t), \quad c\in{[a, b]}.$
 \end{itemize}
 \end{prop*}
\begin{theorem*}
  Seja $\gamma$ suave (ou suave por partes) e f con\t'inua. Ent\~ao, 
  $$
  \int_{}^{}f d \gamma = \int_{a}^{b}f(t)\gamma'(t)dt.
  $$
\end{theorem*}
$\gamma$ ser suave por partes significa que existe $P =\{a=t_{0}, t_1, \cdots, t_{n} = b\} $ tal que 
  $$
  \int_{a}^{b}f(t)d\gamma(t) = \sum\limits_{k=1}^{n}\int_{t_{k-1}}^{t_{k}}f(t)d\gamma(t)
  $$
 \begin{exer*}
   Mostre que para a f\'ormula acima, tamb\'em vale a igualdade com $\int_{t_{k-1}}^{t_{k}}f(t)\gamma'(t)dt.$
 \end{exer*}
 \begin{def*}
   Seja $\gamma$ uma curva em [a, b]. Escreveremos $\{\gamma\} $ para o tra\c co de $\gamma$ dado por 
   $$
   \{\gamma\}:= \{\gamma(t): t\in{[a, b]}\}.
   $$
   Note que $\{\gamma\} $ \'e conexo e compacto. Chamaremos de comprimento de $\gamma$ o valor $v(\gamma)$ caso $\gamma$ seja BV.
Tamb\'em chamaremos $\gamma$ de retific\'avel se $\gamma$ \'e curva BV.
 \end{def*}
\begin{def*}
  Seja $f:G\rightarrow \mathbb{C}$, $\gamma:[a, b]\rightarrow \mathbb{C}$ retific\'avel e $\{\gamma\}\subseteq{G}.$ A integral de 
linha de f ao longo de $\gamma$ \'e 
  $$
  \int_{\gamma}^{}f := \int_{a}^{b}f(\gamma(t))d\gamma(t).
  $$
\end{def*}
\begin{example}
 \begin{itemize}
   \item[i)] Seja $f(z) = z^n, \quad n\in{\mathbb{N}}$, sejam $\alpha, \beta\in \mathbb{C}$ e $\gamma:[0, 1]\rightarrow \mathbb{C}$
  dada por $\gamma(t) = t \beta + (1-t)\alpha.$
    $$
    \int_{\gamma}^{}=\int_{0}^{1}(t\beta + (1-t)\alpha)^n(\beta - \alpha)dt = \frac{(t\beta + (1-t)\alpha)^{n+1}}{n+1}\biggl|_0^1\biggr.
    $$
  \item[ii)] Tome $f(z) = z^{-n}, \quad n\neq1, \quad \gamma(t) = e^{-it}, t\in[-\pi, \pi].$ Ent\~ao,
    $$
    \int_{\gamma}^{}f = \int_{-\pi}^{\pi}e^{nit}(-i)e^{-it}dt = -i \int_{-\pi}^{\pi}e^{ti(n-1)}dt = 0.
    $$
  \item[iii)] Considere $f(z) = \frac{1}{z}, \quad \gamma(t) = e^{-it}, t\in[-\pi, \pi].$ Assim, 
    $$
    \int_{\gamma}^{}f = \int_{-\pi}^{\pi}e^{it}(-i)e^{-it} = -2\pi i.
    $$
 \end{itemize}
\end{example}
\begin{prop*}
  Seja $\gamma:[a, b]\rightarrow \mathbb{C}$ retific\'avel e $\phi:[c, d]\rightarrow [a, b]$crescente \'e cont\'inua. Se $f:G\rightarrow \mathbb{C}$
  \'e cont\'inua em $\{\gamma\} $, ent\~ao 
  $$
  \int_{\gamma}^{}f = \int_{\gamma\circ{\phi}}^{}f
  $$
\end{prop*}
\begin{proof*}
 \begin{exer*}
   Note que $\gamma\circ{\phi}$ \'e BV, $\{\gamma\}=\{\phi\} $ e $v(\gamma\circ{\phi}) = v(\gamma).$
 \end{exer*}
\end{proof*}
  Se dadas $\gamma, \sigma$, existir $\phi:[c, d]\rightarrow [a, b]$ com $\gamma:[a, b]\rightarrow \mathbb{C}\text{ e }\sigma:[c, d]\rightarrow \mathbb{C}, 
\sigma = \gamma\circ{\phi}, \phi$ con\'inua estritamente crescente, dizemos que $\gamma $ \'e equivalente \`a $\sigma$, denotado por $\gamma~\sigma.$
Segue da proposi\c c\~ao que $\gamma~\gamma$ implica 
  $$
  \int_{\gamma}^{}f = \int_{\sigma}^{}f.
  $$
 \begin{exer*}
   \'E verdade que $\{\gamma\}=\{\sigma\}\Rightarrow \int_{\gamma}^{}f = \int_{\sigma}^{}f?$
 \end{exer*}
   Seja $\gamma:[a, b]\rightarrow \mathbb{C}$ retific\'avel. Para cada $t\in{[a, b]},$ considere $\gamma_t:=\gamma|_[a, t]$ e 
note que $t\mapsto{v(\gamma_t)}$ \'e crescente e, portatno, BV. Para f cont\'inua sobre $\{\gamma\},$ definimos
  $$
  \int_{\gamma}^{}f|dz|:= \int_{a}^{b}f(\gamma(t))d|\gamma|(t)
  $$
  com $|\gamma|:[a, b]\rightarrow \mathbb{R}$ dada por $|\gamma|(t) = v(\gamma_t)$. Fixemos a nota\c c\~ao $\gamma_{-}:[-b, -a]\rightarrow \mathbb{C}$
  para $\gamma_{-}(t) = \gamma(-t)$ e $\gamma+c:[a, b]\rightarrow \mathbb{C}$ para $(\gamma + c)(t) = \gamma(t) + c, c\in \mathbb{C}.$
\begin{prop*}
  Se $\gamma$ \'e reific\'avel com $\{\gamma\}\subseteq{A}\subseteq{\mathbb{C}} $ e $f:A\rightarrow \mathbb{C}$ \'e cont\'inua sobre
  $\{\gamma\} $, ent\~ao, 
 \begin{itemize}
   \item[i)] $\int_{\gamma_{-}}^{}f = - \int_{\gamma}^{}f$
   \item[ii)] $\int_{\gamma+c}^{}f(z-c)dz = \int_{\gamma}^{}f$
   \item[iii)] $\biggl|\int_{\gamma}^{}f\biggr|\leq \int_{\gamma}^{}|f||dz|\leq v(\gamma)\max_{z\in \{\gamma\} }|f(z)|$
 \end{itemize}
\end{prop*}
\begin{lmm*}
  Seja $A\subseteq{\mathbb{C}}$ aberto e $\gamma:[a, b]\rightarrow A$ retific\'avel com $\{\gamma\}\subseteq{A}$. Para todo $\epsilon > 0,$
existe $\Gamma$ poligonal em A tal que 
  $$
  \biggl|\int_{\gamma}^{}f - \int_{\Gamma}^{}f\biggr| < \epsilon.
  $$
\end{lmm*}
A seguir, vemos o Teorema Fundamental das Fun\c c\~oes de Vari\'avel Complexa.
\begin{theorem*}
  Seja $A\subseteq{\mathbb{C}}$ aberto, $\gamma$ retific\'avel, $\{\gamma\}\subseteq{A}\text{ e }f:A\rightarrow \mathbb{C} $ com f 
con\t'inua em $\{\gamma\} $. Se existe $F:A\rightarrow \mathbb{C}$ tal que $F' = f,$ ent\~ao
  $$
    \int_{\gamma}^{}f = F(\gamma(b)) - F(\gamma(a)).
  $$
\end{theorem*}
\begin{proof*}
 \begin{itemize}
   \item[i)] Vamos assumir $\gamma$ suave por partes. Sem perda de generalidade, podemos assumir suave. Neste caso,
   $$
     \int_{\gamma}^{}f = \int_{a}^{b} f(\gamma(t))\gamma'(t)dt = (F\circ{\gamma})(t)\biggl|_a^b\biggr. = F(\gamma(b)) - F(\gamma(a)).
   $$
   \item[ii)] No caso geral, considere $\epsilon > 0$ e $\Gamma$ a plligonal dada pelo Lema. Temos
   $$
     \biggl|\int_{\gamma}^{}f - \int_{\Gamma}^{}f\biggr| < \epsilon, \quad\text{ \qedsymbol}
   $$
 \end{itemize}
\end{proof*}
\begin{exer*}
  Mostre que $f(z) = |z|^2$ \'e cont\'inua, mas n\~ao possui primitiva.
\end{exer*}
\begin{crl*}
  Nas condi\c c\~oes e nota\c c\~oes de TFVC, se $\gamma$ \'e fechada, ent\~ao $\int\limits_{\gamma}f = 0.$
\end{crl*}

\subsection{Vers\~ao Introdut\'oria da F\'ormula Integral de Cauchy}
  Um resultado que ser\'a usado de forma recorrente \'e o Lema de Leibniz, como enunciado a seguir
 \begin{lmm*}
   Seja $\phi:[a, b]x[c, d]\rightarrow \mathbb{C}$ cont\'inua e 
   $$
     g(t):= \int_{a}^{b}\phi(s, t)ds.
   $$
   Ent\~ao, g \'e cont\'inua. Adicionalmente, se $\frac{\partial{\phi}}{\partial{t}}$ \'e cont\'inua, ent\~ao g \'e suave e temos
   $$
    g'(t) = \int_{a}^{b}\frac{\partial{\phi}}{\partial{t}}(s, t)ds.
   $$
 \end{lmm*}
\begin{example}
  Se $|z| < 1,$
  $$
    I:= \int_{0}^{2\pi}\frac{e^{is}}{e^{is}-z}ds = 2\pi.
  $$
  Definimos $\phi:[0, 2\pi]x[0, 1]\rightarrow \mathbb{C}$ por 
  $$
    \phi(s, t) = \frac{e^{is}}{e^{is} - tz}, \quad t\in[0, 1], \quad s\in[0, 2\pi].
  $$
  Considere g como no lema e observe que $g(1) = I, g(0) = 2\pi.$ Al\'em disso,
  $$
    g'(t) = \int_{0}^{2\pi}\frac{e^{is}}{(e^{is} - tz)^{2}}ds = \frac{1}{i}\biggl(\frac{1}{e^{is}-tz}\biggr)\biggl|_0^{2\pi}\biggr. = 0
  $$
\end{example}
\newpage

\section{Aula 09 - 17/01/2023}
\subsection{Motiva\c c\~oes}
 \begin{itemize}
   \item F\'ormula Integral de Cauchy para c\'irculos;
   \item Toda fun\c c\~ao anal\'itica pode ser representada em s\'erie de pot\^encia;
   \item Zeros de fun\c c\~oes
 \end{itemize}
 \subsection{Exerc\'icio de Hoje}
 \subsubsection{Edson}
 Tome $\gamma = [1, i], \sigma = [1, 1+i, i], f(z) = |z|^2$ como as poligonais. Al\'em disso, coloque 
 $\gamma(t) = (1-t) + it, \gamma'(t) = -1 + i, t\in[0,1]$ e 
 $$
 \sigma(t) = \left\{\begin{array}{ll}
     it + 1, \quad t\in[0, 1] \\
     2 + i - t, \quad t\in[1, 2]
   \end{array}\right., \quad 
   \sigma'(t) = \left\{\begin{array}{ll}
       i, \quad t\in(0, 1) \\
       -1, \quad t\in(1, 2).
     \end{array}\right.
 $$
  Assim,
  $$
  \int_{\gamma}^{}f = \int_{0}^{1}f(\gamma(t))\gamma'(t)dt = \int_{0}^{1}|it + (1-t)|^2(i-1)dt =
  =(i-1)\int_{0}^{1}(2t^2 - 2t + 1)dt = \frac{2}{3}(i-1)
  $$
  Al\'em disso,
  $$
    \int_{\sigma}^{}f = \int_{0}^{2}f(\sigma(t))\sigma'(t)dt = i \int_{0}^{1}|it + 1|^2 dt - \int_{1}^{2}|2 + i - t|^2dt =
    =i \int_{0}^{1}(t^2 + 1) - \int_{1}^{2}((2-t)^2 + 1)dt = 1 - \frac{7}{3} + i \frac{4}{3}.
  $$

 \subsection{F\'ormula Integral de Cauchy - Vers\~ao Introdut\'oria}
\begin{prop*}
  Seja $f:G\rightarrow \mathbb{C}$ e $\overline{B(a, r)}\subseteq{G}.$ Se $\gamma(t) = a + r e^{it}, t\in[0, \pi],$ ent\~ao
  $$
    f(z) = \frac{1}{2i\pi}\int_{\gamma}^{}\frac{f(w)}{w-z}dw, \quad |z-a| < r.
  $$
\end{prop*}
 \begin{proof*}
   Suponha $\sigma(t) = e^{it}, t\in[0, 2\pi],$ ent\~ao $\gamma = a + r\sigma.$ Note que a f\'ormula buscada equivale a 
   $$
    \frac{1}{2\pi i}\int_{\gamma}^{}\frac{f(w)}{w-z}dw - f(z) = 0,
   $$
   ou seja,
   $$
    \frac{1}{2\pi}\int_{0}^{2\pi}\frac{e^{it}f(e^{it})}{e^{it}-z} - f(z) dt = 0.
   $$
   Considere $\phi(s, t) = \frac{e^{is}f(e^{is}+t(e^{is}-z))}{e^{is}-z} - f(z), t\in[0, 1], s\in[0, 2\pi]$. Tome, tamb\'em,
   $g(t) = \int\limits_{0}^{2\pi}\phi(s, t)dt$. Com isso, observe que 
  \begin{align*}
    &g(1) = \frac{1}{2\pi}\int_{0}^{2\pi}\frac{e^{it}f(e^{it})}{e^{it}-z} - f(z)dz \\
    &g(0) = \int_{0}^{2\pi}\frac{e^{is}f(z)}{e^{is}-z} - f(z) ds = f(z)\int_{0}^{2\pi}\frac{e^{is}}{e^{is}-z} - 1ds = 0.
  \end{align*}
  Temos 
  $$
    g'(t) = \int_{0}^{2\pi}\frac{e^{is}}{e^{is}-z}f'(z + t(e^{is}-z))ds = \int_{0}^{2\pi}e^{is}f'(z + t(e^{is}-z)) = 
    \frac{1}{it}f(z + t(e^{is}-z)) \biggl|_{0}^{2\pi}\biggr. = 0.
  $$
  Portanto, g \'e constante, garantindo a proposi\c c\~ao. \qedsymbol
 \end{proof*}
 \begin{lmm*}
   Seja $\gamma$ retific\'avel em $\mathbb{C},\{f_{n}\} $ sequ\^encia cont\'inua sobre $\{\gamma\} $ e f cont\'inua sobre
 $\{\gamma\}$. Se $f_{n}\to{f}$ em $\{\gamma\}$, ent\~ao
  $$
  \int_{\gamma}^{}f = \lim_{n\to\infty}\int_{\gamma}^{}f_{n}.
  $$
 \end{lmm*} 
\begin{proof*}
  Exerc\'icio.
\end{proof*}
\begin{theorem*}
  Seja f anal\'itica em B(a, r). Ent\~ao, 
  $$
  f(z) = \sum\limits_{n=0}^{\infty}a_{n}(z-a)^{n}, \quad |z-a| < r,
  $$
  em que $a_{n} = \frac{f^{(n)}(a)}{n!}, n = 0, 1, \cdots.$ Al\'em disso, f tem raio de converg\^encia $R\geq{r}.$
\end{theorem*}
\begin{proof*}
  Considere $0<\rho<r$ e suponha f anal\'itica em $\overline{B(a, \rho)}$. Pela proposi\c c\~ao, $f(z) =\displaystyle \frac{1}{2\pi i}\int_{\gamma}^{}\frac{f(w)}{w-z}dw,
\gamma_{\rho} = a + r e^{it}.$ Vamos mostrar que 
  $$  
  f(z) = \sum\limits_{n=0}^{\infty}(z-a)^n\biggl(\frac{1}{2i \pi}\int_{\gamma_{\rho}}^{}\frac{f(w)}{(w-a)^{n+1}}dw.\biggr)
  $$ 
  Com efeito, considere
  $$
  g_{n}(z) = \sum\limits_{k=0}^{n}\underbrace{\frac{f(w)(z-a)^{k}}{(w-a)^{k+1}}}_{f_{n}(z)}
  $$
  Mostremos, agora, que $\sum\limits_{k=0}^{n}f_{n}(z)$ converge uniformemente. Para $w\in{\{\gamma_{\rho}\}},$
  $$
  |f_{k}(w)| = \frac{|f(w)||(z-a)^{k}|}{|w-a|^{k+1}} < |f(w)|\frac{|z-a|^{k}}{\rho^{k+1}}\leq \max_{w\in \{\gamma\}}|f(w)|\frac{1}{\rho}\biggl(\frac{|z-a|^k}{\rho^k}\biggr)^{k}.
  $$
  Pelo working finee M de Weierstras, 
  $$
  g(z):=\sum\limits_{n=0}^{\infty}\frac{f(w)(z-a)^{n}}{(w-a)^{n+1}}
  $$
  \'e cont\'inua. Assim,
  $$
  g(z) = \frac{f(w)}{w - a}\sum\limits_{n=0}^{\infty}\biggl(\frac{z-a}{w-a}\biggr)^{n} = \frac{f(w)}{w-a}\frac{1}{1-\frac{z-a}{w-a}} = \frac{f(w)}{w-z}
  $$
  Como consequ\^encia, conseguimos calcular a integral de g na curva:
 \begin{align*}
   \frac{1}{2i \pi}\int_{\gamma_{\rho}}^{}g(w)dw &= \frac{1}{2i \pi}\int_{\gamma_{\rho}}^{}\frac{f(w)}{w-z}dw = \frac{1}{2i \pi}\lim_{n\to\infty}\int_{\gamma_{\rho}}\frac{f(w)(z-a)^{n}}{(w-a)^{n+1}}dw\\
                                                 &= \frac{1}{2\pi}\sum\limits_{n=0}^{\infty}(z-a)^{n}\underbrace{\int_{\gamma_{\rho}}^{}\frac{f(w)}{(w-a)^{n+1}}dw}_{b_{n}}
 \end{align*}
\end{proof*}
\begin{crl*}
  Se f \'e anal\'itica em B(a, R) e $\gamma$ \'e retific\'avel, $\{\gamma\}\subseteq{B(a, R)}$ e fechada, ent\~ao
  $$
    \int_{\gamma}^{}f = 0.
  $$
\end{crl*}
\begin{proof*}
  Exerc\'icio.
\end{proof*}

\subsection{Zeros de Fun\c c\~oes Anal\'iticas}
\begin{def*}
  Diremos que f \'e inteira se f \'e anal\'itica em $\mathbb{C}$.
\end{def*}
\begin{example}
  Polin\^omios s\~ao fun\c c\~oes inteiras.
\end{example}
\begin{def*}
  Se $f:G\rightarrow \mathbb{C}$ \'e uma fun\c c\~ao anal\'itica com f(a) = 0 e que existe $m\in \mathbb{N}$ tal que 
  $$
    f(z) = (z-a)^{m}g(z),
  $$
  com g anal\'itica e n\~ao-nula em a, diremos que a \'e um zero de multiplicidade m de f.
\end{def*}
  Um teorema muito relevante em FVC \'e o Teorema de Liouville:
 \begin{theorem*}
   Se f \'e inteira e limitada, ent\~ao f \'e constante.
 \end{theorem*}
\begin{proof*}
  Seja $0 < \rho < r$ e sendo f anal\'itica em $B(a, r)$. Ent\~ao, 
  $$
    |f^{(n)}(a)| = \frac{n!}{2 \pi}\biggl|\int_{\gamma_{\rho}}^{}\frac{f(w)}{(w-a)^{n+1}}dw\biggr|
\leq \frac{n!}{2\pi}\int_{\gamma_{\rho}}^{}\frac{|f(w)|}{\rho^{n+1}}|dw| \leq \frac{n!}{2\pi}\frac{Mv(\gamma_{\rho})}{\rho^{n+1}} = \frac{n!M}{\rho^{n+1}}.
  $$
  Fazendo $\rho$ tender a infinito, conclu\'imos que $|f^{(n)}(a)|\leq0,$ ou seja, $f^{(n)}(a) = 0.$
\end{proof*}
  Uma consequ\^encia simples e trivial dessa discuss\~ao toda \'e o Teorema Fundamental da \'Algebra.
\begin{theorem*}
  Todo polin\^omio complexo de grau maior que 1 possui ra\'iz complexa. 
\end{theorem*}
\begin{proof*}
  Se p n\~ao possui zeros, $\frac{1}{p}$ \'e inteira e limitada, visto que $\lim_{z\to\infty}p(z) = \infty$ e $\lim_{z\to\infty}\frac{1}{p(z)} = 0$, 
o que implica na limita\c c\~ao e, por Liouville, $\frac{1}{p}$ \'e constante, fazendo com que p seja constante, uma contradi\c c\~ao.
Portanto, p possui ao menos uma ra\'iz. \qedsymbol
\end{proof*}
\begin{crl*}
  Se p \'e um polin\^omio n\~ao identicamente nulo com zeros $a_1, \cdots, a_{n}$ de multiplicidade $m_1, \cdots, m_{n}$ respectivamente. Ent\~ao,
existe uma constante tal que 
  $$
    p(z) = (z-a_1)^{m_1}\cdots(z-a_{n})^{m_{n}}c
  $$
  e que o grau de p \'e $\sum\limits_{i=1}^{n}m_{i}$.
\end{crl*}
\newpage

\section{Aula 10 - 19/01/2023}
\subsection{Motiva\c c\~oes}
\begin{itemize}
  \item Teorema do M\'aximo M\'odulo;
  \item \'Indice de Curvas Fechadas e Dando Voltas em C\'irculos;
\end{itemize}

\subsection{Continua\c c\~ao de Zeros de Fun\c c\~oes Anal\'iticas}
\begin{prop*}
  Suponha $f:G\rightarrow \mathbb{C}$ anal\'itica na regi\~ao G. S\~ao equivalentes:
 \begin{itemize}
   \item[i)] $f\equiv0$;
   \item[ii)] Existe um a de G tal que $f^{(n)}(a) = 0, n = 0, 1, \cdots;$
   \item[iii)] O conjunto $\{z\in{G}: f(z) = 0\} = F_{0} $ possui ponto de acumula\c c\~ao.
 \end{itemize}
\end{prop*}
\begin{proof*}
 $iii)\Rightarrow ii)$ Seja a ponto de acumula\c c\~ao de $F_{0}$. Considere $\{a_{n}\} $ em $F_{0}$ tal que $a_{n}\to{a}$, f cont\'inua em a.

  Afirmamos que $f^{(n)}(a) = 0.$  Seja m tal que $f^{(m)}(a)\neq 0 $ e $f^{(n)}(a) = 0, n = 0, 1, \cdots, m-1 $. Seja r positivo
e, para z em B(a, r), temos 
  $$
  f(z) = \sum\limits_{n=m}^{\infty}a_{n}(z-a)^{n} = (z-a)^{m}\underbrace{\sum\limits_{n=m}^{\infty}a_{n}(z-a)^{n-m}}_{g(z)\text{ anal\'itica.}}.
  $$
  Temos g anal\'itica e g(a) n\~ao-nulo. Logo, existe $0 < \delta < r$ tal que g(z) \'e n\~ao-nulo para z em $B(a, \delta)$. Portanto,
  $f(z)\neq0$ para z em $B(a, r) - \{a\} $, uma contradi\c c\~ao, donde segue que a terceira afirma\c c\~ao deve implicar na segunda. \qedsymbol

  $ii)\Rightarrow i)$ Considere a tal que $f^{(n)}(a) = 0, n = 0, 1, \cdots$. Seja $A=\{z\in{G}: f^{(n)}(z) = 0, n=0, 1, \cdots\} $. Note que
a pertence a A e, al\'em disso, 
  $$
  A = \bigcap_{n=0}^{\infty}\{z\in{G}: f^{(n)}(z) = 0\}
  $$
\'e fechado. Mostremos que A \'e aberto. Tome b em A e r positivo, tal que $B(b, r)\subseteq{G}.$ Observe que 
 $$
 f(z) = \sum\limits_{n=0}^{\infty}b_{n}(z-b)^{n}, \quad z\in{B(b, r)}.
 $$
 Portanto, A = G e segue a prova. \qedsymbol
\end{proof*}
\begin{crl*}
  Com as hip\'oteses do Teorema, se f(a) = 0, ent\~o existe $n_a\in \mathbb{Z}_{+}$ tal que $f(z) = (z-a)^{n_a}g(z), z\in{G},$
 com g anal\'itica em g e $g(a)\neq0$ Em particular, os zeros de f s\~ao isolados.
\end{crl*}
\begin{theorem*}
  Seja f anal\'itica numa regi\~ao G. Se existe a em G tal que 
  $$
    |f(z)|\leq|f(a)|, \quad z\in{G},
  $$
ent\~ao f \'e constante.
\end{theorem*} 
\begin{proof*}
  Seja $r > 0$ tal que $B(a, r)\subseteq{G}.$ Mostraremos que $|f|:B(a, r)\rightarrow \mathbb{R}$ \'e constante na bola, mas $|f|\equiv|f(a)|\neq0.$
Mostrado isso, se $f = u + iv$, ent\~ao, para z em B(a, r), vale 
  $$
    f(a) = \frac{1}{2\pi i}\int_{\gamma_{v}}^{}\frac{f(w)}{w-a}dw,
  $$
  $\gamma_{v}(t) = a + re^{it}, t\in[0, 2\pi].$
  $$
    |f(a)|\leq \frac{1}{2\pi}\biggl|\int_{0}^{2\pi}\frac{f(a + re^{it})}{re^{it}}rie^{it}dt\biggr|\leq \frac{1}{2\pi}\int_{0}^{2\pi}|f(a + re^{it})|dt
  \leq |f(a)|.
  $$
  Assim, $|f(a+re^{it})| = |f(a)|.$ Como r \'e qualquer tal que $B(a, r)\subseteq{G},$ a afirma\c c\~ao segue.

  Agora, se f = u + iv, z em B(a, r),
    $$
      |f(z)| = u^{2} + v^{2} = |f(a)|\neq 0.
    $$
   \begin{exer*}
     Mostre que 
     $$
      \frac{\partial{u}}{\partial{x}} = \frac{\partial{v}}{\partial{y}} = 0,
     $$
     o que implicar\'a f'(z) = 0. \qedsymbol.
   \end{exer*}
\end{proof*}
\subsection{\'Indice de Curvas Fechadas}
\begin{example}
  Tome $\gamma(t) = a + re^{nit}, t\in[0, 2\pi], n=1, 2, \cdots$. Ent\~ao,
  $$
    \frac{1}{2\pi i}\int_{\gamma_{n}}^{}\frac{1}{w-a}dw = \frac{1}{2\pi  i}\int_{0}^{2\pi}\frac{1}{re^{nit}}rine^{rint}dt = \frac{1}{2\pi i}\int_{0}^{2\pi} indt = n.
  $$
\end{example}
\begin{prop*}
  Considere $\gamma$ retific\'avel e fechada. Ent\~ao, 
  $$
  \frac{1}{2\pi i}\int_{\gamma}^{}\frac{1}{w-a}dw\in \mathbb{Z}, \quad a\not\in \{\gamma\}.
  $$
\end{prop*}
\begin{proof*}
  Seja $\gamma$ curva $C^{1}$ por partes definida em [0, 1]. Considere $g(t) = \int_{0}^{t}\frac{\gamma'(s)}{\gamma(s) - a}ds, t\in[0, 1].$
Note que g(0) = 0 e $g(1) = \int_{\gamma}^{}\frac{1}{w-a}dw.$ Defina $h(t) = e^{-g(t)}(\gamma(t) - a), t\in[0, 1]$ e observe que
$h(0) = (\gamma(0) - a)\neq0.$ Derivando h:
  $$
  h'(t) = -g'(t)e^{-g(t)}(\gamma(t) - a) + e^{-g(t)}\gamma'(t) = 0, t\in[0, 1]. 
  $$
  Assim, $h\equiv{(\gamma(0) - a)}$. Neste caso, $h(1) = e^{-g(1)}(\gamma(0)-a) = \gamma(0) - a.$ Com isso, $e^{-g(1)} = 1.$ Portanto,
 $g(1) = 2\pi ki,$ para algum $k\in \mathbb{Z}.$ Portanto, a f\'ormula est\'a provada. \qedsymbol
\end{proof*}
\newpage

\section{Aula Gravada(ft. AraMat) - 23/01/2023}
\subsection{\'Indice de Curva Fechada - Continuando}
\begin{def*}
  Escrevemos, para o \'indice de uma curva $\gamma$ fechada e retific\'avel em a, 
  $$
  n(\gamma, a):= \frac{1}{2\pi}\int_{\gamma}^{}\frac{1}{z-a}dz, \quad a\not\in \{\gamma\}.
  $$
\end{def*}
\begin{prop*}
  Dadas curvas $\gamma, \sigma$ tais que 
  $$
    \sigma + \gamma(t)  = \left\{\begin{array}{ll}
        \gamma(2t), \quad t\in[0, \frac{1}{2}] \\
        \gamma(2t-1), \quad t\in[\frac{1}{2}, 1],
      \end{array}\right.
  $$
  segue que o \'indice das curvas satisfaz as seguintes propriedades
 \begin{itemize}
   \item[i)] $n(\gamma, a) = -n(\gamma_{-}, a)$
   \item[ii)] $n(\sigma+\gamma, a) = n(\sigma, a) + n(\gamma, a), a\not\in \{\gamma\}\cup \{\sigma\}$
 \end{itemize}
\end{prop*}
\begin{prop*}
  Seja $\gamma$ retific\'avel e fechada. Ent\~ao, $n(\gamma, \cdot):\mathbb{C}-\{\gamma\} \rightarrow \mathbb{Z}$ \'e cont\'inua.
\end{prop*}
\begin{proof*}
  Tome a fora de $\mathbb{C}-\{\gamma\}$ e $r > 0\text{ tal que } B(a, r)\subseteq{\mathbb{C}-\{\gamma\}}$. Ent\~ao, 
  \begin{align*}
    &n(\gamma, a) - n(\gamma, b) = \frac{1}{2\pi i}\int_{\gamma}^{}\frac{a - b}{(z-a)(z-b)}dz\\
    &\Rightarrow|n(\gamma, a) - n(\gamma, b)| = \frac{1}{2\pi}\biggl|\int_{\gamma}^{}\frac{a-b}{(z-a)(z-b)}dz\biggr|\leq \frac{1}{2\pi}\int_{\gamma}^{}\frac{|a-b|}{|z-a||z-b|}|dz|\\
    &\leq\frac{1}{2\pi}\sup_{\{\gamma\} }\frac{|a-b|}{|z-a||z-b|}v(\gamma)\overbrace{\to}^{b-a}0
  \end{align*}
\end{proof*}
\begin{theorem*}
  Seja $\gamma$ retific\'avel e fechada. A fun\c c\~ao \'indice \'e constante em cada componente conexa de $\mathbb{C}-\{\gamma\}.$
  Em particular, anula na componente conexa ilimitada de $\mathbb{C}-\{\gamma\} $.
\end{theorem*}
\begin{proof*}
  Seja $r > 0$ tal que $\{\gamma\}\subseteq{B(0, \frac{r}{2})}.$ Se $|a| > r,$
  $$
  |n(\gamma, a)|\leq \frac{1}{2\pi}\max_{\{\gamma\}}|z-a|^{-1}v(\gamma)\overbrace{\to}^{r\to\infty}0. \quad\text{\qedsymbol}
  $$
\end{proof*}

\subsection{Teorema e F\'ormulas Integrais de Cauchy}
\begin{lmm*}
  Seja $\gamma$ retific\'avel, $\phi$ cont\'inua em $\{\gamma\} $ e 
  $$
  F_{m}(z):= \int_{\gamma}^{}\frac{\phi(w)}{(w-z)^{m}}dw, \quad z\in \mathbb{C}-\{\gamma\}, m=1, 2, \cdots
  $$
  Ent\~ao, $F_{m}$ \'e anal\'itica com $F_{m} = mF_{m+1}, m = 1, 2, \cdots$.
\end{lmm*}
\begin{proof*}
  Note que 
  \begin{align*}
    F_{m}(a) - F_{m}(b) &= \int_{\gamma}^{}\frac{\phi(w)}{(w-a)^{m}} - \frac{\phi(w)}{(w-b)^{m}}dw \\
                        &= \int_{\gamma}^{}\phi(w)\underbrace{\biggl[\frac{1}{(w-a)^{m}} - \frac{1}{(w-b)^{m}}\biggr]}_{\sum\limits_{n=1}^{m}(w-a)^{-n}(w-b)^{-m-1+n}}dw.
  \end{align*}
  Assim,
  $$
  F_{m}'(a) = \lim_{b\to{a}}\sum\limits_{n=1}^{m}\int_{\gamma}^{}\frac{\phi(w)}{(w-a)^{n}(w-b)^{m+1-n}}dw = mF_{m+1}(a).
  $$
\end{proof*}
\begin{theorem*}
  Seja $f:G\rightarrow \mathbb{C}$ anal\'itica, $\gamma$ retific\'avel e fechada tal que $n(\gamma, a) = 0$ para todo $a\in \mathbb{C}-G.$
Ent\~ao, 
  $$
    f(b)n(\gamma, b) = \frac{1}{2\pi i}\int_{\gamma}^{}\frac{f(w)}{w-b}dw, \quad b\in{G-\{\gamma\}}.
  $$
\end{theorem*}
\begin{proof*}
  Considere $\phi:GxG\rightarrow \mathbb{C}$ dada por 
  $$
    \phi(z, w) = \left\{\begin{array}{ll}
        \frac{f(w)-f(z)}{w-z}, \quad z\neq{w} \\
        f'(z), \quad \ = w.
      \end{array}\right.
  $$
  Observe que $\phi(\cdot, w):G\rightarrow \mathbb{C}$ \'e anal\'itica (Exerc\'icio.). Al\'em disso, defina
$H:=\{w\in{\mathbb{C}-\{\gamma\}}: n(\gamma, w) = 0\}$. Assim, $\mathbb{C}-G\subseteq{H}\text{ e } \mathbb{C} = G\cup{H}$.
Coloque $g:\mathbb{C}\rightarrow \mathbb{C}$ da forma 
  $$
    g(z) = \left\{\begin{array}{ll}
        \int_{\gamma}^{}\phi(z, w)dw, \quad z\in{G} \\
        \int_{\gamma}^{}\frac{f(w)}{w-z}dw, \quad z\in{H}.
      \end{array}\right.
  $$
  Vamos mostrar que essa g est\'a bem-definida, tomando z em $H\cap{G}.$ Ent\~ao,
  $$
    \int_{\gamma}^{}\phi(z, w)dw = \int_{\gamma}^{}\frac{f(w)-f(z)}{w-z}dw = \int_{\gamma}^{}\frac{f(w)}{w-z}dw - f(z)n(\gamma, z)
  $$
  Um exerc\'icio \'e mostrar que o lema garante que g \'e anal\'itica e, logo, inteira. Al\'em disso, mostre que g \'e limitada em
bolas de $\mathbb{C}-G$ como $B(a, r)\subseteq{G}\Rightarrow |g(z)|\leq \int\limits_{\gamma}|\phi(z, w)||dw|\leq \sup\limits_{B(a, r)x \{\gamma\} }|\phi|v(\gamma)$.
Portanto, g \'e limitada. Como consequ\^encia de Liouville, g \'e constante. Mostre, por fim, que g \'e identicamente nula tal que, dado b em
 $G-\{\gamma\}$,
 $$
  g(b) = 0 = \int_{\gamma}^{}\frac{f(w) - f(b)}{w-b} = \int_{\gamma}^{}\frac{f(w)}{w-b}dw - f(b) \int_{\gamma}^{}\frac{1}{w-b}dw
 $$
 Portanto,
 $$
 2\pi if(b)n(\gamma, b) = \int_{\gamma}^{}\frac{f(w)}{w-b}dw. \quad\text{\qedsymbol}
 $$
\end{proof*}
\newpage

\section{Aula 11 - 24/01/2023}
\subsection{Motiva\c c\~oes}
\begin{itemize}
  \item Consequ\^encias da F\'ormula Integral de Cauchy;
  \item Teorema de Morera (mais uma vers\~ao do Teorema de Cauchy);
  \item Teorema de Goursat.
\end{itemize}
 \subsection{Exerc\'icios de Hoje}
 \subsubsection{Jo\~ao}
\begin{align*}
  f(a)\sum\limits_{k=1}^{n}n(\gamma_{k}, a) &= \sum\limits_{k=1}^{n}f(a)n(\gamma_{k}, a) \\
                                            &= \sum\limits_{k=1}^{n}\frac{1}{2\pi i}\int_{\gamma_{k}}^{}\frac{f(z)}{z-a}dz \\
                                            &= \frac{1}{2\pi i}\int_{\gamma}^{}\frac{f(z)}{z-a}dz.
\end{align*}

\subsection{Consequ\^encias da F\'ormula Integral de Cauchy}
  Come\c camos esta aula com os respectivos Teorema de Morera e Teorema de Goursat.
\begin{theorem*}
  Seja $f:G\rightarrow \mathbb{C}$, G aberto. Se $\int_{\Delta}^{}f = 0$ para $\Delta = [a, b, c, a]\subseteq{G},$ ent\~ao
f \'e anal\'itica.
\end{theorem*}
\begin{proof*}
  Sejam z um elemento de G e $r > 0$ tal que $B(a, r)\subseteq{G}.$ Defina
  $F_{z}:B(z, r)\rightarrow \mathbb{C}$ por $F_{z}(u) = \int_{[z, u]}^{}f dw.$
Temos,
  $$
  \frac{F_{z}(a) - F_{z}(b)}{a - b} - f(a) = \frac{1}{a-b}\biggl(\int_{[z, a]}^{}fdw + \int_{[z, b]}^{}fdw\biggr) - f(a),
  $$
de onde segue 
 \begin{align*}
   \biggl|\frac{F_{z}(a) - F_{z}(b)}{a-b} - f(a)\biggr| &= \frac{1}{|a-b|}\biggl|\int_{[b, a]}f(w) - f(a)\biggr| \\
                                                        &\leq \frac{1}{|a-b|}\int_{[b, a]}^{}|f(w) - f(a)||dw| \\
                                                        &\leq \frac{1}{|a-b|}\sup_{w\in[b, a]}|f(w) - f(a)|v([b, a)]) = \sup_{w\in[b, a]}|f(w)-f(a)|.
 \end{align*}
 Como $\sup\limits_{w\in[b, a]}|f(w) - f(a)|\to0, b\to{a},$ segue a prova. \qedsymbol
\end{proof*}
\begin{theorem*}
  Seja $f:G\rightarrow \mathbb{C}$ aberto. Se f \'e diferenci\'avel, ent\~ao f \'e anal\'itica.
\end{theorem*}
\begin{proof*}
  Verificamos que f satisfaz as hip\'oteses do Teorema de Morera. 
Sem perda de generalidade, suponha que G = B(w, r) e seja $\Delta=[a, b, c, a].$
Queremos verificar que $\int_{\Delta}^{}f = 0.$
  
  Com efeito, considere $\epsilon > 0$ e mostraremos que $|\int_{\Delta}^{}f| < \epsilon.$ Se 
  $m_{ab}, m_{ba}, m_{ca}$ s\~ao os pontos m\'edios de [a, b], [b, c], [c, a], respectivamente, 
ent\~ao
  $$
    \int_{\Delta}^{}f = \sum\limits_{j=1}^{a}\int_{\Delta_{j}}^{}f.
  $$
  Al\'em disso, assumiremos que $|\int_{\Delta_j}^{}f|$ \'e o m\'aximo de 
 $|\int_{\Delta_{j}}^{}f|, j = 1, \cdots, 4$. Assim, segue que 
  $$
  \biggl|\int_{\Delta}^{}f\biggr|\leq4 \biggl|\int_{\Delta_{1}}^{}f.\biggr|
  $$
  Seja $l(\Delta)$ o per\'imetro de $\Delta$ e $diam \Delta$. Note que 
  $$
  l(\Delta_{1}) = \frac{1}{2}l(\Delta)\quad\text{ e } diam \Delta_{1} = \frac{1}{2}diam \Delta.
  $$
  Indutivamente, consideramos $\Delta_{n}, n = 1, 2, \cdots$ tais que
  \begin{itemize}
    \item[1)] $l(\Delta_{n}) = \frac{1}{2^n}l(\Delta)\to0, n\to\infty;$
    \item[2)] $diam(\Delta_{n}) = \frac{1}{2^n}diam \Delta;$
    \item[3)] $\biggl|\int_{\Delta}^{}f\biggr|\leq 4^n\biggl|\int_{\Delta_{n}}^{}f\biggr|.$
  \end{itemize}
  Considere, agora, $F_{n}$ sendo o fecho do tri\^angulo fechado. Temos 
  $F_{n+1}\subseteq{F_{n}}, n=1, 2, \cdots.$ Logo, pelo Teorema de Cantor, $\bigcap_{n=1}^{\infty}F_{n}=\{u\}. $
  Dado $\epsilon > 0,$ seja $\delta > 0$ tais que 
  $$
    |f(w) - f(u) - f'(u)(w-u)| < \epsilon|w-u|.
  $$
  Se n \'e tal que $diam \Delta_{n} < \delta,$ ent\~ao
  $$
    \int_{\Delta_{n}}^{}f = \int_{\Delta_{n}}^{}(f(z) - f(u)) - f'(u)(z-u)dz.
  $$
  Assim, $|\int_{\Delta_{n}}^{}f|\leq \int_{\Delta_{n}}^{}\epsilon|z-u|dz\leq \epsilon\sup_{z\in{\Delta_{n}}}|z-u|l(\Delta_{n}).$
  Portanto, $|\int_{\Delta}^{}f \leq \epsilon l(\Delta) diam(\Delta).|$ \qedsymbol
\end{proof*}
\newpage

\section{Aula 12 - 26/01/2023}
\subsection{Motiva\c c\~oes}
\begin{itemize}
  \item Homotopia e curvas;
  \item Vers\~ao Homot\'opica do Teorema de Cauchy;
  \item Exist\^encia de Primitivas para Fun\c c\~oes Anal\'iticas 
\end{itemize}

\subsection{Exerc\'icios do Dia}
\subsubsection{Francisco Jonat\~a}
Seja P(z) de grau n e considere $\{z:|z|\geq R\}, \gamma(t) = Re^{it}.$ Ent\~ao, P(z) pode ser escrito como 
  $$
    P(z) = c(z-r_{1})(z-r_2)\cdots(z-r_{n}),\quad P'(z) = c(z-r_2)\cdots(z-r_{n}) + c(z-r_{1})\cdots(z-r_{n}) + \cdots + c(z-r_1)\cdots(z-r_{n-1})
  $$
  Assim,
  $$
  \int_{\gamma}^{}\frac{P'(z)}{P(z)}dz = \int_{\gamma}^{}\sum\limits_{k=1}^{n}\frac{1}{z-r_{k}}dz = \sum\limits_{k=1}^{n}\int_{\gamma}^{}\frac{1}{z-r_{k}}dz = \sum\limits_{k=1}^{n}n(\gamma, r_{k})2\pi i = n2\pi i.\quad\text{ \qedsymbol}
  $$

 \subsubsection{Gabriel Passareli}
 Seja $\gamma:[0, 1]\rightarrow \mathbb{C}, \gamma(0) = 1, \gamma(1) = w.$ Existe um k inteiro tal que $\int_{\gamma}^{}\frac{1}{z}dz = \ln{(r)} + i\theta + 2k\pi i, w=re^{i\theta}.$
Considere 
  $$
  \{\tilde\gamma\} = \{\gamma\}\cup{[w, 1]}.
  $$
Da defini\c c\~ao, 
  $$
  2\pi i = n(\tilde \gamma, 0) = \int_{\gamma}^{}\frac{1}{z}dz = \int_{\gamma}^{}\frac{1}{z}dz + \int_{[w, 1]}^{}\frac{1}{z}dz.
  $$
  Tome
  $$
    G = \{z\in \mathbb{C}: d(z, [w, 1]) < \epsilon\}.
  $$
  Sobre G, tome tamb\'em
  $$
  \ln{(re^{i \theta})} = \ln{(r)} + i \theta, \quad -\pi < \theta < \pi.
  $$
  Ent\~ao, pelo TFVC,
  $$
  \int_{\gamma}^{}\frac{1}{z}dz = -k\pi i + \int_{[w, 1]}^{}\frac{1}{z}dz. \quad\text{ \qedsymbol}
  $$

  \subsection{Vers\~ao Homot\'opica do Teorema de Cauchy}
  Escreveremos I = [0, 1] para o intervalo mencionado. Seja G uma regi\~ao e $\gamma_{1}, \gamma_{2}$ curvas tais que existe
 $\Gamma:I^{2}\rightarrow G$ cont\'inuo tal que 
  $$
    \left\{\begin{array}{ll}
        \Gamma(s, 0) = \gamma_{1}(s)\\
        \Gamma(s, 1) = \gamma_{2}(s)\\
        \Gamma(0, t) = \gamma_{1}(0) = \gamma_{2}(0)\\
        \Gamma(1, t) = \gamma_{1}(1) = \gamma_{2}(1).
      \end{array}\right.
  $$
  Ent\~ao, diremos que $\gamma_{1}$ \'e homot\'opica \`a $\gamma_{2}$ e escreveremos $\gamma_{1}\sim\gamma_{2}.$ Fica de exerc\'icio
mostrar que $\sim$ \'e uma rela\c c\~ao de equival\^encia.
 \begin{theorem*}
   Sejam $\gamma_{1}, \gamma_{2}$ retific\'aveis na regi\~ao G. Se $\gamma_{1}\sim \gamma_{2},$ ent\~ao
   $$
    \int_{\gamma_{1}}f = \int_{\gamma_{2}}f
   $$
  para toda f anal\'itica em G. 
 \end{theorem*}
\begin{proof*}
  O caso em que as duas curvas s\~ao n\~ao fechadas fica como exerc\'icio. Destarte, suponha $\gamma_{1}, \gamma_{2}$ fechadas. 
Supondo que $\Gamma\in{C^{2}}$, temos
  $$
    \int_{\gamma_{1}}^{}f = \int_{0}^{1}f(\Gamma(s, 0))\frac{\partial{\Gamma}}{\partial{s}}(s, 0)dS
  $$
  e
  $$
    \int_{\gamma_{t}}^{}f = g(t), t\in[0, 1], \quad \gamma_{t}(s) = \Gamma(s, t).
  $$
  Temos
  $$
    g'(t) = \frac{\partial{}}{\partial{t}}\biggl(\int_{0}^{1}f(\Gamma(s, t))\frac{\partial{\Gamma}}{\partial{s}}(s, t)ds\biggr) = \biggl(f(\Gamma(s,t ))\frac{\partial{\Gamma}}{\partial{t}}(s, t)\biggr)\biggl|_0^1\biggr. = 0.
  $$
  Agora, no caso geral, suponha $\Gamma(I^2)$ completo, $\Gamma$ \'e uniformemente cont\'inua. Coloque $\gamma_{t}(s):= \Gamma(s, t), s\in{I}.$
 \begin{itemize}
 \item[1)] Existe $\epsilon > 0$ tal que $B(z, 13 \epsilon)\subseteq{G},\forall z\in{\Gamma(I^2)};$
 \item[2)] Existe $\gamma > 0$ tal que $|s_1 - s_2| < \delta\Rightarrow |\Gamma(s_1, t) - \Gamma(s_2, t)| < 2 \epsilon.$ 
 \end{itemize}
 Mostremos que $\int\limits_{\Gamma(\cdot, s_1)}^{}f = \int_{\Gamma(\cdot, s_2)}^{}f.$ Fica como exerc\'icio, tamb\'em, particionar
 I como $P=\{s_{0}, \cdots, s_{n}\}.$ de forma que $|P| < \delta.$ Com efeito, considere $z_{0}, \cdots, z_{n}, w_{1}, \cdots, w_{n}$ e bolas
 $B_{i}, i = 1, 2, \cdots, n$ centradas em $\Gamma(I^2)$ de raio $\epsilon$ com $z_{i}, w_{i}, z_{i+1}, w_{i+1}\in{B_{i}}, i = 1, \cdots, n-1.$
 Temos 
 $$
    g'(t) = \frac{\partial{}}{\partial{t}}\biggl(\int_{0}^{1}f(\Gamma(s, t))\frac{\partial{}}{\partial{t}}(s, t)ds\biggr) = \biggl(f(\Gamma(s, t))\frac{\partial{\Gamma}}{\partial{t}}(s, t)\biggr)\biggl|_0^1\biggr. = 0.
 $$
 Para cada i, f \'e primitiv\'avel em $B_{i}.$ Neste caso, se $F_{i}$ \'e a primitiva de f em $B_{i}$, ent\~ao
 $$
    F_{i} - F_{i+1} = c \in B_{i}\cap{B_{i+1}}, \quad c\in \mathbb{C}, i=1, \cdots, n=1.
 $$
 Logo,
 $$
    F_{i}(z_{i+1}) - F_{i+1}(z_{i+1}) = F_{i}(w_{i+1}) - F_{i+1}(w_{i+1}).
 $$
 Consequentemente, 
 $$
  F_{i}(z_{i+1}) - F_{i}(w_{i+1}) = F_{i+1}(z_{i}) - F_{i+1}(w_{i+1}).
 $$
 Finalmente, 
\begin{itemize}
  \item[I)] $\int\limits_{\Gamma(\cdot, s_{1})}^{}f = \sum\limits_{i=1}^{n-1}F_{i}(z)\biggl|_{z_{i}}^{z_{i+1}}\biggr.$
  \item[II)] $\int\limits_{\Gamma(\cdot, s_2)}^{}f = \sum\limits_{i=1}^{n-1}F_{i}(z)\biggl|_{w_{i}}^{w_{i+1}}\biggr.$
\end{itemize}
Subtraindo, obtemos
  $$
  \sum\limits_{i=1}^{n-1}\biggl[F_{i}(z_{i+1}) - F_{i}(z_{i}) - \biggl(F_{i}(w_{i+1}) - F_{i}(w_{i})\biggr)\biggr] = 0.\quad\text{ \qedsymbol}
  $$
\end{proof*}
\begin{exer*}
  Sejam $\gamma_{1}(t) = e^{2\pi it}, \gamma_{2}(t) = e^{-2\pi it}, t\in[0, 1]$. Mostre que $\gamma_{1}\sim \gamma_{2}.$
\end{exer*}
  Escrevemos $\gamma\sim0$ se $\gamma$ \'e homot\'opica a uma curva constante. Al\'em disso, dada uma regi\~ao G, diremos que
 G \'e simplesmente conexa se $\gamma\sim0$ para toda curva em G.
\begin{exer*}
  A regi\~ao G = B(a, r) \'e simplesmente conexa.
\end{exer*}
\begin{theorem*}
  Toda fun\c c\~ao anal\'itica em uma regi\~ao simplesmente conexa possui uma primitiva.
\end{theorem*}
\begin{proof*}
  Seja $a\in{G}$ e, para cada z em G, seja $\gamma_{z}$ retific\'avel tal que $\gamma_{z}(0) = a, \gamma_{z}(1) = z.$ Defina 
 $F:G\rightarrow \mathbb{C}$ por 
 $$
  F(z) = \int_{\gamma_{z}}^{}fdw.
 $$
 Note que se $\sigma_{2}$ \'e outra curva tal que $\sigma_{z}(0) = a$ e $\sigma_{z}(1) = z$, ent\~ao $\sigma_{z} - \gamma_{z}$
\'e fechada, fazendo com que sejam homot\'opicas a 0 e 
 $$
  \int_{\sigma_{z}-\gamma_{z}}^{}f = 0\Rightarrow \int_{\gamma_{z}}^{}f = \int_{\sigma_{z}}^{}f.
 $$
 Observe que 
 $$
 \frac{F(z+h) - F(z)}{h} = \frac{1}{h}\int_{\gamma_{z+h}}^{}f - \int_{\gamma_{z}}^{}f = \frac{1}{h}\int_{[z, z+h]}^{}f = \frac{1}{h}\int_{0}^{1}f(zt + (1-t)(z+h))-hdt
 $$
\end{proof*}
\newpage

\section{Aula 13 - 27/01/2023}
\subsection{Motiva\c c\~oes}
 \begin{itemize}
  \item Vers\~ao de M\'ultiplos Zeros da Integral de Cauchy;
  \item Teorema da Aplica\c c\~ao Aberta.
 \end{itemize}
 
 \subsection{Teorema Integral de Cauchy com V\'arios Zeros}
\begin{theorem*}
  Seja f anal\'itica na regi\~ao G com zeros $a_{1}, \cdots, a_{m}.$ Se $\gamma$ \'e retific\'avel e fechada \'e tal que $\gamma\sim0$
  e $a_{i}\not\in \{\gamma\}, i = 1, \cdots, m, $ ent\~ao
  $$
    \sum\limits_{k=1}^{m}n(\gamma, a_{k}) = \frac{1}{2\pi i}\int_{\gamma}^{}\frac{f'(z)}{f(z)}dz
  $$
\end{theorem*}
\begin{proof*}
  Podemos escrever 
  $$
    f(z) = (z-a_{1})\cdots(z-a_{m})g(z),
  $$
para alguma g anal\'itica tal que $g(z)\neq0$ para qualquer z em G. Com isso, $\displaystyle \frac{g'(z)}{g(z)}$ anal\'itica em G;
ja que $\gamma\approx0$, o Teorema de Cauchy faz com que 
  $$
    \int_{\gamma}^{}\frac{g'(z)}{g(z)}dz = 0.
  $$
  Assim, segue de 
  $$
    \frac{f'(z)}{f(z)} = \frac{1}{z-a_{1}} + \frac{1}{z-a_{2}} + \cdots + \frac{1}{z-a_{m}} + \frac{g'(z)}{g(z)},\quad z\neq a_{1}, \cdots, a_{m},
  $$
  que 
  $$
  \int_{\gamma}^{}\frac{g'(z)}{g(z)}dz = \int_{\gamma}^{}\frac{1}{z-a_{1}} + \int_{\gamma}^{}\frac{1}{z-a_{2}} + \cdots + \int_{\gamma}^{}\frac{1}{z-a_{m}} - \int_{\gamma}^{}\frac{f'(z)}{f(z)}.
  $$
  Portanto, 
  $$
  \frac{1}{2\pi i}\int_{\gamma}^{}\frac{f'(z)}{f(z)} = \int_{\gamma}^{}\frac{1}{z-a_{1}} + \int_{\gamma}^{}\frac{1}{z-a_{2}} + \cdots + \int_{\gamma}^{}\frac{1}{z-a_{m}} = \sum\limits_{k=1}^{m}n(\gamma, a_{k}).\text{\qedsymbol}
  $$
\end{proof*}
\begin{prop*}
  Seja f anal\'itica na regi\~ao G, $\gamma$ retific\'avel e fechado em G com $n(\gamma, z)\in \{0, 1\}, z\in{G}.$ Se $n(\gamma, a) = 1$
e $\alpha:= f(a)\not\in \{\sigma:= f\circ{\gamma}\} $, ent\~ao $f-\beta$ possui pelo menos uma ra\'iz para todo $\beta$ na mesma componente
de $\alpha$ em $f(G)\cap{\mathbb{C}-\{\sigma\}}$.
\end{prop*}
\begin{proof*}
  Seja $\beta$ como na proposi\c c\~ao. Ent\~ao,
  $$
    n(\sigma, \alpha) = n(\sigma, \beta).
  $$
  Vamos assumir que $\gamma\in{C^{1}}$. Neste caso, temos duas igualdades relevantes:
  \begin{align*}
    &n(\sigma, \alpha) = \frac{1}{2\pi i}\int_{\sigma=f\circ{\gamma}}^{}\frac{1}{w-\alpha}dw = \frac{1}{2\pi i}\int_{0}^{1}\frac{1}{f(\gamma(t))-\alpha}f'(\gamma(t))d\gamma(t) = \sum\limits_{k=1}^{m}n(\gamma, a_{k}(\alpha)). \\
    &n(\sigma, \alpha) = n(\sigma, \beta) = \sum\limits_{k=1}^{m}n(\sigma, a_{k}(\beta)),
  \end{align*}
  em que $a_{k}(\alpha)$ \'e zero de $f-\alpha.$ \qedsymbol
\end{proof*}
\subsection{Teorema da Aplica\c c\~ao Aberta}
  \begin{theorem*}
    Seja G regi\~ao e f anal\'itica em G. Ent\~ao, f \'e aberat, ou seja, f(A) \'e aberto para todo aberto em G. 
  \end{theorem*}
 \begin{proof*}
   Seja $A\subseteq{G}, \alpha:=f(a)\in{f(A)}.$ Afirmamos que se $f:B(a, r)\rightarrow \mathbb{C}$ e $\alpha:=f(a)$ \'e zero de 
multiplicidade m de f-a, ent\~ao existem $\epsilon, \delta > 0$ tais que 
  $$
  \beta\in{B(\alpha, \delta)- \{\alpha\} }\Rightarrow f-\beta\text{ possui m zeros simples em }B(a, \epsilon).
  $$
  Com esta afirma\c c\~ao, tome $\epsilon > 0$ tal que $B(a, \epsilon)\subseteq{A}$ e seja $\delta > 0$ como na afirma\c c\~ao.
Note que
  $$
    B(\alpha, \delta)\subseteq{f(B(a, \epsilon))}\subseteq{f(A)}.
  $$
  Para provar a afirma\c c\~ao, suponha que existe $\epsilon > 0$ tal que $f - \alpha$ n\~ao possua zeros em $B(a, \epsilon)$. Considere
  $\gamma(t) = a + \epsilon e^{2\pi it}, t\in[0, 1].$ Al\'em disso, existe $\gamma > 0$ tal que $B(\alpha, \delta)\cap \{\sigma:=f\circ{\gamma}\}=\emptyset$.
  Dado $\beta\in{B(a, \delta)}$ pela proposi\c c\~ao anterior
  $$
  n(\gamma, \beta) = n(\gamma, \alpha) = \sum\limits_{k=1}^{m}n(\gamma, a_{k}(\alpha)) = m.\quad\text{ \qedsymbol}
  $$
 \end{proof*}
\newpage
\section{COMENT\'ARIO \`A PARTE}

\paragraph{}  A partir desse momento, as aulas s\~ao dadas pelos alunos atrav\'es da forma de Semin\'arios. Neste prisma, o cr\'edito ser\'a
devidamente dado a cada grupo e, caso n\~ao ocorra isso, por favor, me contate para que possa ser corrigido. Ademais, os cr\'editos
tamb\'em devem ser dados \`a professora Tha\'is pela organiza\c c\~ao e orienta\c c\~ao.

\section{Aula 14 - 30/01/2023 - Pedro Rangel, Renan Wenzel, Roberta Agnes Mendes Melo.}
\end{document}


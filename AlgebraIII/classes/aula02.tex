\documentclass[../algebraIII_notes.tex]{subfiles}
\begin{document}
\section{Aula 02 - 13 de Março, 2025}
\subsection{Motivações}
\begin{itemize}
	\item Grau de Extensão de Corpos.
\end{itemize}
\subsection{Introdução à Teoria de Galois - Parte 2}
Ao longo do curso, utilizaremos o seguinte recorrentemente: se \(\mathbb{F}\) é um corpo, o anel \(\mathbb{F}[x]\) é um domínio de ideais principais (D.I.P.), ou seja, dado um ideal \(\mathfrak{i}\trianglelefteq \mathbb{F}[x]\), então
\[
	\mathfrak{i} = \langle f(x) \rangle = \{f(x)g(x): g(x)\in \mathbb{F}[x]\}.
\]
Quando esta f(x) é irredutível, o ideal gerado por ela é maximal e, em particular,
\[
	\mathbb{E} \coloneqq \frac{F[x]}{\langle f(x) \rangle}
\]
será um corpo. Quase todos, se não todos, os exemplos deste curso serão desta forma; se não, serão isomorfos a um corpo desta forma. Conseguimos algumas coisas bem legais com isso; por exemplo, a função
\begin{align*}
	\varphi : & \mathbb{F}\rightarrow \mathbb{E}                 \\
	          & a\mapsto \overline{a} = a + \langle f(x) \rangle
\end{align*}
é um monomorfismo. Mostraremos isto exibindo que se \(\varphi (a) = 0\), então a também deve ser nula. Com efeito,
\[
	\varphi (a) = 0 \Rightarrow \overline{a} = 0 \Rightarrow a\in \langle f(x) \rangle.
\]
Por pertencer ao ideal gerado por \(f(x)\), a será divisível por f(x), o que será possível apenas se a = 0. Além disso,
\[
	\mathrm{Im}(\varphi ) = \{\overline{a}: a \in \mathbb{F}\}
\]
é um subcorpo de \(\mathbb{E}\). Quando encontramos um monomorfismo desse estilo, dizemos que ele é um \textbf{mergulho}. Nesse caso, então, \(\mathbb{F}\overbracket[0pt]{\hookrightarrow}^{\varphi }\mathbb{E}\) determina um mergulho de \(\mathbb{F}\) em \(\mathbb{E}\).
O que nós ganhamos com isso é a possibilidade de enxergar \(\mathbb{F}\) como um subcorpo de \(\mathbb{E}\), e conseguimos estudar o \textbf{grau de extensão} de \(\mathbb{F}\) por meio da proposição
\begin{prop*}
	Se \(\mathbb{F}\) é subcorpo de \(\mathbb{E}\coloneqq \frac{F[x]}{\langle f(x) \rangle}\), então
	\[
		[\mathbb{E}:\mathbb{F}] = \mathrm{dim}_{\mathbb{F}}\mathbb{E} = \mathrm{deg}(f).
	\]
\end{prop*}
\begin{proof*}
	Tome um elemento b de \(\mathbb{E}\) tal que
	\[
		b = \overline{g(x)} = g(x) + \langle f(x) \rangle.
	\]
	Podemos dividir g(x) por f(x) usando o algoritmo de divisão de Euclides: existem \(q(x)\) e \(r(x)\) em \(\mathbb{F}[x]\) tais que
	\[
		g(x) = q(x)f(x) + r(x),\quad 0\leq \deg{(r(n))} < \deg{(f)}.
	\]
	Logo,
	\[
		b = \overline{g(x)} = \overline{q(x)f(x) + r(x)} = \overline{q(x)f(x)} + \overline{r(x)} = \overline{r(x)}.
	\]
	Seja, agora, n o grau de f(x) e
	\[
		r(x) = c_{0} + c_{1}x + \dotsc + c_{r-1}x^{n-1}
	\]
	para alguns \(c_{i}\in \mathbb{F}[x]\), determinando um polinômio de grau menor que f. Com isso, descobrimos que podemos representar b por
	\begin{align*}
		b = \overline{r} & = \overline{c_{0}+ c_{1}x + \dotsc + c_{n-1}x^{n-1}}                                              \\
		                 & = \overline{c_{0}}+\overline{c_{1}}\overline{x} + \dotsc + \overline{c_{n-1}}\overline{x}^{n-1}   \\
		                 & = c_{0}+c_{1}\overline{x} + \dotsc + c_{n-1}\overline{x}^{n-1}                                  .
	\end{align*}
	Em outras palavras, b é uma combinação linear de elementos do conjunto
	\[
		\{\overline{1}, \overline{x}, \overline{x}^{2}, \dotsc , \overline{x}^{n-1}\},
	\]
	isto é, como \(\mathbb{F}\)-espaço vetorial, \(\mathbb{E}\) é gerado por \(\{\overline{1}, \dotsc , \overline{x}^{n-1}\}\), o que implica na seguinte desigualdade:
	\[
		[\mathbb{E}:\mathbb{F}] = \mathrm{dim}_{\mathbb{F}}(\mathbb{E})\leq |\{\overline{1}, \dotsc , \overline{x}^{n-1}\}| = n = \deg{(f)}.
	\]

	Para finalizar, falta mostrar que este conjunto, \(\{\overline{1}, \overline{x}, \dotsc \overline{x}^{n-1}\}\) é linearmente independente. Sendo assim, vamos encurtar a notação colocando \(d_{i}=\overline{d_{i}}\) seja
	\[
		d_{0}\overline{1} + d_{1}\overline{x} + \dotsc + d_{n-1}\overline{x}^{n-1} = \overline{0}.
	\]
	Como consequência,
	\[
		\overline{d_{0} + d_{1}x + \dotsc +d_{n-1}x^{n-1}} = \overline{0} \Rightarrow d_{0} + d_{1}x + \dotsc +d_{n-1}x^{n-1}\in \langle f(x) \rangle.
	\]
	Assim como antes, isto significa que f(x) divide este polinômio:
	\[
		f(x)|d_{0} + d_{1}x + \dotsc +d_{n-1}x^{n-1},
	\]
	o que, como o grau de f é maior, a única possibilidade é que
	\[
		d_{0} + d_{1}x + \dotsc +d_{n-1}x^{n-1} = 0.
	\]
	Para que isto seja possível, é necessário que
	\[
		d_{0} = d_{1} = \dotsc = d_{n-1} = 0.
	\]

	Portanto, \(\{\overline{1}, \overline{x}, \dotsc , \overline{x}^{n-1}\}\) define uma base para \(\mathbb{E}\) como \(\mathbb{F}\)-espaço vetorial.
\end{proof*}
Nesta proposição, usamos o fato que todo polinômio divide 0, e que se um primeiro polinômio divide um segundo, o grau deste segundo deve ser maior que o do primeiro; caso contrário, o segundo polinômio será nulo. Isto aconteceu tanto com a, como um polinômio de grau 1, logo no começo da aula, quanto com a combinação linear para demonstração de ser linearmente independente.

Sob as mesmas condições de \(\mathbb{F}\) e \(\mathbb{E}\) que colocamos, tome \(g(x) = a_{0}+a_{1}x + \dotsc + a_{m}x^{m}\), tal que
\[
	\overline{g(x)} = a_{0} + a_1\overline{x}+\dotsc +a_{m}\overline{x}^{m}
\]
e seja \(\alpha  = \overline{x}.\) Então, todo elemento de \(\mathbb{E}\) é da forma \(h(\alpha )\), em que h(x) é um polinômio com coeficientes em \(\mathbb{F}\); devido a isso, denotamos \(\mathbb{E}\) por \(\mathbb{F}[\alpha ]\):
\[
	\mathbb{E} = \mathbb{F}[\alpha ] = \{h(\alpha )\in \mathbb{E}:\: h(x)\in \mathbb{F}[x]\}.
\]

Considere o anel dos polinômios sobre \(\mathbb{E}\), \(\mathbb{E}[y]\), que será um D.I.P, e pegue um elemento \(f(Y)\) em \(\mathbb{E}[y]\). Ele terá, então, a forma
\[
	f(y) = e_{0} + e_{1}y + \dotsc + e_{n}y^{n},
\]
de forma que, para o \(\alpha \in \mathbb{E}\) de antes, teremos
\[
	f(\alpha ) = f(\overline{x}) = \overline{f(x)} = 0.
\]
Por conta disso, \(\alpha \) pode fazer parte de uma fatoração de \(f(y)\) por meio do algoritmo de divisão:
\[
	f(y) = (y-\alpha )h(y),\quad h(y)\in \mathbb{E}[x].
\]
\begin{exr}
	Seja \(\mathbb{F}\) um corpo, \(\alpha \) um elemento dele e f(x) um polinômio tendo \(\alpha \) como raiz. Então,
	\[
		f(x) = (x-\alpha )h(x)
	\]
	por um \(h(x)\in \mathbb{F}[x]\).
\end{exr}
\begin{example}
	\begin{itemize}
		\item[1)] Em \(\mathbb{F} = \mathbb{Q}\), considere o polinômio
		      \[
			      f(x) = x^{2} - 3\in \mathbb{Q}[x],
		      \]
		      que é irredutível (não tem raiz racional). Aqui, então,
		      \[
			      \mathbb{E}=\frac{\mathbb{Q}[x]}{\langle x^{2}-3 \rangle} = \mathbb{F}[\alpha ],
		      \]
		      onde \(\alpha \) é raiz de \(x^{2}-3\) em \(\mathbb{E}\). Com isso,
		      \[
			      [\mathbb{E}:\mathbb{Q}] = \mathrm{deg}(x^{2}-3) = 2.
		      \]
		      Vale, também, que uma base da extensão \(\mathbb{E}/\mathbb{Q}\) é \(\{1, \alpha \}\); então,
		      \[
			      \mathbb{E} = \{a_{0}1 + a_{1}\alpha : \; a_{0}, \; a_{1}\in \mathbb{Q}\}.
		      \]

		      Considere uma raiz de \(x^{2}-3\) em \(\mathbb{C}\), como \(\sqrt[]{3}\). A partir dela, podemos construir um anel e um isomorfismo, fazendo
		      \[
			      \mathbb{E}' = \{a_{0} + a_{1}\sqrt[]{3}:\; a_{0}, \; a_{1}\in \mathbb{Q}\}\subseteq \mathbb{C}
		      \]
		      e
		      \begin{align*}
			      \varphi: & \mathbb{E}\rightarrow \mathbb{E}'                    \\
			               & a_{0} + a_{1}\alpha \mapsto a_{0} + a_{1}\sqrt[]{3}.
		      \end{align*}
		      Como esta \(\varphi \) é isomorfismo, segue que \(\mathbb{E}'\) é um corpo - conseguimos outra forma de construí-los. Normalmente, denotamos este tipo de corpos por
		      \[
			      \mathbb{Q}(\sqrt[]{3})\coloneqq \frac{\mathbb{Q}[x]}{\langle x^{2}-3 \rangle} = \{a_{0} + a_{1}\sqrt[]{3}:\; a_{0}, \; a_{1}\in \mathbb{Q}\}.
		      \]
		      Por ser um corpo, sabemos que ele terá um elemento inverso sempre que o elemento for diferente de zero. Qual será a forma dele? Dá pra encontrar fazendo
		      \begin{align*}
			                  & (a_{0} + a_{1}\sqrt[]{3})(a_{0} - a_1\sqrt[]{3}) = a_{0}^{2} - 3 a_{1}^{2}\neq 0                                          \\
			      \Rightarrow & (a_{0} + a_{1}\sqrt[]{3})^{-1} = \frac{a_{0}}{a_{0}^{2} - 3 a_{1}^{2}} - \frac{a_{1}}{a_{0}^{2} - 3 a_{1}^{2}}\sqrt[]{3}.
		      \end{align*}
		\item[2)] Agora, tome \(\mathbb{F}\) como sendo os racionais gaussianos, \(\mathbb{F} = \mathbb{Q}[i]\). Aqui,
		      \[
			      f(x) = x^{2}-3\in \mathbb{F}[x]
		      \]
		      é um polinômio irredutível também. Deste jeito, podemos considerar
		      \[
			      \mathbb{E} = F[\alpha ] = \mathbb{Q}[i][\alpha ], \quad \alpha =\overline{x}.
		      \]
		      Vale que \([\mathbb{E}:\mathbb{Q}[i]] = 2\). Além disso, temos
		      \[
			      \mathbb{Q}\subseteq \mathbb{Q}[i]\subseteq \mathbb{E},
		      \]
		      em que ambas as partes da cadeia de subconjuntos têm grau de extensão 2 e \(\mathbb{Q}\) é subcorpo de \(\mathbb{E}\). Como consequência,
		      \[
			      [\mathbb{E}:\mathbb{Q}] = 2 \cdot 2 = 4.
		      \]
		      Generalizamos isto a seguir!
	\end{itemize}
\end{example}
\begin{prop*}
	Seja \(\mathbb{F}\subseteq \mathbb{E}\subseteq \mathbb{K}\) uma cadeia de corpos. Então,
	\[
		[\mathbb{K}:\mathbb{F}] = [\mathbb{K}:\mathbb{E}] \cdot [\mathbb{E}:\mathbb{F}].
	\]
\end{prop*}
\begin{exr}
	Demonstre a proposição acima. Dica: pegue uma base para \(\mathbb{E}/\mathbb{F}\), outra para \(\mathbb{K}/\mathbb{E}\) e mostre que
	\[
		\{\alpha_{i}\beta_{j}:\; i\in I, \; j\in J\}
	\]
	é uma base para \(\mathbb{K}/\mathbb{F}\).
\end{exr}
\begin{example}
	Agora, tome como exemplo o corpo finito de tamanho 5, ou seja, \(\mathbb{F} = \mathbb{F}_{5} = \{0, 1, 2, 3, 4\}\), e considere um polinômio \(f(x)\in \mathbb{F}_{5}\) irredutível de grau 2, tal como \(f(x) = x^{2} + 2\) (cheque que é!). Colocando
	\[
		\mathbb{E} = \frac{\mathbb{F}_{5}[x]}{\langle x^{2}+2 \rangle},
	\]
	temos \([\mathbb{E}:\mathbb{F}_{5}] = 2\), tal que
	\[
		\mathbb{E} \cong \mathbb{F}_{5}\times \mathbb{F}_{5} \Rightarrow |\mathbb{E}| = 5^{2} = 25.
	\]
\end{example}
\begin{exr}
	Seja \(\mathbb{F}\) um corpo e \(f(x)\in \mathbb{F}[x]\) com \(\deg{(f(x))}\leq 3\). Então, f(x) é irredutível se, e somente se, ela não tem nenhuma raiz em \(\mathbb{F}.\)
\end{exr}

\end{document}

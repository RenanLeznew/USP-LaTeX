\documentclass{article}
\usepackage{bookmark}
\usepackage{amsmath}
\usepackage{amsthm}
\usepackage{amssymb}
\usepackage{pgfplots}
\usepackage[utf8]{inputenc}
\usepackage{amsfonts}
\usepackage[margin=2.5cm]{geometry}
\usepackage{graphicx}
\usepackage[export]{adjustbox}
\usepackage{fancyhdr}
\usepackage[portuguese]{babel}
\usepackage{hyperref}
\usepackage{lastpage}
\usepackage{mathtools}
\usepackage[T1]{fontenc}
\usepackage[sfdefault,scaled=.85]{FiraSans}
\usepackage{newtxsf}
\setcounter{section}{-1}

\pagestyle{fancy}
\fancyhf{}

\pgfplotsset{compat = 1.18}

\hypersetup{
   colorlinks,
   citecolor=black,
   filecolor=black,
   linkcolor=black,
   urlcolor=black
}
\newtheorem*{def*}{\underline{Defini\c c\~ao}}
\newtheorem*{theorem*}{\underline{Teorema}}
\newtheorem*{lemma*}{\underline{Lema}}
\newtheorem*{prop*}{\underline{Proposi\c c\~ao}}
\newtheorem{example}{\underline{Exemplo}}
\newtheorem*{proof*}{\underline{Prova}}
\renewcommand\qedsymbol{$\blacksquare$}
\newcommand{\Lin}[1]{Lin_{\mathbb{K}}({#1})}

\rfoot{P\'agina \thepage \hspace{1pt} de \pageref{LastPage}}

\begin{document}
 \begin{figure}[ht]
  \minipage{0.76\textwidth}
    \includegraphics[width=4cm]{icmc.png}
    \hspace{7cm}
    \includegraphics[height=4.9cm,width=4cm]{brasao_usp_cor.jpg}
  \endminipage  
\end{figure}

\begin{center}
  \vspace{1cm}
  \LARGE
  UNIVERSIDADE DE S\~AO PAULO

  \vspace{1.3cm}
  \LARGE
  INSTITUTO DE CI\^ENCIAS MATEM\'ATICAS E COMPUTACIONAIS - ICMC

  \vspace{1.7cm}
  \Large
  \textbf{Notas de Física II}

  \vspace{1.3cm}
  \large
  \textbf{Renan Wenzel - 11169472}

  \vspace{1.3cm}
  \large
  \textbf{Professor - Luis Gustavo Marcassa}

  \textbf{E-mail: marcassa@ifsc.usp.br}

  \vspace{1.3cm}
  \today
\end{center}

 \newpage
 \textbf{{\Huge Disclaimer}}
 \vspace{5cm}

  {\huge Essas notas não possuem relação com professor algum. 

  Qualquer erro é responsabilidade solene do autor.

Caso julgue necessário, contatar: renan.wenzel.rw@gmail.com}
 \newpage

 \tableofcontents

 \newpage

 \section{Aula 00 - 07/08/2023}
   Avisos sobre o curso (Ler e baixar o pdf no e-disciplinas!!!!); 
  
\section{Aula 01 - 09/08/2023}
\subsection{Motivações}
\begin{itemize}
  \item Ângulos, velocidade angular e aceleração angular;
  \item Energia em sistemas com rotação.
\end{itemize}
\subsection{Rotação}
  Antes de qualquer coisa, convencionamos o sentido antihorário como aquele em que \(\Delta \theta >0\) e
o sentido horário como o que \(\Delta \theta <0.\) Uma volta completa em torno do círculo é dada pela versão com \(2\pi\) da fórmula do arco de círculo
 \(\Delta S_{i} = r_{i}\Delta \theta = 2\pi r_{i}\) e, com isso, a variação do ângulo em uma volta completa é dada por 
   \[
     \Delta \theta = \frac{S_{i}}{r_{i}} = \frac{2\pi r_{i}}{r_{i}} = 2\pi rad.
   \]
  Um dos assuntos de importância para nós é o estudo da variação temporal do ângulo. Definimos, nessa lógica, a 
velocidade angular média por 
  \[
    \omega _{med} = \frac{\Delta \theta }{\Delta t}.
  \]
  De modo análogo ao que vimos com cinemática, existe também a velocidade angular instantânea, obtida tomando o limite: 
    \[
      \omega = \lim_{\Delta t\to 0}\frac{\Delta \theta }{\Delta t} = \frac{d\theta }{dt}.
    \]
    Observa-se de cara que, se \(\omega >0, \theta \) aumenta e, se \(\omega <0, \theta \) diminui. Assim como antes,
  precisamos ver, também, a unidade. Em cinemática, a unidade de velocidade era metro por segundo. Dessa vez, já que
  o ângulo move-se em radianos, mas há outras unidades, como a revolução e o grau. Logo, as unidades de \(\omega \) podem ser \([\omega ] = \frac{radianos}{tempo} = \frac{rad}{s}, \frac{graus}{s}, \frac{\text{revolução}}{s},\)
  em que \(1\text{revolução} = 2\pi rad = 360\deg\)

    Por exemplo, se um CD roda a 3000rpm, pode-se expressar essa velocidade de rotação como 
      \[
        \omega = 3000rpm = \frac{3000 \cdot 2\pi}{60} = \frac{600}{6}\pi = 100\pi \frac{rad}{s}.
      \]

  Analogamente, é possível analisar a variação da própria velocidade angular com o tempo, resultando na chamada
  acelerações angulares média e instantânea: 
    \[
      \alpha_{med} = \frac{\Delta \omega }{\Delta t}\quad \alpha  = \frac{d\omega }{dt} = \frac{d}{dt}\biggl(\frac{d\theta }{dt}\biggr) = \frac{d^{2}\theta }{dt^{2}}.
    \]
    A unidade dessa grandeza, novamente, similar à versão linear dela, será dada em \([\alpha ]= \frac{radiano}{s^{2}}\), ou \([\alpha ]=\frac{grau}{s^{2}}\), etc. Nessas
  situações todas, se \(\alpha >0, \omega \) aumenta e, se \(\alpha <0, \omega \) diminui. 

    Agora, suponha que \(\alpha \) é constante. Todos os processos de movimento uniformemente acelerado são válidos aqui também:
    \begin{table}[h!]
    \centering
    \begin{tabular}{|c|c|c|}
        \hline
        & \textbf{Variáveis angulares} & \textbf{Variáveis escalares} \\
        \hline
        \textbf{Posição} & $\theta(t) = \theta_{0} + \omega_{0}t + \frac{\alpha t^{2}}{2}$ & $s(t) = R\theta(t)$ \\
        \hline
        \textbf{Velocidade} & $\omega(t) = \omega_{0} + \alpha t = \frac{d\theta }{dt}$ & $v(t) = v_{0} + at = \frac{dx}{dt}$ \\
        \hline
        \textbf{Aceleração} & $\alpha(t) = \frac{d\omega(t)}{dt} = \frac{d^2\theta(t)}{dt^2}$ & $|\vec{a}(t)| = \frac{dv(t)}{dt} = R\alpha(t),\quad |\vec{a}_{cp}| = \frac{v^2}{R}$ \\
        \hline
        \textbf{Torricelli} & $\omega^{2}(t) = \omega_{0}^{2} + 2\alpha \Delta \theta $ & $v^{2} = v_{0}^{2} + 2a\Delta s.$ \\
        \hline
    \end{tabular}
    \caption{Resumo movimento circular.}
    \label{tab:my_label}
  \end{table}

\begin{example}
  Suponha que há um CD que começa no repouso. Ele começa a girar, indo de 0 a 500rpm em 5.5s. Pergunta-se:
 \begin{itemize}
   \item[a)] Quanto vale \(\alpha \)?
     \item[b)] Quantas voltas o CD dá em 5.5s?
       \item[c)] Qual é a distância percorrida por uma ponta a 6cm do eixo de rotação?
 \end{itemize}

 \textbf{Soluções:}
 \begin{itemize}
   \item[a)] Temos \(\omega (0) = 0, \omega (5.5) = 500rpm.\) Segue que 
     \[
       \omega(t) = \omega_{0} + \alpha t \Rightarrow \alpha  = \frac{\omega (t)}{t} = \frac{500 \cdot 2\pi}{5.5 \cdot 60}\approx 9.52 \frac{rad}{s^{2}}
     \]

    \item[b)] Aplicamos o Torricelli angular com os dados que temos: 
      \[
        \omega^{2} = 2\alpha \Delta \theta \Rightarrow \delta \theta = \frac{\omega^{2}}{2\alpha }\approx 144 rad \Rightarrow \frac{144}{2\pi}rad\approx 23\text{rotações}.
      \]

    \item[c)] Por fim, multiplicando a variação do ângulo pelo raio, obtemos 
      \[
        \Delta S_{i} = r\Delta \theta = 6 \cdot 10^{-2}\cdot 144\approx 8.65m.
      \]
 \end{itemize}
\end{example}

  Olhando de forma cautelosa a fórmula de arco de circulo, podemos derivá-la com respeito ao tempo utilizando o que vimos até agora: 
    \[
      \frac{dS_{i}}{dt} = V_{t} = r_{i}\frac{d\theta }{dt} = r_{i}\omega.
    \]
  Essa derivação resulta em uma velocidade linear, que também pode ser derivada a fim de obter uma aceleração linear 
    \[
      \frac{dV_{t}}{dt} = r_{i}\frac{d\omega }{dt} \Rightarrow a_{t} = r_{i}\alpha.
    \]
    Note a relação entre as duas acelerações que obtivemos, \(a_{c} = \frac{V_{t}^{2}}{r_{i}}= \frac{r_{i}^{2}\omega^{2}}{r_{i}} = r_{i}\omega^{2}.\)

\subsection{Energia Cinética de Rotação}
  A energia cinética, como vista previamente, é dada por 
    \[
      \mathcal{K} = \frac{1}{2}m_{i}v_{i}^{2}.
    \]
  Agora, imagine um corpo discreto (formado por vários pontos). Somemos as energias deles, tal que a energia cinética total é 
    \[
      \mathcal{K}_{T} = \sum\limits_{}^{}\frac{1}{2}m_{i}v_{i}^{2}.
    \]
  Mas, sabemos que \(v_{i} = r_{i}\omega, \) tal que 
    \[
      \mathcal{K} = \frac{1}{2}\sum\limits_{}^{}m_{i}r_{i}^{2}\omega^{2} = \frac{1}{2}\biggl[\sum\limits_{}^{}m_{i}r_{i}^{2}\biggr]\omega^{2}
    \]
  Chamemos o termo em colchete de momento de inércia, denotado por \(I:= \sum\limits_{}^{}m_{i}r_{i}^{2} = \sum\limits_{}^{}I_{i}\). Logo, 
    \[
      \boxed{\hypertarget{kin_en}{\mathcal{K}_{T} = \frac{1}{2}I\omega^{2}.}}
    \]
\newpage
\section{Aula 02 - 10/08/2023}
\subsection{Motivações}
 \begin{itemize}
   \item Momento de Inércia
 \end{itemize}
\subsection{Distribuição Contínua de Massa}
  No caso de distribuições discretas de massa, vimos que o momento de inércia é dado por 
    \[
      I=\sum\limits_{i}^{}m_{i}r_{i}^{2}.
    \]
No entanto, muitas situações do mundo precisam que tratemos a distribuição de massa como algo único, uma
quantidade contínua. Para isso, passamos de somar cada massa para uma integral com respeito a ela: 
  \[
    \hypertarget{momentum_of_inertia_continuous}{\boxed{I = \int_{}^{}r^{2}dm.}}
  \]
  Para o caso de uma barra, por exemplo, na qual a distribuição de massa é dada por 
    \[
      \lambda = \frac{M}{L},
    \]
  segue que \(dm = \lambda dx \Rightarrow dI = x^{2}dm = x^{2}\lambda dx\). Portanto, 
    \[
      I = \lambda \int_{}^{}x^{2}dx = \lambda \frac{x^{3}}{3}.
    \]
  Por exemplo, se o tamanho da barra é 1 e o eixo de rotação está em uma extremidade, o momento de inércia será 
    \[
      I = \lambda \int_{0}^{1}x^{2}dx =\frac{1}{3}ML^{2}.
    \]
    Há outros casos importantes que devem ser tratados. O primeiro deles é o eixo central,
  no qual o eixo de rotação é posicionado na metade do tamanho da barra. Assim, 
    \[
      I = \lambda \int_{-\frac{1}{2}}^{\frac{1}{2}}x^{2}dx = \lambda \frac{x^{3}}{3}\biggl|_{-\frac{1}{2}}^{\frac{1}{2}}\biggr. = \lambda \frac{L^{3}}{12} = \frac{ML^{2}}{12}.
    \]
  O outro engloba a situação em que toda a massa na mesma distância. Neste caso, \(\lambda = \frac{M}{2\pi R}\)
    \[
      I = R^{2} \int_{}^{}dm = MR^{2}, 
    \]
  que também pode ser obtido fazendo uma integral com respeito ao ângulo \(\theta \): 
    \[
      I = R^{2}\lambda \int_{0}^{2\pi } R d\theta = R^{2}\lambda R\times2\pi = MR^{2}.
    \]
  Por fim, é importante olhar o caso dos discos. Discos consistem de dois círculos, um maior e outro menor dentro dele.
Chamaremos de R o raio do maior e de r o do menor. Para eles, há uma distribuição de massa
 \(\sigma = \frac{M}{\pi R^{2}}\), de maneira que o diferencial de massa será 
   \[
     dm = 2\pi r dr\sigma.
   \]
   Com isso, conseguimos encontrar que o momento de inércia é 
     \[
       I = \int_{0}^{R}2\pi \sigma r^{3}dr = 2\pi \sigma \frac{\pi^{4}}{4}\biggl|_{0}^{R}\biggr. = \frac{1}{2}MR^{2}
     \]

\newpage

\section{Aula 03 - 16/08/2023}
\subsection{Motivações}
\begin{itemize}
  \item Disco com buraco;
  \item Rodando disco e cilindro em torno do plano.
\end{itemize}
\subsection{Momento de Inércia em um Disco}
  Vamos considerar um disco de raio \(R_{2}\) que contém dentro de si um buraco de
raio \(R_{1}\). Nisso, consideramos o momento de inércia do disco inteiro como 
  \[
    I = I^{+} + I^{-}.
  \]
  Aqui, \(I^{+}\) desconsidera a existência do buraco, ou seja, tem valor 
    \[
      I^{+} = \frac{\pi R_{2}^{2}\sigma R_{2}^{2}}{2} = \frac{1}{2}M^{+}R_{2}^{2}
    \]
  e o valor de \(I^{-}\) vale 
    \[
      I^{-} = \frac{1}{2}M^{-}R_{1}^{2} = \frac{\pi R_{1}^{2}}{2}\sigma R_{1}^{2}.
    \]
    Assim, considerando o valor total, obtivemos o mesmo resultado que o de antes: 
      \[
        I = \frac{\pi \sigma }{2}(R_{2}^{4} - R_{1}^{4}).
      \]
    Em particular, a densidade de massa após o buraco ser feito, \(\sigma^{*} \), valerá
      \[
        \sigma ^{*} = \frac{M}{\pi(R_{2}^{2} - R_{1}^{2})},
      \]
    de forma que, através de \(I = \frac{\pi \sigma^{*}}{2}(R_{2}^{4} - R_{1}^{4}\), obtemos 
      \[
        I = \frac{\pi M}{2} \frac{(R_{2}^{2}-R_{1}^{2})(R_{2}^{2}+R_{1}^{2})}{\pi (R_{2}^{2}-R_{1}^{2})} = \frac{M}{2}(R_{2}^{2}+R_{1}^{2}).
      \]

    Agora, suponha que deixamos um disco girar em torno de um eixo com velocidade \(\omega \). Como podemos descrever esse sistema e seu momento de inércia? 
Faremos uso do Teorema dos Eixos Paralelos. Apesar de não conhecermos o momento de inércia, sabemos que em algum ponto, encontra-se o centro
de massa do objeto, estando a uma distância h do eixo. Este centro de massa move-se com velocidade \(\vec{v}_{cm}\). Como a energia cinética total
tem valor \(\mathbb{K}_{T} = \frac{1}{2}Mv_{cm}^{2} + \mathbb{K}_{relcm}\), utilizamos que \(\mathbb{K} = \frac{1}{2}I\omega^{2}\) e que \(\mathbb{K}_{relcm}=\frac{1}{2}I_{cm}\omega ^{2}\).
Logo, como \(v_{cm} = h\omega ,\)
 \begin{align*}
   \frac{1}{2}I\omega^{2} &= \frac{1}{2} Mv_{cm}^{2} + \frac{1}{2}I_{cm}\omega^{2}\\
                          &= Mh^{2}\omega^{2} + I_{cm}\omega ^{2}\\
                          &\Rightarrow I = Mh^{2} + I_{cm}.
 \end{align*}
 \begin{example}
   Considerando uma barra em a uma distância de \(\frac{L}{2}\) do eixo de rotação e com momento de inércia 
  \(I = \frac{1}{3}ML^{2},\) podemos utilizar a fórmula para obter 
 \begin{align*}
   &I = Mh^{2} + I_{cm}\\
   &\frac{1}{3}ML^{2} = M \frac{L^{2}}{4} + I_{cm}\\
   &I_{cm} = \frac{1}{12}ML^{2}.
 \end{align*}
 \end{example}
    Mas, o que aconteceria se o disco rodasse em eixos x, y contidos no plano do disco? O que sabemos é que 
      \[
        I_{z} = \sum\limits_{}^{}m_{i}r_{i}^{2}.
      \]
  Além disso, \(r_{i}^{2} = (x_{i}^{2} + y_{i}^{2})\), ou seja, 
 \begin{align*}
   I_{z} &= \sum\limits_{}^{}m_{i}x_{i}^{2} + \sum\limits_{}^{}m_{i}y_{i}^{2}\\
         &= I_{x} + I_{y}.
 \end{align*}
  Este resultado é conhecido como teorema dos eixos perpendiculares, mas vale apenas para corpos bidimensionais.
Em particular, no caso do cilindro, em que \(I_{x} = I_{y},\) 
  \[
    I_{x} = I_{y} \Rightarrow 2I_{x} = I_{z} \Rightarrow I_{x} = \frac{1}{4}MR^{2}.
  \]
  Além disso, considerando que \(dI_{x} = \frac{1}{4}dm R^{2} + dm z^{2}\), obtemos 
    \[
      I_{x} = \frac{1}{4}R^{2} \int_{}^{}dm + \int_{}^{}dm z^{2}.
    \]
    Como \(dm = \lambda dz = \frac{M}{L}dz\), em que L é o comprimento, segue o seguinte resultado 
      \[
        I_{x} = \frac{1}{4}R^{2}\int_{-\frac{L}{2}}^{\frac{L}{2}}\frac{M}{L}dz + \int_{-\frac{L}{2}}^{\frac{L}{2}}\frac{M}{L}z^{2}dz
      \]
    Assim, fazendo as contas, 
      \[
        I_{x} = \frac{1}{4}MR^{2} + \frac{ML^{2}}{12}
      \]

\begin{example}
  Considere uma barra de tamanho L e massa M e deixe-a descer em um pivô. Qual é a força que ele terá que fazer?

  Sabe-se que há uma força peso com módulo Mg agindo e que \(E_{mec_{i}} = E_{mec_{f}},\) tal que \(\mathbb{K}_{i} + U_{i} = \mathbb{K}_{f} + U_{f}\).
Mas, \(\mathbb{K}_{i} = U_{i} = 0\) e \(\mathbb{K}_{f} = \frac{1}{2}I\omega_{f}^{2}, U_{f} = Mg(\frac{-L}{2}).\) Assim, 
  \[
    \frac{1}{2}I\omega_{f}^{2} - Mg \frac{L}{2} = 0 \Rightarrow \omega_{f}^{2} = \frac{MgL}{I} = \frac{MgL}{\frac{1}{3}ML^{2}} = \frac{3g}{L}.
  \]
  Assim, usando que \(a_{cm} = r\omega_{f}^{2},\)
 \begin{align*}
   F- Mg &= Ma_{cm} \Rightarrow F = Mg + M \frac{L}{2}\omega_{f}^{2}\\
         &= Mg + \frac{ML}{2}\frac{3g}{L}\\
         &=Mg + \frac{3}{2}Mg = \frac{5}{2}Mg.
 \end{align*}
\end{example}
\begin{example}
  Considere uma roldana de raio R e massa \(m_{r}\). Atrele a ela, com uma corda de massa \(m_{c}\) e tamanho L, um balde de massa \(m_{b}\). 
Em seguida, solte-o para cair uma distância d. Qual é a velocidade do sistema?

  Sabemos que \(E_{mec_i} = E_{mec_f}\), ou seja, 
    \[
      \mathbb{K}_{i} + U_{i} = \mathbb{K}_{f} + U_{f}.
    \]
  Suponha que \(\mathbb{K}_{i} = U_{i} = 0.\) Quando o balde descer, sendo \(m_{c}^{*} = \frac{d}{L}m_{c}\) a massa da fração de corda que desceu, a potencial final passará a valer
 \(U_{f} = m_{b}(-d)g + m_{c}^{*}(\frac{-d}{2})g = -m_{b}gd - \frac{1}{2}m_{c}^{*}gd.\) Com relação à cinética, 
   \[
     \mathbb{K}_{f} = \frac{1}{2}m_{r}v^{2} + \frac{1}{2}m_{c}v^{2} + \frac{1}{2}m_{b}v^{2}.
   \]
   Utilizando as relações de energia que vimos, segue que 
  \begin{align*}
    &\mathbb{K}_{f} + U_{f} = 0\\
    &\Rightarrow \frac{1}{2}(m_{c}+m_{r}+m_{b})v^{2} = m_{b}gd + \frac{1}{2}m_{c}^{*}gd\\
    &\Rightarrow (m_{r}+m_{c}+m_{b})v^{2} = 2m_{b}gd + m_{c}g \frac{d^{2}}{L}\\
    &\Rightarrow v^{2} = \frac{(2m_{b}L + m_{c}d)}{m_{r}+m_{c}+m_{b}}\frac{gd}{L}\\
    &\Rightarrow v = \sqrt[]{\frac{(2m_{b}L + m_{c}d)}{m_{r}+m_{c}+m_{b}}\frac{gd}{L}}.
  \end{align*}
\end{example}
\newpage

\section{Aula 04 - 17/08/2023}
\subsection{Motivações }
\begin{itemize}
  \item Segunda lei de Newton do Movimento Circular;
  \item Torque da Gravidade.
\end{itemize}
\subsection{Segunda Lei de Newton}
Ao considerarmos uma força aplicada a um objeto em torno de um círculo de raio r, essa força faz um ângulo
 \(\theta \) com a paralela ao raio. Além disso, há uma componente dessa força que será tangente 
 à trajetória do objeto ao longo do círculo. Denotando essa segunda por \(F_{t},\) há duas formas de expressá-la: 
   \[
     F\sin{(\theta )} = F_{t},\quad F_{t} = ma_{t}.
   \]
   Além disso, a aceleração tangencial \(a_{t}\) satisfaz \(a_{t} = r\alpha \). Assim, obtemos a relação 
     \[
       F sin(\theta ) = ma_{t} \Rightarrow F\sin{\theta } = mr\alpha \Longleftrightarrow rF\sin{(\theta )} = mr^{2}\alpha.
     \]
     Esse termo à esquerda é conhecido como \textbf{torque} 
       \[
         \hypertarget{torque}{\boxed{\tau = mr^{2}\alpha = rF\sin{(\theta )}}}
       \]
  Em particular, sendo o torque total a soma de todos os torques, obtemos 
    \[
      \tau = \sum\limits_{}^{}\tau_{i} = \sum\limits_{}^{}m_{i}r_{i}^{2}\alpha = I\alpha 
    \]
    Uma propriedade é que a soma dos torques das forças internas vale zero.

    Olhando um caso mais específico, ao considerarmos um círculo de raio r e uma força que
faz um ângulo \(\theta \) com a paralela ao raio e outro círculo menor de raio r' com a mesma força aplicada,
mas ângulo \(\theta ',\) então \(l=r'\sin{(\theta ')}\) é a componente perpendicular à linha na qual a força está atuando.
  A vantagem disso é que o torque pode ser, então, expresso através de \(\tau = Fl = F_{t}r'\)
\subsection{O Torque da Força da Gravidade}
  Se considerarmos um corpo sofrendo a ação da força peso, o torque desse corpo pode ser descrito por
  \(\tau_{i} = m_{i}gx_{i}\) e, o torque total, será a soma desses torques: 
    \[
      \tau_{r} = \sum\limits_{}^{}\tau_{i} = \sum\limits_{}^{}[m_{i}x_{i}]g
    \]
  Mas, esse é exatamente o torque do centro de massa do objeto \(\tau_{r} = Mx_{cm}g\). Outro assunto que é
importante ressaltar é que, durante os estudos de dinâmica, a forma de estudar as forças em um sistema é através
dos chamados diagramas de força, o que traz à tona a questão do que funcionaria pro estudo do torque.
\begin{example}
  Considere uma roda de bicicleta e a catraca, que sofre uma força F de 18N. Suponha que o raio r
da catraca é de 7cm e o da roda, R, vale 35cm. Além disso, a massa vale 2.4kg. Qual é a velocidade angular para t=5,5s?

Começamos afirmando que o torque é \(\tau = I\alpha = Fr_{c}\). Assim, 
\[
  \alpha = \frac{Fr_{c}}{I} = \frac{Fr_{c}}{MR^{2}} = \frac{18 \cdot (0,07)}{2,4(0,35)^{2}}\frac{rad}{s^{2}}.
\]
Com isso, 
  \[
    \omega = \omega_{0} + \alpha t = \alpha t = \frac{18 \cdot (0,07)}{2,4(0,35)^{2}} \cdot 5,5 = 21,4 \frac{rad}{s}
  \]
\end{example}
\begin{example}
  Considere uma barra de massa m e comprimento l está presa por um pivô, o qual realiza uma força F. Após soltá-la, qual é a força que o pivô realiza?

  Sabemos que \(\tau = mg \frac{l}{2} = I\alpha = \frac{1}{3}ml^{2}.\) Logo, 
 \begin{align*}
   &mg \frac{l}{2} = \frac{1}{3}ml^{2}\alpha \\
   &\alpha = \frac{3}{2}\frac{g}{l}.
 \end{align*}
 Olhando no eixo y, sabemos que \(a_{cm_{y}} = r\alpha  = \frac{l}{2}\frac{3}{2}\frac{g}{l} = \frac{3}{4}g\), tal que 
   \[
     F - mg = -ma_{cm_y} \Rightarrow F = mg - \frac{3}{4}mg = \frac{1}{4}mg
   \]
\end{example}
\begin{example}
  Suponha que temos uma roldana de raio R e momento de inércia I. Pendura-se um corpo de massa m na roldana. Qual é a aceleração de queda do corpo?

\textbf{Roldana:}
As forças que atuam na roldana são o Peso dela, \(P_{r}\), a tensão T
e a força resultante ao peso \(F_{r}\). Assim, 
\begin{align*}
  &F_{r} = P_{r} + T\\
  &TR = I\alpha,\quad a = \alpha R. 
\end{align*}

\textbf{Corpo:}
No corpo, por outro lado, tem-se apenas a tensão T e o peso mg, de forma que 
  \[
    mg-T = ma.\quad a = \alpha R
  \]
\end{example}
  Continua na próxima aula...
  \newpage

\section{Aula 05 - 21/08/2023}
\subsection{Motivações}
\begin{itemize}
  \item Continuação do exemplo e outros;
  \item Potência;
  \item Corpos que rolam sem deslizar.
\end{itemize}
\subsection{Continuando o Exemplo}
\begin{example}[continuando...]
  Segue a relação de tração 
    \[
      TR = I \frac{a}{R} \Rightarrow T = \frac{I}{R^{2}}a.
    \] 
  Com isso, 
    \[
      mg = ma + T = ma + \frac{I}{R^{2}}a = a[1 + \frac{I}{mR^{2}} \Rightarrow a = \frac{g}{1 + \frac{I}{mR^{2}}}.
    \]
  Descobrimos, assim, os valores de T e de \(F_{s}\)
 \begin{align*}
   &T = \frac{I}{R^{2}}\frac{g}{1+ \frac{I}{mR^{2}}}\\
   &F_{s} = Mg + \frac{I}{R^{2}}\frac{g}{1+\frac{I}{mR^{2}}}.
 \end{align*}
 \end{example} 
\begin{example}
  Considere a máquina de Atroos - dois blocos presos a uma roldana, um de massa \(m_{1}\) e outro de massa \(m_{2}\) tais que \(m_{1} > m_{2}\).
A roldana tem massa M, momento de inércia I e raio R. Vejamos as forças

  \textbf{Bloco 1:}
    No primeiro bloco, agem forças de tração \(T_{1}\) e peso \(m_{1}g\). Escrevendo as equações,
   \begin{align*}
     m_{1}g - T_{1} = m_{1}a
   \end{align*}

  \textbf{Bloco 2:}
    No bloco dois, agem a tração \(T_{2}\) e o peso \(m_{2}g\)
   \begin{align*}
     T_{2}-m_{2}g = m_{2}a
   \end{align*}

  \textbf{Roldana:}
    Tem-se a equação 
   \begin{align*}
     (T_{1} - T_{2})R = I\alpha. \Longleftrightarrow T_{1} - T_{2} = \frac{I}{R}\frac{a}{R} = \frac{Ia}{R^{2}}
   \end{align*}

  Como a roldana está rodando, tem-se a relação \(m_{1g} > T_{1} > T_{2} > m_{2}g\)
A seguir, soma-se a equação do bloco 2 com a da Roldana, tal que 
  \[
    m_{1}g - m_{2}g - (T_{1}-T_{2}) = (m_{1}+m_{2})a \Longleftrightarrow (m_{1}-m_{2})g - \frac{Ia}{R^{2}} = (m_{1}+m_{2})a
  \]
  Isola-se a equação no a: 
    \[
      a = \frac{(m_{1}-m_{2})g}{(m_{1}+m_{2})+\frac{I}{R^{2}}}
    \]
\end{example}
\begin{example}
  Considere uma roldana com massa M, momento de inércia I e raio R presa à quina uma mesa. Atrela-se a ela dois corpos, um com massa \(m_{1}\) e que está em cima da mesa
e outro, de massa \(m_{2}\), suspenso pela corda. 
  \textbf{Corpo 1:}
    As forças atuando no bloco 1 são a normal \(F_{N_{1}}\), a peso \(m_{1}g\) e a tração \(T_{1}\), de forma que 
      \[
        T_{1}=m_{1}a
      \]

  \textbf{Corpo 2:}
    Para o bloco 2, podemos descrever o sistema considerando a tração \(T_{2}\) e o peso \(m_{2}g\), tal que 
      \[
        m_{2}g - T_{2} = m_{2}a.
      \]

  \textbf{Roldana:}
    As forças que atuam na roldana são a tração na direção do bloco 1, \(T_{1}\), a outra na do bloco 2, \(T_{2}\), o peso
  \(Mg\) e uma força da quinta nela \(\vec{F_{s}}\). Além disso, \(a=R\alpha \). A equação do sistema será
 \begin{align*}
   (T_{2}-T_{1})R = I\alpha \Rightarrow  T_{2} - T_{1} = \frac{Ia}{R^{2}}.
 \end{align*}

  Somando a equação do bloco 1 e a do bloco 2, chega-se em 
    \[
      m_{2}g - (T_{2}-T_{1}) = (m_{1}+m_{2})a
    \]
  Assim, 
 \begin{align*}
   &T_{2} - T_{1} = \frac{Ia}{R^{2}}\\
   &m_{2}g - \frac{Ia}{R^{2}} = (m_{1}+m_{2})a\\
   &a = \frac{m_{2}g}{(m_{1}+m_2) + \frac{I}{R^{2}}}
 \end{align*}
 Além disso, 
   \[
     T_{1} = \frac{m_{1}m_{2}g}{(m_{1}+m_{2})+\frac{I}{R^{2}}}.
   \]
\end{example}
\subsection{Potência}
  Previamente, a potência era dada pela relação \(dW = F ds.\) Considerando o caso de uma força agindo
emu ma situação circular, isso torna-se \(dW = FRd\theta \). No entanto, esse termo à direita lembra muito um torque. De fato,
a relação que obtemos é que \(dW = \tau d\theta \). Portanto, 
  \[
    \hypertarget{power_torque}{\boxed{P = \frac{dW}{dt} = \tau \frac{d\theta }{dt} = \tau \omega }}
  \]
 \begin{example}
   Um motor de combustão de um carro fornece um torque de \(\tau  = 678Nm\) e está rodando a \(\omega = 4500rpm \approx 471 \frac{rad}{s}\).
Com isso, a potência será 
  \[
    P\approx 315kW.
  \]
 \end{example}
\begin{example}
  Tome uma roda gigante \textbf{(em Londres).} Ela tem um diâmetro de \(135m\), pesa 1600 toneladas e dá 2 revoluções por hora.
Qual é o torque necessário para parar a roda em 10m?
  
  Para começar, observe que \(W = \tau \Delta \theta \) e que \(S = R\Delta \theta = 10m.\) Em particular, temos o valor de R, tal que 
    \[
      S = R\Delta \theta \Longleftrightarrow 10 = 67.5\Delta \theta \Rightarrow \Delta \theta \approx 0,148rad.
    \]
  Note que 
    \[
      W = -(\mathbb{K}_{f} - \mathbb{K}_{i}) \Longleftrightarrow \tau \Delta \theta = -\biggl[0 - \frac{I\omega^{2}}{2}\biggr].
    \]
  Logo, convertendo \(\omega \) para radianos por segundo (\(\omega = 3,5 \cdot 10^{-3}\frac{rad}{s}\),
    \[
      \tau = \frac{I\omega^{2}}{2\Delta \theta } = \frac{MR^{2}\omega^{2}}{2\Delta \theta } \approx 3 \cdot 10^{5}Nm.
    \]
    Em particular, 
      \[
        F = \frac{\tau }{R}\approx 4,4 \cdot 10^{3}N.
      \]
\end{example}
\subsection{Corpos que Rolam sem Deslizar}
  Imagine um sistema em que um disco de raio R está a rolar com velocidade do centro de massa \(\vec{v}_{cm}.\) Considerando o ponto que tangencia o chão, em que a velocidade é nula,
ele se mexe com velocidade angular \(\omega \) em um raio \(\vec{r}\). Assim, a energia cinética desse sistema será 
  \[
    \mathbb{K}_{T} = \frac{1}{2} Mv_{cm}^{2} + \mathbb{K}_{rel} = \frac{1}{2}Mv_{cm}^{2} + \frac{1}{2}I_{cm}\omega ^{2},
  \]
  em que considera-se que \(v_{cm} = R\omega.\) 

  Agora, considere um plano inclinado e uma bola de massa m, momento de inércia I e raio R que irá subir este plano inclinado até parar. Como podemos achar a altura que ela para, 
fornecida velocidade inicial do centro de massa \(v_{cm}\). Utilizando a conservação da energia mecânica, 
  \[
    E_{mec_{i}} = E_{mec_{f}}.
  \]
  Sabemos que 
    \[
      E_{mec_{i}} = \frac{1}{2}mv_{cm}^{2} + \frac{I_{cm}}{2}\omega ^{2}\quad \& E_{mec_{f}} = mgh.
    \]
  Assim, 
  \begin{align*}
    &mgh = \frac{1}{2} mv_{cm}^{2} + \frac{I_{cm}}{2}\omega^{2}\\
    &\Rightarrow h = \frac{1}{2}\biggl[v_{cm}^{2} + \frac{I_{cm}}{2m}\frac{v_{cm}^{2}}{R^{2}}\biggr]\\
    &\Rightarrow h = \frac{v_{cm}^{2}}{2g}\biggl[1 + \frac{I_{cm}}{mR^{2}}\biggr].
  \end{align*}
 \begin{example}
   Para o caso da esfera, em que \(I = \frac{2}{5}mR^{2},\) 
     \[
       h = \frac{v_{cm}^{2}}{2g}\biggl[1 + \frac{2}{5}\biggr] = \frac{7}{10}\frac{v_{cm}^{2}}{g}
     \]
 \end{example}
\begin{example}
  Considere um cenário de sinuca em que um taco aplica uma força F. Se ela é aplicada acima do eixo de rotação, a bola roda para frente. Caso seja exatamente no eixo de rotação,
ela apenas deslizará. Por fim, se for atingida abaixo do eixo de rotação, ela rodará ao contrário. Como fazer ela não rodar?

  Em qualquer um desses pontos, a força é \(F=ma.\) No caso em que ela roda para frente, ou seja, é atingida a uma distância d acima do eixo de rotação,
temos \(\tau = Fd = I\alpha.\) Segue que, para que ela não rode, 
\begin{align*}
  & Fd = I\alpha = \frac{Ia}{R}\\
  &\Rightarrow d = \frac{I}{mR} = \frac{2}{5}\frac{mR^{2}}{mR} = \frac{2}{5}R.
\end{align*}
\end{example}
\newpage

\section{Aula 06 - 23/08/2023}
\subsection{Motivações}
\begin{itemize}
  \item Planos inclinados;
  \item Refrigerante VS Cilindro;
  \item Bola deslizando.
\end{itemize}
\subsection{Objeto Rolante em Plano Inclinado}
  Coloque uma bola de massa M, momento de inércia I e raio R em um plano inclinado a uma ângulo \(\theta \).
Analisando as forças presentes, estão inclusas a da gravidade, a normal e uma força de atrito. Considerando x a direção em que a bola
está indo para. Nessa direção, as relações das forças serão 
  \[
    Mg\sin{(\theta )} - f_{at} = Ma_{cm}.
  \]
  Quanto à rotação, temos 
    \[
      f_{at}R = I_{cm}\alpha.
    \]
  Utilizando \(a_{cm} = R\alpha \), relacionamos as duas como segue:
 \begin{align*}
   &f_{at} = Mg\sin{(\theta )} - Ma_{cm}\\
   &Mg\sin{(\theta )} - Ma_{cm} = \frac{I_{cm}}{R}\frac{a_{cm}}{R}\\
   &Mg\sin{(\theta )} = a_{cm}\biggl[M + \frac{I_{cm}}{R^{2}}\biggr]\\
   &Mg\sin{(\theta )} = Ma_{cm}\biggl[1 + \frac{I_{cm}}{MR^{2}}\biggr]\\
   &a_{cm} = \frac{g\sin{(\theta )}}{1 + \frac{I_{cm}}{MR^{2}}}.
 \end{align*}
  Com isso, conseguimos descrever a força de atrito utilizando que, na esfera, \(I = \frac{2}{5}MR^{2}\). Logo, 
 \begin{align*}
   &a_{cm} = \frac{5}{7}g\sin{(\theta )}\\
   &f_{at} = \frac{2}{5}\frac{MR^{2}}{R^{2}}\frac{5}{7}g\sin{(\theta )} = \frac{2}{7}Mg\sin{(\theta )}.
 \end{align*}
\subsubsection{Simetrias}
  Se um corpo tem simetria esférica ou cilíndrica, então \(I = \beta mR^{2}\). Em particular, podemos usar isso para
generalizar o raciocínio realizado acima. De fato, tanto \(a_{cm}\) quanto \(f_{at}\) podem ser reescritos como segue:
\begin{align*}
  &a_{cm} = \frac{g\sin{(\theta )}}{1+\beta }\\
  f_{at} = \frac{\beta Mg\sin{(\theta )}}{1+\beta }& = \frac{Mg\sin{(\theta )}}{\frac{1}{\beta }(1+\beta )} = \frac{Mg\sin{(\theta )}}{\beta^{-1}+1}.
\end{align*}
  Observa-se, assim, que entre uma esfera, um cilindro e um aro num plano inclinado (de mesmo raio), a esfera chegará primeiro ao chão,
visto que ela tem um \(\beta \) menor.
  
  Agora, considere um cilindro de massa M e raio R iguais aos de uma lata de refrigerante e coloque os dois juntos para descer um plano inclinado. Quem chegará primeiro?
A resposta (por incrível que pareça) é o refrigerante, pois o liquido dentro não possui momento de inércia, ou seja, soltar ela com líquido ou vazia resultará no mesmo! Analisando a energia
desse sistema, temos 
  \[
    E_{mec_{i}} = mgh\quad \& E_{mec_{f}} = \frac{1}{2}mv_{cm}^{2} + \frac{1}{2}I_{cm}\omega^{2},
  \]
  em que \(\omega = \frac{v_{cm}}{R}\). Assim, 
    \[
      E_{mec_{f}}=\frac{1}{2}mv_{cm}^{2} + \frac{1}{2}\beta mR^{2}\frac{v_{cm}^{2}}{R^{2}}.
    \]
  Juntando ambas, 
    \[
      \frac{1}{2}mv_{cm}^{2}[1+\beta ] = mgh \Rightarrow v_{cm}^{2} = \frac{2gh}{1+\beta }
    \]
  Como a latinha é considerada um aro, ela ganha!

  Vamos voltar nossa atenção à força de atrito. Vimos antes que 
    \[
      f_{at} = \frac{mg\sin{(\theta )}}{1 + \beta^{-1}}.
    \]
  Sabe-se que a força de atrito tem um valor máximo, dado por \(f_{at_{max}} = \mu_{e}N\), ou seja,
 \(f_{at}\leq f_{at_{max}}\). Em outras palavras, 
   \[
     \frac{mg\sin{(\theta )}}{1+\beta^{-1}}\leq \mu_{e}mg\cos{(\theta )} \Longleftrightarrow \tan{(\theta )}\leq \mu_{e}(1+\beta^{-1}).
   \]
   Com isso, obtemos um ângulo máximo no qual o plano inclinado pode estar sem que a bola comece a rolar deslizando.
  \begin{example}
    Para uma esfera, em que \(\beta =\frac{2}{5}\), o ângulo máximo é 
      \[
        \tan{(\theta )}\leq 3.5\mu_{e}.
      \]
    Para um cilindro, com \(\beta = \frac{1}{2}\), tem-se 
      \[
        \tan{(\theta )}\leq 3\mu_{e}.
      \]
    Por fim, para um aro, no qual \(\beta =1\), 
      \[
        \tan{(\theta )}\leq 2\mu_{e}.
      \]
  \end{example}

  \subsection{Bola Deslizando}
    Agora, suponha que colocamos uma bola de massa m deslizando (apenas movimento translacional) com velocidade v em uma superfície com atrito. Note que as forças agindo são
  a peso, a normal e a força de atrito. Em termos de translação, então, temos 
    \[
      -f_{at} = ma_{cm} \Rightarrow a_{cm} = -\frac{f_{at}}{m} = -\frac{\mu_{c}N}{m} = -\mu_{c}g.
    \]
  Por outro lado, quanto à rotação, começamos notando que o torque vale \(\tau = I_{cm}\alpha \). Assim, 
  \[
    f_{at}R = I_{cm}\alpha \Longleftrightarrow a_{c}mgR = \frac{2}{5}mR^{2}\alpha
  \]
  Portanto, 
    \[
      \alpha = \frac{5}{2}\frac{\mu_{c}g}{R}.
    \]
  A velocidade do centro de massa, então, pode ser obtida com \(v_{cm} = v - a_{cm}t = v - \mu_{c}gt\). Nota-se, então,
que há uma redução na velocidade, até que, eventualmente, ela para de deslizar e passa a rotacionar, fazendo com que \(v_{cm} = \omega R\). 
Nesse instante, tem-se 
  \[
    v-\mu_{c}gt = R\frac{5}{2}\frac{\mu_{c}g}{R}t \Rightarrow v = \mu_{c}gt\biggl[1+\frac{5}{2}\biggr] = \mu_{c}gt \frac{7}{2}.
  \]
Portanto, o tempo parar ela parar de deslizar é 
  \[
    \boxed{t^{*} = \frac{2v}{7\mu_{c}g}.}
  \]
  A partir disso, podemos também encontrar a distância percorrida:
 \begin{align*}
   x(t^{*}) &= vt^{*} - \frac{\mu_{c}gt^{*^{2}}}{2}\\
            &= \frac{v2v}{7\mu_{c}g} - \frac{\mu_{c}g}{2}\frac{4v^{2}}{49\mu_{c}^{2}g^{2}}\\
            &= \frac{2v^{2}}{7\mu_{c}g} - \frac{2}{49}\frac{v^{2}}{\mu_{c}g}\\
            &= \frac{v^{2}}{\mu_{c}g}\biggl[\frac{14-2}{49}\biggr] = \boxed{\frac{12}{49}\frac{v^{2}}{\mu_{c}g}.}
 \end{align*}
 \newpage

\section{Aula 07 - 24/08/2023}
\subsection{Motivações} 
\begin{itemize}
  \item A natureza vetorial do torque. 
\end{itemize}
\subsection{Vetores}
  Vamos começar com um exemplo de como a nossa construção atual falha em algumas descrições.
Quando um peão está rodando, ou um giroscópio, o torque não é apenas em um dimensão. Assim, não podemos usar nosso modelo atual.

  A descrição do torque como grandeza vetorial é simplesmente \(\vec{\tau } = I \vec{\alpha } = I \frac{d \vec{\omega }}{dt}\). Disto, observa-se 
que a direção dele é a mesma da velocidade angular \(\vec{\omega }\). Considerando uma situação tridimensional e colocarmos 
 \(\vec{r}\) e a força \(\vec{F}\), fazendo ângulo \(\varphi \) com a reta de \(\vec{r}\), no plano xy, o torque sairá na direção z, visto que ele é o produto vetorial da força com 
 \(\vec{r}\): \(\vec{\tau } = \vec{r}\times \vec{F}\). Em particular, segue que \(|\vec{\tau }| = rF\sin{(\varphi )}\)
Para operacionalizar melhor, vamos padronizar \(\hat{i}, \hat{j}, \hat{k}\) como os versores nas direções x, y e z respectivamente. Segue que 
\begin{itemize}
  \item[a)] \(\vec{i}\times \vec{j} = \vec{k}\);
  \item[b)] \(\vec{j}\times \vec{k} = \vec{i}\);
  \item[c)] \(\vec{k}\times \vec{i} = \vec{j}\);
\end{itemize}
  Além disso, vetores em três dimensões serão da forma \(\vec{A} = A_{x}\hat{i} + A_{y}\hat{j} + A_{z}\hat{k}\) e \(\vec{B} = B_{x}\hat{i} + B_{y}\hat{j} + B_{z}\hat{k}\). 
Lembre-se que \(\vec{C} = \vec{A} \times \vec{B} = -\vec{B}\times \vec{A}\) e \(\vec{A} \times \vec{A} = 0.\) Assim, 
\begin{align*}
  \vec{A} \times \vec{B} &= (A_{x}\hat{i} + A_{y}\hat{j} + A_{z}\hat{k})\times(B_{x}\hat{i} + B_{y}\hat{j} + B_{z}\hat{k})\\
                         &= \hat{i}\biggl[A_{y}B_{z} - A_{z}B_{y}\biggr] + \hat{j}\biggl[-A_{x}B_{z} + A_{z}B_{x}\biggr] + \hat{k}\biggl[A_{x}B_{y}-A_{y}B_{x}\biggr]\\
                         &= \biggl[A_{y}B_{z} - A_{z}B_{y}\biggr]\hat{i} + \biggl[A_{z}B_{x} - A_{x}B_{z}\biggr] + \biggl[A_{x}B_{y}-A_{y}B_{x}\biggr]\hat{k}.
\end{align*}
  Outra propriedade é com relação ao quadrado do produto vetorial:
 \begin{align*}
   &|\vec{A}\times \vec{B}|^{2} = |\vec{A}|^{2}|\vec{B}|^{2}\sin^{2}{(\varphi )}\\
   &|\vec{A}\cdot \vec{B}|^{2} = |\vec{A}|^{2}|\vec{B}|^{2}\cos^{2}{(\varphi )}\\
   &\Rightarrow \frac{|\vec{A}\cdot \vec{B}|^{2}}{|\vec{A}|^{2}|\vec{B}|^{2}} + \frac{|\vec{A}\times \vec{B}|^{2}}{|\vec{A}|^{2}|\vec{B}|^{2}} = 1\\
   &\Rightarrow  |\vec{A}\cdot \vec{B}|^{2} + |\vec{A}\times \vec{B}|^{2} = |\vec{A}|^{2}|\vec{B}|^{2}.
 \end{align*}
 \newpage

\section{Aula 08 - 28/08/2023}
\subsection{Motivações}
\begin{itemize}
  \item Momento Angular;
  \item Torque em termos do produto vetorial.
\end{itemize}
\subsection{Um pouco mais de produto vetorial.}
 \begin{example}
  Considere que \(A = 2 \hat{j}\) e \(B = 5\hat{i} + 2 \hat{j}\). Então, 
     \[
       \vec{A} \times{\vec{B}} = 2\hat{j} \times (5\hat{i} + 2\hat{j}) = 2\hat{i}\times 5\hat{j} = -10\hat{k}.
     \]
  Além disso, \(|\vec{A}\times \vec{B}|^{2} = |\vec{A}|^{2}|\vec{B}|^{2} - (\vec{A}\cdot \vec{B})^{2} = 100.\) Note que 
 \(|\vec{A}|^{2}|\vec{B}|^{2} = 4 (25+4) = 29\times 4\) e que \(\vec{A}\cdot \vec{B} = 4, \) ou seja, 
   \[
     |\vec{A}|^{2}|\vec{B}|^{2} - (\vec{A}\cdot \vec{B})^{2} = 4\times 25 + 16 - 16 = 100.
   \]
 \end{example}
 \begin{example}
   Com os mesmos A e B de antes, coloque \(\vec{C} = 3\hat{j} + 2 \hat{k}.\) Vamos calcular \(\vec{A}\times(\vec{B} + \vec{C}):\)
  \begin{align*}
    \vec{a}\times(\vec{B}+\vec{C}) &= 2\hat{j} \times [5\hat{i} + 5\hat{j} + 2\hat{k}]\\
                                   &= - 10\hat{k} + 4\hat{i} = 4\hat{i} - 10\hat{k}.
  \end{align*}
  Por outro lado, 
 \begin{align*}
   \vec{A} \times \vec{B} + \vec{A} \times \vec{C} &= 2\hat{j}\times(5\hat{i} + 2\hat{j}) + 2\hat{j}\times(3\hat{j}+2\hat{k})\\
                                                   &= 4\hat{i} - 10\hat{k}.
 \end{align*}
 Ou seja, os dois de fato coincidem.
 \end{example}
 \begin{example}
  Agora, olhemos para \(\vec{A}\times(\vec{B}\times \vec{C}) = (\vec{A}\cdot \vec{C})\vec{B} - (\vec{A}\cdot \vec{B})\vec{C}\): 
 \begin{align*}
   \vec{A}\times(\vec{B}\times \vec{C}) &= 2\hat{j}\times \biggl[(5\hat{i} + 2\hat{j})\times(3\hat{j}+2\hat{k})\biggr] \\
                                        &= 2\hat{j}\times [15\hat{k} - 10\hat{j} + 4\hat{i}] = 30\hat{i} - 8\hat{k}.
 \end{align*}
 Quanto ao lado direito da igualdade,
\begin{align*}
  &(\vec{A}\cdot \vec{C})\vec{B} = (2\hat{j}\cdot (3\hat{j}+2\hat{k}))(5\hat{i}+2\hat{j}) = 30\hat{i} + 12\hat{j}\\
  &(\vec{A}\cdot \vec{B})\vec{C} = (2\hat{j}\cdot (5\hat{i} + 2\hat{j}))(3\hat{j} + 2\hat{k}) = 12\hat{j} + 8\hat{k}\\
  (\vec{A}\cdot \vec{C})\vec{B} - &(\vec{A}\cdot \vec{B})\vec{C} = 30\hat{i} + 12\hat{j} - 12\hat{j} - 8\hat{k} = 30\hat{i} - 8\hat{k}.
\end{align*}
 \end{example}
 A seguir, vamos considerar um exemplo mais aplicado. 
\begin{example}
  Considere que o movimento de uma partícula em duas dimensões é descrito por \(\vec{r}(t) = v_{0}t\hat{i} + y_{0}\hat{j}\), sendo \(y_{0}\) o tanto que ela se moveu no eixo y.
Temos \(\frac{d \vec{r}}{dt} = v_{0}\hat{i} = \vec{v}\) e, considerando \(\vec{B} = B_{0}t \hat{j}\), segue que 
\begin{align*}
  \vec{r}\times \vec{B} &= (v_{0}t\hat{i} + y_{0}\hat{j})\times(B_{0}t\hat{j})\\
                        &= v_{0}t^{2}B_{0}\hat{k}.  
\end{align*}
  Em particular, 
    \[
      \frac{d(\vec{r}\times \vec{B})}{dt} = 2v_{0}tB_{0}\hat{k}.
    \]
  De fato, em geral, temos uma versão da regra do produto para produtos escalares:
    \[
      \hypertarget{vector_product_rule}{\boxed{\frac{d(\vec{r}\times \vec{B})}{dt} = \frac{d \vec{r}}{dt}\times \vec{B} + \vec{r}\times \frac{d \vec{B}}{dt}.}}
    \]
\end{example}
  Essas formas que lidamos com produtos escalares até agora funcionam bem para vetores mais simples. No entanto, quando mais termos começam a surgir, pode tornar-se algo explicado
muito rapidamente. Para isso, é útil ter em mente a forma que o produto escalar realmente toma - a de um determinante de uma matriz. 
  \[
    \hypertarget{vector_product}{    \vec{A}\times \vec{B} = \det \begin{pmatrix}
          \hat{i} & \hat{j} & \hat{k}\\
          A_{x} & A_{y} & A_{z}\\
          B_{x} & B_{y} & B_{z}
      \end{pmatrix} = \boxed{\hat{i}(A_{y}B_{z} - A_{z}B_{y}) + \hat{j}(A_{z}B_{x} - A_{x}B_{z}) + \hat{k} (A_{x}B_{y}-A_{y}B_{x}).}}
  \]
  Estamos, agora, habilitados para aplicar esses conceitos na física.
\subsection{Momento Angular}
  Considere um sistema xyz e um vetor \(\vec{r}\) contido no plano xy. Aplica-se uma força \(\vec{F}\), também contida no plano xy. 
Definimos, sendo \(\vec{p} = m \vec{v}\) o momento linear, o momento angular da partícula como 
  \[
    \hypertarget{angular_momentum}{\boxed{\vec{L} = \vec{r}\times \vec{p}}.}
  \]
  Equivalentemente, \(\vec{L} = \vec{r}\times m \vec{v} = m(\vec{r}\times \vec{v}) = mrv\hat{k}.\) Fazendo
 \(v = r\omega,\) segue que \(\vec{L} = mr^{2}\omega \hat{k} = I\omega \hat{k} = I \vec{\omega }\). 

  Com relação ao torque, sabe-se que \(\vec{\tau} = \vec{r}\times \vec{F}\). Além disso, 
    \[
      \frac{d \vec{L}}{dt} = \frac{d \vec{r}}{dt}\times \vec{p} + \vec{r} \times \frac{d \vec{p}}{dt} = \vec{v}\times m \vec{v} + \vec{r}\times \frac{d \vec{p}}{dt}
    \]
  Mas, \(\frac{d \vec{p}}{dt} = \vec{F}\), tal que 
    \[
      \vec{r} \times \frac{d \vec{p}}{dt} = \vec{r}\times \vec{F} = \tau.
    \]
  Em outras palavras, 
    \[
      \vec{\tau } = \frac{d \vec{L}}{dt} = I \frac{d \vec{\omega }}{dt} = I \vec{\alpha }.
    \]
  Quando lidamos com movimento linear, havíamos definido a quantidade ``Impulso'' como a variação do momento linear. Fazemos o análogo aqui: 
    \[
      \Delta \vec{L} = \int_{t_{1}}^{t_{2}}\vec{\tau }dt.
    \]
  Com a partícula do início da seção, segue que 
    \[
      \vec{L} = m(v_{0}t\hat{i} + y_{0}\hat{j})\times v_{0}\hat{i} = -my_{0}v_{0}\hat{k}.
    \]
  Para uma partícula com movimento descrito por \(\vec{r}(t)\) no plano xyz, decompomos esse vetor como sendo 
    \[
      \vec{r} = \vec{r}_{rad} + \vec{r}_{z}.
    \]
  Aplicando uma força \(\vec{F}\), na mesma lógica, faremos 
    \[
      \vec{F} = \vec{F}_{xy} + \vec{F}_{z}
    \]
  Com isso, o torque é dado por 
  \[
    \vec{\tau } = \vec{r}\times \vec{F} = (\vec{r}_{rad} + \vec{r}_{z})\times(\vec{F}_{xy} + \vec{F}_{z})
  \]
  A componente na direção z do torque, \(\vec{\tau }_{z}\), pode ser encontrada no termo 
    \[
      \vec{\tau }_{z} = \vec{r}_{rad} + \vec{F}_{xy}.
    \]
  No entanto, observe que \(\vec{L} = \vec{r}\times (\vec{p}_{xy} + \vec{p}_{z}) = (\vec{r}_{rad} + \vec{r}_{z}) + (\vec{p}_{xy}+\vec{p}_{z})\), tal que,
chamando de \(\vec{L}_{z} = \vec{r}_{rad}\times \vec{p}_{xy}\) a componente z do momento angular, chegamos em 
  \[
    \vec{\tau }_{z} = \frac{d \vec{L}_{z}}{dt}.
  \]
  \begin{example}
    Considere a máquina de Atwood com dois corpos tais que \(m_{1} > m_{2}\). Sendo I o momento de inércia da polia, M sua massa e R seu raio, o momento angular total na direção z desse sistema será 
      \[
        L_{total_z} = m_{1}vR + m_{2}vR + I\omega.
      \]
    O torque em z desse sistema é 
      \[
        \tau_{z} = m_{1}gR - m_{2}gR = (m_{1}-m_{2})gR = \frac{dL_{z}}{dt}.
      \]
    Com isso, sendo \(I = \frac{1}{2}mR^{2}\) e \(a  = R\alpha \),
   \begin{align*}
     (m_{1}-m_{2})gR &= m_{1}aR + m_{2}aR + I\alpha \\
                     &= (m_{1}+m_{2}+\frac{1}{2}MR^{2})aR.\\
   \end{align*}
   Portanto, 
  \begin{align*}
    &a = \frac{(m_{1}-m_{2})g}{m_{1}+m_{2}+\frac{M}{2}}\\
    &\alpha = \frac{1}{R}\frac{(m_{1}-m_{2})g}{m_{1}+m_{2}+\frac{M}{2}}.\
  \end{align*}
  \end{example}
\newpage

\section{Aula 09 - 30/08/2023}
\subsection{Motivações}
\begin{itemize}
  \item Giroscópio;
  \item Conservação de Momento Angular.
\end{itemize}

\subsection{O Giroscópio}
  Considere uma barra apoiado a uma haste e passando por um disco de grande massa. Uma das forças atuando nele é a normal, atuando na barra, o peso, com valor M \(\vec{g}\) no centro de massa do disco
e considere \(\vec{r}_{cm}\) o raio da barra até o centro de massa. Considere que o disco tem um momento de inércia \(I_{s}\) e roda com velocidade angular \(\omega_{s}\). Por ele estar rodando,
há um momento angular \(\vec{L}\)
Calculando o torque nesse sistema, 
  \[
    \vec{\tau } = \vec{r}_{cm}\times M \vec{g},\quad |\vec{r}_{cm}| = D,
  \]
observa-se que ele faz um ângulo \(\theta\) que vale 90 graus com o momento angular. Passado um tempo \(\Delta t\), o torque estará
em \(\vec{\tau }\Delta t\) e L irá para \(L'\), mas, como o torque não pode mudá-lo em valor absoluto, \(|\vec{L}|=|\vec{L}'|\). Com isso, \(\vec{L}' = \vec{L} + \vec{dL}\),
em que \(\vec{dL} = \vec{\tau} dt\). No entanto, sabemos, nesse caso, quanto vale o torque, tal que 
  \[
    dL = MgDdt.
  \]
  O ângulo \(d\varphi \) entre L e L', também, será dado por \(d\varphi = \frac{dL}{L} = \frac{MgD}{L}dt = \frac{MgD}{I_{s}\omega_{s}}.\) A partir disso,
chamamos de velocidade de precessão o valor 
  \[
    \frac{d\varphi }{dt}= \omega_{p} = \frac{MgD}{I_{s}\omega_{s}}.
  \]
  ``Velocidade de precessão'' é o nome dado ao fenômeno responsável por fazer o momento angular ``seguir'' o torque quando o giroscópio está rodando - o movimento circular do eixo de rotação. Com isso,
vimos que o torque resultante externo é dado por 
  \[
    \vec{\tau }_{res_{ext}} = \frac{d \vec{L}}{dt},
  \]
  ou seja, quando não há torque externo, o momento angular é constante!

  Agora, assuma dois corpos de massas \(m_{1}, m_{2}\) que exercem forças \(\vec{F}_{12}\) e \(\vec{F}_{21}\) uma na outra. Além disso, coloque-as
a distâncias \(\vec{r}_{1}\) e \(\vec{r}_{2}\) de um referencial 0 (O desenho forma um triângulo). O torque total dessas forças será 
  \[
    \vec{\tau}_{T} = \vec{r}_{1}\times \vec{F}_{12} + \vec{r}_{2}\times \vec{F}_{21} = \vec{r}_{1}\times \vec{F}_{12}- \vec{r}_{2}\times \vec{F}_{12} = (\vec{r}_{1}-\vec{r}_{2})\times \vec{F}_{12},
  \]
  mas \(\vec{r}_{1}-\vec{r}_{2}\) é exatamente a distância entre os dois corpos, ou seja, a força é paralela a essa distância: \(\vec{r}_{1}-\vec{r}_{2}\parallel \vec{F}_{12}.\) 
  
 \begin{example}
   Suponha uma colisão inelástica entre corpos e imagine que o primeiro corpo tem momento de inércia \(I_{1}\), velocidade angular inicial \(\omega_i = \omega_{0}\),
   enquanto o segundo tem \(I_{2}, \omega _{i} = 0\). Após a colisão, quais são o momento de inércia e a velocidade angular finais?

 Aqui, o momento angular deve ser conservado - \(L_{i} = L_{f}\). Assim,
   \[
     I_{1}\omega_{0} = (I_{1}+I_{2})\omega_{f} \Rightarrow \omega_{f} = \frac{I_{1}}{I_{1}+I_{2}}\omega_{0}
   \]
 Quanto à energia cinética, 
   \[
     \mathbb{K} = \frac{1}{2}mv^{2} = \frac{m^{2}v^{2}}{2m} = \frac{p^{2}}{2m}
   \]
  Analogamente,
  \[
   \mathbb{K} = \frac{1}{2}I\omega^{2} = \frac{I^{2}\omega^{2}}{2I} = \frac{L^{2}}{2I}\\
  \]
  Aplicando isso aos corpos do exercício, 
 \begin{align*}
   &\mathbb{K}_{i} = \frac{1}{2}\frac{L^{2}}{I_{1}}\\
   &\mathbb{K}_{f} = \frac{1}{2}\frac{L^{2}}{I_{1}+I_{2}}\\
   \Rightarrow& \frac{\mathbb{K}_{i}}{\mathbb{K}_{f}} = \frac{I_{1}+I_{2}}{I_{1}} = 1 + \frac{I_{2}}{I_{1}}
 \end{align*}
\end{example}
\begin{example}
  Dado um disco preso a um ponto em seu centro, faça uma colisão inelástica dele com um objeto de massa m e velocidade v. O momento de inércia do disco é I, sua massa é M e seu raio é R.
Após a colisão, a massa gruda bem na borda do disco, passando a rodar com velocidade \(\omega \). Quanto vale essa velocidade e qual é a razão entre as energias cinéticas?

  Novamente, usando a conservação de momento angular, \(L_{i} = L_{f}\). Aqui, 
 \begin{align*}
   &L_{i} = mvR\\
   &L_{f} = (I+mR^{2})\omega\\
   \Rightarrow& mvR = (I+mR^{2})\omega\\
   \Rightarrow& \omega = \frac{mvR}{I+mR^{2}} = \frac{mRv}{mR^{2}(1+\frac{I}{mR^{2}})} = \frac{v}{R(1+\frac{I}{mR^{2}})}
 \end{align*}
  Sabemos de antes que 
 \begin{align*}
   &\mathbb{K}_{f}=\frac{L^{2}}{2I_{f}}\\
   &\mathbb{K}_{i} = \frac{1}{2}mv^{2}\\
   \Rightarrow& \frac{\mathbb{K}_{i}}{\mathbb{K}_{f}} = \frac{1}{2}\frac{mv^{2}}{\frac{1}{2}\frac{m^{2}v^{2}R^{2}}{I+mR^{2}}} = \frac{1}{mR^{2}}(I+mR^{2}) = 1 + \frac{I}{mR^{2}}
 \end{align*}
\end{example}

  Suponha que temos um eixo xyz, um objeto em movimento circular no plano xy com velocidade v e massa m, a uma distância \(\vec{R}\) do centro.
Escolhendo o centro das coordenadas, o momento angular \(\vec{L}\) será na direção z, já que 
  \[
    \vec{L} = \vec{R}\times m \vec{v} = \vec{R}\times \vec{p}.
  \]
  Considerando, por outro lado, um ponto abaixo do plano xy, formando um comprimento \(\vec{R}'\) até o círculo de antes, seguirá que \(\vec{L}'\), perpendicular a 
 \(\vec{R}'\), não será na direção z. Além disso, conforme a partícula precessiona, ele também moverá-se, formando um cone. Num sistema em que há uma outra partícula diametralmente
 oposta fazendo a mesma coisa, a soma dos dois momentos angulares apontaria, sim, para o eixo z.
\newpage

\section{Aula 10 - 31/08/2023}
\subsection{Motivações}
\begin{itemize}
  \item Exemplos para Trabalhar com os Conceitos.
\end{itemize}
\subsection{Exemplos}
\begin{example}
  Considere uma barra vertical de tamanho L, massa M e presa em um pivô. Além disso, tome uma partícula movendo-se em direção à barra vertical com velocidade \(\vec{v}\) e massa m.
Ela atinge a barra no ponto x e passa a fazer parte dela. Encontre a razão entre a energia cinética final e inicial do sistema e o valor da energia cinética após a colisão, dado
que a barra faz um ângulo máximo \(\theta_{max}\) após a colisão, deixando-a com uma nova altura de \(\frac{L}{2}\).

\textbf{Descrevendo o sistema:} 

  Num momento inicial, a energia cinética do sistema é dominada pela partícula, já que ela é a única em movimento. Sua energia cinética é calculável fazendo 
  \[
    \mathbb{K}_{i} = \frac{1}{2}mv^{2}.
  \]
  Com relação à barra, ela possui um certo momento angular no ponto x, valendo 
  \[
    L_{i} = mvx.
  \]

  Analisando o sistema após a colisão, chamando \(I_{t}\) como o momento de inércia total do sistema barra e partícula juntos, a energia cinética final
é descrita por 
  \[
    \mathbb{K}_{f} = \frac{1}{2}\frac{L^{2}}{I_{T}}
  \]
  e o momento angular final por 
  \[
    \vec{L_{f}} = I_{t}\omega_{f}.
  \]
\textbf{Desenvolvendo as contas:}

  Conseguimos explicitar uma forma para \(I_{t}\) como \(I_{t} = \frac{1}{3}ML^{2} + mx^{2}\). Já que o sistema é livre de forças externas, o momento angular deve ser conservado, isto é,
 \(L_{i} = L_{f}\). Podemos resolver isso para encontrar o valor de \(\omega_{f}:\)
 \begin{align*}
   &mvx = \biggl[\frac{1}{3}ML^{2} + mx^{2}\biggr]\omega_{f}\\
   \Rightarrow &\omega_{f} = \frac{mvx}{\biggl[\frac{1}{3}ML^{2}+mx^{2}\biggr]}
 \end{align*}

  Por fim, calculamos a razão entre as energias cinéticas
 \begin{align*}
   \frac{\mathbb{K}_{i}}{\mathbb{K}_{f}} &= \frac{\frac{1}{2}mv^{2}}{\frac{1}{2}\frac{L^{2}}{I_{t}}} = \frac{mv^{2}}{m^{2}v^{2}x^{2}}\frac{1}{3}(ML^{2}+mx^{2})\\
                                         &= \frac{1}{mx^{2}}\biggl[mx^{2}+\frac{1}{3}ML^{2}\biggr].
 \end{align*}
\textbf{Pós colisão:}
  Utilizando as informações dadas pelo exercício, podemos descrever a energia cinética final do sistema como 
    \[
      \mathbb{K}_{f}=Mg \biggl(\frac{L}{2}(1-\cos{(\theta_{max} )})\biggr) + mgx(1-\cos{(\theta_{max} )}),
    \]
    em que o termo de cosseno aparece manipulando as relações de tamanhos em triângulos.
\end{example}
\subsection{Disclaimer}
  Tiveram outros exemplos nessa aula. No entanto, devido à quantidade de desenho que eles exigem, optei por não colocá-los (talvez por não saber
descrever o sistema física em texto como costumo fazer nos desenhos, os que apareceram eram complexos demais pra mim).
\newpage

\section{Aula 11 - 11/09/2023}
\subsection{Motivações}
\begin{itemize}
  \item Aula de Exercícios.
\end{itemize}
\subsection{Exercícios}
\subsubsection{Exercício 1}
  Considere um plano inclinado com ângulo \(\theta \), sobre o qual há um objeto de massa M, raio R e momento de inércia I.
Este objeto está posicionado a uma altura h em relação ao chão. Qual é a velocidade do centro de massa após o objeto rolar sem deslizar pelo plano?

  Começamos com o princípio da conservação de energia - \(E_{mec_{i}} = E_{_mec_{f}}\). Para esse sistema, a energia mecânica inicial
vale Mgh, enquanto que a final vale 
  \[
    E_{mec_{f}} = \frac{1}{2}Mv_{cm}^{2} + \frac{1}{2}I\omega^{2}.
  \]
  Colocando os valores na equivalência, obtemos 
    \[
      Mgh = \frac{1}{2}Mv_{cm}^{2} + \frac{1}{2}I\omega^{2}.
    \]
  Sabe-se que, quando o disco rola sem deslizar, \(v_{cm} = R\omega \). Com isso,
 \begin{align*}
   &2Mgh = Mv_{cm}^{2} + I \frac{v_{cm}^{2}}{R^{2}}\\
   \Rightarrow &2gh = v_{cm}^{2}\biggl[1 + \frac{I}{MR^{2}}\biggr]\\
   \Rightarrow &v_{cm}^{2} = \frac{2gh}{\biggl[1 + \frac{I}{MR^{2}}\biggr]}.
 \end{align*}
 Para o aro, \(I = MR^{2},\) ou seja, 
   \[
     v_{cm}^{2} = \frac{2gh}{2} \Rightarrow v_{cm}=\sqrt[]{gh}.
   \]
  Quando é um disco, \(I=\frac{1}{2}MR^{2}\), tal que 
    \[
      v_{cm}^{2} = \frac{2gh}{1 + \frac{1}{2}} = \frac{4gh}{3} \Rightarrow v_{cm} = \frac{2}{\sqrt[]{3}}\sqrt[]{gh}
    \]
  Assim, \(v_{cm_{disco}} > v_{cm_{aro}}\)
    \[
      \frac{v_{cm_{disco}}}{v_{cm_{aro}}} = \frac{2}{\sqrt[]{3}}.
    \]
  Elaborando, sabemos que 
    \[
      E_{mec_{f}} = \mathbb{K}_{trans} + \mathbb{K}_{rot}.
    \]
  Essas duas quantidades relacionam-se com 
    \[
      \frac{\mathbb{K}_{trans}}{\mathbb{K}_{rot}} = \frac{1}{2}\frac{Mv_{cm}^{2}}{\frac{1}{2}I\omega^{2}} = \frac{Mv_{cm}^{2}}{I\omega^{2}}.
    \]
  Como \(v_{cm}=R\omega \), 
    \[
      \frac{\mathbb{K}_{tran}}{\mathbb{K}_{rot}} = \frac{Mv_{cm}^{2}}{I\omega^{2}}=\frac{Mv_{cm}^{2}}{I \frac{v_{cm}^{2}}{R^{2}}} = \frac{MR^{2}}{I}.
    \]
  Para o aro, isto torna-se 
    \[
      \frac{\mathbb{K}_{trans}}{\mathbb{K}_{rot}} = 1
    \]
  e, para o disco, 
    \[
      \frac{\mathbb{K}_{trans}}{\mathbb{K}_{rot}} = 2
    \]
\subsubsection{Exercício 2}
  Considere uma situação em que um disco de raio R tem um buraco, de raio \(\frac{R}{2}\), centrado no eixo x. Buscamos o momento de inércia do disco com relação ao eixo z. 
Dado que o sistema tem massa M, começamos definindo a densidade superficial de massa 
  \[
    \sigma = \frac{M}{\pi \biggl(R^{2}-\frac{R^{2}}{4}\biggr)} = \frac{4M}{3\pi R^{2}}.
  \]
Sabemos que 
  \[
    I_{z} = I_{z}^{+} + I_{z}^{-} 
  \]
  e que 
  \[
    I_{z} = \frac{\pi R^{2}\sigma R^{2}}{2} - \frac{\pi R^{2}}{4}\frac{\sigma }{2}\frac{R^{2}}{4} - \frac{M'R^{2}}{4},
  \]
  em que \(M'\) é a massa ``removida'' do buraco. Assim, 
  \begin{align*}
    I_{z} &= \frac{\pi R^{4}\sigma }{2} - \frac{\pi R^{4}\sigma }{2}\frac{1}{16} - \frac{\pi R^{2}\sigma }{4}\frac{R^{2}}{4}\\
    &= \frac{\pi \sigma R^{4}}{2}\biggl[1-\frac{1}{16}-\frac{2}{16}\biggr] = \frac{2}{3}MR^{2}\biggl(\frac{13}{16}\biggr)\\
    &= \frac{13}{24}MR^{2}.
  \end{align*}
  Usando que \(I_{z} = I_{x} + I_{y},\) podemos encontrar 
 \begin{align*}
   &I_{x} = \frac{M^{+}R^{2}}{4} - \frac{M^{-}}{4}\frac{R^{2}}{4}\\
   &I_{y} = \frac{M^{+}R^{2}}{4} - \frac{M^{-}}{4}\frac{R^{2}}{4} - M^{-}\frac{R^{2}}{4}.
 \end{align*}
\subsubsection{Exercício 3}
  Considere um ioiô - um objeto circular de peso Mg, raio R, aceleração vertical \(\vec{a},\) tração \(\vec{T}\) e momento de inércia I. Quanto vale a aceleração e a tração?

  Sabe-se pelo diagrama de forças que \(TR = I\alpha \), já que a tração é a única força aqui que pode gerar torque, e 
    \[
      Mg - T = Ma,\quad a = \alpha R.
    \]
    Somando as duas coisas, 
      \[
        Mg = Ma + \frac{I\alpha }{R} \Rightarrow Mg = Ma + \frac{Ia}{R^{2}}.
      \]
    Com isso, 
      \[
        Mg = Ma \biggl[1 + \frac{I}{MR^{2}}\biggr] \Rightarrow a = \frac{g}{\biggl[1 + \frac{I}{MR^{2}}\biggr]}.
      \]
    Portanto, a tração vale 
      \[
        T = \frac{I\alpha }{R} = \frac{I}{R^{2}}\frac{g}{\biggl[1 + \frac{I}{MR^{2}}\biggr]}.
      \]
\subsubsection{Exercício 4}
  Novamente, considerando um ioiô, apoie-o sobre uma mesa. Dessa vez, o ioiô é composto de 3 cilindros curtos, tal que a parte interna desse disco tem um raio r e,
nesse disco de dentro, há uma corda aplicando uma força \(F\) (O ioiô de perfil, enrolado, parece um H). Qual é o ângulo que ele deve ser lançado para rolar
sem deslizar?

  A pergunte equivale a achar o ângulo pelo qual a força deve passar sem que haja torque no ponto de apoio. Com isso, percebe-se que 
    \[
      \frac{r}{R} = \cos{(\theta )}.
    \]
    Portanto, \(\theta = \arccos{\biggl(\frac{r}{R}\biggr)}\)
\subsubsection{Exercício 5}
  Considere um sistema de dois discos, ambos com massa m, mas o segundo com o raio duas vezes maior que o primeiro. Posicionando-os centrados no mesmo eixo vertical, mas
o menor acima do primeiro, considere que ele roda com velocidade \(\omega_{0}\) no sentido horário e que o maior roda, também com velocidade \(\omega_{0}\), no sentido anti-horário. 
Postulando que o primeiro tem momento angular negativo e, o segundo, positivo, assume-se que os discos colidem. Qual será a frequência de rotação após a colisão?
E a razão entre \(\mathbb{K}_{inicial}\) e \(\mathbb{K}_{final}\)?

  Pela conservação de momento angular, sabe-se que 
    \[
      L_{i} = L_{f}.
    \]
  Além disso, 
  \[
  L_{i} = L^{+}-L^{-} = I_{1}\omega_{0} - I_{2}\omega_{0} = \frac{1}{2}m(2r^{2})\omega_{0} - \frac{1}{2}mr^{2}\omega_{0} = \frac{3}{2}mr^{2}\omega_{0}
  \]
  Com relação ao momento angular final, 
 \begin{align*}
   L_{f} &= I_{f}\omega_{f} = (I_{2} + I_{1})\omega_{f} = \biggl(\frac{1}{2}m4r^{2}+\frac{1}{2}mr^{2}\biggr)\omega_{f}\\
         &=\frac{5}{2}mr^{2}\omega_{f}.
 \end{align*}
 Portanto, 
   \[
     \frac{3}{2}mr^{2}\omega_{0} = \frac{5}{2}mr^{2}\omega_{f} \Rightarrow \omega_{f}=\frac{3}{5}\omega_{0}.
   \]
  Agora, vejamos a questão das energias cinéticas. A inicial vale 
    \[
      \mathbb{K}_{i} = \frac{1}{2}I_{1}\omega_{0}^{2} + \frac{1}{2}I_{2}\omega_{0}^{2} = \frac{1}{2}\omega_{0}^{2}\frac{5}{2}mr^{2}. 
    \]
  A final, por outro lado, é dada por 
    \[
      \mathbb{K}_{f} = \frac{1}{2}(I_{1}+I_{2})\omega_{f}^{2} = \frac{1}{2}\frac{5}{2}mr^{2}\frac{9}{25}\omega_{0}^{2} = \frac{1}{4}\frac{9}{5}mr^{2}\omega_{0}^{2}.
    \]
  Dividindo um valor pelo outro, 
    \[
      \frac{\mathbb{K}_{i}}{\mathbb{K}_{f}} = \frac{5}{4}\frac{1}{\frac{9}{5}\frac{1}{4}} = \frac{25}{9}.
    \]
\newpage

\section{Aula 12 - 18/09/2023}
\subsection{Motivações}
\begin{itemize}
  \item Fluídos;
  \item Pressão e Densidade.
\end{itemize}
\subsection{Fluídos e Massa Específica}
  Existem dois tipos de fluidos. Um líquido sempre ocupa a parte mais baixa do recipiente, enquanto que um gás 
sempre ocupa todo o espaço do recipiente. Utilizamos líquidos para coisas como geração de energia, aerodinâmica e medicina,
e há a presença de ligações atômicas/moleculares. 
  
Definimos a densidade por 
  \[
    \rho = \frac{M}{V} = \frac{dm}{dV}.
  \]
  Para uma noção mais mundana, a densidade da água vale \(\rho = 1 \frac{g}{cm^{3}} = 100 \frac{kg}{m^{3}} = 1 \frac{kg}{l}\). Por outro lado, estrelas de neutrons 
têm densidade de \(\rho = 7\times 10^{17}\frac{kg}{m^{3}}\). Caso um objeto na água tenha densidade \(\rho > \rho_{H_{2}O},\) ele afundará e, se
 \(\rho < \rho_{H_{2}O}\), o objeto flutuará.
\begin{example}
  Considere um recipiente com 200ml de água a 4 graus Celsius. Esquentemos este líquido até 80 graus Celsius.
Para 6g de \(H_{2}O\) perdidos, qual será a nova densidade da água? 
  \[
    \rho = \frac{(200 - 6)g}{200ml}\approx 0.97 \frac{g}{l} = 970 \frac{kg}{m^{3}}.
  \]
\end{example}
  Outra noção importante é a de pressão, definida como 
    \[
      P = \frac{F}{A}.
    \]
  A unidade padrão da pressão é o Pascal, \([P]= \frac{N}{m^{2}} =\) Pascal. Outras unidades incluem o psi, 
  ou libra por polegada quadrada, e a atmosfera, ou atm. Segue que 
    \[
      1atm = 101325 Pa = 14.7 \frac{lb}{in^{2}}.
    \]
  Nossa intuição diz que, ao aplicarmos \(\Delta P\), também estaremos causando uma variação de volume \(\Delta V\), mas negativo.
Por isso, falamos de módulo volumétrico, definido como 
  \[
    \beta = \frac{-\Delta P}{\frac{\Delta V}{V}}.
  \]
  A compressibilidade é definida pelo inverso do módulo volumétrico: 
    \[
      \frac{1}{\beta } = \frac{\frac{\Delta V}{V}}{\Delta P} = \frac{-\Delta V}{V\Delta P}.
    \]
  Neste curso, trabalharemos apenas com líquidos não compressíveis.

  Considere um disco de área A, submerso a uma profundidade z em um líquido de densidade \(\rho \).
Este disco está sujeito a uma pressão P(z) por cima e P(z+dz) por baixo, em que dz é a variação de profundidade que ele sofre.
Além disso, há a força peso \(dm \vec{g}\). Utilizando que \(F = P A,\) obtemos a equação 
  \[
    \biggl[P(z+dz) - P(z)\biggr]A = dmg = Adz\rho g
  \]
Com isso, chegamos em 
  \[
    P(z+dz) - P(z) = \rho g dz \Rightarrow \frac{dP}{dz} = \rho g.
  \]
  Integrando com respeito a z, encontramos a relação da pressão conforme submergimos num líquido: 
    \[
      \hypertarget{pressure_submerging}{\boxed{P(z) = P_{0}+\rho gz.}}
    \]
\newpage

\section{Aula 13 - 20/09/2023}
\subsection{Motivações}
\begin{itemize}
  \item Pressão em Fluídos e Paradoxo Hidrostático;
  \item Variação de Pressão;
  \item Princípio de Arquimedes.
\end{itemize}
\subsection{Pressão}
\begin{example}
  Considere uma barragem de tamanho L e pressão inicial de 1 atm. Essa barragem impede um líquido com distância da terra até ele sendo h.
Qual é a variação da força com a profundidade z?

  Sabemos da aula anterior a fórmula da \textit{\hyperlink{pressure_submerging}{pressão com profundidade}}. Vimos, também, a relação da pressão com a força
como sendo \(P(z)dA = dF\). Unindo-as, tem-se 
  \[
    dF = (P_{0}+\rho gz)Ldz \Rightarrow F = \int_{0}^{h}P_{0}Ldz + \rho gL \int_{0}^{h}z dz.
  \]
  Portanto, 
    \[
      F = P_{0}Lh + \rho gL \frac{z^{2}}{2}\biggl|_{0}^{h}\biggr. = P_{0}Lh + \rho gL \frac{h^{2}}{2}.
    \]
  Usualmente, barragens no mundo real não são simplesmente retas, possuindo uma curvatura na base conforme aprofunda-se. Podemos descrever 
a força que essa base curvada sofrerá fazendo \(\vec{dF}' = dS L P_{0}\). Como essa forçá faz um ângulo \(\theta \) com a horizontal, vale também
\(dF'_{x} = dF'\cos{(\theta )} = LP_{0}ds\cos{(\theta )} = LP_{0}dl.\) Podemos integrar com respeito ao tamanho para chegar em 
   \[
     F_{x}' = \int_{0}^{h}LP_{0}dl = LP_{0}h.
   \]
 Assim, a força que o fluído exerce nessa barreira é dado pela diferença dos dois resultados 
   \[
     F - F_{x}' = \rho g \frac{h^{2}}{2}L.
   \]
\end{example}
\begin{example}
  Considerando uma caverna subaquática com um ponto \(P_{1}\) próximo ao topo da coluna d'água,
outro \(P_{2}\) na reta vertical de \(P_{1}\), mas mais profundo e \(P_{3}\) na reta horizontal de \(P_{3}\) 
(formam um triângulo), como as diferentes pressões relacionam-se?

  Para os pontos 1 e 2, note que, por estarem na mesma vertical, a força atuando neles é a gravitacional apenas, ou seja, 
    \[
      P_{2} - P_{1} = \frac{mg}{A} = \frac{\rho Ahg}{A} = \rho hg.
    \]
  Para os pontos 2 e 3, eles estão na mesma altura, então não há forças a serem compensadas, ou seja, \(P_{2} = P_{3}\).
\end{example}
  O fenômeno da pressão em um fluído não depender do formato do recipiente que o contém é conhecido como paradoxo da hidrostática (recomendo pesquisar na internet pra ver uma imagem).

\subsection{Medindo Pressões}
\subsubsection{Pressão Barométrica}
  Considere um recipiente sob pressão P conectado uma mangueira com um fluído dentro e uma diferença entre as colunas d'água
do topo da mangueira ao ponto que o líquido para dada por h. Suponha, também, que, fora do sistema, há uma pressão atmosférica \(P_{atm}\).
Então, a pressão do recipiente será dada por 
  \[
    P = P_{atm} + \rho gh
  \]
\subsubsection{Barômetro de Mercúrio}
  Tome um tubo de vidro submerso em mercúrio líquido, fazendo com que o mercúrio desça. A parte que o mercúrio não alcança fica um vácuo, ou seja, com pressão 0.
Com isso, é possível calcular a pressão do exterior do vidro utilizando o tamanho da coluna de mercúrio, h: 
  \[
    P = \rho_{Hg}gh.
  \]
  Utilizando este método no nível do mar, foi possível estimar que \(1atm = 760mmHg\).

\subsection{A Pressão Variando com Altura}
  Vamos chamar de z uma certa altura de atmosfera. Consideramos uma parte dela, sofrendo pressão P(z) por baixo e P(z+dz) por cima. Essa parte tem área A, espessura dz e sofre
da força gravitacional por mg. Analisando esse sistema, 
  \[
    [P(z) - P(z+dz)]A = mg.
  \]
  Utilizando \(m = Adz\rho \), temos 
    \[
      -[P(z+dz)-P(z)]A = Adz\rho g \Rightarrow dP = -\rho g dz.
    \]
  Relembrando o caso dos gases, 
    \[
      Pv = NkT \Rightarrow P = \rho kT \Rightarrow \frac{\rho }{P} = \frac{1}{kT} \Rightarrow \frac{\rho }{P}=\frac{\rho_{0}}{P_{0}},
    \]
  em que \(\rho_{0}\) e \(P_{0}\) são os valore no nível do mar. Assim, \(\rho = \rho_{0} \frac{P}{P_{0}}\) e, utilizando a relação descoberta, 
    \[
      dP = -\frac{\rho_{0}}{P_{0}}gPdz.
    \]
  Integrando esse resultado, 
    \[
      \int_{P_{0}}^{P}\frac{dP}{P} = \frac{\rho_{0}}{P_{0}}\int_{0}^{z}gdz \Rightarrow \ln{(P)}-\ln{(P_{0})} = -\frac{\rho_{0}}{P_{0}}gz.
    \]
  Manipulando e resolvendo para P, chega-se em 
    \[
      \hypertarget{pressure_decay}{\boxed{P = P_{0}e^{-\frac{\rho_{0}gz}{P_{0}}}}}
    \]
    Em outras palavras, o decaimento da pressão com a altura é exponencial.
\begin{example}
  Dado que \(\rho_{0} = 1.23\frac{kg}{m^{3}}\) é a densidade do ar e \(P_{0} = 1atm\), a altura necessária para a pressão valer metade do valor de \(P_{0}\) é \(h = 5.5km\)
Numa altura de um avião comercial, isto é, \(h=10km\), a pressão é \(P = \frac{P_{0}}{4}\).
\end{example}

\subsection{Princípio de Arquimedes}
  Ao entrar na água, nota-se que uma pessoa com um peso \(\vec{P}\) registrará numa balança um valor \(\vec{P}'\) menor que o registrado em terra. Isto deve-se graças ao empuxo,
a reação que o líquido gera à força gravitacional. Para calcular este empuxo, toma-se a diferença entre a força peso e o peso aparente: 
  \[
    \vec{E} = \vec{F}_{g}-\vec{F}_{gap}.
  \]
  Quando não tem-se uma balança, podemos descrever o empuxo levando em conta, primeiramente, que o empuxo age em sentido oposto ao peso. Assim, 
  \[
    \vec{E} = \vec{P} = mg.
  \]
  Na expressão inicial, o peso aparente recebe o valor de \(\vec{F}_{gap}=\vec{F}_{g} - E\).
 \begin{example}
   Para um anel de ouro com peso \(F_{g} = 0.158N\), após ser submerso, aparentou um peso aparente de \(\vec{F}_{gap}=0.15N\). Considerando que
  \(\rho_{Au} = 19.3\frac{g}{cm^{3}},\) determine se o material do anel é realmente ouro.

  Segue que, sendo \(m_{f}, \rho_{f}\) massa e densidade do fluído
    \[
      \frac{E}{F_{g}} = \frac{m_{f}g}{m_{A}g} = \frac{\rho_{f}V}{\rho_{Au}V} = \frac{\rho_{f}}{\rho_{Au}}.
    \]
  Com isso, 
    \[
      \rho_{Au} = \rho_{f}\frac{F_{g}}{E} = \rho_{H_{2}O}\frac{F_{g}}{F_{g}-F_{gap}}=\rho_{H_{2}O}\frac{0.158}{0.158-0.15}\approx 20\frac{g}{cm^{3}}.
    \]
 \end{example}
\begin{example}
  Após submergir uma pessoa, seu peso aparente é \(5\%\) do peso original. Considerando que a densidade da gordura é de
 \(\rho_{g} = 0.9 \frac{g}{cm^{3}}\) e a densidade do resto é \(\rho_{r} = 1.1 \frac{g}{cm^{3}}\), qual é a massa de gordura da pessoa?

  Temos 
    \[
      \frac{\rho }{\rho_{H_{2}O}} = \frac{F_{g}}{F_{g}-F_{gap}} = \frac{F_{g}}{(1-0.05)F_{g}}\approx 1.05.
    \]
    Considerando que o volume total é a soma do volume da gordura com o do resto, segue que 
      \[
        \frac{m_{T}}{\rho }= \frac{m_{g}}{\rho_{g}}+\frac{m_{r}}{\rho_{r}}.
      \]
    Como 
      \[
         \left.\begin{array}{ll}
            m_{g} = f_{g}m_{T}\\
            m_{r} = f_{r}m_{T}
          \end{array}\right\},
      \]
    obtemos \((m_{g}+m_{r}) = (f_{g}+f_{r})m_{T}.\) Desta forma, \(f_{g}+f_{r}=1\) e
   \begin{align*}
     &\frac{m_{T}}{\rho } = \frac{f_{g}m_{T}}{\rho_{g}} + \frac{f_{r}m_{T}}{\rho_{r}}\\
     &\frac{1}{\rho }=\frac{1}{\rho_{g}}f_{g} + \frac{(1-f_{g})}{\rho_{r}}\\
     &\frac{1}{\rho }-\frac{1}{\rho_{r}}=\biggl(\frac{1}{\rho_{g}}-\frac{1}{\rho_{r}}\biggr)f_{g}\\
     &\frac{\frac{1}{\rho_{r}}\biggl(\frac{\rho_{r}}{\rho }-1\biggr)}{\frac{1}{\rho_{r}}\biggl(\frac{\rho_{r}}{\rho_{g}}-1\biggr)} = \rho_{g}\approx0,21.
   \end{align*}
\end{example}
\newpage

\section{Aula 14 - 21/09/2023}
\subsection{Motivações}
\begin{itemize}
  \item Exemplos de Empuxo;
  \item Hidrodinâmica.
\end{itemize}
\begin{example}
  Tome 5 Beckeres iguais. No primeiro, há um barquinho flutuando. No segundo, o barquinho afundou e tem o dobro da densidade. No terceiro,
coloca-se um bloco de gelo flutuando. No quarto, o bloco de gelo está preso e possui metade da densidade. No último, não faz-se nada. Como são os pesos?
Suponha que todos os objetos colocados têm o mesmo peso \(F_{g_{o}}\)

\textbf{Becker E:}
  O peso de E é fácil, já que não têm nada, utilizaremos ele como base, sendo seu peso \(F_{g_{E}}\).  

\textbf{Becker A:}
  Inicialmente, antes de inserir o barco, \(F_{g_{A}} = F_{g_{E}}\). Após a adição dele, o deslocamento de líquido equivale ao peso do barco,
ou seja, \(F_{g_{A}} = F_{g_{E}}-F_{g_{O}}.\) Como o barco está agora incluso, adiciona-se seu peso novamente, donde obtemos 
  \[
    F_{g_{A}} = F_{g_{E}} - F_{g_{O}} + F_{g_{O}} = F_{g_{E}}.
  \]

\textbf{Becker B:}
  Como o barco tem o dobro da densidade, mas a massa é a mesma, o volume do barco é metade do primeiro caso, tal que 
    \[
      F_{g_{B}} = F_{g_{E}} - \frac{F_{g_{O}}}{2} + F_{g_{O}} = F_{g_{E}} + \frac{F_{g_{O}}}{2}
    \]

  \textbf{Becker C:}
    O raciocínio do A vale para este, donde obtemos 
      \[
        F_{g_{C}} = F_{g_{E}} - F_{g_{O}} + F_{g_{O}} = F_{g_{E}}.
      \]

  \textbf{Becker D:}
    Analogamente ao B, como a massa é a mesma, mas a densidade é metade, o volume deslocado é duas vezes maior, ou seja, 
      \[
        F_{g_{D}} = F_{g_{E}} - 2 F_{g_{O}} + F_{g_{O}} = F_{g_{E}} - F_{g_{O}}.
      \]
\end{example}
\begin{example}
  Dado que um iceberg está no mar flutuando, calcule a fração dele que está visível.

  A força gravitacional do iceberg é dada pela relação 
    \[
      F_{g_{ic}} = \rho_{gelo}Vg.
    \]
  Sabemos, também, que \(E = F_{g_{ic}}\). Assim, 
    \[
      \rho_{mar}V_{submerso}g = \rho_{gelo}Vg \Rightarrow \frac{V_{submerso}}{V} = \frac{\rho_{gelo}}{\rho_{mar}} = \frac{0.9\frac{g}{cm^{3}}}{1.022\frac{g}{cm^{3}}}.
    \]
  Portanto, \(\frac{V_{sub}}{V}\approx 0.9\).
\end{example}
\subsection{Hidrodinâmica}
  Pegue um tubo, com área na parte maior \(A_{1}\) que estreita-se um pouco conforme é percorrido, formando quase uma garrafa, com a ponta mais estreita tendo área \(A_{2}\).
Preencha-o com um fluído. Na parte não estreitada, o fluído tem velocidade \(\vec{v}_{1}\) e, após chegar na parte mais estreita, passa a ter velocidade \(\vec{v_{2}}\).
A massa do fluído que passa pela primeira parte do percurso - antes de estreitar - a cada instante de tempo, então, será 
  \[
    \Delta m_{1} = \rho_{1}A_{1}v_{1}\Delta t
  \]
  e, pela segunda parte, 
  \[
    \Delta m_{2} = \rho_{2}A_{2}v_{2}\Delta t
  \]
  Chamamos as vazões de massas nos pontos respectivos por 
    \[
      \text{vazão massica} = \left\{\begin{array}{ll}
          \frac{\Delta m_{1}}{\Delta t} = I_{m_{1}}\\
          \frac{\Delta m_{2}}{\Delta t} = I_{m_{2}}.
        \end{array}\right.
    \]
  A quantidade de massa passando entre os pontos pode, assim, ser obtida calculando-se a diferença entre as duas vazões acima,
resultando numa equação conhecida como \textit{equação da continuidade}:
    \[
      \hypertarget{continuity_equation}{\boxed{\frac{dm_{12}}{dt} = I_{m_{1}}-I_{m_{2}}}}
    \]
  Observe que, se \(\frac{dm_{12}}{dt} = 0\), então \(I_{m_{1}} = I_{m_{2}}\), ou seja, \(A_{1}v_{1}=A_{2}v_{2}\). Nestes casos, chamamos de vazão
volumétrica o valor \(I_{v}=Av\).
\newpage

\section{Aula 15 - 25/09/2023}
\subsection{Motivações}
 \begin{itemize}
   \item Equação de Bernoulli;
   \item Tubo de Venturi;
   \item Tubo de Pitot.
 \end{itemize}
\subsection{Pressão}
  Imagine um tubo que começa a subir na vertical e está preenchido por um fluído. Considere um ponto 1 na parte mais inferior deste tubo, com uma altura \(y_{1}\),
e um ponto 2 na parte superior do mesmo, a uma altura \(y_{2} > y_{1}\). No ponto 1, hão velocidade e pressão \(v_{1}, P_{1}\) e, respectivamente no ponto 2, \(v_{2}, P_{2}\).
Passam, por esses pontos, quantidades de massa \(\Delta m\). Vamos analisar esse sistema.

  Sabe-se que \(W = \Delta \mathbb{K}\). As forças que realizam trabalho, nesta configuração, são a pressão e a gravidade. Considerando as quantidades em questão, 
    \[
      \Delta \mathbb{K} = \frac{1}{2}\Delta m v_{2}^{2} - \frac{1}{2}\Delta m v_{1}^{2}
    \]
Quanto ao trabalho, também, para a gravidade, tem-se
\[
  W_{g} = \Delta mgy_{1} - \Delta mgy_{2}
\]
e, para as pressões, 
  \[
    W_{p_{1}} = F_{1}\Delta x_{1} = p_{1}A_{1}\Delta x_{1} = p_{1}\Delta V\quad\&\quad W_{p_{2}} = -p_{2}\Delta V,
  \]
totalizando 
  \[
    W_{p_{t}} = W_{p_{1}} + W_{p_{2}} = (p_{1}-p_{2})\Delta V.
  \]
  Com isso, obtemos 
  \[
    W_{g}+W_{p_{t}} = \Delta \mathbb{K}.
  \]
  Podemos, então, destrinchar a equação como segue
 \begin{align*}
   &\Delta mgy_{1}-\Delta mgy_{2} + (p_{1}-p_{2})\Delta V = \frac{1}{2}\Delta mv_{2}^{2}-\Delta mv_{1}^{2}\\
  \Longleftrightarrow\quad  & \frac{1}{2}\Delta mv_{1}^{2} + p_{1}\Delta V + \Delta mgy_{1} = \frac{1}{2}\Delta mv_{2}^{2} + p_{2}\Delta v + \Delta mgy_{2}\\
  \underbrace{\Longrightarrow}_{\Delta m=\rho \Delta V}         & \frac{1}{2}\rho v_{1}^{2} + p_{1} + \rho gy_{1} = \frac{1}{2}\rho v_{2}^{2} + p_{2} + \rho gy_{2}.
 \end{align*}
 Em outras palavras, estes valores das igualdades são \textbf{constantes}, o que nos leva à equação de Bernoulli: 
   \[
     \hypertarget{bernoulli}{\boxed{\frac{1}{2}\rho v_{1}^{2}+p_{1}+\rho gy_{1}=constante}}
   \]
   O problema com ela é que ela descreve apenas sistemas idealizados, que muitas vezes não ocorrem na vida real. Por exemplo, 
ela não leva em conta a viscosidade do líquido, o que resulta em inconsistências no mundo real.
\begin{example}
  Considere um tanque preenchido com um fluído até uma altura \(y_{1}\), com área de abertura \(A_{1}\). Neste tanque, há um buraco
 a uma altura \(y_{2}\) menor que \(y_{1}\) e pressão \(A_{2}\). Suponhamos que \(A_{1}>> A_{2}\). Utilizando a equação 
   \[
     A_{1}v_{1} = A_{2}v_{2},
   \]
  concluímos que \(v_{1}\approx0\). Assim, utilizando Bernoulli, 
    \[
      p_{1} + \rho gy_{1} + \underbrace{\frac{1}{2}\rho v_{1}^{2}}_{0} = p_{2} + \rho gy_{2} + \frac{1}{2}\rho v_{2}^{2}. 
    \]
  Como as pressões são as mesmas, 
    \[
      \frac{1}{2}\rho v_{2}^{2} = \rho g(y_{1}-y_{2}) \Rightarrow v_{2}^{2} = 2g\Delta y \Rightarrow v_{2}=\sqrt[]{2g\Delta y}
    \]
  Até mesmo sem fazer a suposição das áreas, sabe-se que 
    \[
      v_{1} = \frac{A_{2}}{A_{1}}v_{2},
    \]
  tal que 
    \[
      \rho gy_{1} + \frac{1}{2}\rho \frac{A_{2}^{2}}{A_{1}^{2}}v_{2}^{2} = \frac{1}{2}\rho v_{2}^{2} + \rho gy_{2},
    \]
  donde segue que 
    \[
      g(y_{1}-y_{2}) = \frac{v_{2}^{2}}{2}\biggl(1-\frac{A_{2}^{2}}{A_{1}^{2}}\biggr),
    \]
  ou seja, 
    \[
      v_{2}^{2} = \frac{2g\Delta y}{1-\frac{A_{2}^{2}}{A_{1}^{2}}} \Rightarrow v_{2} = \sqrt[]{\frac{2g\Delta y}{1-\frac{A_{2}^{2}}{A_{1}^{2}}}}.
    \]
\end{example}
\subsection{Tubo de Venturi}
  O tubo de Venturi é um aparato para medir a velocidade de um certo fluído (recomendo olhar uma imagem no Google!), baseado no princípio de Bernoulli.

Ele consiste de três seções principais, uma de entrada, uma de ``garganta'' - um caminho estreito ligando a primeira e a terceira seção, e a terceira, a seção de saída.
Conecta-se a garganta e a primeira seção, além da abertura delas, com uma coluna de líquidos em formato de U.

Dado que a área das seções um e três é \(A_{1}\), a pressão é \(p_{1}\), a densidade é \(\rho_{f}\)
e a velocidade é \(v_{1}\) e que, na garganta, \(A_{2}, v_{2}, p_{2}\) são os valores respectivos, 
temos duas equações descrevendo este sistema, \hyperlink{bernoulli}{Bernoulli}
e \(A_{1}v_{1} = A_{2}v_{2}.\) Através de ambas, obtemos 
\begin{align*}
  &p_{1}+\frac{1}{2}\rho_{f}v_{1}^{2} = p_{2} + \frac{1}{2}\rho_{f}v_{2}^{2}\\
  &\frac{1}{2}\rho_{f}(v_{1}^{2}-v_{2}^{2}) = p_{2}-p_{1}\\
  \Longrightarrow &\frac{1}{2}\rho_{f}\biggl(1-\frac{A_{1}^{2}}{A_{2}^{2}}\biggr)v_{1}^{2} = p_{2}-p_{1}\\
  \Longleftrightarrow &\frac{1}{2}\rho_{f} \biggl(\frac{A_{1}^{2}}{A_{2}^{2}}-1\biggr)v_{1}^{2} = p_{1}-p_{2}.
\end{align*}
  Sendo \(\rho_{L}\) a densidade do líquido dentro do tubo em U e h a diferença da altura das pontas do líquido, aplicamos Bernoulli a ele e à seção 1, tal que 
    \[
      p_{1}+\rho_{F}gh = p_{2}+\rho_{L}gh \Rightarrow p_{1}-p_{2} = gh(\rho_{L}-\rho_{f}).
    \]
  Chamando de \(r = \frac{A_{1}}{A_{2}}\) e juntando as duas equações, 
    \[
      \frac{1}{2}\rho_{f}(r^{2}-1)v_{1}^{2} = gh(\rho_{L}-\rho_{f}).
    \]
  Portanto, 
    \[
      v_{1}^{2} = \frac{2gh(\rho_{L}-\rho _{f})}{(r^{2}-1)\rho_{f}}.
    \]
\subsection{Tubo de Pitot}
  O tubo de Pitot é um instrumento utilizado para medir a velocidade de um fluido
em um ponto específico de um duto ou fluxo livre. Ele foi nomeado em homenagem
ao engenheiro francês Henri Pitot, que o introduziu no início do século XVIII. 
O tubo de Pitot é amplamente utilizado em aplicações aeronáuticas para medir a 
velocidade do ar em relação a aeronaves, mas também tem aplicações em outros
campos da engenharia.

  Ele é composto por um tubo externo chamado tubo estático e um tubo interno
chamado tubo de impacto ou tubo dinâmico. A extremidade aberta do tubo interno
é direcionada diretamente para o fluxo do fluido, enquanto as aberturas laterais
do tubo externo medem a pressão estática do fluido. 

  Quando o fluido atinge a extremidade aberta do tubo de impacto, ele é trazido
ao repouso (velocidade zero). A pressão medida neste ponto é chamada de pressão 
de impacto ou pressão total, denotada por \(p_{1}\). As aberturas laterais do tubo estático medem a pressão
estática do fluido, denotada por \(p_{2}\). A diferença entre a pressão de impacto e a pressão estática é chamada
de pressão dinâmica, e está diretamente relacionada à velocidade do fluido. Para medir a velocidade
do fluído, utiliza-se a equação de Bernoulli: 
  \[
    p_{1} + \underbrace{\frac{1}{2}\rho v_{1}^{2}}_{0} = p_{2} + \frac{1}{2}\rho v_{2}^{2}.
  \]
  Assim,
 \begin{align*}
   &p_{1}-p_{2} = \frac{1}{2}\rho_{f}v_{2}^{2}\\
   &\rho_{L}gh = \frac{1}{2}\rho_{f}v_{2}^{2}\\
   &v_{2} = \sqrt[]{\frac{2\rho_{L}gh}{\rho_{f}}}.
 \end{align*}
\newpage

\section{Aula 16 - 27/09/2023}
\subsection{Motivações}
 \begin{itemize}
   \item Oscilador Harmônico;
   \item Sistema massa-mola;
   \item Amplitude, frequência e fase.
 \end{itemize}
\subsection{Oscilador Harmônico (OH)!}
  Numa situação de movimento harmônico simples, a aceleração e a força resultante são
proporcionais e de sentidos opostos ao deslocamento a partir da posição de equilíbrio.
Em outras palavras, \(F = -kx\), ou seja, 
  \[
    m\frac{d^{2}x}{dt^{2}}=-kx \Rightarrow \frac{d^{2}x}{dt^{2}}=-\frac{k}{m}x,
  \]
o que é uma EDO de segunda ordem da variável x. Existem formas e métodos de lidar com elas,
formando uma área inteira da matemática. Para esta equação em específico, as soluções possíveis são 
  \[
    x(t)=\sin^{}{\biggl(\sqrt[]{\frac{k}{m}}t\biggr)} \quad\&\quad x(t)=\cos^{}{\biggl(\sqrt[]{\frac{k}{m}}t\biggr)} 
  \]
  Em geral, a função posição de um sistema de oscilador harmônico é descrita por \(x(t) = A\cos^{}{(\omega t + \delta )}.\)
Chama-se A de amplitude, \(\omega \) de frequência e \(\delta \) de fase. Derivando-a, obtemos velocidade 
e aceleração: 
  \[
    v(t) = -A\omega \sin^{}{(\omega t + \delta )}\quad a(t) = -A\omega^{2}\cos^{}{(\omega t + \delta )}
  \]
  Utilizando a relação de antes para frequência, \(\omega = 2\pi f = \frac{2\pi }{T}\), podemos
isolar em função do período a fim de obter o período do oscilador harmônico: 
  \[
    T = 2\pi \sqrt[]{\frac{M}{k}}.
  \]
  Para conseguirmos informações sobre a fase do sistema, calculamos a posição no momento t = 0, tal que
 \(x(0) = A\cos^{}{(\omega 0 + \delta )} = A\cos^{}{(\delta )}\), ou seja, \(\cos^{}{(\delta )}=\frac{x(0)}{A}\)
 \begin{example}
  Dado que \(\omega = 8 rad/s\) e que, em \(t=0\), temos as informações 
    \[
       \left.\begin{array}{ll}
         x=4cm\\
         v_{x} = 25 cm/s
     \end{array}\right\}
    \] 
  encontre a amplitude do sistema de oscilações e a equação que descreve o sistema.

  Com efeito, como \(x = A \cos{(\omega t + \delta )}\), podemos utilizar este ponto de partida. Note que 
    \[
      \tan{(\delta )} = \frac{-V_{0}}{\omega x_{0}} = \frac{-25}{32} \Rightarrow \delta \approx -0.66.
    \]
  Assim, \(A = \frac{x_{0}}{\cos{(\delta )}}\approx 5.1cm\). Portanto, \(x = 5.1\cos{(\omega t - 0.66)}.\)
 \end{example}
 \begin{example}
  Dado que um bloco está preso a uma mola, considere \(m=2kg\) a massa dele, \(k = 19.6\frac{N}{m}\) e que, inicialmente,
ele está em \(x_{0} = 5cm\). Encontre a aceleração e velocidade máximas dele.

 Começamos observando que \(\omega^{2} = \frac{k}{m} = \frac{19.6}{2} = 98,\) ou seja, \(\omega \approx 9.9 \frac{rad}{s}\).
A partir disso, encontramos a frequência e, consequentemente, o período 
  \[
    f = \frac{\omega }{2\pi }\approx 1.58Hz \Rightarrow T\approx 0.63s.
  \]
  A equação que descreve esse oscilador é \(x(t) = 5\cos{(9.9t)}\), donde tiramos que a velocidade do sistema é descrita por 
  \[
    v = -9.9\times 5\sin{(9.9t)}.
  \]
  Como a função seno é limitada por 1, fica fácil de encontrar o valor máximo da velocidade: 
  \[
    |v_{max}| = 49.5 \frac{cm}{s}.
  \]
  Analogamente, a aceleração será descrita por 
  \[
    a = -(9.9)^{2}\times 5\cos{(9.9t)}
  \]
  e o mesmo raciocínio mostra que 
  \[
    |a_{max}| = (9.9)^{2}\times 5 \frac{cm}{s}.
  \]
 \end{example}
\begin{example}
  Num sistema de movimento circular uniforme, as equações que descrevem o movimento da partícula no instante t são 
    \[
       \left\{\begin{array}{ll}
          x(t) = R\cos{(\omega t + \delta )}\\
          y(t) = R\sin{(\omega t + \delta )}
        \end{array}\right.
    \] 
tal que o movimento vetorial dele será dado pela combinação de posição, velocidade e aceleração 
  \[
     \left\{\begin{array}{ll}
        \vec{R}(t) = R\cos{(\omega t + \delta )}\hat{i} + R\sin{(\omega t + \delta )}\hat{j}\\
        \vec{v}(t) = -R\omega \sin{(\omega t + \delta )}\hat{i} + \omega R\cos{(\omega t + \delta )}\hat{j}\\
        \vec{a}(t) = -R\omega ^{2}\cos{(\omega t + \delta )}\hat{i} - \omega ^{2}R\sin{(\omega t + \delta )}\hat{j}.
      \end{array}\right.
  \]
  Disto, faz sentido falar sobre componente de oscilação nos eixos x e y, como 
    \[
       \left\{\begin{array}{ll}
          x = A_{x}\cos{(\omega_{x}t + \delta )}\\
          y = A_{y}\cos{(\omega_{y}t)}
        \end{array}\right.
    \]
\end{example}
\subsection{Energia Mecânica no OH}
  Dado um sistema massa mola, sendo que a velocidade angular é \(\omega \), a amplitude da oscilação é A e a constante da mola é K, o sistema pode ser 
descrito, no quesito posição, por 
  \[
    x(t) = A\cos{(\omega t + \delta )}
  \]
  Quanto às energias, temos 
  \[
    U = \frac{1}{2}kx^{2} = \frac{1}{2}A^{2}\cos^{2}{(\omega t + \delta )}\quad\&\quad \mathbb{K} = \frac{1}{2}mv^{2} = \frac{1}{2}A^{2}\omega ^{2}\sin^{2}{(\omega t+\delta )}
  \]
  Sendo que a soma delas é a energia mecânica do sistema:
    \[
      E_{mec} = U + \mathbb{K} = \frac{1}{2}A^{2}\cos^{2}{(\omega t + \delta )} + \frac{1}{2}A^{2}\omega ^{2}\sin^{2}{(\omega t+\delta )}
    \]
  Como \(\omega^{2} = \frac{k}{m},\) vale 
    \[
      E_{mec} = \frac{1}{2}A^{2}k\sin^{2}{(\omega t + \delta )} + \frac{1}{2}kA^{2}\cos^{2}{(\omega t + \delta )} = \frac{1}{2}kA^{2}\biggl[\cos^{2}{(\omega t+\delta )} + \sin^{2}{(\omega t + \delta )}\biggr] = \frac{1}{2}kA^{2}
    \]
\newpage 

\section{Aula 17 - 28/09/2023}
\subsection{Motivações}
\begin{itemize}
  \item Continuação de Energia;
  \item Massa-Mola vertical;
  \item Pêndulo Simples.
\end{itemize}
\subsection{Continuando Energias}
  Utilizando o que vimos na última aula, também podemos encontrar o valor médio da Energia Cinética: 
    \[
      \left< \mathbb{K} \right> = \frac{1}{T}\int_{t}^{t+T}\frac{1}{2}KA^{2}\sin^{2}{(\omega t'+\delta )}dt'
    \] 
Como \(\left< \mathbb{K} \right> = \left< U \right>,\) já que a variação de seno e a de cosseno é a mesma, segue que \(\left< \mathbb{K} \right> = \frac{E_{mec}}{2}\).
\begin{example}
  Considere um sistema massa mola em que há uma massa \(m=3kg, A = 4cm\) e o período é de \(T=2s\). Encontre a energia mecânica, a velocidade máxima e a posição na qual
a velocidade do corpo é metade da velocidade máxima.

  Começamos observando a relação que nos fornece o valor da constante de mola k: 
  \[
    \omega ^{2} = \frac{k}{m} \Rightarrow \biggl(\frac{2\pi }{T}\biggr)^{2} = \frac{k}{m} \Rightarrow k = m \biggl(\frac{2\pi }{T}\biggr)^{2}
  \]
a partir disso, a energia mecânica sai como 
  \[
    E_{mec} = \frac{1}{2}kA^{2} = \frac{1}{2}m \biggl(\frac{2\pi }{T}\biggr)^{2}A^{2}\approx 2.4 10^{-2}J
  \]
A velocidade máxima de um sistema isolado é atingido quando a energia mecânica do sistema é composta apenas por cinética, ou seja, 
  \[
    \frac{1}{2}mv_{max}^{2} = E_{mec}\approx 0.13\frac{m}{s}.
  \]
  Finalmente, para encontrar o x desejado, teremos 
  \[
    E_{mec} = \frac{1}{2}mv^{2} + \frac{1}{2}kx^{2} = \frac{1}{2}kA^{2},
  \]
  do que segue que 
 \begin{align*}
   &\frac{1}{2}m \frac{v_{max}^{2}}{4} + \frac{1}{2}kx^{2} = \frac{1}{2} kA^{2} = \frac{1}{2}mv_{max}^{2}\\
   &\frac{1}{2}kx^{2}=\frac{1}{2}mv_{max}^{2}-\frac{1}{2}m \frac{v_{max}^{2}}{4}\\
   \Rightarrow &\frac{1}{2}kx^{2} = \frac{1}{2}mv_{max}^{2}\frac{3}{4} \Rightarrow x^{2} = \frac{3}{4}\frac{m}{k}v_{max}^{2}\\
               & x\approx\pm 3.5cm 
 \end{align*}
\end{example}
\subsection{Sistemas Massa Mola Vertical}

  Considere uma mola presa a um teto e atrelada a um bloco de massa m na sua outra ponta. Ao soltá-la, o bloco
se desloca uma quantidade \(y_{0}\) por conta da gravidade e puxamos este bloco por uma quantidade \(y'\), totalizando
uma distensão da mola em \(y = y_{0}+y'.\) A equação deste sistema, no equilíbrio, é dada por 
  \[
    m \frac{d^{2}y}{dt^{2}} = -ky - mg = -ky_{0}-ky'+mg = - ky',
  \]
ou seja, temos uma EDO da forma 
  \[
    m \frac{d^{2}y}{dt^{2}} = -ky'.
  \]
Porém, note que, de \(y=y_{0} + y',\) temos 
  \[
    \frac{dy}{dt} = c + \frac{dy'}{dt} \Rightarrow \frac{d^{2}y}{dt^{2}} = \frac{d^{2}y'}{dt^{2}}.
  \]
Logo, nossa equação é a mesma que a do oscilador harmônico:
  \[
    \frac{d^{2}y'}{dt^{2}} = -\frac{k}{m}y'.
  \]
\subsection{Pêndulo Simples}
  Faça uma corda de tamanho L, presa ao teto e com uma massa na ponta, oscilar após puxar essa massa em um ângulo \(\theta \).
Caso \(s\) denote o deslocamento com relação ao ponto de equilíbrio, então segue que 
  \[
    m \frac{d^{2}s}{dt^{2}} = -mg\sin^{}{(\theta )}.
  \]
Como \(s = L\theta \), vale que 
  \[
    \frac{d^{2}\theta }{dt^{2}}=-\frac{g}{L}\sin^{}{(\theta )}.
  \]
  Para pequenas oscilações (\(\theta < 15^{\circ})\), nas quais \(\sin^{}{(\theta )}\approx \theta \), obtemos 
\[
  \frac{d^{2}\theta }{dt^{2}} = -\frac{g}{L}\theta .
\]
\newpage

\section{Aula 18 - 02/10/2023}
\subsection{Motivações}
\begin{itemize}
  \item Continuando o Pêndulo Simples;
  \item Outros tipos de pêndulo;
  \item Caminhadas.
\end{itemize}
  Começamos por relembrar o conteúdo prévio:
  \begin{quote}``Faça uma corda de tamanho L, presa ao teto e com uma massa na ponta, oscilar após puxar essa massa em um ângulo \(\theta \).
Caso \(s\) denote o deslocamento com relação ao ponto de equilíbrio, então segue que 
  \[
    m \frac{d^{2}s}{dt^{2}} = -mg\sin^{}{(\theta )}.
  \]
Como \(s = L\theta \), vale que 
  \[
    \frac{d^{2}\theta }{dt^{2}}=-\frac{g}{L}\sin^{}{(\theta )}.
  \]
  Para pequenas oscilações (\(\theta < 15^{\circ})\), nas quais \(\sin^{}{(\theta )}\approx \theta \), obtemos 
\[
  \frac{d^{2}\theta }{dt^{2}} = -\frac{g}{L}\theta .''
\]
\end{quote}
  A partir disto, como o movimento harmônico é descrito por 
    \[
      \frac{d^{2}x}{dt^{2}}=-\omega^{2}x,
    \]
  segue que, no pêndulo, \(\omega^{2}=\frac{g}{l},\) ou seja, \(\omega = \sqrt[]{\frac{g}{l}}.\)
Em particular, isso permite com que calculemos o \textbf{período} do pêndulo: 
  \[
    \frac{2\pi }{T} = \sqrt[]{\frac{g}{l}} \Rightarrow T = 2\pi \sqrt[]{\frac{l}{g}}.
  \]

  Agora, vamos resolver o pêndulo utilizando energia. Começamos por notar que a energia mecânica do sistema é dada por 
    \[
      E_{mec} = \frac{1}{2}mv^{2} + mgl(1-\cos^{}{(\theta )}
    \]
  Vamos usar que, para ângulos pequenos, \(\cos^{}{(\theta )}\approx 1-\frac{\theta ^{2}}{2} \), chegamos em
   \[
      E_{mec} = \frac{1}{2}mv^{2}+\frac{mgl\theta ^{2}}{2}.  
   \]
   Para \(v=\frac{ds}{dt} = l\dot \theta ,\) reescrevemo a equação acima como 
     \[
       E_{mec} = \frac{1}{2}ml^{2}\dot\theta ^{2} + \frac{mgl}{2}\theta^{2} = \frac{1}{2}m'\theta ^{2} + \frac{1}{2}k'\theta ^{2},
     \]
  ou seja, encontramos uma equação análoga à da mola, o que nos permite calcular o valor da velocidade angular como \(\omega^{2}=\frac{k'}{m} = \frac{g}{l}.\)
\subsection{Pêndulo de Torção}
  Um pêndulo de torção é um no qual o peso preso à ponta é um disco em movimento rotacional. Utilizando as relações de torque e momento de inércia, isto é, 
    \[
      \tau = -k\theta, \quad \tau = I\alpha, \Rightarrow -k\theta = I \frac{d^{2}\theta }{dt^{2}},
    \]
  tiramos disso que 
    \[
      \frac{d^{2}\theta }{dt^{2}} = - \frac{k}{I}\theta \Rightarrow \omega^{2} = \frac{k}{I}.
    \]
\subsection{Pêndulo físico}
  Para um corpo qualquer atuando como pêndulo, dado que seu centro de massa é cm, sofrendo um peso \(mg\), se ele for segurado por um ponto que está a uma distância
D desse centro de massa tal que o ângulo com o eixo vertical é \(\theta \) e seu momento de inércia é I, segue que 
  \[
    \tau  = -mgD\sin^{}{(\theta )}.
  \]
  Além disso, como \(\tau = I \frac{d^{2}\theta }{dt^{2}},\) juntamos as duas informações a fim de obter 
    \[
      -mgD\sin^{}{(\theta )} = I \frac{d^{2}\theta }{dt^{2}}.
    \]
  Com a aproximação de pequenos ângulos de \(\sin^{}{(\theta )}\approx \theta \), 
    \[
      \frac{d^{2}\theta }{dt^{2}} = -\frac{mgD}{I}\theta \Rightarrow \omega^{2}= \frac{mgD}{I}.
    \]
\subsection{Caminhar}
\paragraph{} Quando caminhamos, o movimento antagonista das pernas funciona como um pêndulo. A partir disso, pode-se imaginar que caminhar próximo
a uma ``frequência natural'' pode ser mais confortável. Vamos tentar descrever esse sistema.

  Considere que as pernas são barras uniformes com centros de massa CM, tamanho l, massa m e momento de inércia I. Considere
que o ponto pivô está a uma distância \(\frac{l}{2}\) do centro de massa. Então, 
  \[
    \frac{2\pi }{T} = \sqrt[]{\frac{mgD}{I}} \Rightarrow 2\pi \sqrt[]{\frac{I}{mgD}}.
  \]
Como \(I = \frac{1}{3}ml^{2}\) e \(D = \frac{l}{2},\) o período que encontramos é 
  \[
    2\pi \sqrt[]{\frac{2ml^{2}}{3mgl}} = 2\pi \sqrt[]{\frac{2l}{3g}}
  \]
  Curiosamente, descobrimos experimentalmente que, para humanos, 5 períodos vale 6.7s. No entanto,
calculando pelo modelo, encontra-se \(5T=7.8s,\) que é um pouco maior que o dado empírico. Isso dá-se 
devido ao fato de que as pernas humanas não são realmente uniformes - a massa da coxa é maior que do joelho.
  
  Deixando de lado o contexto da caminhada, considere simplesmente uma barra (não uma perna) como antes. Desta vez,
tome um ponto a uma distância x qualquer do centro de massa. Vamos calcular o período dele.
Usualmente, o período é dado por 
  \[
    T = 2\pi \sqrt[]{\frac{I}{mgD}} 
  \]
que, para o nosso caso, torna-se 
  \[
    T = 2\pi \sqrt[]{\frac{I_{cm}+mx^{2}}{mgx}}
  \]
Utilizando que \(I_{cm}= \frac{1}{12}ml^{2},\) 
  \[
    T = 2\pi \sqrt[]{\frac{1}{gx}\biggl[\frac{1}{12}l^{2}+x^{2}\biggr]} 
  \]
Analisando o comportamento dessa função em extremos, deduz-se que ela deve possuir um mínimo. Vamos buscá-lo.
Coloque \(z = \frac{1}{x}\biggl(\frac{1}{12}l^{2}+x^{2},\biggr)\) tal que 
  \[
    T = 2\pi \sqrt[]{\frac{z}{g}}.
  \]
  Então, \(\frac{dT}{dx}=0\). Pela Regra da Cadeia, \(\frac{dT}{dx} = \frac{dT}{dz}\frac{dZ}{dx} = 0\), o que equivale a 
  \[
    \frac{1}{\sqrt[]{z}}\biggl(-\frac{1}{12}\frac{l^{2}}{x^{2}}+1\biggr) = 0,
  \]
 em que as constantes sumiram porque cancelamos elas com 0. Assim, como z não pode zerar, a única forma desse sistemas ser verdade é que 
  \[
    -\frac{1}{12}\frac{l^{2}}{x^{2}}+1 = 0 \Rightarrow 1 = \frac{l^{2}}{12x^{2}}.
  \]
\subsection{Oscilador Harmônico Amortecido}
  O oscilador harmônico é uma idealização da natureza. Nele, não há perdas de energia e ele essencialmente
oscila para sempre. Para uma simulação mais realista do mundo, utilizamos o \textbf{Oscilador Harmônico Amortecido}.

  Considere uma mola presa a um telhado com constante elástica k e com uma massa m presa à sua ponta. Além disso, suponha que
esse sistema está imerso em um fluído com viscosidade. Para descrever esse sistema, há uma equação diferencial 
  \[
    m \frac{d^{2}x}{dt^{2}} = -kx - b \frac{dx}{dt}.
  \]
A energia do sistema, inicialmente, é dada por 
  \[
    E = \frac{1}{2}kA^{2}.
  \]
  Ao tratar de atrito, vimos que 
  \[
   \frac{dE}{dt} = - P = - Fv, 
  \]
e que isto é proporcional ao quadrado da velocidade. Analogamente, no caso do pêndulo, a velocidade é proporcional à amplitude, pois x 
é proporcional a ela. Assim, 
  \[
    \frac{dE}{dt}\propto -A^{2} \Rightarrow \frac{dE}{dt}\propto -E.
  \]
  Conhecemos uma função cuja derivada é proporcional a ela mesma - a exponencial. Com isso,  
  \[
    E = E_{0} e^{\frac{-t}{\tau }}
  \]
e 
  \[
   A = A_{0} e^{\frac{-t}{2\tau }}.
  \]
  Feito isso, podemos propor uma solução para o oscilador harmônico:
  \[
    x(t) = A_{0}e^{-\gamma t}\cos^{}{(\omega t + \delta )}.
  \]
  Confiramos a solução. De fato, derivando com relação a t,
 \begin{align*}
   &\frac{dx}{dt} = -A_{0}\gamma e^{-\gamma t}\cos^{}{(\omega t + \delta )}-A_{0}\omega e^{-\gamma t}\sin^{}{(\omega t + \delta )}\\
   &\frac{d^{2}x}{dt^{2}} = A_{0}\gamma ^{2}e^{-\gamma t}\cos^{}{(\omega t + \delta )} + A_{0}\gamma \omega e^{-\gamma t}\sin^{}{(\omega t + \delta )}\\
   &+A_{0}\gamma\omega e^{-\gamma t}\sin^{}{(\omega t+\delta )}-A_{0}\omega ^{2}e^{-\gamma t}\cos^{}{(\omega t + \delta )}.
 \end{align*}

Para facilitar, vamos coletar os termos com base em cosseno e seno na equação original.

\textbf{\underline{Termos com Cosseno:}}
\paragraph{}  Manipulando algebricamente - cancelando os \(A_{0}\) e os cossenos com exponenciais - obtemos 
  \[
    m\gamma ^{2} - m\omega^{2} = -k + b\gamma 
  \]
\newpage
\textbf{\underline{Termos com Seno:}}
\paragraph{}  Manipulando algebricamente - cancelando os \(A_{0}\) e os cossenos com exponenciais - obtemos 
  \[
    m\gamma \omega  + m\omega \gamma  = b\omega \Rightarrow 2m\gamma  = b \Rightarrow \gamma = \frac{b}{2m}
  \]

Utilizando essas duas coisas, a equação torna-se 
  \[
    m\gamma ^{2} - b\gamma + k = m\omega ^{2} \Rightarrow \omega ^{2} = \gamma ^{2}-\frac{b\gamma }{m}+\frac{k}{m}\omega_{0}^{2}.
  \]
Com isso, encontramos o valor de \(\omega\) - supondo que \(\frac{b}{2m\omega_{0}} <<1\):
\begin{align*}
  &\omega ^{2} = \omega_{0}^{2} + \gamma ^{2}\biggl(1 - \frac{b}{m\gamma }\biggr)\\
  &\omega ^{2} = \omega_{0}^{2} + \biggl(\frac{b}{2m}\biggr)^{2}\biggl(1-\frac{b}{m \frac{b}{2m}}\biggr)\\
  &\omega ^{2} = \omega_{0}^{2} - \biggl(\frac{b}{2m}\biggr)^{2}\\
  &\omega  = \omega_{0}\sqrt[]{1-\biggl(\frac{b}{2m\omega_{0}}\biggr)^{2}}.
\end{align*}
  Agora que temos tanto \(\omega \) quanto \(\gamma \), a equação original torna-se 
    \[
      \boxed{x(t) = A_{0}e^{-\frac{b}{2m}t}\cos^{}{\biggl(t\omega_{0}\sqrt[]{1-\biggl(\frac{b}{2m\omega_{0}}\biggr)^{2}}+\delta \biggr)}}
    \]
\newpage

\section{Aula 19 - 04/10/2023}
\subsection{Motivações}
\begin{itemize}
  \item Tipos de Osciladores Harmônicos Amortecidos;
  \item Oscilador Harmônico Subamortecido;
  \item Oscilador Harmônico Amortecido Forçado.
\end{itemize}
\subsection{Oscilador Harmônico Amortecido}
  Olhando para o caso estudado na última aula, chegamos à conclusão de que, num sistema 
de oscilador harmônico amortecido, a velocidade angular é dada por 
  \[
    \omega  = \omega_{0} \sqrt[]{1-\biggl(\frac{b}{2m\omega_{0}}\biggr)^{2}}.
  \]

  Em particular, se o termo ao quadrado é muito menor que 1, pode-se aproximar \(\omega \) puramente
por \(\omega_{0}\). Este tipo de oscilador é chamado ``\textit{Oscilador Harmônico Subamortecido} ''. 
Novas ferramentas como geladeiras e ares-condicionados implementam esse tipo de oscilador harmônico por meio 
dos \textit{inverters}.

  Caso \(\frac{b}{2m\omega_{0}} = 1,\) ou seja, \(b = 2m\omega_{0}\), dizemos que o oscilador é um 
``\textit{Oscilador Harmônico Criticamente Amortecido}'', e denotaremos esse b específico por \(b_{c}.\) 
Um sistema que funciona dessa forma é o amortecedor de um carro.
  
  O caso restante é quando \(\frac{b}{2m\omega_{0}} > 1\), mas este caso \textbf{NÃO CAI NA PROVA}. Nestes casos,
o valor de \(\omega \) é complexo. Para lidar com ele, recordemos a fórmula de Euler: 
  \[
    e^{i \theta } = \cos^{}{(\theta )} + i\sin^{}{(\theta )},
  \]
donde segue que 
  \[
    \cos^{}{(\theta )} = \frac{e^{i\theta }+e^{-i\theta }}{2}\quad\&\quad \sin^{}{(\theta )} = \frac{e^{i\theta }-e^{-i\theta }}{2i}.
  \]
  Assim, \(\omega = \pm \lambda i,\) e o caso é conhecido como ``\textit{Oscilador Harmônico Superamortecido}''.

  Nosso enfoque será no caso subamortecido. Nestes casos, \(A(t) = A_{0}e^{-\gamma t}.\) Como 
a energia mecânica é dada por \(E_{mec} = \frac{1}{2}kA^{2},\) colocando \(k=m\omega_{0}^{2}\), temos 
  \[
    E_{mec} = \frac{1}{2}m\omega_{0}^{2}A_{0}^{2}e^{-2\gamma t}.
  \]
Chamando de \(E_{0} = \frac{1}{2}m\omega_{0}^{2}A_{0}^{2},\) a energia mecânica torna-se 
  \[
    E_{mec} = E_{0}e^{-\frac{b}{m}t} = E_{0}e^{-\frac{t}{\tau }}.
  \]
  Chamamos este \(\tau \) de \textit{tempo característico}. Derivando esta energia mecânica em relação ao tempo, 
  \[
    \frac{dE_{mec}}{dt} = -\frac{1}{\tau }E_{0} e^{-\frac{t}{\tau }} = -\frac{E}{\tau }.
  \]
  Equivalentemente, 
  \[
    \frac{dE}{E} = -\frac{dt}{\tau }.
  \]
  Antes de prosseguir, definimos o \textit{fator de qualidade} como \(Q = \omega_{0}\tau \). Para
oscilações muito pequenas, podemos escrever 
  \[
    \biggl(\frac{\Delta E}{E}\biggr)_{1\text{período}} = -\frac{T}{\tau } = -\frac{2\pi }{\omega_{0}\tau } = -\frac{2\pi }{Q}.
  \]
  Ainda mais, note que 
  \[
    \frac{b}{2m\omega_{0}} = \frac{1}{2\tau \omega_{0}} = \frac{1}{2Q}, 
  \]
  tal que \(\omega = \omega_{0}\sqrt[]{1 - \frac{1}{4Q^{2}}}\).
\begin{example}
  Considere um amortecedor de massa \(m=1100kg\) e frequência \(f = 1Hz\). Qual é o valor de b para que o amortecedor
seja criticamente amortecido?

  Pelo que vimos, 
    \[
      b_{c} = 2m\omega_{0} = 2\times(1100)\times 2\pi \frac{kg}{s}.
    \]
\end{example}
\begin{example}
  Suponha que uma corda que reproduz a nota Dó/C no piano tem uma frequência \(f = 262Hz\). Ela perde metade
  da energia em 4 segundos. Qual é o \(\tau \), o Q e o \(\biggl(\frac{\Delta E}{E}\biggr)_{\text{ciclo}}\) do sistema?

  Como houve uma perda de metade da energia, tem-se 
    \[
      \frac{E_{0}}{2} = E_{0}e^{-\frac{t}{\tau }},
    \]
  donde segue que 
    \[
      \ln \biggl(\frac{1}{2}\biggr) = \frac{-t}{\tau } \Rightarrow \tau \approx 5.7s.
    \]
    A partir disso, lembrando que \(Q = \omega_{0}\tau  = 2\pi f \tau \), segue que \(Q = 9.5\times 10^{3}.\) Finalmente, 
      \[
        \biggl(\frac{\Delta E}{E}\biggr) = \frac{2\pi }{Q} = \frac{1}{f\tau }\approx 6.6\times 10^{-4}.
      \]
\end{example}
\subsection{Oscilador Harmônico Amortecido Forçado}
  Prenda uma mola com constante elástica k a uma roda que fica subindo e descendo ela. Coloque um corpo de massa m na sua ponta e 
suponha que está tudo submerso em um líquido. A equação deste sistema é 
  \[
    m \frac{d^{2}x}{dt^{2}} + b \frac{dx}{dt} + kx = F_{0}\cos^{}{(\omega t)}.
  \]
  Propomos uma solução 
  \[
    x(t) = A(\omega ) \cos^{}{(\omega t - \delta )}.
  \]
  Derivando essa equação, 
 \begin{align*}
   &\dot x = -A\omega \sin^{}{(\omega t - \delta )}\\
   &\ddot x = -A\omega ^{2}\cos^{}{(\omega t - \delta )}.
 \end{align*}
  Expandindo o cosseno, obtemos as seguintes equações
 \begin{center}
   \begin{table}[h]
   \centering
     \begin{tabular}{| c | c |}
       \hline 
       Funções & Expansões\\
       \hline
       x & \(A\cos^{}{(\omega t )}\cos^{}{(\delta )} + A \sin^{}{(\omega t)}\sin^{}{(\delta )}\)\\
       \(\dot x\) & \(-A\omega \sin^{}{(\omega t)}\cos^{}{(\delta )} + A\omega \cos^{}{(\omega t)}\sin^{}{(\delta )}\)\\
       \(\ddot x\) & \(-A\omega^{2}\cos^{}{(\omega t)}\cos^{}{(\delta )} - A\omega ^{2}\sin^{}{(\omega t)}\sin^{}{(\delta )}\)\\
       \hline
     \end{tabular}
   \end{table}
 \end{center}
 
 Analisemos separadamente os termos em cosseno e em seno.

 \underline{\textbf{Termos em \(\cos^{}{(\omega t)}\)}}:
\begin{align*}
  &-mA\omega ^{2}\cos^{}{(\delta )} + b A\omega \sin^{}{(\delta )} + kA \cos^{}{(\delta )} = F_{0}\\
  &A\cos^{}{(\delta )}\biggl[-m\omega ^{2} + b \tan^{}{(\delta )} + m\omega_{0}^{2}\biggr] = F_{0}
\end{align*}

\underline{\textbf{Termos em \(\sin^{}{(\omega t)}\)}}:
\begin{align*}
  &-mA\omega ^{2}\sin^{}{(\delta )} - bA\omega\cos^{}{(\delta )} + m\omega_{0}^{2}A\sin^{}{(\delta )} = 0\\
  &m(\omega_{0}^{2}-\omega ^{2})\sin^{}{(\delta )}=b\omega^{2}\cos^{}{(\delta )}\\
  &\tan^{}{(\delta )}=\frac{-b\omega}{m(\omega ^{2}-\omega_{0}^{2})} = \frac{b\omega}{m(\omega_{0}^{2}-\omega ^{2})}.
\end{align*}

Agora, como \(\sin^{2}{(\delta )} + \cos^{2}{(\delta )} = 1\), vale que 
  \[
    \biggl(\frac{\sin^{}{(\delta )}}{\cos^{}{(\delta )}}\biggr)^{2} + 1 = \frac{1}{\cos^{2}{(\delta )}} \Rightarrow \cos^{2}{(\delta )} = \frac{1}{1+\tan^{2}{(\delta )}} = \frac{1}{1+\frac{b^{2}\omega ^{2}}{m(\omega_{0}^{2}-\omega ^{2})^{2}}},
  \]
ou seja, 
  \[
    \cos^{2}{(\delta )} = \frac{m(\omega_{0}^{2}-\omega ^{2})^{2}}{m(\omega_{0}^{2}-\omega ^{2})^{2}+b^{2}\omega ^{2}.}
  \]
Disto, encontramos o valor de A usando a separação dos termos em cosseno como sendo 
  \[
    A = \frac{F_{0}}{\cos^{}{(\delta )}\biggl[m(\omega_{0}^{2}-\omega ^{2})+\frac{b^{2}\omega ^{2}}{m(\omega_{0}^{2}-\omega ^\{2\})}\biggr]} = \frac{F_{0}}{\frac{m(\omega_{0}^{2}-\omega ^{2})}{\sqrt[]{m(\omega_{0}^{2}-\omega ^{2})^{2}+b^{2}\omega ^{2}}}\biggl[m(\omega_{0}^{2}-\omega ^{2})+\frac{b^{2}\omega ^{2}}{m(\omega_{0}^{2}-\omega ^\{2\})}\biggr]}m(\omega_{0}^{2}-\omega ^{2})
  \]
Simplificando e cortando termos, 
  \[
    A = \frac{F_{0}}{\biggl[m(\omega_{0}^{2}-\omega ^{2})^{2}+b^{2}\omega ^{2}\biggr]^{\frac{1}{2}}}
  \]
\newpage

\section{Aula 20 - 05/10/2023}
\subsection{Motivações}
\begin{itemize}
  \item Potência em Osciladores Harmônicos
  \item Exemplos de Osciladores e Sistemas Harmônicos.
\end{itemize}
\subsection{Potência em Osciladores Harmônicos}
  Como vimos nos casos de sistemas anteriores, é possível associar a cada força uma potência, definida como 
    \[
      P = \vec{F}\cdot \vec{v},
    \]
 \(\vec{v}\) a velocidade com que o objeto sofrendo a força moverá-se. No caso dos osciladores, já possuímos conhecimento
de ambas as quantidades, o que leva-nos a calcular a potência para esse caso:
\begin{align*}
  P &= \vec{F}\cdot \vec{v} = F_{0}\cos^{}{(\omega t)}[\omega A(\omega )\cos^{}{(\omega t-\delta )}]\\
    &=F_{0}A\omega \cos^{}{(\omega t)}[\cos^{}{(\omega t)}\cos^{}{(\delta )} + \sin^{}{(\omega t)}\sin^{}{(\delta )}]\\
  \Rightarrow & \left< P\right> = \frac{F_{0}A\omega \cos^{}{(\delta )}}{2} = \frac{F_{0}^{2}\omega^{2}b}{2[m^{2}(\omega_{0}^{2}-\omega^{2})+b^{2}\omega^{2}]}.
\end{align*}
  Com relação a \(\left< P \right>,\) podemos usar que \(b = 2m\gamma \) para calcular 
    \[
      \left< P \right> = \frac{F_{0}^{2}}{2}\frac{2m\gamma \omega^{2}}{[m^{2}(\omega_{0}^{2}-\omega^{2}) + 4m^{2}\gamma^{2}\omega^{2}]} = \frac{F_{0}^{2}}{4m\gamma }\frac{(2\gamma \omega )^{2}}{(\omega_{0}^{2}-\omega ^{2})^{2}+4\gamma^{2}\omega^{2}}.
    \]
Assim, podemos fazer uma análise de como a média da potência varia com relação a \(\omega \). Começamos notando que 
  \[
    \left< P(\omega \pm \Delta \omega ) \right> = \frac{\left< P_{max} \right>}{2},
  \]
donde segue que 
\begin{align*}
  &\frac{(2\gamma \omega )^{2}}{(\omega_{0}^{2}-\omega ^{2})^{2}+(2\gamma \omega )^{2}} = \frac{1}{2}\\
  \Rightarrow &2(2\gamma \omega )^{2} = (\omega_{0}^{2}-\omega ^{2})^{2}+(2\gamma \omega )^{2}\\
  \Rightarrow &(2\gamma \omega )^{2}=(\omega_{0}^{2}-\omega ^{2})^{2}.
\end{align*}
Como \(\omega = \omega \pm \Delta \omega \), \(\omega^{2} = \omega_{0}^{2}\biggl(1\pm \frac{\Delta \omega }{\omega_{0}}\biggr)^{2},\) tal que, para \(\frac{\Delta \omega }{\omega_{0}} <<1,\)
vale \(\omega ^{2} = \omega_{0}^{2}\biggl(1 \pm \frac{2\Delta \omega }{\omega_{0}}\biggr).\) Destarte, juntando isso com a equação de cima,
 \begin{align*}
   &(\omega_{0}^{2}-\omega_{0}^{2}\pm 2\Delta \omega \omega_{0})^{2} = (2\gamma \omega_{0})^{2}\\
   \Rightarrow &\pm 2\Delta \omega \omega_{0} = 2\gamma \omega_{0}\\
   \Rightarrow &\Delta \omega = \pm\gamma .
 \end{align*}
  Sobre \(\gamma \), por sua definição, 
    \[
      \gamma = \frac{b}{2m} = \frac{\omega_{0}}{2\frac{m}{b}\omega 00} = \frac{\omega_{0}}{2\tau \omega_{0}} = \frac{\omega_{0}}{2Q},
    \]
tal que obtemos a expressão
  \[
    \left< P(\omega_{0}) \right> = \frac{F_{0}^{2}2m\gamma \omega_{0}^{2}}{8m^{2}\gamma^{2}\omega_{0}^{2}} = \frac{F_{0}^{2}}{4m\gamma } = \frac{F_{0}^{2}}{4m}\frac{2Q}{\omega_{0}}.
  \]
 \begin{example}
   Considere um sistema em que uma mola, com constante elástica \(k = 600 \frac{N}{m}\), tem atrelada à sua ponta um objeto de massa m=1,5Kg, sofrendo uma força de
 \(F=0,5N\). Suponha, também, que, num ciclo, \(\biggl(\frac{\Delta E}{E}\biggr)_{ciclo}=0,03\). Quanto vale o fator de qualidade Q, a variação \(\delta \omega ,\) b e a amplitude em \(\omega_{0}\) nesse caso?

  Temos \(\omega_{0}^{2} = \frac{k}{m} \Rightarrow \omega_{0} = 2rad/s\). Além disso, 
    \[
      Q = \frac{2\pi }{\biggl(\frac{\Delta E}{E}\biggr)} = 210.      
    \]
Assim, obtemos 
  \[
    \Delta \omega  = \frac{\omega_{0}}{Q} = 0,096\frac{rad}{s},\quad b = \frac{m\omega_{0}}{Q} = 0,144 Kg\dot{}s,\quad A(\omega_{0}) = \frac{F_{0}}{b\omega_{0}} =17cm.
  \]
 \end{example}
\newpage

\section{Aula 21 - 09/10/2023} 
\subsection{Motivações}
\begin{itemize}
  \item Ondas;
  \item Classificação das Ondas;
  \item A Equação da Onda.
\end{itemize}
\subsection{Ondas e Classificações}
  As ondas fazem transporte de energia e momento. Exemplos delas incluem o som, terremotos,
ondas d'água e a própria luz, possuindo aplicações em diversas áreas do cotidiano. Estudaremos
esse fenômeno, desde suas classificações à sua descrição físico-matemática.

  Começando pelas classificações, existem dois tipos principais de ondas - as transversais e as longitudinais.
A melhor forma de pensar nas transversais é pensando que elas têm um formato de, enquanto que as longitudinais
aparentam ser um monte de barra vertical que juntam-se e separam-se. Como exemplo
das ondas transversais, vide o gráfico abaixo:

\begin{center}
\begin{tikzpicture}
    \draw[->] (0,0) -- (6,0) node[right] {$x$};
    \draw[->] (0,-1.5) -- (0,1.5) node[above] {$y$};
    \draw[scale=0.5,domain=0:10,smooth,variable=\y,shift={(0,0)}] 
        plot ({\y},{sin(2*\y r)});
    \node[below] at (3,-1.0) {Ondas Transversais};
\end{tikzpicture}
\end{center}

e, para as longitudinais, a seguir está o exemplo:

\begin{center}
  \begin{tikzpicture}
 \begin{scope}[z={(70:1)},y={(110:1)},local bounding box=coil] 
  \draw plot[domain=0:14400,variable=\t,samples=1441,smooth] 
  ({\t/1200+0.1*pi*sin(\t/20)},{-0.5*sin(\t)},{0.5*cos(\t)}); 
 \end{scope} 
 \path (coil.south west) -- (coil.south east) 
 node[pos=0.25,below]{Compressão} node[pos=0.5,below]{Rarefacão}; 
 \draw[very thick,red,stealth-stealth] 
  ([yshift=2mm]coil.north) -- ([yshift=2mm]coil.north east)
  node[midway,above]{$\lambda$}; 
\end{tikzpicture} 
\end{center}

\subsection{A Equação da Onda - Primeiros Passos}
  Ao analisarmos o gráfico de uma onda, vemos que a posição dela no momento t é dada por 
  \[
    x = vt + x'.
  \]
Desta forma, \(x' = x - vt\), o que permite-nos escrever uma forma para a função de propagação dela. Quando
é da esquerda para a direita - ou seja, + x - \(f(x') = f(x-vt)\) e, da direita pra esquerda, -x, é \(f(x') = f(x+vt)\).
Nosso próximo passo é continuar essa análise a fim de encontrar uma equação que descreverá as ondas, começando pelo exemplo
mais simples da corda.

 Para que uma corda vibre, ela precisa estar tensionada com tensão \(F_{T}\). Suponha, também, que ela possui densidade de massa \(\mu\).
Passados t segundos, ela é deformada tal qual passa a parecer uma rampa inclinada. O pedaço que ela estava tem tamanho
 \(v\Delta t\), enquanto que a altura será \(u\Delta t\), sendo u a velocidade com que ela subiu e v a velocidade da onda.
Para que ela fosse levantada, uma forçá vertical \(F_{y}\) foi exercida. Junto da tração de antes, a soma dessas duas resulta em uma
força \(\vec{F} = \vec{F}_{y} + \vec{F}_{T}.\) De forma mais analítica,

\underline{\textbf{Analisando o Sistema em y:}}

  A equação de momento linear no eixo y será 
    \[
      F_{y}\Delta t = \Delta p = \Delta mu.
    \]

Dividindo as duas forças presentes, segue que 
  \[
    \frac{F_{y}}{F_{t}} = \frac{\Delta tu}{\Delta tv} = \frac{u}{v}.
  \]  
Assim, juntando as duas equações, segue que 
  \[
    \frac{\Delta m u}{\Delta t F_{t}} = \frac{u}{v},
  \]
ou seja, 
  \[
    \frac{\Delta m}{\Delta t F_{t}} = \frac{1}{v}.
  \]
Utilizando aproximações \(\Delta m\approx \mu v\Delta t,\) chegamos em 
  \[
    \frac{\mu v\Delta t}{\Delta t F_{t}} = \frac{1}{v} \Rightarrow v^{2}=\frac{F_{t}}{\mu} \Rightarrow v = \sqrt[]{\frac{F_{T}}{\mu}}.
  \]
\begin{example}
  Considere uma corda presa à ponta de um telhado e, amarrado a ela, há um peso de \(m=10kg\). Esta corda
tem comprimento L de 25m e ela sofre uma perturbação na ponta, gerando um pulso a \(20m \) do começo dela. Além disso,
nesse começo, tem uma lagarta, a \(\Delta x = 2.5cm\) da ponta da corda e movendo-se com \(v'=2.5cm/s\). A massa da
lagarta é de \(1kg\). Quanto tempo leva para o pulso chegar à lagarta?

  A velocidade do pulso é dada por 
    \[
      v = \sqrt[]{\frac{F_{T}}{\mu}}=\sqrt[]{\frac{M_{g}}{\frac{m}{L}}} = \sqrt[]{\frac{10\times 10\times 25}{1}}=50m/s.
    \]
Com isso, somos capazes de descobrir o tempo que leva para o pulso alcançar a lagarta como
    \[
      \Delta t = \frac{l}{v} = \frac{20}{50} = 0.4s.
    \]
O tempo para a lagarta atingir em segurança a borda seria 
    \[
      \Delta t' = \frac{\Delta x}{v'} = 1s.
    \]
\end{example}
\subsection{Velocidade do Som}
  Supondo um gás ideal, no qual PV = NkT, vale que 
    \[
      v = \sqrt[]{\frac{B}{\rho }},
    \]
em que \(B = \frac{-\Delta P}{\frac{\Delta V}{V}}.\) Assim, 
  \[
    v = \sqrt[]{\frac{\gamma RT}{M}},
  \]
e, já que \(\gamma \) e R são constantes, essa velocidade só depende da massa e da temperatura do fluído no qual
o som percorrerá.
\subsection{Deduzindo a Equação da Onda}
  Voltando à situação da corda, pegue um pedacinho dela com comprimento \(\Delta x\) e altura \(\Delta y\). Além disso,
na sua extremidade mais alta, uma força \(\vec{F}_{2}\) faz ângulo \(\theta_{2}\) com o eixo horizontal. Analogamente,
na extremidade mais baixa, uma força \(\vec{F}_{1}\) faz ângulo \(\theta_{1}\) com a reta horizontal. Suponha que
 \(\theta_{1}, \theta_{2} << 1\), tal que \(\Delta m = \mu\Delta x.\) Vejamos os movimentos em cada eixo.

 \underline{\textbf{Eixo x:}}
  
  Como não há movimento em x, a somatória das forças nesse eixo deve ser nula: 
    \[
      \sum\limits_{}^{}F_{x} = 0 \Rightarrow F_{2}\cos^{}{(\theta_{2})} - F_{1}\cos^{}{(\theta_{1})} = 0.
    \]
Como os ângulos são muito pequenos, \(\cos^{}{(\theta_{i})} = 1,  i=1, 2\), ou seja,
  \[
    F_{1} = F_{2} \Rightarrow F_{1} = F_{2} = F_{T}.
  \]

  \underline{\textbf{Eixo y:}}
    
  Aqui, há movimento, de onde vale a segunda Lei de Newton 
    \[
      \sum\limits_{}^{}F_{y} = \Delta m \frac{\partial^{2}y}{\partial t^{2}}.
    \]
Do raciocínio inicial, sabemos que 
\begin{align*}
  &y_{+} = y(x-vt),\\
  &y_{-} = y(x+vt).
\end{align*}
  
  Unindo com a parte de x, obtemos a equação 
  \[
    F_{T}\sin^{}{(\theta_{2})} - F_{T}\sin^{}{(\theta_{1})} = \Delta m \frac{\partial^{2}{y}}{\partial{t^{2}}}
  \]
  Como o ângulo é muito pequeno, \(\sin^{}{(\theta )} = \tan^{}{(\theta )}\). Assim, colocando 
 \begin{align*}
   &s_{2} = \tan^{}{(\theta_{2})} = \frac{\partial^{}y}{\partial x^{}}\biggl|_{2}^{}\biggr.\\
   &s_{1} = \tan^{}{(\theta_{1})} = \frac{\partial^{}y}{\partial x^{}}\biggl|_{1}^{}\biggr.,
 \end{align*}
 a equação torna-se 
 \[
   F_{T}(s_{2}-s_{1}) = \Delta x\mu \frac{\partial^{2}y}{\partial t^{2}}.
 \]
Logo, 
\begin{align*}
  &F_{T}\frac{\Delta S}{\Delta x} = \mu \frac{\partial^{2}y}{\partial t^{2}}\\
  \Rightarrow &\lim_{\Delta x\to 0}\frac{\Delta S}{\Delta x}= \frac{\partial^{}S}{\partial x^{}}=\frac{\partial^{2}y}{\partial x^{2}}\\
  \Rightarrow &F_{T}\frac{\partial^{2}y}{\partial x^{2}}=\mu \frac{\partial^{2}y}{\partial t^{2}}\\
  \Rightarrow &\frac{\partial^{2}y}{\partial x^{2}}=\frac{\mu}{F_{T}}\frac{\partial^{2}y}{\partial t^{2}}.
\end{align*}
  Portanto, a equação da onda na corda é a seguinte:
  \[
    \boxed{\frac{\partial^{2}y}{\partial x^{2}}=\frac{\mu}{F_{T}}\frac{\partial^{2}y}{\partial t^{2}}}
  \]

  Apesar de termos derivado isso para uma corda, ela vale para todos os tipos de ondas. De fato, para
 \(y = y(x-vt)\) qualquer, coloque \(\alpha  = x - vt\). Então, 
\begin{align*}
  &\frac{\partial^{}y}{\partial x^{}} = \frac{\partial^{}y}{\partial \alpha ^{}}\frac{\partial^{}\alpha }{\partial x^{}} = \frac{\partial^{}y}{\partial \alpha ^{}}\\
  \Rightarrow & \frac{\partial^{2}y}{\partial x^{2}} = \frac{\partial^{2}y}{\partial \alpha ^{2}}\\
              & \frac{\partial^{}y}{\partial t^{}}=\frac{\partial^{}y}{\partial \alpha ^{}}\frac{\partial^{}\alpha }{\partial t^{}}=\frac{\partial^{}y}{\partial x^{}}(-v)\\
  \Rightarrow & \frac{\partial^{2}y}{\partial t^{2}}=v^{2}\frac{\partial^{2}y}{\partial \alpha ^{2}}.
\end{align*}
Juntando ambas, obtemos 
 \begin{align*}
   &\frac{1}{v^{2}}\frac{\partial^{2}y}{\partial t^{2}}=\frac{\partial^{2}y}{\partial x^{2}}\\
   \Rightarrow & \frac{\partial^{2}y}{\partial x^{2}}=\frac{1}{v^{2}}\frac{\partial^{2}y}{\partial t^{2}}.
 \end{align*}
Portanto, a equação da onda em sua forma geral é dada por 
  \[
    \hypertarget{wave_eqn}{\boxed{\frac{\partial^{2}y}{\partial x^{2}}=\frac{1}{v^{2}}\frac{\partial^{2}y}{\partial t^{2}}}}
  \]
\newpage

\section{Aula 22 - 16/10/2023}
\subsection{Motivações}
\begin{itemize}
  \item Ondas Periódicas;
  \item Superposição de Ondas;
  \item Interferência e Batimento.
\end{itemize}
\subsection{Ondas Periódicas}
  Suponhamos que \(y(x, t) = A\sin^{}{(kx - \omega t)}\). Derivando essa equação com respeito a t e a x, temos 
 \begin{align*}
   &\frac{\partial^{}y}{\partial t^{}} = -\omega A \cos^{}{(kx-\omega t)}\\
   &\frac{\partial^{2}y}{\partial t^{2}} = -\omega^{2}A \sin^{}{(kx-\omega t)}\\
   &\frac{\partial^{}y}{\partial x^{}} = k A \cos^{}{(kx-\omega t)}\\
   &\frac{\partial^{2}y}{\partial x^{2}} = -k A \sin^{}{(kx-\omega t)}
 \end{align*}
  Pela \hyperlink{wave_eqn}{\textbf{equação da onda}}, obtemos 
    \[
      -k^{2} = \frac{1}{v^{2}}(-\omega^{2}) \Rightarrow \omega = kv.
    \]
Chamamos k de vetor de onda. Ao analisarmos o gráfico de uma onda, vemos que o seu período
é dado por \(T = \frac{\lambda }{v}\), ou seja, \(v = \frac{\lambda }{T} = \lambda f\). Colocando isso na equação que descreve \(\omega \), 
  \[
    k = \frac{\omega }{v} = \frac{2\pi f}{v} = \frac{2\pi }{\lambda },
  \]
o que nos permite escrevermos 
  \[
    y(x, t=0) = A \sin^{}{(kx + \delta )} = A\sin^{}{(\frac{2\pi }{\lambda }x + \delta )}
  \]
e 
  \[
    y(x, t) = A \sin^{}{(kx-kvt)} = A\sin^{}{(k[x-vt])}
  \]
  Assim, as derivadas em t tornam-se 
 \begin{align*}
   &\frac{\partial^{}y}{\partial t^{}} = v_{y} = -Akv\cos^{}{([k(x-vt)])} = -A\omega \cos^{}{([k(x-vt)])}\\
   &\frac{\partial^{2}y}{\partial t^{2}} = a_{y} = -A(kv)^{2}\sin^{}{([k(x-vt)])} = -A\omega^{2}\sin^{}{([k(x-vt)])}.
 \end{align*}
\begin{example}
  Dada uma onda \(y = 0,03\sin^{}{(2,2x - 3,5t)}\), sabe-se que a sua amplitude é 0,03, o vetor de onda é 2,2\(\frac{1}{m}\) e \(\omega = 3,5\frac{rad}{s}\). A partir disso,
conseguimos encontrar \(\lambda \) a partir de 
  \[
    k = \frac{2\pi }{\lambda } = 2,2 \Rightarrow \lambda = 2,9m
  \]
e o período T por 
  \[
    T = \frac{2\pi }{\omega }\approx 1,8s.
  \]
Assim, a velocidade máxima que essa corda pode atingir é dada por 
  \[
    v_{y_{max}} = A\omega = 0,03\times 3,5\approx 0,11\frac{m}{s}.
  \]
\end{example}
\begin{example}
  Pegue uma corda sendo puxada para baixo por uma força \(F_{T}\), que faz um ângulo \(\theta << 1\) com a horizontal, na sua extremidade no eixo x negativo. Isso resulta
na velocidade que ela é puxada pra baixo assumindo um valor \(v_{y}\). A potência dessa onda, então, será dada por 
  \[
    P = \vec{F}\cdot \vec{v} = F_{T}\sin^{}{(\theta )}v_{y} = F_{T}\tan^{}{(\theta )}v_{y} = F_{T}\frac{\partial^{}y}{\partial x^{}}\frac{\partial^{}y}{\partial t^{}}
  \]
Como \(y = A\cos^{}{(kx - \omega t)}\), encontramos 
  \[
    P = F_{T}\biggl[-Ak\sin^{}{(kx - \omega t)}\biggr]\biggl[A\omega \sin^{}{(kx - \omega t)}\biggr],
  \]
que, em módulo, tem valor \(|P|=F_{T}A^{2}k\omega \sin^{}{(kx-\omega t)}\) . Assim, a potência média dessa onda com essa forçá é 
  \[
   \left< P \right> = \frac{F_{T}A^{2}k\omega }{2}.
  \]
  Utilizando \(F_{T} = \mu v^{2}\) e \(v^{2} = \frac{\omega ^{2}}{k^{2}},\) temos \(F_{T} = \mu \frac{\omega^{2}}{k^{2}},\) donde segue que
  \[
    \left< P \right> = \mu \frac{\omega^{2}}{k^{2}}\frac{A^{2}k\omega }{2} = \frac{\mu\omega^{3}A^{2}}{2k} = \frac{\mu v \omega^{2}A^{2}}{2}.
  \]
Utilizando que \(\biggl(\Delta E\biggr)_{med} = \left< P \right> \Delta t,\) portanto, 
  \[
    \frac{\mu v \omega ^{2}A^{2}}{2}\Delta t = \frac{\mu\Delta x\omega^{2}A^{2}}{2}
  \]
\end{example}
\begin{example}
  Considere uma corda de comprimento \(L = 60m\) e que há um objeto de massa \(m =320g\) preso à ela sob a ação de uma força de tração
 \(T = 12N\). Dados que \(\lambda = 25cm, A = 1,2cm, \Delta x = 15cm\) para a onda que origina-se nesta corda, encontramos 
   \[
     v = \sqrt[]{\frac{T}{\mu}} = \sqrt[]{\frac{TL}{m}} = \sqrt[]{\frac{12\times 60}{0,32}}\approx 47\frac{m}{s}.
   \]
  Além disso, \(\omega = 2\pi f = 2\pi \frac{v}{\lambda } = 1190\frac{rad}{s}.\) Assim, podemos encontrar a variação da energia nesta onda: 
  \[
    \Delta E = \frac{\mu\omega ^{2}A^{2}}{2}\Delta x = \frac{m}{L}\frac{\omega ^{2}A^{2}}{2}\Delta x\approx 8,25. 
  \]
\end{example}
\subsection{Ondas Sonoras e Superposição de Ondas}
  Suponha que há um sistema de várias massas-molas acopladas. Quando uma onda sonora passa por elas, as massas vibrarão em torno de uma posição de equilíbrio.
Moldaremos nosso gás por meio desse sistema. Antes de estudá-las, precisamos ver outros conceitos básicos das ondas.
  
  Vimos que a \hyperlink{wave_eqn}{equação de onda} tem a forma 
    \[
      \frac{\partial^{2}y}{\partial x^{2}} = \frac{1}{v^{2}}\frac{\partial^{2}y}{\partial t^{2}}.
    \]
Suponha que existem soluções \(y_{1}\) e \(y_{2}\) dessa equação de onda. Então, uma combinação linear delas também é uma solução, chame-a de 
  \[
    y = c_{1}y_{1} + c_{2}y_{2}.
  \]
Quando as duas ondas chocam-se, elas entram em um estado que chamamos de superposição de ondas, durante o qual as ondas interagem umas com as outras,
resultando numa combinação. Quando elas duas ondas colidem, ocorre um fenômeno chamado interferência - descreveremos ele a seguir.

  Dadas duas ondas \(y_{1} = A\sin^{}{(kx - \omega t)}, y_{2} = A\sin^{}{(kx - \omega t + \delta )}\), a onda superposta delas será 
  \[
    y_{T} = y_{1} + y_{2} = A \biggl[\sin^{}{(kx-\omega t)} + \sin^{}{(kx-\omega t+\delta )}\biggr] = 2A\cos^{}{(\frac{\delta }{2})}\sin^{}{(kx-\omega t+\frac{\delta }{2})},
  \]
em que usamos \(\sin^{}{(\theta_{1})}+\sin^{}{(\theta_{2})} = 2\cos^{}{(\frac{\theta_{1}-\theta_{2}}{2})}\sin^{}{(\frac{\theta_{1}+\theta_{2}}{2})}\). Em particular, a onda continua
oscilando com \(kx-\omega t\), mas sua amplitude está diferente - Agora, ela vale \(2A\cos^{}{(\frac{\delta }{2})}\). Podemos analisar alguns casos a partir disso.
\begin{itemize}
  \item[\(\delta = 0\text{ ou } 2\pi \):] \(A' = 2A\cos^{}{(\frac{\delta }{2})} = 2A;\)
  \item[\(\delta = \pi \):] \(A' = 2A\cos^{}{(\frac{\delta }{2})} = 0.\)
\end{itemize}
  O primeiro dos casos mostrados é chamado \textbf{interferência construtiva}, enquanto que, o segundo, é chamado \textbf{interferência destrutiva.}

  Outro fenômeno que ocorre com as ondas é conhecido como \textbf{batimento}. Considere ondas \(y_{1} = A\sin^{}{(\omega_{1}t)}, y_{2} = A\sin^{}{(\omega_{2}t)}\) e a combinação
delas 
  \[
    y_{T} = y_{1}+y_{2} = 2A\cos^{}{(\frac{\omega_{1}-\omega_{2}}{2}t)}\sin^{}{(\frac{\omega_{1}+\omega_{2}}{2}t)}.
  \]
vamos supor que \(\Delta \omega = \omega_{1} - \omega_{2} <<\omega_{1}, \omega_{2}\), tal que podemos escrever, para 
 \(\omega _{m} = \frac{\omega_{1}+\omega_{2}}{2}\)
  \[
    y_{T} = 2A\cos^{}{(\frac{\Delta \omega }{2})t}\sin^{}{(\omega_{m}t)}
  \]
  Para os ouvidos, o que interessa das ondas sonoras são os máximos, tal que a frequência, que seria \(\frac{\Delta \omega }{2},\) leva um fator de correção
que é multiplicá-la por 2, deixando apenas \(\Delta \omega .\) Com isso, a \textbf{frequência de batimento} é descrita como 
  \[
    f_{bat} = \Delta \omega = \frac{1}{2\pi }(\omega_{1}-\omega _{2}) = f_{1}-f_{2}.
  \]
\newpage

\section{Aula 23 - 18/10/2023}
\subsection{Motivações}
\begin{itemize}
  \item Ondas Estacionárias;
  \item Ondas em Tubos;
  \item Início de Efeito Doppler.
\end{itemize}
\subsection{Ondas Estacionárias}
  Considere uma corda presa entre duas superfícies que começa a vibrar. Dependendo da velocidade que ela vibra,
o comprimento de onda \(\lambda \) irá mudar, assim como a frequência
\begin{center}
  \begin{tikzpicture}
      % Eixos
    \draw[-] (0,0) -- (6.28,0) node[right] {\(\text{onda parada e tamanho da corda}\)};
    \draw[-] (0,-1.5) -- (0,1.5) node[above] {\(\text{parede, n=1}\)};
    \draw[-] (6.28,-1.5) -- (6.28,1.5) node[above] {\(\text{parede, n=1}\)};

      % Onda estacionária sólida
      \draw[thick,blue] plot[domain=0:6.28,samples=200] (\x,{6.28*sin(\x*6.28 - \x*\x)});

      % Onda estacionária tracejada com fase oposta
      \draw[thick,dashed,red] plot[domain=0:6.28,samples=200] (\x,{-6.28*sin(\x*6.28 - \x*\x)});

      % Nós
      \foreach \x in {0, 6.28} {
          \fill (\x,0) circle (2pt);
      }
  \end{tikzpicture}

 \begin{tikzpicture}
     % Eixos
   \draw[-] (0,0) -- (6.28,0) node[right] {\(\text{onda parada e tamanho da corda}\)};
   \draw[-] (0,-1.5) -- (0,1.5) node[above] {\(\text{parede, n=2}\)};
   \draw[-] (6.28,-1.5) -- (6.28,1.5) node[above] {\(\text{parede, n=2}\)};

     % Onda estacionária sólida
     \draw[thick,blue] plot[domain=0:6.28,samples=200] (\x,{6.28*sin((3.14-\x)*(\x*6.28 - \x*\x))});

     % Onda estacionária tracejada com fase oposta
     \draw[thick,dashed,red] plot[domain=0:6.28,samples=200] (\x,{-6.28*sin((3.14-\x)*(\x*6.28 - \x*\x))});

     % Nós
     \foreach \x in {0, 6.28} {
         \fill (\x,0) circle (2pt);
     }
 \end{tikzpicture}

 \begin{tikzpicture}
     % Eixos
   \draw[-] (0,0) -- (6.28,0) node[right] {\(\text{onda parada e tamanho da corda}\)};
   \draw[-] (0,-1.5) -- (0,1.5) node[above] {\(\text{parede, n=3}\)};
   \draw[-] (6.28,-1.5) -- (6.28,1.5) node[above] {\(\text{parede, n=3}\)};

     % Onda estacionária sólida
     \draw[thick,blue] plot[domain=0:6.28,samples=200] (\x,{6.28*sin((4.19-\x)*(2.09-\x)*(\x*6.28 - \x*\x))});

     % Onda estacionária tracejada com fase oposta
     \draw[thick,dashed,red] plot[domain=0:6.28,samples=200] (\x,{-6.28*sin((4.19-\x)*(2.09-\x)*(\x*6.28 - \x*\x))});

     % Nós
     \foreach \x in {0, 6.28} {
         \fill (\x,0) circle (2pt);
     }
 \end{tikzpicture}
  \end{center}
(Os picos deveriam estar do mesmo tamanho!!!!)
Obtemos a seguinte tabela representando as mudanças
 \begin{center}
   \begin{table}[h!]
   \caption{Quantidades de Acordo com a Velocidade}
   \centering
     \begin{tabular}{| c | c | c |}
       \hline
       n & \(\lambda \) & f \\
       \hline
       1 & 2L & \(\frac{2v}{2L}\)\\
       2 & \(\frac{2L}{2}\) & \(\frac{2v}{2L}\)\\
       3 & \(\frac{2L}{3}\) & \(\frac{3v}{2L}\)\\
       4 & \(\frac{2L}{4}\) & \(\frac{4v}{2L}\)\\
       \hline
     \end{tabular}
   \end{table}
 \end{center}
 Observando esta tabela, chamamos o caso n=1 de primeiro harmônico fundamental, n=2 de segundo harmônico fundamental, e assim em diante. Além disso, há a seguinte relação: 
   \[
     \lambda = \frac{2L}{n},\quad f = n \frac{v}{2L},\quad \lambda = \frac{\lambda_{1}}{n},
   \]
   em que n é um natural, \(v = \sqrt[]{\frac{F_{T}}{\mu}}\) e \(\lambda_{1}\) é o comprimento de onda no primeiro harmônico fundamental. Em particular, segue que 
     \[
       f = n f_{1},\quad n\in \mathbb{N}.
     \]
\begin{example}
  Considere uma corda vibrando em \(f = 440Hz\) de tamanho \(L - 0.7m\). Neste caso, temos 
    \[
      \lambda  = 2L = 1.4m,\quad v = f\lambda = 440\times 1.4 = 616\frac{m}{s}.
    \]
\end{example}
\begin{example}
  Considere uma corda de piano de 3m com densidade linear de massa \(\mu = 0.0025 \frac{kg}{m}\) e suponha que ela vibra em duas frequências - \(f_{n} = 252Hz\) e \(f_{n+1} = 336Hz\)
(não sabemos qual nó que é). A tensão máxima que a corda pode ser submetida é \(F_{max} = 700N.\) Esta corda está a ponto de estourar?

Sabemos que \(v^{2} = \frac{F_{T}}{\mu}\) e que \(v = \frac{f}{n}2L\). Assim, 
  \[
    F_{T} = \mu \biggl[\frac{f_{n}2L}{n}\biggr]^{2}.
  \]
  Como \(f_{n} = nf_{1}\) e \(f_{n+1} = (n+1)f_{1}\), temos 
    \[
      \frac{f_{n}}{f_{n+1}} = \frac{n}{n+1} \Rightarrow \frac{252}{336} = \frac{n}{n+1},
    \]
tal que 
  \[
    336n = 252n + 252 \Rightarrow 84n = 252 \Rightarrow n = \frac{252}{84} = 3.
  \]
Em outras palavras, o terceiro harmônico fundamental é o que vale 252, o que permite-nos simplificar 
  \[
    F_{T} = \mu \biggl[\frac{nf_{1}2L}{n}\biggr]^{2} = \mu f_{1}^{2}4L^{2} = \mu \biggl(\frac{f_{3}}{3}\biggr)^{2}4L^{2} = \mu 84^{2}4L^{2} = 635N.
  \]
Portanto, a corda não está prestes a estourar.
\end{example}
  Se a corda não estiver presa a um ponto, os harmônicos ocorrem apenas para n's ímpares, tal que 
    \[
      \lambda_{n} = \frac{4L}{n} = \frac{\lambda_{1}}{n},\quad f_{n} = \frac{nv}{4L} = nf_{1},\quad n = 1, 3, 5, \cdots
    \]

  Agora, considere \(y = A(x)\cos^{}{(\omega t + \delta )}\). Se \(\cos^{}{(\omega t + \delta )}=1,\) segue que 
    \[
      A_{0}\sin^{}{(k_{n}x)} = A(x),\quad k_{n} = \frac{2\pi }{\lambda_{n}}.
    \]
  Se supormos que \(A(0) = 0 = A(L)\), então \(\frac{2\pi }{\lambda_{n}}\) é um número inteiro múltiplo de L, donde obtemos 
    \[
      \frac{2\pi }{n}L = n\pi \Rightarrow \lambda_{n} = \frac{2L}{n},
    \]
  já que seno anula-se nos valores múltiplos de \(\pi \). Com isso, 
    \[
      y(x, t) = A\sin^{}{(k_{n}x)}\cos^{}{(\omega t + \delta )}.
    \]
\subsection{Ondas Sonoras em Tubos}
  A teoria construída até agora será usada para descrever ondas sonoras em tubos abertos e em tubos fechados, pois ela depende fortemente
da questão dos harmônicos. Um disclaimer importante de ser feito é que, dependendo do material base, o desenho mudará - alguns livros representam
o deslocamento das partículas, outros representam a pressão.

  Primeiramente, considere um tubo aberto. Em suas extremidades, devem haver nós, pois a pressão deve ser a mesma que o exterior. Conforme avança-se
em direção ao centro, essa pressão aumenta. No ponto de vista do deslocamento, há mais dele nas extremidades e vai reduzindo conforme avança-se para o
centro. Voltamos ao caso da corda presa em duas posições

  Quanto ao tubo fechado, ao chegar à extremidade sem saída, o deslocamento será na menor quantidade, mas a pressão estará em um máximo.
Em contraste com o primeiro, este é o caso da corda presa em uma posição só.
 \begin{example}
   Considere um órgão (instrumento) e seu tubo aberto, com comprimento L de 1m. Qual é a frequência do modo fundamental?

   Temos \(\lambda  = 2L = 2m. \) Como a velocidade do som é \(343 \frac{m}{s}\), segue que 
    \[
     v = \lambda f \Rightarrow f = \frac{v}{\lambda } = 171.5 Hz 
    \]
  Caso fosse hélio ao invés de ar, como a velocidade do hélio é \(v_{He} = 975m/s\), 
    \[
      f = \frac{v}{\lambda } \approx 480Hz.
    \]
 \end{example}
  Preencha um tubo de água e conecte-o a um reservatório por meio de uma mangueira, sendo este reservatório móvel para cima e para baixo. Com um diapasão, 
você acerta o tubo com uma frequência conhecida e abaixa o reservatório. Quando a coluna estiver em ressonância com a frequência, um som mais alto é produzido.
Em outras palavras, estamos medindo as ondas do tubo, permitindo-nos medir os comprimentos de ondas nesse tubo de acordo com a altura que a coluna d'água entrou em ressonância.

  Certa vez, utilizando um diapasão de 500Hz, foram medindo e mediram os seguintes valores: 
    \[
      L = 16;\quad50,5;\quad85;\quad119,5cm
    \]
    tal que \(\Delta L = 119.5 - 85 = \frac{\lambda }{2}\). Sabendo a frequência e o \(\lambda \), descobriram a velocidade do som como \(v\approx 345m/s.\)
\subsection{Efeito Doppler}
  Para estudar este efeito, assume-se que a fonte da onda é móvel tal que ela desloca-se \(u_{f}T\). Para um observador olhando, o comprimento da onda de antes dela
mover-se comparado com o comprimento após o movimento será maior no segundo caso. Antes dela mover-se, a velocidade vale \(f_{O}\lambda_{O} = v\).
Sabemos o comprimento de onda como sendo \(\lambda_{0} = (v-u_{f})T\), donde obtemos, para o caso da fonte aproximando-se, a frequência percebida pelo observador
  \[
    f_{O} = \frac{v}{(v-u_{f})T} = \frac{vf_{F}}{v-u_{f}} = \frac{f_{f}}{1-\frac{u_{f}}{v}},
  \]
em que \(f_{F}\) é a frequência da fonte. Analogamente, caso a fonte esteja afastando-se, 
  \[
    f_{O} = \frac{f_{F}}{1+\frac{u_{f}}{v}}.
  \]
\newpage

\section{Aula 24 - 19/10/2023}
\subsection{Motivações}
\begin{itemize}
  \item Continuanddo Efeito Doppler
\end{itemize}
\subsection{Efeito Doppler}
  O caso que vimos previamente foi um no qual o observador encontra-se parado e, a fonte, movendo-se. Porém, há a possibilidade de ambos estarem em movimento. Para um observador que 
move-se com velocidade \(u_{O}\), a equação 
  \[
    f_{O} = \frac{f_{F}}{1+\frac{u_{f}}{v}}.
  \]
torna-se 
  \[
    f_{O} = \frac{\biggl(1\pm \frac{u_{O}}{v}\biggr)}{\biggl(1\pm \frac{u_{f}}{v}\biggr)}f_{F}.
  \]
  Vamos analisar o caso em que \(\frac{u_{O}}{v} << 1\) e \(\frac{u_{f}}{v} << 1\). Então, 
  \[
    f_{O} = \biggl(1\pm \frac{u_{O}}{v}\biggr)\biggl(1\pm \frac{u_{f}}{v}\biggr)f_{F} - f_{F},
  \]
ou seja, 
  \[
    f_{O} = f_{F} - f_{F}\pm \frac{(u_{O}\pm u_{f})}{v}f_{F}.
  \]
  Escrevendo \(u = u_{O} \pm u_{f}\), portanto, 
  \[
    f_{O} = \pm\frac{u}{v}f_{F}.
  \]
\begin{example}
  Assuma que há um fonte movendo-se a \(34 \frac{m}{s}\) de um observador, e que ela emite uma onda de frequência \(f_{F} = 4470Hz\). Encontre a frequência percebida pelo
observador e, em seguida, faça o mesmo, mas considerando que o observador é quem move-se.

  Temos o comprimento de onda que chega até o observador como sendo
  \[
    \lambda_{O} = \frac{v-u_{f}}{f_{F}} = \frac{343-34}{400} = 0,77m.
  \]
Além disso, o comprimento da onda saindo da fonte vale 
  \[
    \lambda = \frac{v}{f_{F}} = \frac{343}{400} = 0,83m.
  \]
  Com esses dados, a frequência que o observador percebe é de 
    \[
      f_{O} = \biggl(\frac{1}{1-\frac{u_{F}}{v}}\biggr)f_{F} = \frac{1}{\biggl(1-\frac{34}{343}\biggr)}\times 400 = 444Hz.
    \]
  Com relação ao caso do observador móvel e fonte parada, temos 
  \[
    f_{O} = \biggl(\frac{u_{0}}{v}+1\biggr)f_{F} = \biggl(1 + \frac{34}{343}\biggr)\times 400 = 440Hz.
  \]
\end{example}
\begin{example}
  Para este exemplo, considere o caso de uma fonte parada emitindo uma onda de \(4400Hz\). Em direção a ela, com velocidade u, há uma onda enorme movimentando-se.
A fim de descobrir quão rápida esta onda está aproximando-se, foi colocado um captador do som, que funciona por meio do efeito Doppler após a frequência emitida ser refletida
pela onda em movimento. Encontre o valor de u.

  Começamos pela frequência recebida pela onda, cujo expressão é 
  \[
    f_{O} = \biggl(1 + \frac{u}{v}\biggr)f_{F}.
  \]
Analogamente, a frequência que a gravação capta do som refletido na onda é 
  \[
    f_{O}' = \biggl(\frac{1}{1-\frac{u}{v}}\biggr)f_{O} = \biggl(\frac{1}{1-\frac{u}{v}}\biggr)\biggl(1+\frac{u}{v}\biggr)f_{F}
  \]
  Rearranjando a expressão, 
  \[
    f_{O}' = \frac{1}{\frac{v-u}{v}}\biggl(\frac{v+u}{v}\biggr)f_{F}.
  \]
  Logo,
 \begin{align*}
   &vf_{O}' - f_{O}'u = vf_{F} + f_{F}u\\
   &v(f_{O}' - f_{F}) = u(f_{F}+f_{O}')\\
   &u = v\frac{(f_{O}'-f_{F})}{(f_{F}+f_{O}')} = 16,3 \frac{m}{s}.
 \end{align*}
\end{example}
\newpage

\section{Aula 25 - 23/10/2023}
\subsection{Motivações}
\begin{itemize}
 \item Exercícios.
\end{itemize}
\subsection{Exercícios}
\begin{example}
 O sifão descreve um sistema em que um recipiente com líquido recebe um cano cuja extremidade final está abaixo de onde ele é colocado, com uma diferença h
em relação à superfície do líquido. Qual é a altura máxima que ele pode ser posto? E qual é o volume do líquido na extremidade final?

 Sabemos que 
   \[
     P_{1} + \rho gy_{1} + \frac{\rho v_{1}^{2}}{2} = cte.
   \]
 Assim, 
   \[
     P_{A} + \rho gy_{A} + \frac{1}{2}\rho V_{A}^{2} = P_{B} + \rho gy_{B} + \frac{1}{2}\rho V_{B}^{2}
   \]
 e, disto, \(A_{A}V_{A} = A_{B}V_{B}\), tal que \(A_{A} >> A_{B}\) e \(V_{A}\approx 0\). Suponha que a altura da extremidade final \(y_{B} = 0\). Com isso, 
\begin{align*}
  &P_{atm} + \rho gh = P_{atm} + \frac{1}{2}\rho V_{B}^{2}\\
  \Rightarrow& v_{B} = \sqrt[]{2gh}.
\end{align*}
 Concluímos, por fim, o valor da altura máxima:
\begin{align*}
  P_{A} + \rho gy_{A} + \frac{1}{2}\rho V_{A}^{2} &= P_{c} + \rho gy_{c} + \frac{1}{2}\rho V_{c}^{2}\\
  \underbrace{\Rightarrow}_{y_{A}=0, y_{c} = y_{max}} & P_{A} = P_{c} + \rho gy_{max} + \frac{1}{2}\rho V_{c}^{2}
                                         \Rightarrow  & y_{max} = \frac{P_{A}}{\rho g} \approx \frac{10^{5}}{10^{3}\times 10} \approx 10m
\end{align*}
\end{example}
\begin{example}
 Considere agora um sistema composto de três massas-molas separadas e todas na vertical presas ao chão. A primeira mola tem tamanho L; a segunda, l e, a terceira,
tem um bloco à altura h. Supondo que a segunda mola recebe uma martelada e desloca-se com velocidade \(v_{0}\) para baixo, encontre a amplitude A e a altura h. Além disso,
encontre o tempo necessário para que a segunda mola suba até a amplitude máxima. Para qual \(v_{0}\) a mola fica frouxa - isto é, não distenda-se na subida?

 Utilizaremos os conceitos de osciladores harmônicos. Sabe-se que a velocidade máxima do deslocamento que a mola causa é quando cosseno ou seno vale 1, ou seja, 
   \[
     V_{max} = A\omega .
   \]
 Além disso, analisando o sistema dinâmico, 
   \[
     k(L-l) = mg,
   \]
em que k é o coeficiente da mola. Com isso, 
   \[
     \frac{k}{m} = \frac{mg}{m(L-l)} \Rightarrow \frac{k}{m} = \frac{g}{L-l}.
   \]
 Disto segue que a velocidade da martelada será 
   \[
     v_{0}=A\sqrt[]{\frac{k}{m}} = A\sqrt[]{\frac{g}{L-l}}
   \]
e, portanto, 
   \[
     A = v_{0}\sqrt[]{\frac{L-l}{g}}.
   \]
 Assim, a altura h será dada por 
   \[
     h = l + A = l + v_{0}\sqrt[]{\frac{L-l}{g}}.
   \]
 Com relação ao tempo necessário para que a mola atinja a amplitude máxima, note que ela levará \(\Delta t = \frac{3}{4}T\). Isto pode ser reescrito como 
   \[
     \Delta t = \frac{3}{4}T = \frac{3}{4}\frac{1}{f} = \frac{3}{4\frac{\omega }{2\pi }} = \frac{3\pi }{2\omega } = \frac{3\pi }{2\sqrt[]{\frac{g}{L-l}}} = \frac{3}{2}\pi \sqrt[]{\frac{L-l}{g}}
   \]
 Finalmente, sobre a velocidade, ela deverá satisfazer 
   \[
     \frac{1}{2}mv_{0}^{2} = mg(L-l) + \frac{1}{2}k(L-l)^{2} = 0 \Rightarrow v_{0}^{2} = 2g(L-l) - \frac{k}{m}(L-l)^{2} = 2g(L-l)-g(L-l) = g(L-l).
   \]
\end{example}
\begin{example}
Considere dois auto falantes distando d um do outro e, em uma distância D, coloque um estudante de graduação. Suponha que as ondas sonoras emitidas são esféricas e formam
um ângulo \(\theta \) com a reta horizontal. A distância entre uma onda lançada pelo primeiro pra uma onda lançada pelo segundo é \(\Delta S = d\sin^{}{(\theta )}\) - a diferença de caminho entre
as duas ondas.

Para que o primeiro ângulo mínimo que faça o estudante ouvir o som ocorra, é preciso que \(\Delta S = \frac{\lambda }{2}\), tal que, como os ângulos são pequenos,
 \[
   \frac{\lambda }{2} = d\theta_{0} \Rightarrow \lambda.
 \]
Usando que \(\lambda  = \frac{v_{som}}{f},\) encontra-se o valor de \(\theta \): 
 \[
   \theta_{0} = \frac{\lambda }{2d} = \frac{v_{som}}{2df}.
 \]
Sabendo o valor do ângulo mínimo, segue que o primeiro ângulo máximo será 
 \[
   \theta_{max} = \frac{\lambda }{df}\approx \frac{y}{D}.
 \]
\end{example}
\begin{example}
Suponha uma barra de tamanho L presa ao teto e, na metade dela, uma mola de constante elástica k é ligada, sendo ela presa horizontalmente a outra superfície.
Qual é a velocidade angular desse sistema?

  A energia desse sistema é 
  \[
    E = \frac{1}{2}mv^{2} + mg \frac{L}{2}\biggl(1-\cos^{}{(\theta )}\biggr)
  \]
  Com a forma que ele foi descrito, sendo \(\theta \) pequeno, vale que 
  \[
    \frac{x}{\frac{L}{2}} = \theta,\quad \cos^{}{(\theta )} = 1 -\frac{\theta^{2}}{2},\quad x = \theta \frac{L}{2}. 
  \]
  Assim, utilizando \(v = \frac{L}{2}\dot\theta \) 
  \begin{align*}
    E &= \frac{1}{2}mv^{2} + mg \frac{L}{2}\biggl(1 - 1 + \frac{\theta ^{2}}{2}\biggr) + \frac{1}{2}k\theta^{2}\frac{L^{2}}{4} \\
      &= \frac{1}{2}mv^{2} + \frac{mgL\theta ^{2}}{4} + \frac{1}{2}k \frac{L^{2}}{4}\theta^{2}\\
      &= \frac{1}{2}mv^{2} + \frac{1}{2}\biggl[\frac{mgL}{2} + \frac{kL^{2}}{4}\biggr]\theta^{2}\\
      &= \frac{1}{2}m \frac{L^{2}}{4}\dot \theta ^{2} + \frac{1}{2}\biggl[\frac{mgL}{2}+ \frac{kL^{2}}{4}\biggr]\theta^{2}\\
      &= \frac{1}{2}m'v^{2} + \frac{1}{2}k'x^{2}.
  \end{align*}
  Finalmente, como \(\omega^{2} = \frac{k'}{m'}\), 
    \[
      \omega^{2} = \biggl[\frac{mgL}{2}+ \frac{kL^{2}}{4}\biggr]\frac{1}{m \frac{L^{2}}{4}} = \frac{2g}{L} + \frac{k}{m}.
    \]
\end{example}
\newpage

\section{Aula 26 - 30/10/2023}
\subsection{Motivações} 
\begin{itemize}
  \item a
\end{itemize}
\end{document}


\documentclass[PhysicsII/phsyicsII_notes.tex]{subfiles}
\begin{document}
\section{Aula 26 - 30/10/2023}
\subsection{Motivações}
\begin{itemize}
	\item Temperatura;
	\item Escalas de Temperatura;
	\item Lei dos Gáses Ideais.
\end{itemize}
\subsection{Temperatura}
\paragraph{} O senso de tato é o sentido responsável por nossa noção de temperatura, permitindo-nos diferenciar
entre quente e frio. Esse fenômeno é conhecido como contato térmico, pelo menos um dos tipos disso. Entre líquidos e sólidos,
a consequência é aquecer/esfriar, resultando numa variação de volume. Para um gás com volume constante, no entanto, uma
das consequências é a variação de pressão e, se a pressão for constante, é a variação de volume.

Dados dois blocos, digamos um de cobre e outro de ferro, com temperaturas respectivas \(T_{1}, T_{2}, T_{1} > T_{2}\), então ao
juntarmos os dois, de forma que eles estejam em contato térmico, a temperatura do ``blocão'' unido, \(T_{3}\), terá a seguinte relação -
\(T_{2} < T_{3} < T_{1}\). Isso tem relação direta com a chama \textbf{Lei Zero da Termodinâmica}:
\begin{quote}
	\textit{``Se o bloco 1 está em equilíbrio térmico com o bloco 2 e o bloco 2 está em equilíbrio com um bloco 3, então
		o bloco 1 está em equilíbrio com o bloco 3''}
\end{quote}
Uma pergunta básica a ser feita é como exatamente fazemos para medir temperatura? Colocar a mão não costuma ser evolutivamente favorável, visto que pode resultar
no fim da continuação genética da espécia (morte). Para evitar isso, foram desenvolvidas as escalas de temperatura, como Celsius e Fahrenheit, que baseiam-se
em pontos específicos referenciais. No caso do grau Celsius, baseia-se na temperatura em que a água, \(H_{2}O\), vira gelo na pressão de 1atm, chamado ponto zero.
Outro referencial dessa escala é a temperatura para que a água entre em ebulição a 1atm de pressão, que leva o valor de \(100^{\circ{}}C\)

No passado, as pessoas utilizavam termômetros de mercúrio com marcações que denotavam esses dois valores referenciais da escala Celsius. Quando a temperatura subia, a coluna
de mercúrio subia junto. Se a coluna tem tamanho \(L_{0}\) no ponto zero graus e \(L_{100}\) é o tamanho em 100 graus, chamamos de \(L_{T}\) o tamanho para uma temperatura T e obtemos a
temperatura medida \(T_{L}\) como
\[
	T_{L} = \frac{L_{T} - L_{0}}{L_{100}-L_{0}}\times 100^{\circ{}}C.
\]
Para converter entre Celsius e Fahrenheit, os referenciais tornam-se \(0^{\circ{}}C\mapsto 32^{\circ{}}F, 100^{\circ{}}C\mapsto 212^{\circ{}}F\), mas esses \textbf{não são os valores da conversão.}
Para a conversão em si, denotando a temperatura em Celsius por \(T_{C}\) e a temperatura em Fahrenheit por \(T_{F}\),
\[
	T_{F} = \frac{9}{5}T_{c} + 32,\quad T_{C} = \frac{5}{9}\biggl(T_{F}-32\biggr).
\]
As duas escalas coincidem na temperatura \(-40^{\circ{}}F = -40^{\circ{}}C\). Curiosamente, essa temperatura é fria o suficiente para congelar água fervendo antes dela chegar ao chão
\begin{example}
	Suponha que uma pessoa está com febre a uma temperatura de \(40^{\circ{}}C\) e essa pessoa viaja para um médico estadunidense. Qual temperatura ele deve alertar ao médico?

	Fazendo a conversão,
	\[
		T_{F} = \frac{9}{5}40 +32 = 104^{\circ{}}F.
	\]
\end{example}
Devido à toxicidade do mercúrio, o uso do termômetro feito com ele foi cessado. Outro método permite medir as temperaturas com um termômetro à gás. Para seu funcionamento,
considera-se um recipiente ligado a uma coluna de mercúrio. Em seguida, mede-se a pressão dentro do recipiente, que será dada por \(P = P_{atm} = \rho gh\) e a temperatura será
\[
	T_{p} = \frac{P_{T} - P_{0^{\circ{}C}}}{P_{100^{\circ{}}C}-P_{0^{\circ{}}C}}\times 100^{\circ{}}C.
\]
A água possui um ponto triplo no qual os três estados básicos - líquido, sólido e gasoso - ocorrem. Ela é dada por \(4,8mm\) de Hg, ou \(0,01^{\circ{}}C = 273,16K\).
Este ``momento'' é denotado por um índice 3, então \(P_{3}, T_{3}\) denotam a pressão e a temperatura do ponto triplo d'água.

É definida uma escala absoluta de temperatura, usada em produção científica e acadêmica e chamada Kelvin. Ela é a temperatura do Sistema Interacional e será a que usaremos
daqui pra frente. Para converter de Celsius para Kelvin, usa-se
\[
	T_{C} = T_{K} - 273,15.
\]
\subsection{A Lei dos Gases Ideais}
No final do século XVIII, começa-se o desenvolvimento do estudo dos gases, eventualmente levando à invenção do motor à gas. A primeira lei que foi declarada
é a Lei de Boyle, o qual notou que, mantida constante a temperatura, o produto da pressão pelo volume era uma constante, ou seja,
\[
	\hypertarget{boyle}{\boxed{\text{Se a \textbf{temperatura} é constante, }PV = cte}}
\]
Em seguida, Jacques Charles e Gay-Lussac observaram que, caso a temperatura varie,
\[
	\hypertarget{gay}{\boxed{PV = kT.}}
\]
Com base nelas, dados dois reservatórios com valores \(P_{1}, V_{1}, T_{1}\), ao realizar contato entre os dois, \(P_{1}, T_{1}\) deverão ser os mesmos após a junção.
Além disso, o volume valerá \(2V_{1}.\) Assim, o k que aparece na \hyperlink{gay}{segunda relação} deve ser o valor do número de constituintes do gás. Atualmente, sabemos que
\(k = Nk_{B},\) em que \(k_{B} = 1,38 \cdot 10^{-23}\frac{1}{J} = 8,617 \cdot 10^{-5}\frac{eV}{K}\) é a \textbf{Constante de Boltzmann}. Como esse número é muito pequeno, alternativamente,
costuma-se usar o \textbf{Número de Avogadro } ao invés da Constante de Boltzmann, \(N_{A} = 6,022 \cdot 10^{23}\), fazendo com que \(N = n \cdot N_{A}\), em que n é o número
de moles, e a fórmula para k torna-se \(k = n \cdot N_{A} \cdot k_{B} T = nRT\), donde obtém-se a \textbf{Lei dos Gases Ideais:}
\[
	\hypertarget{pvnrt}{\boxed{PV = nRT}},
\]
em que \(R = 8,314 \frac{J}{molK} = 0,68206 \frac{\ell atm}{mol K}\). Apesar de ser uma lei, ela possui alguns problemas, tal como, ao fazer um gráfico Pressão por Volume,
não é possível encontrar, por exemplo, o momento em que a água liquefaz-se. Isto ocorre pois a Lei dos Gases Ideais não leva em conta a interação entre as partículas. Outra questão
é que, ao tomar o limite de pressão infinita, mantendo-se a temperatura constante, ela não considera que o volume deve ser finito. Em suma, esse modelo não prevê a mudança de fase
e nem o volume dos constituintes. Além disso,, note que
\[
	\frac{PV}{nT} = R,
\]
ou seja, visto como uma função da pressão e colocado num gráfico \(\frac{PV}{nT}\times P,\) cada elemento terá um valor diferente de R, chegando ao valor \(8,314 \frac{J}{mol K}\) apenas
se a pressão for muito pequena, justamente por não levar em conta as interações entre as partículas.

\end{document}

\documentclass[phsyicsII_notes.tex]{subfiles}
\begin{document}
\section{Aula 32 - 23/11/2023}
\subsection{Motivações}
\begin{itemize}
	\item Refrigerador;
	\item Bomba de Calor.
\end{itemize}
\subsection{Refrigerador}
Um \textbf{refrigerador ideal} é uma máquina térmica que transfere o calor frio para o calor quente. Como isso não é possível
naturalmente, é necessário exercer trabalho nesses tipos de reservatórios. Buscamos otimizar essa troca - gastar o menor
trabalho possível para transferir a maior quantidade possível. Essa eficácia é medida pelo \textbf{coeficiente de desempenho},
dado pela razão
\[
	\boxed{CD = \frac{Q_{f}}{W} = \frac{Q_{f}}{Q_{q}-Q_{f}}}.
\]
\begin{example}
	Uma pessoa está organizando uma festa e precisa de gelo para ela. Ela tem 1L de \(H_{2}O\) a uma temperatura de \({10}^{\mathrm{o}}\) e dispões de 30 minutos até
	o início da festa. Dado que seu refrigerador atua com uma potência de \(P=550W\), com eficácia de refrigeração de \(10\%\) para resfriar,
	e que seu coeficiente de desempenho é de \(CD=5,5\), quanto tempo leva para fazer esse gelo?

	O primeiro passo é descobrir quanto calor deve sair dessa água para ela virar gelo, ou seja, qual o valor de \(Q_{f}\). Temos
	\[
		Q_{f} = Q_{\text{gelar}} + Q_{\text{solidificar}} = mc\Delta T + mL
	\]
	A massa foi dada, vale 1000g, o coeficiente de calor específico da água é de 4,18 e, por fim, a variação de temperatura é de \({10}^{\mathrm{o}}\).
	Quanto à solidificação, o m também é 1000g e L vale 333,5, tal que
	\[
		Q_{f} = 10^{3}\times4,18\times 10\approx 4,18\times10^{4}J + 10^{3}\times 333,5 = 333,5\times 10^{3}J = (41,8+333,5)kJ.
	\]
	A princípio, de \(0.1P = P' = \frac{W}{\Delta t},\) segue que \(\Delta t = \frac{W}{P}\), em que \(W = \frac{Q_{f}}{CD}\). Assim,
	\[
		\Delta t = \frac{Q_{f}}{CD\times P'} = \frac{Q_{f}}{5,5P'} = 1240s\approx 20,5 \text{min.}
	\]
\end{example}

Vimos dois enunciados da segunda lei da termodinâmica na última aula. Note que ambos são, na verdade,
antíteses:

Considere uma máquina térmica que sofre um trabalho de 50J para mover 100J de calor frio para um reservatório quente, resultando
num fluxo total de 150J até ele. Esta máquina inclui uma conexão do reservatório quente para o cano do trabalho, violando um dos postulados,
no sentido dela entregar todo o calor como trabalho. Este exemplo pode ser visto, também, como um refrigerador ideal que passa todo o calor
frio para o reservatório quente, sem perda. Nenhuma dessas situações exsitem na vida real, ou seja, recuperamos o primeiro postulado a partir do segundo,
pois mostramos que, nas condições dele, um reservatório não pode converter totalmente o calor em trabalho.
Um processo análogo pode mostrar o caminho oposto - supondo que o primeiro postulado é falso, obtemos uma contradição no segundo.
\subsection{Bomba de Calor}
Uma \textbf{bomba de calor} é um equema no qual, dado um reservatório quente e outro frio, buscamos levar calor do reservatório frio, por meio de um trabalho, para o reservatório quente.
Buscamos, como antes, maximizar a eficácia desse trabalho. No entanto, como a direção é outra, o coeficiente muda também:
\[
	\boxed{CD_{BT} = \frac{Q_{q}}{W} = \frac{1}{1-\frac{Q_{f}}{Q_{q}}}}.
\]
Como \(Q_{q} - Q_{f} = W,\) segue que \(Q_{q} = W + Q_{f}\). Não bastando, utilizando a fŕomula acima, chegamos em
\[
	CD_{BT} = \frac{W+Q_{f}}{W} = 1 + \frac{Q_{f}}{W} = 1 + CD.
\]

\end{document}

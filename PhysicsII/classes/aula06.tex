\documentclass[physicsII_notes.tex]{subfiles}
\begin{document}
\section{Aula 06 - 23/08/2023}
\subsection{Motivações}
\begin{itemize}
	\item Planos inclinados;
	\item Refrigerante VS Cilindro;
	\item Bola deslizando.
\end{itemize}
\subsection{Objeto Rolante em Plano Inclinado}
Coloque uma bola de massa M, momento de inércia I e raio R em um plano inclinado a uma ângulo \(\theta \).
Analisando as forças presentes, estão inclusas a da gravidade, a normal e uma força de atrito. Considerando x a direção em que a bola
está indo para. Nessa direção, as relações das forças serão
\[
	Mg\sin{(\theta )} - f_{at} = Ma_{cm}.
\]
Quanto à rotação, temos
\[
	f_{at}R = I_{cm}\alpha.
\]
Utilizando \(a_{cm} = R\alpha \), relacionamos as duas como segue:
\begin{align*}
	 & f_{at} = Mg\sin{(\theta )} - Ma_{cm}                               \\
	 & Mg\sin{(\theta )} - Ma_{cm} = \frac{I_{cm}}{R}\frac{a_{cm}}{R}     \\
	 & Mg\sin{(\theta )} = a_{cm}\biggl[M + \frac{I_{cm}}{R^{2}}\biggr]   \\
	 & Mg\sin{(\theta )} = Ma_{cm}\biggl[1 + \frac{I_{cm}}{MR^{2}}\biggr] \\
	 & a_{cm} = \frac{g\sin{(\theta )}}{1 + \frac{I_{cm}}{MR^{2}}}.
\end{align*}
Com isso, conseguimos descrever a força de atrito utilizando que, na esfera, \(I = \frac{2}{5}MR^{2}\). Logo,
\begin{align*}
	 & a_{cm} = \frac{5}{7}g\sin{(\theta )}                                                                \\
	 & f_{at} = \frac{2}{5}\frac{MR^{2}}{R^{2}}\frac{5}{7}g\sin{(\theta )} = \frac{2}{7}Mg\sin{(\theta )}.
\end{align*}
\subsubsection{Simetrias}
Se um corpo tem simetria esférica ou cilíndrica, então \(I = \beta mR^{2}\). Em particular, podemos usar isso para
generalizar o raciocínio realizado acima. De fato, tanto \(a_{cm}\) quanto \(f_{at}\) podem ser reescritos como segue:
\begin{align*}
	                                                  & a_{cm} = \frac{g\sin{(\theta )}}{1+\beta }                                                       \\
	f_{at} = \frac{\beta Mg\sin{(\theta )}}{1+\beta } & = \frac{Mg\sin{(\theta )}}{\frac{1}{\beta }(1+\beta )} = \frac{Mg\sin{(\theta )}}{\beta^{-1}+1}.
\end{align*}
Observa-se, assim, que entre uma esfera, um cilindro e um aro num plano inclinado (de mesmo raio), a esfera chegará primeiro ao chão,
visto que ela tem um \(\beta \) menor.

Agora, considere um cilindro de massa M e raio R iguais aos de uma lata de refrigerante e coloque os dois juntos para descer um plano inclinado. Quem chegará primeiro?
A resposta (por incrível que pareça) é o refrigerante, pois o liquido dentro não possui momento de inércia, ou seja, soltar ela com líquido ou vazia resultará no mesmo! Analisando a energia
desse sistema, temos
\[
	E_{mec_{i}} = mgh\quad \&\quad E_{mec_{f}} = \frac{1}{2}mv_{cm}^{2} + \frac{1}{2}I_{cm}\omega^{2},
\]
em que \(\omega = \frac{v_{cm}}{R}\). Assim,
\[
	E_{mec_{f}}=\frac{1}{2}mv_{cm}^{2} + \frac{1}{2}\beta mR^{2}\frac{v_{cm}^{2}}{R^{2}}.
\]
Juntando ambas,
\[
	\frac{1}{2}mv_{cm}^{2}[1+\beta ] = mgh \Rightarrow v_{cm}^{2} = \frac{2gh}{1+\beta }
\]
Como a latinha é considerada um aro, ela ganha!

Vamos voltar nossa atenção à força de atrito. Vimos antes que
\[
	f_{at} = \frac{mg\sin{(\theta )}}{1 + \beta^{-1}}.
\]
Sabe-se que a força de atrito tem um valor máximo, dado por \(f_{at_{max}} = \mu_{e}N\), ou seja,
\(f_{at}\leq f_{at_{max}}\). Em outras palavras,
\[
	\frac{mg\sin{(\theta )}}{1+\beta^{-1}}\leq \mu_{e}mg\cos{(\theta )} \Longleftrightarrow \tan{(\theta )}\leq \mu_{e}(1+\beta^{-1}).
\]
Com isso, obtemos um ângulo máximo no qual o plano inclinado pode estar sem que a bola comece a rolar deslizando.
\begin{example}
	Para uma esfera, em que \(\beta =\frac{2}{5}\), o ângulo máximo é
	\[
		\tan{(\theta )}\leq 3.5\mu_{e}.
	\]
	Para um cilindro, com \(\beta = \frac{1}{2}\), tem-se
	\[
		\tan{(\theta )}\leq 3\mu_{e}.
	\]
	Por fim, para um aro, no qual \(\beta =1\),
	\[
		\tan{(\theta )}\leq 2\mu_{e}.
	\]
\end{example}

\subsection{Bola Deslizando}
Agora, suponha que colocamos uma bola de massa m deslizando (apenas movimento translacional) com velocidade v em uma superfície com atrito. Note que as forças agindo são
a peso, a normal e a força de atrito. Em termos de translação, então, temos
\[
	-f_{at} = ma_{cm} \Rightarrow a_{cm} = -\frac{f_{at}}{m} = -\frac{\mu_{c}N}{m} = -\mu_{c}g.
\]
Por outro lado, quanto à rotação, começamos notando que o torque vale \(\tau = I_{cm}\alpha \). Assim,
\[
	f_{at}R = I_{cm}\alpha \Longleftrightarrow a_{c}mgR = \frac{2}{5}mR^{2}\alpha
\]
Portanto,
\[
	\alpha = \frac{5}{2}\frac{\mu_{c}g}{R}.
\]
A velocidade do centro de massa, então, pode ser obtida com \(v_{cm} = v - a_{cm}t = v - \mu_{c}gt\). Nota-se, então,
que há uma redução na velocidade, até que, eventualmente, ela para de deslizar e passa a rotacionar, fazendo com que \(v_{cm} = \omega R\).
Nesse instante, tem-se
\[
	v-\mu_{c}gt = R\frac{5}{2}\frac{\mu_{c}g}{R}t \Rightarrow v = \mu_{c}gt\biggl[1+\frac{5}{2}\biggr] = \mu_{c}gt \frac{7}{2}.
\]
Portanto, o tempo parar ela parar de deslizar é
\[
	\boxed{t^{*} = \frac{2v}{7\mu_{c}g}.}
\]
A partir disso, podemos também encontrar a distância percorrida:
\begin{align*}
	x(t^{*}) & = vt^{*} - \frac{\mu_{c}gt^{*^{2}}}{2}                                                               \\
	         & = \frac{v2v}{7\mu_{c}g} - \frac{\mu_{c}g}{2}\frac{4v^{2}}{49\mu_{c}^{2}g^{2}}                        \\
	         & = \frac{2v^{2}}{7\mu_{c}g} - \frac{2}{49}\frac{v^{2}}{\mu_{c}g}                                      \\
	         & = \frac{v^{2}}{\mu_{c}g}\biggl[\frac{14-2}{49}\biggr] = \boxed{\frac{12}{49}\frac{v^{2}}{\mu_{c}g}.}
\end{align*}
\end{document}

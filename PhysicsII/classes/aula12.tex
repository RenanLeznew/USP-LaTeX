\documentclass[PhysicsII/physicsII_notes.tex]{subfiles}
\begin{document}
\section{Aula 12 - 18/09/2023}
\subsection{Motivações}
\begin{itemize}
	\item Fluídos;
	\item Pressão e Densidade.
\end{itemize}
\subsection{Fluídos e Massa Específica}
Existem dois tipos de fluidos. Um líquido sempre ocupa a parte mais baixa do recipiente, enquanto que um gás
sempre ocupa todo o espaço do recipiente. Utilizamos líquidos para coisas como geração de energia, aerodinâmica e medicina,
e há a presença de ligações atômicas/moleculares.

Definimos a densidade por
\[
	\rho = \frac{M}{V} = \frac{dm}{dV}.
\]
Para uma noção mais mundana, a densidade da água vale \(\rho = 1 \frac{g}{cm^{3}} = 100 \frac{kg}{m^{3}} = 1 \frac{kg}{l}\). Por outro lado, estrelas de neutrons
têm densidade de \(\rho = 7\times 10^{17}\frac{kg}{m^{3}}\). Caso um objeto na água tenha densidade \(\rho > \rho_{H_{2}O},\) ele afundará e, se
\(\rho < \rho_{H_{2}O}\), o objeto flutuará.
\begin{example}
	Considere um recipiente com 200ml de água a 4 graus Celsius. Esquentemos este líquido até 80 graus Celsius.
	Para 6g de \(H_{2}O\) perdidos, qual será a nova densidade da água?
	\[
		\rho = \frac{(200 - 6)g}{200ml}\approx 0.97 \frac{g}{l} = 970 \frac{kg}{m^{3}}.
	\]
\end{example}
Outra noção importante é a de pressão, definida como
\[
	P = \frac{F}{A}.
\]
A unidade padrão da pressão é o Pascal, \([P]= \frac{N}{m^{2}} =\) Pascal. Outras unidades incluem o psi,
ou libra por polegada quadrada, e a atmosfera, ou atm. Segue que
\[
	1atm = 101325 Pa = 14.7 \frac{lb}{in^{2}}.
\]
Nossa intuição diz que, ao aplicarmos \(\Delta P\), também estaremos causando uma variação de volume \(\Delta V\), mas negativo.
Por isso, falamos de módulo volumétrico, definido como
\[
	\beta = \frac{-\Delta P}{\frac{\Delta V}{V}}.
\]
A compressibilidade é definida pelo inverso do módulo volumétrico:
\[
	\frac{1}{\beta } = \frac{\frac{\Delta V}{V}}{\Delta P} = \frac{-\Delta V}{V\Delta P}.
\]
Neste curso, trabalharemos apenas com líquidos não compressíveis.

Considere um disco de área A, submerso a uma profundidade z em um líquido de densidade \(\rho \).
Este disco está sujeito a uma pressão P(z) por cima e P(z+dz) por baixo, em que dz é a variação de profundidade que ele sofre.
Além disso, há a força peso \(dm \vec{g}\). Utilizando que \(F = P A,\) obtemos a equação
\[
	\biggl[P(z+dz) - P(z)\biggr]A = dmg = Adz\rho g
\]
Com isso, chegamos em
\[
	P(z+dz) - P(z) = \rho g dz \Rightarrow \frac{dP}{dz} = \rho g.
\]
Integrando com respeito a z, encontramos a relação da pressão conforme submergimos num líquido:
\[
	\hypertarget{pressure_submerging}{\boxed{P(z) = P_{0}+\rho gz.}}
\]

\end{document}

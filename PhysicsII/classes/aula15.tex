\documentclass[physicsII_notes.tex]{subfiles}
\begin{document}
\section{Aula 15 - 25/09/2023}
\subsection{Motivações}
\begin{itemize}
	\item Equação de Bernoulli;
	\item Tubo de Venturi;
	\item Tubo de Pitot.
\end{itemize}
\subsection{Pressão}
Imagine um tubo que começa a subir na vertical e está preenchido por um fluído. Considere um ponto 1 na parte mais inferior deste tubo, com uma altura \(y_{1}\),
e um ponto 2 na parte superior do mesmo, a uma altura \(y_{2} > y_{1}\). No ponto 1, hão velocidade e pressão \(v_{1}, P_{1}\) e, respectivamente no ponto 2, \(v_{2}, P_{2}\).
Passam, por esses pontos, quantidades de massa \(\Delta m\). Vamos analisar esse sistema.

Sabe-se que \(W = \Delta \mathbb{K}\). As forças que realizam trabalho, nesta configuração, são a pressão e a gravidade. Considerando as quantidades em questão,
\[
	\Delta \mathbb{K} = \frac{1}{2}\Delta m v_{2}^{2} - \frac{1}{2}\Delta m v_{1}^{2}
\]
Quanto ao trabalho, também, para a gravidade, tem-se
\[
	W_{g} = \Delta mgy_{1} - \Delta mgy_{2}
\]
e, para as pressões,
\[
	W_{p_{1}} = F_{1}\Delta x_{1} = p_{1}A_{1}\Delta x_{1} = p_{1}\Delta V\quad\&\quad W_{p_{2}} = -p_{2}\Delta V,
\]
totalizando
\[
	W_{p_{t}} = W_{p_{1}} + W_{p_{2}} = (p_{1}-p_{2})\Delta V.
\]
Com isso, obtemos
\[
	W_{g}+W_{p_{t}} = \Delta \mathbb{K}.
\]
Podemos, então, destrinchar a equação como segue
\begin{align*}
	                                                      & \Delta mgy_{1}-\Delta mgy_{2} + (p_{1}-p_{2})\Delta V = \frac{1}{2}\Delta mv_{2}^{2}-\Delta mv_{1}^{2}                        \\
	\Longleftrightarrow\quad                              & \frac{1}{2}\Delta mv_{1}^{2} + p_{1}\Delta V + \Delta mgy_{1} = \frac{1}{2}\Delta mv_{2}^{2} + p_{2}\Delta v + \Delta mgy_{2} \\
	\underbrace{\Longrightarrow}_{\Delta m=\rho \Delta V} & \frac{1}{2}\rho v_{1}^{2} + p_{1} + \rho gy_{1} = \frac{1}{2}\rho v_{2}^{2} + p_{2} + \rho gy_{2}.
\end{align*}
Em outras palavras, estes valores das igualdades são \textbf{constantes}, o que nos leva à equação de Bernoulli:
\[
	\hypertarget{bernoulli}{\boxed{\frac{1}{2}\rho v_{1}^{2}+p_{1}+\rho gy_{1}=constante}}
\]
O problema com ela é que ela descreve apenas sistemas idealizados, que muitas vezes não ocorrem na vida real. Por exemplo,
ela não leva em conta a viscosidade do líquido, o que resulta em inconsistências no mundo real.
\begin{example}
	Considere um tanque preenchido com um fluído até uma altura \(y_{1}\), com área de abertura \(A_{1}\). Neste tanque, há um buraco
	a uma altura \(y_{2}\) menor que \(y_{1}\) e pressão \(A_{2}\). Suponhamos que \(A_{1}>> A_{2}\). Utilizando a equação
	\[
		A_{1}v_{1} = A_{2}v_{2},
	\]
	concluímos que \(v_{1}\approx0\). Assim, utilizando Bernoulli,
	\[
		p_{1} + \rho gy_{1} + \underbrace{\frac{1}{2}\rho v_{1}^{2}}_{0} = p_{2} + \rho gy_{2} + \frac{1}{2}\rho v_{2}^{2}.
	\]
	Como as pressões são as mesmas,
	\[
		\frac{1}{2}\rho v_{2}^{2} = \rho g(y_{1}-y_{2}) \Rightarrow v_{2}^{2} = 2g\Delta y \Rightarrow v_{2}=\sqrt[]{2g\Delta y}
	\]
	Até mesmo sem fazer a suposição das áreas, sabe-se que
	\[
		v_{1} = \frac{A_{2}}{A_{1}}v_{2},
	\]
	tal que
	\[
		\rho gy_{1} + \frac{1}{2}\rho \frac{A_{2}^{2}}{A_{1}^{2}}v_{2}^{2} = \frac{1}{2}\rho v_{2}^{2} + \rho gy_{2},
	\]
	donde segue que
	\[
		g(y_{1}-y_{2}) = \frac{v_{2}^{2}}{2}\biggl(1-\frac{A_{2}^{2}}{A_{1}^{2}}\biggr),
	\]
	ou seja,
	\[
		v_{2}^{2} = \frac{2g\Delta y}{1-\frac{A_{2}^{2}}{A_{1}^{2}}} \Rightarrow v_{2} = \sqrt[]{\frac{2g\Delta y}{1-\frac{A_{2}^{2}}{A_{1}^{2}}}}.
	\]
\end{example}
\subsection{Tubo de Venturi}
O tubo de Venturi é um aparato para medir a velocidade de um certo fluído (recomendo olhar uma imagem no Google!), baseado no princípio de Bernoulli.

Ele consiste de três seções principais, uma de entrada, uma de ``garganta'' - um caminho estreito ligando a primeira e a terceira seção, e a terceira, a seção de saída.
Conecta-se a garganta e a primeira seção, além da abertura delas, com uma coluna de líquidos em formato de U.

Dado que a área das seções um e três é \(A_{1}\), a pressão é \(p_{1}\), a densidade é \(\rho_{f}\)
e a velocidade é \(v_{1}\) e que, na garganta, \(A_{2}, v_{2}, p_{2}\) são os valores respectivos,
temos duas equações descrevendo este sistema, \hyperlink{bernoulli}{Bernoulli}
e \(A_{1}v_{1} = A_{2}v_{2}.\) Através de ambas, obtemos
\begin{align*}
	                    & p_{1}+\frac{1}{2}\rho_{f}v_{1}^{2} = p_{2} + \frac{1}{2}\rho_{f}v_{2}^{2}               \\
	                    & \frac{1}{2}\rho_{f}(v_{1}^{2}-v_{2}^{2}) = p_{2}-p_{1}                                  \\
	\Longrightarrow     & \frac{1}{2}\rho_{f}\biggl(1-\frac{A_{1}^{2}}{A_{2}^{2}}\biggr)v_{1}^{2} = p_{2}-p_{1}   \\
	\Longleftrightarrow & \frac{1}{2}\rho_{f} \biggl(\frac{A_{1}^{2}}{A_{2}^{2}}-1\biggr)v_{1}^{2} = p_{1}-p_{2}.
\end{align*}
Sendo \(\rho_{L}\) a densidade do líquido dentro do tubo em U e h a diferença da altura das pontas do líquido, aplicamos Bernoulli a ele e à seção 1, tal que
\[
	p_{1}+\rho_{F}gh = p_{2}+\rho_{L}gh \Rightarrow p_{1}-p_{2} = gh(\rho_{L}-\rho_{f}).
\]
Chamando de \(r = \frac{A_{1}}{A_{2}}\) e juntando as duas equações,
\[
	\frac{1}{2}\rho_{f}(r^{2}-1)v_{1}^{2} = gh(\rho_{L}-\rho_{f}).
\]
Portanto,
\[
	v_{1}^{2} = \frac{2gh(\rho_{L}-\rho _{f})}{(r^{2}-1)\rho_{f}}.
\]
\subsection{Tubo de Pitot}
O tubo de Pitot é um instrumento utilizado para medir a velocidade de um fluido
em um ponto específico de um duto ou fluxo livre. Ele foi nomeado em homenagem
ao engenheiro francês Henri Pitot, que o introduziu no início do século XVIII.
O tubo de Pitot é amplamente utilizado em aplicações aeronáuticas para medir a
velocidade do ar em relação a aeronaves, mas também tem aplicações em outros
campos da engenharia.

Ele é composto por um tubo externo chamado tubo estático e um tubo interno
chamado tubo de impacto ou tubo dinâmico. A extremidade aberta do tubo interno
é direcionada diretamente para o fluxo do fluido, enquanto as aberturas laterais
do tubo externo medem a pressão estática do fluido.

Quando o fluido atinge a extremidade aberta do tubo de impacto, ele é trazido
ao repouso (velocidade zero). A pressão medida neste ponto é chamada de pressão
de impacto ou pressão total, denotada por \(p_{1}\). As aberturas laterais do tubo estático medem a pressão
estática do fluido, denotada por \(p_{2}\). A diferença entre a pressão de impacto e a pressão estática é chamada
de pressão dinâmica, e está diretamente relacionada à velocidade do fluido. Para medir a velocidade
do fluído, utiliza-se a equação de Bernoulli:
\[
	p_{1} + \underbrace{\frac{1}{2}\rho v_{1}^{2}}_{0} = p_{2} + \frac{1}{2}\rho v_{2}^{2}.
\]
Assim,
\begin{align*}
	 & p_{1}-p_{2} = \frac{1}{2}\rho_{f}v_{2}^{2}     \\
	 & \rho_{L}gh = \frac{1}{2}\rho_{f}v_{2}^{2}      \\
	 & v_{2} = \sqrt[]{\frac{2\rho_{L}gh}{\rho_{f}}}.
\end{align*}
\end{document}

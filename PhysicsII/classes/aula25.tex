\documentclass[phsyicsII_notes.tex]{subfiles}
\begin{document}
\section{Aula 25 - 23/10/2023}
\subsection{Motivações}
\begin{itemize}
	\item Exercícios.
\end{itemize}
\subsection{Exercícios}
\begin{example}
	O sifão descreve um sistema em que um recipiente com líquido recebe um cano cuja extremidade final está abaixo de onde ele é colocado, com uma diferença h
	em relação à superfície do líquido. Qual é a altura máxima que ele pode ser posto? E qual é o volume do líquido na extremidade final?

	Sabemos que
	\[
		P_{1} + \rho gy_{1} + \frac{\rho v_{1}^{2}}{2} = cte.
	\]
	Assim,
	\[
		P_{A} + \rho gy_{A} + \frac{1}{2}\rho V_{A}^{2} = P_{B} + \rho gy_{B} + \frac{1}{2}\rho V_{B}^{2}
	\]
	e, disto, \(A_{A}V_{A} = A_{B}V_{B}\), tal que \(A_{A} >> A_{B}\) e \(V_{A}\approx 0\). Suponha que a altura da extremidade final \(y_{B} = 0\). Com isso,
	\begin{align*}
		            & P_{atm} + \rho gh = P_{atm} + \frac{1}{2}\rho V_{B}^{2} \\
		\Rightarrow & v_{B} = \sqrt[]{2gh}.
	\end{align*}
	Concluímos, por fim, o valor da altura máxima:
	\begin{align*}
		P_{A} + \rho gy_{A} + \frac{1}{2}\rho V_{A}^{2}     & = P_{c} + \rho gy_{c} + \frac{1}{2}\rho V_{c}^{2}                                 \\
		\underbrace{\Rightarrow}_{y_{A}=0, y_{c} = y_{max}} & P_{A} = P_{c} + \rho gy_{max} + \frac{1}{2}\rho V_{c}^{2}
		\Rightarrow                                         & y_{max} = \frac{P_{A}}{\rho g} \approx \frac{10^{5}}{10^{3}\times 10} \approx 10m
	\end{align*}
\end{example}
\begin{example}
	Considere agora um sistema composto de três massas-molas separadas e todas na vertical presas ao chão. A primeira mola tem tamanho L; a segunda, l e, a terceira,
	tem um bloco à altura h. Supondo que a segunda mola recebe uma martelada e desloca-se com velocidade \(v_{0}\) para baixo, encontre a amplitude A e a altura h. Além disso,
	encontre o tempo necessário para que a segunda mola suba até a amplitude máxima. Para qual \(v_{0}\) a mola fica frouxa - isto é, não distenda-se na subida?

	Utilizaremos os conceitos de osciladores harmônicos. Sabe-se que a velocidade máxima do deslocamento que a mola causa é quando cosseno ou seno vale 1, ou seja,
	\[
		V_{max} = A\omega .
	\]
	Além disso, analisando o sistema dinâmico,
	\[
		k(L-l) = mg,
	\]
	em que k é o coeficiente da mola. Com isso,
	\[
		\frac{k}{m} = \frac{mg}{m(L-l)} \Rightarrow \frac{k}{m} = \frac{g}{L-l}.
	\]
	Disto segue que a velocidade da martelada será
	\[
		v_{0}=A\sqrt[]{\frac{k}{m}} = A\sqrt[]{\frac{g}{L-l}}
	\]
	e, portanto,
	\[
		A = v_{0}\sqrt[]{\frac{L-l}{g}}.
	\]
	Assim, a altura h será dada por
	\[
		h = l + A = l + v_{0}\sqrt[]{\frac{L-l}{g}}.
	\]
	Com relação ao tempo necessário para que a mola atinja a amplitude máxima, note que ela levará \(\Delta t = \frac{3}{4}T\). Isto pode ser reescrito como
	\[
		\Delta t = \frac{3}{4}T = \frac{3}{4}\frac{1}{f} = \frac{3}{4\frac{\omega }{2\pi }} = \frac{3\pi }{2\omega } = \frac{3\pi }{2\sqrt[]{\frac{g}{L-l}}} = \frac{3}{2}\pi \sqrt[]{\frac{L-l}{g}}
	\]
	Finalmente, sobre a velocidade, ela deverá satisfazer
	\[
		\frac{1}{2}mv_{0}^{2} = mg(L-l) + \frac{1}{2}k(L-l)^{2} = 0 \Rightarrow v_{0}^{2} = 2g(L-l) - \frac{k}{m}(L-l)^{2} = 2g(L-l)-g(L-l) = g(L-l).
	\]
\end{example}
\begin{example}
	Considere dois auto falantes distando d um do outro e, em uma distância D, coloque um estudante de graduação. Suponha que as ondas sonoras emitidas são esféricas e formam
	um ângulo \(\theta \) com a reta horizontal. A distância entre uma onda lançada pelo primeiro pra uma onda lançada pelo segundo é \(\Delta S = d\sin^{}{(\theta )}\) - a diferença de caminho entre
	as duas ondas.

	Para que o primeiro ângulo mínimo que faça o estudante ouvir o som ocorra, é preciso que \(\Delta S = \frac{\lambda }{2}\), tal que, como os ângulos são pequenos,
	\[
		\frac{\lambda }{2} = d\theta_{0} \Rightarrow \lambda.
	\]
	Usando que \(\lambda  = \frac{v_{som}}{f},\) encontra-se o valor de \(\theta \):
	\[
		\theta_{0} = \frac{\lambda }{2d} = \frac{v_{som}}{2df}.
	\]
	Sabendo o valor do ângulo mínimo, segue que o primeiro ângulo máximo será
	\[
		\theta_{max} = \frac{\lambda }{df}\approx \frac{y}{D}.
	\]
\end{example}
\begin{example}
	Suponha uma barra de tamanho L presa ao teto e, na metade dela, uma mola de constante elástica k é ligada, sendo ela presa horizontalmente a outra superfície.
	Qual é a velocidade angular desse sistema?

	A energia desse sistema é
	\[
		E = \frac{1}{2}mv^{2} + mg \frac{L}{2}\biggl(1-\cos^{}{(\theta )}\biggr)
	\]
	Com a forma que ele foi descrito, sendo \(\theta \) pequeno, vale que
	\[
		\frac{x}{\frac{L}{2}} = \theta,\quad \cos^{}{(\theta )} = 1 -\frac{\theta^{2}}{2},\quad x = \theta \frac{L}{2}.
	\]
	Assim, utilizando \(v = \frac{L}{2}\dot\theta \)
	\begin{align*}
		E & = \frac{1}{2}mv^{2} + mg \frac{L}{2}\biggl(1 - 1 + \frac{\theta ^{2}}{2}\biggr) + \frac{1}{2}k\theta^{2}\frac{L^{2}}{4} \\
		  & = \frac{1}{2}mv^{2} + \frac{mgL\theta ^{2}}{4} + \frac{1}{2}k \frac{L^{2}}{4}\theta^{2}                                 \\
		  & = \frac{1}{2}mv^{2} + \frac{1}{2}\biggl[\frac{mgL}{2} + \frac{kL^{2}}{4}\biggr]\theta^{2}                               \\
		  & = \frac{1}{2}m \frac{L^{2}}{4}\dot \theta ^{2} + \frac{1}{2}\biggl[\frac{mgL}{2}+ \frac{kL^{2}}{4}\biggr]\theta^{2}     \\
		  & = \frac{1}{2}m'v^{2} + \frac{1}{2}k'x^{2}.
	\end{align*}
	Finalmente, como \(\omega^{2} = \frac{k'}{m'}\),
	\[
		\omega^{2} = \biggl[\frac{mgL}{2}+ \frac{kL^{2}}{4}\biggr]\frac{1}{m \frac{L^{2}}{4}} = \frac{2g}{L} + \frac{k}{m}.
	\]
\end{example}
\end{document}

\documentclass[PhysicsII/physicsII_notes.tex]{subfiles}
\begin{document}
\section{Aula 17 - 28/09/2023}
\subsection{Motivações}
\begin{itemize}
	\item Continuação de Energia;
	\item Massa-Mola vertical;
	\item Pêndulo Simples.
\end{itemize}
\subsection{Continuando Energias}
Utilizando o que vimos na última aula, também podemos encontrar o valor médio da Energia Cinética:
\[
	\left< \mathbb{K} \right> = \frac{1}{T}\int_{t}^{t+T}\frac{1}{2}KA^{2}\sin^{2}{(\omega t'+\delta )}dt'
\]
Como \(\left< \mathbb{K} \right> = \left< U \right>,\) já que a variação de seno e a de cosseno é a mesma, segue que \(\left< \mathbb{K} \right> = \frac{E_{mec}}{2}\).
\begin{example}
	Considere um sistema massa mola em que há uma massa \(m=3kg, A = 4cm\) e o período é de \(T=2s\). Encontre a energia mecânica, a velocidade máxima e a posição na qual
	a velocidade do corpo é metade da velocidade máxima.

	Começamos observando a relação que nos fornece o valor da constante de mola k:
	\[
		\omega ^{2} = \frac{k}{m} \Rightarrow \biggl(\frac{2\pi }{T}\biggr)^{2} = \frac{k}{m} \Rightarrow k = m \biggl(\frac{2\pi }{T}\biggr)^{2}
	\]
	a partir disso, a energia mecânica sai como
	\[
		E_{mec} = \frac{1}{2}kA^{2} = \frac{1}{2}m \biggl(\frac{2\pi }{T}\biggr)^{2}A^{2}\approx 2.4 10^{-2}J
	\]
	A velocidade máxima de um sistema isolado é atingido quando a energia mecânica do sistema é composta apenas por cinética, ou seja,
	\[
		\frac{1}{2}mv_{max}^{2} = E_{mec}\approx 0.13\frac{m}{s}.
	\]
	Finalmente, para encontrar o x desejado, teremos
	\[
		E_{mec} = \frac{1}{2}mv^{2} + \frac{1}{2}kx^{2} = \frac{1}{2}kA^{2},
	\]
	do que segue que
	\begin{align*}
		            & \frac{1}{2}m \frac{v_{max}^{2}}{4} + \frac{1}{2}kx^{2} = \frac{1}{2} kA^{2} = \frac{1}{2}mv_{max}^{2}        \\
		            & \frac{1}{2}kx^{2}=\frac{1}{2}mv_{max}^{2}-\frac{1}{2}m \frac{v_{max}^{2}}{4}                                 \\
		\Rightarrow & \frac{1}{2}kx^{2} = \frac{1}{2}mv_{max}^{2}\frac{3}{4} \Rightarrow x^{2} = \frac{3}{4}\frac{m}{k}v_{max}^{2} \\
		            & x\approx\pm 3.5cm
	\end{align*}
\end{example}
\subsection{Sistemas Massa Mola Vertical}

Considere uma mola presa a um teto e atrelada a um bloco de massa m na sua outra ponta. Ao soltá-la, o bloco
se desloca uma quantidade \(y_{0}\) por conta da gravidade e puxamos este bloco por uma quantidade \(y'\), totalizando
uma distensão da mola em \(y = y_{0}+y'.\) A equação deste sistema, no equilíbrio, é dada por
\[
	m \frac{d^{2}y}{dt^{2}} = -ky - mg = -ky_{0}-ky'+mg = - ky',
\]
ou seja, temos uma EDO da forma
\[
	m \frac{d^{2}y}{dt^{2}} = -ky'.
\]
Porém, note que, de \(y=y_{0} + y',\) temos
\[
	\frac{dy}{dt} = c + \frac{dy'}{dt} \Rightarrow \frac{d^{2}y}{dt^{2}} = \frac{d^{2}y'}{dt^{2}}.
\]
Logo, nossa equação é a mesma que a do oscilador harmônico:
\[
	\frac{d^{2}y'}{dt^{2}} = -\frac{k}{m}y'.
\]
\subsection{Pêndulo Simples}
Faça uma corda de tamanho L, presa ao teto e com uma massa na ponta, oscilar após puxar essa massa em um ângulo \(\theta \).
Caso \(s\) denote o deslocamento com relação ao ponto de equilíbrio, então segue que
\[
	m \frac{d^{2}s}{dt^{2}} = -mg\sin^{}{(\theta )}.
\]
Como \(s = L\theta \), vale que
\[
	\frac{d^{2}\theta }{dt^{2}}=-\frac{g}{L}\sin^{}{(\theta )}.
\]
Para pequenas oscilações (\(\theta < 15^{\circ})\), nas quais \(\sin^{}{(\theta )}\approx \theta \), obtemos
\[
	\frac{d^{2}\theta }{dt^{2}} = -\frac{g}{L}\theta .
\]
\end{document}

\documentclass[PhysicsII/physicsII_notes.tex]{subfiles}
\begin{document}
\section{Aula 05 - 21/08/2023}
\subsection{Motivações}
\begin{itemize}
	\item Continuação do exemplo e outros;
	\item Potência;
	\item Corpos que rolam sem deslizar.
\end{itemize}
\subsection{Continuando o Exemplo}
\begin{example}[continuando...]
	Segue a relação de tração
	\[
		TR = I \frac{a}{R} \Rightarrow T = \frac{I}{R^{2}}a.
	\]
	Com isso,
	\[
		mg = ma + T = ma + \frac{I}{R^{2}}a = a[1 + \frac{I}{mR^{2}} \Rightarrow a = \frac{g}{1 + \frac{I}{mR^{2}}}.
	\]
	Descobrimos, assim, os valores de T e de \(F_{s}\)
	\begin{align*}
		 & T = \frac{I}{R^{2}}\frac{g}{1+ \frac{I}{mR^{2}}}          \\
		 & F_{s} = Mg + \frac{I}{R^{2}}\frac{g}{1+\frac{I}{mR^{2}}}.
	\end{align*}
\end{example}
\begin{example}
	Considere a máquina de Atwood - dois blocos presos a uma roldana, um de massa \(m_{1}\) e outro de massa \(m_{2}\) tais que \(m_{1} > m_{2}\).
	A roldana tem massa M, momento de inércia I e raio R. Vejamos as forças

	\textbf{Bloco 1:}
	No primeiro bloco, agem forças de tração \(T_{1}\) e peso \(m_{1}g\). Escrevendo as equações,
	\begin{align*}
		m_{1}g - T_{1} = m_{1}a
	\end{align*}

	\textbf{Bloco 2:}
	No bloco dois, agem a tração \(T_{2}\) e o peso \(m_{2}g\)
	\begin{align*}
		T_{2}-m_{2}g = m_{2}a
	\end{align*}

	\textbf{Roldana:}
	Tem-se a equação
	\begin{align*}
		(T_{1} - T_{2})R = I\alpha. \Longleftrightarrow T_{1} - T_{2} = \frac{I}{R}\frac{a}{R} = \frac{Ia}{R^{2}}
	\end{align*}

	Como a roldana está rodando, tem-se a relação \(m_{1g} > T_{1} > T_{2} > m_{2}g\)
	A seguir, soma-se a equação do bloco 2 com a da Roldana, tal que
	\[
		m_{1}g - m_{2}g - (T_{1}-T_{2}) = (m_{1}+m_{2})a \Longleftrightarrow (m_{1}-m_{2})g - \frac{Ia}{R^{2}} = (m_{1}+m_{2})a
	\]
	Isola-se a equação no a:
	\[
		a = \frac{(m_{1}-m_{2})g}{(m_{1}+m_{2})+\frac{I}{R^{2}}}
	\]
\end{example}
\begin{example}
	Considere uma roldana com massa M, momento de inércia I e raio R presa à quina uma mesa. Atrela-se a ela dois corpos, um com massa \(m_{1}\) e que está em cima da mesa
	e outro, de massa \(m_{2}\), suspenso pela corda.
	\textbf{Corpo 1:}
	As forças atuando no bloco 1 são a normal \(F_{N_{1}}\), a peso \(m_{1}g\) e a tração \(T_{1}\), de forma que
	\[
		T_{1}=m_{1}a
	\]

	\textbf{Corpo 2:}
	Para o bloco 2, podemos descrever o sistema considerando a tração \(T_{2}\) e o peso \(m_{2}g\), tal que
	\[
		m_{2}g - T_{2} = m_{2}a.
	\]

	\textbf{Roldana:}
	As forças que atuam na roldana são a tração na direção do bloco 1, \(T_{1}\), a outra na do bloco 2, \(T_{2}\), o peso
	\(Mg\) e uma força da quinta nela \(\vec{F_{s}}\). Além disso, \(a=R\alpha \). A equação do sistema será
	\begin{align*}
		(T_{2}-T_{1})R = I\alpha \Rightarrow  T_{2} - T_{1} = \frac{Ia}{R^{2}}.
	\end{align*}

	Somando a equação do bloco 1 e a do bloco 2, chega-se em
	\[
		m_{2}g - (T_{2}-T_{1}) = (m_{1}+m_{2})a
	\]
	Assim,
	\begin{align*}
		 & T_{2} - T_{1} = \frac{Ia}{R^{2}}                 \\
		 & m_{2}g - \frac{Ia}{R^{2}} = (m_{1}+m_{2})a       \\
		 & a = \frac{m_{2}g}{(m_{1}+m_2) + \frac{I}{R^{2}}}
	\end{align*}
	Além disso,
	\[
		T_{1} = \frac{m_{1}m_{2}g}{(m_{1}+m_{2})+\frac{I}{R^{2}}}.
	\]
\end{example}
\subsection{Potência}
Previamente, a potência era dada pela relação \(dW = F ds.\) Considerando o caso de uma força agindo
emu ma situação circular, isso torna-se \(dW = FRd\theta \). No entanto, esse termo à direita lembra muito um torque. De fato,
a relação que obtemos é que \(dW = \tau d\theta \). Portanto,
\[
	\hypertarget{power_torque}{\boxed{P = \frac{dW}{dt} = \tau \frac{d\theta }{dt} = \tau \omega }}
\]
\begin{example}
	Um motor de combustão de um carro fornece um torque de \(\tau  = 678Nm\) e está rodando a \(\omega = 4500rpm \approx 471 \frac{rad}{s}\).
	Com isso, a potência será
	\[
		P\approx 315kW.
	\]
\end{example}
\begin{example}
	Tome uma roda gigante \textbf{(em Londres).} Ela tem um diâmetro de \(135m\), pesa 1600 toneladas e dá 2 revoluções por hora.
	Qual é o torque necessário para parar a roda em 10m?

	Para começar, observe que \(W = \tau \Delta \theta \) e que \(S = R\Delta \theta = 10m.\) Em particular, temos o valor de R, tal que
	\[
		S = R\Delta \theta \Longleftrightarrow 10 = 67.5\Delta \theta \Rightarrow \Delta \theta \approx 0,148rad.
	\]
	Note que
	\[
		W = -(\mathbb{K}_{f} - \mathbb{K}_{i}) \Longleftrightarrow \tau \Delta \theta = -\biggl[0 - \frac{I\omega^{2}}{2}\biggr].
	\]
	Logo, convertendo \(\omega \) para radianos por segundo (\(\omega = 3,5 \cdot 10^{-3}\frac{rad}{s}\),
	\[
		\tau = \frac{I\omega^{2}}{2\Delta \theta } = \frac{MR^{2}\omega^{2}}{2\Delta \theta } \approx 3 \cdot 10^{5}Nm.
	\]
	Em particular,
	\[
		F = \frac{\tau }{R}\approx 4,4 \cdot 10^{3}N.
	\]
\end{example}
\subsection{Corpos que Rolam sem Deslizar}
Imagine um sistema em que um disco de raio R está a rolar com velocidade do centro de massa \(\vec{v}_{cm}.\) Considerando o ponto que tangencia o chão, em que a velocidade é nula,
ele se mexe com velocidade angular \(\omega \) em um raio \(\vec{r}\). Assim, a energia cinética desse sistema será
\[
	\mathbb{K}_{T} = \frac{1}{2} Mv_{cm}^{2} + \mathbb{K}_{rel} = \frac{1}{2}Mv_{cm}^{2} + \frac{1}{2}I_{cm}\omega ^{2},
\]
em que considera-se que \(v_{cm} = R\omega.\)

Agora, considere um plano inclinado e uma bola de massa m, momento de inércia I e raio R que irá subir este plano inclinado até parar. Como podemos achar a altura que ela para,
fornecida velocidade inicial do centro de massa \(v_{cm}\). Utilizando a conservação da energia mecânica,
\[
	E_{mec_{i}} = E_{mec_{f}}.
\]
Sabemos que
\[
	E_{mec_{i}} = \frac{1}{2}mv_{cm}^{2} + \frac{I_{cm}}{2}\omega ^{2}\quad \& E_{mec_{f}} = mgh.
\]
Assim,
\begin{align*}
	 & mgh = \frac{1}{2} mv_{cm}^{2} + \frac{I_{cm}}{2}\omega^{2}                                      \\
	 & \Rightarrow h = \frac{1}{2}\biggl[v_{cm}^{2} + \frac{I_{cm}}{2m}\frac{v_{cm}^{2}}{R^{2}}\biggr] \\
	 & \Rightarrow h = \frac{v_{cm}^{2}}{2g}\biggl[1 + \frac{I_{cm}}{mR^{2}}\biggr].
\end{align*}
\begin{example}
	Para o caso da esfera, em que \(I = \frac{2}{5}mR^{2},\)
	\[
		h = \frac{v_{cm}^{2}}{2g}\biggl[1 + \frac{2}{5}\biggr] = \frac{7}{10}\frac{v_{cm}^{2}}{g}
	\]
\end{example}
\begin{example}
	Considere um cenário de sinuca em que um taco aplica uma força F. Se ela é aplicada acima do eixo de rotação, a bola roda para frente. Caso seja exatamente no eixo de rotação,
	ela apenas deslizará. Por fim, se for atingida abaixo do eixo de rotação, ela rodará ao contrário. Como fazer ela não rodar?

	Em qualquer um desses pontos, a força é \(F=ma.\) No caso em que ela roda para frente, ou seja, é atingida a uma distância d acima do eixo de rotação,
	temos \(\tau = Fd = I\alpha.\) Segue que, para que ela não rode,
	\begin{align*}
		 & Fd = I\alpha = \frac{Ia}{R}                                                 \\
		 & \Rightarrow d = \frac{I}{mR} = \frac{2}{5}\frac{mR^{2}}{mR} = \frac{2}{5}R.
	\end{align*}
\end{example}
\end{document}

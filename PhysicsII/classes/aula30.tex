\documentclass[PhysicsII/phsyicsII_notes.tex]{subfiles}
\begin{document}
\section{Aula 30 - 09/11/2023}
\subsection{Motivações}
\begin{itemize}
	\item Gases Monoatômicos e Diatômicos.
\end{itemize}
\subsection{Gases Monoatômicos e Diatômicos}
Para alguns tipos específicos de gases, somos capazes de determinar propriamente uma fórmula para estes coeficientes, e é o que veremos a seguir.
A priori, considere um gás monoatômico, no qual sabemos que há apenas 3 graus de liberdade, o que permite-nos afirmar que
\[
	E_{int} = \frac{3}{2}k_{B}TN = \frac{3}{2}nRT.
\]
Como vimos acima, \(c_{V} = \frac{dE_{int}}{dt},\) tal que
\[
	d_{V} = \frac{d}{dt}\biggl(\frac{3}{2}nRT\biggr) = \frac{3}{2}nR.
\]
Além disso, como \(c_{P} = c_{V} + nR,\) isso permite que escrevamos \(c_{P}\) em termos de constantes puramente da seguinte
forma
\[
	c_{P} = c_{V} + nR = \frac{3}{2}nR + nR = \frac{5}{2}nR \Rightarrow \frac{c_{P}}{n} = \frac{5}{2}R.
\]
Vejamos, agora, o caso de um gás diatômico alinhado ao eixo Z, que possui 5 graus de liberdade e, logo, a energia interna vale
\[
	E_{int} = \frac{5}{2}nRT.
\]
Antes de progredir, observe que a energia total de um gás diatômico pode ser descrita pela componente cinética somada à potencial, de modo que
\[
	E = \frac{1}{2}mv_{x}^{2} + \frac{1}{2}mv_{y}^{2} + \frac{1}{2}mv_{z}^{2} + \frac{1}{2}I_{x}\omega_{x}^{2} + \frac{1}{2}I_{y}\omega_{y}^{2}.
\]
Analogamente ao caso monoatômico, chegamos em
\begin{align*}
	c_{V} = \frac{5}{2}nR \Rightarrow & c_{P} = \frac{5}{2}nR + \frac{5}{2}nR \\
	\Rightarrow                       & \frac{c_{P}}{n} = \frac{7}{2}R.
\end{align*}
(\textbf{DISCLAIMER: TALVEZ ESSA PARTE ESTEJA ERRADA. SE ESTIVER MESMO, ALGUÉM ME CORRIJA, POR FAVOR}) Se ele não estivesse necessariamente alinhado ao eixo Z, ele ganharia dois graus de liberdade a mais,
de modo que
\[
	c_{v} = \frac{7}{2}nR \quad\&\quad c_{p} = \frac{9}{2}nR.
\]
Para fixar as ideias, vejamos dois exemplos:

\begin{center}
	\begin{table}[h!]
		\caption{Exemplo com gás diatômico e monoatômico}
		\centering
		\begin{tabular}{| c | c | c | c | c |}
			\hline
			gás       & \(c_{p}\)(J/molK) & \(c_{V}\)(J/molK) & \(c_{V}/R\) & \((c_{P}-c_{V})/R\) \\
			\hline
			He        & 20,79             & 12,52             & 1,51        & 0,99                \\
			\(N_{2}\) & 29,12             & 208               & 2,5         & 1                   \\
			\hline
		\end{tabular}
	\end{table}
\end{center}
\begin{example}
	Suponha um sistema que possui \(n=0,32\) mols de um gás monoatômico. Este sistema sofrerá processos térmicos em três tempos,
	descritas por três momentos. O primeiro é uma expansão isobárica, com pressão \(P = 2,4atm\) e ocorre durante os momentos A até B, sendo os volumes final \(4,4L\) e inicial \(2,2L\). O segundo,
	um processo isocórico no volume \(V = 4,4L\), durante os momentos B até C, sendo as pressões inicial e final dadas por \(2,4atm\) e \(1,2atm\), respectivamente.
	Por fim, o sistema passa por um processo isotérmico do momento C até o A, com os valores de pressão e volume iniciais de \(4,4L\) e \(1,2atm\), terminando
	em \(2,2L\) e \(2,4atm\). Encontre os valores de Q, E e W em cada trecho do processo.

	Comece pelo valor de \(R = 0,08206 \frac{J \cdot atm \cdot L}{K}\). Podemos determinar, através da Lei dos Gases Ideais,
	as temperaturas em cada trecho. Temos:
	\begin{itemize}
		\item[A)] \(T_{A} = \frac{P_{A}V_{A}}{nR}\approx 201K\);
		\item[B)] Como \(\frac{P_{B}V_{B}}{T_{B}} = \frac{P_{A}V_{A}}{T_{A}}\) e \(V_{B} = 2V_{A},\) temos \(T_{B} = 2T_{A} = 402K;\)
		\item[C)] Finalmente, \(T_{C} = T_{A}.\)
	\end{itemize}
	Agora, analisemos cada processo individualmente.

	\textbf{\underline{Processo AB}:} Temos os seguintes valores para os três itens pedidos
	\begin{itemize}
		\item[\(W_{AB}\):] Segue que \(W_{AB} = -P_{A}(V_{f}-V_{i}) = -P_{A}(V_{B}-V_{A}) = -534,9J\);
		\item[\(Q_{AB}\):] Temos \(Q_{AB} = c_{P}\Delta T = \frac{5}{2}nR(T_{B}-T_{A}) = 1337J\);
		\item[\(E_{AB}\):] A variação da energia interna é dada por \(\Delta E_{int} = Q_{AB} + W_{AB} = 0,8kJ.\) Em particular,
		      o sistema ganhou temperatura.
	\end{itemize}

	\textbf{\underline{Processo BC}:} Aqui, o processo é isocórico, o que simplifica as coisas, já que
	\begin{itemize}
		\item[\(W_{BC}\):] Num processo isocórico, \(W_{BC} = 0\);
		\item[\(Q_{BC}\):] Vale que \(Q_{BC} = c_{V}\Delta T = \frac{3}{2}nR(T_{C} - T_{B}) = - 0,8kJ\);
		\item[\(E_{BC}\):] Como o trabalho é nulo, a variação de energia interna e o calor são os mesmos.
	\end{itemize}

	\textbf{\underline{Processo CA}:} Neste caso, como a transformação é isotérmica, temos:
	\begin{itemize}
		\item[\(E_{CA}\):] Num processo isotérmico, \(\Delta E_{int} = 0\);
		\item[\(W_{CA}\):] O trabalho de aqui é dado por \(W_{CA} = nRT_{A}\ln^{}{\biggl(\frac{V_{C}}{V_{A}}\biggr)} = 0,37kJ\);
		\item[\(Q_{CA}\):] Note que \(\Delta E_{int} = 0\) implica que \(Q_{CA} = - W_{CA} = -0,37kJ\).
	\end{itemize}

	\textbf{\underline{Processo Total}:} Para o processo total, comece observando que a transformação
	satisfaz \(\Delta E_{int} = 0\). Além disso, o trabalho total é a soma de cada trabalho, tal que
	\[
		W_{\text{total}} = W_{AB} + W_{BC} + W_{CA} = -0,13kJ.
	\]
	Logo, em particular,
	\[
		Q_{\text{total}} = - W_{\text{total}} = 0,13kJ
	\]
\end{example}
\subsection{Processos Adiabáticos}
Quando o diferencial da quantidade de calor é nulo, dizemos que o processo é \textbf{adiabático} - não há transferência
de calor nem para fora e nem para dentro do sistema. Em particular, isto significa que a energia interna do sistema é totalmente
governada pelo trabalho realizado. Com efeito,
\[
	dE_{int} = dQ + dW = 0 + dW = dW.
\]
Podemos elaborar nesta fórmula ainda mais:
\begin{align*}
	dE_{int} = dW \Rightarrow c_{V}dT & = - PdV                                              \\
	\Rightarrow                       & c_{V}dT = -\frac{nRT}{V}dV
	\Rightarrow                       & c_{V}\frac{dT}{T} + \frac{nR}{c_{V}}\frac{dV}{V} = 0
\end{align*}
Podemos integrar essa expressão nos dois lados e obtemos
\begin{align*}
	 & \ln^{}{(T)} + \frac{nR}{c_{V}\ln^{}{(V)}} = \mathrm{cte} \\
	 & \ln^{}{(TV)^{\frac{nR}{c_{V}}}} = \mathrm{cte}           \\
	 & TV^{\frac{nR}{c_{V}}} = e^{cte} = \mathrm{cte}.
\end{align*}
Colocando \(\gamma  = \frac{c_{P}}{c_{V}},\) note que, como \(c_{P} = nR + c_{V},\) vale
\[
	\frac{c_{P}-c_{V}}{c_{V}} = \frac{nR}{c_{V}}.
\]
Com a introdução de \(\gamma \), podemos escrever isso na forma
\[
	\frac{nR}{c_{V}} = \gamma - 1.
\]
Juntando isto ao raciocínio feito antes, segue que
\begin{align*}
	 & T_{i}V_{i}^{\gamma - 1} = T_{f}V_{f}^{\gamma - 1}                                              \\
	 & \frac{PV}{nR}V^{\gamma - 1} = PV^{\gamma } = \mathrm{cte}                                      \\
	 & \frac{T(nRT)^{\gamma - 1}}{P^{\gamma - 1}} = \frac{T^{\gamma }}{P^{\gamma - 1}} = \mathrm{cte}
\end{align*}

\end{document}

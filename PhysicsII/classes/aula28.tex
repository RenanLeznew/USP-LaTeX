\documentclass[phsyicsII_notes.tex]{subfiles}
\begin{document}
\section{Aula 28 - 06/11/2023}
\subsection{Motivações}
\begin{itemize}
	\item Calorimetria;
	\item Calor Específico e Calor Latente.
\end{itemize}
\subsection{Calor}
Podemos definir \textbf{calor} como \textbf{transferência de energia por conta da diferença de temperatura}. Sabemos algumas coisas
a priori já, como por exemplo que a diferença de temperatura está relacionada à energia cinética dos átomos. Previamente,
para explicar essa noção, utilizaram uma entidade imaginária chamada \textit{calórico}, que era transferida do corpo de maior temperatura
para o de menor temperatura.

Vamos supor que temos dois blocos A e B em contato e que o corpo A tem temperatura \(T_{A} \) maior que a do corpo B, \(T_{B}\).
Chamemos a quantidade de calor de Q, tal que, pelo raciocínio que fizemos,
\[
	Q = \Delta E_{int}.
\]
Como ela depende da variação da temperatura, podemos colocar, em geral,
\[
	Q = C\Delta T,
\]
em que o coeficiente c é chamado \textbf{capacidade térmica.} Esse coeficiente depende de duas coisas - da massa do objeto (quanto mais massa, maior a capacidade térmica)
e de outro coeficiente, conhecido como \textbf{calor específico} - de forma que
\[
	C = mc,\quad \&\quad Q = mc\Delta T.
\]
A unidade de calor é a caloria, que pode ser convertida em Joule através de \([Q] = 1\text{caloria} = 4,184J.\) Outras utilizadas são, por exemplo,
o BTU (\textbf{B}ritish \textbf{T}hermal \textbf{U}nit), padronizada como o calor necessário para aquecer uma libra de água em um Fahrenheit, e pode ser convertida por
\(1\text{BTU} = 252cal = 1,054kJ.\) Quando lidamos com elementos, podemos também definir o chamado \textbf{calor específico molar}, dado por
\[
	c'= \frac{c}{n}.
\]
Alguns valores de alguns elementos seguem na tabela
\begin{center}
	\begin{table}[h!]
		\caption{Calor Específico Molar de Alguns Elementos}
		\centering
		\begin{tabular}{| c | c |}
			\hline
			Elemento   & c'(J/molK) \\
			\hline
			Al         & 24.3       \\
			Hg         & 28.3       \\
			Zn         & 25.2       \\
			\(H_{2}O\) & 75.2       \\
			\hline
		\end{tabular}
	\end{table}
\end{center}
\begin{example}
	Considere uma massa de ouro de \(3\text{kg}\), inicialmente a uma temperatura de \(22^{\circ{}}C\) e que termina com
	uma temperatura de \(1063^{\circ{}}C\). Sabendo que \(c_{Au}=0,127\frac{\text{kJ}}{\text{KgK}},\) encontre a quantidade de calor para essa
	variação de temperatura.

	Temos
	\[
		Q = mc\Delta T = 3\times 0,126\times 1041\approx 333kJ.
	\]
\end{example}
\subsection{Calorimetria}
Para podermos medir o calor dos objetos, a área responsável por isso é a \textbf{calorimetria}. A ferramenta utilizada chama-se calorímetro.
Ao estudar isso, assumiremos que a soma dos calores é 0 e, inicialmente., que não há mudança de fase

Consideremos um recipiente contendo água, com massa d'água \(m_{H_{2}O},\) temperatura inicial \(T_{i}\), um calorímetro
de massa \(m_{c}\) e temperatura inicial igual à da água. Colocamos um bloco de massa, coeficiente de calor específico e temperatura inicial \(m_{0}, c_{0}, T_{i_{0}}\),
respectivamente. Assim, utilizando o que foi dito sobre a soma dos calores,
\[
	Q_{0} + Q_{H_{2}O} + Q_{c} = 0.
\]
Para cada um desses, vale
\[
	Q = mc\Delta T = mc(T_{f}-T_{i}),
\]
ou seja, podemos reescrever a igualdade como
\[
	m_{0}c_{0}(T_{F} - T_{i_{0}}) + m_{H_{2}O}c_{H_{2}O}(T_{f}-T_{i}) + m_{c}c_{c}(T_{f}-T_{i}) = 0
\]
Agora, por exemplo, caso quiséssemos descobrir o valor do coeficiente de calor específico do objeto, poderíamos resolver para ele, obtendo
\[
	c_{0} = \frac{1}{m_{0}(T_{f}-T_{i_{0}})}\biggl[m_{H_{2}O}c_{H_{2}O}(-T_{f}+T_{i}) + m_{c}c_{c}(-T_{f}+T_{i})\biggr]
\]
\begin{example}
	Considere uma massa de chumbo (Pb) de \(m_{0} = 600g\), inicialmente com uma temperatura de \(T_{i_{0}} = 100^{\circ{}}C\). Utilizando um calorímetro de
	\(200g\) feito de alumínio, cujo coeficiente de calor específico vale \(0.9\frac{\text{kJ}}{\text{kgK}}\) mergulha-se a massa de chumbo em um recipiente que contém uma massa d'água de valor \(m_{H_{2}O}=500g\) e
	que está em uma temperatura inicial de \(17.3^{\circ{}}C.\) Após as trocas térmicas, a temperatura final d'água é de \(20^{\circ{}C}\). Qual é o coeficiente
	de calor específico do chumbo?

	Começamos notando alguns valores:
	\begin{align*}
		 & (T_{f}-T_{i_{0}}) = -80^{\circ{}}C \\
		 & (T_{f}-T_{i}) = 2.7^{\circ{}}C.
	\end{align*}
	Com isso, utilizamos a fórmula encontrada e obtemos
	\[
		c_{0} = \frac{-2.7}{600\times(-80)}\biggl[500\times 4.18 + 200\times 09\biggr]\approx 0.13 \frac{\text{kJ}}{\text{kgK}}.
	\]
\end{example}
\subsection{Calor Latente}
O \textbf{calor latente} é a \textbf{quantidade de calor necessária para quebrar as ligações moleculares} de um dado elemento - em outras palavras,
o calor necessário para a mudança de fase. Nessa lógica, temos, por exemplo, \textbf{calor latente de fusão}, \textbf{calor latente de vaporização},
\textbf{calor latente de condensação}, etc. A quantidade de calor latente depende apenas da massa e do coeficiente, isto é,
\[
	Q = mL,\quad\&\quad Q_{\text{fusão}} = mL_{\text{fusão}},\quad Q_{\text{vaporização}} = m L_{\text{vaporização}}.
\]
\begin{example}
	Dado uma massa de gelo de \(m = 1.5kg\) a \(-20^{\circ{}}C\). Quanto de calor precisamos fornecer para evaporar essa massa de gelo?

	Nesse esquema, os tipos de calor que lidaremos são: Calor para esquentar o gelo de \(-20^{\circ{}}C\) a \(0^{\circ{}}\), calor para fundir o gelo, calor para esquentar a água até chegar na temperatura
	de vaporização e calor para vaporizar. Denotaremos eles por \(Q_{1}, Q_{2}, Q_{3} e Q_{4}\)
	respectivamente. Assim,
	\begin{align*}
		 & Q_{1} = mc\Delta T = 1.5\times 2.05\times 20\approx 61.5kJ \\
		 & Q_{2} = mL_{F} = 1.5\times 333.5\approx 500kJ              \\
		 & Q_{3} = mc\Delta T = 1.5\times 2.05\times 100\approx 657kJ \\
		 & Q_{4} = mL_{F} = 1.5\times 2.26 = 3.39MJ.
	\end{align*}
	Somando todos eles, obtemos o calor total necessário para todas as mudanças de estado:
	\[
		Q_{T} = Q_{1} + Q_{2} + Q_{3} + Q_{4}\approx 4.4MJ.
	\]
\end{example}
\begin{example}[Marcassa Disse que Gosta]
	Considere um copo de limonada, com uma massa de \(0.24kg\) e a uma temperatura de \(33^{\circ{}}C.\)
	Pegamos dois cubos de gelo, cada um de massa \(m_{\text{gelo}} = 0.025kg\) de temperatura \(T = 0^{\circ{}}C.\) Ao colocá-los na limonada,
	quanto valerá a temperatura final da limonada?

	Primeiramente, o gelo irá derreter ao entrar em contato com a limonada. Assim, o primeiro passo será encontrar
	o calor latente do gelo, ou seja,
	\[
		m_{\text{gelo}}L.
	\]
	Após isso, a massa de gelo e da limonada têm que esquentar até a temperatura final, isto é,
	\[
		m_{\text{gelo}}c_{H_{2}O}(T_{f} - T_{i_{\text{gelo}}}) + m_{\text{limonada}}c_{H_{2}O}(T_{f}-T_{i_{\text{limonada}}}).
	\]
	Somando tudo, devemos obter 0, tal que
	\[
		m_{\text{gelo}}L + m_{\text{gelo}}c_{H_{2}O}(T_{f} - T_{i_{\text{gelo}}}) + m_{\text{limonada}}c_{H_{2}O}(T_{f}-T_{i_{\text{limonada}}}) = 0
	\]
	Com alguns dos valores dados, simplifica-se para
	\[
		m_{\text{gelo}}L + m_{\text{gelo}}c_{H_{2}O}T_{f} + m_{\text{limonada}}c_{H_{2}O}T_{f} - m_{\text{limonada}}c_{H_{2}O}T_{i_{\text{limonada}}} = 0
	\]
	Assim,
	\[
		(m_{\text{gelo}}+m_{\text{limonada}})c_{H_{2}O}T_{f} = m_{\text{limonada}}c_{H_{2}O}T_{i_{\text{limonada}}}c_{H_{2}O}T_{i_{\text{limonada}}} - m_{\text{gelo}}L.
	\]
	Resolvendo para temperatura final,
	\[
		T_{f} = \frac{m_{\text{limonada}}c_{H_{2}O}T_{i_{\text{limonada}}} - m_{\text{gelo}}L}{(m_{\text{gelo}}+m_{\text{limonada}})c_{H_{2}O}}.
	\]
	Como adicionamos dois cubos de gelo, \(m_{\text{gelo}} = 0.05\), tal que \(T_{f}\approx 14^{\circ{}}C\). No entanto, é preciso tomar cuidado com algo neste exercício -
	se, no lugar de 2, adicionássemos 6 cubos de gelo, as contas dariam \(T_{f} = -10^{\circ{}}C\), o que é um absurdo, porque isso estaria menor do que
	a temperatura do próprio gelo. O ponto que causa essa falha é que, adicionando 6 cubos de gelo, \textbf{nem todos eles irão derreter}. Para
	descobrir a massa de gelo que não derrete, é preciso escrever
	\[
		m^{*}L = m_{\text{limonada}}c_{H_{2}O}\Delta T\quad \text{(Pode estar errado, estava difícil enxergar na lousa)}
	\]
\end{example}
\end{document}

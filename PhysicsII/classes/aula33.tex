\documentclass[phsyicsII_notes.tex]{subfiles}
\begin{document}
\section{Aula 33 - 27/11/2023}
\subsection{Motivações}
\begin{itemize}
	\item Máquina de Carnot;
	\item Ciclo de Carnot.
\end{itemize}
\subsection{Máquina de Carnot}
Vimos previamente que a eficiência de uma máquina térmica é dada por
\[
	\varepsilon = 1 - \frac{Q_{f}}{Q_{q}}.
\]
Qual seria o valor máximo da eficiência de uma máquina? Para isso, foi inventada a ideia da \textbf{máquina de Carnot}, baseado no postulado de Carnot:
\begin{quote}
	``Nenhuma máquina operando entre dois reservatórios é mais eficiente que uma máquina reversível.''
\end{quote}
Entenda por \textit{máquina reversível} uma na qual todos os processos podem ser revertidos. Por exemplo,

\begin{center}
	\begin{table}[h!]
		\caption{Reversível vs Irreversível}
		\centering
		\begin{tabular}{| c | c | c |}
			\hline
			Tipos        & \multicolumn{2}{c |}{Processos}                                                 \\
			\hline
			Reversível   & Expansão adiabática quase estática      & Compressão quase estática isotérmica  \\
			\hline
			Irreversível & Passagem de calor do quente para o frio & Transformar atrito em Energia Térmica \\
			\hline
		\end{tabular}
	\end{table}
\end{center}

Isto indica que há condições para que a máquina reversível exista. De fato, a energia mecânica não pode ser transformada
em energia térmica por atrito, a troca de calor deve ser feita entre corpos com diferença de temperatura infinitesimal e o processo
deve ser quase estático, para que o sistema esteja sempre em equilíbrio. Vejamos, em diagrama, o ciclo de Carnot.

\begin{center}
	\begin{tikzpicture}
		\def\Th{1.20}
		\def\Tc{0.45}
		\def\Ch{0.4}
		\def\Cc{1.9}
		\def\N{40}
		\def\gam{2.2}
		\def\isotherm#1#2{{ #2/(#1) }}
		\def\adiabatic#1#2{{ #2/(#1)^(\gam) }}
		\def\xA{ (\Th/\Ch)^(1/(1-\gam)) }
		\def\xB{ (\Th/\Cc)^(1/(1-\gam)) }
		\def\xC{ (\Tc/\Cc)^(1/(1-\gam)) }
		\def\xD{ (\Tc/\Ch)^(1/(1-\gam)) }
		\def\N{40} % number of plot samples
		\def\xmax{3}
		\def\ymax{2.5}
		\coordinate (A) at ({\xA},{\isotherm{\xA}{\Th}});
		\coordinate (B) at ({\xB},{\isotherm{\xB}{\Th}});
		\coordinate (C) at ({\xC},{\isotherm{\xC}{\Tc}});
		\coordinate (D) at ({\xD},{\isotherm{\xD}{\Tc}});

		%\clip (-0.1*\xmax,-0.12*\ymax) rectangle (1.05*\xmax,1.1*\ymax);

		% WORK
		\fill[mylightblue,samples=\N]
		plot[domain={\xA:\xB}] (\x,\isotherm{\x}{\Th}) --
		plot[domain={\xB:\xC}] (\x,\adiabatic{\x}{\Cc}) --
		plot[domain={\xC:\xD}] (\x,\isotherm{\x}{\Tc}) --
		plot[domain={\xD:\xA}] (\x,\adiabatic{\x}{\Ch});
		\node[blue,scale=.9] at ($(B)!.5!(D)$) {$W$};

		% ADIABATIC & ISOTHERMIC TRANSFORMATIONS
		\draw[myred,thick,midarr=.60,domain={\xA:\xB},samples=\N]
		plot (\x,\isotherm{\x}{\Th}); % hot
		\draw[blue,thick,midarr=.45,domain={\xB:\xC},samples=\N]
		plot (\x,\adiabatic{\x}{\Cc}); % cold
		\draw[blue,thick,midarr=.65,domain={\xC:\xD},samples=\N]
		plot (\x,\isotherm{\x}{\Tc}); % cold
		\draw[myred,thick,midarr=.40,domain={\xD:\xA},samples=\N]
		plot(\x,\adiabatic{\x}{\Ch}); % hot

		% POINTS
		\fill[mydarkblue]
		(A) circle(0.05) node[above=1,scale=.8] {I}
		(B) circle(0.05) node[above right,scale=.8] {II}
		(C) circle(0.05) node[above=1,scale=.8] {III}
		(D) circle(0.05) node[below left,scale=.8] {IV};

		% AXIS
		\draw[->,thick] (0,-0.1*\ymax) -- (0,\ymax+0.1)
		node[anchor=north east,inner sep=4,scale=1] {$P$};
		\draw[->,thick] (-0.1*\xmax,0) -- (\xmax+0.1,0)
		node[anchor=north east,inner sep=4,scale=1] {$V$};

	\end{tikzpicture}
\end{center}

O processo (I) é uma expansão quase estática isotérmica, o (II) é uma expansão quase estática adiabática,
o (III) uma compressão quase estática isotérmica e, o (IV), uma compressão quase estática adiabática.

Note que processos (I) e (III) implicam que \(\Delta E_{int} = 0.\) Como \(\Delta E_{int} = W + Q,\) em particular,
\(Q_{q} = W_{I}\) e \(Q_{f} = W_{III}.\) Além disso,
\begin{align*}
	W = - \int_{}^{}PdV & = \int_{V_{1}}^{V_{2}}PdV                                \\
	                    & = \int_{V_{1}}^{V_{2}}\frac{nRT}{V}dV                    \\
	                    & = nRT \ln^{}{\biggl(\frac{V_{2}}{V_{1}}\biggr)} = Q_{q}.
\end{align*}
Analogamente,
\begin{align*}
	W_{I} = nRT\ln^{}{\biggl(\frac{V_{3}}{V_{4}}\biggr)} = Q_{f}.
\end{align*}
Consequentemente,
\begin{align*}
	\varepsilon = 1 - \frac{Q_{f}}{Q_{q}} & = 1 - \frac{nRT_{f}\ln^{}{\biggl(\frac{V_{3}}{V_{4}}\biggr)}}{nRT_{q}\ln^{}{\biggl(\frac{V_{2}}{V_{1}}\biggr)}}      \\
	                                      & = 1 - \frac{\ln^{}{\biggl(\frac{V_{3}}{V_{4}}\biggr)}}{\ln^{}{\biggl(\frac{V_{2}}{V_{1}}\biggr)}}\frac{T_{f}}{T_{q}}
\end{align*}
Analisando os processos adiabáticos (II) e (IV), note que
\begin{itemize}
	\item[II)] \(T_{q}V_{2}^{\gamma -1} = T_{f}V_{3}^{\gamma -1}\)
	\item[IV)] \(T_{f}V_{4}^{\gamma -1} = T_{f}V_{4}^{\gamma -1}\)
\end{itemize}
Dividindo uma igualdade pela outra, segue que
\[
	\biggl(\frac{V_{2}}{V_{1}}\biggr)^{\gamma -1} = \biggl(\frac{V_{3}}{V_{4}}\biggr)^{\gamma -1} \Rightarrow \frac{V_{2}}{V_{1}} = \frac{V_{3}}{V_{4}}.
\]
Assim,
\[
	\varepsilon  = 1 - \frac{\ln^{}{\biggl(\frac{V_{3}}{V_{4}}\biggr)}}{\ln^{}{\biggl(\frac{V_{3}}{V_{4}}\biggr)}}\frac{T_{f}}{T_{q}} = 1 - \frac{T_{f}}{T_{q}}.
\]
\begin{example}
	Uma máquina a vapor trabalha entre um reservatório quente a \(T_{q}={100}^{\mathrm{o}}C (373K)\) e um
	reservatório frio a \(T_{f}={0}^{\mathrm{o}}C (273K).\) Para uma máquina deste tipo, a eficiência será
	\[
		\varepsilon  = 1 - \frac{273}{373}\approx 0.268 = 26,8\%
	\]
	Como refrigerador, usando \(CD = \frac{Q_{f}}{W} = \frac{Q_{f}}{Q_{q}-Q_{f}} = \frac{1}{\frac{T_{q}}{T_{f}}-1},\) segue que
	\[
		CD = \frac{1}{\frac{T_{q}}{T_{f}}-1} = 2,73.
	\]
\end{example}
\begin{example}
	Uma máquina térmica opera entre dois reservatório, retirando 200J do reservatório quente e entregando 48J de Trabalho. Ao fazer isso, libera, ao reservatório frio, 152J.
	Quanto desse trabalho foi feito por processos irreversíveis, como atrito?

	Sabemos que o trabalho máximo dessa máquina será
	\[
		\varepsilon_{max} = \frac{W_{max}}{Q_{q}} \Rightarrow W_{max} = \varepsilon_{max}Q_{q}
	\]
	e que o trabalho geral, analogamente, é
	\[
		\varepsilon_{\text{real}}  = \frac{W_{\text{real}}}{Q}.
	\]
	Com isso, o trabalho perdido será \(W_{p} = W_{max} - W_{\text{realizado}}\), ou seja,
	\[
		W_{p} = \biggl(1 - \frac{T_{f}}{T_{q}}\biggr)Q_{q} - W_{\text{realizado}}
	\]
	Portanto,
	\[
		W_{p} = 0,268\times 200 - 48 = 5,6J.
	\]
\end{example}
Na máquina de Carnot, vimos que
\[
	\varepsilon  = 1 - \frac{T_{f}}{T_{q}}.
\]
Poderíamos definir uma \textit{temperatura termodinâmica}, fazendo uma escala na qual
\[
	\frac{T_{f}}{T_{q}} = \frac{Q_{f}}{Q_{q}}.
\]
Apesar disso, ela não é utilizada de maneira prática.
\end{document}

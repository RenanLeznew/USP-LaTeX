\documentclass[physicsII_notes.tex]{subfiles}
\begin{document}
\section{Aula 02 - 10/08/2023}
\subsection{Motivações}
\begin{itemize}
	\item Momento de Inércia
\end{itemize}
\subsection{Distribuição Contínua de Massa}
No caso de distribuições discretas de massa, vimos que o momento de inércia é dado por
\[
	I=\sum\limits_{i}^{}m_{i}r_{i}^{2}.
\]
No entanto, muitas situações do mundo precisam que tratemos a distribuição de massa como algo único, uma
quantidade contínua. Para isso, passamos de somar cada massa para uma integral com respeito a ela:
\[
	\hypertarget{momentum_of_inertia_continuous}{\boxed{I = \int_{}^{}r^{2}dm.}}
\]
Para o caso de uma barra, por exemplo, na qual a distribuição de massa é dada por
\[
	\lambda = \frac{M}{L},
\]
segue que \(dm = \lambda dx \Rightarrow dI = x^{2}dm = x^{2}\lambda dx\). Portanto,
\[
	I = \lambda \int_{}^{}x^{2}dx = \lambda \frac{x^{3}}{3}.
\]
Por exemplo, se o tamanho da barra é 1 e o eixo de rotação está em uma extremidade, o momento de inércia será
\[
	I = \lambda \int_{0}^{1}x^{2}dx =\frac{1}{3}ML^{2}.
\]
Há outros casos importantes que devem ser tratados. O primeiro deles é o eixo central,
no qual o eixo de rotação é posicionado na metade do tamanho da barra. Assim,
\[
	I = \lambda \int_{-\frac{1}{2}}^{\frac{1}{2}}x^{2}dx = \lambda \frac{x^{3}}{3}\biggl|_{-\frac{1}{2}}^{\frac{1}{2}}\biggr. = \lambda \frac{L^{3}}{12} = \frac{ML^{2}}{12}.
\]
O outro engloba a situação em que toda a massa na mesma distância. Neste caso, \(\lambda = \frac{M}{2\pi R}\)
\[
	I = R^{2} \int_{}^{}dm = MR^{2},
\]
que também pode ser obtido fazendo uma integral com respeito ao ângulo \(\theta \):
\[
	I = R^{2}\lambda \int_{0}^{2\pi } R d\theta = R^{2}\lambda R\times2\pi = MR^{2}.
\]
Por fim, é importante olhar o caso dos discos. Discos consistem de dois círculos, um maior e outro menor dentro dele.
Chamaremos de R o raio do maior e de r o do menor. Para eles, há uma distribuição de massa
\(\sigma = \frac{M}{\pi R^{2}}\), de maneira que o diferencial de massa será
\[
	dm = 2\pi r dr\sigma.
\]
Com isso, conseguimos encontrar que o momento de inércia é
\[
	I = \int_{0}^{R}2\pi \sigma r^{3}dr = 2\pi \sigma \frac{\pi^{4}}{4}\biggl|_{0}^{R}\biggr. = \frac{1}{2}MR^{2}
\]


\end{document}

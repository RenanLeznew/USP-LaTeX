\documentclass[PhysicsII/physicsII_notes.tex]{subfiles}
\begin{document}
\section{Aula 11 - 11/09/2023}
\subsection{Motivações}
\begin{itemize}
	\item Aula de Exercícios.
\end{itemize}
\subsection{Exercícios}
\subsubsection{Exercício 1}
Considere um plano inclinado com ângulo \(\theta \), sobre o qual há um objeto de massa M, raio R e momento de inércia I.
Este objeto está posicionado a uma altura h em relação ao chão. Qual é a velocidade do centro de massa após o objeto rolar sem deslizar pelo plano?

Começamos com o princípio da conservação de energia - \(E_{mec_{i}} = E_{_mec_{f}}\). Para esse sistema, a energia mecânica inicial
vale Mgh, enquanto que a final vale
\[
	E_{mec_{f}} = \frac{1}{2}Mv_{cm}^{2} + \frac{1}{2}I\omega^{2}.
\]
Colocando os valores na equivalência, obtemos
\[
	Mgh = \frac{1}{2}Mv_{cm}^{2} + \frac{1}{2}I\omega^{2}.
\]
Sabe-se que, quando o disco rola sem deslizar, \(v_{cm} = R\omega \). Com isso,
\begin{align*}
	            & 2Mgh = Mv_{cm}^{2} + I \frac{v_{cm}^{2}}{R^{2}}              \\
	\Rightarrow & 2gh = v_{cm}^{2}\biggl[1 + \frac{I}{MR^{2}}\biggr]           \\
	\Rightarrow & v_{cm}^{2} = \frac{2gh}{\biggl[1 + \frac{I}{MR^{2}}\biggr]}.
\end{align*}
Para o aro, \(I = MR^{2},\) ou seja,
\[
	v_{cm}^{2} = \frac{2gh}{2} \Rightarrow v_{cm}=\sqrt[]{gh}.
\]
Quando é um disco, \(I=\frac{1}{2}MR^{2}\), tal que
\[
	v_{cm}^{2} = \frac{2gh}{1 + \frac{1}{2}} = \frac{4gh}{3} \Rightarrow v_{cm} = \frac{2}{\sqrt[]{3}}\sqrt[]{gh}
\]
Assim, \(v_{cm_{disco}} > v_{cm_{aro}}\)
\[
	\frac{v_{cm_{disco}}}{v_{cm_{aro}}} = \frac{2}{\sqrt[]{3}}.
\]
Elaborando, sabemos que
\[
	E_{mec_{f}} = \mathbb{K}_{trans} + \mathbb{K}_{rot}.
\]
Essas duas quantidades relacionam-se com
\[
	\frac{\mathbb{K}_{trans}}{\mathbb{K}_{rot}} = \frac{1}{2}\frac{Mv_{cm}^{2}}{\frac{1}{2}I\omega^{2}} = \frac{Mv_{cm}^{2}}{I\omega^{2}}.
\]
Como \(v_{cm}=R\omega \),
\[
	\frac{\mathbb{K}_{tran}}{\mathbb{K}_{rot}} = \frac{Mv_{cm}^{2}}{I\omega^{2}}=\frac{Mv_{cm}^{2}}{I \frac{v_{cm}^{2}}{R^{2}}} = \frac{MR^{2}}{I}.
\]
Para o aro, isto torna-se
\[
	\frac{\mathbb{K}_{trans}}{\mathbb{K}_{rot}} = 1
\]
e, para o disco,
\[
	\frac{\mathbb{K}_{trans}}{\mathbb{K}_{rot}} = 2
\]
\subsubsection{Exercício 2}
Considere uma situação em que um disco de raio R tem um buraco, de raio \(\frac{R}{2}\), centrado no eixo x. Buscamos o momento de inércia do disco com relação ao eixo z.
Dado que o sistema tem massa M, começamos definindo a densidade superficial de massa
\[
	\sigma = \frac{M}{\pi \biggl(R^{2}-\frac{R^{2}}{4}\biggr)} = \frac{4M}{3\pi R^{2}}.
\]
Sabemos que
\[
	I_{z} = I_{z}^{+} + I_{z}^{-}
\]
e que
\[
	I_{z} = \frac{\pi R^{2}\sigma R^{2}}{2} - \frac{\pi R^{2}}{4}\frac{\sigma }{2}\frac{R^{2}}{4} - \frac{M'R^{2}}{4},
\]
em que \(M'\) é a massa ``removida'' do buraco. Assim,
\begin{align*}
	I_{z} & = \frac{\pi R^{4}\sigma }{2} - \frac{\pi R^{4}\sigma }{2}\frac{1}{16} - \frac{\pi R^{2}\sigma }{4}\frac{R^{2}}{4}    \\
	      & = \frac{\pi \sigma R^{4}}{2}\biggl[1-\frac{1}{16}-\frac{2}{16}\biggr] = \frac{2}{3}MR^{2}\biggl(\frac{13}{16}\biggr) \\
	      & = \frac{13}{24}MR^{2}.
\end{align*}
Usando que \(I_{z} = I_{x} + I_{y},\) podemos encontrar
\begin{align*}
	 & I_{x} = \frac{M^{+}R^{2}}{4} - \frac{M^{-}}{4}\frac{R^{2}}{4}                         \\
	 & I_{y} = \frac{M^{+}R^{2}}{4} - \frac{M^{-}}{4}\frac{R^{2}}{4} - M^{-}\frac{R^{2}}{4}.
\end{align*}
\subsubsection{Exercício 3}
Considere um ioiô - um objeto circular de peso Mg, raio R, aceleração vertical \(\vec{a},\) tração \(\vec{T}\) e momento de inércia I. Quanto vale a aceleração e a tração?

Sabe-se pelo diagrama de forças que \(TR = I\alpha \), já que a tração é a única força aqui que pode gerar torque, e
\[
	Mg - T = Ma,\quad a = \alpha R.
\]
Somando as duas coisas,
\[
	Mg = Ma + \frac{I\alpha }{R} \Rightarrow Mg = Ma + \frac{Ia}{R^{2}}.
\]
Com isso,
\[
	Mg = Ma \biggl[1 + \frac{I}{MR^{2}}\biggr] \Rightarrow a = \frac{g}{\biggl[1 + \frac{I}{MR^{2}}\biggr]}.
\]
Portanto, a tração vale
\[
	T = \frac{I\alpha }{R} = \frac{I}{R^{2}}\frac{g}{\biggl[1 + \frac{I}{MR^{2}}\biggr]}.
\]
\subsubsection{Exercício 4}
Novamente, considerando um ioiô, apoie-o sobre uma mesa. Dessa vez, o ioiô é composto de 3 cilindros curtos, tal que a parte interna desse disco tem um raio r e,
nesse disco de dentro, há uma corda aplicando uma força \(F\) (O ioiô de perfil, enrolado, parece um H). Qual é o ângulo que ele deve ser lançado para rolar
sem deslizar?

A pergunte equivale a achar o ângulo pelo qual a força deve passar sem que haja torque no ponto de apoio. Com isso, percebe-se que
\[
	\frac{r}{R} = \cos{(\theta )}.
\]
Portanto, \(\theta = \arccos{\biggl(\frac{r}{R}\biggr)}\)
\subsubsection{Exercício 5}
Considere um sistema de dois discos, ambos com massa m, mas o segundo com o raio duas vezes maior que o primeiro. Posicionando-os centrados no mesmo eixo vertical, mas
o menor acima do primeiro, considere que ele roda com velocidade \(\omega_{0}\) no sentido horário e que o maior roda, também com velocidade \(\omega_{0}\), no sentido anti-horário.
Postulando que o primeiro tem momento angular negativo e, o segundo, positivo, assume-se que os discos colidem. Qual será a frequência de rotação após a colisão?
E a razão entre \(\mathbb{K}_{inicial}\) e \(\mathbb{K}_{final}\)?

Pela conservação de momento angular, sabe-se que
\[
	L_{i} = L_{f}.
\]
Além disso,
\[
	L_{i} = L^{+}-L^{-} = I_{1}\omega_{0} - I_{2}\omega_{0} = \frac{1}{2}m(2r^{2})\omega_{0} - \frac{1}{2}mr^{2}\omega_{0} = \frac{3}{2}mr^{2}\omega_{0}
\]
Com relação ao momento angular final,
\begin{align*}
	L_{f} & = I_{f}\omega_{f} = (I_{2} + I_{1})\omega_{f} = \biggl(\frac{1}{2}m4r^{2}+\frac{1}{2}mr^{2}\biggr)\omega_{f} \\
	      & =\frac{5}{2}mr^{2}\omega_{f}.
\end{align*}
Portanto,
\[
	\frac{3}{2}mr^{2}\omega_{0} = \frac{5}{2}mr^{2}\omega_{f} \Rightarrow \omega_{f}=\frac{3}{5}\omega_{0}.
\]
Agora, vejamos a questão das energias cinéticas. A inicial vale
\[
	\mathbb{K}_{i} = \frac{1}{2}I_{1}\omega_{0}^{2} + \frac{1}{2}I_{2}\omega_{0}^{2} = \frac{1}{2}\omega_{0}^{2}\frac{5}{2}mr^{2}.
\]
A final, por outro lado, é dada por
\[
	\mathbb{K}_{f} = \frac{1}{2}(I_{1}+I_{2})\omega_{f}^{2} = \frac{1}{2}\frac{5}{2}mr^{2}\frac{9}{25}\omega_{0}^{2} = \frac{1}{4}\frac{9}{5}mr^{2}\omega_{0}^{2}.
\]
Dividindo um valor pelo outro,
\[
	\frac{\mathbb{K}_{i}}{\mathbb{K}_{f}} = \frac{5}{4}\frac{1}{\frac{9}{5}\frac{1}{4}} = \frac{25}{9}.
\]
\end{document}

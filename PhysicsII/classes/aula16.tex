\documentclass[physicsII_notes.tex]{subfiles}
\begin{document}
\section{Aula 16 - 27/09/2023}
\subsection{Motivações}
\begin{itemize}
	\item Oscilador Harmônico;
	\item Sistema massa-mola;
	\item Amplitude, frequência e fase.
\end{itemize}
\subsection{Oscilador Harmônico (OH)!}
Numa situação de movimento harmônico simples, a aceleração e a força resultante são
proporcionais e de sentidos opostos ao deslocamento a partir da posição de equilíbrio.
Em outras palavras, \(F = -kx\), ou seja,
\[
	m\frac{d^{2}x}{dt^{2}}=-kx \Rightarrow \frac{d^{2}x}{dt^{2}}=-\frac{k}{m}x,
\]
o que é uma EDO de segunda ordem da variável x. Existem formas e métodos de lidar com elas,
formando uma área inteira da matemática. Para esta equação em específico, as soluções possíveis são
\[
	x(t)=\sin^{}{\biggl(\sqrt[]{\frac{k}{m}}t\biggr)} \quad\&\quad x(t)=\cos^{}{\biggl(\sqrt[]{\frac{k}{m}}t\biggr)}
\]
Em geral, a função posição de um sistema de oscilador harmônico é descrita por \(x(t) = A\cos^{}{(\omega t + \delta )}.\)
Chama-se A de amplitude, \(\omega \) de frequência e \(\delta \) de fase. Derivando-a, obtemos velocidade
e aceleração:
\[
	v(t) = -A\omega \sin^{}{(\omega t + \delta )}\quad a(t) = -A\omega^{2}\cos^{}{(\omega t + \delta )}
\]
Utilizando a relação de antes para frequência, \(\omega = 2\pi f = \frac{2\pi }{T}\), podemos
isolar em função do período a fim de obter o período do oscilador harmônico:
\[
	T = 2\pi \sqrt[]{\frac{M}{k}}.
\]
Para conseguirmos informações sobre a fase do sistema, calculamos a posição no momento t = 0, tal que
\(x(0) = A\cos^{}{(\omega 0 + \delta )} = A\cos^{}{(\delta )}\), ou seja, \(\cos^{}{(\delta )}=\frac{x(0)}{A}\)
\begin{example}
	Dado que \(\omega = 8 rad/s\) e que, em \(t=0\), temos as informações
	\[
		\left.\begin{array}{ll}
			x=4cm \\
			v_{x} = 25 cm/s
		\end{array}\right\}
	\]
	encontre a amplitude do sistema de oscilações e a equação que descreve o sistema.

	Com efeito, como \(x = A \cos{(\omega t + \delta )}\), podemos utilizar este ponto de partida. Note que
	\[
		\tan{(\delta )} = \frac{-V_{0}}{\omega x_{0}} = \frac{-25}{32} \Rightarrow \delta \approx -0.66.
	\]
	Assim, \(A = \frac{x_{0}}{\cos{(\delta )}}\approx 5.1cm\). Portanto, \(x = 5.1\cos{(\omega t - 0.66)}.\)
\end{example}
\begin{example}
	Dado que um bloco está preso a uma mola, considere \(m=2kg\) a massa dele, \(k = 19.6\frac{N}{m}\) e que, inicialmente,
	ele está em \(x_{0} = 5cm\). Encontre a aceleração e velocidade máximas dele.

	Começamos observando que \(\omega^{2} = \frac{k}{m} = \frac{19.6}{2} = 98,\) ou seja, \(\omega \approx 9.9 \frac{rad}{s}\).
	A partir disso, encontramos a frequência e, consequentemente, o período
	\[
		f = \frac{\omega }{2\pi }\approx 1.58Hz \Rightarrow T\approx 0.63s.
	\]
	A equação que descreve esse oscilador é \(x(t) = 5\cos{(9.9t)}\), donde tiramos que a velocidade do sistema é descrita por
	\[
		v = -9.9\times 5\sin{(9.9t)}.
	\]
	Como a função seno é limitada por 1, fica fácil de encontrar o valor máximo da velocidade:
	\[
		|v_{max}| = 49.5 \frac{cm}{s}.
	\]
	Analogamente, a aceleração será descrita por
	\[
		a = -(9.9)^{2}\times 5\cos{(9.9t)}
	\]
	e o mesmo raciocínio mostra que
	\[
		|a_{max}| = (9.9)^{2}\times 5 \frac{cm}{s}.
	\]
\end{example}
\begin{example}
	Num sistema de movimento circular uniforme, as equações que descrevem o movimento da partícula no instante t são
	\[
		\left\{\begin{array}{ll}
			x(t) = R\cos{(\omega t + \delta )} \\
			y(t) = R\sin{(\omega t + \delta )}
		\end{array}\right.
	\]
	tal que o movimento vetorial dele será dado pela combinação de posição, velocidade e aceleração
	\[
		\left\{\begin{array}{ll}
			\vec{R}(t) = R\cos{(\omega t + \delta )}\hat{i} + R\sin{(\omega t + \delta )}\hat{j}                \\
			\vec{v}(t) = -R\omega \sin{(\omega t + \delta )}\hat{i} + \omega R\cos{(\omega t + \delta )}\hat{j} \\
			\vec{a}(t) = -R\omega ^{2}\cos{(\omega t + \delta )}\hat{i} - \omega ^{2}R\sin{(\omega t + \delta )}\hat{j}.
		\end{array}\right.
	\]
	Disto, faz sentido falar sobre componente de oscilação nos eixos x e y, como
	\[
		\left\{\begin{array}{ll}
			x = A_{x}\cos{(\omega_{x}t + \delta )} \\
			y = A_{y}\cos{(\omega_{y}t)}
		\end{array}\right.
	\]
\end{example}
\subsection{Energia Mecânica no OH}
Dado um sistema massa mola, sendo que a velocidade angular é \(\omega \), a amplitude da oscilação é A e a constante da mola é K, o sistema pode ser
descrito, no quesito posição, por
\[
	x(t) = A\cos{(\omega t + \delta )}
\]
Quanto às energias, temos
\[
	U = \frac{1}{2}kx^{2} = \frac{1}{2}A^{2}\cos^{2}{(\omega t + \delta )}\quad\&\quad \mathbb{K} = \frac{1}{2}mv^{2} = \frac{1}{2}A^{2}\omega ^{2}\sin^{2}{(\omega t+\delta )}
\]
Sendo que a soma delas é a energia mecânica do sistema:
\[
	E_{mec} = U + \mathbb{K} = \frac{1}{2}A^{2}\cos^{2}{(\omega t + \delta )} + \frac{1}{2}A^{2}\omega ^{2}\sin^{2}{(\omega t+\delta )}
\]
Como \(\omega^{2} = \frac{k}{m},\) vale
\[
	E_{mec} = \frac{1}{2}A^{2}k\sin^{2}{(\omega t + \delta )} + \frac{1}{2}kA^{2}\cos^{2}{(\omega t + \delta )} = \frac{1}{2}kA^{2}\biggl[\cos^{2}{(\omega t+\delta )} + \sin^{2}{(\omega t + \delta )}\biggr] = \frac{1}{2}kA^{2}
\]

\end{document}

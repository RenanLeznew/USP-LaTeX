\documentclass[PhysicsII/physicsII_notes.tex]{subfiles}
\begin{document}
\section{Aula 07 - 24/08/2023}
\subsection{Motivações}
\begin{itemize}
	\item A natureza vetorial do torque.
\end{itemize}
\subsection{Vetores}
Vamos começar com um exemplo de como a nossa construção atual falha em algumas descrições.
Quando um peão está rodando, ou um giroscópio, o torque não é apenas em um dimensão. Assim, não podemos usar nosso modelo atual.

A descrição do torque como grandeza vetorial é simplesmente \(\vec{\tau } = I \vec{\alpha } = I \frac{d \vec{\omega }}{dt}\). Disto, observa-se
que a direção dele é a mesma da velocidade angular \(\vec{\omega }\). Considerando uma situação tridimensional e colocarmos
\(\vec{r}\) e a força \(\vec{F}\), fazendo ângulo \(\varphi \) com a reta de \(\vec{r}\), no plano xy, o torque sairá na direção z, visto que ele é o produto vetorial da força com
\(\vec{r}\): \(\vec{\tau } = \vec{r}\times \vec{F}\). Em particular, segue que \(|\vec{\tau }| = rF\sin{(\varphi )}\)
Para operacionalizar melhor, vamos padronizar \(\hat{i}, \hat{j}, \hat{k}\) como os versores nas direções x, y e z respectivamente. Segue que
\begin{itemize}
	\item[a)] \(\vec{i}\times \vec{j} = \vec{k}\);
	\item[b)] \(\vec{j}\times \vec{k} = \vec{i}\);
	\item[c)] \(\vec{k}\times \vec{i} = \vec{j}\);
\end{itemize}
Além disso, vetores em três dimensões serão da forma \(\vec{A} = A_{x}\hat{i} + A_{y}\hat{j} + A_{z}\hat{k}\) e \(\vec{B} = B_{x}\hat{i} + B_{y}\hat{j} + B_{z}\hat{k}\).
Lembre-se que \(\vec{C} = \vec{A} \times \vec{B} = -\vec{B}\times \vec{A}\) e \(\vec{A} \times \vec{A} = 0.\) Assim,
\begin{align*}
	\vec{A} \times \vec{B} & = (A_{x}\hat{i} + A_{y}\hat{j} + A_{z}\hat{k})\times(B_{x}\hat{i} + B_{y}\hat{j} + B_{z}\hat{k})                                            \\
	                       & = \hat{i}\biggl[A_{y}B_{z} - A_{z}B_{y}\biggr] + \hat{j}\biggl[-A_{x}B_{z} + A_{z}B_{x}\biggr] + \hat{k}\biggl[A_{x}B_{y}-A_{y}B_{x}\biggr] \\
	                       & = \biggl[A_{y}B_{z} - A_{z}B_{y}\biggr]\hat{i} + \biggl[A_{z}B_{x} - A_{x}B_{z}\biggr] + \biggl[A_{x}B_{y}-A_{y}B_{x}\biggr]\hat{k}.
\end{align*}
Outra propriedade é com relação ao quadrado do produto vetorial:
\begin{align*}
	 & |\vec{A}\times \vec{B}|^{2} = |\vec{A}|^{2}|\vec{B}|^{2}\sin^{2}{(\varphi )}                                                                   \\
	 & |\vec{A}\cdot \vec{B}|^{2} = |\vec{A}|^{2}|\vec{B}|^{2}\cos^{2}{(\varphi )}                                                                    \\
	 & \Rightarrow \frac{|\vec{A}\cdot \vec{B}|^{2}}{|\vec{A}|^{2}|\vec{B}|^{2}} + \frac{|\vec{A}\times \vec{B}|^{2}}{|\vec{A}|^{2}|\vec{B}|^{2}} = 1 \\
	 & \Rightarrow  |\vec{A}\cdot \vec{B}|^{2} + |\vec{A}\times \vec{B}|^{2} = |\vec{A}|^{2}|\vec{B}|^{2}.
\end{align*}

\end{document}

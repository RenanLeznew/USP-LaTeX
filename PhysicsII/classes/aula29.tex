\documentclass[PhysicsII/phsyicsII_notes.tex]{subfiles}
\begin{document}
\section{Aula 29 - 08/11/2023}
\subsection{Motivações}
\begin{itemize}
	\item Primeira Lei da Termodinâmica e Trabalho de um Gás;
	\item Tipos de Transformações Térmicas;
	\item Calor Específico e Capacidade Térmica dos Gases.
\end{itemize}
\subsection{Trabalho de um Gás}
Consideramos um sistema em que um recipiente de volume V comporta um gás ideal a uma pressão P e
temperatura T. Esse sistema sofre e realiza trabalho e trocas térmicas, resultando em mudanças em sua energia interna.
Para descrever essas mudanças, introduzimos a \textbf{Primeira Lei da Termodinâmica:}
\hypertarget{first_thermodynamics}{
	\[
		\boxed{dE_{int} = dW + dQ.}
	\]}
Uma aplicação direta disso é no contexto de uma expansão livre de um gás, no qual a pressão externa é nula, ou seja,
o gás está num vácuo. Considerando que a temperatura não varia, isto é, \(T_{f} = T_{i}\), segue que
\[
	W = Q = 0 = \Delta U.
\]
Assim, a energia cinética vale
\[
	\mathbb{K}_{i} = \frac{3}{2}k_{B}TN = \mathbb{K}_{f},
\]
mas e no caso de uma expansão real?

Para estudar essa situação, consideremos um recipiente de volume V, tampado, que contém um gás a uma temperatura T e que está
a uma pressão P. Aplica-se, na tampa, uma força F, que faz com que ela desloque uma quantidade infinitesimal \(\Delta x\). Assim,
para um gás expandindo em alta pressão, segue que
\[
	dW = Fdx + PAdx.
\]
Como o trabalho realizado pelo sistema será \(dW_{\text{pelo}} = PdV,\) então, segue que o trabalho sobre o sistema é
\(dW_{s} = -PdV\). Portanto, o trabalho exercido sobre o sistema é dado pela integral
\hypertarget{work_gas}{
	\[
		W_{s} = - \int_{V_{1}}^{V_{2}}PdV.
	\]}
\subsection{Tipos de Transformações Térmicas}
Com base na pressão, na temperatura e no volume de uma dada transferência de energia térmica, fazemos
classificações com base em quais dessas propriedades são preservadas. Respectivamente, chamam-se
de \textbf{transformações isobáricas, isotérmicas, ou isocóricas}. Através delas, colocando os valores em um gráfico
de pressão por volume, somos capazes de determinar o comportamento do trabalho necessário para realizar essa transferência.
Deixando de lado o caso isobárico, vejamos os outros dois:
\begin{itemize}
	\item [Isocórico:)] Como o volume final e inicial da transformação são os mesmos
	      e como o cálculo da área do gráfico fornece o trabalho necessário, segue que o trabalho feito por um gás durante
	      um processo isocórico é 0 (Neste gráfico, a área será de uma simples linha vertical, por isso é 0);
	\item [Isobárico:)] Para processos isobáricos, fixando a pressão e variando o volume, o gráfico formado é o de
	      de um retângulo, com altura sendo a pressão \(P\) e área da base \(V_{f} - V_{i}\), tal que
	      \[
		      W_{s} = P(V_{f} - V_{i}).
	      \]
	      Em particular, quando o trabalho realizado é positivo, dizemos que o processo é uma \textbf{comrpessão}, enquanto que,
	      quando o trabalho é negativo, chamamos de \textbf{descompressão}.
	\item [Isotérmica:)] O caso dos processos isotérmicos é um pouco mais complicado, pois tanto a pressão quanto o volume variam.
	      No entanto, o produto deles permanece o mesmo pela Lei dos Gases Ideais (já que \(PV = nRT\)). Com isso,
	      \[
		      W_{s} = -\int_{V_{i}}^{V_{f}}PdV = - \int_{V_{i}}^{V_{f}} \frac{nRT}{V}dV = -nRT[\ln{(V_{f})}-\ln{(V_{i})}] = nRT\ln{\biggl(\frac{V_{i}}{V_{f}}\biggr)}.
	      \]
\end{itemize}
\begin{example}
	Considere uma transformação térmica composta de quatro tempos, A, B, C e D. De D até A, o processo ocorre
	de forma isocórica, mas a pressão varia de 1 até 2 atm. Dos tempos A até B, o processo é isobárico, mas o volume varia
	de 1 até 2,5. Isto se repete, respectivamente, nos processos B até C e C até A, mas com sinais opostos. Encontre o trabalho total
	desta transformação. Encontre, também, a quantidade de calor gerado.

	Vamos separar com base nos processos para facilitar a organização.

	\textbf{\underline{Processo Isocórico}:}
	Nos processos isocóricos, temos
	\[
		W_{BC} = W_{DA} = 0
	\]

	\textbf{\underline{Processo Isobárico}:}
	Nos dois processos isobáricos, segue que:
	\begin{align*}
		 & W_{s_{AB}} = -P_{2}(V_{f}-V_{i}) = -3atmL   \\
		 & W_{s_{CD}} = -P_{1}(V_{f}-V_{i}) = 1,5atmL.
	\end{align*}

	Com estes dados, somos capazes de determinar o trabalho total do processo como sendo:
	\[
		W_{\text{total}} = W_{AB} + W_{BC} + W_{CD} + W_{DA} = -1,5atmL.
	\]
	Por fim, como a variação de energia térmica é nula no processo isotérmico,
	\[
		Q = +1,5atmL = 151,95J.
	\]
\end{example}
\subsection{Calor Específico e Capacidade Térmica dos Gases}
Denotemos por \(c_{V}, Q_{V}\) o calor específico do gás e a quantidade de calor dele em um contexto de volume constante e por \(c_{P}, Q_{P}\) o
calor específico do gás e sua quantidade de calor, mas em pressão constante. Para estudar a capacidade térmica dos gases, vamos
supor um contexto em que a variação da energia interna é positiva. Assim, começamos vendo o caso em que ao volume é constante, ou seja,
\begin{align*}
	            & dE_{int} = dQ_{v} + dW                                     \\
	            & dQ_{v} = c_{V}dT                                           \\
	            & dW = PdV = 0                                               \\
	\Rightarrow & dE_{int} = c_{V}dT \Rightarrow c_{V} = \frac{dE_{int}}{dT}
\end{align*}
Para o caso da pressão constante, temos
\begin{align*}
	 & dQ_{P} = c_{P}dt         \\
	 & dE_{int} = dQ_{P} + dW   \\
	 & dE_{int} = dQ_{P} - PdV  \\
	 & c_{V}dT = c_{P}dT - PdV.
\end{align*}
Utilizando a Lei dos Gases Ideais, isto é, \(PV = nRT\), podemos derivar a igualdade dos dois lados e obter a seguinte relação:
\begin{align*}
	 & PdV + VdP = nRdT         \\
	 & c_{V}dT + nRdT = c_{P}dT \\
	 & c_{P} = c_{V} + nR.
\end{align*}
Concluímos, então, que a calor específico de um gás a pressão constante é maior que o calor específico de um gás a
volume constante - \(c_{P} > c_{V}\).
\end{document}

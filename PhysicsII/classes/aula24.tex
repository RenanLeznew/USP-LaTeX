\documentclass[PhysicsII/phsyicsII_notes.tex]{subfiles}
\begin{document}
\section{Aula 24 - 19/10/2023}
\subsection{Motivações}
\begin{itemize}
	\item Continuanddo Efeito Doppler
\end{itemize}
\subsection{Efeito Doppler}
O caso que vimos previamente foi um no qual o observador encontra-se parado e, a fonte, movendo-se. Porém, há a possibilidade de ambos estarem em movimento. Para um observador que
move-se com velocidade \(u_{O}\), a equação
\[
	f_{O} = \frac{f_{F}}{1+\frac{u_{f}}{v}}.
\]
torna-se
\[
	f_{O} = \frac{\biggl(1\pm \frac{u_{O}}{v}\biggr)}{\biggl(1\pm \frac{u_{f}}{v}\biggr)}f_{F}.
\]
Vamos analisar o caso em que \(\frac{u_{O}}{v} << 1\) e \(\frac{u_{f}}{v} << 1\). Então,
\[
	f_{O} = \biggl(1\pm \frac{u_{O}}{v}\biggr)\biggl(1\pm \frac{u_{f}}{v}\biggr)f_{F} - f_{F},
\]
ou seja,
\[
	f_{O} = f_{F} - f_{F}\pm \frac{(u_{O}\pm u_{f})}{v}f_{F}.
\]
Escrevendo \(u = u_{O} \pm u_{f}\), portanto,
\[
	f_{O} = \pm\frac{u}{v}f_{F}.
\]
\begin{example}
	Assuma que há um fonte movendo-se a \(34 \frac{m}{s}\) de um observador, e que ela emite uma onda de frequência \(f_{F} = 4470Hz\). Encontre a frequência percebida pelo
	observador e, em seguida, faça o mesmo, mas considerando que o observador é quem move-se.

	Temos o comprimento de onda que chega até o observador como sendo
	\[
		\lambda_{O} = \frac{v-u_{f}}{f_{F}} = \frac{343-34}{400} = 0,77m.
	\]
	Além disso, o comprimento da onda saindo da fonte vale
	\[
		\lambda = \frac{v}{f_{F}} = \frac{343}{400} = 0,83m.
	\]
	Com esses dados, a frequência que o observador percebe é de
	\[
		f_{O} = \biggl(\frac{1}{1-\frac{u_{F}}{v}}\biggr)f_{F} = \frac{1}{\biggl(1-\frac{34}{343}\biggr)}\times 400 = 444Hz.
	\]
	Com relação ao caso do observador móvel e fonte parada, temos
	\[
		f_{O} = \biggl(\frac{u_{0}}{v}+1\biggr)f_{F} = \biggl(1 + \frac{34}{343}\biggr)\times 400 = 440Hz.
	\]
\end{example}
\begin{example}
	Para este exemplo, considere o caso de uma fonte parada emitindo uma onda de \(4400Hz\). Em direção a ela, com velocidade u, há uma onda enorme movimentando-se.
	A fim de descobrir quão rápida esta onda está aproximando-se, foi colocado um captador do som, que funciona por meio do efeito Doppler após a frequência emitida ser refletida
	pela onda em movimento. Encontre o valor de u.

	Começamos pela frequência recebida pela onda, cujo expressão é
	\[
		f_{O} = \biggl(1 + \frac{u}{v}\biggr)f_{F}.
	\]
	Analogamente, a frequência que a gravação capta do som refletido na onda é
	\[
		f_{O}' = \biggl(\frac{1}{1-\frac{u}{v}}\biggr)f_{O} = \biggl(\frac{1}{1-\frac{u}{v}}\biggr)\biggl(1+\frac{u}{v}\biggr)f_{F}
	\]
	Rearranjando a expressão,
	\[
		f_{O}' = \frac{1}{\frac{v-u}{v}}\biggl(\frac{v+u}{v}\biggr)f_{F}.
	\]
	Logo,
	\begin{align*}
		 & vf_{O}' - f_{O}'u = vf_{F} + f_{F}u                            \\
		 & v(f_{O}' - f_{F}) = u(f_{F}+f_{O}')                            \\
		 & u = v\frac{(f_{O}'-f_{F})}{(f_{F}+f_{O}')} = 16,3 \frac{m}{s}.
	\end{align*}
\end{example}
\end{document}

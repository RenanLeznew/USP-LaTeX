\documentclass[PhysicsII/physicsII_notes.tex]{subfiles}
\begin{document}
\section{Aula 10 - 31/08/2023}
\subsection{Motivações}
\begin{itemize}
	\item Exemplos para Trabalhar com os Conceitos.
\end{itemize}
\subsection{Exemplos}
\begin{example}
	Considere uma barra vertical de tamanho L, massa M e presa em um pivô. Além disso, tome uma partícula movendo-se em direção à barra vertical com velocidade \(\vec{v}\) e massa m.
	Ela atinge a barra no ponto x e passa a fazer parte dela. Encontre a razão entre a energia cinética final e inicial do sistema e o valor da energia cinética após a colisão, dado
	que a barra faz um ângulo máximo \(\theta_{max}\) após a colisão, deixando-a com uma nova altura de \(\frac{L}{2}\).

	\textbf{Descrevendo o sistema:}

	Num momento inicial, a energia cinética do sistema é dominada pela partícula, já que ela é a única em movimento. Sua energia cinética é calculável fazendo
	\[
		\mathbb{K}_{i} = \frac{1}{2}mv^{2}.
	\]
	Com relação à barra, ela possui um certo momento angular no ponto x, valendo
	\[
		L_{i} = mvx.
	\]

	Analisando o sistema após a colisão, chamando \(I_{t}\) como o momento de inércia total do sistema barra e partícula juntos, a energia cinética final
	é descrita por
	\[
		\mathbb{K}_{f} = \frac{1}{2}\frac{L^{2}}{I_{T}}
	\]
	e o momento angular final por
	\[
		\vec{L_{f}} = I_{t}\omega_{f}.
	\]
	\textbf{Desenvolvendo as contas:}

	Conseguimos explicitar uma forma para \(I_{t}\) como \(I_{t} = \frac{1}{3}ML^{2} + mx^{2}\). Já que o sistema é livre de forças externas, o momento angular deve ser conservado, isto é,
	\(L_{i} = L_{f}\). Podemos resolver isso para encontrar o valor de \(\omega_{f}:\)
	\begin{align*}
		            & mvx = \biggl[\frac{1}{3}ML^{2} + mx^{2}\biggr]\omega_{f}        \\
		\Rightarrow & \omega_{f} = \frac{mvx}{\biggl[\frac{1}{3}ML^{2}+mx^{2}\biggr]}
	\end{align*}

	Por fim, calculamos a razão entre as energias cinéticas
	\begin{align*}
		\frac{\mathbb{K}_{i}}{\mathbb{K}_{f}} & = \frac{\frac{1}{2}mv^{2}}{\frac{1}{2}\frac{L^{2}}{I_{t}}} = \frac{mv^{2}}{m^{2}v^{2}x^{2}}\frac{1}{3}(ML^{2}+mx^{2}) \\
		                                      & = \frac{1}{mx^{2}}\biggl[mx^{2}+\frac{1}{3}ML^{2}\biggr].
	\end{align*}
	\textbf{Pós colisão:}
	Utilizando as informações dadas pelo exercício, podemos descrever a energia cinética final do sistema como
	\[
		\mathbb{K}_{f}=Mg \biggl(\frac{L}{2}(1-\cos{(\theta_{max} )})\biggr) + mgx(1-\cos{(\theta_{max} )}),
	\]
	em que o termo de cosseno aparece manipulando as relações de tamanhos em triângulos.
\end{example}
\subsection{Disclaimer}
Tiveram outros exemplos nessa aula. No entanto, devido à quantidade de desenho que eles exigem, optei por não colocá-los (talvez por não saber
descrever o sistema física em texto como costumo fazer nos desenhos, os que apareceram eram complexos demais pra mim).

\end{document}

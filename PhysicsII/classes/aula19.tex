\documentclass[PhysicsII/physicsII_notes.tex]{subfiles}
\begin{document}
\section{Aula 19 - 04/10/2023}
\subsection{Motivações}
\begin{itemize}
	\item Tipos de Osciladores Harmônicos Amortecidos;
	\item Oscilador Harmônico Subamortecido;
	\item Oscilador Harmônico Amortecido Forçado.
\end{itemize}
\subsection{Oscilador Harmônico Amortecido}
Olhando para o caso estudado na última aula, chegamos à conclusão de que, num sistema
de oscilador harmônico amortecido, a velocidade angular é dada por
\[
	\omega  = \omega_{0} \sqrt[]{1-\biggl(\frac{b}{2m\omega_{0}}\biggr)^{2}}.
\]

Em particular, se o termo ao quadrado é muito menor que 1, pode-se aproximar \(\omega \) puramente
por \(\omega_{0}\). Este tipo de oscilador é chamado ``\textit{Oscilador Harmônico Subamortecido} ''.
Novas ferramentas como geladeiras e ares-condicionados implementam esse tipo de oscilador harmônico por meio
dos \textit{inverters}.

Caso \(\frac{b}{2m\omega_{0}} = 1,\) ou seja, \(b = 2m\omega_{0}\), dizemos que o oscilador é um
``\textit{Oscilador Harmônico Criticamente Amortecido}'', e denotaremos esse b específico por \(b_{c}.\)
Um sistema que funciona dessa forma é o amortecedor de um carro.

O caso restante é quando \(\frac{b}{2m\omega_{0}} > 1\), mas este caso \textbf{NÃO CAI NA PROVA}. Nestes casos,
o valor de \(\omega \) é complexo. Para lidar com ele, recordemos a fórmula de Euler:
\[
	e^{i \theta } = \cos^{}{(\theta )} + i\sin^{}{(\theta )},
\]
donde segue que
\[
	\cos^{}{(\theta )} = \frac{e^{i\theta }+e^{-i\theta }}{2}\quad\&\quad \sin^{}{(\theta )} = \frac{e^{i\theta }-e^{-i\theta }}{2i}.
\]
Assim, \(\omega = \pm \lambda i,\) e o caso é conhecido como ``\textit{Oscilador Harmônico Superamortecido}''.

Nosso enfoque será no caso subamortecido. Nestes casos, \(A(t) = A_{0}e^{-\gamma t}.\) Como
a energia mecânica é dada por \(E_{mec} = \frac{1}{2}kA^{2},\) colocando \(k=m\omega_{0}^{2}\), temos
\[
	E_{mec} = \frac{1}{2}m\omega_{0}^{2}A_{0}^{2}e^{-2\gamma t}.
\]
Chamando de \(E_{0} = \frac{1}{2}m\omega_{0}^{2}A_{0}^{2},\) a energia mecânica torna-se
\[
	E_{mec} = E_{0}e^{-\frac{b}{m}t} = E_{0}e^{-\frac{t}{\tau }}.
\]
Chamamos este \(\tau \) de \textit{tempo característico}. Derivando esta energia mecânica em relação ao tempo,
\[
	\frac{dE_{mec}}{dt} = -\frac{1}{\tau }E_{0} e^{-\frac{t}{\tau }} = -\frac{E}{\tau }.
\]
Equivalentemente,
\[
	\frac{dE}{E} = -\frac{dt}{\tau }.
\]
Antes de prosseguir, definimos o \textit{fator de qualidade} como \(Q = \omega_{0}\tau \). Para
oscilações muito pequenas, podemos escrever
\[
	\biggl(\frac{\Delta E}{E}\biggr)_{1\text{período}} = -\frac{T}{\tau } = -\frac{2\pi }{\omega_{0}\tau } = -\frac{2\pi }{Q}.
\]
Ainda mais, note que
\[
	\frac{b}{2m\omega_{0}} = \frac{1}{2\tau \omega_{0}} = \frac{1}{2Q},
\]
tal que \(\omega = \omega_{0}\sqrt[]{1 - \frac{1}{4Q^{2}}}\).
\begin{example}
	Considere um amortecedor de massa \(m=1100kg\) e frequência \(f = 1Hz\). Qual é o valor de b para que o amortecedor
	seja criticamente amortecido?

	Pelo que vimos,
	\[
		b_{c} = 2m\omega_{0} = 2\times(1100)\times 2\pi \frac{kg}{s}.
	\]
\end{example}
\begin{example}
	Suponha que uma corda que reproduz a nota Dó/C no piano tem uma frequência \(f = 262Hz\). Ela perde metade
	da energia em 4 segundos. Qual é o \(\tau \), o Q e o \(\biggl(\frac{\Delta E}{E}\biggr)_{\text{ciclo}}\) do sistema?

	Como houve uma perda de metade da energia, tem-se
	\[
		\frac{E_{0}}{2} = E_{0}e^{-\frac{t}{\tau }},
	\]
	donde segue que
	\[
		\ln \biggl(\frac{1}{2}\biggr) = \frac{-t}{\tau } \Rightarrow \tau \approx 5.7s.
	\]
	A partir disso, lembrando que \(Q = \omega_{0}\tau  = 2\pi f \tau \), segue que \(Q = 9.5\times 10^{3}.\) Finalmente,
	\[
		\biggl(\frac{\Delta E}{E}\biggr) = \frac{2\pi }{Q} = \frac{1}{f\tau }\approx 6.6\times 10^{-4}.
	\]
\end{example}
\subsection{Oscilador Harmônico Amortecido Forçado}
Prenda uma mola com constante elástica k a uma roda que fica subindo e descendo ela. Coloque um corpo de massa m na sua ponta e
suponha que está tudo submerso em um líquido. A equação deste sistema é
\[
	m \frac{d^{2}x}{dt^{2}} + b \frac{dx}{dt} + kx = F_{0}\cos^{}{(\omega t)}.
\]
Propomos uma solução
\[
	x(t) = A(\omega ) \cos^{}{(\omega t - \delta )}.
\]
Derivando essa equação,
\begin{align*}
	 & \dot x = -A\omega \sin^{}{(\omega t - \delta )}       \\
	 & \ddot x = -A\omega ^{2}\cos^{}{(\omega t - \delta )}.
\end{align*}
Expandindo o cosseno, obtemos as seguintes equações
\begin{center}
	\begin{table}[h]
		\centering
		\begin{tabular}{| c | c |}
			\hline
			Funções     & Expansões                                                                                                 \\
			\hline
			x           & \(A\cos^{}{(\omega t )}\cos^{}{(\delta )} + A \sin^{}{(\omega t)}\sin^{}{(\delta )}\)                     \\
			\(\dot x\)  & \(-A\omega \sin^{}{(\omega t)}\cos^{}{(\delta )} + A\omega \cos^{}{(\omega t)}\sin^{}{(\delta )}\)        \\
			\(\ddot x\) & \(-A\omega^{2}\cos^{}{(\omega t)}\cos^{}{(\delta )} - A\omega ^{2}\sin^{}{(\omega t)}\sin^{}{(\delta )}\) \\
			\hline
		\end{tabular}
	\end{table}
\end{center}

Analisemos separadamente os termos em cosseno e em seno.

\underline{\textbf{Termos em \(\cos^{}{(\omega t)}\)}}:
\begin{align*}
	 & -mA\omega ^{2}\cos^{}{(\delta )} + b A\omega \sin^{}{(\delta )} + kA \cos^{}{(\delta )} = F_{0} \\
	 & A\cos^{}{(\delta )}\biggl[-m\omega ^{2} + b \tan^{}{(\delta )} + m\omega_{0}^{2}\biggr] = F_{0}
\end{align*}

\underline{\textbf{Termos em \(\sin^{}{(\omega t)}\)}}:
\begin{align*}
	 & -mA\omega ^{2}\sin^{}{(\delta )} - bA\omega\cos^{}{(\delta )} + m\omega_{0}^{2}A\sin^{}{(\delta )} = 0             \\
	 & m(\omega_{0}^{2}-\omega ^{2})\sin^{}{(\delta )}=b\omega^{2}\cos^{}{(\delta )}                                      \\
	 & \tan^{}{(\delta )}=\frac{-b\omega}{m(\omega ^{2}-\omega_{0}^{2})} = \frac{b\omega}{m(\omega_{0}^{2}-\omega ^{2})}.
\end{align*}

Agora, como \(\sin^{2}{(\delta )} + \cos^{2}{(\delta )} = 1\), vale que
\[
	\biggl(\frac{\sin^{}{(\delta )}}{\cos^{}{(\delta )}}\biggr)^{2} + 1 = \frac{1}{\cos^{2}{(\delta )}} \Rightarrow \cos^{2}{(\delta )} = \frac{1}{1+\tan^{2}{(\delta )}} = \frac{1}{1+\frac{b^{2}\omega ^{2}}{m(\omega_{0}^{2}-\omega ^{2})^{2}}},
\]
ou seja,
\[
	\cos^{2}{(\delta )} = \frac{m(\omega_{0}^{2}-\omega ^{2})^{2}}{m(\omega_{0}^{2}-\omega ^{2})^{2}+b^{2}\omega ^{2}.}
\]
Disto, encontramos o valor de A usando a separação dos termos em cosseno como sendo
\[
	A = \frac{F_{0}}{\cos^{}{(\delta )}\biggl[m(\omega_{0}^{2}-\omega ^{2})+\frac{b^{2}\omega ^{2}}{m(\omega_{0}^{2}-\omega ^\{2\})}\biggr]} = \frac{F_{0}}{\frac{m(\omega_{0}^{2}-\omega ^{2})}{\sqrt[]{m(\omega_{0}^{2}-\omega ^{2})^{2}+b^{2}\omega ^{2}}}\biggl[m(\omega_{0}^{2}-\omega ^{2})+\frac{b^{2}\omega ^{2}}{m(\omega_{0}^{2}-\omega ^\{2\})}\biggr]}m(\omega_{0}^{2}-\omega ^{2})
\]
Simplificando e cortando termos,
\[
	A = \frac{F_{0}}{\biggl[m(\omega_{0}^{2}-\omega ^{2})^{2}+b^{2}\omega ^{2}\biggr]^{\frac{1}{2}}}
\]
\end{document}

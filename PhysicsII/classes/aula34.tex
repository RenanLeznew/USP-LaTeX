\documentclass[PhysicsII/phsyicsII_notes.tex]{subfiles}
\begin{document}
\section{Aula 34 - 29/11/2023}
\subsection{Motivações}
\begin{itemize}
	\item Entropia.
\end{itemize}
\subsection{Entropia}
A entropia é, coloquialmente falando, o conceito de Caos, da tendência das coisas tenderem à desordem.
Para fins de exemplificar, tome uma caixa com velocidade v preenchida por um gás. Essa velocidade pode ser transformada
em trabalho, por exemplo, com uma colisão. Assuma, então, que ela colide inelasticamente com uma parede. Ao colidir, tanto a
caixa quanto a parede esquentam - Em outras palavras, energia que poderia ter sido usada em trabalho foi transformada em calor.
A entropia mede essa quantidade de energia desperdiçada, donde definimos a entropia por
\[
	\hypertarget{entropy}{\boxed{dS = \frac{dQ_{rev}}{T}}}\quad\quad [S] = \frac{J}{K}.
\]
Para encontrar a variação de entropia, utiliza-se uma integral:
\[
	S_{2}-S_{1} = \int_{1}^{2}\frac{dQ_{rev}}{T}.
\]
Para um gás ideal, lembre que \(dE_{int} = dW + dQ_{rev}\). Podemos reescrever como
\[
	dE_{int} = -PdV + dQ_{rev}\Rightarrow c_{V}dT = -PdV + dQ_{rev}.
\]
Assim,
\[
	\frac{dQ_{rev}}{T} = \frac{c_{V}dT}{T}+\frac{PdV}{T},
\]
ou seja,
\[
	\Delta S = \int_{1}^{2} c_{v}\frac{dT}{T} + \int_{1}^{2} \frac{nRT}{V}\frac{dV}{T} = c_{v}\ln^{}{\biggl(\frac{T_{2}}{T_{1}}\biggr)} + nR\ln^{}{\biggl(\frac{V_{2}}{V_{1}}\biggr)}.
\]
Suponha que um gás ideal sofre uma expansão isotérmica. Como \(T_{2} = T_{1}, \ln^{}{\biggl(\frac{T_{2}}{T_{1}}\biggr)} = \ln^{}{(1)}=0\),
tal que
\[
	\Delta S = nR\ln^{}{\biggl(\frac{V_{2}}{V_{1}}\biggr)}.
\]
De maneira análoga, se a expansão for isocórica,
\[
	\Delta S = c_{V}\ln^{}{\biggl(\frac{T_{2}}{T_{1}}\biggr)}.
\]
Para transformações isotérmicas, nas quais \(Q_{rev} = W = nRT\ln^{}{\biggl(\frac{V_{2}}{V_{1}}\biggr)},\) então
\[
	\Delta S_{gas} = \frac{Q_{rev}}{T}\quad\&\quad \Delta S_{res} = \frac{-Q_{rev}}{T}.
\]
Em particular, observe que a entropia do sistema universo (reservatórios + gás) não varia, pois
\[
	\Delta S = \Delta S_{gas} + \Delta S_{res} = 0.
\]
Isto ocorre pois o processo é reversível.

Restam as isobáricas. Neste caso, \(dQ_{rev} = c_{P}dT\), tal que
\[
	\Delta S = \int_{1}^{2} c_{P}\frac{dT}{T} = c_{P}\ln^{}{\biggl(\frac{T_{2}}{T_{1}}\biggr)}
\]
Qual é a entropia da expansão livre de um gás?

Vamos tratar como um processo irreversível no qual o volume varia, tal que
\[
	\Delta S_{gas} = nR\ln^{}{\biggl(\frac{V_{2}}{V_{1}}\biggr)}
\]
Para o resto do sistema, essa variação é nula, tal que
\[
	\Delta S_{u} > 0
\]
\begin{example}
	Para um objeto em queda livre, como a geração de calor é dada pela colisão com o chão, segue que
	\[
		\Delta S = \frac{Q_{rev}}{T} = \frac{mgh}{T}
	\]
	Além disso, para uma pessoa morta (????), basta calcular quanto de calor ele cede. Supondo que ele é essencialmente
	água, então
	\[
		\Delta S = \frac{m_{\text{morto}}c_{\text{água}}(T_{\text{ambiente}}-35.5^{\mathrm{o}})}{T_{\text{ambiente}}}.
	\]
\end{example}
\begin{example}
	Podemos calcular a variação de entropia da transferência de calor entre 2 reservatórios como
	\[
		\Delta S_{q} = -\frac{Q}{T_{q}}\quad\&\quad \Delta S_{f} = \frac{Q}{T_{f}}
	\]
	Assim,
	\[
		\Delta S_{u} = \frac{Q}{T_{f}} - \frac{Q}{T_{q}}.
	\]
\end{example}
Levando em conta um sistema de ciclo de Carnot, a variação de entropia do reservatório quente pode ser encontrada por
\[
	\Delta S_{q} = -\frac{Q_{q}}{T_{q}}.
\]
A do reservatório frio, por outro lado, é
\[
	\Delta S_{f} = \frac{Q_{f}}{T_{f}}.
\]
Logo, a variação total de entropia será
\[
	\Delta S_{T} = \Delta S_{q} + \Delta S_{f} = \frac{Q_{f}}{T_{f}} - \frac{Q_{q}}{T_{q}}.
\]
Como \(\frac{T_{f}}{T_{q}} = \frac{Q_{f}}{Q_{q}},\) temos
\[
	\Delta S_{T} = \frac{Q_{f}}{T_{f}} - \frac{Q_{q}}{T_{f}}\frac{Q_{f}}{Q_{q}} = 0
\]
\begin{example}
	Um copo com uma massa m de água, inicialmente a uma temperatura de \({0}^{\mathrm{o}}C\), é armazenado em um refrigerador, que está a uma temperatura
	de \(T_{R}={-15}^{\mathrm{o}}C\). Qual é a variação da entropia da água, do refrigerador e do universo?

	Primeiramente, a água deve perder calor para congelar, tal que, a priori
	\[
		\Delta S_{H_{2}O} = \frac{-Q_{resfriar}}{T_{i}} = -\frac{mL}{T_{i}}.
	\]
	Para a segunda etapa, ou seja, a da resfriação até \({-15}^{\mathrm{o}}C\), note que \(dQ_{res} = mcdT.\) Com isso,
	\[
		\Delta S = \int_{i}^{R}mc \frac{dT}{T} = mc \ln^{}{\biggl(\frac{T_{R}}{T_{i}}\biggr)},
	\]
	do que segue que
	\[
		\Delta S_{H_{2}O} = \frac{-mL}{T_{i}} + mc\ln^{}{\biggl(\frac{T_{R}}{T_{i}}\biggr)}.
	\]
	Agora, com relação ao refrigerador, ele recebe calor para que a água vire gelo, ou seja, começamos por
	\[
		\Delta S_{ref} = \frac{mL}{T_{R}} + \frac{mc(T_{R}-T_{i})}{T_{R}}.
	\]
	Consequentemente,
	\[
		\Delta S_{u} = \Delta S_{H_{2}O} + \Delta S_{ref} = \frac{-mL}{T_{i}} + mc\ln^{}{\biggl(\frac{T_{R}}{T_{i}}\biggr)} + \frac{mL}{T_{R}} + \frac{mc(T_{R}-T_{i})}{T_{R}}.
	\]
\end{example}
\end{document}

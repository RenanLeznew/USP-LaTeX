\documentclass[PhysicsII/physicsII_notes.tex]{subfiles}
\begin{document}
\section{Aula 04 - 17/08/2023}
\subsection{Motivações }
\begin{itemize}
	\item Segunda lei de Newton do Movimento Circular;
	\item Torque da Gravidade.
\end{itemize}
\subsection{Segunda Lei de Newton}
Ao considerarmos uma força aplicada a um objeto em torno de um círculo de raio r, essa força faz um ângulo
\(\theta \) com a paralela ao raio. Além disso, há uma componente dessa força que será tangente
à trajetória do objeto ao longo do círculo. Denotando essa segunda por \(F_{t},\) há duas formas de expressá-la:
\[
	F\sin{(\theta )} = F_{t},\quad F_{t} = ma_{t}.
\]
Além disso, a aceleração tangencial \(a_{t}\) satisfaz \(a_{t} = r\alpha \). Assim, obtemos a relação
\[
	F sin(\theta ) = ma_{t} \Rightarrow F\sin{\theta } = mr\alpha \Longleftrightarrow rF\sin{(\theta )} = mr^{2}\alpha.
\]
Esse termo à esquerda é conhecido como \textbf{torque}
\[
	\hypertarget{torque}{\boxed{\tau = mr^{2}\alpha = rF\sin{(\theta )}}}
\]
Em particular, sendo o torque total a soma de todos os torques, obtemos
\[
	\tau = \sum\limits_{}^{}\tau_{i} = \sum\limits_{}^{}m_{i}r_{i}^{2}\alpha = I\alpha
\]
Uma propriedade é que a soma dos torques das forças internas vale zero.

Olhando um caso mais específico, ao considerarmos um círculo de raio r e uma força que
faz um ângulo \(\theta \) com a paralela ao raio e outro círculo menor de raio r' com a mesma força aplicada,
mas ângulo \(\theta ',\) então \(l=r'\sin{(\theta ')}\) é a componente perpendicular à linha na qual a força está atuando.
A vantagem disso é que o torque pode ser, então, expresso através de \(\tau = Fl = F_{t}r'\)
\subsection{O Torque da Força da Gravidade}
Se considerarmos um corpo sofrendo a ação da força peso, o torque desse corpo pode ser descrito por
\(\tau_{i} = m_{i}gx_{i}\) e, o torque total, será a soma desses torques:
\[
	\tau_{r} = \sum\limits_{}^{}\tau_{i} = \sum\limits_{}^{}[m_{i}x_{i}]g
\]
Mas, esse é exatamente o torque do centro de massa do objeto \(\tau_{r} = Mx_{cm}g\). Outro assunto que é
importante ressaltar é que, durante os estudos de dinâmica, a forma de estudar as forças em um sistema é através
dos chamados diagramas de força, o que traz à tona a questão do que funcionaria pro estudo do torque.
\begin{example}
	Considere uma roda de bicicleta e a catraca, que sofre uma força F de 18N. Suponha que o raio r
	da catraca é de 7cm e o da roda, R, vale 35cm. Além disso, a massa vale 2.4kg. Qual é a velocidade angular para t=5,5s?

	Começamos afirmando que o torque é \(\tau = I\alpha = Fr_{c}\). Assim,
	\[
		\alpha = \frac{Fr_{c}}{I} = \frac{Fr_{c}}{MR^{2}} = \frac{18 \cdot (0,07)}{2,4(0,35)^{2}}\frac{rad}{s^{2}}.
	\]
	Com isso,
	\[
		\omega = \omega_{0} + \alpha t = \alpha t = \frac{18 \cdot (0,07)}{2,4(0,35)^{2}} \cdot 5,5 = 21,4 \frac{rad}{s}
	\]
\end{example}
\begin{example}
	Considere uma barra de massa m e comprimento l está presa por um pivô, o qual realiza uma força F. Após soltá-la, qual é a força que o pivô realiza?

	Sabemos que \(\tau = mg \frac{l}{2} = I\alpha = \frac{1}{3}ml^{2}.\) Logo,
	\begin{align*}
		 & mg \frac{l}{2} = \frac{1}{3}ml^{2}\alpha \\
		 & \alpha = \frac{3}{2}\frac{g}{l}.
	\end{align*}
	Olhando no eixo y, sabemos que \(a_{cm_{y}} = r\alpha  = \frac{l}{2}\frac{3}{2}\frac{g}{l} = \frac{3}{4}g\), tal que
	\[
		F - mg = -ma_{cm_y} \Rightarrow F = mg - \frac{3}{4}mg = \frac{1}{4}mg
	\]
\end{example}
\begin{example}
	Suponha que temos uma roldana de raio R e momento de inércia I. Pendura-se um corpo de massa m na roldana. Qual é a aceleração de queda do corpo?

	\textbf{Roldana:}
	As forças que atuam na roldana são o Peso dela, \(P_{r}\), a tensão T
	e a força resultante ao peso \(F_{r}\). Assim,
	\begin{align*}
		 & F_{r} = P_{r} + T                \\
		 & TR = I\alpha,\quad a = \alpha R.
	\end{align*}

	\textbf{Corpo:}
	No corpo, por outro lado, tem-se apenas a tensão T e o peso mg, de forma que
	\[
		mg-T = ma.\quad a = \alpha R
	\]
\end{example}
Continua na próxima aula...

\end{document}

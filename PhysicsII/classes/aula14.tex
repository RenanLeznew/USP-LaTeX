\documentclass[physicsII_notes.tex]{subfiles}
\begin{document}
\section{Aula 14 - 21/09/2023}
\subsection{Motivações}
\begin{itemize}
	\item Exemplos de Empuxo;
	\item Hidrodinâmica.
\end{itemize}
\begin{example}
	Tome 5 Beckeres iguais. No primeiro, há um barquinho flutuando. No segundo, o barquinho afundou e tem o dobro da densidade. No terceiro,
	coloca-se um bloco de gelo flutuando. No quarto, o bloco de gelo está preso e possui metade da densidade. No último, não faz-se nada. Como são os pesos?
	Suponha que todos os objetos colocados têm o mesmo peso \(F_{g_{o}}\)

	\textbf{Becker E:}
	O peso de E é fácil, já que não têm nada, utilizaremos ele como base, sendo seu peso \(F_{g_{E}}\).

	\textbf{Becker A:}
	Inicialmente, antes de inserir o barco, \(F_{g_{A}} = F_{g_{E}}\). Após a adição dele, o deslocamento de líquido equivale ao peso do barco,
	ou seja, \(F_{g_{A}} = F_{g_{E}}-F_{g_{O}}.\) Como o barco está agora incluso, adiciona-se seu peso novamente, donde obtemos
	\[
		F_{g_{A}} = F_{g_{E}} - F_{g_{O}} + F_{g_{O}} = F_{g_{E}}.
	\]

	\textbf{Becker B:}
	Como o barco tem o dobro da densidade, mas a massa é a mesma, o volume do barco é metade do primeiro caso, tal que
	\[
		F_{g_{B}} = F_{g_{E}} - \frac{F_{g_{O}}}{2} + F_{g_{O}} = F_{g_{E}} + \frac{F_{g_{O}}}{2}
	\]

	\textbf{Becker C:}
	O raciocínio do A vale para este, donde obtemos
	\[
		F_{g_{C}} = F_{g_{E}} - F_{g_{O}} + F_{g_{O}} = F_{g_{E}}.
	\]

	\textbf{Becker D:}
	Analogamente ao B, como a massa é a mesma, mas a densidade é metade, o volume deslocado é duas vezes maior, ou seja,
	\[
		F_{g_{D}} = F_{g_{E}} - 2 F_{g_{O}} + F_{g_{O}} = F_{g_{E}} - F_{g_{O}}.
	\]
\end{example}
\begin{example}
	Dado que um iceberg está no mar flutuando, calcule a fração dele que está visível.

	A força gravitacional do iceberg é dada pela relação
	\[
		F_{g_{ic}} = \rho_{gelo}Vg.
	\]
	Sabemos, também, que \(E = F_{g_{ic}}\). Assim,
	\[
		\rho_{mar}V_{submerso}g = \rho_{gelo}Vg \Rightarrow \frac{V_{submerso}}{V} = \frac{\rho_{gelo}}{\rho_{mar}} = \frac{0.9\frac{g}{cm^{3}}}{1.022\frac{g}{cm^{3}}}.
	\]
	Portanto, \(\frac{V_{sub}}{V}\approx 0.9\).
\end{example}
\subsection{Hidrodinâmica}
Pegue um tubo, com área na parte maior \(A_{1}\) que estreita-se um pouco conforme é percorrido, formando quase uma garrafa, com a ponta mais estreita tendo área \(A_{2}\).
Preencha-o com um fluído. Na parte não estreitada, o fluído tem velocidade \(\vec{v}_{1}\) e, após chegar na parte mais estreita, passa a ter velocidade \(\vec{v_{2}}\).
A massa do fluído que passa pela primeira parte do percurso - antes de estreitar - a cada instante de tempo, então, será
\[
	\Delta m_{1} = \rho_{1}A_{1}v_{1}\Delta t
\]
e, pela segunda parte,
\[
	\Delta m_{2} = \rho_{2}A_{2}v_{2}\Delta t
\]
Chamamos as vazões de massas nos pontos respectivos por
\[
	\text{vazão massica} = \left\{\begin{array}{ll}
		\frac{\Delta m_{1}}{\Delta t} = I_{m_{1}} \\
		\frac{\Delta m_{2}}{\Delta t} = I_{m_{2}}.
	\end{array}\right.
\]
A quantidade de massa passando entre os pontos pode, assim, ser obtida calculando-se a diferença entre as duas vazões acima,
resultando numa equação conhecida como \textit{equação da continuidade}:
\[
	\hypertarget{continuity_equation}{\boxed{\frac{dm_{12}}{dt} = I_{m_{1}}-I_{m_{2}}}}
\]
Observe que, se \(\frac{dm_{12}}{dt} = 0\), então \(I_{m_{1}} = I_{m_{2}}\), ou seja, \(A_{1}v_{1}=A_{2}v_{2}\). Nestes casos, chamamos de vazão
volumétrica o valor \(I_{v}=Av\).

\end{document}

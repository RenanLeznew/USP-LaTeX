\documentclass[PhysicsII/physicsII_notes.tex]{subfiles}
\begin{document}
\section{Aula 01 - 09/08/2023}
\subsection{Motivações}
\begin{itemize}
	\item Ângulos, velocidade angular e aceleração angular;
	\item Energia em sistemas com rotação.
\end{itemize}
\subsection{Rotação}
Antes de qualquer coisa, convencionamos o sentido antihorário como aquele em que \(\Delta \theta >0\) e
o sentido horário como o que \(\Delta \theta <0.\) Uma volta completa em torno do círculo é dada pela versão com \(2\pi\) da fórmula do arco de círculo
\(\Delta S_{i} = r_{i}\Delta \theta = 2\pi r_{i}\) e, com isso, a variação do ângulo em uma volta completa é dada por
\[
	\Delta \theta = \frac{S_{i}}{r_{i}} = \frac{2\pi r_{i}}{r_{i}} = 2\pi rad.
\]
Um dos assuntos de importância para nós é o estudo da variação temporal do ângulo. Definimos, nessa lógica, a
velocidade angular média por
\[
	\omega _{med} = \frac{\Delta \theta }{\Delta t}.
\]
De modo análogo ao que vimos com cinemática, existe também a velocidade angular instantânea, obtida tomando o limite:
\[
	\omega = \lim_{\Delta t\to 0}\frac{\Delta \theta }{\Delta t} = \frac{d\theta }{dt}.
\]
Observa-se de cara que, se \(\omega >0, \theta \) aumenta e, se \(\omega <0, \theta \) diminui. Assim como antes,
precisamos ver, também, a unidade. Em cinemática, a unidade de velocidade era metro por segundo. Dessa vez, já que
o ângulo move-se em radianos, mas há outras unidades, como a revolução e o grau. Logo, as unidades de \(\omega \) podem ser \([\omega ] = \frac{radianos}{tempo} = \frac{rad}{s}, \frac{graus}{s}, \frac{\text{revolução}}{s},\)
em que \(1\text{revolução} = 2\pi rad = 360\deg\)

Por exemplo, se um CD roda a 3000rpm, pode-se expressar essa velocidade de rotação como
\[
	\omega = 3000rpm = \frac{3000 \cdot 2\pi}{60} = \frac{600}{6}\pi = 100\pi \frac{rad}{s}.
\]

Analogamente, é possível analisar a variação da própria velocidade angular com o tempo, resultando na chamada
acelerações angulares média e instantânea:
\[
	\alpha_{med} = \frac{\Delta \omega }{\Delta t}\quad \alpha  = \frac{d\omega }{dt} = \frac{d}{dt}\biggl(\frac{d\theta }{dt}\biggr) = \frac{d^{2}\theta }{dt^{2}}.
\]
A unidade dessa grandeza, novamente, similar à versão linear dela, será dada em \([\alpha ]= \frac{radiano}{s^{2}}\), ou \([\alpha ]=\frac{grau}{s^{2}}\), etc. Nessas
situações todas, se \(\alpha >0, \omega \) aumenta e, se \(\alpha <0, \omega \) diminui.

Agora, suponha que \(\alpha \) é constante. Todos os processos de movimento uniformemente acelerado são válidos aqui também:
\begin{table}[h!]
	\centering
	\begin{tabular}{|c|c|c|}
		\hline
		                    & \textbf{Variáveis angulares}                                    & \textbf{Variáveis escalares}                                                        \\
		\hline
		\textbf{Posição}    & $\theta(t) = \theta_{0} + \omega_{0}t + \frac{\alpha t^{2}}{2}$ & $s(t) = R\theta(t)$                                                                 \\
		\hline
		\textbf{Velocidade} & $\omega(t) = \omega_{0} + \alpha t = \frac{d\theta }{dt}$       & $v(t) = v_{0} + at = \frac{dx}{dt}$                                                 \\
		\hline
		\textbf{Aceleração} & $\alpha(t) = \frac{d\omega(t)}{dt} = \frac{d^2\theta(t)}{dt^2}$ & $|\vec{a}(t)| = \frac{dv(t)}{dt} = R\alpha(t),\quad |\vec{a}_{cp}| = \frac{v^2}{R}$ \\
		\hline
		\textbf{Torricelli} & $\omega^{2}(t) = \omega_{0}^{2} + 2\alpha \Delta \theta $       & $v^{2} = v_{0}^{2} + 2a\Delta s.$                                                   \\
		\hline
	\end{tabular}
	\caption{Resumo movimento circular.}
	\label{tab:my_label}
\end{table}

\begin{example}
	Suponha que há um CD que começa no repouso. Ele começa a girar, indo de 0 a 500rpm em 5.5s. Pergunta-se:
	\begin{itemize}
		\item[a)] Quanto vale \(\alpha \)?
		\item[b)] Quantas voltas o CD dá em 5.5s?
		\item[c)] Qual é a distância percorrida por uma ponta a 6cm do eixo de rotação?
	\end{itemize}

	\textbf{Soluções:}
	\begin{itemize}
		\item[a)] Temos \(\omega (0) = 0, \omega (5.5) = 500rpm.\) Segue que
		      \[
			      \omega(t) = \omega_{0} + \alpha t \Rightarrow \alpha  = \frac{\omega (t)}{t} = \frac{500 \cdot 2\pi}{5.5 \cdot 60}\approx 9.52 \frac{rad}{s^{2}}
		      \]

		\item[b)] Aplicamos o Torricelli angular com os dados que temos:
		      \[
			      \omega^{2} = 2\alpha \Delta \theta \Rightarrow \delta \theta = \frac{\omega^{2}}{2\alpha }\approx 144 rad \Rightarrow \frac{144}{2\pi}rad\approx 23\text{rotações}.
		      \]

		\item[c)] Por fim, multiplicando a variação do ângulo pelo raio, obtemos
		      \[
			      \Delta S_{i} = r\Delta \theta = 6 \cdot 10^{-2}\cdot 144\approx 8.65m.
		      \]
	\end{itemize}
\end{example}

Olhando de forma cautelosa a fórmula de arco de circulo, podemos derivá-la com respeito ao tempo utilizando o que vimos até agora:
\[
	\frac{dS_{i}}{dt} = V_{t} = r_{i}\frac{d\theta }{dt} = r_{i}\omega.
\]
Essa derivação resulta em uma velocidade linear, que também pode ser derivada a fim de obter uma aceleração linear
\[
	\frac{dV_{t}}{dt} = r_{i}\frac{d\omega }{dt} \Rightarrow a_{t} = r_{i}\alpha.
\]
Note a relação entre as duas acelerações que obtivemos, \(a_{c} = \frac{V_{t}^{2}}{r_{i}}= \frac{r_{i}^{2}\omega^{2}}{r_{i}} = r_{i}\omega^{2}.\)

\subsection{Energia Cinética de Rotação}
A energia cinética, como vista previamente, é dada por
\[
	\mathcal{K} = \frac{1}{2}m_{i}v_{i}^{2}.
\]
Agora, imagine um corpo discreto (formado por vários pontos). Somemos as energias deles, tal que a energia cinética total é
\[
	\mathcal{K}_{T} = \sum\limits_{}^{}\frac{1}{2}m_{i}v_{i}^{2}.
\]
Mas, sabemos que \(v_{i} = r_{i}\omega, \) tal que
\[
	\mathcal{K} = \frac{1}{2}\sum\limits_{}^{}m_{i}r_{i}^{2}\omega^{2} = \frac{1}{2}\biggl[\sum\limits_{}^{}m_{i}r_{i}^{2}\biggr]\omega^{2}
\]
Chamemos o termo em colchete de momento de inércia, denotado por \(I:= \sum\limits_{}^{}m_{i}r_{i}^{2} = \sum\limits_{}^{}I_{i}\). Logo,
\[
	\boxed{\hypertarget{kin_en}{\mathcal{K}_{T} = \frac{1}{2}I\omega^{2}.}}
\]
\end{document}

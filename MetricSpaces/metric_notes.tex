\documentclass{article}
 \usepackage{amsmath}
 \usepackage{microtype}
 \usepackage{amsthm}
 \usepackage{amssymb}
 \usepackage{pgfplots}
 \usepackage[utf8]{inputenc}
 \usepackage{amsfonts}
 \usepackage[margin=2.5cm]{geometry}
 \usepackage{graphicx}
 \usepackage[export]{adjustbox}
 \usepackage{fancyhdr}
 \usepackage[portuguese]{babel}
 \usepackage{hyperref}
 \usepackage{lastpage}
 \usepackage{mathtools}
 \usepackage{sansmathfonts}
 \usepackage[T1]{fontenc}
 \setcounter{section}{-1}

 \pagestyle{fancy}
 \fancyhf{}

 \pgfplotsset{compat = 1.18}

 \hypersetup{
     colorlinks,
     citecolor=black,
     filecolor=black,
     linkcolor=black,
     urlcolor=black
 }
 \newtheorem*{def*}{\underline{Defini\c c\~ao}}
 \newtheorem*{theorem*}{\underline{Teorema}}
 \newtheorem*{lemma*}{\underline{Lema}}
 \newtheorem*{prop*}{\underline{Proposi\c c\~ao}}
 \newtheorem*{crl*}{\underline{Corolário}}
 \newtheorem{example}{\underline{Exemplo}}
 \newtheorem*{proof*}{\underline{Prova}}
 \newtheorem*{claim*}{\underline{Afirmação}}
 \renewcommand\qedsymbol{$\blacksquare$}

 \rfoot{P\'agina \thepage \hspace{1pt} de \pageref{LastPage}}

 \begin{document}
 \begin{figure}[ht]
  \minipage{0.76\textwidth}
    \includegraphics[width=4cm]{icmc.png}
    \hspace{7cm}
    \includegraphics[height=4.9cm,width=4cm]{brasao_usp_cor.jpg}
  \endminipage  
\end{figure}

\begin{center}
  \vspace{1cm}
  \LARGE
  UNIVERSIDADE DE S\~AO PAULO

  \vspace{1.3cm}
  \LARGE
  INSTITUTO DE CI\^ENCIAS MATEM\'ATICAS E COMPUTACIONAIS - ICMC

  \vspace{1.7cm}
  \Large
  \textbf{Notas de Espaços Métricos}

  \vspace{1.3cm}
  \large
  \textbf{Renan Wenzel - 11169472}

  \vspace{1.3cm}
  \large
  \textbf{Professora - Thaís Jordão}

  \textbf{E-mail: tjordao@icmc.usp.br}

  \vspace{1.3cm}
  \today
\end{center}

 \newpage

 \tableofcontents
 \newpage
 \section{Informações (Possivelmente) Úteis}
 \subsection{Datas das Provas:}
\begin{itemize}
  \item[P1)] 31/08 - Peso 1;
  \item[P2)] 03/10 - Peso 2;
  \item[P3)] 31/10 - Peso 2;
  \item[P4)] 23/11 - Peso 3;
  \item[P5)] 14/12 - Peso 3.
\end{itemize}
\subsection{Bibliografia}
 \begin{itemize}
   \item LIMA, E. L.  ``Espaços Métricos'', Rio de Janeiro: Projeto Euclides, 2005.
   \item DOMINGUES, H. H. ``Espaços Métricos e Introdução à Topologia'', Atual Editora, 1982.
 \end{itemize}
\subsection*{Monitoria}
  Às terças-feiras, às 19h (Sete horas da noite).

 \newpage

\section{Aula 01 - 08/08/2023}
\subsection{Motivações}
\begin{itemize}
  \item Introdução ao Material do Curso.
\end{itemize}
\subsection*{O que é um espaço métrico?}
  Ao longo deste curso, trabalharemos com um conjunto M não-vazio.
\hypertarget{def_metric}{ \begin{def*}
   Uma função \(d:M\times M\rightarrow \mathbb{R}\) é dita ser uma métrica em M se: 
  \begin{itemize}
    \item[i)] \(d(x, y)\geq 0, x, y\in M\);
    \item[ii)] \(d(x, y) = 0 \Leftrightarrow x = y, x, y\in M\);
    \item[iii)] \(d(x, y) = d(y, x), x, y\in M\);
    \item[iv)] \(d(x, y)\leq d(x, z) + d(z, y), x, y, z\in M\).
  \end{itemize}
  Neste caso, o par \((M, d)\) é chamado espaço métrico.
 \end{def*}
}
\begin{example}
 \begin{itemize}
   
   \item[1)] \((\mathbb{R}, d)\), em que \(d:\mathbb{R}\times \mathbb{R}\rightarrow [0, \infty)\) é dado por 
    d(x, y) = \(|x-y|.\) É claro que, olhando para \(d(x,y),\) vale para quaisquer x, y reais que 
      \[
        d(x, y) = |x-y| = |-1(y-x)| = 1|y-x| = d(y, x).
      \]
    Assim, resta verificarmos os itens dois e quatro da \hyperlink{def_metric}{definição de métrica}. Para o item (ii), 
    \[
      |x-y| = 0 \Longleftrightarrow x = y.
    \]
    Com relação ao último item, observe que 
    \[
      |x+y|\leq |x| + |y|.
    \]
    De fato, como \(|a|\geq a\) para todo número real a, 
    \[
      |x+y|^{2} = (x+y)^{2} = x^{2} + 2xy +y^{2}\leq x^{2} + 2|x||y| + y^{2} = (|x| + |y|)^{2}.
    \]
    Logo, tomando a raíz dos dois lados, segue a afirmação:
    \[
        |x+y|\leq |x| + |y|.
    \]
    Com isso, temos 
      \[
        d(x, y) = |x-y| = |x-z+z-y| = |(x-z)-(y-z)|\leq |x-z| + |y-z|.
      \]
    Portanto, \(d(x, y)\leq d(x, z) + d(z, y)\), o que torna \((\mathbb{R}, d)\) um espaço métrico.

    \item[2)] Seja X um conjunto não-vazio. Definimos 
      \[
        d:X\times X\rightarrow [0, \infty)
      \]
      por 
        \[
          d(x, y) = \left\{\begin{array}{ll}
              0,\quad x = y\\
              1,\quad x\neq y.
            \end{array}\right.
        \]
      Esta métrica é conhecida como \textit{métrica discreta}. Verifiquemos as propriedades dela.

Com efeito, como a imagem dela pode ser apenas 0 ou 1, o item 1 é trivial. Por definição, a métrica vale 0 se,
e somente se, x e y são iguais, tal que o item (ii) está feito. O item (iii) segue automaticamente se x e y são iguais. Caso
eles sejam diferentes, temos \(d(x, y) = 1, d(y, x) = 1\), ou seja, o item (iii) é válido para todos os casos. Por fim, a desigualdade triangular fica como exercício.
 \end{itemize} 
\end{example}
\newpage

\section{Aula 02 - 10/08/2023}
\subsection{Motivações}
\begin{itemize}
  \item Exemplos de espaços métricos;
  \item Similaridade entre métricas;  
  \item Produto de espaços métricos.
\end{itemize}
\subsection{Uma nota histórica}
  Algumas dessas informações podem ser vistas com mais detalhe do livro de espaços métricos
de Jean Cerqueira, do IME. 

  Para um contexto temporal, em 1906, Maurice Fréchet publicou sua tese de doutorado, nomeada
``Sus quelques du calcul functionnel''. A seguir, em 1910, David Hilbert faz sua tentativa de axiomatizar
as ideias vistas na tese de Fréchet e, dois anos depois, Felix Hausdorff, trabalhando no conceito de separação de pontos
a partir do ponto de vista de conjuntos, acaba contribuindo com essa axiomatização. Antes disso, Henri Poincaré, havia sistematizado as ideias
de forma prototípica. Outro nome mencionável é o de Pavel Urysohnn, responsável por aprofundar-se na parte da separação de pontos.

\subsection{Exercício 1 da Lista 1}
  Considere \(d:\mathbb{R}^{2}\times \mathbb{R}^{2}\rightarrow [0, \infty) \) definida por 
    \[
      d((x_{1}, x_{2}), (y_{1}, y_{2})) = |x_{2}-y_{2}|.
    \]
  Então, d não é uma métrica. Chequemos este fato. 

  Com efeito, considere x um número real qualquer. Assim, 
    \[
      d((1, x), (2, x)) = |x-x| = 0.
    \]
  No entanto, (1, x) não é igual a (2, x), ou seja, já falha logo na primeira condição de métrica!
Portanto, conclui-se que d não pode ser métrica. Analogamente, definindo \(d':\mathbb{R}^{2}\times \mathbb{R}^{2}\rightarrow [0, \infty)\) por 
 \(d'((x_{1}, x_{2}), (y_{1}, y_{2}))=|x_{1}-y_{1}|\), ela também não será métrica.

 \textbf{Obs.:} Definiremos, ainda essa aula, a métrica \(d_{s}:\mathbb{R}^{2}\times \mathbb{R}^{2}\rightarrow [0, \infty)\) por 
   \[
     d_{s}((x_{1}, x_{2}), (y_{1}, y_{2})) = |x_{1}-y_{1}|+|x_{2}-y_{2}|,
   \]
   a qual torna \((\mathbb{R}^{2}, d_{s})\) num espaço métrico. Mostremos isso. Os itens \hyperlink{def_metric}{(i), (ii) e (iii)}
da definição de métrica estão trivialmente cumpridos. Para a desigualdade triangular, observe que 
\begin{align*}
  d((x_{1}, x_{2}), (y_{1}, y_{2})) &= |x_{1}-y_{1}| + |x_{2}-y_{2}| \\
                                    &\leq |x_{1}-z_{1}| + |z_{1}-y_{1}| + |x_{2}-z_{2}| + |x_{2}-y_{2}| \\
                                    &=d((x_{1}, x_{2}), (z_{1}, z_{2})) + d((z_{1}, z_{2}), (y_{1}, y_{2}))
\end{align*}
\subsection{Espaço de funções contínuas}
   Considere o intervalo I=[0, 1] e tome o conjunto 
     \[
       \mathcal{C}(I, \mathbb{R})\coloneqq \{f:I\rightarrow \mathbb{R}: f \text{ contínua}\}.
     \]
     Então, são métrica em \(\mathcal{C}(I, \mathbb{R})\) as funções \(d:\mathcal{C}(I, \mathbb{R})\rightarrow [0, \infty), (f, g)\mapsto \sup_{x\in I}\{|f(x)-g(x)|\}\)
e \(\rho :\mathcal{C}(I, \mathbb{R})\times \mathcal{C}(I, \mathbb{R})\rightarrow [0, +\infty)\) definida por 
  \[
    \rho(f, g) = \int_{0}^{1}|f(x)-g(x)|dx,\quad f, g\in \mathcal{C}(I, \mathbb{R}).
  \]
    Mostremos que elas são \hyperlink{def_metric}{métricas}, começando pela d.

    Observe que, se \(d(f, g) = 0,\), então 
      \[
        |f(y)-g(y)|\leq \sup_{x\in I}\{|f(x)-g(x)|\}\quad \forall y\in I.
      \]
    Logo, \(|f(y) - g(y)| = 0\) para todo y em I, garantindo que f e g são as mesmas.
    Para a desigualdade triangular, observe que, para qualquer y em I, e h em \(\mathcal{C}(I, \mathbb{R}),\) vale que
    \[
      |f(y)\pm h(y) - g(y)|\leq |f(y) - h(y)| + |h(y) - g(y)|.
    \]
    Tomando o supremo, obtemos 
      \[
        d(f, g)\leq d(f, h) + d(h, g).
      \]
    Assim, \((\mathcal{C}(I, \mathbb{R}), d)\) é um espaço métrico.

    Agora, analisando a questão de \(\rho\), se \(\rho (f, g) = 0,\) então 
      \[
        \int_{0}^{1}|f(x)-g(x)|dx = 0
      \]
    Como o Teorema da Conservação de Sinal implicaria em \(\int_{0}^{1}|f(x)-g(x)|dx > 0\) se \(|f(x)-g(x)| > 0\) para algum x em I.
    Logo, \(|f(x)-g(x)|=0\) para todo x em I. Para mostrar a desigualdade triangular, considere \(f, g, h\in \mathcal{C}(I, \mathbb{R}).\)
    Temos, para todo \(x\in I\), 
      \[
        |f(\overbrace{x)-g(}^{\pm h(x)}x)|\leq |f(x) - h(x)| + |h(x) - g(x)|
      \]
    Integrando os dois lados da desigualdade, 
      \[
        \int_{0}^{1}|f(x)-g(x)|\leq \int_{0}^{1}|f(x) - h(x)|dx + \int_{0}^{1}|h(x) - g(x)|dx,
      \]
      de onde segue a desigualdade triangular. Portanto, \((\mathcal{C}(I, \mathbb{R}), \rho)\) é um espaço métrico.
\subsection{Espaço euclidiano n-dimensional}
  Seja n um natural, \(n\geq 1.\) O espaço euclidiano é 
    \[
      \mathbb{R}^{n}\coloneqq \{(x_{1}, \cdots, x_{n}): x_{i}\in \mathbb{R}, i = 1, 2, \cdots, n\},
    \]
    munido da métrica usual/euclidiana, definida por 
      \[
        d(x, y) = \biggl(\sum\limits_{i=1}^{n}(x_{i}-y_{i})^{2}\biggr)^{\frac{1}{2}},
      \]
      em que \(x=(x_{1}, x_{2}, \cdots, x_{n})\) e \(y = (y_{1}, y_{2}, \cdots, y_{n}\) são elementos de \(\mathbb{R}^{n}.\)
  Sobre \(\mathbb{R}^{n}\), podemos considerar duas outras métricas, a métrica da soma e do máximo. Definimo-las, respectivamente, por
 \begin{align*}
   &d_{s}(x, y)\coloneqq \sum\limits_{i=1}^{n}|x_{i}-y_{i}|\\
   &d_{m}(x, y)\coloneqq \max\{|x_{i}-y_{i}|, i = 1, 2, \cdots, n\},
 \end{align*}
 para  \(x=(x_{1}, x_{2}, \cdots, x_{n})\) e \(y = (y_{1}, y_{2}, \cdots, y_{n}\) são elementos de \(\mathbb{R}^{n}.\) Vamos mostrar que a primeira d é métrica.
 
  Pra começar, não é difícil ver que os itens \hyperlink{def_metric}{(i) e (ii)} da definição são satisfeitos. Para mostrar a desigualdade triangular, será necessário
utilizar Cauchy-Schwarz: 
\begin{lemma*}[Desigualdade de Cauchy-Schwars]Dados \((x_{1}, \cdots, x_{n})\) e \((y_{1}, \cdots, y_{n})\), vale a desigualdade
  \[
    \hypertarget{cauchy_schwarz}{\sum\limits_{i=1}^{n}|x_{i}\cdot y_{i}|\leq \biggl(\sum\limits_{i=1}^{n}|x_{i}|^{2}\biggr)^{\frac{1}{2}}\biggl(\sum\limits_{i=1}^{n}|y_{i}|^{2}\biggr)^{\frac{1}{2}} }
  \]
\end{lemma*}
\begin{proof*}
  Observe que, dados \(x, y\in \mathbb{R}\), vale que \(2xy\leq x^{2} + y^{2}\).
  Temos, para \(x=(\sum\limits_{i=1}^{n}|x_{i}|^{2})^{\frac{1}{2}}\) e \(y=(\sum\limits_{i=1}^{n}|y_{i}|^{2})^{\frac{1}{2}}\). Com isso, aplicando a desigualdade vista para
  \(\frac{|x_{i}|}{x}\) e \(\frac{|y_{i}|}{y}\), ganhamos 
    \[
      2\frac{|x_{i}||y_{i}|}{xy}\leq \frac{|x_{i}|^{2}}{x^{2}} + \frac{|y_{i}|^{2}}{y^{2}}.
    \]
    Somando de \(i=1, \cdots, n,\) obtemos 
      \[
        \frac{2}{xy}\sum\limits_{i=1}^{n}|x_{i}||y_{i}| \leq \frac{1}{x^{2}}\sum\limits_{i=1}^{n}|x_{i}|^{2} + \frac{1}{y^{2}}\sum\limits_{i=1}^{n}|y_{i}|^{2} = 1 + 1 = 2.
      \]
    Portanto, isolando a soma à esquerda, temos 
      \[
        \sum\limits_{i=1}^{n}|x_{i}||y_{i}|\leq \frac{2}{2}\cdot xy = xy.\text{\qedsymbol}
      \]
    \end{proof*}
  A desigualdade triangular seguirá do seguinte 
  \begin{align*}
    \bigl[d(\overline{x}, \overline{y})\bigr]^{2} = \sum\limits_{i=1}^{n}(x_{i}-y_{i})^{2} &=\sum\limits_{i=1}^{n}(x_{i}-z_{i}+z_{i}-y_{i})^{2}\\
                                                                                           &=\sum\limits_{i=1}^{n}(x_{i}-z_{i})^{2} + 2\sum\limits_{i=1}^{n}(x_{i}-z_{i})(z_{i}-y_{i}) + \sum\limits_{i=1}^{n}(z_{i}-y_{i})^{2}\\
                                                                                           &\leq \sum\limits_{i=1}^{n}(x_{i}-z_{i})^{2} + 2\biggl(\sum\limits_{i=1}^{n}|x_{i}-z_{i}|^{2}\biggr)^{\frac{1}{2}}\biggl(\sum\limits_{i=1}^{n}|z_{i}-y_{i}|^{2}\biggr)^{\frac{1}{2}} + \sum\limits_{i=1}^{n}(z_{i}-y_{i})^{2}\\
                                                                                           &=\bigl[d(\overline{x}, \overline{y})\bigl]^{2} + 2 d(\overline{x}, \overline{z})d(\overline{z}, \overline{y}) + \bigl[d(\overline{z}, \overline{y})\bigr]^{2} \\
                                                                                           & = (d(\overline{x}, \overline{z})+d(\overline{z}, \overline{y}))^{2}.
  \end{align*}
  O fim da prova da desigualdade triangular é um exercício.
\newpage

\section{Aula 03 - 17/08/2023}
\subsection{Motivações}
\begin{itemize}
  \item Métricas similares;
  \item Produtos de Espaços Métricos;
  \item Espaços vetoriais normados;
  \item Desigualdade de Hölder e de Minkowski.
\end{itemize}
\subsection{Similaridade de Métricas}
 \begin{def*}
   Seja M um conjunto não vazio. Duas métricas d e \(\rho \) em M são similares se existem \(c_{1}, c_{2} > 0\) tais que 
     \[
       c_{1}d(x, y)\leq \rho (x, y)\leq c_{2}d(x, y)
     \]
 \end{def*}
\begin{example}
  Seja d a métrica usual e \(\delta  \) a métrica discreta em \(\mathbb{R}\). Para todo c positivo, existem
x, y em \(\mathbb{R}\) tais que \(d(x, y) > c\delta (x, y)\), ou seja, não vale \(d(z, w)\leq c\delta (z, w)\) para todos
z, w reais. 

  De fato, dado \(c > 0\), tome \(x=2c\) e \(y=c-1.\) Tem-se \(d(x, y) = c + 1\) e \(\delta (x, y) = 1\). Logo, não são similares.
\end{example}
\begin{prop*}
  Para quaisquer x, y em \(\mathbb{R}^{n},\) vale a desigualdade 
    \[
      d_{m}(x,y)\leq d(x,y)\leq d_{s}(x, y)\leq nd_{m}(x,y)
    \]
\end{prop*}
\begin{proof*}
  Seja \(x=(x_{1}, x_{2}, \cdots, x_{n}), y = (y_{1}, \cdots, y_{n})\). Temos: 
    \[
      d_{m}(x, y) = \max \biggl\{(|x_{i}-y_{i}|^{2})^{\frac{1}{2}}\biggr\}\leq \biggl(\sum\limits_{i=1}^{n}|x_{i}-y_{i}|^{2}\biggr)^{\frac{1}{2}} = d(x, y)
    \]
  e também \(d_{s}(x, y) = \sum\limits_{i=1}^{n}|x_{i}-y_{i}|\leq n\max \biggl\{|x_{i}-y_{i}|\biggr\}\),
  provando a desigualdade do meio  
  \[
    (d(x,y))^{2} = \sum\limits_{i=1}^{n}(x_{i}-y_{i})^{2} = \underbrace{\biggl(\sum\limits_{i=1}^{n}|x_{i}-y_{i}|\biggr)}_{d_{s}(x, y)^{2}} - A,\quad A\geq 0,
  \]
  de onde segue que \((d(x, y))^{2}\leq (d_{s}(x, y))^{2} - A\) e prova a desigualdade. \qedsymbol
\end{proof*}
\subsection{Produto de Espaços Métricos}
Considere \((M_{1}, d_{2}), (M_{2}, d_{2}), \cdots, (M_{n}, d_{n})\) espaços métricos. Para \(M=\Pi_{i=1}^{n}M_{i}\) e definimos as métricas anteriores
\begin{itemize}
  \item \(d(x, y) = \biggl\{\sum\limits_{i=1}^{n}[d_{i}(x_{i}, y_{i})]^{2}\biggr\}^{\frac{1}{2}}\)
  \item \(d_{s}(x, y) = \sum\limits_{i=1}^{n}d_{i}(x_{i}, y_{i})\);
  \item \(d_{max}(x, y) = \max\{d_{i}(x_{i}, y_{i})\}\).
\end{itemize}
\begin{example}
  Tome \(M = \mathcal{C}(I, \mathbb{R})\times \mathbb{R}\). Se \(x = (f, s), y=(h, t)\), então 
    \[
      d_{m}(x, y) = \max \biggl\{\rho (f, h), d(s, t)\biggr\}
    \]
\end{example}
\subsection{Espaços Vetoriais Normados}
\begin{def*}
  Dado V um espaço vetorial, uma norma em V é uma função \(||\cdot ||:V\rightarrow [0, +\infty)\) que satisfaz
 \begin{itemize}
   \item[i)] \(||\vec{v}|| = 0 \Longleftrightarrow \vec{v}=0\);
   \item[ii)] \(||\lambda \vec{v}|| = |\lambda |||\vec{v}||\);
   \item[iii)] \(||\vec{v}+\vec{w}||\leq ||\vec{v}|| + ||\vec{w}||\).
 \end{itemize}
\end{def*}
\begin{example}
  Seja \((\mathbb{R}^{2n}, +, \cdot )\) é um espaço vetorial e \(||\cdot ||:\mathbb{R}^{n}\rightarrow [0, \infty)\) dada por 
    \[
      x\mapsto ||x|| = \sqrt[]{\sum\limits_{i=1}^{n}x_{i}^{2}}
    \]
\end{example}
\begin{example}
  Considere \((\mathcal{C}(I, \mathbb{R}), +, \cdot )\) e defina \(||f||_{\infty} = \max \biggl\{|f(x)|: x\in I\biggr\}, f\in \mathcal{C}(I, \mathbb{R}).\) Isto define uma norma em
 \(\mathcal{C}(I, \mathbb{R}).\) 

  Note que \(||f||_{\infty} = 0 = \max\{|f(x)|\}.\) Logo, \(|f(y)| = |f|_{\infty} = 0\) para todo y em I e \(f(y) = 0\) para todo y em I.
Isso demonstra a primeira propriedade. Para a segunda propriedade, 
  \[
    ||\lambda f||_{\infty} = \max\{|\lambda f(x)|: x\in I\} = |\lambda |||f||_{\infty}
  \]
  pelas propriedades de módulo. Por fim, temos 
    \[
      ||f+g||_{\infty} = \max\{|f(x)+g(x)|: x\in I\}\leq \max\{|f(x)| + |g(x)|: x\in I\}\leq \max\{|f(x)|\} + \max\{|g(x)|\}\leq ||f||_{\infty} + ||g||_{\infty}.
    \]
\end{example}
\begin{example}
  Definiremos a norma p em \(\mathbb{R}^{n}\). Se \(p=\infty,\) coloquemos 
    \[
      ||x||_{\infty} = \sup\{|x_{i}|: 1\leq i\leq n\}.
    \]
  Se \(p\neq\infty\), definimos 
    \[
      ||x||_{p} = \biggl(\sum\limits_{i=1}^{n}|x_{i}|^{p}\biggr)^{\frac{1}{p}}
    \]
\end{example}
\subsection{A Desigualdade de Minkowski}
\begin{lemma*}
  Se \(p, q\in (1, \infty)\) é tal que \(\frac{1}{p} + \frac{1}{q} = 1\) e \(a, b\in [0, \infty),\) então 
    \[
      a^{\frac{1}{p}} + b^{\frac{1}{q}}\leq \frac{a}{q} + \frac{b}{q}.
    \]
\end{lemma*}
\begin{proof*}
  Se \(b > 0\), então \(\frac{a^{\frac{1}{p}}+b^{\frac{1}{q}}}{b} = \frac{a}{bp} + \frac{1}{q}\), isto é, 
    \[
      \biggl(\frac{a}{b}\biggr)^{\frac{1}{p}}\leq \frac{a}{bp} + \frac{1}{q}.
    \]
    Tomando \(z=\frac{a}{b}\) e \(\alpha =\frac{1}{p},\) segue que \(t^{\alpha }\leq \alpha t + 1 - \alpha.\) Com isso, defina 
  \(f_{\alpha }:\mathbb{R}^{+}\rightarrow \mathbb{R}\) por 
    \[
      f_{\alpha }(t) = \alpha t +1 - \alpha - z^{\alpha }.
    \]
    Segue que \(f_{\alpha }(t)\geq 0\) pela sua derivada. \qedsymbol
\end{proof*} 
\begin{lemma*}[Desigualdade de Hölder]
  Se \(p\in(1, \infty), q\in (1, \infty)\) é tal que \(\frac{1}{p}+\frac{1}{q}=1\) e \(a, b\in [0, \infty),\) então 
    \[
      \sum\limits_{i=1}^{n}|x_{i}y_{i}|\leq \biggl[\sum\limits_{i=1}^{n}|x_{i}|^{p}\biggr]^{\frac{1}{p}}\biggl[\sum\limits_{i=1}^{n}|y_{i}|^{q}\biggr]^{\frac{1}{q}}
    \] 
  para todo \(x=(x_{1}, \cdots, y_{n}), y=(y_{1}, \cdots, y_{n}) \in \mathbb{R}^{n}.\)
\end{lemma*}
\begin{proof*}
  Se x = 0 ou y = 0, a desigualdade é trivial. Se \(x\neq0\) e \(y\neq0\), então defina 
    \[
      a_{j} = \frac{|x_{j}|^{p}}{\sum\limits_{i=1}^{n}|x_{i}|^{p}}\quad b_{j}=\frac{|y_{j}|^{q}}{\sum\limits_{i=1}^{n}|y_{i}|^{q}}.
    \] 
  Observamos que \(\sum\limits_{j=1}^{n}a_{j} = \sum\limits_{j=1}^{n}b_{j}=1.\) Aplicando a desigualdade de Young, 
    \[
    a_{j}^{\frac{1}{p}}b_{j}^{\frac{1}{q}} = \frac{|x_{j}y_{j}|}{\biggl[\sum\limits_{i=1}^{n}|x_{i}|^{p}\biggr]^{\frac{1}{p}}\biggl[\sum\limits_{i=1}^{n}|y_{i}|^{q}\biggr]^{\frac{1}{q}}}\leq \frac{1}{p}a_{j} + \frac{1}{q}b_{j},
    \]
  para \(j=1, \cdots, n.\) Assim, 
    \[
      \frac{\sum\limits_{j=1}^{n}|x_{j}y_{j}|}{\biggl[\sum\limits_{i=1}^{n}|x_{i}|^{p}\biggr]^{\frac{1}{p}}\biggl[\sum\limits_{i=1}^{n}|y_{i}|^{q}\biggr]^{\frac{1}{q}}}\leq \frac{1}{p} + \frac{1}{q}=1,
    \]
  donde segue a desigualdade. \qedsymbol
\end{proof*}
\begin{prop*}
  Se \(p\in[1, \infty)\), então 
    \[
      \biggl[\sum\limits_{i=1}^{n}|x_{i}+y_{i}|^{p}\biggr]^{\frac{1}{p}}\leq \biggl[\sum\limits_{i=1}^{n}|x_{i}|^{p}\biggr]^{\frac{1}{p}} + \biggl[\sum\limits_{i=1}^{n}|y_{i}|^{p}\biggr]^{\frac{1}{p}},
    \]
  para todo \(x=(x_{1}, \cdots, x_{n}), y=(y_{1}, \cdots, y_{n})\in \mathbb{R}^{n}\).
\end{prop*}
\begin{proof*}
  Os casos \(p=1, \infty\) são deixados como exercício. Se \(p\in(1, \infty),\) então 
    \[
      \biggl[\sum\limits_{i=1}^{n}|x_{i}+y_{i}|^{p}\biggr]^{\frac{1}{p}}\leq \biggl[\sum\limits_{i=1}^{n}(|x_{i}|+|y_{i}|)^{p}\biggr]^{\frac{1}{p}}.
    \]
  Podemos escrever 
    \[
      (|x_{i}|+|y_{i}|)^{p} = (|x_{i}|+|y_{i}|)^{p-1}|x_{i}|+(|x_{i}|+|y_{i}|)^{p-1}|y_{i}|,\quad i = 1, \cdots, n.
    \]
    Somando os elementos à esquerda da desigualdade anterior, obtemos 
      \[
        \sum\limits_{i=1}^{n}(|x_{i}|+|y_{i}|)^{p} = x_{n}+y_{n}
      \]
    com 
      \[
        x_{n} + y_{n}\coloneqq \sum\limits_{i=1}^{n}(|x_{i}|+|y_{i}|)^{p-1}|x_{i}| + \sum\limits_{i=1}^{n}(|x_{i}|+|y_{i}|)^{p-1}|y_{i}|.
      \]
    Aplicando a Desigualdade de Hölder, temos 
   \begin{align*}
     x_{n}&\leq \biggl[\sum\limits_{i=1}^{n}|x_{i}|^{p}\biggr]^{\frac{1}{p}}\biggl[\sum\limits_{i=1}^{n}(|x_{i}|+|y_{i}|)^{(p-1)q}\biggr]^{\frac{1}{q}}\\
          &\leq \biggl[\sum\limits_{i=1}^{n}|x_{i}|^{p}\biggr]^{\frac{1}{p}}\biggl[\sum\limits_{i=1}^{n}(|x_{i}|+|y_{i}|)^{p}\biggr]^{\frac{1}{q}}.
   \end{align*}
   De forma análoga, temos 
     \[
       y_{n}\leq \biggl[\sum\limits_{i=1}^{n}|y_{i}|^{p}\biggr]^{\frac{1}{p}}\biggl[\sum\limits_{i=1}^{n}(|x_{i}|+|y_{i}|)^{p}\biggr]^{\frac{1}{q}}.
     \]
  Portanto, 
    \[
      \sum\limits_{i=1}^{n}(|x_{i}|+|y_{i}|)^{p} = x_{n} + y_{n}\leq \cdots.\quad\text{\qedsymbol}
    \]
\end{proof*}
\begin{example}
  Dado \(p\in[0, \infty], (\mathbb{R}^{n}, d_{p})\) é um espaço métrico, em que \(d_{p}\) está representando a métrica induzida por \(||\cdot ||_{p}.\) Então, para \(p=2\),
recuperamos o espaço euclidiano n-dimensional.
\end{example}
\newpage

\section{Aula 04 - 21/08/2023}
\subsection{Motivações}
\begin{itemize}
  \item Subespaços métricos e distância entre conjuntos;
  \item Conjuntos limitados, abertos e fechados;
  \item A topologia de espaços métricos.
\end{itemize}
\subsection{Subespaços Métricos e Distância entre Conjuntos}
\begin{def*}
  Seja (X, d) um espaço métrico e \(M\subseteq{X}.\) Então, \(d|_{M}M\times M:\rightarrow \mathbb{R}\)
define uma métrica, chamada métrica induzida em M. Isso faz de \((M, d|_M)\) um subespaço métrico. \(\square\)
\end{def*}
  Além da distância entre pontos, pode-se falar da distância entre um ponto e um subconjunto do espaço métrico
e da distância entre dois subconjuntos de um espaço métrico. Para isso, considere \((X, \rho )\) um espaço métrico, 
 \(x\in X\) e \(E, F\subseteq{X}.\) Definimos, então,
\begin{align*}
  &d(x, E)\coloneqq \inf\{\rho (x, e): e\in E\}\\
  &d(E, F)\coloneqq \inf\{d(e, F): e\in E\}.
\end{align*}
\textbf{Observação:} O nome ``distância'', aqui, não é sinônimo de métrico. De fato, há um exercício na lista que mostra que a distância entre conjuntos
\textbf{NÃO} é simétrica, ou seja, não define uma métrica.
\begin{prop*}
  Seja \((X, \rho )\) um espaço métrico e \(E\subseteq{X}.\) Então, 
    \[
      |d(x, E) - d(y, E)|\leq \rho (x, y) \quad \forall x, y\in X.
    \]
\end{prop*}
\begin{proof*}
  Note que, para todo e em E, 
    \[
      d(x, E)\leq \rho (x, e)\leq \rho (x, y) + \rho (y, E).
    \]
  Logo, para todo \(x, y\in X\), 
    \[
      d(x, E)\leq \rho (x, y) + d(y, E).
    \]
  Assim, temos 
    \[
      d(x, E) - d(y, E)\leq \rho (x, y).
    \]
  Analogamente, 
    \[
      d(y, E) - d(x, E)\leq \rho (x, y)
    \]
  Portanto, 
    \[
      |d(x, E) - d(y, E)|\leq \rho (x, y).\quad\text{\qedsymbol}
    \]
\end{proof*}
\begin{example}
  Considere \((\mathbb{R}, |\cdot |), A = (-1, 0]\) e x = -2. Por definição, 
    \[
      d(x, A) = \inf\{|-2-a|: a\in A\} = \inf\{|2+a|: a\in A\} = 1\quad [d(-2, 1+\varepsilon )= 1 +\varepsilon \forall \varepsilon >0]
    \]
  No entanto, \(d(-2, a) > 1\) para todo a em A.
\end{example}
\begin{crl*}
   Seja \((X, \rho )\) um espaço métrico. Vale a desigualdade 
     \[
       |\rho (x, z) - \rho (y, z)|\leq \rho (x, y).
     \]
 \end{crl*}
\begin{proof*}
  Seja \(E=\{z\}\). Pela proposição, o resultado já segue. \qedsymbol 
\end{proof*}
\subsection{Topologia de Espaços Métricos}
\begin{def*}
  Seja \((X, d)\) um espaço métrico. Dado x em X e \(r > 0\), o conjunto 
    \[
      B_{r}(x)\coloneqq \{y\in X: d(x, y) < r\}
    \]
  é chamado bola aberta de centro em x e raio r. O conjunto
    \[
      D_{r}(x)\coloneqq \{y\in X: d(x, y)\leq r\}
    \]
  é chamado bola fechada de centro em x e raio r. \(\square\)
\end{def*}
\begin{example}
  Considere \((\mathbb{R}^{2}, d_{p})\), em que \(d_{p}(x, y)=||x-y||_{p}, 1\leq p\leq \infty.\)
 \begin{align*}
   &d_{2}((x_{1}, y_{1}), (x_{2}, y_{2})) = [(x_{1}-x_{2})^{2}+(y_{1}-y_{2})^{2}]^{\frac{1}{2}} \Rightarrow B_{1}(0) = \{(x, y)\in \mathbb{R}^{2}: d_{2}((0, 0), (x, y)) < 1\}\\
   &d_{\infty}((x_{1}, y_{1}), (x_{2}, y_{2})) = \max\{|x_{1} - x_{2}|, |y_{1} - y_{2}|\} \Rightarrow B_{1}(0) = \{(x, y)\in \mathbb{R}^{2}: \max\{|x|, |y|\} < 1\}\\
   &d_{1}((x_{1}, y_{1}), (x_{2}, y_{2})) = |x_{1}-y_{1}| + |x_{2} - y_{2}| \Rightarrow B_{1}(0)=\{(x, y)\in \mathbb{R}^{2}: |x|+|y| < 1\}
 \end{align*}
\end{example}
\begin{def*}
  Seja \((X, d)\) um espaço métrico e x um ponto de X. Chamamos x de ponto isolado se existe \(r > 0\) tal que \(B_{r}(x) = \{x\}. \quad\square\)
\end{def*}
\begin{example}
  Seja \(S = [0, 1]\cup \{2\}\) munido da métrica usual, induzida da reta. Temos 
    \[
      B_{\frac{1}{2}}(2)=\{y\in S: |x-2| < \frac{1}{2}\} = \{2\}.
    \]
\end{example}
\begin{example}
  Seja \(X \neq\emptyset\) e d a métrica discreta. Então, 
 \begin{itemize}
   \item[-] Todo ponto é isolado;
   \item[-] \(D_{r}(x) = \{x\}\) se \(r > 1\);
   \item[-] \(B_{1}(x) = \{x\}\) e \(D_{1}(x) = X;\)
   \item[-] \(B_{r}(x) = X\) se \(r > 1\).
  De fato, \(D_{r}(x) = \{y\in X: d(x, y)\leq r\}.\) Por isso, se r = 1, \(D_{1}(x) = X.\) Pela mesma lógica, prova-se os outros itens.
 \end{itemize}
\end{example}
\begin{example}
\begin{itemize}
  \item[i)] Dado \((\mathbb{R}, d),\) d a métrica usual, nenhum ponto é isolado e \(B_{r}(x) = (x-r, x+r), x\in \mathbb{R}, r > 0.\)
  \item[ii)] Se \(M=[0, 2]\) com métrica induzida, \(B_{1}(0) = [0, 1).\)
  \item[iii)] Se \(M=\mathbb{Z}\) com a métrica induzida, então \(B_{1}(n) = \{n\}\)
  \item[iv)] Se \(M=\{\frac{1}{n}:n\in \mathbb{N}, n\neq0\}\cup\{0\}\), com a métrica induzida, então \(B_{1}(0)\neq\{0\}\) para todo \(r>0.\) 
Neste caso, se \(n, m\neq 0\), então existe \(r > 0\) tal que \(B_{r}\biggl(\frac{1}{n}\biggr)=\{\frac{1}{n}\}\) e \(B_{r}\biggl(\frac{1}{m}\biggr)=\{\frac{1}{m}\}.\) Porém,
para qualquer \(r> 0\), seja \(n_{0}\) tal que \(0 < \frac{1}{n_{0}} < \frac{r}{2}\) e \(\frac{1}{n_{0}}\in B_{r}(0).\) Portanto, \(B_{r}(0)\neq\{0\}\) para todo \(r> 0\).
\end{itemize}
\end{example}
\begin{example}
  Considere \((C([a, b]), ||\cdot ||_{\infty}\), com \(||f||_{\infty} = \sup\{|f(x)|: x\in[a, b]\}.\) Seja
 \(h\in C([a, b])\) e \(r > 0\). Temos 
   \[
     B_{r}(h) = \{f\in C([a, b]): ||f-h||_{\infty} < r\} = \{f\in C([a, b]):\max_{x\in[a, b]}\{|f(x)-h(x)|\} < r\}.
   \]
\end{example}
\begin{def*}
  Um subconjunto \(M \neq\emptyset\) de um espaço métrico (X, d) é limitado se 
    \[
      diam(X)\coloneqq \sup\{d(x, y):x, y\in X\} < \infty.
    \]
    Neste caso, diam(X) é chamado diâmetro de X. Caso contrário, diz-se que M é ilimitado e \(diam(X) = \infty.\quad\square\)
  \end{def*}
\begin{example}
  Seja \((X, d)\) métrico. Para todo x em X e \(r>0, B_{r}(x)\) é ilimitado e, além disso, \(diam(B_{r}(x))\leq 2r\), o que segue da relação
 \(d(y, z)\leq d(y, x) + d(x, z)\leq r + r = 2r.\) Além disso, se \((X, ||\cdot ||)\) é um espaço vetorial normado e a métrica d é induzida pela norma,
 então \(diam(B_{r}(x)) = 2r.\) Com efeito, seja \(s < 2r.\) Tome y em X com \(x\neq0.\) Definimos 
   \[
     v = \frac{t}{||y||}y,
   \]
   para algum t satisfazendo \(s < 2t < 2r.\) Neste caso, \(x-v, x+v\in B_{r}(x)\)  e 
     \[
       d(x+v, x-v) = 2||v|| = 2t > s,
     \]
     ou seja, a afirmação feita está garantida.
\end{example}
\newpage

\section{Aula 05 - 24/08/2023}
\subsection{Motivações}
\begin{itemize}
  \item Propriedades do diâmetro;
  \item Exemplo de limitado;
  \item Conjuntos abertos;
  \item Propriedade dos abertos.
\end{itemize}
\subsection{Propriedades do Diâmetro}
 \begin{example}
   A função 
     \[
       \rho_{p}(x,y) = \frac{||x-y||_{p}}{1+||x-y||_{p}},\quad \forall x, y\in \mathbb{R}^{n}
     \]
     é uma métrica em \(\mathbb{R}^{n}\). Para ver isso, comece definindo \(f(t) = \frac{t}{1+t}\) e exiba que essa função é
crescente, de forma que \(f(||x-y||_{p}\leq f(||x-z||_{p}+||z-y||_{p})\). Com essa métrica, afirmamos que o seguinte ocorre: 
  \[
    \mathbb{R}^{n} = B_{1}(0)\quad\&\quad diam(\mathbb{R}^{n})\leq 1.
  \]
  A inclusão \(B_{1}(0)\subseteq{\mathbb{R}^{n}}\) é automática. Por outro lado, seja \(y\in \mathbb{R}^{n}.\) Temos 
    \[
      \rho_{p}(y, 0) = \frac{||y||_{p}}{1 + ||y||_{p}} < 1. \Rightarrow \mathbb{R}^{n}\subseteq{B_{1}(0)}.
    \]
  Portanto, \(\mathbb{R}^{n}\) é limitado no espaço métrico \((\mathbb{R}^{n}, \rho_{p}),\) mas não é limitado em \((\mathbb{R}^{n}, d_{p})\).
 \end{example}
 Vale uma observação - apesar dessa diferença entre \(\rho_{p}\) e \(d_{p}\), veremos futuramente que as duas métricas induzem a mesma
estrutura de formato do espaço - em outras palavras, a mesma topologia.
\begin{prop*}
  Seja \((X, \rho )\) um espaço métrico.
 \begin{itemize}
   \item[1)] \(E\subseteq{X}\) é limitado se, e somente se, existe \(r>0\) tal que \(E\subseteq{B_{r}(x)}\) para todo \(x\in E\).
     \item[2)] Se \(E\subseteq{X}\) é limitado e não-vazio, então 
       \[
         diam(E) = \inf\{r > 0: E \subseteq{B_{r}(x)}, x \in E\}.
       \]
 \end{itemize}
\end{prop*}
\begin{proof*}
  \(1 \Rightarrow )\) A volta é simples, segue das propriedades do supremo. Por outro lado, se E é limitado, então \(diam(E) < \infty\). Seja
 \(r=2diam(E)\) e \(B_{r}(x)\supseteq{E}\) para todo x em E. De fato, se \(e\in E\), temos 
   \[
     d(e, x)\leq diam(E) < r.
   \]

   \(2 \Rightarrow )\) Seja \(A = \{r>0: E \subseteq{B_{r}(x)}, x\in E\}\). Mostremos que \(diam(E) = \inf{(A)}.\) Se \(r > diam(E)\) e x
é um ponto de E, então 
  \[
    d(x, y)\leq diam(E) < r\quad \forall y\in E,
  \]
  logo, \(E\subseteq{B_{r}(x)}\). Assim, o intervalo \((diam(E), \infty)\subseteq{A}.\) Deste modo, \(\inf{(A)}\leq diam(E).\)
  Por outro lado, se \(r < diam(E),\) então existem \(x, y\in E\) tais que 
    \[
      d(x, y) > r,\quad y\not\in B_{r}(x),\quad \&\quad r\not\in A.
    \]
    Consequentemente, \((0, diam(E))\cap A = \emptyset.\) Portanto, \(diam(E) = \inf{(A)}.\) \qedsymbol 
\end{proof*}
\begin{def*}
  Seja \((X, d)\) um espaço métrico. Um subconjunto \(E\subseteq{X}\) é dito aberto em \((X, d)\) se, para cada
x em E, existir \(r_{x} > 0\) tal que \(B_{r_{x}}(x)\subseteq{E}.\quad\square\)
\end{def*}
\newpage

\section{Aula 06 - 29/08/2023}
\subsection{Motivações}
\begin{itemize}
  \item Propriedades dos abertos;
  \item Propriedades dos fechados.
\end{itemize}
\subsection{Propriedades de Abertos e Fechados}
\begin{example}
  Considere \((\mathbb{R}, |\cdot |)\). Com \(E = B_{\frac{1}{2}}\biggl(\frac{1}{2}\biggr) = (0, 1)\) é aberto. De fato, se \(x\in(0, 1)\),
então consideramos \(0 < r < \min\{x, 1-x\}.\) Neste caso, temos 
  \[
    B_{r}(x) = (x-r, x+r)\subseteq{(0, 1)}\quad \text{Exercício.]}
  \]
\end{example}
\begin{prop*}
  Seja \((X, \rho )\) um espaço métrico. Então,
 \begin{itemize}
   \item[a)] X e \(\emptyset\) são abertos;
   \item[b)] \(B_{r}(x)\) é aberto;
   \item[c)] A união arbitrária de abertos é um aberto;
   \item[d)] A intersecção finita de abertos é aberto.
 \end{itemize}
\end{prop*}
\begin{proof*}
  O item a é trivial.

  Para o item b, dado y em \(B_{r}(x)\), consideramos \(0 < r_{y} < r - \rho (x, y)\). Se z pertence a \(B_{r_{y}}(y)\), a desigualdade triangular implica 
    \[
      \rho (x, z)\leq \rho (x, y)\leq \rho (y, z) < \rho (x, y) + r_{y} < r.
    \]
  Logo, \(\rho_{r_{y}}(y)\subseteq{B_{r}(x)}\), ou seja, a bola aberta é aberta.

  Quanto ao c, seja \(\{A_{\lambda }\}_{\lambda \in I}\) uma coleção de abertos de X e \(A = \bigcup_{\lambda \in I}^{}{A_{\lambda }}\). Vamos mostrar que A é aberto.
  Com efeito, dado x em A, segue que \(x\in A_{\lambda}\) para algum \(\lambda \in I.\) Como \(A_{\lambda }\) é aberto, existe \(r_{x} > 0\) tal que \(B_{r_{x}}(x)\subseteq{A_{\lambda}}\subseteq{A}.\)
  Destarte, A é aberto.

  Finalmente, para o item d, tome \(\{A_{i}\}_{i=1}^{n}\) uma coleção finita de abertos em X e seja \(A = \bigcap_{i=1}^{n}{A_{i}}\). Seja \(x\in A\). Temos x pertencente a \(A_{i}\) para algum \(i=1, 2, \cdots, n\).
Como \(A_{i}\) é sempre aberto, existe \(r_{i} > 0\) tal que \(B_{r_{i}}(x)\subseteq{A_{i}}.\) Fazendo \(r=\min\{r_{i}: 1\leq i\leq n\}\), segue que 
  \[
    B_{r}(x)\subseteq{A_{i}},\quad \forall i=1, \cdots, n.
  \]
Deste modo, 
  \[
    B_{r}(x)\subseteq{\bigcap_{i=1}^{n}{A_{i}}} = A.
  \]
  Portanto, A é aberto. \qedsymbol
\end{proof*}
\begin{def*}
  Um subconjunto \(F\subseteq{X}\) é dito fechado em \((X, d)\) se \(F^{c}\) é aberto em \((X, d).\quad\square\)
\end{def*}
  Para proceder, vale a pena relembrar uma das formas mais úteis de manipulação de operações de conjuntos, as chamadas Leis de deMorgan. Com a forma que abertos e fechados foram definidos,
torna-se fundamental entender como complementares agem. Sem mais delongas, as Leis de deMorgan são as seguintes: 
  \[
    \hypertarget{demorgan}{\boxed{\biggl(\bigcap_{i\in I}^{}{A_{i}}\biggr)^{c} = \bigcup_{i\in I}^{}{A_{i}^{c}}}\quad\boxed{\biggl(\bigcup_{i\in I}^{}{A_{i}}\biggr)^{c} = \bigcap_{i\in I}^{}{A_{i}^{c}}}}
  \]
  \textbf{\underline{Observação}:} Pessoalmente, gosto de pensar que a ação do complementar é ``virar'' o sinal da operação de cabeça pra baixo, então \(\cup\) vira \(\cap\) e vice-versa.
\begin{prop*}
  Seja \((X, \rho )\) um espaço métrico. Então, 
 \begin{itemize}
   \item[a)] \(X, \emptyset\) são fechados;
   \item[b)] \(D_{r}(x)\) é fechado;
   \item[c)] A intersecção arbitrária de fechados é um conjunto fechado;
   \item[d)] A união finita de fechados é um fechado.
 \end{itemize}
\end{prop*}
\begin{proof*}
  As provas de todos os itens, menos do b, seguem a partir das \hyperlink{demorgan}{Leis de DeMorgan}

  Para o item b, dado y fora de \(D_{r}(x)\), vale \(d(x, y) < r.\) Seja \(0 < r_{y} < \rho (x, y) - r.\) Se z pertence a \(B_{r_{y}}(y),\), pela desigualdade triangular, 
    \[
      \rho (x, z)\geq \rho (x, y) - \rho (y, z)\geq \rho (x, y) - r_{y} > r.
    \]
    Logo, \(z\not\in D_{r}(x)\) e \(B_{r_{y}}(y)\subseteq{(D_{r}(x))^{c}}.\) \qedsymbol
\end{proof*}
\begin{example}
  Seja \(\mathbb{R}\) com a métrica usual. Temos:
 \begin{align*}
   &[x-r, x+r]=\bigcap_{n=1}^{\infty}{(x-r-\frac{1}{n}, x+r+\frac{1}{n})}\\
   &\{x\} = \bigcap_{i=1}^{n}{(x-\frac{1}{n}, x+\frac{1}{n})}.
 \end{align*}
\end{example}
\newpage

\section{Aula 07 - 12/09/2023}
\subsection{Motivações} 
\begin{itemize}
  \item Fecho e bordo de conjuntos;
  \item Conjuntos densos;
  \item Pontos isolados e de acumulação.
\end{itemize}
\subsection{Fechos e Bordos}
 \begin{def*}
   O interior \({E}^{\mathrm{o}}\) de \(E\subseteq{X}\) é a união de todos os abertos de \((X, \rho )\) contidos em E. \(\square\)
 \end{def*}
\begin{def*}
  O fecho \(\overline{E}\) de \(E\subseteq{X}\) é a intersecção de todos os fechados contendo E. \(\square\)
\end{def*}
Note que \({E}^{\mathrm{o}}\coloneqq \{x\in E: \exists r > 0, B_{r}(x)\subseteq{E}\}\) e \(\overline{E}\coloneqq \{x\in M: B_{r}(x)\cap E \neq\emptyset, \forall r>0\}\).
Além disso, por definição, \({E}^{\mathrm{o}}\subseteq{E}\subseteq{\overline{E}}\).
\begin{def*}
  Um conjunto \(E\subseteq{X}\) é dito denso se \(\overline{E}=X.\square\)
\end{def*}
\begin{def*}
  Um ponto \(x\in X\) é dito um ponto de fronteira de E se 
    \[
      B_{r}(x)\cap E \neq\emptyset\quad\& B_{r}(x)\cap E^{c}\neq\emptyset,\quad \forall r>0.
    \]
    O conjunto de todos os pontos de fronteira de E é denotado por \(\partial E.\square\)
\end{def*}
\begin{example}
  Para \((\mathbb{R}, |\cdot |),\) temos \(\mathbb{Q}\subseteq{\mathbb{R}}\) denso em \(\mathbb{R}\).
  De fato, \(\mathbb{Q}=\mathbb{R}\), pois, para todo x em \(\mathbb{R}\) e \(r > 0\), 
    \[
      B_{r}(x)\cap \mathbb{Q}\neq\emptyset,
    \]
    o que implica em \(x\in \overline{\mathbb{R}}\)
\end{example} 
\begin{example}
 \begin{itemize}
   \item[1)] Vale que \(E = \overline{E}\) se, e somente se, E é fechado. Afinal, se E é fechado, o menor aberto
  que o contém é ele mesmo, ou seja, \(\overline{E}\subseteq{E}\). O outro lado da inclusão já foi comentado anteriormente.
   \item[2)] O interior da bola fechada nem sempre é a bola aberta;
   \item[3)] Para d discreta, \(B_{r}(x) = \overline{B_{r}}(x).\)
 \end{itemize}
\end{example}
\begin{def*}
  Se \((X, \rho )\) é um espaço métrico, uma sequência em X é \(\varphi:\mathbb{N}\rightarrow X\) com \(\varphi(n) = x_{n}\) e
\(Im\varphi = \{x_{n}\}\). Diremos que uma sequência \(\{x_{n}\}\) em \((X, \rho )\) converge se existe \(x\in X\)
tal que \(\{\rho(x_{n}, x)\}\) converge para zero em \(\mathbb{R}\). Escrevemos \(x_{n}\rightarrow x\) para denotar que \(\{x_{n}\}\) 
converge para x. \(\square\)
\end{def*}
\begin{example}
  Considere \(\mathcal{B}(I; \mathbb{R})\), em que \(I = [0, 1]\), com a norma 
    \[
      ||f||_{1}\coloneqq \int_{0}^{1}|f(x)|dx.
    \]
    Seja \(\{f_{n}\}\) em \(\mathcal{B}(I; \mathbb{R})\) dada por \(f_{n}(x) =x^{n}, x\in[0,1]\). Para cada \(x\in[0, 1)\), vale
 \(\lim_{n\to \infty}x_{n} = 0\). Assim, \(\lim_{n\to \infty}f_{n}(x) = 0, x\in[0, 1)\) e \(\lim_{n\to \infty}f_{n}(1) = 1.\) Considere \(g\in \mathcal{B}(I; \mathbb{R})\)
 dada por \(g(x) = \left\{\begin{array}{ll}
     0,\quad x\in[0, 1)\\
     1,\quad x=1.
   \end{array}\right.\)
  Para cada \(x\in I\), vale \(\lim_{n\to \infty}f_{n}(x)=g(x)\). Temos, de fato, 
    \[
      d(f_{n}, g)= \int_{0}^{1}|x^{n}|dx = \frac{x^{n+1}}{n+1}\biggl|_{0}^{1}\biggr.\longrightarrow 0,\quad n\to\infty.
    \] 
\end{example}
\begin{prop*}
  Seja \((X, \rho )\) um espaço métrico e \(E\subseteq{X}\). Equivalem: 
 \begin{itemize}
   \item[a)] \(x\in \overline{E}\);
   \item[b)] \(B_{r}(x)\cap E \neq\emptyset,\quad \forall r>0\);
   \item[c)] Existe \(\{x_{n}\}\) em E tal que \(\lim_{n\to \infty}x_{n} = x.\)
 \end{itemize}
\end{prop*}
\begin{proof*}
  \(1) \Rightarrow 2):\) Assuma que existe \(r > 0\) tal que \(B_{r}(x)\subseteq{E^{c}}\). Então,
\(({E}^{\mathrm{o}})^{c}.\) Como \([({E}^{\mathrm{o}})^{c}]^{c}\) é fechado e contém E, concluímos que 
\(x\not\in \overline{E}.\) De fato, se existe \(r > 0\) tal que \(B_{r}(x)\cap E = \emptyset\), então \(x\not\in \overline{E}\). 
Assim, se \(x\in \overline{E},\) então \(B_{r}(x)\cap E \neq\emptyset\) para todo \(r > 0\).

  \(2)\Rightarrow 3):\) Se \(B_{r}(x)\cap E \neq\emptyset\) para todo \(r > 0\), então \(B_{\frac{1}{n}}(x)\cap \neq\emptyset.\) Seja 
 \(x_{n}\) um elemento de \(B_{\frac{1}{n}}(x)\cap E,\) para cada n natural. Então, \(\{x_{n}\}\rightarrow x\), visto que vale \(0\leq d(x_{n}, x) < \frac{1}{n}\rightarrow 0\) 
 quando \(n\rightarrow\infty.\)


  \(3) \Rightarrow 1):\) Se existe \(\{x_{n}\}\) em E, \(\lim_{n\to \infty}x_{n} = x\) tal que \(x\not\in \overline{E},\) então \(B_{r}(x)\subseteq{\overline{E}^{c}}\) 
para algum \(r>0.\) Devido ao fato de \(x_{n}\rightarrow x\) quando \(n\rightarrow \infty\), existe \(n_{0}\in \mathbb{N}\) tal que \(\rho(x_{n}, x) < \varepsilon \).
Em particular, para \(\varepsilon = r, \rho(x, x) < r\)  e \(x_{n}\in B_{r}(x)\subseteq{\overline{E}^{c}}\) para \(n\geq n_{0}\), um absurdo. \qedsymbol
\end{proof*}
\begin{def*}
  Um ponto \(x\in X\) é chamado de ponto de acumulação de \(E\subseteq{X}\) se 
    \[
      (B_{r}(x)\setminus\{x\})\cap E \neq\emptyset,\quad \forall r>0.
    \]
    Um ponto \(x\in E\subseteq{X}\) é chamado ponto isolado de E se existe \(r > 0\) tal que 
   \(B_{r}(x)\cap{E} = \{x\}\). O conjunto de pontos de acumulação de E, chamado conjunto derivado 
  de E, será denotado por \(E'. \square\)
\end{def*}
\begin{prop*}
  Seja \((X, \rho )\) um espaço métrico. Então:
 \begin{itemize}
   \item[a)] \(E = E^{c}\cup \partial E\);
   \item[b)] \(\overline{E} = E\cup (\partial E\cap E')\);
   \item[c)] \({E}^{\mathrm{o}} = E\setminus{\partial E}\);
   \item[d)] \(\partial E\) é um fechado.
 \end{itemize}
\end{prop*}
\begin{proof*}
a) Como \([({E}^{\mathrm{o}})^{c}]^{c} = E^{c\mathrm{o}c}={E}^{\mathrm{o}}\cup \partial E\) é fechado
e contém E, concluímos que \(\overline{E}\subseteq{{E}^{\mathrm{o}}}\cup \partial E\). 

  Por outro lado, sabe-se que \({E}^{\mathrm{o}}\subseteq{E}\subseteq{\overline{E}}\). Se \(x\in \partial E\) e \(x\not\in {E}^{\mathrm{o}}\),
então \(B_{\frac{1}{n}}(x)\cap E \neq\emptyset\) para todo n natural. Seja \(\{x_{n}\}\) tal que \(x_{n}\overbracket[0pt]{\longrightarrow}^{n\to\infty }x.\)
Temos \(x\in \overline{E}\) e segue que \({E}^{\mathrm{o}}\cup \partial E \subseteq{\overline{E}}.\) 

b) Observe que \(E\cup(\partial E\cap E')\subseteq{\overline{E}}\). Se \(x\in\overline{E}\setminus{E}\), segue que 
existe uma sequência \(\{x_{n}\}\) em E tal que \(x_{n}\overbracket[0pt]{\longrightarrow}^{n\to \infty}x.\) Logo, 
  \[
    B_{r}(x)\cap E \neq\emptyset\quad\&\quad B_{r}(x)\cap E^{c} \neq\emptyset,\quad \forall r>0,
  \]
e para todo \(x\in \partial E\cap E'.\) Portanto, \(\overline{E}\subseteq{E\cup(\partial E\cap E')}.\)

c) Segue que \(E\subseteq{{E}^{\mathrm{o}}\cup \partial E}\) dos anteriores. A partir disto, basta tomar 
a diferença dos conjuntos.

d) Segue da igualdade \(\partial E = ({E}^{\mathrm{o}}\cup ({E}^{\mathrm{o}})^{c})^{c}\).
\qedsymbol

\end{proof*}
\newpage

\section{Aula 08 - 14/09/2023}
\subsection{Motivações}
\begin{itemize}
  \item Continuidade e suas diferentes formas;
  \item Tipos de funções contínuas.
\end{itemize}
\subsection{Continuidade - Noções Prévias}
  Na reta real, as funções contínuas apresentam um papel importante tanto em análise quanto nos diferentes cálculos ao longo da graduação. Não apenas isso, mas
pode-se dizer sem exagero que formam uma parte fundamental da matemática em muitos níveis, possuindo graus de generalização. Ao estudar cálculo I, a definição é
a epsilon-delta comum:
\begin{quote}
  ``\textit{Uma função} \(f:X\rightarrow Y, X, Y\subseteq{\mathbb{R}}\) \textit{é dita contínua em x se para todo} \(\varepsilon >0\), \textit{existe um} \(\delta >0\) \textit{tal que, para }\(t\in X,\) 
    \[
      |t-x| < \delta \Rightarrow |f(x)-f(t)| < \varepsilon."
    \]
\end{quote}

  Ao entrar em cálculo de muitas variáveis, uma versão análoga de continuidade é apresentada:

\begin{quote}
  ``\textit{Uma função} \(f:X\rightarrow Y, X\subseteq{\mathbb{R}^{n}}, Y\subseteq{\mathbb{R}^{m}}\) \textit{é dita contínua em x se, para todo} \(\varepsilon >0\), \textit{existe um }\(\delta >0\)
\textit{tal que, para todo} \(y\in X\), 
  \[
    ||x-y|| < \delta \Rightarrow ||f(x)-f(y)|| < \varepsilon, 
  \]
  \textit{em que} \(||\cdot ||\) \textit{é a norma usual em} \(\mathbb{R}^{n}\) \textit{ou }\(\mathbb{R}^{m}.\)''
\end{quote}
  Em ambos os casos, se levarmos em conta a relação entre módulo/norma e intervalos da reta, no sentido de uma norma ou um módulo descrevem as bolas abertas no conjunto, temos uma noção
do que uma função contínua deve fazer - mapear intervalos abertos em abertos. Porém, observe que, na definição, a liberdade que temos é no \(\varepsilon \), ou seja, no aberto da imagem da função
e, para cada \(\varepsilon \), é preciso encontrar um \(\delta \). Em outras palavras, ela mapeia intervalos abertos em outros abertos, mas de uma forma específica: Dado um aberto na imagem da função,
é preciso encontrar um aberto em sua pré-imagem. Essa é, de fato, a caracterização que utilizaremos nesse curso. 
\subsection{Continuidade}
 Sejam \((X, \rho ), (Y, d)\) espaços métricos. Denotaremos as bolas abertas nos espaços respectivos por um superscript acima - \(B^{X}_{r}(x)\) é uma bola aberta
em X e \(B_{r}^{Y}(x)\) uma em Y.
\begin{def*}
  Diremos que uma função \(f:X\rightarrow Y, (X, \rho ), (Y, d)\) espaços métricos, é contínua se para todo \(\varepsilon >0\) existir um \(\delta = \delta (\varepsilon, p) > 0\) tal que 
    \[
      B_{\delta }^{X}(p)\subseteq{f^{-1}(B_{\delta }^{Y}(f(p))}.
    \]
    Equivalentemente, \(f(B_{\delta }^{X}(p))\subseteq{B_{\delta }^{Y}(f(p))}.\square\)
\end{def*}
  Explicitamente, se \(\rho (x, p) < \delta \), então \(x\in B_{\delta }^{X}(p).\) Logo, x pertence a \(f^{-1}(B_{\varepsilon }^{Y}(f(p)))\) e
 \(f(x)\in B_{\varepsilon }^{Y}(f(p)).\) Por definição da bola, \(d(f(x), f(p)) < \varepsilon.\) Isso pode ser reescrito na seguinte proposição:
\begin{prop*}
  Uma função \(f:X\rightarrow Y\) é contínua em \(p\in X\) se, e somente se, para todo \(\varepsilon >0\) existe
 \(\delta = \delta (\varepsilon , p) > 0\) tal que 
   \[
     \rho (x, p)<\delta  \Rightarrow d(f(x), f(p)) <\varepsilon .
   \]
\end{prop*}

  A seguir, vamos apresentar algumas outras caracterizações de continuidade - a topológica e a por sequências.
 \begin{def*}
   A caracterização topológica das funções contínuas é: \(f:X\rightarrow Y\) é continuas se, e somente se, \(f^{-1}(A)\) é aberto em X para todo aberto A em Y. \(\square\)
 \end{def*}
  Mostraremos que a caracterização topológica e a com bolas são equivalentes. Se f é contínua e U é um aberto de Y, se
\(y\in f^{-1}(U)\) e \(\varepsilon > 0\) é tal que \(B_{\varepsilon }(f(y)) \subseteq{U},\) então existe \(\delta  > 0\)
tal que \(f^{-1}(B_{\varepsilon }(f(y)))\supseteq{B_{\delta }(y)}.\) Logo, y é interior a \(f^{-1}(U).\) Como
consequência, já que y era arbitrário, \(f^{-1}(U)\) é aberto.

  Por outro lado, se \(f^{-1}(U)\) é aberto em \((X, \rho )\) sempre que U é aberto em
 \((Y, d)\), segue que, para todo x em X e \(\varepsilon >0, f^{-1}(B(f(x)))\) é aberto e contém x.
 Logo, existe \(\delta  > 0\) tal que \(B_{\delta }(x)\subseteq{f^{-1}(B_{\varepsilon }(f(x)))}\) e f é contínua em x.
 Portanto, f é contínua para todo x em X.
 \begin{example}
   A função constante \(f:X\rightarrow Y\) dada por \(f(x) = c,\) para todo x em X e algum c em Y, é contínua.
  De fato, seja A um aberto de Y. Se c pertence a A, então \(f^{-1}(A) = X\), que é aberto em X. Por 
  outro lado, se c não pertence a A, então \(f^{-1}(A) = \emptyset,\) que também é aberto em X. Portanto, f é contínua.
 \end{example}
 \begin{example}
   A função identidade \(f:X\rightarrow X\) é contínua para todo x em X. De fato, se A é um aberto de X,
como \(f^{-1}(A) = A, f^{-1}(A)\) é aberto em X.
 \end{example}
 \begin{def*}
   A caracterização por sequências de uma função contínua é: \(f:X\rightarrow Y\) é contínua em \(x\in X\) se, e somente se, dada uma sequência \(\{x_{n}\}_{n\in \mathbb{N}}\) que
converge para x (\(x_{n}\longrightarrow x)\), vale \(f(x_{n})\overbracket[0pt]{\longrightarrow}^{n\to \infty}f(x).\square\)
 \end{def*}
 Vamos verificar que a noção de continuidade que fornecemos implica na definição de sequências. Note que, como \(\rho (x_{n}, x) < \delta \)(já que a sequência converge), vale que existe um \(n_{0}\in \mathbb{N}\) tal que
 \(x_{n}\in B_{\delta }^{X}(x)\) para todo \(n\geq n_{0}\). Com isso, como f é contínua, \(B_{\delta }^{X}(x)\subseteq{f^{-1}(B_{\varepsilon }^{Y}(f(x)))}\), ou seja, \(x_{n}\in f^{-1}(B_{\varepsilon }^{Y}(f(x)))\) para todo \(n\geq n_{0}.\)
 Assim, \(f(x_{n})\in B_{\varepsilon }^{Y}(f(x))\) para todo \(n\geq n_{0}\), o que equivale a \(d(f(x_{n}), f(x)) < \varepsilon,\) ou seja, \(f(x_{n})\overbracket[0pt]{\longrightarrow}^{n\to \infty}f(x)\)
\subsection{Outras Classes de Continuidade}
 \begin{def*}
   Diremos que uma função \(f:X\rightarrow Y\) é uniformemente contínua se, para todo \(\varepsilon >0\), existe \(\delta =\delta (\varepsilon )>0\) tal que 
     \[
       \rho (x, y) < \delta \Rightarrow d(f(x), f(y)) < \varepsilon \quad \forall x, y\in X.\quad\square
     \]
 \end{def*}
  A diferença entre uma função contínua e uniformemente é sutil - enquanto que, na continuidade normal, existe um \(\delta \) diferente para cada ponto x, isso não acontece
na continuidade uniforme, ou seja, um \(\delta \) funciona para todos os pontos do espaço.
\begin{def*}
  Diremos que uma função \(f:X\rightarrow Y\) é Lipschitz contínua se, para algum \(\lambda >0\), vale 
    \[
      d(f(x), f(y))\leq \lambda \rho (x, y).\quad \forall x, y\in X.
    \]
\end{def*}
\begin{def*}
  Uma função \(f:X\rightarrow Y\) é localmente Lipschitz se, para todo x em X, existir \(r_{x}>0\) e \(\lambda_{x}>0\) tais que 
    \[
      d(f(z), f(w))\leq \rho (x, w)\lambda_{x},\quad \forall z, w\in B_{r_{x}}^{X}(x).
    \]
\end{def*}
  Tanto as funções Lipschitz quanto as localmente Lipschitz são contínuas.
 \begin{def*}
  Uma função \(f:X\rightarrow Y\) é uma imersão isométrica de X em Y se 
    \[
      d(f(x), f(y)) = \rho (x, y).
    \]
  Além disso, se \(f(X) = Y\), ou seja, f é sobrejetora, então f é uma isometria entre X e Y. \(\square\)
 \end{def*}
  Toda imersão isométrica é injetora, visto que 
    \[
      d(f(x), f(y)) = \rho (x, y) = 0
    \]
  é o mesmo que \(\rho (x, y) = 0\) que, como \(\rho \) é uma métrica, significa que \(x=y\).
  \begin{example}
    Consideremos \(\mathbb{R}^{n}\) munido da métrica euclidiana e b em \(\mathbb{R}^{n}.\) A função translação por b,
  \(f:\mathbb{R}^{n}\rightarrow \mathbb{R}^{n}\) definida por \(f(x) = x + b\) é uma imersão isométrica.
  \end{example}
 \begin{example}
   Se \(f:X\rightarrow Y\) for injetiva e definirmos em \(Y' = f(X)\) a métrica
  \(d_{f}(y_{1}, y_{2}) = \rho (f^{-1}(y_{1}), f^{-1}(y_{2}))\) para todo \(y_{1}, y_{2}\in Y'\), então
  \(f:X\rightarrow Y'\) é uma isometria.
 \end{example}
 \begin{example}
   Se \(M\subseteq{X}\) e \(\rho _{M}:M\times M\rightarrow \mathbb{R}^{+}\) é a métrica induzida, a inclusão
\(i:M\rightarrow X\) definida por \(i(x) = x\), é uma imersão isométrica.
 \end{example}
 \begin{prop*}
   Um espaço métrico \((X, \rho )\) pode ser imerso isometricamente no espaço vetorial normado \(\mathcal{B}(X, \mathbb{R})\) das funções limitadas \(f:X\rightarrow \mathbb{R}\), com a norma
  \(||f||_{\infty}=\sup\{|f(x)|\}\)
 \end{prop*}
 \begin{proof*}
   Considere \((X, \rho )\) e \((\mathcal{B}(X, \mathbb{R}), ||\cdot ||_{\infty}\) os espaços métricos no enunciado. Defina
 \(f:X\rightarrow \mathcal{B}(X, \mathbb{R})\) por \(x\mapsto f(x):X\rightarrow \mathbb{R}\) dada por 
 \(f(x)(y) = \rho (x, y)-\rho (p, y).\) Note que:
\begin{itemize}
  \item[i)] \(|f(x)(y) - f(z)(y)| = |\rho (x, y)-\rho (p, y)-\rho (z, y) + \rho (p, y)| = |\rho (x, y)-\rho (z, y)|\)
  \item[ii)] \(|f(x)(z)-f(z)(z)| = |\rho (x, z) - \rho (p, z) - \rho (z, z) + \rho (p,z)| = |\rho (x, z)|.\)
\end{itemize}
  Disto, conclui-se que 
    \[
      ||f(x)-f(z)||_{\infty} = \sup\{|f(x)(y) - f(z)(y)|\}\leq \rho (x, y)\quad\&\quad ||f(x)-f(z)||_{\infty} = \rho (x, z).
    \]
  Portanto, f é uma imersão isométrica. \qedsymbol
 \end{proof*}
\newpage

\section{Aula 09 - 19/09/2023}
  Essa aula e a próxima foram planejadas, apresentadas e ministradas pela Bruna, aluna da Thaís. Todos os créditos vão para ela.
Além disso, por convenção, trocarei a ordem das aulas 09 e 10 para melhor compreensão das mesmas.
\subsection{Motivações}
\begin{itemize}
  \item Breve revisão de continuidade;
  \item Exemplos de funções continuas;
  \item Continuidade da Transformação Linear.
\end{itemize}
\subsection{Uma breve revisão}
 \begin{def*}
   Uma função \(f:X\rightarrow Y\) é contínua em \(p\in X\) se para todo \(\varepsilon >0\) existir um
  \(\delta =\delta (p, \varepsilon)>0\) tal que \(B_{\delta }^{X}(p)\subseteq{f^{-1}(B_{\varepsilon }^{Y}(f(p))}.\)
  Equivalentemente, \(f(B_{\delta }^{X}(p))\subseteq{B_{\varepsilon }^{Y}(f(p))}.\square\)
 \end{def*}
\begin{def*}
  Uma função \(f:(X, \rho )\rightarrow (Y, d)\) é uma imersão isométrica se \(d(f(x), f(y)) = \rho (x, y).\square\)
\end{def*}
  Anteriormente, provamos que funções isométricas são contínuas.
\subsection{Exemplos de Funções Contínuas}
\subsubsection{Inclusão}
  Sejam \((X, \rho )\) um espaço métrico e \(M\subseteq{X}\) um subespaço de X. Definimos a função inclusão por 
    \[
      i:M\rightarrow X,\quad \underbrace{x}_{\text{Como elemento de M}}\longmapsto \underbrace{x}_{\text{Como elemento de X}}
    \]
  Note que \(\rho (i(x), i(y)) = \rho (x, y) = \rho_{M}(x, y)\) para todos \(x, y\in M\), tal que i é uma imersão isométrica
e, portanto, é contínua.
\subsubsection{Projeções}
  Sejam \((X_{i}, \rho_{i}), i\in \mathbb{N},\) espaços métricos. Definimos a i-ésima projeção como
 \(\pi :\prod\limits_{i=1}^{n}X_{i}\rightarrow X_{i}, \pi_{i}(x_1, x_2, \cdots, x_{n}) = x_{i}.\) A projeção
 é contínua. De fato, seja \(A\subseteq{X}\) um aberto, \(i=1, \cdots, n.\) Mostremos que \(\pi^{-1}(A)\) é aberto. 
 Com efeito, 
   \[
     \pi_{i}^{-1}(A) = X_{1}\times X_{2}\times \cdots\times X_{i-1}\times A_{i}\times X_{i+1}\times \cdots\times X_{n}.
   \]
   Sendo o produto finito de abertos um aberto, portanto, \(\pi_{i}\) é contínua.
\subsubsection{Contração Fraca}
 \begin{def*}
   Uma contração fraca é uma aplicação \(f:(X, \rho )\rightarrow (Y, d)\) tal que 
     \[
       d(f(x), f(y))\leq \rho(x, y),\quad \forall x,y \in X.\square
     \]
 \end{def*}
 Toda contração fraca é contínua. Com efeito, dado \(\varepsilon > 0\), escolhemos \(\delta = \varepsilon\). A partir
disso, obtemos 
  \[
    \rho (x, y) < \delta \Rightarrow d(f(x), f(y))\leq \rho (x, y) < \delta =\varepsilon .
  \]
  Portanto, f é contínua.
\subsubsection{Soma} 
  Sejam \(\mathbb{E}\) um espaço vetorial normado com a métrica induzida d, e considere
\((\mathbb{E}\times \mathbb{E}, D_{1})\), em que \(D_{1}\) é a métrica da soma. Definimos a soma como
\(s:\mathbb{E}\times \mathbb{E}\rightarrow \mathbb{E}, s(x, y) = x + y.\)
\textbf{Afirmação:} A soma é uma contração fraca.

  Com efeito, sejam \((x_{1}, y_{1}), (x_{2}, y_{2})\in \mathbb{E}.\) Temos
 \begin{align*}
   d(s(x_{1}, y_{1}), s(x_{2}, y_{2})) &= d(x_{1}+y_{1}, x_{2}+y_{2}) = ||(x_{1}+y_{1})-(x_{2}+y_{2})||\\
                                       &\leq ||x_{1}-x_{2}|| + ||y_{1}-y_{2}|| = d(x_{1}, x_{2}) + d(y_{1}, y_{2})\\
                                       &=D_{1}((x_{1}, y_{1}), (x_{2}, y_{2})).
 \end{align*}
Portanto, s é uma contração fraca, ou seja, é contínua.
\subsubsection{Norma}
  Sejam \(x, y\in \mathbb{E}, \mathbb{E}\) espaço vetorial normado. Temos, para \((\mathbb{R}, d)=(\mathbb{R}, |\cdot |),\)
    \[
      |||x||-||y||| < ||x-y||.
    \]
  Como \(|||x||-||y||| = d(||x||, ||y||),\) isto equivale a 
    \[
      d(||x||, ||y||)\leq ||x-y||.
    \]
  Portanto, a norma é uma contração fraca, ou seja, contínua.
\subsubsection{Multiplicação por Escalar}
  Seja \(\mathbb{E}\) um espaço vetorial normado com métrica d induzida pela norma, \((\mathbb{R}, |\cdot |)\) e \((\mathbb{R}\times \mathbb{E}, D_{1})\).
Definimos a multiplicação por escalar como 
  \[
    m:\mathbb{R}\times \mathbb{E}\rightarrow \mathbb{E},\quad (\alpha ,x)\mapsto \alpha x
  \]
\begin{def*}
  Uma função \(f:X\rightarrow Y\) é localmente Lipschitz se, para todo x em X, existir \(r_{x}>0\) e \(\lambda_{x}>0\) tais que 
    \[
      d(f(z), f(w))\leq \rho (x, w)\lambda_{x},\quad \forall z, w\in B_{r_{x}}^{X}(x).
    \]
\end{def*}
  Toda função localmente Lipschitz é contínua.

\textbf{Afirmação:} m é localmente Lipschitz.

  Com efeito, seja \(x\in \mathbb{R}\times \mathbb{E}\). Consideramos a bola \(B\coloneqq B_{r_{x}}^{\mathbb{R}\times \mathbb{E}}\), em que \(r_{x}>0\)
é tal que \(B\subseteq{\mathbb{R}\times \mathbb{E}}.\) Escolhemos \(\lambda_{x} = r_{x}+d(O, x),\) em que \(O = (0, 0)\) é a origem 
de \(\mathbb{R}\times \mathbb{E}.\) Dessa escolha, \(B_{1}=B_{\lambda_{x}}^{\mathbb{R}\times \mathbb{E}}(O)\). Sejam \((\alpha , x), (\beta, y)\in B\subseteq{B_{1}}.\)
Então, 
  \[
    D_{1}(O, (\alpha , x)) < \lambda_{x}\quad\&\quad D_{1}(O, (\beta, y)) < \lambda_{x}.
  \]
  Pela definição da métrica da soma, 
    \[
      |\alpha - 0| + d(x, 0) < \lambda_{x}\quad\&\quad |\beta - 0| + d(y, 0) < \lambda_{x},
    \]
  ou seja, \(|\alpha | + ||x|| < \lambda_{x}\) e \(|\beta |+||y||<\lambda_{x}\). Em particular, \(||x|| < \lambda_{x}\) e \(||\beta || < \lambda_{x}\). Assim, 
 \begin{align*}
   d(m(\alpha, x), m(\beta, y)) &= d(\alpha x, \beta y) = ||\alpha x - \beta y|| = ||\alpha x-\beta _{x}+\beta _{x}-\beta _{y}||\\
                                &=||x(\alpha -\beta ) + \beta (x-y)||\leq ||x(\alpha -\beta )|| + ||\beta (x-y)||\\
                                &=|\alpha -\beta |||x||+|\beta |||x-y||\leq |\alpha -\beta |\lambda_{x} + \lambda_{x}||x-y||\\
                                &=\lambda_{x}(|\alpha -\beta | + ||x-y||) = \lambda_{x}[D_{1}((\alpha , x), (\beta , y))].
 \end{align*}
 Então, \(d(m(\alpha , x), m(\beta , y))\leq \lambda_{x}[D_{1}((\alpha, x), (\beta , y))] \). Portanto, m é localmente Lipschitziana, logo, contínua.
\subsection{Continuidade da Transformação Linear}
  Sejam X, Y espaços vetoriais sobre um corpo \(\mathbb{K}\).
 \begin{def*}
   Uma transformação linear \(T:X\rightarrow Y\) é definida por:
 \begin{itemize}
   \item[a)] \(T(u+v)=T(u)+T(v),\quad \forall u, v\in X\)
   \item[b)] \(T(\alpha u) = \alpha T(u),\quad \forall \alpha \in \mathbb{K}, u\in X.\quad\square\)
 \end{itemize}
 \end{def*}
  Observe que, de (a) e (b), segue que \(T(0) = T(0u) = 0T(u) = 0, T(0) = T(u+(-u)) = T(u) + T(-u),\) ou seja,
 \(T(-u) = -T(u)\) e \(T(u-v) = T(u) + T(-v) = T(u) - T(v)\).
\begin{example}
  Para X = Y espaços vetoriais sobre \(\mathbb{R}\), um exemplo de transformação linear é 
    \[
      T[x] = \alpha x,\quad T:\mathbb{R}\rightarrow \mathbb{R},
    \]
    pois \(T[x+y] = \alpha (x+y) = \alpha x + \alpha y = T[x] + T[y]\) e \(T[\beta x] = \beta \alpha x = \beta T[x].\)
\end{example}
\begin{theorem*}
  Se X, Y são espaços vetoriais normados sobre \(\mathbb{R}\) e se \(T:X\rightarrow Y\) é uma transformação linear, então, são equivalentes:
 \begin{itemize}
   \item[a)] T é contínua;
   \item[b)] T é contínua na origem;
   \item[c)] Existe um \(k > 0\) tal que \(||T(u)|| < k||u||,\) para todo \(u\in X\);
   \item[d)] T é Lipschitziana.
 \end{itemize}
\end{theorem*}
\begin{proof*}
  (a) \(\Rightarrow \)(b) é automático.

  (b) \(\Rightarrow \) (c): Se T é contínua na origem, tome \(\varepsilon = 1.\) Com isso, existe \(\delta > 0\) tal que 
    \[
      ||u-0|| = ||u|| < \delta \Rightarrow ||T(u)-T(0)|| = ||T(u)|| < 1.
    \]
  Escolha \(k > 0\) tal que \(\frac{1}{\delta } < k.\) Notemos que, para u em X, \(u\neq0\), o vetor
 \(\frac{1}{k}\frac{u}{||u||}\) tem norma \(\frac{1}{k} < \delta .\) Com isso, \(||u|| < \delta \) implica que
 \(||T(u)|| < 1\) e, em particular,
 \begin{align*}
   \biggl|\biggl|\frac{u}{k||u||}\biggr|\biggr| < \delta &\Rightarrow \biggl|\biggl|T \biggl(\frac{u}{k||u||}\biggr)\biggr|\biggr| = \biggl|\biggl|\frac{1}{k||u||}T(u)\biggr|\biggr|\\
                                                         &= \frac{1}{k||u||}||T(u)|| < 1\\
                                                         &\Rightarrow ||T(u)|| < k||u||.
  \end{align*}
  Além disso, caso \(u=0\), segue que \(0 = ||T(0)|| = 0k = ||u||k.\) Portanto, \(||T(u)||\leq k||u||\) para todo u em X.

  (c) \(\Rightarrow \) (d): Segue que, para todos u, v em X, 
    \[
      ||T(u-v)||\leq k||u-v||.
    \]
    Como T é linear, isto equivale a 
    \[
      ||T(u)-T(v)||\leq k||u-v||.
    \]
    Portanto, T é Lipschitziana com constante k.

  (d) \(\Rightarrow \) (a): Fixado \(\varepsilon > 0\), caso T seja Lipschitz, então, para todos u, v em X, existe uma constante \(k > 0\) tal que 
    \[
      ||T(u) - T(v)||\leq  k ||u-v||
    \]
    Assim, escolhendo \(\delta  = \frac{\varepsilon }{k} > 0,\) obtemos o seguinte: Caso \(||u-v|| < \delta \), então 
      \[
        ||T(u)-T(v)||\leq k||u-v|| < k \frac{\varepsilon }{k} = \varepsilon .
      \]
    Portanto, T é contínua. \qedsymbol

\end{proof*}
\newpage

\section{Aula 10 - 21/09/2023}
\subsection{Motivações}
 \begin{itemize}
   \item Composta de Funções Contínuas;
   \item Propriedades de Funções Contínuas.
 \end{itemize}
\subsection{Composta de Funções Contínuas}
 \begin{prop*}
   Sejam \(f:M\rightarrow N\) e \(g:N\rightarrow P\) são funções contínuas nos pontos p em M e f(p) em N,
respectivamente. Então, \(g\circ{f}:M\rightarrow P\) é contínua no ponto p em M.
 \end{prop*}
 \begin{proof*}
   Seja \(\varepsilon >0.\) Considere a bola \(B_{\varepsilon }^{P}(g\overline{f}(p)) = B_{1}.\) Como g é contínua em f(p), então
existe \(\delta_{g}>0\) tal que \(B_{2}\coloneqq B_{\delta_{g}}^{N}(f(p))\subseteq{g^{-1}(B_{1}),}\) tal que 
  \[
    f^{-1}(B_{2})\subseteq{f^{-1}(g^{-1}(B_{1}))} = (f^{-1}\circ{g^{-1}})(B_{1}) = (g\overline{f})^{-1}(B_{1}).
  \]
  Agora, por f ser contínua, existe \(\delta_{f}>0\) tal que 
  \[
    B_{\delta_{f}^{M}(p)}\subseteq{f^{-1}(B_{2})}.
  \]
  Assim, juntando tudo, 
  \[
    B_{\delta_{f}}^{M}(p)\subseteq{f^{-1}(B_{2})}\subseteq{(g\circ{f})^{-1}(B_{1})} = (g\circ{f})^{-1}(B_{\varepsilon }^{P}(g\circ{f}(p))).
  \]
  Portanto, \(g\circ{f}\) é contínua em p em M. \qedsymbol
 \end{proof*} 
\subsection{Propriedades de Funções Contínuas}
\begin{def*}
  Sejam \(\mathbb{E}\) um espaço vetorial normado e \(X \neq\emptyset, f:X\rightarrow \mathbb{E}, g:X\rightarrow \mathbb{E}.\)
Definimos \(f+g:X\rightarrow \mathbb{E}\) por \((f+g)(x)\coloneqq f(x)+g(x).\square\)
\end{def*}
\begin{prop*}
  Se f e g são contínuas, então f + g é contínua.
\end{prop*}
\begin{proof*}
  Definimos \(h:X\rightarrow \mathbb{E}\times \mathbb{E}\) por \(h(x) = (f(x), g(x))\), a qual é contínua. A função soma
 \(s:\mathbb{E}\times \mathbb{E}\rightarrow \mathbb{E}\) dada por \(s(x, y) = x + y\), como vimos, é contínua. Com isso, 
   \[
     f(x)+g(x)=s(f(x), g(x)) = s(h(x)) = (s\circ{h})(x).
   \]
  Portanto, (f+g)(x) é contínua.\qedsymbol
\end{proof*}
\begin{def*}
  Sejam \(\mathbb{E}\) um espaço vetorial normado e \(X \neq\emptyset, f:X\rightarrow \mathbb{E}, g:X\rightarrow \mathbb{E}.\)
Definimos \(fg:X\rightarrow \mathbb{E}\) por \((fg)(x)\coloneqq f(x)g(x).\square\)
\end{def*}
\begin{prop*}
  Se f e g são contínuas, então fg é contínua.
\end{prop*}
\begin{proof*}
  Definimos \(h:X\rightarrow \mathbb{E}\times \mathbb{E}\) por \(h(x) = (f(x), g(x))\), a qual é contínua. A função multiplicação
 \(m:\mathbb{E}\times \mathbb{E}\rightarrow \mathbb{E}\) dada por \(m(x, y) = xy\), como vimos, é contínua. Com isso, 
   \[
     f(x)g(x)=m(f(x), g(x)) = m(h(x)) = (m\circ{h})(x).
   \]
  Portanto, (fg)(x) é contínua.\qedsymbol
\end{proof*}
\begin{prop*}
  Se f e g são contínuas e \(g(x)\neq0\) para todo x em X, então \(\frac{f}{g}\) é contínua.
\end{prop*}
\begin{proof*}
  Definimos \(h:\mathbb{R}^{\times}\rightarrow \mathbb{R}, h(t)=\frac{1}{t}\), a qual é contínua. Então, 
    \[
      \frac{f(x)}{g(x)} = f(x)\frac{1}{g(x)} = f(x)h(g(x)) = f(x)(h\circ{g})(x).
    \]
    Portanto, como o produto de funções contínuas é contínua, \(\frac{f(x)}{g(x)}\) é contínua. \qedsymbol
\end{proof*}
\newpage

\section{Aula 11 - 28/09/2023}
\subsection{Motivações}
\begin{itemize}
  \item Homeomorfismos
\end{itemize}
\subsection{Espaços Homeomorfos}
 \begin{def*}
   Sejam M e N espaços métricos. Uma função \(f:M\rightarrow N\) é chamada homeomorfismo se
ela é bijetora e tanto ela quanto sua inversão são funções contínuas. Dizemos, neste caso, que M e N
são homeomorfos. \(\square\)
 \end{def*}   
  Resumidamente: ``\textit{Função contínua com inversa contínua}''.
\begin{example}
  Considere \(\varphi :[0, 2\pi )\rightarrow \mathbb{S}^{1},\) com 
    \[
      \mathbb{S}^{1}\coloneqq \{(x, y)\in \mathbb{R}^{2}: x^{2}+y^{2}=1\},
    \]
e \(\varphi (t) = (\cos{(t)}, \sin{(t)}), t\in[0, 2\pi ).\) Para uma bola aberta em 
 \(\mathbb{S}^{1}\), \(\varphi ^{-1}(B)\) não é um aberto, ou seja, a inversa
 de \(\varphi \) não é contínua.

  Apesar disso, o hemisfério norte \(\mathbb{S}^{1}=\{(x, y)\in \mathbb{S}^{1}: y > 0\}\) é
homeomorfo à bola aberta \(B_{1}(0) = B(0, 1) = (-1, 1)\), uma vez que esse hemisfério é o gráfico
de \(f(x) = \sqrt[]{1-x^{2}}, x\in (-1, 1).\) De maneira geral, o hemisfério norte 
da esfera 
  \[
    \mathbb{S}_{+}^{n} = \{(x_{1}, \cdots, x_{n})\in \mathbb{S}^{n}: x_{n}>0\}
  \]
  é homeomorfo à bola aberta \(B(0, 1)\subseteq{\mathbb{R}^{n}}\).
\end{example}
\begin{example}
  Seja \(p=(0, 0, 1)\) o polo norte da esfera \(\mathbb{S}^{2}=\{(x, y, z): x^{2}+y^{2}+z^{2}=1\}\). Então, 
 \(S^{2}\setminus\{p\}\) é homeomorfo a \(\mathbb{R}^{2}.\) A projeção estereográfica
 \(\varphi :\mathbb{S}^{2}\setminus\{p\}\rightarrow \mathbb{R}^{2}\), definida por 
   \[
     \varphi (x, y, z) = \biggl(\frac{x}{1-z}, \frac{y}{1-z}\biggr)
   \]
é contínua, bem como sua inversa, \(\varphi ^{-1}:\mathbb{R}^{2}\rightarrow \mathbb{S}^{2}\setminus\{p\}\), dada por 
  \[
    \varphi ^{-1}(x, y) = \biggl(\frac{2x}{x^{2}+y^{2}+1}, \frac{2y}{x^{2}+y^{2}+1}, \frac{x^{2}+y^{2}-1}{x^{2}+y^{2}+1}\biggr).
  \]
\end{example}
\begin{example}
  O círculo 
    \[
      \mathbb{S}^{1}=\{(x, y)\in \mathbb{R}^{2}: x^{2}+y^{2}=1\}
    \]
  e o quadrado 
    \[
      Q = \{(x, y)\in \mathbb{R}^{2}: |x|+|y|=1\}
    \]
  do espaço euclidiano \(\mathbb{R}^{2}\) são homeomorfos. A função \(f:Q\rightarrow \mathbb{S}^{1}\),
definida por 
  \[
    f(x, y) = \biggl(\frac{x}{\sqrt[]{x^{2}+y^{2}}}, \frac{y}{\sqrt[]{x^{2}+y^{2}}}\biggr)
  \]
  é um homeomorfismo
\end{example}
\begin{example}
  O plano perfurado 
  \[
    X = \{(x, y)\in \mathbb{R}^{2}: (x, y)\neq(0, 0)\}
  \]
e o cilindro circular reto 
  \[
    Y = \{(x, y, z)\in \mathbb{R}^{3}: x^{2}+y^{2}=1\}
  \]
são homeomorfos pelo homeomorfismo \(f:Y\rightarrow X\)
  \[
    f(x, y, z) = (x e^{z}, y e^{z}),\quad (x, y, z)\in Y.
  \]
\end{example}
\begin{example}
  Seja \(f:M\rightarrow N\) contínua. O gráfico de f, 
    \[
      G(f)\coloneqq \{(x, f(x)): x\in M\},
    \]
  é naturalmente homeomorfo a M, o domínio da função. Para ver isso, define-se
\(F:G(f)\rightarrow M, G(x, f(x))=x\), que é contínua por ser a composta da projeção na primeira coordenada com a inclusão.
F é sobrejetora, claramente, e 
  \[
    F(x, f(x)) = F(y, f(y)) \Longleftrightarrow x = y,
  \]
tal que f é injetora e, destarte, bijetora. Sua inversa é contínua pois a função
 \(F^{-1}:M\rightarrow G(f)\) dada por \(\pi_{1}\circ{f}\) também é contínua.
\end{example}
\begin{example}
  Se E é um espaço vetorial normado, então toda bola aberta \(B_{\varepsilon }(p) = B(p, \varepsilon )\)
é homeomorfa ao espaço E. Mostraremos que E E é homeomorfo a \(B_{1}(0) = B(0, 1)\), pois \(B_{\varepsilon }(p)\)
é homeomorfa a \(B_{1}(0).\) De fato, defina \(f:E\rightarrow B_{1}(0)\) por 
  \[
    f(u) = \frac{u}{1+||u||},\quad u\in E,
  \]
e \(g:B_{1}(0)\rightarrow E\) por 
  \[
    g(u) = \frac{u}{1-||u||},\quad u\in B_{1}(0),
  \]
que são contínuas e inversas uma da outra.
\end{example}
\begin{prop*}
  Sejam d e \(\rho \) métricas sobre um conjunto M. Para que d e \(\rho \) sejam equivalentes é
necessário e suficiente que a inclusão seja um homeomorfismo.
\end{prop*}
\begin{proof*}
  Suponha que a inclusão \(i:(M, d)\rightarrow (M, \rho )\) é um homeomorfismo. Então, 
\(i^{-1}(B_{\varepsilon }^{\rho }(x))\) é aberto em M. Logo, existe \(\delta > 0\) tal que 
\(B_{\delta }^{d}(x) \subseteq{i^{-1}(B_{\varepsilon }^{\rho }(x))}\)
Além disso, \(i^{-1}:(M, \rho )\rightarrow (M, d)\) é contínua e então \((i^{-1})^{-1}(B_{\varepsilon }^{d}(x))\)
é aberto, logo, \(\delta >0\) existe satisfazendo \(B_{\delta }^{\rho }(x) \subseteq{B_{\varepsilon }^{d}(x)}\), o 
que define, em conjunto com a primeira parte, duas métricas equivalentes.

  Por outro lado, assuma d e \(\rho \) equivalentes. Neste caso, precisamos verificar
que \(i^{-1}(A)\) é aberto para todo aberto de \((M, \rho) \) e \((i^{-1})^{-1}(B)\) é aberto em \((M, d)\).
Com efeito, dado \(A\subseteq{M}\) aberto, seja \(x\in i^{-1}(A)\). Como x pertence a A, existe \(\delta >0\) de
maneira que \(B_{\delta }^{\rho }(x)\subseteq{A}.\) Seja \(r > 0\) tal que \(B_{r}^{d}(x)\subseteq{B_{\delta }^{r}}\subseteq{A}\),
o que garante a continuidade de i. O análogo pode ser feito para mostrar que \(i^{-1}\) é contínua (exercício).
Portanto, i é homeomorfismo. \qedsymbol
\end{proof*}
\newpage

\section{Aula 13 - 17/10/2023}
\subsection{Motivações}
\begin{itemize}
  \item Espaço Conexos;
  \item Exemplos de Espaços Conexos;
  \item Teorema do Valor Intermediário.
\end{itemize}
\subsection{Espaço Conexos}
 \begin{def*}
   Uma cisão de um espaço métrico M é uma decomposição 
     \[
       M = A\cup{B},
     \]
  em que A e B são conjuntos abertos e disjuntos de M. \(\square\)
 \end{def*}
  Observe que, em uma cisão, os conjuntos A e B são simultaneamente abertos e fechados 
de M, visto que 
  \[
    M = A\cup B\quad\&\quad A\cap B = \emptyset
  \]
se, e somente se, 
  \[
    A = B^{c} \quad\&\ B = A^{c}.
  \]
  Além disso, é possível definir uma \textit{Cisão Trivial}, na qual A ou B é vazio, fazendo com que
a cisão tenha a forma \(M = M\cup\emptyset\)
\begin{def*}
  Um espaço métrico M é dito ser conexo se a única cisão possível para M é a trivial. Um
subconjunto A de M é um conjunto conexo dado que \((A, \rho_{A})\) é conexo com a métrica induzida.\(\square\)
\end{def*}
  Quando A admite uma cisão não-trivial, dizemos que A é \textit{desconexo}.
 \begin{example}
   Seja \(M_{1} = [0,1]\cup [2, 3]\). Podemos escrever 
     \[
       M_{1} = \biggl((-\varepsilon , 1 + \varepsilon )\cap M\biggr)\bigcup_{}^{}{\biggl((2-\varepsilon , 3+\varepsilon )\cap M\biggr)},
     \]
em que \(0 < \varepsilon < \frac{1}{2}.\) Tome, também, 
  \[
    M_{2} = \{1, 2, 3\}.
  \]
  Neste caso, \(M_{2} = (B_{\frac{1}{2}}(1)\cap M)\cup ((B_{\frac{1}{2}}(2)\cup B_{\frac{1}{2}}(3))\cap M)\)
Em ambos os casos, \(M_{1}\) e \(M_{2}\) são desconexos.
 \end{example}
 \begin{example}
   São desconexos os seguintes subconjuntos de \(\mathbb{R}\): 
     \[
       \mathbb{Q}, \quad \mathbb{Z},\quad\text{e}\quad A = \mathbb{R}\setminus\{0\}.
     \]
Vejamos estes exemplos. 

  Nos casos de \(\mathbb{Q}, \mathbb{Z}\), vale que 
    \[
      \mathbb{Q} = \underbrace{\biggl((-\infty, \pi )\cap \mathbb{Q}\biggr)}_{A}\cup \underbrace{\biggl((\pi , \infty)\cap \mathbb{Q}\biggr)}_{B}
    \]
e 
    \[
    \mathbb{Z} = \underbrace{\biggl((-\infty, \pi )\cap \mathbb{Z}\biggr)}_{C}\cup \underbrace{\biggl((\pi , \infty)\cap \mathbb{Z}\biggr)}_{D}
    \]
  Como \(A\cap B = C\cap D = \emptyset, A,\text{ e }B\) são abertos em \(\mathbb{Q}\) e \(C, D\) em \(\mathbb{Z}\), tal que \(\mathbb{Q} = A\cup B\)
e \(\mathbb{Z} = C\cup D\) são ambos desconexos.

  Colocando \(\mathbb{R}\setminus\{0\} = M\), temos a cisão não-trivial de M
    \[
      M = \biggl((-\infty, 0)\cap M\biggr)\bigcup_{}^{}{\biggl((0, +\infty)\cap M\biggr)},
    \]
tornando-o desconexo.

 \end{example}
 \begin{example}
   \(\mathbb{R}\) com a métrica usual é um conjunto conexo. De fato, suponha que exisstam
   \(A, B\subseteq{\mathbb{R}}\) não-vazios que são abertos e fechados e tais que \(A\cup B = \mathbb{R}\).
   Seja \(a\in A\) e \(b\in B\) e digamos que \(a < b\). O conjunto 
     \[
       A_{b}\coloneqq \{x\in A: x < b\}\neq\emptyset
     \]
  é limitado superiormente. Coloque \(c = \sup A_{b}\leq b\) e \(c\in \overline{A_{b}}\subseteq{\overline{A}}.\) 
Como \(A = \overline{A}, c\in A\) (Lembrando que A é aberto e fechado) e \(c < b\). Como A é
aberto, existe \(\varepsilon >0\) tal que \(c+\varepsilon < b\) e \((c-\varepsilon , c + \varepsilon )\subseteq{A}\),
contradizendo a hipótese de \(c = \sup A_{b}.\)
 \end{example}
\begin{prop*}
  Seja \((X, \rho )\) um espaço métrico. São equivalentes:
 \begin{itemize}
   \item[1)] X é conexo;
   \item[2)] X e \(\emptyset\) são os únicos subconjuntos de X que são simultaneamente
abertos e fechados;
   \item[3)] Se \(A\subseteq X\) tem fronteira vazia, então \(A = X\) ou \(A = \emptyset\)
 \end{itemize}
\end{prop*}
\begin{proof*}
  \(1) \Rightarrow 2)\) Se \(A \subseteq{X}\) for, ao mesmo tempo, aberto e fechado,
então \(A^{c}\) será, também, aberto e fechado ao mesmo tempo. Além disso, \(A\cup A^{c} = X\).
Segue que \(A = X\) ou \(A = \emptyset.\)

  \(2) \Rightarrow 3)\) Se \(A\subseteq{X}\) e \(\partial A = \emptyset,\) A é aberto e \(A^{c}\) é 
  aberto. Logo, A e \(A^{c}\) também são fechados e \(A=X\) ou \(A=\emptyset\).

  \(3) \Rightarrow 1)\) Se \(X = A\cup A^{c}\) com A e \(A^{c}\) abertos, então \(\partial A = \emptyset\) e \(A = X\)
ou \(A = \emptyset.\) Logo, X é conexo. \qedsymbol
\end{proof*}
 \begin{prop*}
  \begin{itemize}
    \item[1)] A imagem de um conjunto conexo por uma função contínua é um conjunto conexo (continuidade preserva conexidade)
    \item[2)] O fecho de um conjunto conexo é conexo
    \item[3)] Se \(A_{1}\) e \(A_{2}\) são subconjuntos conexos de um espaço M e 
  \(A_{1}\cap A_{2}\neq\emptyset\), então \(A_{1}\cup A_{2}\) também é conexo.
  \end{itemize}
 \end{prop*}
 \begin{proof*}
\textbf{Prova do Item 1:} Considere o caso em que \(f:M\rightarrow N\) é contínua, sobrejetiva e M é conexo.
Vamos provar que \(N = f(M)\) é conexo. Com efeito, seja \(N=A\cup{B}\) uma cisão.
Então, \(M = f^{-1}(A)\cup f^{-1}(B)\) é uma cisão. Como M é conexo, segue que ou \(f^{-1}(A)\) ou \(f^{-1}(B)\) é vazio.
Já que f é sobrejetiva, A ou B deve ser vazio. 

  Por fim, se \(f:M\rightarrow N\) é contínua e \(X\subseteq{M}\) é conexo, então
\(f:X\rightarrow f(X)\) é uma sobrejeção, caso o qual já demonstramos que será conexo.
  
  \textbf{Prova do Item 2:} Consideremos, primeiramente, o caso em que \(X\subseteq{M}\) é um conjunto conexo tal que
 \(\overline{X} = M\), ou seja, X é denso em M. Neste caso, se \(M = A\cup B\) for uma cisão, temos a cisão
 \(X = (A\cap X)\cup(B\cap X)\). Como X é conexo, segue que \(A\cap X = \emptyset\) ou \(B\cap X = \emptyset\). Sendo X denso
em M, isto implica \(A = \emptyset\) ou \(B = \emptyset\).

  Agora, com relação ao caso geral, se X é conexo, segue de \(\overline{X} = \overline{\overline{X}}\) que X é denso em \(\overline{X}\), reduzindo, assim,
ao caso anterior.

\textbf{Prova do Item 3:} Consideremos \(X\coloneqq A_{1}\cup A_{2}\) e seja \(X = C\cup D\) uma cisão. Nosso objetivo é mostrar que
ou C ou D devem ser vazios. Tome \(a\in A_{1}\cap A_{2}\) e assuma que \(a\in C\). As cisões 
  \[
    A_{1} = (A_{1}\cap C)\cup (A_{2}\cap D)\quad\text{e}\quad A_{2} = (A_{2}\cap C)\cup(A_{2}\cap D)
  \]
  implicam que \(A_{1}\cap D = A_{2}\cap D = \emptyset, \) já que \(A_{1}\) e \(A_{2}\) são conexos. Logo, 
    \[
      D = D\cap X = D\cap(A_{1}\cup A_{2}) = (D\cap A_{1})\cup(D\cap A_{2}) = \emptyset.\text{\qedsymbol} 
    \]
 \end{proof*}
\begin{crl*}
 \begin{itemize}
   \item[1)] Se M é conexo e N é homeomorfo a M, então N também é conexo
   \item[2)] Se \(X\subseteq{Y}\subseteq{\overline{X}}\) e X é conexo, então Y é conexo
   \item[3)] Dados dois espaços métricos M e N, então \(M\times N\) é conexo se, e somente se,
M e N são conexos.
 \end{itemize}
\end{crl*}
\begin{proof*}
  \textbf{Prova do Item 3:} Se \(M\times N\) é conexo, então \(M = \pi_{1}(M\times N)\) e \(N = \pi_{2}(M\times N)\) são conexos, pois
a projeção é uma função contínua. 

  Por outro lado, suponha que M e N são conexos. Para todo x em M, é possível provar
que \(\{x\}\times N\) é homeomorfo a N, tal que \(\{x\}\times N\) é conexo para todo x em M. Analogamente, \(M\times\{y\}\) é 
homeomorfo a M e, assim, conexo para todo y em N. Com isso, seja \(n\in N\) e considere 
  \[
    C_{x} = \biggl(\{x\}\times N\biggr)\bigcup_{}^{}{\biggl(M\times \{n\}\biggr)},
  \]
em que \(\biggl(\{x\}\times N\biggr)\cap \biggl(M\times \{n\}\biggr)\neq\emptyset.\) Neste caso, \(C_{x}\) é conexo para todo
x em M. Observamos, também, que 
  \[
    M\times \{n\}\subseteq{\bigcap_{x\in M}^{}{C_{x}}}\neq\emptyset.
  \]
  Portanto, 
    \[
      M\times N = \bigcup_{x\in M}^{}{C_{x}},
    \]
o qual é conexo.\qedsymbol
\end{proof*}
\begin{example}
  \(\mathbb{S}^{1}\) é conexa. Com efeito, tomando \(p = (0, 1)\in \mathbb{S}^{1}\) e \(X = \mathbb{S}^{1}\setminus\{p\}\), sabemos que X é homeomorfo
à reta \(\mathbb{R}\) pela projeção estereográfica. Logo, X é conexo. Concluímos, assim, que \(\mathbb{S}^{1}\) é conexo observando que
 \(\overline{X} = \mathbb{S}^{1}.\)
\end{example}
\begin{example}
  Podemos definir uma ``projeção estereográfica'' \(\pi_{u}:\mathbb{S}^{1}\setminus\{u\}\rightarrow \mathbb{R}\) a partir de qualquer ponto 
 \(u\in \mathbb{S}^{1}\) usado como polo no lugar de p, de forma que \(\mathbb{S}^{1}\setminus\{u\}\) é um conjunto conexo. No entanto, se omitirmos dois pontos distintos \(u, v\in \mathbb{S}^{1},\) obtemos o conjunto
 desconexo \(\mathbb{S}^{1}\setminus\{u, v\}\). Com efeito, seja \(ax + by = c\) a equação da reta que passa
 pelos dois pontos u e v. Então, \(A\cup B\) é uma cisão não trivial tal que \(A = \{(x, y)\in \mathbb{S}^{1}: ax + by > c\}\) e 
 \(B = \{(x, y)\in \mathbb{S}^{1}: ax + by < c\}\).
\end{example}
\begin{example}
  Todo intervalo aberto de \(\mathbb{R}\) é conexo (de fato, a recíproca também vale, como veremos logo mais). Com efeito, todo intervalo
aberto é homeomorfo a \(\mathbb{R}\). Caso o intervalo seja limitado, esse fato decorre dos seguintes homeomorfismos:
  \[
    h:B_{1}(0)\rightarrow B_{r}(u),\quad v\mapsto h(v) = u + rv
  \]
e 
  \[
    g:V\rightarrow B_{1}(0),\quad v\mapsto g(v) = \frac{v}{(1+||v||)}
  \]
Se o intervalo é ilimitado, então decorre dos seguintes homeomorfismos:
  \[
    h:\mathbb{R}\rightarrow (a, \infty),\quad x\mapsto a + e^{x}
  \]
e 
  \[
    h^{-1}:(a, \infty)\rightarrow \mathbb{R},\quad x\mapsto \ln{(x-a)}.
  \]
Para o outro tipo de intervalo ilimitado, eles são 
  \[
    h:\mathbb{R}\rightarrow (-\infty, a),\quad x\mapsto a - e^{-x}
  \]
e 
  \[
    h^{-1}:(-\infty, a)\rightarrow \mathbb{R},\quad x\mapsto -\ln{(a-x)}.
  \]
\end{example}
Esse último exemplo mostra uma parte do seguinte teorema:
 \begin{theorem*}
    Um subconjunto de \(\mathbb{R}\) é conexo se, e somente se, ele é um intervalo. 
 \end{theorem*}
 \begin{proof*}
   A parte de todo intervalo ser conexo foi mostrada no exemplo anterior.

   \(\Rightarrow )\) Reciprocamente, seja \(X\subseteq{\mathbb{R}}\) conexo. Suponha que \(a, b\in X\) e que
  \(a < c < b.\) Provaremos que, neste caso, \(c\in X\). Com efeito, suponha, para fins de contradição, que
  \(c\not\in X.\) Então, teríamos a cisão não trivial de X
    \[
      X = [X\cap(-\infty, c)]\cup[X\cap(c, +\infty)],
    \]
    pois \(a\in X\cap(-\infty, c)\) e \(b\in X\cap(c, +\infty).\) Assim, \(a < c < b\) com \(a, b\in X\) implica que c pertence a X, propriedade que
garante que X seja um intervalo. \qedsymbol
 \end{proof*}
 \hypertarget{intermediate_value}{\begin{crl*}
     Seja \(f:[a, b]\rightarrow \mathbb{R}\) contínua. Se \(f(a) < d < f(b)\), então existe c em \((a, b)\) tal que \(f(c) = d.\)  
 \end{crl*}}
 \begin{proof*}
   A imagem \(f([a, b])\) é um intervalo que contém os pontos \(f(a)\) e \(f(b)\), logo \([f(a), f(b)]\subseteq{f([a, b])}.\) Assim,
existe \(c\in[a, b]\) tal que \(f(c) = d\). Porém, \(f(a) < f(c) < f(b)\) exclui a possibilidade de \(c = a\) ou \(c = b\). Portanto,
 \(c\in (a, b)\). \qedsymbol
 \end{proof*}
\newpage

\section{Aula 14 - 19/10/2023}
\subsection{Motivações} 
\begin{itemize}
  \item Exemplos de Espaços Conexos;
  \item Relação de continuidade e conexidade.
\end{itemize}
\subsection{Conexidade - Continuação}
\begin{theorem*}[Alfândega]
  Sejam Y, X subconjuntos de um espaço métrico M. Se Y é conexo e \(Y\cap X \neq\emptyset\) e \(Y\cap X^{c}\neq\emptyset\), então
 \(Y\cap \partial X \neq\emptyset.\)
\end{theorem*}
\begin{proof*}
  Como \(Y\cap X \neq\emptyset\) e \(Y\cap X^{c} \neq\emptyset\), o subconjunto \(Y \cap X\) do espaço métrico conexo
não é nem vazio e nem o espaço todo. Logo, existe algum c pertencente à fronteira de \(Y\cap X\) no subespaço Y. De fato,
na verdade, vamos mostrar que c pertence à fronteira de X em M. Com efeito, dado \(\varepsilon >0\), existe \(s\in Y\cap X\subseteq{X}\) e 
 \(t\in Y\setminus\{Y\cap X\} = Y\setminus{X}\subseteq{M\setminus\{X\}}\) com \(d(c, s) < \varepsilon \) e \(d(c, t) < \varepsilon \), tal que
 \(c\in \partial X\). Portanto, \(Y\cap \partial X \neq\emptyset.\) \qedsymbol
\end{proof*}
\begin{def*}
  Um \textbf{caminho} num espaço métrico M é uma função contínua \(f:[0, 1]\rightarrow M.\) Os pontos \(a = f(0)\) e \(b = f(1)\) são os extremos
do caminho. Neste cado, f é dito ligar o ponto a ao ponto b em M e escreveremos \(x\thicksim a.\) Quando \(a = b,\) dizemos que f é um caminho fechado
em M. \(\square\)
\end{def*}
Essa relação é de equivalência. 

Para ver que é reflexiva, seja x em M e defina \(f:[0, 1]\rightarrow M\) por \(f(t) = x, t\in [0, 1],\) tal que temos \(x\thicksim x.\)

Agora, para a simetria, sejam \(x, y\in M\) tais que \(x\thicksim y\) e \(f:[0, 1]\rightarrow M\) tal que \(f(0) = x\) e \(f(1) = y\). Definimos \(g:[0, 1]\rightarrow M\) por
 \(g(t) = f(1-t), t\in [0, 1],\) que é contínua por ser a composta de contínua e \(g(0) = f(1) = y\) e \(g(1) = x,\) de modo que \(y\thicksim x\).

Finalmente, a transitividade segue do seguinte - sejam \(x, y, z\in M\) com \(x\thicksim y\) e \(y\thicksim z\). Seja f o caminho com ponto inicial x, final y e g o 
caminho com ponto inicial y e final z. Definimos \(h:[0, 1]\rightarrow M \) por 
  \[
    h(t) = \left\{\begin{array}{ll}
        f(2t),\quad 0\leq t\leq \frac{1}{2}\\
        g(2t-1),\quad \frac{1}{2}\leq t\leq 1.
      \end{array}\right.
  \]
  Essa função h é chamada \textbf{caminho justaposto}.
 \begin{def*}
   Um espaço métrico M é dito ser \textbf{conexo por caminho} se, para quaisquer a, b em M, \(a\thicksim b\), ou seja,
existe um caminho f em M que liga a a b, ou seja, \(f(0)=a\) e \(f(1)=b.\) Isto equivale a dizer que só há uma 
classe de equivalência com respeito a ``\(\thicksim\)''. \(\square\)
 \end{def*}
 \begin{example}
   Para todo \(n\geq 2, \mathbb{R}^{n}\setminus\{0\}\) é conexo por caminhos. De fato, sejam x, y em \(\mathbb{R}^{n}\setminus\{0\}\) e
consideremos dois casos.

  (1) Se \(0\not\in[x, y]\coloneqq \{ty + (1-t)x: t\in[0, 1]\},\) então consideramos o caminho 
    \[
      \gamma (t) = ty + (1-t)x,\quad t\in[0,1],
    \]
  que liga x a y em \(\mathbb{R}^{n}\setminus\{0\}.\)

  (2) Se \(0 = t_{0}y + (1-t_{0})x\) para algum \(t_{0}\in (0, 1),\) então existe 
    \[
      z\in \mathbb{R}^{n}\setminus\biggl\{t \biggl(x, \frac{yt_{0}}{(1-t_{0})}\biggr):t\in \mathbb{R}\biggr\}
    \] 
 \end{example}
 \begin{theorem*}
   Seja \(h:M\rightarrow N\) um homeomorfismo. Então, M é conexo por caminho se, e somente se, N é conexo por caminhos.
 \end{theorem*}
 \begin{proof*}
   Basta ver que, se M é conexo por caminho, dados \(f(x), f(y)\in N\), os elementos \(x, y\in M\) são ligados por um caminho \(\varphi \). Como h
é contínua, \(h\circ{\varphi }\) é contínua e sai de \([0, 1]\), provando que N é conexo por caminhos. A volta é análoga, mas com \(h^{-1}.\) \qedsymbol
 \end{proof*}
 \begin{theorem*}
   Se o espaço métrico M é conexo por caminhos, então M é conexo.
 \end{theorem*}
 \begin{proof*}
   Se \(M = A \cup B\) fosse uma cisão não-trivial de M, dados \(a\in A\) e \(b\in B\), existiria um caminho 
  \(f:[0, 1]\rightarrow M\) tal que \(f(0) = a, f(1) = b\) e, assim, teríamos 
    \[
      [0, 1] = f^{-1}(M) = f^{-1}(A)\cup f^{-1}(B),
    \]
    uma cisão de \([0, 1]\), já que \(0\in f^{-1}(A)\) e \(1\in f^{-1}(B)\), formando uma cisão do intervalo. Contradição.
Portanto, M é conexo. \qedsymbol
 \end{proof*}
 \begin{example}[Espaço Pente]
  Seja \(P\subseteq{\mathbb{R}^{2}}, P\coloneqq \{0\}\times (0,1]\cup D\cup H,\) o espaço pente, com (dentes e haste) 
    \[
      D = \bigcup_{n\in \mathbb{N}}^{}{\biggl\{\frac{1}{n}\biggr\}\times[0, 1]}\quad\text{e}\quad H = (0, 1]\times\{0\}.
    \]
  Esse espaço é conexo e não é conexo por caminhos. Suponha que existe f contínua, \(f:[0, 1]\rightarrow X\) tal que 
    \[
      f(0) = (0, 1)\quad\text{e}\quad f(1) = (1,1).
    \]
Defina \(\alpha  = \sup{\{x\in[0, 1]: f(0, x)\subseteq{B}\}}\). Então, \(f(\alpha )\) está bem-definida e vamos verificar o pertencimento de \(f(\alpha )\) a A ou a B.

\textbf{Caso \(f(\alpha )\in A\):}
  Se isso acontece, então 
    \[
      B_{r}^{f(\alpha )}\cap B = \emptyset,\quad r < d(f(\alpha ), B),
    \]
ou seja, \(A\subseteq{\mathbb{R}^{2}}\setminus{(B\cup \{0, 0\})}\), mostrando que A é aberto. Por f ser contínua, existe \(\delta \) tal que 
 \(f(\delta -\alpha , \delta +\alpha ) \subseteq{B_{r}^{f(\alpha )}}.\) Tomemos \(y\in (\alpha -\delta , \alpha ),\) de forma que \(f(y)\cap B = \emptyset\).
 Como \(y < \alpha \), então, \(f(y)\in B\), mas, pela definição de \(\alpha \), deve existir y em \((\alpha , \alpha +\delta )\) tal que não existe nenhum \(\delta \)
 satisfazendo \(f(\alpha -\delta ,\alpha +\delta )\cap B = \emptyset \)

\textbf{Caso \(f(\alpha )\in B\):}
  Analogamente, suponha que \(f(\alpha )\in B\). Sempre é possível obter \(\varepsilon >0\) tal que \((0, 0)\not\in B_{\varepsilon }(f(\alpha ))\) e, pela continuidade
de f em \(\alpha \), existe um intervalo aberto V tal que \(f(V)\subseteq{B_{\varepsilon }(f(\alpha ))}.\) Um argumento similar ao anterior mostra a existência de
um intervalo \((\alpha , \alpha +\delta ) \subseteq{V}\) para o qual \(f((\alpha , \alpha +\varepsilon ))\subseteq{B_{\varepsilon }(f(\alpha ))}.\) Assim, pela definição
de \(\alpha ,\) deve existir um \(\beta \in (\alpha , \alpha +\delta )\) tal que \(f(\beta)\in A\), já que, se não fosse o caso, \(\alpha \) não seria o supremo.
Com isso, no intervalo V, f assume valores em B e em A, o que impede que \(f(V)\) seja conexo, visto que \(f(V)\subseteq{B_{\varepsilon }(f(\alpha ))}\) e este conjunto
é composto por retas verticais. 

  Encontramos elementos tanto em B quanto em A. Isso é uma contradição ao fato de funções contínuas preservarem conexidade, ou seja, não pode existir tal função f. Portanto,
X não é conexo por caminhos, pois provamos que não há função contínua de \((0, 1)\) à \((1, 1)\).
  \end{example}
  \begin{prop*}
    Um espaço métrico \((M, \rho )\) é conexo se, e somente se, quaisquer dois pontos de M estiverem contidos em algum subconjunto
conexo.
  \end{prop*}
  \begin{proof*}
    Se \((M, \rho )\) é conexo, o resultado segue automaticamente. Para a recíproca, fixe \(x_{1}\in M\) e seja
\(C_{x}\) um conexo contendo \(\{x_{1}, x\}\). Temos \(M = \bigcup_{x\in M}^{}{C_{x}}\), que é conexo, já que \(\bigcap_{x\in M}^{}{C_{x}}\neq\emptyset.\) \qedsymbol
  \end{proof*}
  \begin{def*}
    Seja \((M, \rho )\) um espaço métrico e x um elemento de M. A \textbf{componente conexa} do ponto x em M é o subconjunto conexo maximal
 \(C_{x}\) de M com x em \(C_{x}\). 
  \end{def*}
  Entenda por maximal que, se \(D_{x}\) é qualquer conexo de M contendo x, vale \(D_{x}\subseteq{C_{x}}.\)
 \begin{example}
   As componentes conexas de \(\mathbb{R}\setminus{\{0\}}\) se referem aos conexos \((-\infty, 0)\) e \((0, \infty).\)
 \end{example}
 \begin{example}
   No espaço métrico \(\mathbb{Q},\) cada componente conexa reduz-se ao conjunto unitário formado por um ponto. Em outras palavras,
nenhum subconjunto conexo de \(\mathbb{Q}\) pode conter dois pontos distintos. Com efeito, seja \(A\subseteq{\mathbb{Q}}\) com \(A \neq\emptyset\)
e \(A\neq\{q\}.\) Sejam \(q_{1}, q_{2}\in A\) distintos e \(r\in(q_{1}, q_{2})\setminus{\mathbb{Q}}\). Temos 
  \[
    A = \biggl((-\infty, r)\cap A\biggr)\bigcup_{}^{}{\biggl((r, +\infty)\cap A\biggr)},
  \]
  formando uma cisão não trivial de A. Portanto, os únicos conjuntos conexos de \(\mathbb{Q}\) são os unitários.
 \end{example}
  Em geral, a componente maximal tem a forma 
    \[
      C_{x} = \bigcup_{\alpha \in J}^{}{D_{\alpha }^{x}},
    \]
  em que \(\{D_{\alpha }^{x}: \alpha \in J\}\neq\emptyset\) é a família de todos os subconjuntos conexos de M contendo x.

  Além disso, se \(C_{x}\cap C_{y} \neq\emptyset\), então \(C_{x} = C_{y}.\) De fato, 
    \[
      C_{x} \subseteq\underbrace{{C_{x}\cup C_{y}}}_{\text{conexos}} \subseteq{C_{y}}
    \]
  e 
    \[
      C_{y}\subseteq{C_{x}\cup C_{y}}\subseteq{C_{x}}.
    \]

  Note também que a família \(\{C_{x}: x\in M\}\) das componentes conexas de M nos fornece uma partição de M
em partes disjuntas da seguinte forma: 
  \[
    M = \bigcup_{x\in M}^{}{C_{x}}.
  \]
  Devido ao fato do fecho da componente conexa ser conexa, toda componente conexa é fechada. Ademais, se M tiver uma quantidade finita de componentes conexas,
elas serão abertas. Com efeito, se \(M = \cup_{i=1}^{n}C_{i},\) com \(C_{i}, i = 1,\cdots,n\) componentes conexas, então 
  \[
    C_{i}^{\complement} = \cup_{j=1, j\neq i}^{n}C_{j}
  \]
  que é fechado, visto que cada \(C_{j}\) é fechado e, portanto, \(C_{i}\) é aberto.
 \begin{example}
   Se M não tiver um número finito, as componentes podem não ser conjuntos abertos, vide o exemplo de \(M = \mathbb{Q}\), tal que a decomposição em suas componentes aberats é 
     \[
       \mathbb{Q} = \bigcup_{q\in \mathbb{Q}}^{}{\{q\}}.
     \]
 \end{example}
 \begin{theorem*}
   Seja \(h:M\rightarrow N\) um homeomorfismo. Então, \(C_{x}\subseteq{M}\) é componente conexa de M se, e somente se,
 \(h(C_{x})\) é uma componente conexa de N.
 \end{theorem*}
\begin{proof*}
  Se \(C_{x}\subseteq{M}\) é uma componente conexa em M e \(C_{y}\) é a componente conexa de \(y=h(x),\) então \(h(C_{x})\) é
um conexo contendo \(y = h(x)\) e, por isso, \(h(C_{x})\subseteq{C_{y}}\). Além disso, como \(h^{-1}\) é contínua, 
  \[
    h^{-1}(C_{y})\subseteq{C_{x}},
  \]
  pois \(h^{-1}(C_{y})\) é um conexo contendo x. Segue que \(C_{y}\subseteq{h(C_{x})}\), donde segue que \(h(C_{x}) = C_{y}\) é 
uma componente de N.

  Analogamente, mostra-se que \(h(C)\) é uma componente conexa de N e \(C_{x}\) é a componente conexa e \(x=h^{-1}(y)\) para algum y em h(C), então
 \(h^{-1}(C_{y})\) é uma componente de M. \qedsymbol
\end{proof*}
\newpage

\section{Aula 14 - 24/10/2023}
\subsection{Motivações} 
\begin{itemize}
  \item Sequências de Cauchy;
  \item Espaço Métrico Completo.
\end{itemize}
\subsection{Espaços Completos}
 \begin{def*}
   Seja \((X, \rho )\) espaço métrico. Uma sequência \(\{x_{n}\}\in X\) é chamada \textbf{sequência de Cauchy }se dado \(\varepsilon > 0\), existe \(n_{0}\in \mathbb{N}\) tal que 
   \[
     \rho (x_{n}, x_{m}) < \varepsilon ,\quad \forall n, m\geq n_{0}.\square
   \]
 \end{def*}
 \begin{def*}
   Uma \textbf{subsequência} de \(\{x_{n}\}\) é uma sequência \(\{x_{n, k}\}_{k}\subseteq{\{x_{n}\}}.\square\)
 \end{def*}
Vejamos algumas propriedades
\begin{lemma*}
 \begin{itemize}
   \item[1)] Toda subsequência de uma sequência de Cauchy é também de Cauchy;
   \item[2)] Toda sequência convergente é de Cauchy;
   \item[3)] Uma sequência de Cauchy pode não ser convergente;
   \item[4)] Toda sequência de Cauchy é limitada;
   \item[5)] Nem toda sequência limitada é de Cauchy;
   \item[6)] Uma sequência de Cauchy que possui subsequência convergente é convergente.
 \end{itemize}
\end{lemma*}
\begin{proof*}
  2.) Suponha que \(x_{n}\overbracket[0pt]{\longrightarrow}^{n\to \infty}x\) em M. Dado \(\varepsilon > 0\), existe \(n_{0}\in \mathbb{N}\) tal que 
    \[
      d(x_{n}, x) < \frac{\varepsilon }{2},\quad n\geq n_{0}.
    \]
  Assim, se \(n, m\geq n_{0}\), então 
    \[
      d(x_{n}, x_{m})\leq d(x_{n}, x) + d(x, x_{m}) < \frac{\varepsilon }{2}+\frac{\varepsilon }{2}=\varepsilon .
    \]
  Logo, \(\{x_{n}\}\) é de Cauchy. 

  3.) Para um exemplo de uma sequência de Cauchy que não é convergente, considere \(M=(0, 1), x_{n}=\frac{1}{n},n\in \mathbb{N}\).

  4.) Seja \(\{x_{n}\}\) de Cauchy. Existe \(n_{0}\in \mathbb{N}\) tal que 
    \[
      d(x_{n}, x_{m}) < 1,\quad \forall n\, m\geq n_{0}.
    \]
  Com isso, para \(n\geq n_{0}\), vale 
    \[
      d(x_{n_{0}}, x_{n}) < 1.
    \]
  Concluímos que \(x_{n}\in B_{1}(n_{0})\) para \(n\geq n_{0}\). Seja \(r = \max_{1\leq i\leq n_{0}-1}\{1, d(x_{n_{0}}, x_{i})\}\). Temos, então,
 \(\{x_{n}\}\subseteq{B_{2r}(x_{n_{0}})}\).

  5.) Um exemplo é \(\{(-1)^{n}\}.\)

  6.) Considere \(\{x_{n}\}\) sequência de Cauchy com uma subsequência \(\{x_{n, k}\}_{k}\) convergente com limite x.
Dado \(\varepsilon >0\), existe \(n_{0}\in \mathbb{N}\) tal que \(d(x_{n}, x_{m}) < \varepsilon , n\geq n_{0}\), já que \(\{x_{n}\}\) é de Cauchy,
e \(d(x_{n_{k}}, x) < \varepsilon , k\geq n_{1}.\) Para \(m_{0}=\max{n_{0}, n_{1}}\), valem para
todo \(n, k\geq n_{0}\) as duas coisas, tal que, para \(n\geq m_{0},\)
  \[
    d(x_{n}, x)\leq d(x_{n}, x_{m_{0}}) + d(x_{m_{0}}, x) < \frac{\varepsilon }{2} + \frac{\varepsilon }{2} = \varepsilon .
  \]
  Portanto, \(x_{n}\overbracket[0pt]{\longrightarrow}^{n\to \infty}x.\)
  \qedsymbol
\end{proof*}
\begin{def*}
  Um espaço métrico \((X, \rho )\) é dito \textbf{completo} se toda sequência de Cauchy em X converge. Um subconjunto \(E\subseteq{X}\) é \textbf{completo} se toda sequência de Cauchy de E for convergente em E. \(\square\)
\end{def*}
\begin{example}
  \(\mathbb{R}\) é completo. Seja \(\{x_{n}\}\) uma sequência de Cauchy em \(\mathbb{R}\). Consideremos \(X_{n}= \{x_{j}:j = n, n + 1, \cdots\}, n = 1, 2, \cdots\).
Então, 
  \[
    X_{n} \supsetneq{X_{n+1}}\quad \forall n\in \mathbb{N}.
  \]
  Seja \(a_{n} = \inf\{X_{n}\}, n\in \mathbb{N}.\) Pela relação acima, \(a_{n} = \inf\{X_{n}\}\leq \inf\{X_{n+1}\} = a_{n+1}\), mostrando que \(\{a_{n}\}\) é crescente e 
limitada. Em outras palavras, \(a = \sup\{a_{n}: n\in \mathbb{N}\} = \lim_{n\to \infty}a_{n}.\)
  Mostremos agora que \(\{x_{n}\}\) tem uma subsequência convergente para a e, nesse caso, \(x_{n}\overbracket[0pt]{\longrightarrow}^{n\to \infty}a.\)

  Com efeito, dado \(\varepsilon_{k} = \frac{1}{k},\) existe \(n_{0, k}\in \mathbb{N}\) tal que 
  \[
  a-\frac{1}{k} < a_{n} < a + \frac{1}{k},\quad \forall n\geq n_{0_{k}}.
  \]
Como \(a_{n_{0_{k}}}=\inf\{X_{n_{0_{k}}}\},\) existe \(x_{n_{k}}\in X_{n_{0_{k}}}\) tal que 
  \[
    a - \frac{1}{k} < a_{n_{k}}\leq x_{n_{k}} < a + \frac{1}{k}.
  \]
  Portanto, a sequência \(\{x_{n_{k}}\}\) converge para a e o resultado está provado. 
\end{example}
\begin{example}
  Considere \(M = (0, 1)\) com a métrica usual. Então, M não é completo, visto que existe uma sequência de Cauchy que não é convergente.
\end{example}
\begin{example}
  \((X, \rho )\) com \(X \neq\emptyset\) e \(\rho \) a métrica discreta em X é completo. 

  De fato, seja \(\{x_{n}\}\) de Cauchy em X. Por definição, para \(0 < \varepsilon < 1,\) existe \(n_{0}\in \mathbb{N}\) tal que  
  \[
    d(x_{n}, x_{m}) < \varepsilon, \quad \forall n,m\geq n_{0}.
  \]
Como \(d(x_{n}, x_{m}) < \varepsilon \) se, e somente se, \(x_{n} = x_{m},\) conclui-se que a sequência \(\{x_{n}\}\) é constante a partir do \(n_{0}\),
portanto ela converge, já que \(\{x_{n, k}\}_{k\geq n_{0}}\) é uma subsequência convergente.
\end{example}
\begin{example}
  \((\mathbb{R}^{n}, ||\cdot||_{p})\) é completo e \(\mathbb{Q}^{n}\) não é completo.

  Com efeito, temos 
  \[
    ||x||_{p} = \biggl(\sum\limits_{i=1}^{n}|x_{i}|^{p}\biggr)^{\frac{1}{p}},\quad p\geq 1.
  \]
Considere \(\{x_{n}\}\) de Cauchy em \(\mathbb{R}^{n}\). Observamos que se \(x_{n} = (x_{m}^{1}, \cdots, x_{m}^{n}), m = 1, 2, \cdots.\), então 
  \[
    |x_{m}^{i}-x_{q}^{i}|\leq ||x_{m}-x_{q}||_{p}
  \]
Logo, \(\{x_{m}^{i}\}\) é uma sequência de Cauchy para cada \(i=1, \cdots, n.\) Como \(\mathbb{R}\) é completo,
existe \(y_{i}\) tal que \(x_{m}^{i}\overbracket[0pt]{\longrightarrow}^{m\to \infty}y_{i}.\) Seja \(y= (y_{1}, \cdots, y_{n}).\)
Mostremos que \(x_{m}\overbracket[0pt]{\longrightarrow}^{m\to \infty}y.\) De fato, dado \(\varepsilon >0\), existe \(n_{0}\in \mathbb{N}\)
tal que 
  \[
    ||x_{m}-y||_{p} = \biggl(\sum\limits_{i=1}^{n}|x_{m}^{i}-y_{i}|^{p}\biggr)^{\frac{1}{p}} < \varepsilon.
  \]
Da convergência de \(x_{m}^{i}\overbracket[0pt]{\longrightarrow}^{m\to \infty}y_{i},\) existe \(n_{i}\in \mathbb{N}\) tal que 
  \[
    |x_{m}^{i}-y_{i}| < \frac{\varepsilon}{n},\quad i=1,\cdots,n.
  \]
  Temos, então, 
  \[
    ||x_{m}-y||_{p} < \biggl(\sum\limits_{i=1}^{n}\frac{\varepsilon^{p}}{n}\biggr)^{\frac{1}{p}} = \varepsilon.
  \]
  Com relação a \(\mathbb{Q},\) veja que \(\mathbb{Q}\) não é completo, pois 
  \[
    x_{n} = \biggl(1 + \frac{1}{n}\biggr)^{n}\in \mathbb{Q},
  \]
  mas \(x_{n}\overbracket[0pt]{\longrightarrow}^{\to }e\in \mathbb{R}\setminus{\mathbb{Q}}.\) 

\end{example}
\begin{def*}
  Um espaço vetorial normado \((V, ||\cdot ||\) é chamado \textbf{espaço de Banach} se é um espaço métrico completo com a distância induzida pela norma, isto é,
 \(d(x, y) = ||x-y||, x, y\in V.\square\)
\end{def*}
\begin{example}
  \(\mathcal{C}([a, b], \mathbb{R})\) com a norma 
  \[
    ||f|| = \max_{x\in [a, b]}|f(x)|,\quad f\in \mathcal{C}([a, b], \mathbb{R}),
  \]
é completo. Em outras palavras, \(\mathcal{C}([a, b]), ||||\) é um espaço de Banach.

Com efeito, seja \(\{h_{n}\}\) de Cauchy em \(\mathcal{C}([a, b])\) para todo \(x\in [a, b]\). Segue que 
  \[
    |h_{n}(x) - h_{m}(x)|\leq ||h_{n}-h_{m}||
  \]
  Logo, \(\{h_{n}(x)\}\subseteq{\mathbb{R}}\) é de Cauchy para cada x em \([a, b]\). Seja \(h(x)\) o limite de \(\{h_{n}(x)\}\) e
  considere \(h:[a, b]\rightarrow \mathbb{R}, x\mapsto h(x) = \lim_{n\to \infty}h_{n}(x).\) Mostraremos que \(h\in \mathcal{C}([a, b])\) e 
    \[
      ||h_{n}-h|| \rightarrow 0, n\rightarrow \infty.
    \]
Dado \(\varepsilon >0,\) do fato de \(\{h_{n}\}\) ser de Cauchy, existe \(n_{0}\in \mathbb{N}\) tal que 
  \[
    |h_{n}(x) - h_{m}(x)|\leq ||h_{n}-h_{m}|| < \frac{\varepsilon }{3},
  \] 
para todo \(x\in[a, b]\) e \(n, m\geq n_{0}\). Fazendo \(n\longrightarrow \infty,\) obtemos 
  \[
    |h(x)-h_{m}(x)|\leq \frac{\varepsilon }{3},\quad m\geq n_{0}, \forall x\in[a, b].
  \]
  Assim, se \(h\in \mathcal{C}([a, b]), \) então 
  \[
    \underbrace{\sup_{x}\{|h(x) - h_{m}(x)|\}}_{||h-h_{m}||}\leq \frac{\varepsilon }{3},\quad m\geq n_{0}.
  \]
e \(h_{n}\overbracket[0pt]{\longrightarrow}^{n\to \infty}h\) em \(\mathcal{C}([a, b]).\) Mostremos a continuidade de h.
Seja \(y\in [a, b]\) e verifiquemos que h é contínua em y. Dado \(\varepsilon >0,\) existe \(n_{0}\in \mathbb{N}\) tal que 
valha
  \[
    |h(x)-h_{m}(x)|\leq \frac{\varepsilon }{3},\quad m\geq n_{0}, \forall x\in[a, b].
  \]
Como \(h_{n_{0}}\in \mathcal{C}([a, b]),\) existe \(\delta > 0\) tal que 
  \[
    |x-y| < \delta \Rightarrow |h_{n_{0}}(x) - h(y)| < \frac{\varepsilon }{3}.
  \]
  Com isso, temos o seguinte - se \(|x-y| < \delta \), então 
  \[
    |h(x)-h(y)|\leq |h(x)-h_{n_{0}}(x)| + |h_{n_{0}}(x) - h(y)| < \frac{\varepsilon }{3} + \frac{\varepsilon }{3} = \frac{2\varepsilon }{3} < \varepsilon .
  \]
Portanto, h é contínua em y.
\end{example}
\begin{example}
  O espaço \((\ell_{\infty}, ||\cdot ||_{\infty}), \ell_{\infty}\coloneqq \biggl\{\{x_{n}:\sup_{i}|x_{i}|<\infty\}\biggr\}\) é Banach. A norma em \(\ell_{\infty}\) é dada por 
  \[
    ||x||_{\infty}=\sup\{|x_{i}|:i\in \mathbb{N}\},\quad x = \{x_{n}\}\in \ell_{\infty}.
  \]

  De fato, seja \(\{x_{n}\}\subseteq{\ell_{\infty}}\) de Cauchy. Escreveremos 
  \[
    x_{n} = \{x_{n}^{k}\}_{k}, \quad n = 1, 2, \cdots.
  \]
  Temos 
  \[
    |x_{n}^{k}-x_{m}^{k}|\leq \sup_{j}|x_{n}^{j}-x_{m}^{j} |. = ||x_{n}-x_{m}||,
  \]
garantindo que \(\{x_{n}^{k}\}_{k}\) é de Cauchy na reta \(\mathbb{R}\). Sendo \(\mathbb{R}\) um espaço completo, podemos tomar
 \(y_{n} = \lim_{k\to \infty}x_{n}^{k}\) e definir \(y = \{y_{n}\}\). A partir disso, devemos mostrar que 
\begin{itemize}
  \item[1)] \(y\in \ell_{\infty}\);
  \item[2)] \(||x_{n}-y||\longrightarrow 0\) quando \(n\longrightarrow\infty\)
\end{itemize}
  Dado \(\varepsilon > 0\), existe \(n_{0}\in \mathbb{N}\) tal que 
  \[
    |x_{n}^{k}-x_{m}^{k}| < \frac{\varepsilon }{2},\quad m,n\geq n_{0}.
  \]
  Isso implica que 
  \[
    |y_{n}-x_{n}^{k}|\leq \frac{\varepsilon }{2}, m\geq n_{0}.
  \]
  Logo, \(x_{m}^{k}-\frac{\varepsilon }{2} < y_{n} < x_{m}^{k}+\frac{\varepsilon }{2}\) para \(m, n\geq n_{0}.\) Como
  \(\{x_{m}^{k}\}_{k}\) é limitada, segue que \(\{y_{n}\}_{n}\) também é. Assim, 
  \[
    |y_{n}-x_{m}^{k}|\leq \frac{\varepsilon }{2},\quad m\geq n_{0}.
  \]
  Agora, 
  \[
    ||y-x_{n}||_{\infty}=\sup_{i}\{|y_{i}-x_{n}^{i}|\}\leq \frac{\varepsilon }{2},\quad m\geq n_{0}.
  \] 
  Portanto, a sequência de Cauchy convergiu para y.
\end{example}
\newpage

\section{Aula 15 - 26/10/2023}
\subsection{Motivações}
\begin{itemize}
  \item Mais Exemplos de Espaços de Banach;
  \item Propriedade Topológicas de Espaços Métricos Completos.
\end{itemize}
\subsection{Espaços de Banach e Topologia de Espaços Completos}
  Continuemos com outro exemplo de um espaço de Banach:
  \begin{example}
    O espaço métrico \(\mathcal{B}(X, \mathbb{R})\coloneqq \{f:X\rightarrow \mathbb{R}: f\text{ é limitada}\}\) é um espaço de Banach com a norma 
      \[
        ||f||=\sup_{x\in X}\{|f(x)|\}.
      \]

    De fato, seja \(\{f_{n}\}\subseteq{\mathcal{B}(X, \mathbb{R})}\) uma sequência de Cauchy. Logo, para todo \(\varepsilon >0\), existe \(n_{0}\in \mathbb{N}\) tal que 
  se \(m, n\geq n_{0}\), 
    \[
      |f_{n}-f_{m}|\leq \sup_{x\in X}\{f_{n}(x)-f_{m}(x)\} = ||f_{n}-f_{m}|| < \varepsilon .
    \]
  para todo x em X. Logo, \(\{f_{n}(x)\}\subseteq{\mathbb{R}}\) é uma sequência de Cauchy e, portanto, converge para algum \(f(x),\) pois \(\mathbb{R}\) é um espaço métrico completo.
Assim, temos \(f:X\rightarrow \mathbb{R}\) definida por 
  \[
    f(x) = \lim_{n\to \infty}f_{n}(x),\quad x\in X.
  \]
  Fazendo \(n\longrightarrow \infty\) na primeira desigualdade e colocando \(m=n_{0}\), obtemos 
  \[
    |f(x)-f_{n_{0}}(x)|\leq \varepsilon .
  \]
  Disto segue que 
  \[
    |f(x)| - |f_{n_{0}}(x)|\leq \varepsilon \Rightarrow |f(x)|\leq \varepsilon + |f_{n_{0}}(x)|.
  \]
  Como \(f_{n_{0}}\in \mathcal{B}(X, \mathbb{R}),\) concluímos que \(f\in \mathcal{B}(X, \mathbb{R}).\) Por aquela mesma desigualdade, segue que 
  \[
    \sup_{x\in X}\{f(x)-f_{m}(x)\}\leq \varepsilon ,
  \]
  ou seja, \(||f-f_{m}|| < \varepsilon \) para todo \(m\geq n_{0}\), o que equivale a dizer que \(f_{m}\) converge para f em \(\mathcal{B}(X, \mathbb{R}).\)
  \end{example}
  Um resultado muito importante sobre a topologia dos espaços métricos completos diz que todo subconjunto fechado do espaço métrico é completo e vice-versa - em outras
palavras, há uma equivalência entre completude e ser fechado! 
 \begin{prop*}
   Seja \((M, d)\) um espaço métrico.
 \begin{itemize}
   \item[a)] Se M é completo, então \(F\subseteq{M}\) é completo se F é fechado.
   \item[b)] Se \(F\subseteq{M}\) é completo, então F é fechado.
 \end{itemize}
\end{prop*}
\begin{proof*}
  b) Se F é completo e x pertence ao seu fecho, então 
 \begin{itemize}
   \item[i)] Existe \(\{x_{n}\}\subseteq{F}\) tal que \(x_{n}\overbracket[0pt]{\longrightarrow}^{n\to \infty}x\)
   \item[ii)] Como \(\{x_{n}\}\) é convergente, ela é de Cauchy em F e converge em F.
 \end{itemize}
 Portanto, \(x\in F\) e \(\overline{F}\subseteq{F},\) donde segue que \(F = \overline{F}\) e que F é fechado.

 a) Se F é fechado, toda sequência em F converge para um elemento de F. Sendo assim, dada uma sequência de Cauchy \(\{x_{n}\}\subseteq{F},\) ela é,
em particular, de Cauchy em M, o qual é completo. Sendo assim, existe \(x\in M\) tal que \(x_{n}\overbracket[0pt]{\longrightarrow}^{n\to \infty}x.\) No entanto,
já que F é fechado, o limite de uma sequência de F deve estar lá, o que significa que \(x\in F\). Portanto, \(\{x_{n}\}\) de Cauchy tem seu limite em F, provando a 
completude do mesmo. \qedsymbol
\end{proof*}
\newpage

\section{Aula 16 - 07/11/2023}
\subsection{Motivações}
\begin{itemize}
  \item Completamento de Espaços Métricos;
  \item Contrações;
  \item Princípio da Contração de Banach.
\end{itemize}
\subsection{Completamento}
  Começamos com um exemplo já visto e já provado (vide Aula 8):
\begin{example}
  Todo espaço métrico \((X, \rho )\) pode ser imerso isometricamente no espaço vetorial normado \((\mathcal{B}(X;\mathbb{R}), \Vert \cdot \Vert_{\infty})\) das funções
limitadas com a norma do supremo, ou seja, 
  \[
    \Vert f \Vert_{\infty} = \sup\{|f(x)|: x\in X\}
  \]
\end{example}
  Este exemplo foi utilizado pois faremos ideias similares ao estudar o \textit{completamento} de espaços métricos - estaremos ``mergulhando'' espaços métricos incompletos
isometricamente de um jeito específico em espaços métricos completos, essencialmente ``completando'' ele.
\begin{def*}
  Um \textbf{completamento} de um espaço métrico (M, d) é um par \(((\hat{M}, \rho ), f)\) satisfazendo:
\begin{itemize}
  \item[i)] \((\hat{M}, \rho )\) é um espaço métrico completo;
  \item[ii)] \(f:M\rightarrow \hat{M}\) é uma imersão isométrica;
  \item[iii)] \(f(M)\) é denso em \(\hat{M}\), ou seja, \(\overline{f(M)} = \hat{M}\).
\end{itemize}
\end{def*}
\begin{lemma*}
  Sejam \((M, d), (N, \rho )\) espaços métricos, \(S\subseteq{M}\) e N completo. Se \(f:S\rightarrow N\) é contínua, então existe uma única extensão contínua de f a \(\overline{S}.\)
\end{lemma*}
\begin{proof*}
  Seja \(x\in \overline{S}.\) Existe uma sequência \(\{x_{n}\}\in S\) tal que \(x_{n}\overbracket[0pt]{\longrightarrow}^{n\to \infty}x\). A sequência \(\{f(x_{n})\}\subseteq{f(S)}\)
é de Cauchy (exercício) e, sendo M completo, existe uma função h tal que \(f(x_{n})\overbracket[0pt]{\longrightarrow}^{n\to \infty}h(x)\). 

  Precisamos checar a boa-definição de h. Se \(\{y_{n}\}\subseteq{S}\) é uma sequência tal que \(y_{n}\longrightarrow x, n\rightarrow \infty\), então \(d(x_{n}, y_{n})\longrightarrow 0, n\rightarrow \infty\).
Com isso, \(d(f(x_{n}), f(y_{n}))\longrightarrow 0, n\rightarrow \infty\) e a boa-definição de h é garantida. Definimos, então, \(h:\overline{S}\rightarrow N\) por
 \(h(x) = \lim_{n\to \infty}f(x_{n}),\) em que \(\{x_{n}\}\subseteq{S}, x_{n}\overbracket[0pt]{\longrightarrow}^{n\to \infty}x.\) Note que a sequência contínua
  \(\{x_{n} = x\}\) cumpre \(h(x) = \lim_{n\to \infty}f(x) = f(x),\) tal que h estende a f. Portanto, provamos a existência e unicidade da extensão. \qedsymbol
\end{proof*}
  \textit{Existir uma extensão} de \(f:S\rightarrow N\) a \(\overline{S}\) significa que existe uma função \(h:\overline{S}\rightarrow N\) tal que \(f = h |_S\), ou seja,
 \(h(x) = f(x)\) para todo \(x\in S\).
\begin{theorem*}
  Todo espaço métrico admite um completamento. Além disso, esse completamento é único, a menos de isometria.  
\end{theorem*}
  Quando dizemos que um completamento é único, a menos de isometria, queremos dizer que, se \(((\hat{M}, \rho ), f)\) e \(((\tilde{M}, \beta ), g)\) são completamentos de M,
então \((\tilde{M}, \beta )\) e \((\hat{M}, \rho )\) são isométricos. Faz sentido falar em unicidade por isometria porque quando dois espaços métricos são isométricos, eles podem
ser vistos com essencialmente a mesma estrutura métrica - os abertos, conexidade, etc. 
\begin{proof*}
  Vimos que \(\mathcal{B}(M, \mathbb{R})\) é completo e que existe uma imersão isométrica \(T:M\rightarrow \mathcal{B}(M, \mathbb{R}).\) Considere o conjunto \(\hat{M} = \overline{T(M)}\)
e observe que \(\hat{M}\) é fechado (já que o fecho do fecho é o próprio fecho, tal que \(\overline{\hat{M}} = \overline{\overline{T(M)}} = \overline{T(M)} = \hat{M}\) e, portanto,
completo, pois conjuntos fechados dentro de completos são, também, completos. Assim, \(((\hat{M}, \Vert \cdot \Vert _{\infty}), T)\) é um completamento de M. Resta mostrarmos a unicidade e, para
isso, utilizaremos o lema.

  Sejam \(\hat{M}, \tilde{M}\) espaços métricos completos e isometrias \(T:M\rightarrow \tilde{M}\) e \(S:M\rightarrow \hat{M},\) com imagens densas. Definimos a isometria
 \(I:\tilde{M}\rightarrow \hat{M}\) como a única extensão contínua a \(\tilde{M} = \overline{T(M)}\) da isometria 
  \[
    V = S \circ{T^{-1}}:T(M)\rightarrow S(M).
  \]
\end{proof*}
\subsection{Contrações} 
\begin{def*}
  Seja \((X, \rho )\) um espaço métrico completo. Uma aplicação \(T:X\rightarrow X\) é uma \textbf{contração} em X se existe \(0 < \kappa < 1,\) tal que 
  \[
    \rho (T_{x}, T_{y})\leq \kappa \rho (x, y),\quad \forall x, y\in X.\quad\square
  \]
\end{def*}
  As contrações permitem-nos falar de um dos resultados mais fundamentais da matéria, com inúmeras aplicações - o \textit{Princípio da Contração de Banach}.
\hypertarget{banach_contraction}{\begin{theorem*}[Princípio da Contração de Banach]
  Se X é um espaço métrico completo e T é uma contração em X, então T possui um único ponto fixo em X, isto é, existe um único \(x_{0}\in X\) tal que \(T(x_{0}) = x_{0}.\)
\end{theorem*}}
\begin{proof*}
  Primeiramente, vamos mostrar que T tem no máximo um ponto fixo. Para isso, sejam x e y dois pontos fixos de T, de forma que 
  \[
    \rho (x, y) = \rho (Tx, Ty)\leq \kappa \rho (x, y) < \rho (x, y).
  \]
  Logo, \(x=y.\) Agora, vamos provar a existência de um ponto fixo. 

  Seja \(x\in X\) e considere a sequência \(\{x, Tx, T^{2}x, \dotsc\}\) de X. Dados \(n, p\in \mathbb{N},\) vale
 \begin{align*}
  \rho (T^{n+p}x, T^{n}x) &\leq \kappa^{n}[\rho (T^{p}x, T^{p-1}x) + \dotsc + \rho (Tx, x)]\\
                          &\leq \kappa^{n}[\kappa^{p-1}\rho (Tx, x) + \dotsc + \rho(Tx, x)]\quad \text{(Desigualdade Triangular)}\\
                          &= \kappa ^{n}\underbrace{[\kappa^{p-1} + \dotsc + 1]}_{\text{série geométrica}}\rho(Tx, x)\\
                          &= \kappa ^{n}\frac{1}{1-\kappa }\rho (Tx, x).
 \end{align*}
 Como \(0 < \kappa < 1, \kappa ^{n}\overbracket[0pt]{\longrightarrow}^{n\to \infty}0.\) Logo, \(\{T^{n}x\}\) é convergente para algum \(x_{0}\in X.\)
Para ver que \(x_{0}\) é ponto fixo de T, note que 
  \[
    Tx_{0} = T \lim_{n\to \infty}T^{n}x = \lim_{n\to \infty}T^{n+1}x = x_{0}.
  \]
  Portanto, T tem um único ponto fixo. \qedsymbol
\end{proof*}
  Um comentário do escritor: não apenas mostramos que a função tem um ponto fixo, como também provamos que \textbf{aplicações repetidas de T convergem muito rápido para o ponto fixo.}
Isso pode ser visto no meio da prova de que \(\{T^{n}x\}\) é de Cauchy, já que, a cada aplicação de T a si mesma, multiplicou-se por \(\kappa \) o fator \(\kappa^{n}.\)
\newpage

\section{Aula 17 - 14/11/2023}
\subsection{Motivações} 
\begin{itemize}
  \item Compacidade e Coberturas;
\end{itemize}
\subsection{Coberturas e Limitação Total}
\begin{def*}
  Seja \((X, \rho )\) um espaço métrico e \(E\subseteq{X}.\) Uma família \(\{V_{\alpha }\}_{\alpha \in A}\) de subconjuntos de X satisfazendo 
  \[
    E \subseteq \bigcup_{\alpha \in A}^{}{V_{\alpha }}
  \]
  é chamada \textbf{cobertura}. Se a cobertura \(\{V_{\alpha }\}_{\alpha \in A}\) for composta de conjuntos abertos, chamaremos ela de \textbf{cobertura aberta}. \(\square\)
\end{def*}
\begin{example}
  Para \(\mathbb{R}\), \(\{[-n, n]\}\) é uma cobertura e \(\{(-n, n)\}\) é uma cobertura aberta, \(n\in \mathbb{N}\).
\end{example}
\begin{def*}
  Seja \(\{V_{\alpha }\}_{\alpha \in A}\) uma cobertura de E. Se \(\Omega \subseteq{A}\) e 
  \[
    E \subseteq{\bigcup_{\alpha \in \Omega }^{}{V_{\alpha }}},
  \]
então \(\{V_{\alpha }\}_{\alpha \in \Omega }\) é uma \textbf{subcobertura} de \(\{V_{\alpha }\}_{\alpha \in A}.\) Se \(\{U_{\gamma }\}_{\gamma \in \Omega }\) é uma cobertura 
de E tal que para todo \(\gamma \in \Omega, \) existe \(\alpha \in A\) tal que \(U_{\gamma }\subseteq{V_{\alpha }}\), então \(\{U_{\gamma }\}_{\gamma \in\Omega }\)
é chamado de \textbf{refinamento.} \(\square\)
\end{def*}
\begin{def*}
  Dizemos que \(E\subseteq{X}\) é \textbf{totalmente limitado} se para qualquer \(\varepsilon > 0\), existem \(x_{1}, x_{2}, \dotsc, x_{n}\in E\) tais que 
  \[
    E \subseteq{\bigcup_{i=1}^{n}{B_{\varepsilon }(x_{i})}}.\quad\square
  \]
\end{def*}
\begin{example}
  Seja \((X, \rho )\) um espaço métrico.
\begin{itemize}
  \item[1)] Se \(E\subseteq{X},\) então, para todo \(\varepsilon >0,\) os conjuntos 
  \[
    \{B_{\varepsilon }(x)\}_{x\in X}\quad\&\quad \{B_{\varepsilon }(x)\}_{x\in E}
  \]
são coberturas de E.
  \item[2)] Se \(E = \{e_{1}, e_{2}, \dotsc, e_{n}\}\subseteq{X},\) então para todo \(\varepsilon >0,\) a coleção \(\{B_{\varepsilon }(x_{i})\}_{i=1}^{n}\) é uma subcobertura
da cobertura \(\{B_{\varepsilon }(x)\}_{x\in X}\).
\item[3)] Se \(E\subseteq{X}\), então para todo \(\varepsilon >0\), a coleção \(\{B_{\frac{\varepsilon }{2}(x)}\}_{x\in X}\) é um refinamento da cobertura \(\{B_{\varepsilon }(x)\}_{x\in X}\).
\item[4)] Seja \(E\subseteq{X}\) tal que existe \(r > 0\) e \(x_{1}, x_{2}, \dotsc, x_{n}\in E\) tal que 
  \[
    E\subseteq{\bigcup_{i=1}^{n}{B_{r}(x_{i})}},
  \]
  então \(\mathrm{diam}(E) < \infty.\)
\end{itemize}
\end{example}
\begin{prop*}
  Todo conjunto totalmente limitado é limitado. Se E for totalmente limitado, então \(\overline{E}\) é totalmente limitado.
\end{prop*}
 Observe que E pode ser limitado e não ser totalmente limitado.
\begin{example}
  Tome o espaço \((\ell_{\infty}, \Vert \cdot  \Vert_{\infty}),\) em que \(\ell_{\infty}=\{\{a_{n}\}: \sup\{|a_{n}|\} < \infty\}, \Vert a_{n} \Vert = \sup\{|a_{n}|:n\in \mathbb{N}.\}\)
Para cada \(i\in \mathbb{N},\) seja \(e_{i}\) a sequência nula, exceto na i-ésima entrada, que vale 1. Temos \(\Vert e_{i} \Vert_{\infty} = 1\) para todo i em \(\mathbb{N}.\) Logo,
o conjunto \(L = \{e_{i}:i\in \mathbb{N}\}\) é limitado. No entanto, para \(\varepsilon = \frac{1}{2},\) como \(\Vert e_{i}-e_{j} \Vert_{\infty} = 1\) para \(i\neq j,\)
segue que 
  \[
    L\not\subseteq{\bigcup_{i=1}^{n}{B_{\frac{1}{2}(x_{i})}}}\quad \forall n\in \mathbb{N}.
  \]
\end{example}
\begin{example}[Exercício]
  Em \(\mathbb{R}^{n},\) um conjunto é totalmente limitado se, e somente se, é limitado.
\end{example}
\begin{proof*}
  Consideremos em \(\mathbb{R}^{n}\) a métrica do máximo. Se \(E\subseteq{\mathbb{R}^{n}}\) é limitado, existe \(r > 0\) e \(x = (x_{1}, \dotsc, x_{n})\in E\) tal que 
  \[
    E\subseteq{B_{r}(x)} = \prod\limits_{i=1}^{n}(x_{i}-r, x_{i}+r).
  \]
  Dado \(\varepsilon >0\), existe \(a_{1}^{n}, \dotsc, a_{m_{\varepsilon }^{n}}^{n}\in(x_{n}-r, x_{n}+r)\) satisfazendo 
  \[
    (x_{n}-r,x_{n}+r) \subseteq{\bigcup_{j_{n}=1}^{m_{\varepsilon }^{n}}{(a_{j_{n}}^{n}-\varepsilon , a_{j_{n}}^{n}+\varepsilon )}}.
  \]
  Logo, 
 \begin{align*}
  E \subseteq{B_{r}(x)} &=\prod\limits_{i=1}^{n}(x_{i}-r, x_{i}+r)\\
                        &\subseteq{}\prod\limits_{i=1}^{n-1}(x_{i}-r, x_{i}+r)\times \bigcup_{j_{n}=1}^{m_{\varepsilon }^{n}}{(a_{j_{n}}^{n}-\varepsilon , a_{j_{n}}^{n}+\varepsilon )}\\
                        &\subseteq{\bigcup_{j_{n}=1}^{m_{\varepsilon }^{n}}\prod\limits_{i=1}^{n-1}(x_{i}-r, x_{i}+r)\times (a_{j_{n}}^{n}-\varepsilon , a_{j_{n}}^{n}+\varepsilon ).}
 \end{align*}
 Repetimos o argumento para o intervalo \((x_{n-1}-\varepsilon , x_{n-1}+\varepsilon )\) e obtemos 
  \[
    (x_{n-1}-r, x_{n-1}+r)\subseteq{\bigcup_{j_{n-1}=1}^{m_{\varepsilon }^{n}}{(a_{j_{n-1}}^{n-1}-\varepsilon, a_{j_{n-1}}^{n-1}+\varepsilon  )}}
  \]
Consequentemente,
\begin{align*}
  &E\subseteq{\bigcup_{j_{n}=1}^{m_{\varepsilon }^{n}}\prod\limits_{i=1}^{n-1}(x_{i}-r, x_{i}+r)\times (a_{j_{n}}^{n}-\varepsilon , a_{j_{n}}^{n}+\varepsilon ).}\subseteq{}\\
  &\subseteq{\bigcup_{j_{n-1}=1}^{m_{\varepsilon }^{n-1}}\bigcup_{j_{n}=1}^{m_{\varepsilon }^{n}}\prod\limits_{i=1}^{n-2}(x_{i}-r, x_{i}+r)\times(a_{j_{n-1}}^{n-1}-\varepsilon , a_{j_{n-1}}^{n-1}+\varepsilon )\times (a_{j_{n}}^{n}-\varepsilon , a_{j_{n}}^{n}+\varepsilon ).}
\end{align*}
  Por indução, concluímos que 
 \begin{align*}
  E \subseteq{B_{r}(x)} &=\prod\limits_{i=1}^{n}(x_{i}-r, x_{i}+r)\\
                        &\subseteq{}\prod\limits_{j_{1}=1}^{m_{\varepsilon }^{1}}\dotsc \prod\limits_{j_{n}=1}^{m_{\varepsilon }^{n}}\prod\limits_{i=1}^{n}(a_{j_{i}}^{i}-\varepsilon , a_{j_{i}}^{i}+\varepsilon )\\
                        &=\bigcup_{j_{1}=1}^{m_{\varepsilon }^{1}}{\dotsc}\bigcup_{j_{n}=1}^{m_{\varepsilon }^{n}}{B_{\varepsilon }(a_{j, k}}),
 \end{align*}
 com \(a_{j, k} = (a_{j_{k}1}^{1},\dotsc,a_{j_{k}n}^{n})\), para cada \(k=1, \dotsc, m_{\varepsilon }^{k}\) e \(j=j_{1},\dotsc,j_{n}\). Portanto, E é totalmente limitado. \qedsymbol
\end{proof*}
\begin{theorem*}
  Seja \((X, \rho )\) um espaço métrico e \(E\subseteq{X}.\) São equivalentes:
 \begin{itemize}
  \item[1)]E é completo e totalmente limitado
  \item[2)] Toda sequência em E possui subsequência convergente em E;
  \item[3)] Toda cobertura aberta de E possui subcobertura finita.
 \end{itemize}
\end{theorem*}
\begin{proof*}
  Faremos a prova mostrando a equivalência entre 1 e 2, depois que 1 e 2 juntos implicam 3 e, por fim, que 3 implica 2.

  \(1 \Rightarrow 2):\) Seja \(\{x_{n}\}\subseteq{E}\) sem subsequência convergente. Como E é totalmente limitado para \(\varepsilon = \frac{1}{2},\) existe
 \(y_{1}, \dotsc, y_{n}\in E\) tais que 
  \[
    E\subseteq{\bigcup_{i=1}^{n}{B_{\frac{1}{2}(y_{i})}}}.
  \]
  Então, existe i tal que \(\{x_{n}\}\cap B_{\frac{1}{2}}(y_{i})\) contém infinitos elementos. Seja \(N_{1}\) o conjunto \(N_{1}\subseteq{\mathbb{N}}\) tal que 
  \[
    \{x_{k}\}\subseteq{\{x_{n}\}\cap B_{\frac{1}{2}}(y_{i}),}\quad k\in N_{1}.
  \]
  Como \(E\cap B_{\frac{1}{2}}(y_{i})\) é totalmente limitado, para \(\varepsilon = \frac{1}{2^{2}},\) existem \(z_{1}, \dotsc, x_{n_{2}}\in E\cap B_{\frac{1}{2}(y_{i})}\)
com 
  \[
    E\cap B_{\frac{1}{2}}(y_{i})\subseteq{\bigcup_{i=1}^{n_{2}}{B_{\frac{1}{2}}(z_{i})}}.
  \]
  Analogamente ao que foi feito anteriormente, existe \(N_{2}\subseteq{N_{1}}\) com \(x_{k}\in E\cap B_{\frac{1}{2^{2}}}(z_{i}), k \in N_{2}\). Indutivamente, construímos uma coleção
\(\{B_{n}\}_{n}\) de bolas com diâmetro \(\mathrm{diam}(B_{n})\leq \frac{1}{2^{n}}\) tais que \(x_{n}\in E\cap B_{k}, n\in N_{k}\) e que \(N_{1}\supseteq{N_{2}}\dotsc \supseteq{N_{k}}\supseteq{\dotsc}\)
Para cada \(k\in N_{k},\) seja \(y_{k}\in E\cap B_{k}.\) A sequência \(\{y_{k}\}\) é uma subsequência de \(\{x_{n}\}\) que é de Cauchy em E (Verfique). Como E é completo, 
 \(y_{k}\) deve ser convergente, o que é uma contradição ao que havia sido assumido.

  \(2 \Rightarrow 1):\) Suponha que toda sequência \(\{x_{n}\}\) possui subsequência \(\{x_{n_{k}}\}\) convergente e que E não seja completo. Com isso, existe uma
sequência de Cauchy sem subsequência convergente. Já obtivemos uma contradição, pois é diretamente contrário à hipótese de que toda sequência de Cauchy converge. Agora, suponha que E
não seja totalmente limitado e considere \(\varepsilon >0\) tal que 
  \[
    E\not\subseteq{B_{\varepsilon }(x)}, \quad x\in E.
  \]
  Existe \(x_{1}\in E\setminus{B_{\varepsilon }(x)}\). Além disso, também vale que 
  \[
    E\not\subseteq{B_{\varepsilon }(x)\cup B_{\varepsilon }(x_{1})}.
  \]
  Com isso, existe também \(x_{2}\in E\setminus{(B_{\varepsilon }(x)\cup B_{\varepsilon }(x_{1}))}\). Chamando 
 \(x_{0} = x\), indutivamente, construímos a sequência \(\{x_{n}\}\subseteq{E}\) tal que 
  \[
    d(x_{n}, x_{m})\geq \varepsilon, \quad n\neq m,\quad n, m\in \mathbb{N}.
  \]
  Justamente por isso, \(\{x_{n}\}\) não possui subsequência convergente. (Continua na próxima aula).
\end{proof*}
\newpage

\section{Aula 18 - 16/11/2023}
\subsection{Motivações}
\begin{itemize}
  \item Continuando o Teorema;
  \item Fechamento e Completude.
\end{itemize}
\subsection{As Faces da Compacidade}
  Vamos dar continuidade a prova. 
 \begin{proof*}
  \(1 + 2 \Rightarrow 3):\) Seja \(\{V_{\alpha }\}_{\alpha \in \Omega }\) uma cobertura aberta de E. A ideia
é assumir a existência de uma cobertura sem qualquer subcobertura finita e chegar em uma contradição.

\begin{claim*}Para todo x em E, existe \(r_{x}> 0\) tal que \(B_{r_{x}}(x)\cap E \neq\emptyset, \) então 
 \(B_{r_{x}}(x)\subseteq V_{\alpha }\) para algum \(\alpha \in \Omega \).
\end{claim*}

  De fato, suponha que exista \(B_{r}\) uma bola tal que \(B_{r}\cap E \neq\emptyset\) e \(B_{r}\not\subseteq V_{\alpha },\) para todo
 \(\alpha \in \Omega \). Para cada natural \(n,\) vale que 
  \[
    B_{\frac{1}{2^{n}}}(x)\cap E \neq\emptyset\quad\&\quad B_{\frac{1}{2^{n}}}(x)\not\subseteq V_{\alpha }, \quad\alpha \in \Omega .
  \] 
  Considere \(x_{n}\in B_{\frac{1}{2^{n}}}(x)\cap E\), \(n\in \mathbb{N}.\) Assim, uma sequência \(\{x_{n}\}\) em E é
de Cauchy. Agora, por hipótese, \(x_{n}\overbracket[0pt]{\longrightarrow}^{n\to \infty}y\) para algum \(y\in E\) por
E ser aberto. Como \(\{V_{\alpha }\}_{\alpha \in \Omega }\) é uma cobertura de E, existe \(\alpha_{y} \in \Omega \) tal que
\(y\in V_{\alpha_{y}}\). Sendo \(V_{\alpha_{y}}\) aberto, existe um raio \(r > 0\) tal que \(B_{r}(y)\subseteq V_{\alpha _{y}}.\)
Seja \(n\) satisfazendo \(\frac{1}{2^{n}} < \frac{r}{3}\), de forma que \(B_{\frac{1}{2^{n}}}(x)\subseteq B_{r}(y)\), o que
contraria a propriedade 
  \[
    B_{\frac{1}{2^{n}}}(x)\not\subseteq V_{\alpha }, \quad\alpha \in \Omega .
  \]
  A seguir, extrairemos uma subcobertura finita a partir da encontrada. Para cada \(x\in E, \) considere
 \(\varepsilon > 0\) da afirmação, a coleção \(\{B_{\varepsilon }(x)\}_{x\in E}\) cobre E. Observe que, com isso,
 temos um refinamento da cobertura \(\{V_{\alpha }\}_{\alpha \in \Omega }\) e, além disso, existe \(\alpha_{x}\subseteq \Omega \)
 tal que 
  \[
    B_{\varepsilon }(x)\subseteq V_{\alpha_{x}}
  \]
  para algum \(\alpha_{x}\in \Omega \). Como E é totalmente limitado, existem \(x_{1}, \dotsc, x_{n}\) tal que 
  \[
    E\subseteq \bigcup_{i=1}^{n}B_{\varepsilon }(x_{i})\subseteq \bigcup_{i=1}^{n}V_{\alpha_{x_{i}}}.
  \]
  Com isso, obtivemos a cobertura finita para E desejada. \qedsymbol
 \end{proof*}
 \begin{def*}
  Diremos que um subconjunto E de um espaço métrico M é \textbf{compacto} se toda cobertura aberta de E admite subcobertura finita. \(\square\)
 \end{def*}
 De imediato, podemos reescrever o teorema como
 \begin{crl*}
  Seja \((X, \rho )\) um espaço métrico e \(E\subseteq X\). São equivalentes:
 \begin{itemize}
  \item[1)] E é completo e totalmente limitado;
  \item[2)] Toda sequência em E possui uma subsequência convergente em E;
  \item[3)] E é compacto.
 \end{itemize}
 \end{crl*}
 \begin{example}
  \begin{itemize}
    \item[1)] \([a, b]\) é compacto em \(\mathbb{R}\) com a métrica usual;
    \item[2)] \([a, b)\) não é compacto em \(\mathbb{R}\) com a métrica usual;
    \item[3)] \(M \neq\emptyset\) munido da métrica discreta é compacto se, e somente se, 
  M finito não é compacto.

    Este segue do fato que, se M é infinito, então 
    \[
      M = \bigcup_{x\in M}^{}B_{\frac{1}{2}}(x) = \bigcup_{x\in M}^{}\{x\}.
    \]
    \item[4)] A esfera S em \(\ell_{\infty}\), dada por 
    \[
      S = \{x\in \ell_{\infty}: \Vert x \Vert = 1\},
    \]
  não é compacta.

    De fato, da aula passada, se considerarmos 
      \[
        e_{i} = \{a_{n}\}_{n},
      \]
  com \(a_{n} = 1\) e \(a_{i} = 0, i\neq n.\) Vimos que 
    \[
      \Vert e_{i}-e_{n} \Vert_{\infty} = 1,\quad i\neq j\quad\&\quad \Vert e_{i} \Vert = 1.
    \]
  A coleção \(\{e_{n}\}_{n\in \mathbb{N}}\) é limitado e não é totalmente limitado em S.
  \end{itemize}
 \end{example}
\begin{theorem*}
  Um subconjunto \(E\subseteq \mathbb{R}^{n}\) é compacto se, e somente se, E é fechado e limitado.
\end{theorem*}
\begin{proof*}
  Pela carcterização de totalmente limitado no espaço euclidiano, vale que E é totalmente limitado se, e somente se, E é limitado.
A caracterização de compactos diz que E é compacto se, e somente se, E é completo e totalmente limitado.

  Logo, se \(E\subseteq \mathbb{R}^{n}\) é compacto, então E é completo e, portanto, fechado. Caso E seja limitado no espaço euclidiano,
então E é totalmente limitado. Por outro lado, se E é fechado, como \(\mathbb{R}^{n}\) é completo, então E é completo. \qedsymbol
\end{proof*}
\begin{theorem*}
  Sejam \((X, \rho )\) e \((M, d)\) espaços métricos. Se X é compacto e \(f:X\rightarrow M\) é contínua e sobrejetora, então M é compacto.
\end{theorem*}
\begin{proof*}
  Mostremos que M é compacto pela definição. Seja \(\{A_{\alpha }\}_{\alpha }\) uma cobertura aberta de M. Como f é contínua,
 \(f^{-1}(A_{\alpha })\) é aberto. Além disso, como \(f(X) = M,\) e \(M = \bigcup_{\alpha \in A}^{}A_{\alpha },\)
  \[
    X \subseteq f^{-1}(M) = f^{-1}\biggl(\bigcup_{\alpha \in \Omega }^{}A_{\alpha }\biggr) = \bigcup_{\alpha \in A}^{}f^{-1}(A_{\alpha })
  \]
  Logo, a família \(\{f^{-1}(A_{\alpha })\}_{\alpha }\) é uma cobertura aberta de X. Como X é compacto, existem \(n_{1}, \dotsc, n_{k}\in \Omega \)
que satisfazem 
  \[
    X\subseteq \bigcup_{i=1}^{k}f^{-1}(A_{n_{i}}).
  \]
  Segue que 
    \[
      Y = f(X)\subseteq f \biggl(\bigcup_{i=1}^{k}f^{-1}(A_{n_{i}})\biggr)\subseteq \bigcup_{i=1}^{k}A_{n_{i}},
    \]
    e \(\{A_{n_{i}}\}_{i=1}^{k}\) é uma subcobertura finita da cobertura aberta \(\{A_{\alpha }\}_{\alpha \in \Omega }\). Portanto,
M é compacto. \qedsymbol
\end{proof*}
\begin{crl*}
  Se X e Y são compactos e \(f:X\rightarrow Y\) é contínua e inversível, então f é um homeomorfismo.
\end{crl*}
\begin{proof*}
  Sendo f contínua e inversível, considere \(h:Y\rightarrow X\) sua inversa. Mostremos que h é contínua.
Com efeito, mostraremos que a pré-imagem de fechado é fechado. Considere F fechado de X. Então,
 \(h^{-1}(F) = f(F)\) é fechado, pois como F é compacto, \(f(F)\) é compacto, que é completo e, portanto, é fechado. \qedsymbol
\end{proof*}
\subsection{Aplicações}
  Seja \((X, d)\) um espaço métrico e \(\mathbb{R}\) munido da métrica usual.
 \begin{theorem*}
  Se X é compacto e \(f:X\rightarrow \mathbb{R}\) é contínua, então existem \(x, y\in X\) tais que 
  \[
    f(x)\leq f(z)\leq f(y),\quad \forall z\in X.
  \]
  Neste caso, colocamos \(f(x)=\inf\{f(x):x\in X\}\) e \(f(x) = \sup\{f(x):x\in X\}\)
 \end{theorem*}
 \begin{proof*}
  Pelo resultado anterior, \(f(X)\) é compacto em \(\mathbb{R},\) logo é fechado e limitado. Assim, estão bem definidos 
    \[
      m\coloneqq \inf\{f(x):x\in X\}\quad\&\quad M\coloneqq \sup\{f(x):x\in X\}.
    \]
    Além disso, como \(m, M\in \overline{f(X)}\) e \(\overline{f(X)} = f(X),\) existem \(x, y\in X\) tais que
      \[
        f(x) = m\quad\&\quad M = f(y).\quad \text{\qedsymbol}
      \]
 \end{proof*}
 \begin{crl*}
  Se \(C\subseteq X\) é compacto e \(A\subseteq X\), então existe \(p\in C\) tal que  
    \[
      \rho (C, A) = \rho (p, A).
    \]
 \end{crl*}
 \begin{proof*}
  Sejam \(C, A\subseteq X\), C compacto e tome \(p\in C\). Considere \(h:C\rightarrow \mathbb{R}\) dada por 
    \[
      h(x) = d(A, C) - d(A, x).
    \]
  Como h é contínua, C é compacto e existe \(p\in C\) tal que, portanto,
 \begin{align*}
  h(p) &= \sup\{h(x):x\in C\}\\ 
       &= 0.
 \end{align*}
 \end{proof*}
 \begin{crl*}
  Se \(X\) é compacto e \(f:X\rightarrow \mathbb{R}^{n}\) é contínua, então existem \(x, y\in X\) tais que 
    \[
      \Vert f(x) \Vert\leq \Vert f(z) \Vert\leq \Vert f(y) \Vert,\quad \forall z\in X.
    \]
 \end{crl*}
\begin{prop*}
  Se X é compacto, \((Y, \rho )\) é outro espaço métrico e \(f:X\rightarrow Y \) é contínua, então f é uniformemente contínua
\end{prop*}
\begin{proof*}
  Suponha que X é compacto, \(f:X\rightarrow Y\) é contínua, mas não uniformemente contínua. Existe \(\varepsilon >0\) tal que
para todo \(n\in \mathbb{N}\), existem \(x_{n}, y_{n}\in X\) satisfazendo 
  \[
    d(x_{n},y_{n}) < \frac{1}{n}\quad\&\quad \rho (f(x_{n}), f(y_{n}))\geq \varepsilon .
  \]
  Por compacidade, as duas sequências \(\{x_{n}\}\) e \(\{y_{n}\}\) possuem subsequências convergentes, que podem
ser consideradas tendo o mesmo limite (\textbf{exercício}). Neste caso, existem \(N\subseteq \mathbb{N}\) e \(x\in X\) tais que 
  \[
    x_{n_{k}}, y_{n_{k}}\overbracket[0pt]{\longrightarrow}^{n\to \infty}x,
  \]
  enquanto que \(\rho (f(x_{n_{k}}), f(y_{n_{k}}))\geq \varepsilon \) para todo \(n_{k}\in N\).
  Mas isso é uma contradição, pois f ser contínua quer dizer que 
    \[
      \rho (f(x_{n_{k}}), f(y_{n_{k}}))\overbracket[0pt]{\longrightarrow}^{k\to \infty}\rho(f(x), f(x)) = 0.\quad \text{\qedsymbol}
    \]
\end{proof*}
\newpage
\end{document}

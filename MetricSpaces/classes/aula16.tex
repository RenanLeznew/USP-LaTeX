\documentclass[MetricSpaces/metric_notes.tex]{subfiles}
\begin{document}
\section{Aula 16 - 07/11/2023}
\subsection{Motivações}
\begin{itemize}
	\item Completamento de Espaços Métricos;
	\item Contrações;
	\item Princípio da Contração de Banach.
\end{itemize}
\subsection{Completamento}
Começamos com um exemplo já visto e já provado (vide Aula 8):
\begin{example}
	Todo espaço métrico \((X, \rho )\) pode ser imerso isometricamente no espaço vetorial normado \((\mathcal{B}(X;\mathbb{R}), \Vert \cdot \Vert_{\infty})\) das funções
	limitadas com a norma do supremo, ou seja,
	\[
		\Vert f \Vert_{\infty} = \sup\{|f(x)|: x\in X\}
	\]
\end{example}
Este exemplo foi utilizado pois faremos ideias similares ao estudar o \textit{completamento} de espaços métricos - estaremos ``mergulhando'' espaços métricos incompletos
isometricamente de um jeito específico em espaços métricos completos, essencialmente ``completando'' ele.
\begin{def*}
	Um \textbf{completamento} de um espaço métrico (M, d) é um par \(((\hat{M}, \rho ), f)\) satisfazendo:
	\begin{itemize}
		\item[i)] \((\hat{M}, \rho )\) é um espaço métrico completo;
		\item[ii)] \(f:M\rightarrow \hat{M}\) é uma imersão isométrica;
		\item[iii)] \(f(M)\) é denso em \(\hat{M}\), ou seja, \(\overline{f(M)} = \hat{M}\).
	\end{itemize}
\end{def*}
\begin{lemma*}
	Sejam \((M, d), (N, \rho )\) espaços métricos, \(S\subseteq{M}\) e N completo. Se \(f:S\rightarrow N\) é contínua, então existe uma única extensão contínua de f a \(\overline{S}.\)
\end{lemma*}
\begin{proof*}
	Seja \(x\in \overline{S}.\) Existe uma sequência \(\{x_{n}\}\in S\) tal que \(x_{n}\overbracket[0pt]{\longrightarrow}^{n\to \infty}x\). A sequência \(\{f(x_{n})\}\subseteq{f(S)}\)
	é de Cauchy (exercício) e, sendo M completo, existe uma função h tal que \(f(x_{n})\overbracket[0pt]{\longrightarrow}^{n\to \infty}h(x)\).

	Precisamos checar a boa-definição de h. Se \(\{y_{n}\}\subseteq{S}\) é uma sequência tal que \(y_{n}\longrightarrow x, n\rightarrow \infty\), então \(d(x_{n}, y_{n})\longrightarrow 0, n\rightarrow \infty\).
	Com isso, \(d(f(x_{n}), f(y_{n}))\longrightarrow 0, n\rightarrow \infty\) e a boa-definição de h é garantida. Definimos, então, \(h:\overline{S}\rightarrow N\) por
	\(h(x) = \lim_{n\to \infty}f(x_{n}),\) em que \(\{x_{n}\}\subseteq{S}, x_{n}\overbracket[0pt]{\longrightarrow}^{n\to \infty}x.\) Note que a sequência contínua
	\(\{x_{n} = x\}\) cumpre \(h(x) = \lim_{n\to \infty}f(x) = f(x),\) tal que h estende a f. Portanto, provamos a existência e unicidade da extensão. \qedsymbol
\end{proof*}
\textit{Existir uma extensão} de \(f:S\rightarrow N\) a \(\overline{S}\) significa que existe uma função \(h:\overline{S}\rightarrow N\) tal que \(f = h |_S\), ou seja,
\(h(x) = f(x)\) para todo \(x\in S\).
\begin{theorem*}
	Todo espaço métrico admite um completamento. Além disso, esse completamento é único, a menos de isometria.
\end{theorem*}
Quando dizemos que um completamento é único, a menos de isometria, queremos dizer que, se \(((\hat{M}, \rho ), f)\) e \(((\tilde{M}, \beta ), g)\) são completamentos de M,
então \((\tilde{M}, \beta )\) e \((\hat{M}, \rho )\) são isométricos. Faz sentido falar em unicidade por isometria porque quando dois espaços métricos são isométricos, eles podem
ser vistos com essencialmente a mesma estrutura métrica - os abertos, conexidade, etc.
\begin{proof*}
	Vimos que \(\mathcal{B}(M, \mathbb{R})\) é completo e que existe uma imersão isométrica \(T:M\rightarrow \mathcal{B}(M, \mathbb{R}).\) Considere o conjunto \(\hat{M} = \overline{T(M)}\)
	e observe que \(\hat{M}\) é fechado (já que o fecho do fecho é o próprio fecho, tal que \(\overline{\hat{M}} = \overline{\overline{T(M)}} = \overline{T(M)} = \hat{M}\) e, portanto,
	completo, pois conjuntos fechados dentro de completos são, também, completos. Assim, \(((\hat{M}, \Vert \cdot \Vert _{\infty}), T)\) é um completamento de M. Resta mostrarmos a unicidade e, para
	isso, utilizaremos o lema.

	Sejam \(\hat{M}, \tilde{M}\) espaços métricos completos e isometrias \(T:M\rightarrow \tilde{M}\) e \(S:M\rightarrow \hat{M},\) com imagens densas. Definimos a isometria
	\(I:\tilde{M}\rightarrow \hat{M}\) como a única extensão contínua a \(\tilde{M} = \overline{T(M)}\) da isometria
	\[
		V = S \circ{T^{-1}}:T(M)\rightarrow S(M).
	\]
\end{proof*}
\subsection{Contrações}
\begin{def*}
	Seja \((X, \rho )\) um espaço métrico completo. Uma aplicação \(T:X\rightarrow X\) é uma \textbf{contração} em X se existe \(0 < \kappa < 1,\) tal que
	\[
		\rho (T_{x}, T_{y})\leq \kappa \rho (x, y),\quad \forall x, y\in X.\quad\square
	\]
\end{def*}
As contrações permitem-nos falar de um dos resultados mais fundamentais da matéria, com inúmeras aplicações - o \textit{Princípio da Contração de Banach}.
\hypertarget{banach_contraction}{\begin{theorem*}[Princípio da Contração de Banach]
		Se X é um espaço métrico completo e T é uma contração em X, então T possui um único ponto fixo em X, isto é, existe um único \(x_{0}\in X\) tal que \(T(x_{0}) = x_{0}.\)
	\end{theorem*}}
\begin{proof*}
	Primeiramente, vamos mostrar que T tem no máximo um ponto fixo. Para isso, sejam x e y dois pontos fixos de T, de forma que
	\[
		\rho (x, y) = \rho (Tx, Ty)\leq \kappa \rho (x, y) < \rho (x, y).
	\]
	Logo, \(x=y.\) Agora, vamos provar a existência de um ponto fixo.

	Seja \(x\in X\) e considere a sequência \(\{x, Tx, T^{2}x, \dotsc\}\) de X. Dados \(n, p\in \mathbb{N},\) vale
	\begin{align*}
		\rho (T^{n+p}x, T^{n}x) & \leq \kappa^{n}[\rho (T^{p}x, T^{p-1}x) + \dotsc + \rho (Tx, x)]                                       \\
		                        & \leq \kappa^{n}[\kappa^{p-1}\rho (Tx, x) + \dotsc + \rho(Tx, x)]\quad \text{(Desigualdade Triangular)} \\
		                        & = \kappa ^{n}\underbrace{[\kappa^{p-1} + \dotsc + 1]}_{\text{série geométrica}}\rho(Tx, x)             \\
		                        & = \kappa ^{n}\frac{1}{1-\kappa }\rho (Tx, x).
	\end{align*}
	Como \(0 < \kappa < 1, \kappa ^{n}\overbracket[0pt]{\longrightarrow}^{n\to \infty}0.\) Logo, \(\{T^{n}x\}\) é convergente para algum \(x_{0}\in X.\)
	Para ver que \(x_{0}\) é ponto fixo de T, note que
	\[
		Tx_{0} = T \lim_{n\to \infty}T^{n}x = \lim_{n\to \infty}T^{n+1}x = x_{0}.
	\]
	Portanto, T tem um único ponto fixo. \qedsymbol
\end{proof*}
Um comentário do escritor: não apenas mostramos que a função tem um ponto fixo, como também provamos que \textbf{aplicações repetidas de T convergem muito rápido para o ponto fixo.}
Isso pode ser visto no meio da prova de que \(\{T^{n}x\}\) é de Cauchy, já que, a cada aplicação de T a si mesma, multiplicou-se por \(\kappa \) o fator \(\kappa^{n}.\)
\end{document}

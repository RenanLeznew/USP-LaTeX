\documentclass[MetricSpaces/metric_notes.tex]{subfiles}
\begin{document}
\section{Aula 06 - 29/08/2023}
\subsection{Motivações}
\begin{itemize}
	\item Propriedades dos abertos;
	\item Propriedades dos fechados.
\end{itemize}
\subsection{Propriedades de Abertos e Fechados}
\begin{example}
	Considere \((\mathbb{R}, |\cdot |)\). Com \(E = B_{\frac{1}{2}}\biggl(\frac{1}{2}\biggr) = (0, 1)\) é aberto. De fato, se \(x\in(0, 1)\),
	então consideramos \(0 < r < \min\{x, 1-x\}.\) Neste caso, temos
	\[
		B_{r}(x) = (x-r, x+r)\subseteq{(0, 1)}\quad \text{Exercício.]}
	\]
\end{example}
\begin{prop*}
	Seja \((X, \rho )\) um espaço métrico. Então,
	\begin{itemize}
		\item[a)] X e \(\emptyset\) são abertos;
		\item[b)] \(B_{r}(x)\) é aberto;
		\item[c)] A união arbitrária de abertos é um aberto;
		\item[d)] A intersecção finita de abertos é aberto.
	\end{itemize}
\end{prop*}
\begin{proof*}
	O item a é trivial.

	Para o item b, dado y em \(B_{r}(x)\), consideramos \(0 < r_{y} < r - \rho (x, y)\). Se z pertence a \(B_{r_{y}}(y)\), a desigualdade triangular implica
	\[
		\rho (x, z)\leq \rho (x, y)\leq \rho (y, z) < \rho (x, y) + r_{y} < r.
	\]
	Logo, \(\rho_{r_{y}}(y)\subseteq{B_{r}(x)}\), ou seja, a bola aberta é aberta.

	Quanto ao c, seja \(\{A_{\lambda }\}_{\lambda \in I}\) uma coleção de abertos de X e \(A = \bigcup_{\lambda \in I}^{}{A_{\lambda }}\). Vamos mostrar que A é aberto.
	Com efeito, dado x em A, segue que \(x\in A_{\lambda}\) para algum \(\lambda \in I.\) Como \(A_{\lambda }\) é aberto, existe \(r_{x} > 0\) tal que \(B_{r_{x}}(x)\subseteq{A_{\lambda}}\subseteq{A}.\)
	Destarte, A é aberto.

	Finalmente, para o item d, tome \(\{A_{i}\}_{i=1}^{n}\) uma coleção finita de abertos em X e seja \(A = \bigcap_{i=1}^{n}{A_{i}}\). Seja \(x\in A\). Temos x pertencente a \(A_{i}\) para algum \(i=1, 2, \cdots, n\).
	Como \(A_{i}\) é sempre aberto, existe \(r_{i} > 0\) tal que \(B_{r_{i}}(x)\subseteq{A_{i}}.\) Fazendo \(r=\min\{r_{i}: 1\leq i\leq n\}\), segue que
	\[
		B_{r}(x)\subseteq{A_{i}},\quad \forall i=1, \cdots, n.
	\]
	Deste modo,
	\[
		B_{r}(x)\subseteq{\bigcap_{i=1}^{n}{A_{i}}} = A.
	\]
	Portanto, A é aberto. \qedsymbol
\end{proof*}
\begin{def*}
	Um subconjunto \(F\subseteq{X}\) é dito fechado em \((X, d)\) se \(F^{c}\) é aberto em \((X, d).\quad\square\)
\end{def*}
Para proceder, vale a pena relembrar uma das formas mais úteis de manipulação de operações de conjuntos, as chamadas Leis de deMorgan. Com a forma que abertos e fechados foram definidos,
torna-se fundamental entender como complementares agem. Sem mais delongas, as Leis de deMorgan são as seguintes:
\[
	\hypertarget{demorgan}{\boxed{\biggl(\bigcap_{i\in I}^{}{A_{i}}\biggr)^{c} = \bigcup_{i\in I}^{}{A_{i}^{c}}}\quad\boxed{\biggl(\bigcup_{i\in I}^{}{A_{i}}\biggr)^{c} = \bigcap_{i\in I}^{}{A_{i}^{c}}}}
\]
\textbf{\underline{Observação}:} Pessoalmente, gosto de pensar que a ação do complementar é ``virar'' o sinal da operação de cabeça pra baixo, então \(\cup\) vira \(\cap\) e vice-versa.
\begin{prop*}
	Seja \((X, \rho )\) um espaço métrico. Então,
	\begin{itemize}
		\item[a)] \(X, \emptyset\) são fechados;
		\item[b)] \(D_{r}(x)\) é fechado;
		\item[c)] A intersecção arbitrária de fechados é um conjunto fechado;
		\item[d)] A união finita de fechados é um fechado.
	\end{itemize}
\end{prop*}
\begin{proof*}
	As provas de todos os itens, menos do b, seguem a partir das \hyperlink{demorgan}{Leis de DeMorgan}

	Para o item b, dado y fora de \(D_{r}(x)\), vale \(d(x, y) < r.\) Seja \(0 < r_{y} < \rho (x, y) - r.\) Se z pertence a \(B_{r_{y}}(y),\), pela desigualdade triangular,
	\[
		\rho (x, z)\geq \rho (x, y) - \rho (y, z)\geq \rho (x, y) - r_{y} > r.
	\]
	Logo, \(z\not\in D_{r}(x)\) e \(B_{r_{y}}(y)\subseteq{(D_{r}(x))^{c}}.\) \qedsymbol
\end{proof*}
\begin{example}
	Seja \(\mathbb{R}\) com a métrica usual. Temos:
	\begin{align*}
		 & [x-r, x+r]=\bigcap_{n=1}^{\infty}{(x-r-\frac{1}{n}, x+r+\frac{1}{n})} \\
		 & \{x\} = \bigcap_{i=1}^{n}{(x-\frac{1}{n}, x+\frac{1}{n})}.
	\end{align*}
\end{example}
\end{document}

\documentclass[MetricSpaces/metric_notes.tex]{subfiles}
\begin{document}
\section{Aula 07 - 12/09/2023}
\subsection{Motivações}
\begin{itemize}
	\item Fecho e bordo de conjuntos;
	\item Conjuntos densos;
	\item Pontos isolados e de acumulação.
\end{itemize}
\subsection{Fechos e Bordos}
\begin{def*}
	O interior \({E}^{\mathrm{o}}\) de \(E\subseteq{X}\) é a união de todos os abertos de \((X, \rho )\) contidos em E. \(\square\)
\end{def*}
\begin{def*}
	O fecho \(\overline{E}\) de \(E\subseteq{X}\) é a intersecção de todos os fechados contendo E. \(\square\)
\end{def*}
Note que \({E}^{\mathrm{o}}\coloneqq \{x\in E: \exists r > 0, B_{r}(x)\subseteq{E}\}\) e \(\overline{E}\coloneqq \{x\in M: B_{r}(x)\cap E \neq\emptyset, \forall r>0\}\).
Além disso, por definição, \({E}^{\mathrm{o}}\subseteq{E}\subseteq{\overline{E}}\).
\begin{def*}
	Um conjunto \(E\subseteq{X}\) é dito denso se \(\overline{E}=X.\square\)
\end{def*}
\begin{def*}
	Um ponto \(x\in X\) é dito um ponto de fronteira de E se
	\[
		B_{r}(x)\cap E \neq\emptyset\quad\& B_{r}(x)\cap E^{c}\neq\emptyset,\quad \forall r>0.
	\]
	O conjunto de todos os pontos de fronteira de E é denotado por \(\partial E.\square\)
\end{def*}
\begin{example}
	Para \((\mathbb{R}, |\cdot |),\) temos \(\mathbb{Q}\subseteq{\mathbb{R}}\) denso em \(\mathbb{R}\).
	De fato, \(\mathbb{Q}=\mathbb{R}\), pois, para todo x em \(\mathbb{R}\) e \(r > 0\),
	\[
		B_{r}(x)\cap \mathbb{Q}\neq\emptyset,
	\]
	o que implica em \(x\in \overline{\mathbb{R}}\)
\end{example}
\begin{example}
	\begin{itemize}
		\item[1)] Vale que \(E = \overline{E}\) se, e somente se, E é fechado. Afinal, se E é fechado, o menor aberto
		      que o contém é ele mesmo, ou seja, \(\overline{E}\subseteq{E}\). O outro lado da inclusão já foi comentado anteriormente.
		\item[2)] O interior da bola fechada nem sempre é a bola aberta;
		\item[3)] Para d discreta, \(B_{r}(x) = \overline{B_{r}}(x).\)
	\end{itemize}
\end{example}
\begin{def*}
	Se \((X, \rho )\) é um espaço métrico, uma sequência em X é \(\varphi:\mathbb{N}\rightarrow X\) com \(\varphi(n) = x_{n}\) e
	\(Im\varphi = \{x_{n}\}\). Diremos que uma sequência \(\{x_{n}\}\) em \((X, \rho )\) converge se existe \(x\in X\)
	tal que \(\{\rho(x_{n}, x)\}\) converge para zero em \(\mathbb{R}\). Escrevemos \(x_{n}\rightarrow x\) para denotar que \(\{x_{n}\}\)
	converge para x. \(\square\)
\end{def*}
\begin{example}
	Considere \(\mathcal{B}(I; \mathbb{R})\), em que \(I = [0, 1]\), com a norma
	\[
		||f||_{1}\coloneqq \int_{0}^{1}|f(x)|dx.
	\]
	Seja \(\{f_{n}\}\) em \(\mathcal{B}(I; \mathbb{R})\) dada por \(f_{n}(x) =x^{n}, x\in[0,1]\). Para cada \(x\in[0, 1)\), vale
	\(\lim_{n\to \infty}x_{n} = 0\). Assim, \(\lim_{n\to \infty}f_{n}(x) = 0, x\in[0, 1)\) e \(\lim_{n\to \infty}f_{n}(1) = 1.\) Considere \(g\in \mathcal{B}(I; \mathbb{R})\)
	dada por \(g(x) = \left\{\begin{array}{ll}
		0,\quad x\in[0, 1) \\
		1,\quad x=1.
	\end{array}\right.\)
	Para cada \(x\in I\), vale \(\lim_{n\to \infty}f_{n}(x)=g(x)\). Temos, de fato,
	\[
		d(f_{n}, g)= \int_{0}^{1}|x^{n}|dx = \frac{x^{n+1}}{n+1}\biggl|_{0}^{1}\biggr.\longrightarrow 0,\quad n\to\infty.
	\]
\end{example}
\begin{prop*}
	Seja \((X, \rho )\) um espaço métrico e \(E\subseteq{X}\). Equivalem:
	\begin{itemize}
		\item[a)] \(x\in \overline{E}\);
		\item[b)] \(B_{r}(x)\cap E \neq\emptyset,\quad \forall r>0\);
		\item[c)] Existe \(\{x_{n}\}\) em E tal que \(\lim_{n\to \infty}x_{n} = x.\)
	\end{itemize}
\end{prop*}
\begin{proof*}
	\(1) \Rightarrow 2):\) Assuma que existe \(r > 0\) tal que \(B_{r}(x)\subseteq{E^{c}}\). Então,
	\(({E}^{\mathrm{o}})^{c}.\) Como \([({E}^{\mathrm{o}})^{c}]^{c}\) é fechado e contém E, concluímos que
	\(x\not\in \overline{E}.\) De fato, se existe \(r > 0\) tal que \(B_{r}(x)\cap E = \emptyset\), então \(x\not\in \overline{E}\).
	Assim, se \(x\in \overline{E},\) então \(B_{r}(x)\cap E \neq\emptyset\) para todo \(r > 0\).

	\(2)\Rightarrow 3):\) Se \(B_{r}(x)\cap E \neq\emptyset\) para todo \(r > 0\), então \(B_{\frac{1}{n}}(x)\cap \neq\emptyset.\) Seja
	\(x_{n}\) um elemento de \(B_{\frac{1}{n}}(x)\cap E,\) para cada n natural. Então, \(\{x_{n}\}\rightarrow x\), visto que vale \(0\leq d(x_{n}, x) < \frac{1}{n}\rightarrow 0\)
	quando \(n\rightarrow\infty.\)


	\(3) \Rightarrow 1):\) Se existe \(\{x_{n}\}\) em E, \(\lim_{n\to \infty}x_{n} = x\) tal que \(x\not\in \overline{E},\) então \(B_{r}(x)\subseteq{\overline{E}^{c}}\)
	para algum \(r>0.\) Devido ao fato de \(x_{n}\rightarrow x\) quando \(n\rightarrow \infty\), existe \(n_{0}\in \mathbb{N}\) tal que \(\rho(x_{n}, x) < \varepsilon \).
	Em particular, para \(\varepsilon = r, \rho(x, x) < r\)  e \(x_{n}\in B_{r}(x)\subseteq{\overline{E}^{c}}\) para \(n\geq n_{0}\), um absurdo. \qedsymbol
\end{proof*}
\begin{def*}
	Um ponto \(x\in X\) é chamado de ponto de acumulação de \(E\subseteq{X}\) se
	\[
		(B_{r}(x)\setminus\{x\})\cap E \neq\emptyset,\quad \forall r>0.
	\]
	Um ponto \(x\in E\subseteq{X}\) é chamado ponto isolado de E se existe \(r > 0\) tal que
	\(B_{r}(x)\cap{E} = \{x\}\). O conjunto de pontos de acumulação de E, chamado conjunto derivado
	de E, será denotado por \(E'. \square\)
\end{def*}
\begin{prop*}
	Seja \((X, \rho )\) um espaço métrico. Então:
	\begin{itemize}
		\item[a)] \(E = E^{c}\cup \partial E\);
		\item[b)] \(\overline{E} = E\cup (\partial E\cap E')\);
		\item[c)] \({E}^{\mathrm{o}} = E\setminus{\partial E}\);
		\item[d)] \(\partial E\) é um fechado.
	\end{itemize}
\end{prop*}
\begin{proof*}
	a) Como \([({E}^{\mathrm{o}})^{c}]^{c} = E^{c\mathrm{o}c}={E}^{\mathrm{o}}\cup \partial E\) é fechado
	e contém E, concluímos que \(\overline{E}\subseteq{{E}^{\mathrm{o}}}\cup \partial E\).

	Por outro lado, sabe-se que \({E}^{\mathrm{o}}\subseteq{E}\subseteq{\overline{E}}\). Se \(x\in \partial E\) e \(x\not\in {E}^{\mathrm{o}}\),
	então \(B_{\frac{1}{n}}(x)\cap E \neq\emptyset\) para todo n natural. Seja \(\{x_{n}\}\) tal que \(x_{n}\overbracket[0pt]{\longrightarrow}^{n\to\infty }x.\)
	Temos \(x\in \overline{E}\) e segue que \({E}^{\mathrm{o}}\cup \partial E \subseteq{\overline{E}}.\)

	b) Observe que \(E\cup(\partial E\cap E')\subseteq{\overline{E}}\). Se \(x\in\overline{E}\setminus{E}\), segue que
	existe uma sequência \(\{x_{n}\}\) em E tal que \(x_{n}\overbracket[0pt]{\longrightarrow}^{n\to \infty}x.\) Logo,
	\[
		B_{r}(x)\cap E \neq\emptyset\quad\&\quad B_{r}(x)\cap E^{c} \neq\emptyset,\quad \forall r>0,
	\]
	e para todo \(x\in \partial E\cap E'.\) Portanto, \(\overline{E}\subseteq{E\cup(\partial E\cap E')}.\)

	c) Segue que \(E\subseteq{{E}^{\mathrm{o}}\cup \partial E}\) dos anteriores. A partir disto, basta tomar
	a diferença dos conjuntos.

	d) Segue da igualdade \(\partial E = ({E}^{\mathrm{o}}\cup ({E}^{\mathrm{o}})^{c})^{c}\).
	\qedsymbol

\end{proof*}
\end{document}

\documentclass[metric_notes.tex]{subfiles}
\begin{document}
\section{Aula 11 - 28/09/2023}
\subsection{Motivações}
\begin{itemize}
	\item Homeomorfismos.
\end{itemize}
\subsection{Espaços Homeomorfos}
\begin{def*}
	Sejam M e N espaços métricos. Uma função \(f:M\rightarrow N\) é chamada homeomorfismo se
	ela é bijetora e tanto ela quanto sua inversão são funções contínuas. Dizemos, neste caso, que M e N
	são homeomorfos. \(\square\)
\end{def*}
Resumidamente: ``\textit{Função contínua com inversa contínua}''.
\begin{example}
	Considere \(\varphi :[0, 2\pi )\rightarrow \mathbb{S}^{1},\) com
	\[
		\mathbb{S}^{1}\coloneqq \{(x, y)\in \mathbb{R}^{2}: x^{2}+y^{2}=1\},
	\]
	e \(\varphi (t) = (\cos{(t)}, \sin{(t)}), t\in[0, 2\pi ).\) Para uma bola aberta em
	\(\mathbb{S}^{1}\), \(\varphi ^{-1}(B)\) não é um aberto, ou seja, a inversa
	de \(\varphi \) não é contínua.

	Apesar disso, o hemisfério norte \(\mathbb{S}^{1}=\{(x, y)\in \mathbb{S}^{1}: y > 0\}\) é
	homeomorfo à bola aberta \(B_{1}(0) = B(0, 1) = (-1, 1)\), uma vez que esse hemisfério é o gráfico
	de \(f(x) = \sqrt[]{1-x^{2}}, x\in (-1, 1).\) De maneira geral, o hemisfério norte
	da esfera
	\[
		\mathbb{S}_{+}^{n} = \{(x_{1}, \cdots, x_{n})\in \mathbb{S}^{n}: x_{n}>0\}
	\]
	é homeomorfo à bola aberta \(B(0, 1)\subseteq{\mathbb{R}^{n}}\).
\end{example}
\begin{example}
	Seja \(p=(0, 0, 1)\) o polo norte da esfera \(\mathbb{S}^{2}=\{(x, y, z): x^{2}+y^{2}+z^{2}=1\}\). Então,
	\(S^{2}\setminus\{p\}\) é homeomorfo a \(\mathbb{R}^{2}.\) A projeção estereográfica
	\(\varphi :\mathbb{S}^{2}\setminus\{p\}\rightarrow \mathbb{R}^{2}\), definida por
	\[
		\varphi (x, y, z) = \biggl(\frac{x}{1-z}, \frac{y}{1-z}\biggr)
	\]
	é contínua, bem como sua inversa, \(\varphi ^{-1}:\mathbb{R}^{2}\rightarrow \mathbb{S}^{2}\setminus\{p\}\), dada por
	\[
		\varphi ^{-1}(x, y) = \biggl(\frac{2x}{x^{2}+y^{2}+1}, \frac{2y}{x^{2}+y^{2}+1}, \frac{x^{2}+y^{2}-1}{x^{2}+y^{2}+1}\biggr).
	\]
\end{example}
\begin{example}
	O círculo
	\[
		\mathbb{S}^{1}=\{(x, y)\in \mathbb{R}^{2}: x^{2}+y^{2}=1\}
	\]
	e o quadrado
	\[
		Q = \{(x, y)\in \mathbb{R}^{2}: |x|+|y|=1\}
	\]
	do espaço euclidiano \(\mathbb{R}^{2}\) são homeomorfos. A função \(f:Q\rightarrow \mathbb{S}^{1}\),
	definida por
	\[
		f(x, y) = \biggl(\frac{x}{\sqrt[]{x^{2}+y^{2}}}, \frac{y}{\sqrt[]{x^{2}+y^{2}}}\biggr)
	\]
	é um homeomorfismo
\end{example}
\begin{example}
	O plano perfurado
	\[
		X = \{(x, y)\in \mathbb{R}^{2}: (x, y)\neq(0, 0)\}
	\]
	e o cilindro circular reto
	\[
		Y = \{(x, y, z)\in \mathbb{R}^{3}: x^{2}+y^{2}=1\}
	\]
	são homeomorfos pelo homeomorfismo \(f:Y\rightarrow X\)
	\[
		f(x, y, z) = (x e^{z}, y e^{z}),\quad (x, y, z)\in Y.
	\]
\end{example}
\begin{example}
	Seja \(f:M\rightarrow N\) contínua. O gráfico de f,
	\[
		G(f)\coloneqq \{(x, f(x)): x\in M\},
	\]
	é naturalmente homeomorfo a M, o domínio da função. Para ver isso, define-se
	\(F:G(f)\rightarrow M, G(x, f(x))=x\), que é contínua por ser a composta da projeção na primeira coordenada com a inclusão.
	F é sobrejetora, claramente, e
	\[
		F(x, f(x)) = F(y, f(y)) \Longleftrightarrow x = y,
	\]
	tal que f é injetora e, destarte, bijetora. Sua inversa é contínua pois a função
	\(F^{-1}:M\rightarrow G(f)\) dada por \(\pi_{1}\circ{f}\) também é contínua.
\end{example}
\begin{example}
	Se E é um espaço vetorial normado, então toda bola aberta \(B_{\varepsilon }(p) = B(p, \varepsilon )\)
	é homeomorfa ao espaço E. Mostraremos que E E é homeomorfo a \(B_{1}(0) = B(0, 1)\), pois \(B_{\varepsilon }(p)\)
	é homeomorfa a \(B_{1}(0).\) De fato, defina \(f:E\rightarrow B_{1}(0)\) por
	\[
		f(u) = \frac{u}{1+||u||},\quad u\in E,
	\]
	e \(g:B_{1}(0)\rightarrow E\) por
	\[
		g(u) = \frac{u}{1-||u||},\quad u\in B_{1}(0),
	\]
	que são contínuas e inversas uma da outra.
\end{example}
\begin{prop*}
	Sejam d e \(\rho \) métricas sobre um conjunto M. Para que d e \(\rho \) sejam equivalentes é
	necessário e suficiente que a inclusão seja um homeomorfismo.
\end{prop*}
\begin{proof*}
	Suponha que a inclusão \(i:(M, d)\rightarrow (M, \rho )\) é um homeomorfismo. Então,
	\(i^{-1}(B_{\varepsilon }^{\rho }(x))\) é aberto em M. Logo, existe \(\delta > 0\) tal que
	\(B_{\delta }^{d}(x) \subseteq{i^{-1}(B_{\varepsilon }^{\rho }(x))}\)
	Além disso, \(i^{-1}:(M, \rho )\rightarrow (M, d)\) é contínua e então \((i^{-1})^{-1}(B_{\varepsilon }^{d}(x))\)
	é aberto, logo, \(\delta >0\) existe satisfazendo \(B_{\delta }^{\rho }(x) \subseteq{B_{\varepsilon }^{d}(x)}\), o
	que define, em conjunto com a primeira parte, duas métricas equivalentes.

	Por outro lado, assuma d e \(\rho \) equivalentes. Neste caso, precisamos verificar
	que \(i^{-1}(A)\) é aberto para todo aberto de \((M, \rho) \) e \((i^{-1})^{-1}(B)\) é aberto em \((M, d)\).
	Com efeito, dado \(A\subseteq{M}\) aberto, seja \(x\in i^{-1}(A)\). Como x pertence a A, existe \(\delta >0\) de
	maneira que \(B_{\delta }^{\rho }(x)\subseteq{A}.\) Seja \(r > 0\) tal que \(B_{r}^{d}(x)\subseteq{B_{\delta }^{r}}\subseteq{A}\),
	o que garante a continuidade de i. O análogo pode ser feito para mostrar que \(i^{-1}\) é contínua (exercício).
	Portanto, i é homeomorfismo. \qedsymbol
\end{proof*}
\end{document}

\documentclass[metric_notes.tex]{subfiles}
\begin{document}

\section{Aula 04 - 21/08/2023}
\subsection{Motivações}
\begin{itemize}
	\item Subespaços métricos e distância entre conjuntos;
	\item Conjuntos limitados, abertos e fechados;
	\item A topologia de espaços métricos.
\end{itemize}
\subsection{Subespaços Métricos e Distância entre Conjuntos}
\begin{def*}
	Seja (X, d) um espaço métrico e \(M\subseteq{X}.\) Então, \(d|_{M}M\times M:\rightarrow \mathbb{R}\)
	define uma métrica, chamada métrica induzida em M. Isso faz de \((M, d|_M)\) um subespaço métrico. \(\square\)
\end{def*}
Além da distância entre pontos, pode-se falar da distância entre um ponto e um subconjunto do espaço métrico
e da distância entre dois subconjuntos de um espaço métrico. Para isso, considere \((X, \rho )\) um espaço métrico,
\(x\in X\) e \(E, F\subseteq{X}.\) Definimos, então,
\begin{align*}
	 & d(x, E)\coloneqq \inf\{\rho (x, e): e\in E\} \\
	 & d(E, F)\coloneqq \inf\{d(e, F): e\in E\}.
\end{align*}
\textbf{Observação:} O nome ``distância'', aqui, não é sinônimo de métrico. De fato, há um exercício na lista que mostra que a distância entre conjuntos
\textbf{NÃO} é simétrica, ou seja, não define uma métrica.
\begin{prop*}
	Seja \((X, \rho )\) um espaço métrico e \(E\subseteq{X}.\) Então,
	\[
		|d(x, E) - d(y, E)|\leq \rho (x, y) \quad \forall x, y\in X.
	\]
\end{prop*}
\begin{proof*}
	Note que, para todo e em E,
	\[
		d(x, E)\leq \rho (x, e)\leq \rho (x, y) + \rho (y, E).
	\]
	Logo, para todo \(x, y\in X\),
	\[
		d(x, E)\leq \rho (x, y) + d(y, E).
	\]
	Assim, temos
	\[
		d(x, E) - d(y, E)\leq \rho (x, y).
	\]
	Analogamente,
	\[
		d(y, E) - d(x, E)\leq \rho (x, y)
	\]
	Portanto,
	\[
		|d(x, E) - d(y, E)|\leq \rho (x, y).\quad\text{\qedsymbol}
	\]
\end{proof*}
\begin{example}
	Considere \((\mathbb{R}, |\cdot |), A = (-1, 0]\) e x = -2. Por definição,
	\[
		d(x, A) = \inf\{|-2-a|: a\in A\} = \inf\{|2+a|: a\in A\} = 1\quad [d(-2, 1+\varepsilon )= 1 +\varepsilon \forall \varepsilon >0]
	\]
	No entanto, \(d(-2, a) > 1\) para todo a em A.
\end{example}
\begin{crl*}
	Seja \((X, \rho )\) um espaço métrico. Vale a desigualdade
	\[
		|\rho (x, z) - \rho (y, z)|\leq \rho (x, y).
	\]
\end{crl*}
\begin{proof*}
	Seja \(E=\{z\}\). Pela proposição, o resultado já segue. \qedsymbol
\end{proof*}
\subsection{Topologia de Espaços Métricos}
\begin{def*}
	Seja \((X, d)\) um espaço métrico. Dado x em X e \(r > 0\), o conjunto
	\[
		B_{r}(x)\coloneqq \{y\in X: d(x, y) < r\}
	\]
	é chamado bola aberta de centro em x e raio r. O conjunto
	\[
		D_{r}(x)\coloneqq \{y\in X: d(x, y)\leq r\}
	\]
	é chamado bola fechada de centro em x e raio r. \(\square\)
\end{def*}
\begin{example}
	Considere \((\mathbb{R}^{2}, d_{p})\), em que \(d_{p}(x, y)=||x-y||_{p}, 1\leq p\leq \infty.\)
	\begin{align*}
		 & d_{2}((x_{1}, y_{1}), (x_{2}, y_{2})) = [(x_{1}-x_{2})^{2}+(y_{1}-y_{2})^{2}]^{\frac{1}{2}} \Rightarrow B_{1}(0) = \{(x, y)\in \mathbb{R}^{2}: d_{2}((0, 0), (x, y)) < 1\} \\
		 & d_{\infty}((x_{1}, y_{1}), (x_{2}, y_{2})) = \max\{|x_{1} - x_{2}|, |y_{1} - y_{2}|\} \Rightarrow B_{1}(0) = \{(x, y)\in \mathbb{R}^{2}: \max\{|x|, |y|\} < 1\}            \\
		 & d_{1}((x_{1}, y_{1}), (x_{2}, y_{2})) = |x_{1}-y_{1}| + |x_{2} - y_{2}| \Rightarrow B_{1}(0)=\{(x, y)\in \mathbb{R}^{2}: |x|+|y| < 1\}
	\end{align*}
\end{example}
\begin{def*}
	Seja \((X, d)\) um espaço métrico e x um ponto de X. Chamamos x de ponto isolado se existe \(r > 0\) tal que \(B_{r}(x) = \{x\}. \quad\square\)
\end{def*}
\begin{example}
	Seja \(S = [0, 1]\cup \{2\}\) munido da métrica usual, induzida da reta. Temos
	\[
		B_{\frac{1}{2}}(2)=\{y\in S: |x-2| < \frac{1}{2}\} = \{2\}.
	\]
\end{example}
\begin{example}
	Seja \(X \neq\emptyset\) e d a métrica discreta. Então,
	\begin{itemize}
		\item[-] Todo ponto é isolado;
		\item[-] \(D_{r}(x) = \{x\}\) se \(r > 1\);
		\item[-] \(B_{1}(x) = \{x\}\) e \(D_{1}(x) = X;\)
		\item[-] \(B_{r}(x) = X\) se \(r > 1\).
		      De fato, \(D_{r}(x) = \{y\in X: d(x, y)\leq r\}.\) Por isso, se r = 1, \(D_{1}(x) = X.\) Pela mesma lógica, prova-se os outros itens.
	\end{itemize}
\end{example}
\begin{example}
	\begin{itemize}
		\item[i)] Dado \((\mathbb{R}, d),\) d a métrica usual, nenhum ponto é isolado e \(B_{r}(x) = (x-r, x+r), x\in \mathbb{R}, r > 0.\)
		\item[ii)] Se \(M=[0, 2]\) com métrica induzida, \(B_{1}(0) = [0, 1).\)
		\item[iii)] Se \(M=\mathbb{Z}\) com a métrica induzida, então \(B_{1}(n) = \{n\}\)
		\item[iv)] Se \(M=\{\frac{1}{n}:n\in \mathbb{N}, n\neq0\}\cup\{0\}\), com a métrica induzida, então \(B_{1}(0)\neq\{0\}\) para todo \(r>0.\)
		      Neste caso, se \(n, m\neq 0\), então existe \(r > 0\) tal que \(B_{r}\biggl(\frac{1}{n}\biggr)=\{\frac{1}{n}\}\) e \(B_{r}\biggl(\frac{1}{m}\biggr)=\{\frac{1}{m}\}.\) Porém,
		      para qualquer \(r> 0\), seja \(n_{0}\) tal que \(0 < \frac{1}{n_{0}} < \frac{r}{2}\) e \(\frac{1}{n_{0}}\in B_{r}(0).\) Portanto, \(B_{r}(0)\neq\{0\}\) para todo \(r> 0\).
	\end{itemize}
\end{example}
\begin{example}
	Considere \((C([a, b]), ||\cdot ||_{\infty}\), com \(||f||_{\infty} = \sup\{|f(x)|: x\in[a, b]\}.\) Seja
	\(h\in C([a, b])\) e \(r > 0\). Temos
	\[
		B_{r}(h) = \{f\in C([a, b]): ||f-h||_{\infty} < r\} = \{f\in C([a, b]):\max_{x\in[a, b]}\{|f(x)-h(x)|\} < r\}.
	\]
\end{example}
\begin{def*}
	Um subconjunto \(M \neq\emptyset\) de um espaço métrico (X, d) é limitado se
	\[
		diam(X)\coloneqq \sup\{d(x, y):x, y\in X\} < \infty.
	\]
	Neste caso, diam(X) é chamado diâmetro de X. Caso contrário, diz-se que M é ilimitado e \(diam(X) = \infty.\quad\square\)
\end{def*}
\begin{example}
	Seja \((X, d)\) métrico. Para todo x em X e \(r>0, B_{r}(x)\) é ilimitado e, além disso, \(diam(B_{r}(x))\leq 2r\), o que segue da relação
	\(d(y, z)\leq d(y, x) + d(x, z)\leq r + r = 2r.\) Além disso, se \((X, ||\cdot ||)\) é um espaço vetorial normado e a métrica d é induzida pela norma,
	então \(diam(B_{r}(x)) = 2r.\) Com efeito, seja \(s < 2r.\) Tome y em X com \(x\neq0.\) Definimos
	\[
		v = \frac{t}{||y||}y,
	\]
	para algum t satisfazendo \(s < 2t < 2r.\) Neste caso, \(x-v, x+v\in B_{r}(x)\)  e
	\[
		d(x+v, x-v) = 2||v|| = 2t > s,
	\]
	ou seja, a afirmação feita está garantida.
\end{example}
\end{document}

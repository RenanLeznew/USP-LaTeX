\documentclass[metric_notes.tex]{subfiles}
\begin{document}
\section{Aula 10 - 21/09/2023}
\subsection{Motivações}
\begin{itemize}
	\item Composta de Funções Contínuas;
	\item Propriedades de Funções Contínuas.
\end{itemize}
\subsection{Composta de Funções Contínuas}
\begin{prop*}
	Sejam \(f:M\rightarrow N\) e \(g:N\rightarrow P\) são funções contínuas nos pontos p em M e f(p) em N,
	respectivamente. Então, \(g\circ{f}:M\rightarrow P\) é contínua no ponto p em M.
\end{prop*}
\begin{proof*}
	Seja \(\varepsilon >0.\) Considere a bola \(B_{\varepsilon }^{P}(g\overline{f}(p)) = B_{1}.\) Como g é contínua em f(p), então
	existe \(\delta_{g}>0\) tal que \(B_{2}\coloneqq B_{\delta_{g}}^{N}(f(p))\subseteq{g^{-1}(B_{1}),}\) tal que
	\[
		f^{-1}(B_{2})\subseteq{f^{-1}(g^{-1}(B_{1}))} = (f^{-1}\circ{g^{-1}})(B_{1}) = (g\overline{f})^{-1}(B_{1}).
	\]
	Agora, por f ser contínua, existe \(\delta_{f}>0\) tal que
	\[
		B_{\delta_{f}^{M}(p)}\subseteq{f^{-1}(B_{2})}.
	\]
	Assim, juntando tudo,
	\[
		B_{\delta_{f}}^{M}(p)\subseteq{f^{-1}(B_{2})}\subseteq{(g\circ{f})^{-1}(B_{1})} = (g\circ{f})^{-1}(B_{\varepsilon }^{P}(g\circ{f}(p))).
	\]
	Portanto, \(g\circ{f}\) é contínua em p em M. \qedsymbol
\end{proof*}
\subsection{Propriedades de Funções Contínuas}
\begin{def*}
	Sejam \(\mathbb{E}\) um espaço vetorial normado e \(X \neq\emptyset, f:X\rightarrow \mathbb{E}, g:X\rightarrow \mathbb{E}.\)
	Definimos \(f+g:X\rightarrow \mathbb{E}\) por \((f+g)(x)\coloneqq f(x)+g(x).\square\)
\end{def*}
\begin{prop*}
	Se f e g são contínuas, então f + g é contínua.
\end{prop*}
\begin{proof*}
	Definimos \(h:X\rightarrow \mathbb{E}\times \mathbb{E}\) por \(h(x) = (f(x), g(x))\), a qual é contínua. A função soma
	\(s:\mathbb{E}\times \mathbb{E}\rightarrow \mathbb{E}\) dada por \(s(x, y) = x + y\), como vimos, é contínua. Com isso,
	\[
		f(x)+g(x)=s(f(x), g(x)) = s(h(x)) = (s\circ{h})(x).
	\]
	Portanto, (f+g)(x) é contínua.\qedsymbol
\end{proof*}
\begin{def*}
	Sejam \(\mathbb{E}\) um espaço vetorial normado e \(X \neq\emptyset, f:X\rightarrow \mathbb{E}, g:X\rightarrow \mathbb{E}.\)
	Definimos \(fg:X\rightarrow \mathbb{E}\) por \((fg)(x)\coloneqq f(x)g(x).\square\)
\end{def*}
\begin{prop*}
	Se f e g são contínuas, então fg é contínua.
\end{prop*}
\begin{proof*}
	Definimos \(h:X\rightarrow \mathbb{E}\times \mathbb{E}\) por \(h(x) = (f(x), g(x))\), a qual é contínua. A função multiplicação
	\(m:\mathbb{E}\times \mathbb{E}\rightarrow \mathbb{E}\) dada por \(m(x, y) = xy\), como vimos, é contínua. Com isso,
	\[
		f(x)g(x)=m(f(x), g(x)) = m(h(x)) = (m\circ{h})(x).
	\]
	Portanto, (fg)(x) é contínua.\qedsymbol
\end{proof*}
\begin{prop*}
	Se f e g são contínuas e \(g(x)\neq0\) para todo x em X, então \(\frac{f}{g}\) é contínua.
\end{prop*}
\begin{proof*}
	Definimos \(h:\mathbb{R}^{\times}\rightarrow \mathbb{R}, h(t)=\frac{1}{t}\), a qual é contínua. Então,
	\[
		\frac{f(x)}{g(x)} = f(x)\frac{1}{g(x)} = f(x)h(g(x)) = f(x)(h\circ{g})(x).
	\]
	Portanto, como o produto de funções contínuas é contínua, \(\frac{f(x)}{g(x)}\) é contínua. \qedsymbol
\end{proof*}
\end{document}

\documentclass[MetricSpaces/metric_notes.tex]{subfiles}
\begin{document}
\section{Aula 12 - 17/10/2023}
\subsection{Motivações}
\begin{itemize}
	\item Espaço Conexos;
	\item Exemplos de Espaços Conexos;
	\item Teorema do Valor Intermediário.
\end{itemize}
\subsection{Espaço Conexos}
\begin{def*}
	Uma cisão de um espaço métrico M é uma decomposição
	\[
		M = A\cup{B},
	\]
	em que A e B são conjuntos abertos e disjuntos de M. \(\square\)
\end{def*}
Observe que, em uma cisão, os conjuntos A e B são simultaneamente abertos e fechados
de M, visto que
\[
	M = A\cup B\quad\&\quad A\cap B = \emptyset
\]
se, e somente se,
\[
	A = B^{c} \quad\&\ B = A^{c}.
\]
Além disso, é possível definir uma \textit{Cisão Trivial}, na qual A ou B é vazio, fazendo com que
a cisão tenha a forma \(M = M\cup\emptyset\)
\begin{def*}
	Um espaço métrico M é dito ser conexo se a única cisão possível para M é a trivial. Um
	subconjunto A de M é um conjunto conexo dado que \((A, \rho_{A})\) é conexo com a métrica induzida.\(\square\)
\end{def*}
Quando A admite uma cisão não-trivial, dizemos que A é \textit{desconexo}.
\begin{example}
	Seja \(M_{1} = [0,1]\cup [2, 3]\). Podemos escrever
	\[
		M_{1} = \biggl((-\varepsilon , 1 + \varepsilon )\cap M\biggr)\bigcup_{}^{}{\biggl((2-\varepsilon , 3+\varepsilon )\cap M\biggr)},
	\]
	em que \(0 < \varepsilon < \frac{1}{2}.\) Tome, também,
	\[
		M_{2} = \{1, 2, 3\}.
	\]
	Neste caso, \(M_{2} = (B_{\frac{1}{2}}(1)\cap M)\cup ((B_{\frac{1}{2}}(2)\cup B_{\frac{1}{2}}(3))\cap M)\)
	Em ambos os casos, \(M_{1}\) e \(M_{2}\) são desconexos.
\end{example}
\begin{example}
	São desconexos os seguintes subconjuntos de \(\mathbb{R}\):
	\[
		\mathbb{Q}, \quad \mathbb{Z},\quad\text{e}\quad A = \mathbb{R}\setminus\{0\}.
	\]
	Vejamos estes exemplos.

	Nos casos de \(\mathbb{Q}, \mathbb{Z}\), vale que
	\[
		\mathbb{Q} = \underbrace{\biggl((-\infty, \pi )\cap \mathbb{Q}\biggr)}_{A}\cup \underbrace{\biggl((\pi , \infty)\cap \mathbb{Q}\biggr)}_{B}
	\]
	e
	\[
		\mathbb{Z} = \underbrace{\biggl((-\infty, \pi )\cap \mathbb{Z}\biggr)}_{C}\cup \underbrace{\biggl((\pi , \infty)\cap \mathbb{Z}\biggr)}_{D}
	\]
	Como \(A\cap B = C\cap D = \emptyset, A,\text{ e }B\) são abertos em \(\mathbb{Q}\) e \(C, D\) em \(\mathbb{Z}\), tal que \(\mathbb{Q} = A\cup B\)
	e \(\mathbb{Z} = C\cup D\) são ambos desconexos.

	Colocando \(\mathbb{R}\setminus\{0\} = M\), temos a cisão não-trivial de M
	\[
		M = \biggl((-\infty, 0)\cap M\biggr)\bigcup_{}^{}{\biggl((0, +\infty)\cap M\biggr)},
	\]
	tornando-o desconexo.

\end{example}
\begin{example}
	\(\mathbb{R}\) com a métrica usual é um conjunto conexo. De fato, suponha que exisstam
	\(A, B\subseteq{\mathbb{R}}\) não-vazios que são abertos e fechados e tais que \(A\cup B = \mathbb{R}\).
	Seja \(a\in A\) e \(b\in B\) e digamos que \(a < b\). O conjunto
	\[
		A_{b}\coloneqq \{x\in A: x < b\}\neq\emptyset
	\]
	é limitado superiormente. Coloque \(c = \sup A_{b}\leq b\) e \(c\in \overline{A_{b}}\subseteq{\overline{A}}.\)
	Como \(A = \overline{A}, c\in A\) (Lembrando que A é aberto e fechado) e \(c < b\). Como A é
	aberto, existe \(\varepsilon >0\) tal que \(c+\varepsilon < b\) e \((c-\varepsilon , c + \varepsilon )\subseteq{A}\),
	contradizendo a hipótese de \(c = \sup A_{b}.\)
\end{example}
\begin{prop*}
	Seja \((X, \rho )\) um espaço métrico. São equivalentes:
	\begin{itemize}
		\item[1)] X é conexo;
		\item[2)] X e \(\emptyset\) são os únicos subconjuntos de X que são simultaneamente
		      abertos e fechados;
		\item[3)] Se \(A\subseteq X\) tem fronteira vazia, então \(A = X\) ou \(A = \emptyset\)
	\end{itemize}
\end{prop*}
\begin{proof*}
	\(1) \Rightarrow 2)\) Se \(A \subseteq{X}\) for, ao mesmo tempo, aberto e fechado,
	então \(A^{c}\) será, também, aberto e fechado ao mesmo tempo. Além disso, \(A\cup A^{c} = X\).
	Segue que \(A = X\) ou \(A = \emptyset.\)

	\(2) \Rightarrow 3)\) Se \(A\subseteq{X}\) e \(\partial A = \emptyset,\) A é aberto e \(A^{c}\) é
	aberto. Logo, A e \(A^{c}\) também são fechados e \(A=X\) ou \(A=\emptyset\).

	\(3) \Rightarrow 1)\) Se \(X = A\cup A^{c}\) com A e \(A^{c}\) abertos, então \(\partial A = \emptyset\) e \(A = X\)
	ou \(A = \emptyset.\) Logo, X é conexo. \qedsymbol
\end{proof*}
\begin{prop*}
	\begin{itemize}
		\item[1)] A imagem de um conjunto conexo por uma função contínua é um conjunto conexo (continuidade preserva conexidade)
		\item[2)] O fecho de um conjunto conexo é conexo
		\item[3)] Se \(A_{1}\) e \(A_{2}\) são subconjuntos conexos de um espaço M e
		      \(A_{1}\cap A_{2}\neq\emptyset\), então \(A_{1}\cup A_{2}\) também é conexo.
	\end{itemize}
\end{prop*}
\begin{proof*}
	\textbf{Prova do Item 1:} Considere o caso em que \(f:M\rightarrow N\) é contínua, sobrejetiva e M é conexo.
	Vamos provar que \(N = f(M)\) é conexo. Com efeito, seja \(N=A\cup{B}\) uma cisão.
	Então, \(M = f^{-1}(A)\cup f^{-1}(B)\) é uma cisão. Como M é conexo, segue que ou \(f^{-1}(A)\) ou \(f^{-1}(B)\) é vazio.
	Já que f é sobrejetiva, A ou B deve ser vazio.

	Por fim, se \(f:M\rightarrow N\) é contínua e \(X\subseteq{M}\) é conexo, então
	\(f:X\rightarrow f(X)\) é uma sobrejeção, caso o qual já demonstramos que será conexo.

	\textbf{Prova do Item 2:} Consideremos, primeiramente, o caso em que \(X\subseteq{M}\) é um conjunto conexo tal que
	\(\overline{X} = M\), ou seja, X é denso em M. Neste caso, se \(M = A\cup B\) for uma cisão, temos a cisão
	\(X = (A\cap X)\cup(B\cap X)\). Como X é conexo, segue que \(A\cap X = \emptyset\) ou \(B\cap X = \emptyset\). Sendo X denso
	em M, isto implica \(A = \emptyset\) ou \(B = \emptyset\).

	Agora, com relação ao caso geral, se X é conexo, segue de \(\overline{X} = \overline{\overline{X}}\) que X é denso em \(\overline{X}\), reduzindo, assim,
	ao caso anterior.

	\textbf{Prova do Item 3:} Consideremos \(X\coloneqq A_{1}\cup A_{2}\) e seja \(X = C\cup D\) uma cisão. Nosso objetivo é mostrar que
	ou C ou D devem ser vazios. Tome \(a\in A_{1}\cap A_{2}\) e assuma que \(a\in C\). As cisões
	\[
		A_{1} = (A_{1}\cap C)\cup (A_{2}\cap D)\quad\text{e}\quad A_{2} = (A_{2}\cap C)\cup(A_{2}\cap D)
	\]
	implicam que \(A_{1}\cap D = A_{2}\cap D = \emptyset, \) já que \(A_{1}\) e \(A_{2}\) são conexos. Logo,
	\[
		D = D\cap X = D\cap(A_{1}\cup A_{2}) = (D\cap A_{1})\cup(D\cap A_{2}) = \emptyset.\text{\qedsymbol}
	\]
\end{proof*}
\begin{crl*}
	\begin{itemize}
		\item[1)] Se M é conexo e N é homeomorfo a M, então N também é conexo
		\item[2)] Se \(X\subseteq{Y}\subseteq{\overline{X}}\) e X é conexo, então Y é conexo
		\item[3)] Dados dois espaços métricos M e N, então \(M\times N\) é conexo se, e somente se,
		      M e N são conexos.
	\end{itemize}
\end{crl*}
\begin{proof*}
	\textbf{Prova do Item 3:} Se \(M\times N\) é conexo, então \(M = \pi_{1}(M\times N)\) e \(N = \pi_{2}(M\times N)\) são conexos, pois
	a projeção é uma função contínua.

	Por outro lado, suponha que M e N são conexos. Para todo x em M, é possível provar
	que \(\{x\}\times N\) é homeomorfo a N, tal que \(\{x\}\times N\) é conexo para todo x em M. Analogamente, \(M\times\{y\}\) é
	homeomorfo a M e, assim, conexo para todo y em N. Com isso, seja \(n\in N\) e considere
	\[
		C_{x} = \biggl(\{x\}\times N\biggr)\bigcup_{}^{}{\biggl(M\times \{n\}\biggr)},
	\]
	em que \(\biggl(\{x\}\times N\biggr)\cap \biggl(M\times \{n\}\biggr)\neq\emptyset.\) Neste caso, \(C_{x}\) é conexo para todo
	x em M. Observamos, também, que
	\[
		M\times \{n\}\subseteq{\bigcap_{x\in M}^{}{C_{x}}}\neq\emptyset.
	\]
	Portanto,
	\[
		M\times N = \bigcup_{x\in M}^{}{C_{x}},
	\]
	o qual é conexo.\qedsymbol
\end{proof*}
\begin{example}
	\(\mathbb{S}^{1}\) é conexa. Com efeito, tomando \(p = (0, 1)\in \mathbb{S}^{1}\) e \(X = \mathbb{S}^{1}\setminus\{p\}\), sabemos que X é homeomorfo
	à reta \(\mathbb{R}\) pela projeção estereográfica. Logo, X é conexo. Concluímos, assim, que \(\mathbb{S}^{1}\) é conexo observando que
	\(\overline{X} = \mathbb{S}^{1}.\)
\end{example}
\begin{example}
	Podemos definir uma ``projeção estereográfica'' \(\pi_{u}:\mathbb{S}^{1}\setminus\{u\}\rightarrow \mathbb{R}\) a partir de qualquer ponto
	\(u\in \mathbb{S}^{1}\) usado como polo no lugar de p, de forma que \(\mathbb{S}^{1}\setminus\{u\}\) é um conjunto conexo. No entanto, se omitirmos dois pontos distintos \(u, v\in \mathbb{S}^{1},\) obtemos o conjunto
	desconexo \(\mathbb{S}^{1}\setminus\{u, v\}\). Com efeito, seja \(ax + by = c\) a equação da reta que passa
	pelos dois pontos u e v. Então, \(A\cup B\) é uma cisão não trivial tal que \(A = \{(x, y)\in \mathbb{S}^{1}: ax + by > c\}\) e
	\(B = \{(x, y)\in \mathbb{S}^{1}: ax + by < c\}\).
\end{example}
\begin{example}
	Todo intervalo aberto de \(\mathbb{R}\) é conexo (de fato, a recíproca também vale, como veremos logo mais). Com efeito, todo intervalo
	aberto é homeomorfo a \(\mathbb{R}\). Caso o intervalo seja limitado, esse fato decorre dos seguintes homeomorfismos:
	\[
		h:B_{1}(0)\rightarrow B_{r}(u),\quad v\mapsto h(v) = u + rv
	\]
	e
	\[
		g:V\rightarrow B_{1}(0),\quad v\mapsto g(v) = \frac{v}{(1+||v||)}
	\]
	Se o intervalo é ilimitado, então decorre dos seguintes homeomorfismos:
	\[
		h:\mathbb{R}\rightarrow (a, \infty),\quad x\mapsto a + e^{x}
	\]
	e
	\[
		h^{-1}:(a, \infty)\rightarrow \mathbb{R},\quad x\mapsto \ln{(x-a)}.
	\]
	Para o outro tipo de intervalo ilimitado, eles são
	\[
		h:\mathbb{R}\rightarrow (-\infty, a),\quad x\mapsto a - e^{-x}
	\]
	e
	\[
		h^{-1}:(-\infty, a)\rightarrow \mathbb{R},\quad x\mapsto -\ln{(a-x)}.
	\]
\end{example}
Esse último exemplo mostra uma parte do seguinte teorema:
\begin{theorem*}
	Um subconjunto de \(\mathbb{R}\) é conexo se, e somente se, ele é um intervalo.
\end{theorem*}
\begin{proof*}
	A parte de todo intervalo ser conexo foi mostrada no exemplo anterior.

	\(\Rightarrow )\) Reciprocamente, seja \(X\subseteq{\mathbb{R}}\) conexo. Suponha que \(a, b\in X\) e que
	\(a < c < b.\) Provaremos que, neste caso, \(c\in X\). Com efeito, suponha, para fins de contradição, que
	\(c\not\in X.\) Então, teríamos a cisão não trivial de X
	\[
		X = [X\cap(-\infty, c)]\cup[X\cap(c, +\infty)],
	\]
	pois \(a\in X\cap(-\infty, c)\) e \(b\in X\cap(c, +\infty).\) Assim, \(a < c < b\) com \(a, b\in X\) implica que c pertence a X, propriedade que
	garante que X seja um intervalo. \qedsymbol
\end{proof*}
\hypertarget{intermediate_value}{\begin{crl*}
		Seja \(f:[a, b]\rightarrow \mathbb{R}\) contínua. Se \(f(a) < d < f(b)\), então existe c em \((a, b)\) tal que \(f(c) = d.\)
	\end{crl*}}
\begin{proof*}
	A imagem \(f([a, b])\) é um intervalo que contém os pontos \(f(a)\) e \(f(b)\), logo \([f(a), f(b)]\subseteq{f([a, b])}.\) Assim,
	existe \(c\in[a, b]\) tal que \(f(c) = d\). Porém, \(f(a) < f(c) < f(b)\) exclui a possibilidade de \(c = a\) ou \(c = b\). Portanto,
	\(c\in (a, b)\). \qedsymbol
\end{proof*}
\end{document}

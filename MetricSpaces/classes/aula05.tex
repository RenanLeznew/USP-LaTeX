\documentclass[metric_notes.tex]{subfiles}
\begin{document}
\section{Aula 05 - 24/08/2023}
\subsection{Motivações}
\begin{itemize}
	\item Propriedades do diâmetro;
	\item Exemplo de limitado;
	\item Conjuntos abertos;
	\item Propriedade dos abertos.
\end{itemize}
\subsection{Propriedades do Diâmetro}
\begin{example}
	A função
	\[
		\rho_{p}(x,y) = \frac{||x-y||_{p}}{1+||x-y||_{p}},\quad \forall x, y\in \mathbb{R}^{n}
	\]
	é uma métrica em \(\mathbb{R}^{n}\). Para ver isso, comece definindo \(f(t) = \frac{t}{1+t}\) e exiba que essa função é
	crescente, de forma que \(f(||x-y||_{p}\leq f(||x-z||_{p}+||z-y||_{p})\). Com essa métrica, afirmamos que o seguinte ocorre:
	\[
		\mathbb{R}^{n} = B_{1}(0)\quad\&\quad diam(\mathbb{R}^{n})\leq 1.
	\]
	A inclusão \(B_{1}(0)\subseteq{\mathbb{R}^{n}}\) é automática. Por outro lado, seja \(y\in \mathbb{R}^{n}.\) Temos
	\[
		\rho_{p}(y, 0) = \frac{||y||_{p}}{1 + ||y||_{p}} < 1. \Rightarrow \mathbb{R}^{n}\subseteq{B_{1}(0)}.
	\]
	Portanto, \(\mathbb{R}^{n}\) é limitado no espaço métrico \((\mathbb{R}^{n}, \rho_{p}),\) mas não é limitado em \((\mathbb{R}^{n}, d_{p})\).
\end{example}
Vale uma observação - apesar dessa diferença entre \(\rho_{p}\) e \(d_{p}\), veremos futuramente que as duas métricas induzem a mesma
estrutura de formato do espaço - em outras palavras, a mesma topologia.
\begin{prop*}
	Seja \((X, \rho )\) um espaço métrico.
	\begin{itemize}
		\item[1)] \(E\subseteq{X}\) é limitado se, e somente se, existe \(r>0\) tal que \(E\subseteq{B_{r}(x)}\) para todo \(x\in E\).
		\item[2)] Se \(E\subseteq{X}\) é limitado e não-vazio, então
		      \[
			      diam(E) = \inf\{r > 0: E \subseteq{B_{r}(x)}, x \in E\}.
		      \]
	\end{itemize}
\end{prop*}
\begin{proof*}
	\(1 \Rightarrow )\) A volta é simples, segue das propriedades do supremo. Por outro lado, se E é limitado, então \(diam(E) < \infty\). Seja
	\(r=2diam(E)\) e \(B_{r}(x)\supseteq{E}\) para todo x em E. De fato, se \(e\in E\), temos
	\[
		d(e, x)\leq diam(E) < r.
	\]

	\(2 \Rightarrow )\) Seja \(A = \{r>0: E \subseteq{B_{r}(x)}, x\in E\}\). Mostremos que \(diam(E) = \inf{(A)}.\) Se \(r > diam(E)\) e x
	é um ponto de E, então
	\[
		d(x, y)\leq diam(E) < r\quad \forall y\in E,
	\]
	logo, \(E\subseteq{B_{r}(x)}\). Assim, o intervalo \((diam(E), \infty)\subseteq{A}.\) Deste modo, \(\inf{(A)}\leq diam(E).\)
	Por outro lado, se \(r < diam(E),\) então existem \(x, y\in E\) tais que
	\[
		d(x, y) > r,\quad y\not\in B_{r}(x),\quad \&\quad r\not\in A.
	\]
	Consequentemente, \((0, diam(E))\cap A = \emptyset.\) Portanto, \(diam(E) = \inf{(A)}.\) \qedsymbol
\end{proof*}
\begin{def*}
	Seja \((X, d)\) um espaço métrico. Um subconjunto \(E\subseteq{X}\) é dito aberto em \((X, d)\) se, para cada
	x em E, existir \(r_{x} > 0\) tal que \(B_{r_{x}}(x)\subseteq{E}.\quad\square\)
\end{def*}
\end{document}

\documentclass[MetricSpaces/metric_notes.tex]{subfiles}
\begin{document}
\section{Aula 08 - 14/09/2023}
\subsection{Motivações}
\begin{itemize}
	\item Continuidade e suas diferentes formas;
	\item Tipos de funções contínuas.
\end{itemize}
\subsection{Continuidade - Noções Prévias}
Na reta real, as funções contínuas apresentam um papel importante tanto em análise quanto nos diferentes cálculos ao longo da graduação. Não apenas isso, mas
pode-se dizer sem exagero que formam uma parte fundamental da matemática em muitos níveis, possuindo graus de generalização. Ao estudar cálculo I, a definição é
a epsilon-delta comum:
\begin{quote}
	``\textit{Uma função} \(f:X\rightarrow Y, X, Y\subseteq{\mathbb{R}}\) \textit{é dita contínua em x se para todo} \(\varepsilon >0\), \textit{existe um} \(\delta >0\) \textit{tal que, para }\(t\in X,\)
	\[
		|t-x| < \delta \Rightarrow |f(x)-f(t)| < \varepsilon."
	\]
\end{quote}

Ao entrar em cálculo de muitas variáveis, uma versão análoga de continuidade é apresentada:

\begin{quote}
	``\textit{Uma função} \(f:X\rightarrow Y, X\subseteq{\mathbb{R}^{n}}, Y\subseteq{\mathbb{R}^{m}}\) \textit{é dita contínua em x se, para todo} \(\varepsilon >0\), \textit{existe um }\(\delta >0\)
	\textit{tal que, para todo} \(y\in X\),
	\[
		||x-y|| < \delta \Rightarrow ||f(x)-f(y)|| < \varepsilon,
	\]
	\textit{em que} \(||\cdot ||\) \textit{é a norma usual em} \(\mathbb{R}^{n}\) \textit{ou }\(\mathbb{R}^{m}.\)''
\end{quote}
Em ambos os casos, se levarmos em conta a relação entre módulo/norma e intervalos da reta, no sentido de uma norma ou um módulo descrevem as bolas abertas no conjunto, temos uma noção
do que uma função contínua deve fazer - mapear intervalos abertos em abertos. Porém, observe que, na definição, a liberdade que temos é no \(\varepsilon \), ou seja, no aberto da imagem da função
e, para cada \(\varepsilon \), é preciso encontrar um \(\delta \). Em outras palavras, ela mapeia intervalos abertos em outros abertos, mas de uma forma específica: Dado um aberto na imagem da função,
é preciso encontrar um aberto em sua pré-imagem. Essa é, de fato, a caracterização que utilizaremos nesse curso.
\subsection{Continuidade}
Sejam \((X, \rho ), (Y, d)\) espaços métricos. Denotaremos as bolas abertas nos espaços respectivos por um superscript acima - \(B^{X}_{r}(x)\) é uma bola aberta
em X e \(B_{r}^{Y}(x)\) uma em Y.
\begin{def*}
	Diremos que uma função \(f:X\rightarrow Y, (X, \rho ), (Y, d)\) espaços métricos, é contínua se para todo \(\varepsilon >0\) existir um \(\delta = \delta (\varepsilon, p) > 0\) tal que
	\[
		B_{\delta }^{X}(p)\subseteq{f^{-1}(B_{\delta }^{Y}(f(p))}.
	\]
	Equivalentemente, \(f(B_{\delta }^{X}(p))\subseteq{B_{\delta }^{Y}(f(p))}.\square\)
\end{def*}
Explicitamente, se \(\rho (x, p) < \delta \), então \(x\in B_{\delta }^{X}(p).\) Logo, x pertence a \(f^{-1}(B_{\varepsilon }^{Y}(f(p)))\) e
\(f(x)\in B_{\varepsilon }^{Y}(f(p)).\) Por definição da bola, \(d(f(x), f(p)) < \varepsilon.\) Isso pode ser reescrito na seguinte proposição:
\begin{prop*}
	Uma função \(f:X\rightarrow Y\) é contínua em \(p\in X\) se, e somente se, para todo \(\varepsilon >0\) existe
	\(\delta = \delta (\varepsilon , p) > 0\) tal que
	\[
		\rho (x, p)<\delta  \Rightarrow d(f(x), f(p)) <\varepsilon .
	\]
\end{prop*}

A seguir, vamos apresentar algumas outras caracterizações de continuidade - a topológica e a por sequências.
\begin{def*}
	A caracterização topológica das funções contínuas é: \(f:X\rightarrow Y\) é continuas se, e somente se, \(f^{-1}(A)\) é aberto em X para todo aberto A em Y. \(\square\)
\end{def*}
Mostraremos que a caracterização topológica e a com bolas são equivalentes. Se f é contínua e U é um aberto de Y, se
\(y\in f^{-1}(U)\) e \(\varepsilon > 0\) é tal que \(B_{\varepsilon }(f(y)) \subseteq{U},\) então existe \(\delta  > 0\)
tal que \(f^{-1}(B_{\varepsilon }(f(y)))\supseteq{B_{\delta }(y)}.\) Logo, y é interior a \(f^{-1}(U).\) Como
consequência, já que y era arbitrário, \(f^{-1}(U)\) é aberto.

Por outro lado, se \(f^{-1}(U)\) é aberto em \((X, \rho )\) sempre que U é aberto em
\((Y, d)\), segue que, para todo x em X e \(\varepsilon >0, f^{-1}(B(f(x)))\) é aberto e contém x.
Logo, existe \(\delta  > 0\) tal que \(B_{\delta }(x)\subseteq{f^{-1}(B_{\varepsilon }(f(x)))}\) e f é contínua em x.
Portanto, f é contínua para todo x em X.
\begin{example}
	A função constante \(f:X\rightarrow Y\) dada por \(f(x) = c,\) para todo x em X e algum c em Y, é contínua.
	De fato, seja A um aberto de Y. Se c pertence a A, então \(f^{-1}(A) = X\), que é aberto em X. Por
	outro lado, se c não pertence a A, então \(f^{-1}(A) = \emptyset,\) que também é aberto em X. Portanto, f é contínua.
\end{example}
\begin{example}
	A função identidade \(f:X\rightarrow X\) é contínua para todo x em X. De fato, se A é um aberto de X,
	como \(f^{-1}(A) = A, f^{-1}(A)\) é aberto em X.
\end{example}
\begin{def*}
	A caracterização por sequências de uma função contínua é: \(f:X\rightarrow Y\) é contínua em \(x\in X\) se, e somente se, dada uma sequência \(\{x_{n}\}_{n\in \mathbb{N}}\) que
	converge para x (\(x_{n}\longrightarrow x)\), vale \(f(x_{n})\overbracket[0pt]{\longrightarrow}^{n\to \infty}f(x).\square\)
\end{def*}
Vamos verificar que a noção de continuidade que fornecemos implica na definição de sequências. Note que, como \(\rho (x_{n}, x) < \delta \)(já que a sequência converge), vale que existe um \(n_{0}\in \mathbb{N}\) tal que
\(x_{n}\in B_{\delta }^{X}(x)\) para todo \(n\geq n_{0}\). Com isso, como f é contínua, \(B_{\delta }^{X}(x)\subseteq{f^{-1}(B_{\varepsilon }^{Y}(f(x)))}\), ou seja, \(x_{n}\in f^{-1}(B_{\varepsilon }^{Y}(f(x)))\) para todo \(n\geq n_{0}.\)
Assim, \(f(x_{n})\in B_{\varepsilon }^{Y}(f(x))\) para todo \(n\geq n_{0}\), o que equivale a \(d(f(x_{n}), f(x)) < \varepsilon,\) ou seja, \(f(x_{n})\overbracket[0pt]{\longrightarrow}^{n\to \infty}f(x)\)
\subsection{Outras Classes de Continuidade}
\begin{def*}
	Diremos que uma função \(f:X\rightarrow Y\) é uniformemente contínua se, para todo \(\varepsilon >0\), existe \(\delta =\delta (\varepsilon )>0\) tal que
	\[
		\rho (x, y) < \delta \Rightarrow d(f(x), f(y)) < \varepsilon \quad \forall x, y\in X.\quad\square
	\]
\end{def*}
A diferença entre uma função contínua e uniformemente é sutil - enquanto que, na continuidade normal, existe um \(\delta \) diferente para cada ponto x, isso não acontece
na continuidade uniforme, ou seja, um \(\delta \) funciona para todos os pontos do espaço.
\begin{def*}
	Diremos que uma função \(f:X\rightarrow Y\) é Lipschitz contínua se, para algum \(\lambda >0\), vale
	\[
		d(f(x), f(y))\leq \lambda \rho (x, y).\quad \forall x, y\in X.
	\]
\end{def*}
\begin{def*}
	Uma função \(f:X\rightarrow Y\) é localmente Lipschitz se, para todo x em X, existir \(r_{x}>0\) e \(\lambda_{x}>0\) tais que
	\[
		d(f(z), f(w))\leq \rho (x, w)\lambda_{x},\quad \forall z, w\in B_{r_{x}}^{X}(x).
	\]
\end{def*}
Tanto as funções Lipschitz quanto as localmente Lipschitz são contínuas.
\begin{def*}
	Uma função \(f:X\rightarrow Y\) é uma imersão isométrica de X em Y se
	\[
		d(f(x), f(y)) = \rho (x, y).
	\]
	Além disso, se \(f(X) = Y\), ou seja, f é sobrejetora, então f é uma isometria entre X e Y. \(\square\)
\end{def*}
Toda imersão isométrica é injetora, visto que
\[
	d(f(x), f(y)) = \rho (x, y) = 0
\]
é o mesmo que \(\rho (x, y) = 0\) que, como \(\rho \) é uma métrica, significa que \(x=y\).
\begin{example}
	Consideremos \(\mathbb{R}^{n}\) munido da métrica euclidiana e b em \(\mathbb{R}^{n}.\) A função translação por b,
	\(f:\mathbb{R}^{n}\rightarrow \mathbb{R}^{n}\) definida por \(f(x) = x + b\) é uma imersão isométrica.
\end{example}
\begin{example}
	Se \(f:X\rightarrow Y\) for injetiva e definirmos em \(Y' = f(X)\) a métrica
	\(d_{f}(y_{1}, y_{2}) = \rho (f^{-1}(y_{1}), f^{-1}(y_{2}))\) para todo \(y_{1}, y_{2}\in Y'\), então
	\(f:X\rightarrow Y'\) é uma isometria.
\end{example}
\begin{example}
	Se \(M\subseteq{X}\) e \(\rho _{M}:M\times M\rightarrow \mathbb{R}^{+}\) é a métrica induzida, a inclusão
	\(i:M\rightarrow X\) definida por \(i(x) = x\), é uma imersão isométrica.
\end{example}
\begin{prop*}
	Um espaço métrico \((X, \rho )\) pode ser imerso isometricamente no espaço vetorial normado \(\mathcal{B}(X, \mathbb{R})\) das funções limitadas \(f:X\rightarrow \mathbb{R}\), com a norma
	\(||f||_{\infty}=\sup\{|f(x)|\}\)
\end{prop*}
\begin{proof*}
	Considere \((X, \rho )\) e \((\mathcal{B}(X, \mathbb{R}), ||\cdot ||_{\infty}\) os espaços métricos no enunciado. Defina
	\(f:X\rightarrow \mathcal{B}(X, \mathbb{R})\) por \(x\mapsto f(x):X\rightarrow \mathbb{R}\) dada por
	\(f(x)(y) = \rho (x, y)-\rho (p, y).\) Note que:
	\begin{itemize}
		\item[i)] \(|f(x)(y) - f(z)(y)| = |\rho (x, y)-\rho (p, y)-\rho (z, y) + \rho (p, y)| = |\rho (x, y)-\rho (z, y)|\)
		\item[ii)] \(|f(x)(z)-f(z)(z)| = |\rho (x, z) - \rho (p, z) - \rho (z, z) + \rho (p,z)| = |\rho (x, z)|.\)
	\end{itemize}
	Disto, conclui-se que
	\[
		||f(x)-f(z)||_{\infty} = \sup\{|f(x)(y) - f(z)(y)|\}\leq \rho (x, y)\quad\&\quad ||f(x)-f(z)||_{\infty} = \rho (x, z).
	\]
	Portanto, f é uma imersão isométrica. \qedsymbol
\end{proof*}
\end{document}

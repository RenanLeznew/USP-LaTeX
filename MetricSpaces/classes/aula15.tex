\documentclass[metric_notes.tex]{subfiles}
\begin{document}
\section{Aula 15 - 26/10/2023}
\subsection{Motivações}
\begin{itemize}
	\item Mais Exemplos de Espaços de Banach;
	\item Propriedade Topológicas de Espaços Métricos Completos.
\end{itemize}
\subsection{Espaços de Banach e Topologia de Espaços Completos}
Continuemos com outro exemplo de um espaço de Banach:
\begin{example}
	O espaço métrico \(\mathcal{B}(X, \mathbb{R})\coloneqq \{f:X\rightarrow \mathbb{R}: f\text{ é limitada}\}\) é um espaço de Banach com a norma
	\[
		||f||=\sup_{x\in X}\{|f(x)|\}.
	\]

	De fato, seja \(\{f_{n}\}\subseteq{\mathcal{B}(X, \mathbb{R})}\) uma sequência de Cauchy. Logo, para todo \(\varepsilon >0\), existe \(n_{0}\in \mathbb{N}\) tal que
	se \(m, n\geq n_{0}\),
	\[
		|f_{n}-f_{m}|\leq \sup_{x\in X}\{f_{n}(x)-f_{m}(x)\} = ||f_{n}-f_{m}|| < \varepsilon .
	\]
	para todo x em X. Logo, \(\{f_{n}(x)\}\subseteq{\mathbb{R}}\) é uma sequência de Cauchy e, portanto, converge para algum \(f(x),\) pois \(\mathbb{R}\) é um espaço métrico completo.
	Assim, temos \(f:X\rightarrow \mathbb{R}\) definida por
	\[
		f(x) = \lim_{n\to \infty}f_{n}(x),\quad x\in X.
	\]
	Fazendo \(n\longrightarrow \infty\) na primeira desigualdade e colocando \(m=n_{0}\), obtemos
	\[
		|f(x)-f_{n_{0}}(x)|\leq \varepsilon .
	\]
	Disto segue que
	\[
		|f(x)| - |f_{n_{0}}(x)|\leq \varepsilon \Rightarrow |f(x)|\leq \varepsilon + |f_{n_{0}}(x)|.
	\]
	Como \(f_{n_{0}}\in \mathcal{B}(X, \mathbb{R}),\) concluímos que \(f\in \mathcal{B}(X, \mathbb{R}).\) Por aquela mesma desigualdade, segue que
	\[
		\sup_{x\in X}\{f(x)-f_{m}(x)\}\leq \varepsilon ,
	\]
	ou seja, \(||f-f_{m}|| < \varepsilon \) para todo \(m\geq n_{0}\), o que equivale a dizer que \(f_{m}\) converge para f em \(\mathcal{B}(X, \mathbb{R}).\)
\end{example}
Um resultado muito importante sobre a topologia dos espaços métricos completos diz que todo subconjunto fechado do espaço métrico é completo e vice-versa - em outras
palavras, há uma equivalência entre completude e ser fechado!
\begin{prop*}
	Seja \((M, d)\) um espaço métrico.
	\begin{itemize}
		\item[a)] Se M é completo, então \(F\subseteq{M}\) é completo se F é fechado.
		\item[b)] Se \(F\subseteq{M}\) é completo, então F é fechado.
	\end{itemize}
\end{prop*}
\begin{proof*}
	b) Se F é completo e x pertence ao seu fecho, então
	\begin{itemize}
		\item[i)] Existe \(\{x_{n}\}\subseteq{F}\) tal que \(x_{n}\overbracket[0pt]{\longrightarrow}^{n\to \infty}x\)
		\item[ii)] Como \(\{x_{n}\}\) é convergente, ela é de Cauchy em F e converge em F.
	\end{itemize}
	Portanto, \(x\in F\) e \(\overline{F}\subseteq{F},\) donde segue que \(F = \overline{F}\) e que F é fechado.

	a) Se F é fechado, toda sequência em F converge para um elemento de F. Sendo assim, dada uma sequência de Cauchy \(\{x_{n}\}\subseteq{F},\) ela é,
	em particular, de Cauchy em M, o qual é completo. Sendo assim, existe \(x\in M\) tal que \(x_{n}\overbracket[0pt]{\longrightarrow}^{n\to \infty}x.\) No entanto,
	já que F é fechado, o limite de uma sequência de F deve estar lá, o que significa que \(x\in F\). Portanto, \(\{x_{n}\}\) de Cauchy tem seu limite em F, provando a
	completude do mesmo. \qedsymbol
\end{proof*}
\end{document}

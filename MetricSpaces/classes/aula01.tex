\documentclass[metric_notes.tex]{subfiles}
\begin{document}

\section{Aula 01 - 08/08/2023}
\subsection{Motivações}
\begin{itemize}
	\item Introdução ao Material do Curso.
\end{itemize}
\subsection*{O que é um espaço métrico?}
Ao longo deste curso, trabalharemos com um conjunto M não-vazio.
\hypertarget{def_metric}{ \begin{def*}
		Uma função \(d:M\times M\rightarrow \mathbb{R}\) é dita ser uma métrica em M se:
		\begin{itemize}
			\item[i)] \(d(x, y)\geq 0, x, y\in M\);
			\item[ii)] \(d(x, y) = 0 \Leftrightarrow x = y, x, y\in M\);
			\item[iii)] \(d(x, y) = d(y, x), x, y\in M\);
			\item[iv)] \(d(x, y)\leq d(x, z) + d(z, y), x, y, z\in M\).
		\end{itemize}
		Neste caso, o par \((M, d)\) é chamado espaço métrico.
	\end{def*}
}
\begin{example}
	\begin{itemize}

		\item[1)] \((\mathbb{R}, d)\), em que \(d:\mathbb{R}\times \mathbb{R}\rightarrow [0, \infty)\) é dado por
		      d(x, y) = \(|x-y|.\) É claro que, olhando para \(d(x,y),\) vale para quaisquer x, y reais que
		      \[
			      d(x, y) = |x-y| = |-1(y-x)| = 1|y-x| = d(y, x).
		      \]
		      Assim, resta verificarmos os itens dois e quatro da \hyperlink{def_metric}{definição de métrica}. Para o item (ii),
		      \[
			      |x-y| = 0 \Longleftrightarrow x = y.
		      \]
		      Com relação ao último item, observe que
		      \[
			      |x+y|\leq |x| + |y|.
		      \]
		      De fato, como \(|a|\geq a\) para todo número real a,
		      \[
			      |x+y|^{2} = (x+y)^{2} = x^{2} + 2xy +y^{2}\leq x^{2} + 2|x||y| + y^{2} = (|x| + |y|)^{2}.
		      \]
		      Logo, tomando a raíz dos dois lados, segue a afirmação:
		      \[
			      |x+y|\leq |x| + |y|.
		      \]
		      Com isso, temos
		      \[
			      d(x, y) = |x-y| = |x-z+z-y| = |(x-z)-(y-z)|\leq |x-z| + |y-z|.
		      \]
		      Portanto, \(d(x, y)\leq d(x, z) + d(z, y)\), o que torna \((\mathbb{R}, d)\) um espaço métrico.

		\item[2)] Seja X um conjunto não-vazio. Definimos
		      \[
			      d:X\times X\rightarrow [0, \infty)
		      \]
		      por
		      \[
			      d(x, y) = \left\{\begin{array}{ll}
				      0,\quad x = y \\
				      1,\quad x\neq y.
			      \end{array}\right.
		      \]
		      Esta métrica é conhecida como \textit{métrica discreta}. Verifiquemos as propriedades dela.

		      Com efeito, como a imagem dela pode ser apenas 0 ou 1, o item 1 é trivial. Por definição, a métrica vale 0 se,
		      e somente se, x e y são iguais, tal que o item (ii) está feito. O item (iii) segue automaticamente se x e y são iguais. Caso
		      eles sejam diferentes, temos \(d(x, y) = 1, d(y, x) = 1\), ou seja, o item (iii) é válido para todos os casos. Por fim, a desigualdade triangular fica como exercício.
	\end{itemize}
\end{example}
\end{document}

\documentclass[metric_notes.tex]{subfiles}
\begin{document}
\section{Aula 02 - 10/08/2023}
\subsection{Motivações}
\begin{itemize}
	\item Exemplos de espaços métricos;
	\item Similaridade entre métricas;
	\item Produto de espaços métricos.
\end{itemize}
\subsection{Uma nota histórica}
Algumas dessas informações podem ser vistas com mais detalhe do livro de espaços métricos
de Jean Cerqueira, do IME.

Para um contexto temporal, em 1906, Maurice Fréchet publicou sua tese de doutorado, nomeada
``Sus quelques du calcul functionnel''. A seguir, em 1910, David Hilbert faz sua tentativa de axiomatizar
as ideias vistas na tese de Fréchet e, dois anos depois, Felix Hausdorff, trabalhando no conceito de separação de pontos
a partir do ponto de vista de conjuntos, acaba contribuindo com essa axiomatização. Antes disso, Henri Poincaré, havia sistematizado as ideias
de forma prototípica. Outro nome mencionável é o de Pavel Urysohnn, responsável por aprofundar-se na parte da separação de pontos.

\subsection{Exercício 1 da Lista 1}
Considere \(d:\mathbb{R}^{2}\times \mathbb{R}^{2}\rightarrow [0, \infty) \) definida por
\[
	d((x_{1}, x_{2}), (y_{1}, y_{2})) = |x_{2}-y_{2}|.
\]
Então, d não é uma métrica. Chequemos este fato.

Com efeito, considere x um número real qualquer. Assim,
\[
	d((1, x), (2, x)) = |x-x| = 0.
\]
No entanto, (1, x) não é igual a (2, x), ou seja, já falha logo na primeira condição de métrica!
Portanto, conclui-se que d não pode ser métrica. Analogamente, definindo \(d':\mathbb{R}^{2}\times \mathbb{R}^{2}\rightarrow [0, \infty)\) por
\(d'((x_{1}, x_{2}), (y_{1}, y_{2}))=|x_{1}-y_{1}|\), ela também não será métrica.

\textbf{Obs.:} Definiremos, ainda essa aula, a métrica \(d_{s}:\mathbb{R}^{2}\times \mathbb{R}^{2}\rightarrow [0, \infty)\) por
\[
	d_{s}((x_{1}, x_{2}), (y_{1}, y_{2})) = |x_{1}-y_{1}|+|x_{2}-y_{2}|,
\]
a qual torna \((\mathbb{R}^{2}, d_{s})\) num espaço métrico. Mostremos isso. Os itens \hyperlink{def_metric}{(i), (ii) e (iii)}
da definição de métrica estão trivialmente cumpridos. Para a desigualdade triangular, observe que
\begin{align*}
	d((x_{1}, x_{2}), (y_{1}, y_{2})) & = |x_{1}-y_{1}| + |x_{2}-y_{2}|                                        \\
	                                  & \leq |x_{1}-z_{1}| + |z_{1}-y_{1}| + |x_{2}-z_{2}| + |x_{2}-y_{2}|     \\
	                                  & =d((x_{1}, x_{2}), (z_{1}, z_{2})) + d((z_{1}, z_{2}), (y_{1}, y_{2}))
\end{align*}
\subsection{Espaço de funções contínuas}
Considere o intervalo I=[0, 1] e tome o conjunto
\[
	\mathcal{C}(I, \mathbb{R})\coloneqq \{f:I\rightarrow \mathbb{R}: f \text{ contínua}\}.
\]
Então, são métrica em \(\mathcal{C}(I, \mathbb{R})\) as funções \(d:\mathcal{C}(I, \mathbb{R})\rightarrow [0, \infty), (f, g)\mapsto \sup_{x\in I}\{|f(x)-g(x)|\}\)
e \(\rho :\mathcal{C}(I, \mathbb{R})\times \mathcal{C}(I, \mathbb{R})\rightarrow [0, +\infty)\) definida por
\[
	\rho(f, g) = \int_{0}^{1}|f(x)-g(x)|dx,\quad f, g\in \mathcal{C}(I, \mathbb{R}).
\]
Mostremos que elas são \hyperlink{def_metric}{métricas}, começando pela d.

Observe que, se \(d(f, g) = 0,\), então
\[
	|f(y)-g(y)|\leq \sup_{x\in I}\{|f(x)-g(x)|\}\quad \forall y\in I.
\]
Logo, \(|f(y) - g(y)| = 0\) para todo y em I, garantindo que f e g são as mesmas.
Para a desigualdade triangular, observe que, para qualquer y em I, e h em \(\mathcal{C}(I, \mathbb{R}),\) vale que
\[
	|f(y)\pm h(y) - g(y)|\leq |f(y) - h(y)| + |h(y) - g(y)|.
\]
Tomando o supremo, obtemos
\[
	d(f, g)\leq d(f, h) + d(h, g).
\]
Assim, \((\mathcal{C}(I, \mathbb{R}), d)\) é um espaço métrico.

Agora, analisando a questão de \(\rho\), se \(\rho (f, g) = 0,\) então
\[
	\int_{0}^{1}|f(x)-g(x)|dx = 0
\]
Como o Teorema da Conservação de Sinal implicaria em \(\int_{0}^{1}|f(x)-g(x)|dx > 0\) se \(|f(x)-g(x)| > 0\) para algum x em I.
Logo, \(|f(x)-g(x)|=0\) para todo x em I. Para mostrar a desigualdade triangular, considere \(f, g, h\in \mathcal{C}(I, \mathbb{R}).\)
Temos, para todo \(x\in I\),
\[
	|f(\overbrace{x)-g(}^{\pm h(x)}x)|\leq |f(x) - h(x)| + |h(x) - g(x)|
\]
Integrando os dois lados da desigualdade,
\[
	\int_{0}^{1}|f(x)-g(x)|\leq \int_{0}^{1}|f(x) - h(x)|dx + \int_{0}^{1}|h(x) - g(x)|dx,
\]
de onde segue a desigualdade triangular. Portanto, \((\mathcal{C}(I, \mathbb{R}), \rho)\) é um espaço métrico.
\subsection{Espaço euclidiano n-dimensional}
Seja n um natural, \(n\geq 1.\) O espaço euclidiano é
\[
	\mathbb{R}^{n}\coloneqq \{(x_{1}, \cdots, x_{n}): x_{i}\in \mathbb{R}, i = 1, 2, \cdots, n\},
\]
munido da métrica usual/euclidiana, definida por
\[
	d(x, y) = \biggl(\sum\limits_{i=1}^{n}(x_{i}-y_{i})^{2}\biggr)^{\frac{1}{2}},
\]
em que \(x=(x_{1}, x_{2}, \cdots, x_{n})\) e \(y = (y_{1}, y_{2}, \cdots, y_{n}\) são elementos de \(\mathbb{R}^{n}.\)
Sobre \(\mathbb{R}^{n}\), podemos considerar duas outras métricas, a métrica da soma e do máximo. Definimo-las, respectivamente, por
\begin{align*}
	 & d_{s}(x, y)\coloneqq \sum\limits_{i=1}^{n}|x_{i}-y_{i}|          \\
	 & d_{m}(x, y)\coloneqq \max\{|x_{i}-y_{i}|, i = 1, 2, \cdots, n\},
\end{align*}
para  \(x=(x_{1}, x_{2}, \cdots, x_{n})\) e \(y = (y_{1}, y_{2}, \cdots, y_{n}\) são elementos de \(\mathbb{R}^{n}.\) Vamos mostrar que a primeira d é métrica.

Pra começar, não é difícil ver que os itens \hyperlink{def_metric}{(i) e (ii)} da definição são satisfeitos. Para mostrar a desigualdade triangular, será necessário
utilizar Cauchy-Schwarz:
\begin{lemma*}[Desigualdade de Cauchy-Schwars]Dados \((x_{1}, \cdots, x_{n})\) e \((y_{1}, \cdots, y_{n})\), vale a desigualdade
	\[
		\hypertarget{cauchy_schwarz}{\sum\limits_{i=1}^{n}|x_{i}\cdot y_{i}|\leq \biggl(\sum\limits_{i=1}^{n}|x_{i}|^{2}\biggr)^{\frac{1}{2}}\biggl(\sum\limits_{i=1}^{n}|y_{i}|^{2}\biggr)^{\frac{1}{2}} }
	\]
\end{lemma*}
\begin{proof*}
	Observe que, dados \(x, y\in \mathbb{R}\), vale que \(2xy\leq x^{2} + y^{2}\).
	Temos, para \(x=(\sum\limits_{i=1}^{n}|x_{i}|^{2})^{\frac{1}{2}}\) e \(y=(\sum\limits_{i=1}^{n}|y_{i}|^{2})^{\frac{1}{2}}\). Com isso, aplicando a desigualdade vista para
	\(\frac{|x_{i}|}{x}\) e \(\frac{|y_{i}|}{y}\), ganhamos
	\[
		2\frac{|x_{i}||y_{i}|}{xy}\leq \frac{|x_{i}|^{2}}{x^{2}} + \frac{|y_{i}|^{2}}{y^{2}}.
	\]
	Somando de \(i=1, \cdots, n,\) obtemos
	\[
		\frac{2}{xy}\sum\limits_{i=1}^{n}|x_{i}||y_{i}| \leq \frac{1}{x^{2}}\sum\limits_{i=1}^{n}|x_{i}|^{2} + \frac{1}{y^{2}}\sum\limits_{i=1}^{n}|y_{i}|^{2} = 1 + 1 = 2.
	\]
	Portanto, isolando a soma à esquerda, temos
	\[
		\sum\limits_{i=1}^{n}|x_{i}||y_{i}|\leq \frac{2}{2}\cdot xy = xy.\text{\qedsymbol}
	\]
\end{proof*}
A desigualdade triangular seguirá do seguinte
\begin{align*}
	\bigl[d(\overline{x}, \overline{y})\bigr]^{2} = \sum\limits_{i=1}^{n}(x_{i}-y_{i})^{2} & =\sum\limits_{i=1}^{n}(x_{i}-z_{i}+z_{i}-y_{i})^{2}                                                                                                                                                                          \\
	                                                                                       & =\sum\limits_{i=1}^{n}(x_{i}-z_{i})^{2} + 2\sum\limits_{i=1}^{n}(x_{i}-z_{i})(z_{i}-y_{i}) + \sum\limits_{i=1}^{n}(z_{i}-y_{i})^{2}                                                                                          \\
	                                                                                       & \leq \sum\limits_{i=1}^{n}(x_{i}-z_{i})^{2} + 2\biggl(\sum\limits_{i=1}^{n}|x_{i}-z_{i}|^{2}\biggr)^{\frac{1}{2}}\biggl(\sum\limits_{i=1}^{n}|z_{i}-y_{i}|^{2}\biggr)^{\frac{1}{2}} + \sum\limits_{i=1}^{n}(z_{i}-y_{i})^{2} \\
	                                                                                       & =\bigl[d(\overline{x}, \overline{y})\bigl]^{2} + 2 d(\overline{x}, \overline{z})d(\overline{z}, \overline{y}) + \bigl[d(\overline{z}, \overline{y})\bigr]^{2}                                                                \\
	                                                                                       & = (d(\overline{x}, \overline{z})+d(\overline{z}, \overline{y}))^{2}.
\end{align*}
\end{document}

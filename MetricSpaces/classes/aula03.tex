\documentclass[metric_notes.tex]{subfiles}
\begin{document}
\section{Aula 03 - 17/08/2023}
\subsection{Motivações}
\begin{itemize}
	\item Métricas similares;
	\item Produtos de Espaços Métricos;
	\item Espaços vetoriais normados;
	\item Desigualdade de Hölder e de Minkowski.
\end{itemize}
\subsection{Similaridade de Métricas}
\begin{def*}
	Seja M um conjunto não vazio. Duas métricas d e \(\rho \) em M são similares se existem \(c_{1}, c_{2} > 0\) tais que
	\[
		c_{1}d(x, y)\leq \rho (x, y)\leq c_{2}d(x, y)
	\]
\end{def*}
\begin{example}
	Seja d a métrica usual e \(\delta  \) a métrica discreta em \(\mathbb{R}\). Para todo c positivo, existem
	x, y em \(\mathbb{R}\) tais que \(d(x, y) > c\delta (x, y)\), ou seja, não vale \(d(z, w)\leq c\delta (z, w)\) para todos
	z, w reais.

	De fato, dado \(c > 0\), tome \(x=2c\) e \(y=c-1.\) Tem-se \(d(x, y) = c + 1\) e \(\delta (x, y) = 1\). Logo, não são similares.
\end{example}
\begin{prop*}
	Para quaisquer x, y em \(\mathbb{R}^{n},\) vale a desigualdade
	\[
		d_{m}(x,y)\leq d(x,y)\leq d_{s}(x, y)\leq nd_{m}(x,y)
	\]
\end{prop*}
\begin{proof*}
	Seja \(x=(x_{1}, x_{2}, \cdots, x_{n}), y = (y_{1}, \cdots, y_{n})\). Temos:
	\[
		d_{m}(x, y) = \max \biggl\{(|x_{i}-y_{i}|^{2})^{\frac{1}{2}}\biggr\}\leq \biggl(\sum\limits_{i=1}^{n}|x_{i}-y_{i}|^{2}\biggr)^{\frac{1}{2}} = d(x, y)
	\]
	e também \(d_{s}(x, y) = \sum\limits_{i=1}^{n}|x_{i}-y_{i}|\leq n\max \biggl\{|x_{i}-y_{i}|\biggr\}\),
	provando a desigualdade do meio
	\[
		(d(x,y))^{2} = \sum\limits_{i=1}^{n}(x_{i}-y_{i})^{2} = \underbrace{\biggl(\sum\limits_{i=1}^{n}|x_{i}-y_{i}|\biggr)}_{d_{s}(x, y)^{2}} - A,\quad A\geq 0,
	\]
	de onde segue que \((d(x, y))^{2}\leq (d_{s}(x, y))^{2} - A\) e prova a desigualdade. \qedsymbol
\end{proof*}
\subsection{Produto de Espaços Métricos}
Considere \((M_{1}, d_{2}), (M_{2}, d_{2}), \cdots, (M_{n}, d_{n})\) espaços métricos. Para \(M=\Pi_{i=1}^{n}M_{i}\) e definimos as métricas anteriores
\begin{itemize}
	\item \(d(x, y) = \biggl\{\sum\limits_{i=1}^{n}[d_{i}(x_{i}, y_{i})]^{2}\biggr\}^{\frac{1}{2}}\)
	\item \(d_{s}(x, y) = \sum\limits_{i=1}^{n}d_{i}(x_{i}, y_{i})\);
	\item \(d_{max}(x, y) = \max\{d_{i}(x_{i}, y_{i})\}\).
\end{itemize}
\begin{example}
	Tome \(M = \mathcal{C}(I, \mathbb{R})\times \mathbb{R}\). Se \(x = (f, s), y=(h, t)\), então
	\[
		d_{m}(x, y) = \max \biggl\{\rho (f, h), d(s, t)\biggr\}
	\]
\end{example}
\subsection{Espaços Vetoriais Normados}
\begin{def*}
	Dado V um espaço vetorial, uma norma em V é uma função \(||\cdot ||:V\rightarrow [0, +\infty)\) que satisfaz
	\begin{itemize}
		\item[i)] \(||\vec{v}|| = 0 \Longleftrightarrow \vec{v}=0\);
		\item[ii)] \(||\lambda \vec{v}|| = |\lambda |||\vec{v}||\);
		\item[iii)] \(||\vec{v}+\vec{w}||\leq ||\vec{v}|| + ||\vec{w}||\).
	\end{itemize}
\end{def*}
\begin{example}
	Seja \((\mathbb{R}^{2n}, +, \cdot )\) é um espaço vetorial e \(||\cdot ||:\mathbb{R}^{n}\rightarrow [0, \infty)\) dada por
	\[
		x\mapsto ||x|| = \sqrt[]{\sum\limits_{i=1}^{n}x_{i}^{2}}
	\]
\end{example}
\begin{example}
	Considere \((\mathcal{C}(I, \mathbb{R}), +, \cdot )\) e defina \(||f||_{\infty} = \max \biggl\{|f(x)|: x\in I\biggr\}, f\in \mathcal{C}(I, \mathbb{R}).\) Isto define uma norma em
	\(\mathcal{C}(I, \mathbb{R}).\)

	Note que \(||f||_{\infty} = 0 = \max\{|f(x)|\}.\) Logo, \(|f(y)| = |f|_{\infty} = 0\) para todo y em I e \(f(y) = 0\) para todo y em I.
	Isso demonstra a primeira propriedade. Para a segunda propriedade,
	\[
		||\lambda f||_{\infty} = \max\{|\lambda f(x)|: x\in I\} = |\lambda |||f||_{\infty}
	\]
	pelas propriedades de módulo. Por fim, temos
	\[
		||f+g||_{\infty} = \max\{|f(x)+g(x)|: x\in I\}\leq \max\{|f(x)| + |g(x)|: x\in I\}\leq \max\{|f(x)|\} + \max\{|g(x)|\}\leq ||f||_{\infty} + ||g||_{\infty}.
	\]
\end{example}
\begin{example}
	Definiremos a norma p em \(\mathbb{R}^{n}\). Se \(p=\infty,\) coloquemos
	\[
		||x||_{\infty} = \sup\{|x_{i}|: 1\leq i\leq n\}.
	\]
	Se \(p\neq\infty\), definimos
	\[
		||x||_{p} = \biggl(\sum\limits_{i=1}^{n}|x_{i}|^{p}\biggr)^{\frac{1}{p}}
	\]
\end{example}
\subsection{A Desigualdade de Minkowski}
\begin{lemma*}
	Se \(p, q\in (1, \infty)\) é tal que \(\frac{1}{p} + \frac{1}{q} = 1\) e \(a, b\in [0, \infty),\) então
	\[
		a^{\frac{1}{p}} + b^{\frac{1}{q}}\leq \frac{a}{q} + \frac{b}{q}.
	\]
\end{lemma*}
\begin{proof*}
	Se \(b > 0\), então \(\frac{a^{\frac{1}{p}}+b^{\frac{1}{q}}}{b} = \frac{a}{bp} + \frac{1}{q}\), isto é,
	\[
		\biggl(\frac{a}{b}\biggr)^{\frac{1}{p}}\leq \frac{a}{bp} + \frac{1}{q}.
	\]
	Tomando \(z=\frac{a}{b}\) e \(\alpha =\frac{1}{p},\) segue que \(t^{\alpha }\leq \alpha t + 1 - \alpha.\) Com isso, defina
	\(f_{\alpha }:\mathbb{R}^{+}\rightarrow \mathbb{R}\) por
	\[
		f_{\alpha }(t) = \alpha t +1 - \alpha - z^{\alpha }.
	\]
	Segue que \(f_{\alpha }(t)\geq 0\) pela sua derivada. \qedsymbol
\end{proof*}
\begin{lemma*}[Desigualdade de Hölder]
	Se \(p\in(1, \infty), q\in (1, \infty)\) é tal que \(\frac{1}{p}+\frac{1}{q}=1\) e \(a, b\in [0, \infty),\) então
	\[
		\sum\limits_{i=1}^{n}|x_{i}y_{i}|\leq \biggl[\sum\limits_{i=1}^{n}|x_{i}|^{p}\biggr]^{\frac{1}{p}}\biggl[\sum\limits_{i=1}^{n}|y_{i}|^{q}\biggr]^{\frac{1}{q}}
	\]
	para todo \(x=(x_{1}, \cdots, y_{n}), y=(y_{1}, \cdots, y_{n}) \in \mathbb{R}^{n}.\)
\end{lemma*}
\begin{proof*}
	Se x = 0 ou y = 0, a desigualdade é trivial. Se \(x\neq0\) e \(y\neq0\), então defina
	\[
		a_{j} = \frac{|x_{j}|^{p}}{\sum\limits_{i=1}^{n}|x_{i}|^{p}}\quad b_{j}=\frac{|y_{j}|^{q}}{\sum\limits_{i=1}^{n}|y_{i}|^{q}}.
	\]
	Observamos que \(\sum\limits_{j=1}^{n}a_{j} = \sum\limits_{j=1}^{n}b_{j}=1.\) Aplicando a desigualdade de Young,
	\[
		a_{j}^{\frac{1}{p}}b_{j}^{\frac{1}{q}} = \frac{|x_{j}y_{j}|}{\biggl[\sum\limits_{i=1}^{n}|x_{i}|^{p}\biggr]^{\frac{1}{p}}\biggl[\sum\limits_{i=1}^{n}|y_{i}|^{q}\biggr]^{\frac{1}{q}}}\leq \frac{1}{p}a_{j} + \frac{1}{q}b_{j},
	\]
	para \(j=1, \cdots, n.\) Assim,
	\[
		\frac{\sum\limits_{j=1}^{n}|x_{j}y_{j}|}{\biggl[\sum\limits_{i=1}^{n}|x_{i}|^{p}\biggr]^{\frac{1}{p}}\biggl[\sum\limits_{i=1}^{n}|y_{i}|^{q}\biggr]^{\frac{1}{q}}}\leq \frac{1}{p} + \frac{1}{q}=1,
	\]
	donde segue a desigualdade. \qedsymbol
\end{proof*}
\begin{prop*}
	Se \(p\in[1, \infty)\), então
	\[
		\biggl[\sum\limits_{i=1}^{n}|x_{i}+y_{i}|^{p}\biggr]^{\frac{1}{p}}\leq \biggl[\sum\limits_{i=1}^{n}|x_{i}|^{p}\biggr]^{\frac{1}{p}} + \biggl[\sum\limits_{i=1}^{n}|y_{i}|^{p}\biggr]^{\frac{1}{p}},
	\]
	para todo \(x=(x_{1}, \cdots, x_{n}), y=(y_{1}, \cdots, y_{n})\in \mathbb{R}^{n}\).
\end{prop*}
\begin{proof*}
	Os casos \(p=1, \infty\) são deixados como exercício. Se \(p\in(1, \infty),\) então
	\[
		\biggl[\sum\limits_{i=1}^{n}|x_{i}+y_{i}|^{p}\biggr]^{\frac{1}{p}}\leq \biggl[\sum\limits_{i=1}^{n}(|x_{i}|+|y_{i}|)^{p}\biggr]^{\frac{1}{p}}.
	\]
	Podemos escrever
	\[
		(|x_{i}|+|y_{i}|)^{p} = (|x_{i}|+|y_{i}|)^{p-1}|x_{i}|+(|x_{i}|+|y_{i}|)^{p-1}|y_{i}|,\quad i = 1, \cdots, n.
	\]
	Somando os elementos à esquerda da desigualdade anterior, obtemos
	\[
		\sum\limits_{i=1}^{n}(|x_{i}|+|y_{i}|)^{p} = x_{n}+y_{n}
	\]
	com
	\[
		x_{n} + y_{n}\coloneqq \sum\limits_{i=1}^{n}(|x_{i}|+|y_{i}|)^{p-1}|x_{i}| + \sum\limits_{i=1}^{n}(|x_{i}|+|y_{i}|)^{p-1}|y_{i}|.
	\]
	Aplicando a Desigualdade de Hölder, temos
	\begin{align*}
		x_{n} & \leq \biggl[\sum\limits_{i=1}^{n}|x_{i}|^{p}\biggr]^{\frac{1}{p}}\biggl[\sum\limits_{i=1}^{n}(|x_{i}|+|y_{i}|)^{(p-1)q}\biggr]^{\frac{1}{q}} \\
		      & \leq \biggl[\sum\limits_{i=1}^{n}|x_{i}|^{p}\biggr]^{\frac{1}{p}}\biggl[\sum\limits_{i=1}^{n}(|x_{i}|+|y_{i}|)^{p}\biggr]^{\frac{1}{q}}.
	\end{align*}
	De forma análoga, temos
	\[
		y_{n}\leq \biggl[\sum\limits_{i=1}^{n}|y_{i}|^{p}\biggr]^{\frac{1}{p}}\biggl[\sum\limits_{i=1}^{n}(|x_{i}|+|y_{i}|)^{p}\biggr]^{\frac{1}{q}}.
	\]
	Portanto,
	\[
		\sum\limits_{i=1}^{n}(|x_{i}|+|y_{i}|)^{p} = x_{n} + y_{n}\leq \cdots.\quad\text{\qedsymbol}
	\]
\end{proof*}
\begin{example}
	Dado \(p\in[0, \infty], (\mathbb{R}^{n}, d_{p})\) é um espaço métrico, em que \(d_{p}\) está representando a métrica induzida por \(||\cdot ||_{p}.\) Então, para \(p=2\),
	recuperamos o espaço euclidiano n-dimensional.
\end{example}
\end{document}

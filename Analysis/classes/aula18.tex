\documentclass[Analysis/analysis_notes.tex]{subfiles}
\begin{document}
\section{Aula 18 - 10/05/2023}
\subsection{O que esperar?}
\begin{itemize}
	\item Semicontinuidades;
	\item Diferenciabilidade.
\end{itemize}
\subsection{Semicontinuidade}
\begin{def*}
	Se \(D\subseteq{\mathbb{R}}\), c é um ponto de acumula\c cão de D e \(f:D\rightarrow \mathbb{R}\)
	é uma fun\c cão que é limitada em uma vizinhan\c ca de c em \(\mathbb{R}\), definimos
	\[
		\limsup_{x\to c}f(x)\limsup_{r\to 0^{+}}\{f(x):x\in D, 0 < |x-c|< r\}\quad\text{ e } \liminf_{x\to c}f(x)=\liminf_{r\to 0^{+}}{f(x):x\in D, 0 < |x-c|<r}.
	\]
	Definimos os análogos para pontos em \(c\in D^{-}\) como
	\[
		\overline{\text{Lim}}_{x\to p}f(x)\coloneqq \limsup_{x\to p}f(x)\coloneqq \limsup_{r\to 0^{+}}\{f(x):x\in D, |x-p| < r\}\quad\text{ e } \underline{\text{Lim}}_{x\to p}f(x)\coloneqq \liminf_{r\to 0^{+}}\{f(x):x\in D, |x-p|<r||\}square
	\]
\end{def*}
\begin{def*}
	Seja \(f:D\rightarrow \mathbb{R}\) e c um ponto de D. Então, f é semicontínua
	superiormente em c se
	\[
		f(c) = \overline{Lim}_{x\to c}f(x) \quad(f(c)\geq \limsup_{x\to p}f(x))
	\]
	e semicontínua inferiormente se
	\[
		f(c) = \underline{Lim}_{x\to c}f(x) \quad(f(c)\leq \liminf_{x\to p}f(x))
	\]
	Se f é semicontínua superiormente (inferiormente) em todos os pontos de D, dizemos
	que f é semicontínua superiormente (inferiormente). \(\square\)
\end{def*}
\begin{theorem*}
	Seja \(f:D\rightarrow \mathbb{R}\) uma fun\c cão semicontínua superiormente (inferiormente).
	Se k é um número real, então existe um aberto \(O_{k}\) de \(\mathbb{R}\) tal que
	\[
		O_{k}\cap D = \underbrace{\{x\in D: f(x)<k\}}_{f^{-1}((-\infty, k))} \quad(O_{k}\cap D=\underbrace{\{x\in D: f(x)>k\}}_{f^{-1}((k, \infty))})
	\]
\end{theorem*}
\begin{proof*}
	Para c em D com \(f(c) < k,\), da defini\c cão de semicontinuidade, segue que
	existe \(r_{c}>0\) tal que \(f(x) < k\) para todo x em D, \(|x-c| < r_{c}.\) Seja
	\(I_{c} = (c-r_{c}, c+r_{c})\) e defina
	\[
		O_{k} = \bigcup_{c\in f^{-1}((-\infty, k))}^{}{I_{c}.}
	\]
	De fato, se c pertence a esta interse\c cão, então \(f(c) < k\). Com isso,
	\[
		k > f(c) = \overline{Lim}_{x\to c}f(x) = \limsup_{r\to 0^{+}}\{f(s):s\in D, |s-c|< r\}
	\]
	Logo, existe \(r_{c}\) tal que \(\sup\{f(s):s\in D, |s-c|<r_{c}\} < k.\) Assim,
	\(f(s)<k\) para todo s em \((c-r_{c}, c+r_{c})\) e em D. Finalmente, por conta disso,
	\(c\in D\cap(c-r_{c}, c+r_{c})\subseteq{f^{-1}((-\infty, k))}\) e \(\bigcup_{c\in f^{-1}((-\infty, k))}^{}{D\cap (c-r_{c}, c+r_{c})} = f^{-1}(-\infty, k)\)
	É claro que, para todo x em \(O_{i}\cap D = f^{-1}((-\infty, k))\).
\end{proof*}

\subsection{Motiva\c cão para Diferenciabilidade}
A diferenciabilidade é um conceito fundamental na análise matemática, que
estende o conceito de derivada de uma função em um ponto. Uma função é dita
diferenciável em um ponto se existe um número que pode ser interpretado como
a inclinação da reta tangente à curva representativa da função naquele ponto.

Seja $f: \mathbb{R} \to \mathbb{R}$ uma função. Dizemos que $f$ é
diferenciável em $x_0 \in \mathbb{R}$ se o seguinte limite existe:

\begin{equation}
	f'(x_0) = \lim_{{h \to 0}} \frac{f(x_0 + h) - f(x_0)}{h}
\end{equation}

O número $f'(x_0)$ é chamado de derivada de $f$ em $x_0$.

A diferenciabilidade tem diversas aplicações em matemática e ciências.
Ela permite, por exemplo, calcular a taxa de variação de uma quantidade,
otimizar funções e resolver equações diferenciais. Alguns exemplos incluem

\subsubsection{Função Linear}

Considere a função linear $f(x) = mx + b$, onde $m$ e $b$ são constantes. A derivada de $f$ em qualquer ponto $x_0$ é $f'(x_0) = m$, que é a inclinação da reta.

\subsubsection{Função Quadrática}

Considere a função quadrática $f(x) = ax^2 + bx + c$, onde $a$, $b$ e $c$ são constantes. A derivada de $f$ em qualquer ponto $x_0$ é $f'(x_0) = 2ax_0 + b$.

Pense na diferenciabilidade como uma maneira de medir o quão "suave" é uma função. Se você pode desenhar a curva de uma função sem levantar a caneta do papel, então a função é provavelmente diferenciável. A derivada em um ponto é simplesmente a inclinação da reta que melhor se ajusta à curva naquele ponto.
Vamos entrar mais a fundo nessas ideias

\subsection{Derivadas}
\begin{def*}
	Seja \(f:D_{f}\rightarrow \mathbb{R}\) uma fun\c cão e \(p\in D_{f}\) um ponto de
	acumula\c cão de \(D_{f}\). Se existir o limite
	\[
		\lim_{x\to p}\frac{f(x)-f(p)}{x-p} = L\in \mathbb{R},
	\]
	diremos que L é a derivada de f em p e escrevemos
	\[
		f'(p) = L = \lim_{x\to p}\frac{f(x)-f(p)}{x-p}=\lim_{h\to 0}\frac{f(p+h)-f(p)}{h}. \square
	\]
\end{def*}

Se \(f:D\rightarrow \mathbb{R}\) possui derivada num ponto d de D que é também um
ponto de acumula\c cão de D, para \(h\in \mathbb{R}\) tal que d+h pertence a D, escrevemos
\[
	r(h) = f(d+h)-f(d)-f'(d)h.
\]
Nesses pontos, definimos \(r:\{h\in \mathbb{R}: d+h\in D_{f}\}\rightarrow \mathbb{R}\)
e escrevemos \(f(d+h) = f(d)+f'(d)h + r(h)\) e, fazendo \(\sigma (h) = \frac{r(h)}{h}, h\neq 0,\)
temos \(\lim_{h\to 0}\sigma (h)=0.\)

Note que f é diferenciável em d se, e somente se, existe fun\c cão \(\sigma \)
com \(\lim_{h\to 0}\sigma (h)=0\) tal que \(f(d+h)=f(d)+[f'(d)+\sigma (h)]h\).
\begin{def*}
	Sejam \(f:D_{f}\rightarrow \mathbb{R}\) uma fun\c cão e \(p\in D_{f}\) um ponto
	de acumula\c cão à direita de \(D_{f}\). Se existir o limite
	\[
		\lim_{x\to p^{+}}\frac{f(x)-f(p)}{x-p}=L^+\in \mathbb{R},
	\]
	diremos que \(L^+\) é a derivada à direita de f em p e escrevemos
	\[
		f'(p^+) = L^+=\lim_{x\to p^+}\frac{f(x)-f(p)}{x-p}= \lim_{h\to 0^{+}}\frac{f(p+h)-f(p)}{h}
	\]
	Analogamente, define-se derivada à esquerda. \(\square\)
\end{def*}
Já definimos a derivada de \(f:D_{f}\rightarrow \mathbb{R}\) em pontos p de
\(D_{f}\) que também são pontos de acumula\c cão de \(D_{f}\). Sendo assim,
se
\[
	D_{f'}= \biggl\{x\in D_{f}:x\text{ é um ponto de acumula\c cão de $D_{f}$ e }\lim_{h\to 0}\frac{f(x+h)-f(x)}{h}\text{ existe.}\biggr\}\subseteq{D_{f}},
\]
definimos a fun\c cão \(f':D_{f'}\rightarrow \mathbb{R}\) por
\[
	f'(x) = \lim_{h\to 0}\frac{f(x+h)-f(x)}{h}, \quad x\in D_{f'}.
\]
A fun\c cão f' é dita a fun\c cão derivada ou apenas derivada de f.
\hypertarget{diff_cont}{
	\begin{theorem*}
		Se f for diferenciável em p de \(D_{f}\), então f será contínua em p.
	\end{theorem*}
}

\begin{proof*}
	Recorde que \(p\in D_{f}\) é um ponto de acumula\c cão de \(D_{f}\).
	Vamos mostrar que \(\lim_{x\to p}f(x)=f(p)\) ou que \(\lim_{x\to p}(f(x)-f(p))=0.\)
	Escrevemos
	\[
		f(x)-f(p) = \frac{f(x)-f(p)}{x-p}(x-p).
	\]
	Assim,
	\[
		\lim_{x\to p}(f(x)-f(p)) = \lim_{x\to p}\frac{f(x)-f(p)}{x-p}\lim_{x\to p}(x-p) = f'(p)\cdot 0 = 0.
	\]

	Portanto, f é contínua em p.
\end{proof*}
Note que a recíproca não vale. A fun\c cão \(f(x) = |x|\) é contínua em x, mas
não é diferenciável em x=0.
\hypertarget{negative_diff}{   \begin{example}
		Se f não é contínua em p, então f não é diferenciável em p.
	\end{example}}
\begin{example}
	A fun\c cão
	\[
		f(x) = \left\{\begin{array}{ll}
			x^2,\quad x\geq 2 \\
			2,\quad x\leq 2
		\end{array}\right.
	\]
	não é contínua, logo não diferenciável.
\end{example}
\begin{def*}
	Seja f derivável em \(D_{f'}\). A fun\c cão \(f':D_{f'}\rightarrow \mathbb{R}\)
	é dita derivada de f ou derivada primeira de f. Definimos a derivada segunda de f
	como
	\[
		(f')'(x) = \lim_{h\to 0}\frac{f(x+h)-f'(x)}{h}
	\]
	quando o limite existe. Escrevemos \(f''=f^{(2)} = (f')'\) para denotá-la.

	Para \(n \in\mathbb{N}^{\times}\), podemos definir a derivada n-ésima analogamente. \(\square\).
\end{def*}
\begin{theorem*}
	Se \(k\in \mathbb{R}\) e \(n\in \mathbb{N}^{\times}\), valem as propriedades
	de derivada vistas em cálculo I. Além disso, valem as regras
	da derivada do produto, quociente, soma e multiplica\c cão por constante.
\end{theorem*}
Lembrando:
\[
	(f(x)g(x))' = f'(x)g(x) + f(x)g'(x)\quad \biggl(\frac{f}{g}\biggr)' = \frac{f'(x)g(x) - g'(x)f(x)}{g(x)^2}
\]

\hypertarget{chain_rule}{
	\begin{theorem*}
		Sejam \(f:D_{f}\rightarrow \mathbb{R}, g:D_{g}\rightarrow \mathbb{R}\) diferenciáveis com
		\(Im(g)\subseteq{D_{f}.}\) Se g é diferenciável em p, g(p) é ponto de acumula\c cão de \(D_{f}\),
		f é diferenciável em g(p) e \(h=f\circ{g}\), então h é diferenciável em p e
		\[
			h'(p) = f'(g(p))g'(p).
		\]
	\end{theorem*}
}
\begin{proof*}
	Fa\c ca q = g(p). Sejam \(\sigma _{g}\) e \(\sigma _{f}\) definidas em vizinhan\c cas
	de 0 com \(\lim_{h\to 0}\sigma_{g}(h) = 0\) e \(\lim_{k\to 0}\sigma _{f}(k) = 0\) tais que
	\begin{align*}
		 & g(p+h) = g(p) + [g'(p) + \sigma _{g}(h)]h  \\
		 & f(q+k) = f(q) + [f'(q) + \sigma _{f}(k)]k.
	\end{align*}
	Fazendo \(k=g(p+h)-g(p) = [g'(p)+\sigma _{g}(h)]h\), temos \(g(p+h) = q + k\) e
	\begin{align*}
		f(g(p+h)) = f(q+k) & = f(q) + [f'(q) + \sigma _{f}(k)]k                                                               \\
		                   & = f(q) + [f'(q) + \sigma _{f}(k)][g'(p)+\sigma _{g}(h)]h                                         \\
		                   & =f(g(p))+f'(f(p))g'(p)h + [\sigma _{f}(g(p+h)-g(p))[g'(p)+\sigma _{g}(h)]+f'(q)\sigma _{g}(h)]h.
	\end{align*}
	Agora, se \(\sigma _{f\circ{g}}(h) =[\sigma _{f}(g(p+h)-g(p))[g'(p)+\sigma _{g}(h)]+f'(q)\sigma _{g}(h)]h\),
	temos \(\lim_{h\to 0}\sigma _{f\circ{g}}(h)=0.\) \qedsymbol
\end{proof*}
Seja \(f:D_{f}\rightarrow \mathbb{R}\) uma fun\c cão que tem inversa, \(D_{f^{-1}}=Im(f)\)
e \(f^{-1}:D_{f^{-1}}\rightarrow \mathbb{R}\). Então, para todo x em \(D_{f^{-1}}\),
\[
	f(f^{-1}(x)) = x.
\]
Vimos que se f é contínua em um compacto, \(f^{-1}\) é contínua. Se, além disso,
\(f \) e \(f^{-1}\) forem deriváveis, pela \hyperlink{chain_rule}{Regra da Cadeia,}
\[
	f'(f^{-1}(x))(f^{-1})'(x) = x' = 1.
\]
Logo, \(f'(f^{-1}(x))\neq0\) e
\[
	(f^{-1})'(x) = \frac{1}{f'(f^{-1}(x))}.
\]
Esse estudo sobre a inversa é sumarizado no próximo resultado:
\begin{theorem*}
	Seja f injetiva, p um ponto de acumula\c cão de Im(f). Se f for diferenciável em \(q=f^{-1}(p)\) e
	\(f^{-1}\) é contínua em p, então \(f^{-1}\) é diferenciável em p se, e somente se,
	\(f'(f^{-1}(p))\neq 0.\) Neste caso,
	\[
		(f^{-1})'(x) = \frac{1}{f'(f^{-1}(x))}.
	\]
\end{theorem*}
\begin{proof*}
	Se \(f'(f^{-1}(p))\neq0\), como \(f^{-1}\) é contínua em p, \(\lim_{h\to 0}f^{-1}(p+h)=f^{-1}(p)\).
	Usando \(f(f^{-1}(x)), x\in D_{f^{-1}}\), temos
	\begin{align*}
		(f^{-1})'(p) & = \lim_{h\to 0}\frac{f^{-1}(p+h)-f^{-1}(p)}{h} = \lim_{h\to 0}\frac{1}{\frac{h}{f^{-1}(p+h)-f^{-1}(p)}}      \\
		             & = \lim_{h\to 0}\frac{1}{\frac{f(f^{-1}(p+h))-f(f^{-1}(p))}{f^{-1}(p+h)-f^{-1}(p)}} = \frac{1}{f'(f^{-1}(p))}
	\end{align*}
	Por outro lado, se \(f^{-1}\) é diferenciável em p, da \hyperlink{chain_rule}{regra da cadeia}
	aplicada a \(f\circ{f^{-1}}\), temos \(f'(f^{-1}(p))\cdot (f^{-1})'(p)=1\) e \(f'(f^{-1}(p))\neq0\).
\end{proof*}
\begin{example}
	Se \(g(x)=x^{\frac{1}{n}}\), então \(g'(x) = \frac{1}{n}x^{\frac{1}{n}-1}, 2\leq n\in \mathbb{N}\).
	Lembre-se que, \(x > 0\) se n for par e \(x\neq0\) se n for ímpar.

	Com efeito, note que \(g(x) = x^{\frac{1}{n}} = f^{-1}(x)\), em que \(f(u) = u^n.\) Então,
	\[
		g'(x) = (f^{-1})(x) = \frac{1}{f'(f^{-1}(x))} = \frac{1}{n(x^{\frac{1}{n}})^{n-1}} = \frac{1}{n(x^{\frac{n-1}{n}})} = \frac{1}{n}x^{\frac{1}{n}-1}.
	\]
\end{example}
\end{document}

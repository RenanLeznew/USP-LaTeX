\documentclass[analysis_notes.tex]{subfiles}
\begin{document}
\section{Aula 19 - 12/05/2023}
\subsection{O que esperar?}
\begin{itemize}
	\item Fun\c cões Deriváveis em Intervalos;
	\item Teorema de Darboux;
	\item Descontinuidades da Derivada.
\end{itemize}

\subsection{Fun\c cões Deriváveis em Intervalos}
\begin{example}
	Seja \(f:\mathbb{R}\rightarrow \mathbb{R}\) dada por \(f(x) = x^{3}\). Já
	sabemos que f é contínua. Como
	\begin{align*}
		f(x) - f(y) & =(x-y)(x^{2}+xy+y^{2})                        \\
		            & =(x-y)((x+\frac{y}{2})^{2}+\frac{3}{4}y^{2})  \\
		            & =(x-y)((\frac{x}{2}+y)^{2}+\frac{3}{4}x^{2}).
	\end{align*}
	Segue que \(x\neq y\) implica \(f(x)\neq f(y)\) e f é injetora. Além disso,
	\(\lim_{x\to \pm\infty}f(x)=\pm\infty\)e, pelo teorema do valor intermediário,
	f é uma bije\c cão de \(\mathbb{R}\) em \(\mathbb{R}\). Além disso,
	\(f^{-1}:\mathbb{R}\rightarrow \mathbb{R}\) é contínua pois ela é contínua em
	qualquer intervalo compacto. A inversa de f é denotada \(f^{-1}(x) = x^{\frac{1}{3}} = \sqrt[3]{x}.\)
	Do teorema sobre a derivada da inversa e do fato que \(f'(x) = 3x^{2}\)
	deduzimos que \(f^{-1}:\mathbb{R}\rightarrow \mathbb{R}\) é diferenciável se,
	e somente se, \(x\in \mathbb{R}\backslash\{0\}\) e, para estes valores de x.
	\[
		\overbrace{(\sqrt[3]{x})'}^{(f^{-1})'(x)} = \frac{1}{f'(f^{-1}(x))} = \frac{1}{3(\sqrt[3]{x})^{2}} = \frac{1}{3x^{\frac{2}{3}}}.
	\]
\end{example}
\begin{def*}
	Seja \(f:D\rightarrow \mathbb{R}\). Dizemos que f tem um máximo (mínimo) local no
	ponto de em D se existe \(\delta >0\) tal que \(f(x)\leq f(d) (f(x)\geq f(d))\)
	para todo x em D com \(|x-d| < \delta .\) Quando a desigualdade é estrita,
	dizemos que f tem um máximo (mínimo) local estrito. Os máximos e mínimos locais
	serão chamados de valores extremos e os pontos em que a fun\c cão assume valores máximos
	ou mínimos são chamados de pontos de máximo ou de mínimo. \(\square\)
\end{def*}
Pela defini\c cão de derivada (à direita), segue que
\begin{enumerate}
	\item Se \(f:D\rightarrow \mathbb{R}\) é não-decrescente (não-crescente) e é
	      diferenciável em um ponto d de D, então \(f'(d)\geq 0 (f'(d)\leq 0)\). Vale o mesmo
	      para fun\c cões diferenciáveis à direita.
	\item Se \(f:D\rightarrow \mathbb{R}\) é derivável à direita (esquerda) em um ponto
	      \(d\in D\) e \(f'(d^{+}) > 0 (f'(d^{-}) > 0)\) então existe \(\delta >0\) tal que
	      \(x\in D, x\in (d, d+\delta )(x\in (d-\delta , d)\) implica \(f(x) > f(d) (f(x) < f(d)).\)
	\item Se \(f:D\rightarrow \mathbb{R}\) é derivável à direita (esquerda) em um
	      ponto \(d\in D\) e \(f(d^{-}) < 0 (f'(d^{-}) < 0)\), então existe \(\delta >0\) tal que
	      \(x\in D, x\in(d, d+\delta )(x\in (d-\delta, d))\) implica \(f(x) < f(d) (f(x) > f(d)).\)
	\item Se \(f:D\rightarrow \mathbb{R}\) é derivável em um ponto \(d\in D\),
	      d sendo ponto de acumula\c cão à direita e à esquerda e \(f'(d) > 0\), existe \(\delta  > 0\)
	      tal que \(x\in D, d-\delta < x < d < y < d+\delta\) implica \(f(x) < f(d) < f(y).\)
	\item \(f:D\rightarrow \mathbb{R}\) é derivável em um ponto d em D, d é um ponto
	      de acumula\c cão à direita e à esquerda e f tem um valor extremo local em d, então \(f'(d) = 0.\)
\end{enumerate}
\begin{def*}
	Seja I um intervalo e \(f:I\rightarrow \mathbb{R}\) uma fun\c cão diferenciável.
	Se \(f':I\rightarrow \mathbb{R}\) for contínua diremos que f é continuamente diferenciável
	em I, ou simplesmente f é de classe \(C^{1}\) em I.
\end{def*}
Note que existe fun\c cão diferenciável em um intervalo I que não é
continuamente diferenciável.
\begin{example}
	Seja \(f:\mathbb{R}\rightarrow \mathbb{R}\) dada por
	\[
		f(x) = \left\{\begin{array}{ll}
			x^{2}\sin{(\frac{1}{x})},\quad x\neq0 \\
			f(0) = 0.
		\end{array}\right.
	\]
	Então, f é diferenciável, mas f' não é contínua em x = 0.
\end{example}
\hypertarget{darboux}{
	\begin{theorem*}
		Se \(f:[a, b]\rightarrow \mathbb{R}\) é diferenciável com \(f'(a)\neq f'(b)\), então,
		para todo C entre \(f'(a)\) e \(f'(b)\), existe \(c\in (a, b)\) tal que
		\(f'(c) = C\).
	\end{theorem*}}
\begin{proof*}
	Suponha que \(f'(a) < 0 < f'(b)\). Segue que, para x próximo a a em \([a, b], f(x) < f(a)\)
	e, para x próximo a b em \([a, b], f(x) < f(b).\) Logo, o ponto de mínimo (que existe pelo
	\hyperlink{weierstrass}{Teorema de Weierstrass} c de f ocorre em \((a, b)\) e
	portanto \(f'(c) = 0.\) Para o caso geral, consideramos
	\begin{itemize}
		\item Se \(f'(a) < C < f'(b), g(x) = f(x) - C \cdot x\)
		\item Se \(f'(a) > C > f'(b), g(x) = C \cdot x - f(x).\)
	\end{itemize}
\end{proof*}
\begin{theorem*}
	Se I é um intervalo e \(f:I\rightarrow \mathbb{R}\) é diferenciável, então
	f' não tem descontinuidades de primeira espécie.
\end{theorem*}
\begin{proof*}
	Se a é um ponto de acumula\c cão à direita de I e \(L^{+}=\lim_{x\to a^{+}}f(x)\)
	existe, mostremos que \(L^{+} = f'(a).\)

	De modo análogo (exercício), se a é um ponto de acumula\c cão à esquerda de I
	e \(L^{-}=\lim_{x\to a^{-}}f'(x)\) existe, mostre que \(f'(a) = L^{-}.\)

	Se \(L^{+} > f'(a)\) e \(C\in(f'(a), L^{+})\), existe \(\delta >0\) tal que
	\(f'(x) > C\) para todo \(x\in(a, a-\delta ).\) Escolhendo \(b\in(a, a+\delta )\),
	temos \(f'(b) > C > f'(a)\), uma contradi\c cão com o \hyperlink{darboux}{Teorema de Darboux},
	pois este implica a existência de \(c\in(a, b)\) tal que \(f'(c) = C.\) Logo,
	\(f'(a)\geq L^{+}.\)

	Se \(f'(a) > L^{+}\) e \(C\in(L^{+}, f'(a))\), existe \(\delta >0\) tal que
	\(f'(x) < C\) para todo x em \((a, a+\delta )\). Escolhendo b em \((a, a+\delta )\),
	temos \(f'(b) < C < f'(a)\), novamente contradizendo o \hyperlink{darboux}{Teorema de Darboux},
	visto que ele implica que deve existir c em \((a, b)\) tal que \(f'(c) = C.\) Assim,
	\(f'(a)\leq L^{+}.\) Portanto, \(L^{+} = f'(a).\) \qedsymbol
\end{proof*}
\begin{theorem*}
	Se \(f:[a, b)\rightarrow \mathbb{R}\) é contínua e diferenciável à direita com
	derivada à direita, \(D^{+}f:[a, b)\rightarrow \mathbb{R}.\) Se \(D^{+}f(x)\leq 0 (D^{+}f(x)\geq 0)\)
	para todo x em \([a, b)\) e \(f(a) = 0\), então \(f(x)\leq 0 (f(x)\geq 0)\) em \([a, b).\)
\end{theorem*}
\begin{proof*}
	Suponha primeiramente que \(D^{+}f(x) < 0\) para todo \(x\in[a, b).\) Se o resultado
	é falso, existe ao menos um \(x\in(a, b)\) tal que \(f(x) > 0.\) Seja
	\(x_{0}=\inf\{x\in(a, b): f(x) > 0\}.\)

	Da continuidade de f, \(f(x_{0}) = 0\) e da defini\c cão de \(x_{0}\) existe
	uma sequência \(x_{n}\in(x_{0}, b)\) tal que \(x_{n}\overbracket[0pt]{\longrightarrow}^{n\to \infty}x_{0}.\)
	Assim,
	\[
		D^{+}f(x_{0}) = \lim_{n\to \infty}\frac{f(x_{n})-f(x_{0})}{x_{n}-x_{0}}\geq 0,
	\]
	uma contradi\c cão. Logo, \(f(x)\leq 0\) para todo x em \([a, b)\).

	Agora, consideramos o caso geral \(D^{+}f(x)\leq 0\) para todo \(x\in[a, b).\)
	Neste caso, consideramos a fun\c cão auxiliar \(f_{\varepsilon }(x) = f(x) - \varepsilon (x-a)\)
	e temos \(f_{\varepsilon }(x)\leq 0\) para todo \(x\in[a, b)\) e \(\varepsilon >0.\)
	Disto segue que, para todo \(x\in[a, b), f(x)\leq 0\). O outro caso será deixado como
	exercício. \qedsymbol
\end{proof*}
\begin{example}
	Encontre uma fun\c cão \(f:\mathbb{R}\rightarrow \mathbb{R}\) que é diferenciável
	à direita, tal que \(D^{+}f(x) < 0\) para todo \(x\neq0, D^{+}f(0)=0,\) f é positiva
	para \(x > 0\) e negativa para \(x < 0 (f(x)\overbracket[0pt]{\longrightarrow}^{x\to \pm\infty}\pm\infty.)\)
\end{example}
\begin{crl*}
	Se \(f:[a, b)\rightarrow \mathbb{R}\) é contínua e diferenciável à direita com derivada
	à direita \(D^{+}f:[a, b)\rightarrow \mathbb{R}.\) Se \(D^{+}f(x)\leq 0\) para todo x em \([a, b)\),
	então f é não-crescente em \([a, b).\)
\end{crl*}
\begin{proof*}
	Se \(a\leq c < d < b,\) seja \(g:[c, b)\rightarrow \mathbb{R}\) definida por
	\(g(x) = f(x) - f(x)\) e \(D^{+}g(x)\leq 0\) para todo x em \([c, b)\). Segue do
	teorema que \(g(x)\leq 0\) para todo x em \([c, b).\) Em particular, \(g(d) = f(d) - f(c)\leq 0\). \qedsymbol
\end{proof*}
\begin{crl*}
	Se \(f:[a, b)\rightarrow \mathbb{R}\) é contínua e diferenciável à direita com
	derivada à direita \(D^{+}f:[a, b)\rightarrow \mathbb{R}.\) Se \(D^{+}f(x)\geq 0\)
	para todo x em \([a, b)\), então f é não-decrescente em \([a, b).\)
\end{crl*}
A prova deste exercício é deixada como exercício. Além disso, como exercício,
enuncie e prove resultados semelhantes aos anteriores para a derivada à esquerda.
\end{document}

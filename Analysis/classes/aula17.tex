\documentclass[analysis_notes.tex]{subfiles}
\begin{document}
\section{Aula 17 - 05/04/2023}
\subsection{O que esperar?}
\begin{itemize}
	\item Coberturas e Teorema de Borel-Lebesgue;
	\item Compactos da Reta;
	\item Continuidade Uniforme.
\end{itemize}
\subsection{Motivações}
Um subconjunto de $\mathbb{R}$ é dito compacto se satisfaz uma das condições abaixo - a caracterização por coberturas, por sequências, ou
por fechados.

Um conjunto $K$ é compacto se, para toda cobertura aberta do conjunto, existe uma subcobertura finita que ainda cobre o conjunto.
Ou seja, se temos um conjunto de intervalos abertos que cobrem todo o conjunto, sempre podemos encontrar um número finito desses intervalos
que ainda cobrem o conjunto.

Um conjunto $K$ é compacto se toda sequência de pontos do conjunto tem uma subsequência que converge para um ponto ainda dentro
do conjunto. Ou seja, se pegarmos uma lista infinita de pontos no conjunto, sempre poderemos encontrar uma "sublista" infinita
desses pontos que se aproxima cada vez mais de um único ponto no conjunto.

Um conjunto $K$ é fechado se contém todos os seus pontos de acumulação, e é limitado se existe um número real que é maior que todos os elementos do conjunto.

Além dos compactos, estaremos abordando outro tipo de continuidade. A continuidade uniforme é uma noção mais forte do que a continuidade regular. Seja $f: X \rightarrow Y$ uma função entre espaços métricos. Uma função é dita continua se, para todo $\varepsilon > 0$, existe um $\delta > 0$ tal que, para todos $x, y \in X$, se $|x - y| < \delta$ então $|f(x) - f(y)| < \varepsilon$.

A continuidade uniforme fortalece essa condição ao exigir que a escolha de $\delta$ não dependa de $x$. Em outras palavras,
para toda $\varepsilon > 0$, existe um $\delta > 0$ tal que, para todos $x, y \in X$, se $|x - y| < \delta$ então
$|f(x) - f(y)| < \varepsilon$. Isso implica que podemos encontrar um único "tamanho de passo" $\delta$ que funciona para todo
o domínio da função, em vez de precisar de um passo diferente para cada ponto.
Isso tem implicações importantes quando estamos trabalhando com conjuntos compactos, pois toda função contínua em um
conjunto compacto é uniformemente contínua. Isso é útil porque as funções uniformemente contínuas têm propriedades muito agradáveis,
como a capacidade de serem estendidas de um modo contínuo a partir de um conjunto denso.

A diferença entre continuidade e continuidade uniforme é sutil. No entanto, essa diferença é importante em muitos contextos.
Por exemplo, a função $f(x) = \frac{1}{x}$ é contínua em todos os pontos do seu domínio (todos os números reais não nulos),
mas não é uniformemente contínua. Isso ocorre porque, conforme nos aproximamos de zero, precisamos escolher $\delta$ cada vez
menor para garantir que $d(f(x), f(x')) < \varepsilon$ para um $\varepsilon$ fixo. Não podemos encontrar um único $\delta$ que
funcione para todos os pontos do domínio.

Outro fato importante é que toda função contínua em um conjunto compacto é uniformemente contínua. Isso se deve às propriedades
dos conjuntos compactos, que garantem que podemos encontrar um $\delta$ que funcione para todo o conjunto. Isso é uma
propriedade muito útil, pois a continuidade uniforme garante algumas propriedades agradáveis para as funções, como a
capacidade de serem estendidas de maneira contínua a partir de um conjunto denso.

A seguir, vamos elaborar cada um destes conceitos, a fim de obter um entendimento mais profundo sobre o estudo de funções e da
topologia da reta.

\subsection{Compactos e Coberturas}
\begin{def*}
	Seja $A\subseteq{\mathbb{R}}$ e $\Lambda $ um conjunto. Uma coleção $\{A_{\lambda }\}_{\lambda \in\Lambda }$ de conjuntos
	é chamada uma cobertura de A se $A\subseteq{\bigcup_{\lambda \in\Lambda }^{}{A_{\lambda }}}.\square$
\end{def*}
\begin{def*}
	Se $\{A_{\lambda }\}_{\lambda \in\Lambda }$ é uma cobertura, $\Lambda '\subseteq{\Lambda }$ e $A\subseteq{\bigcup_{\lambda '\in\Lambda '}^{}{A_{\lambda '}}},
		\{A_{\lambda '}\}_{\lambda '\in\Lambda '}$ é dita uma subcobertura da cobertura $\{A_{\lambda }\}_{\lambda \in\Lambda }.$
	Se os conjuntos da cobertura são todos abertos, ela é dita uma cobertura aberta. $\square$
\end{def*}
\begin{theorem*}
	Este resultado é conhecido como Teorema de Borel-Lebesgue. Dada uma cobertura $\{I_{\lambda }\}_{\lambda \in\Lambda }$ de [a, b]
	em que cada $I_{\lambda }$ é um intervalo aberto, existe $\Lambda '\subseteq{\Lambda }$ finito tal que $[a, b]\subseteq{\bigcup_{\lambda '\in\Lambda '}^{}{I_{\lambda '}.}}$
\end{theorem*}
\begin{proof*}
	Seja $A=\{x\in[a,b]: \exists \Lambda'\subseteq{\Lambda }\text{ finito }\&[a,x]\subseteq{\bigcup_{\lambda '\in\Lambda '}^{}{I_{\lambda '}}}\}$.
	Note que $A \neq\emptyset.$ Seja $s=\sup{A}.$ Segue que $s\in[a, b]$ e que existe $\lambda _{s}$ tal que $s\in I_{\lambda _{s}}.$
	Como $I_{\lambda _{s}}$ é aberto, $I_{\lambda _{s}}\cap A \neq\emptyset$. Portanto, s = b e $[a, b]$ está contido em uma
	união finita de $I_{\lambda }'s.$ \qedsymbol
\end{proof*}
\begin{crl*}
	Dada uma cobertura aberta $\{A_{\lambda }\}_{\lambda \in\Lambda }$ de [a, b], existe $\Lambda '\subseteq{\Lambda }$ finito
	tal que $[a,b]\subseteq{\bigcup_{\lambda '\in\Lambda '}^{}{A_{\lambda'}}}.$
\end{crl*}
Basta lembrar que cada aberto da cobertura pode ser escrito como a união enumerável de intervalos abertos disjuntos.
\begin{crl*}
	Dada uma cobertura aberta $\{A_{\lambda }\}_{\lambda \in\Lambda }$ de um conjunto fechado e limitado F existe $\Lambda '\subseteq{\Lambda }$
	finito tal que $F\subseteq{\bigcup_{\lambda '\in\Lambda '}^{}{A_{\lambda '}}}$
\end{crl*}
\begin{proof*}
	Como F é fechado e limitado, $F\subseteq{[a,b]}$. Como $A=[a,b]^{c}$ é aberto e $[a,b]\subseteq{\bigcup_{\lambda \in\Lambda }^{}{A_{\lambda }}\cup{A}}$, temos
	a existência de $\Lambda '\subseteq{\Lambda }$ finito tal que
	$$
		[a,b] \subseteq{\bigcup_{\lambda '\in\Lambda '}^{}{A_{\lambda '}}\cup{A}}.
	$$
	Portanto, $F\subseteq{\bigcup_{\lambda '\in\Lambda '}^{}{A_{\lambda '}}}.$ \qedsymbol
\end{proof*}
\begin{theorem*}
	Dado $K\subseteq{\mathbb{R}}$ são equivalente:
	\begin{itemize}
		\item[1)] K é fechado e limitado;
		\item[2)] Toda cobertura aberta de K possui uma subcobertura finita;
		\item[3)] Todo subconjunto infinito de K possui um ponto de acumulação pertencente a K;
		\item[4)] Toda sequência de pontos de K possui uma subsequência que converge para um ponto de K.
	\end{itemize}
\end{theorem*}
\begin{proof*}
	$1) \Rightarrow 2):$ Segue diretamente do corolário anterior.

	$2) \Rightarrow 3):$ Se $A\subseteq{K}$ é infinito e não tem pontos de acumulação em K, para cada k em K existe $I_{k}=r_{k} > 0$
	tal que $(k-r_{k}, k+r_{k})\cap{A} = \{k\}$ ou  $I_{k}\cap A = \emptyset.$ Segue que $\bigcup_{k\in K}^{}{I_{a}}\supseteq{K}$ é
	uma cobertura aberta sem subcobertura finita.

	$3) \Rightarrow 4):$ Dada uma sequência de pontos $\{k_{n}\}$ em K, ela pode ter um número finito ou infinito de valores.
	Em qualquer um dos casos, possui uma subsequência convergente.

	$4) \Rightarrow 1):$ É claro que K é limitado, visto que, caso contrário, existiria uma sequência $\{x_{n}\}$ em K com $x_{0}\in K$
	e $|x_{n}|\geq |x_{n-1}| + 1$ e esta não teria subsequência convergente. Para ver que K é fechado, basta notar que, se $x\in K^{-}$,
	existe sequência $x_{n}\in K$ tal que $x_{n}\overbracket[0pt]{\longrightarrow}^{n\to \infty}x$. Portanto, pela hipótese de 4, $x\in K.$\qedsymbol
\end{proof*}
\begin{def*}
	Um conjunto é compacto se satisfaz qualquer uma das condições do teorema anterior. $\square$
\end{def*}
Como corolário do teorema passado, temos o Teorema de Bolzano-Weierstrass!
\begin{crl*}
	Todo conjunto infinito e limitado de número reais possui um ponto de acumulação.
\end{crl*}
\begin{crl*}
	Toda sequência decrescente de compactos não vazio têm interseção não vazia.
\end{crl*}
\begin{def*}
	Seja $A\subseteq{\mathbb{R}}$ e $B\subseteq{A}$, diremos que B é aberto em A se, para cada $b\in B$, existe um $r_{b}>0$ tal
	que $A\cap (b-r_{b}, b + r_{b})\subseteq{B}.\square$
\end{def*}
Note que todo conjunto é aberto nele mesmo. Além disso, se $A\subseteq{\mathbb{R}}$ e $B\subseteq{A}$, então B é aberto em A
se, e somente se, existe um aberto $\mathcal{O}_{B}$ de $\mathbb{R}$ tal que $B=\mathcal{O}_{B}\cap A.$ Além disso,
se B é aberto, $A\subseteq{B}$ é aberto em B se, e somente se, B é aberto em $\mathbb{R}.$

Lembre-se que, se $f:D\rightarrow \mathbb{R}$ é uma função, $f^{-1}(\mathcal{O})=\{d\in D: f(d)\in \mathcal{O}\}.$

\begin{theorem*}
	Seja $D\subseteq{\mathbb{R}}$ e $f:D\rightarrow \mathbb{R}.$ A função f é contínua se, e somente se, para todo
	$\mathcal{O}$ de $\mathbb{R}, f^{-1}(\mathcal{O})$ é aberto em D.
\end{theorem*}
\begin{proof*}
	Se $f:D\rightarrow \mathbb{R}$ é contínua, $\mathcal{O}$ é um aberto de $\mathbb{R}$ e $d\in f^{-1}(\mathcal{O}),$
	então $f(d)\in \mathcal{O}$ e dado $\varepsilon  > 0$ tal que $(f(d)-\varepsilon , f(d)+\varepsilon )\subseteq{\mathcal{O}},$
	existe $\delta >0$ tal que $f((d-\delta , d+\delta )\cap D)\subseteq{(f(d)-\varepsilon , f(d)+\varepsilon ).}$ Isot mostra que
	$(d-\delta , d+\delta )\cap D\subseteq{f^{-1}(\mathcal{O})}$ e que $f^{-1}(\mathcal{O})$ é aberto em D.

	Por outro lado, de $f^{-1}(\mathcal{O})$ ser aberto em D para cada $\mathcal{O}$ aberto em $\mathbb{R},$ se d é um elemento
	de D, dado $\varepsilon  > 0$, seja $\mathcal{O} = (f(d)-\varepsilon , f(d)+\varepsilon ).$ Como $d\in f^{-1}((f(d)-\varepsilon , f(d)+\varepsilon ))$
	é aberto em D, existe $\delta >0$ tal que $(d-\lambda , d+\delta )\cap D\subseteq{f^{-1}((f(d)-\varepsilon , f(d)+\varepsilon ))},$
	ou seja,
	$$
		x\in D, |x-d| < \delta \Rightarrow |f(x)-f(d)| < \varepsilon
	$$
	e f é contínua em d. \qedsymbol
\end{proof*}
\begin{theorem*}
	Se $I\subseteq{\mathbb{R}}$ é um intervalo e $f:I\rightarrow \mathbb{R}$ é uma função contínua, então f(I) é um intervalo.
\end{theorem*}
\begin{proof*}
	Basta notar que, dados dois pontos $f(a)\neq f(b)$ em f(I) com $a < b,$ tomando a restrição de f ao intervalo [a, b],
	segue do teorema do valor intermediário que, para todo k entre f(a) e f(b), existe um c em (a, b) tal que f(c) = k. Portanto,
	f(I) é um intervalo. \qedsymbol
\end{proof*}
\begin{theorem*}
	Se $K\subseteq{\mathbb{R}}$ é um conjunto compacto e $f:K\rightarrow \mathbb{R}$ é uma função contínua, então f(K) é compacto.
\end{theorem*}
\begin{proof*}
	Seja $\{\mathcal{O}_{\lambda }:\lambda \in\Lambda \}$ uma cobertura aberta de f(K). Como, para cada $\lambda \in \Lambda, f^{-1}(\mathcal{O}_{\lambda })$
	é aberto em K, existe $U_{\lambda }$ aberto em $\mathbb{R}$ tal que $U_{\lambda }\cap K = f^{-1}(\mathcal{O}_{\lambda }).$
	Assim, $\{U_{\lambda }:\lambda \in\Lambda \}$ é uma cobertura aberta de K. Como K é compacto, existe $\lambda '\subseteq{\Lambda }$
	finito tal que $\bigcup_{\lambda '\in\Lambda '}^{}{U_{\lambda '}}\supseteq{K}.$ Segue que $\{\mathcal{O}_{\lambda '}: \lambda '\in\Lambda '\}$
	é uma subcobertura finita da cobertura $\{\mathcal{O}_{\lambda }:\lambda \in \Lambda \}$ de f(K). Portanto, f(K) é compacto. \qedsymbol
\end{proof*}
\begin{proof*}
	Uma prova alternativa faz uso de sequências: Seja $\{y_{n}\}$ uma sequência em f(K). Então, existe uma sequência
	$\{x_{n}\}$ em K tal que $y_{n}=f(x_{n}).$ Como K é compacto, $\{x_{n}\}$ tem uma subsequência $\{x_{\varphi (n)}\}(\varphi :\mathbb{N}\rightarrow \mathbb{N}$
	estritamente crescente) convergente com limite $\overline{x}\in K.$ Como $x_{\varphi (n)}\overbracket[0pt]{\longrightarrow}^{n\to \infty}\overline{x},
		y_{\varphi (n)}=f(x_{\varphi (n)})\overbracket[0pt]{\longrightarrow}^{n\to \infty}f(x)$ e $\{y_{n}\}$ tem uma
	subsequência convergente com limite em f(K). Portanto, f(K) é compacto. \qedsymbol
\end{proof*}
\begin{theorem*}
	Se $K\subseteq{\mathbb{R}}$ é um conjunto compacto e $f:K\rightarrow \mathbb{R}$ é contínua, existem $x_{1},x_{2}\in K$ tal que
	$f(x_{1})\leq f(x)\leq f(x_{2})$ para todo x em K.
\end{theorem*}
\begin{proof*}
	Como f(K) é compacto, $L=\sup{\{y:y\in f(K)\}}$ e $l=\inf{\{y:y\in f(K)\}}$ pertencem a f(K). Portanto, existem $x_{1},x_{2}\in K$
	tais que $f(x_{1})=l\leq f(x)\leq L = f(x_{2})$ para todo x de K. \qedsymbol
\end{proof*}
\end{document}

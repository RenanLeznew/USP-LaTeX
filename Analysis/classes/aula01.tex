\documentclass[Analysis/analysis_notes.tex]{subfiles}
\begin{document}
\section{Aula 01 - 13/03/2023}
\subsection{Motiva\c c\~ao}
\begin{itemize}
	\item Relembrar sistemas b\'asicos da matem\'atica;
	\item Relembrar propriedades b\'asicas das principais estruturas ($\mathbb{N}, \mathbb{Z}, \mathbb{Q}$).
\end{itemize}

\subsection{Os N\'umeros Naturais}
Os n\'umeros naturais s\~ao os que utilizamos para contar objetos, e s\~ao caracterizados pelos Axiomas de Peano:
\begin{itemize}
	\item[1)] Todo n\'umero natural tem um \'unico sucessor;
	\item[2)] N\'umeros naturais diferentes t\^em sucessores diferentes;
	\item[3)] Existe um \'unico n\'umero natural, zero (0), que n\~ao \'e sucessor de nenhum n\'umero natural.
	\item[4)] Seja $X \subseteq{\mathbb{N}}$ tal que $0\in{X}$ e, se n pertence a X, seu sucessor n+1 tamb\'em pertence
	      a X. Ent\~ao, X = $\mathbb{N}.$ (Propriedade de Indu\c c\~ao).
\end{itemize}

\begin{def*}
	Definimos a adi\c c\~ao por: $n + 0 = n, n\in \mathbb{N},\text{ e }n+(p+1) = (n+p)+1, p\in{\mathbb{N}}$. Al\'em disso,
	a multiplica\c c\~ao \'e dada por: $n.0 = 0, n.(p+1) = n.p + n, n, p\in\mathbb{N}.$ Ou seja, sabendo somar ou multiplicar um n\'umero,
	sabemos somar e multiplicar seu sucessor.
\end{def*}
Com rela\c c\~ao ao quarto axioma, ele leva este nome porque um dos m\'etodos de demonstra\c c\~ao, conhecido como
prova por indu\c c\~ao. Nele, mostramos um caso base, o caso 0, e utilizamos a segunda parte para provar que, se um
resultado vale para o caso n, ele vale para n+1, portanto sendo verdadeiro para todos os naturais.

\begin{lemma*}
	Para todo n natural, 1 + n = n + 1.
\end{lemma*}
\begin{proof*}
	Note que o resultado \'e verdadeiro para n = 0. Suponha que o resultado seja v\'alido para n = k e mostremos que
	vale tamb\'em para n = k+1. Com efeito, segue pela propriedade de indu\c c\~ao e pela defini\c c\~ao de soma que
	$$
		1 + (k + 1) = (1 + k) + 1 =  (k + 1) + 1.
	$$
	Segue que o resultado vale para todo n natural. \qedsymbol
\end{proof*}
A seguir, mostramos a associatividade e a comutatividade, respectivamente, das opera\c c\~oes nos naturais.
\begin{lemma*}
	Para todo n, p, r naturais, (n + p) + r = n + (p + r).
\end{lemma*}
\begin{proof*}
	Note que o resultado \'e v\'alido trivialmente para r = 0 e r = 1. Suponha que o resultado seja v\'alido para
	r = k e mostremos que vale tamb\'em para r = k + 1. Com efeito, pela hip\'otese de indu\c c\~ao e defini\c c\~ao de adi\c c\~ao,
	$$
		n + (p + (k + 1)) = n + ((p + k) + 1) = (n + (p + k)) + 1 = ((n + p) + k) + 1 = (n + p) + (k + 1).
	$$
	Segue o resultado por indu\c c\~ao. \qedsymbol
\end{proof*}
\begin{lemma*}
	Para todo n, p naturais, n + p = p + n.
\end{lemma*}
\begin{proof*}
	Observe que j\'a mostramos o caso em que p = 1. Suponha que o resultado vale para p = k e vamos mostrar o caso
	p = k + 1. De fato, pela hip\'otese de indu\c c\~ao e defini\c c\~ao de adi\c c\~ao, junto do lema de associatividade,
	temos
	$$
		n + (k + 1) = (n + k) + 1 = (k + n) + 1 = 1 + (k + n) = (1 + k) + n = (k + 1) + n.
	$$
	Por indu\c c\~ao, segue que isso vale para todo natural n. \qedsymbol
\end{proof*}

\begin{def*}
	Definimos uma ordem em $\mathbb{N}$ colocando que $m\leq{n}$ se existe p natural tal que $n = m + p. \square$
\end{def*}
A rela\c c\~ao de ordem possui as seguintes propriedades:
\begin{itemize}
	\item[i)] Reflexiva: Para todo n natural, $n\leq{n};$
	\item[ii)] Antissim\'etrica: Se $m\leq n$ e $n\leq m,$ ent\~ao $m = n;$
	\item[iii)] Transitiva: Se $m \leq n$ e $n \leq p$, ent\~ao $m\leq p;$
	\item[i] Dados m, n naturais, temos ou $m \leq n$, ou $n \leq m;$
	\item[v] Se $m \leq n$ e p \'e um natural, ent\~ao $n + p\leq n\text{ e } mp\leq np$
\end{itemize}

\subsection{N\'umeros Inteiros e Racionais}
Usualmente, construimos os inteiros a partir dos naturais tomando os pares ordenados de n\'umeros naturais
com a seguinte identifica\c c\~ao (a, b) $\mathtt{\sim}$ (c, d) se a + d = b + c. Assim, podemos representar
$$
	\mathbb{N} = \{(0, 0), (1, 0), (2, 0), (3, 0), \ldots\}, \quad -\mathbb{N}^* = \{\cdots, (0, 3), (0, 2), (0, 1)\}.
$$
Tomar o sucessor ser\'a somar 1 \`a primeira coordenada e, para os inteiros negativos, voltar a identificar (1, n) com (0, n-1).

Os n\'umeros racionais s\~ao constru\'idos tomando o conjunto $\mathbb{Z}\times{\mathbb{Z}^*}$ e identificando os pares $(a, b)\mathtt{\sim}(c, d)$
para os quais ad = bc. Representamos um par (a, b) neste conjunto por $\displaystyle \frac{a}{b}.$ A soma e o produto em $\mathbb{Q}$
s\~ao dados, respectivamente, por:
\begin{align*}
	 & \frac{a}{b} + \frac{c}{d} \coloneqq \frac{ad + bc}{bd} \\
	 & \frac{a}{b}\cdot\frac{c}{d} \coloneqq \frac{ac}{bd}.
\end{align*}
Chamamos a adi\c c\~ao a opera\c c\~ao que a cada par $(x, y)\in \mathbb{Q}\times{\mathbb{Q}}$ associa sua soma $x+y\in \mathbb{Q}$
e chamamos multiplica\c c\~ao a opera\c c\~ao que a cada par $(x, y)\in \mathbb{Q}\times \mathbb{Q}$ associa seu produto $x.y\in \mathbb{Q}.$
A terna $(\mathbb{Q}, +, \cdot)$ satisfaz as propriedades de um corpo, i.e.,
\begin{align*}
	 & (A1) (x + y) + z = x + (y + z), \quad\forall x, y, z\in \mathbb{Q}             \\
	 & (A2) x + y = y + x, \quad\forall x, y\in \mathbb{Q}                            \\
	 & (A3) \exists 0\in \mathbb{Q}: x + 0 = x, \quad\forall x\in \mathbb{Q}          \\
	 & (A4) \forall x\in \mathbb{Q}, \exists y\in \mathbb{Q} (y = -x): x + y = 0      \\
	 & (M1) (xy)z = x(yz), \quad\forall x, y, z\in \mathbb{Q}                         \\
	 & (M2) xy = yx, \quad x, y\in \mathbb{Q}                                         \\
	 & (M3) \exists 1\in \mathbb{Q}: 1.x = x.1 = x, \quad\forall x\in \mathbb{Q}      \\
	 & (M4) \forall x\in \mathbb{Q}^*, \exists y = \frac{1}{x}\in \mathbb{Q}: x.y = 1 \\
	 & (D) x(y+z) = xy + xz,\quad\forall x, y, z\in \mathbb{Q}.
\end{align*}
\end{document}

\documentclass[analysis_notes.tex]{subfiles}
\begin{document}
\section{Aula 13 - 17/04/2023}
\subsection{Motiva\c c\~oes}
\begin{itemize}
	\item Limites e Continuidade de Fun\c c\~oes;
	\item Limites laterais.
\end{itemize}
\subsection{Intui\c c\~ao e Exemplos Iniciais}
Entender continuidade e limites é fundamental no cálculo. Aqui estão alguns exemplos que podem ajudar a esclarecer esses conceitos:

\begin{enumerate}
	\item Continuidade: \\
	      Uma função é contínua em um ponto se o seu valor nesse ponto é igual ao limite da função quando se aproxima desse ponto. Em outras palavras, não há quebras ou saltos no gráfico da função. Um exemplo clássico de uma função contínua é a função quadrática:

	      \begin{equation*}
		      f(x) = x^2
	      \end{equation*}

	      Essa função é contínua para todos os números reais, pois não há quebras ou saltos no gráfico.

	      \begin{center}
		      \begin{tikzpicture}
			      \begin{axis}[
					      xlabel={$x$},
					      ylabel={$f(x)=x^2$},
					      domain=-3:3,
					      samples=100,
					      axis lines=middle
				      ]
				      \addplot[blue, thick] {x^2};
			      \end{axis}
		      \end{tikzpicture}
	      \end{center}

	\item Limites: \\
	      Um limite é o valor que uma função se aproxima quando a entrada (valor x) se aproxima de um determinado valor. O conceito de limites nos ajuda a entender o comportamento das funções próximas aos pontos onde elas não são definidas ou não são contínuas.

	      Exemplo 1: Limite simples \\
	      Considere a função:

	      \begin{equation*}
		      g(x) = 3x + 2
	      \end{equation*}

	      Encontre o limite quando x se aproxima de 1:

	      \begin{equation*}
		      \lim_{x \to 1} (3x + 2)
	      \end{equation*}

	      Como essa é uma função linear e contínua em todos os lugares, o limite é o mesmo que o valor da função no ponto:

	      \begin{align*}
		      g(1) & = 3(1) + 2 \\
		           & = 5
	      \end{align*}

	      Portanto, $\lim_{x \to 1} (3x + 2) = 5$.

	      \begin{center}
		      \begin{tikzpicture}
			      \begin{axis}[
					      xlabel={$x$},
					      ylabel={$g(x)=3x+2$},
					      domain=-2:3,
					      samples=100,
					      axis lines=middle
				      ]
				      \addplot[red, thick] {3*x+2};
			      \end{axis}
		      \end{tikzpicture}
	      \end{center}

	      Exemplo 2: Limite de uma função com descontinuidade \\
	      Considere a função:

	      \begin{equation*}
		      h(x) = \frac{x^2 - 1}{x - 1}
	      \end{equation*}

	      Essa função não é definida para x = 1, mas ainda podemos encontrar o limite quando x se aproxima de 1:

	      \begin{equation*}
		      \lim_{x \to 1} \frac{x^2 - 1}{x - 1}
	      \end{equation*}

	      Observe que $(x^2 - 1)$ pode ser fatorado como $(x + 1)(x - 1)$. Então, obtemos:

	      \begin{equation*}
		      \lim_{x \to 1} \frac{(x + 1)(x - 1)}{x - 1}
	      \end{equation*}

	      Agora, podemos cancelar os termos $(x - 1)$:

	      \begin{equation*}
		      \lim_{x \to 1} (x + 1)
	      \end{equation*}

	      Neste ponto, podemos substituir x = 1:

	      \begin{align*}
		      1 + 1 & = 2
	      \end{align*}

	      Portanto, $\lim_{x \to 1} (h(x)) = 2$, embora a função em si não seja definida para x = 1.

	      \begin{center}
		      \begin{tikzpicture}
			      \begin{axis}[
					      xlabel={$x$},
					      ylabel={$h(x)=\frac{x^2 - 1}{x - 1}$},
					      domain=-3:3,
					      ymin=-5, ymax=5,
					      samples=100,
					      axis lines=middle,
					      restrict y to domain=-5:5
				      ]
				      \addplot[green, thick] {((x^2 - 1)/(x - 1))};
				      \addplot[only marks, mark=o, mark options={scale=1.5}] coordinates {(1,2)};
			      \end{axis}
		      \end{tikzpicture}
	      \end{center}
\end{enumerate}

Esses exemplos demonstram como a continuidade e os limites nos ajudam a analisar e entender o comportamento das funções, mesmo em pontos onde elas podem não ser definidas.
A seguir, formalizaremos essas ideias e provaremos algumas propriedades.

\subsection{Limites de Fun\c c\~oes}
\begin{def*}
	Seja D um subconjunto de $\mathbb{R}, f:D\rightarrow \mathbb{R}$ uma fun\c c\~ao e p um ponto de acumula\c c\~ao de D. Diremos que
	o limite de f(x) quando x tende a p \'e L se, dado $\varepsilon > 0,$ existe um $\delta > 0$ tal que
	$$
		x\in D\text{ e } 0 < |x-p| < \delta, \Rightarrow |f(x) - L | < \varepsilon.
	$$
	Em outras palavras, dado $\varepsilon > 0$, existe $\delta = \delta(\varepsilon, p) > 0$ tal que
	$$
		f(D\cap(p-\delta, p+\delta)/a)\subseteq{(L-\varepsilon, L+\varepsilon).} \square
	$$
\end{def*}
Note que, se n\~ao existe um n\'umero real L tal que $\lim_{x\to p}= L,$ ent\~ao diremos que o limite n\~oa existe. Al\'em disso,
o ponto p n\~ao precisa ser um ponto do dom\'inio D e, mesmo que perten\c ca, o valor de f em p n\~ao \'e importante para a defini\c c\~ao.
Apenas os valores de f em pontos arbitrariamente pr\'oximos a p s\~ao importantes para a defini\c c\~ao.
\begin{theorem*}
	Seja $f:D\rightarrow \mathbb{R}$ uma fun\c c\~ao e p um ponto de acumula\c c\~ao de D. O limite de f(x) quando x tende a p,
	caso existe, \'e \'unico, e ser\'a dneotado por
	$$
		\lim_{x\to p}f(x) = L.
	$$
\end{theorem*}
\begin{proof*}
	De fato, se L e L' s\~ao limites de f(x) quando x tende a p, dado $\varepsilon > 0$, existe $\delta > 0$ tal que
	$$
		x\in D, 0 < |x-p| < \delta \Rightarrow |f(x)- L| < \varepsilon, |f(x) - L'|< \varepsilon.
	$$
	Logo, dado $\varepsilon > 0$, com a escolha de $\delta$ acima e x em D satisfazendo $0 <|x-p|<\delta$, temos
	$$
		|L - L'| = |L -f(x) + f(x) - L'| \leq{} |L - f(x)| + |f(x) - L'| < 2\varepsilon.
	$$
	Portanto, como $\varepsilon$ \'e arbitr\'ario, L = L'. \qedsymbol
\end{proof*}
Quando nos referimos a uma fun\c c\~ao, fica impl\'icito que ela tem um dom\'inio especificado. Dada a fun\c c\~ao
$f:D\rightarrow \mathbb{R}$ e $D'\subseteq{D},$ denotaremos por $f_{|_{D'}}:D'\rightarrow \mathbb{R}$ a fun\c c\~ao
definida por $f_{|_{D'}}(x) = f(x)$ para x em D'. Segue dessa defini\c c\~ao que
\begin{theorem*}
	Seja D um subconjunto dos reais, $f:D\rightarrow \mathbb{R}$ uma fun\c c\~ao, D' um subconjunto de D e p um ponto de aucmula\c c\~ao de D'.
	Se $\lim_{x\to p}f(x) = L,$ ent\~ao $\lim_{x\to p}f_{|_{D'}}(x) = L.$
\end{theorem*}
\begin{theorem*}
	Seja $f:D\rightarrow \mathbb{R}$ uma fun\c c\~ao, D' e D'' subconjuntos de D e p um n\'umero real que \'e ponto de acumula\c c\~ao
	de D' e de D''.
	\begin{itemize}
		\item[i)] Se um dos limites $\lim_{x\to p}f_{|_{D'}}(x)$ ou $\lim_{x\to p}f_{|_{D''}}(x)$ n\~ao existe, ou ambos existem,
		      mas s\~ao diferentes, ent\~ao o limite $\lim_{x\to p}f(x)$ n\~ao existe.
		\item[ii)] Se $(D'\cup D'')/\{p\} = D/\{p\}$, o limite $\lim_{x\to p}f(x)$ existe se, e somente se, $\lim_{x\to p}f_{|_{D'}}(x)$ e
		      $\lim_{x\to p}f_{|_{D''}}(x)$ existem e t\^em o mesmo valor.
	\end{itemize}
\end{theorem*}
\begin{proof*}
	A prova da primeira parte segue direto. Para a segunda, existe $\lim_{x\to p }f(x) = L$ se, e somente se, dado $\varepsilon > 0$,
	existe $\delta > 0$ tal que
	$$
		x\in D, 0 < |x-p| < \delta \Rightarrow |f(x)-L| < \varepsilon,
	$$
	o que equivale a dizer que, dado $\varepsilon > 0$, existe $\delta > 0$ tal que
	$$
		x\in D', 0 < |x-p| < \delta \Rightarrow |f(x) - L|< \varepsilon \Rightarrow |f(x)_{|_{D'}}|<\varepsilon
	$$
	e
	$$
		x\in D'', 0 < |x-p| < \delta \Rightarrow |f(x) - L|< \varepsilon \Rightarrow |f(x)_{|_{D''}}|<\varepsilon
	$$
	Portanto, $\lim_{x\to p}f_{|_{D''}}(x) = \lim_{x\to p}f_{|_{D'}}(x)$. \qedsymbol
\end{proof*}
\begin{def*}
	Se D \'e um subconjunto de $\mathbb{R}$, diremos que p real \'e um ponto de acumula\c c\~ao \`a direita de D se \'e um ponto
	de acumula\c c\~ao de $D_{p}^{+} = D\cap{(p, \infty)}$. Seja $f:D\rightarrow \mathbb{R}$ uma fun\c c\~ao e p um ponto de
	acumula\c c\~ao \`a direita de D. O limite de f(x) quando x tende a p pela direita \'e
	$$
		\lim_{x\to p^{+}}f(x)\coloneqq \lim_{x\to p}f_{|_{D_{p}^{+}}}(x).
	$$
	Define-se analogamente limite de f(x) quando x tende a pela esquerda e ponto de acumula\c c\~ao \`a esquerda.
\end{def*}
\begin{crl*}
	Seja $f:D\rightarrow \mathbb{R}$ uma fun\c c\~ao e p um ponto de acumula\c c\~ao \`a direita e \`a esquerda de D. Ent\~ao,
	$$
		\lim_{x\to p}f(x)
	$$
	existe se, e somente se, os limites laterais existem e s\~ao iguais.
\end{crl*}
\begin{theorem*}
	Seja D subconjunto real e $f:D\rightarrow \mathbb{R}$ uma fun\c c\~ao. Tome p como um ponto de acumula\c c\~ao de D. Se
	existe $\lim_{x\to p}f(x) - L$, ent\~ao f \'e limitada em uma vizinha\c ca de p, i.e., existem $M>0$ e $\delta > 0$ tais que
	se x pertence a D e $0 <|x-p| < \delta,$ ent\~ao $|f(x)|< M$
\end{theorem*}
\begin{proof*}
	Existem $\delta > 0$ tal que se x pertence a D e $0<|x-p|<\delta,$ ent\~ao $|f(x)-L|<1.$ Logo,
	$$
		|f(x)| \leq{|f(x)-L| + |L|} \leq{1 + |L| = M},
	$$
\end{proof*}
\begin{theorem*}
	Seja $D\subseteq{\mathbb{R}}, f, g, h:D\rightarrow \mathbb{R}$ fun\c c\~oes e p um ponto de acumula\c c\~ao de D. Se
	para todo x em D diferente de p, $f(x)\leq{g(x)}\leq{h(x)}$ e $\lim_{x\to p}f(x) = \lim_{x\to p }h(x) = L,$ ent\~ao $\lim_{x\to p}g(x) = L.$
\end{theorem*}
\begin{proof*}
	Com efeito, dado $\varepsilon > 0$, existe $\delta > 0$ tal que se x pertence a D e $0 < |x-p| < \delta,$ ent\~ao
	$|f(x)-L|<\varepsilon, |h(x)-L|<\varepsilon.$ Logo,
	$$
		L - \varepsilon < f(x) \leq{g(x)} \leq{h(x)} < L +\varepsilon, \quad \forall x\in D, 0<|x-p|<\delta.
	$$
	Segue que $L-\varepsilon < g(x) < L+\varepsilon,$ para todo x em D que satisfa\c ca a condi\c c\~ao. Em outras palavras,
	$$
		|g(x)-L| < \varepsilon, \quad \forall x\in D, 0<|x-p|<\delta.\text{ \qedsymbol}
	$$
\end{proof*}
\begin{theorem*}
	Seja D subconjunto de $\mathbb{R}, f:D\rightarrow \mathbb{R}$ e p um ponto de acumula\c c\~ao de D. Se $\lim_{x\to p}f(x) = L > 0,$
	ent\~ao existe $\delta > 0$ tal que $f(x) > 0$ para todo x em D com $0<|x-p|<\delta.$
\end{theorem*}
\begin{proof*}
	Dado $\varepsilon = \frac{L}{2}$ tal que
	$$
		-\frac{L}{2} < f(x) - L < \frac{L}{2}
	$$
	para todo x em D, $0 < |x-p|<\delta.$ Logo, $0 <\frac{L}{2}<f(x)$ para todo x em D com $0 <|x-p|<\delta.$ \qedsymbol
\end{proof*}
\begin{theorem*}
	Seja $D\subseteq{\mathbb{R}}, f, g:D\rightarrow \mathbb{R}<$ uma fun\c c\~ao e p um ponto de acumula\c c\~ao de D. Se existe $\delta > 0$
	tal que $f(x)\leq{g(x)}$ para todo x em D com $0 <|x-p|<\delta$ e existe $\lim_{x\to p}f(x) = L_{g}$, ent\~ao $L_{f}\leq{L_{g}}.$
\end{theorem*}
\begin{proof*}
	De fato, dado $\varepsilon > 0$, existe $\delta > 0$ tal que x pertence a D e satisfaz a propriedade, ent\~ao
	$$
		L_{f} - \frac{\varepsilon}{2} \leq{f(x)}\leq{g(x)}\leq{L_{g}+\frac{\varepsilon}{2}.}
	$$
	Portanto, $L_{f}-L_{g}\leq{\varepsilon}$ e, como $\varepsilon > 0$ \'e abritr\'ario, o resultado segue. \qedsymbol
\end{proof*}
\begin{theorem*}
	Seja D subconjunto real, $f:D\rightarrow \mathbb{R}$ e p ponto de acumula\c c\~ao. O limite $\lim_{x\to p}f(x)=L$ se, e s\'o se,
	$\lim_{n\to\infty}f(x_{n})$ existe para toda sequ\^encia $\{x_{n}\}$ em $D/\{p\}$ que converge para p.
\end{theorem*}
\begin{proof*}
	Se $\lim_{x\to p }f(x) = L$ e $\{x_{n}\}$ \'e uma sequ\^encia em $D/\{p\}$ com $x_{n}\overbracket[0pt]{\longrightarrow}^{n\to \infty}p$, dado
	$\varepsilon > 0$, podemos encontrar $\delta > 0$ tal que $|f(x)-L|<\varepsilon$ para quaisquer $x\in D, 0<|x-p|<\delta.$

	Seja N natural tal que $|x_{n}-p|<\delta$ para todo $n\geq{N}.$ Logo, $|f(x_{n})-L| < \varepsilon$ para todo $n\geq{N}.$
	Portanto, $\lim_{n\to\infty}f(x_{n})=L.$

	Por outro lado, note que, se $\lim_{n\to\infty}f(x_{n})$ existe para toda sequ\^encia $\{x_{n}\}$ em $D/\{p\}$ que converge
	para p, todas as sequ\^encias $\{f(x_{n})\}$ t\^em o mesmo limite, visto que se elas n\~ao tivessem, construir\'iamos uma sequ\^encia
	$\{x_{n}'\}$ em $D/\{p\}$ que converge para p, mas $\{f(x_{n}')\}$ n\~ao convergiria.

	Agora, se $\lim_{x\to p}f(x)$ n\~ao \'e L, existe $\varepsilon > 0$ tal que para todo n natural n\~ao nulo, $x_{n}\in D, 0 <|x_{n}-p|<\frac{1}{n}$ tal
	que $|f(x_{n})-L|\geq{\varepsilon}$. Logo, $\lim_{n\to\infty}f(x_{n})$ n\~ao \'e L.\qedsymbol
\end{proof*}
\begin{theorem*}
	Seja $D\subseteq{\mathbb{R}}, f, g:D\rightarrow \mathbb{R}$ fun\c c\~oes e p um ponto de acumula\c c\~ao de D e $\lambda\in \mathbb{R}.$
	\begin{itemize}
		\item[i)] Se existem M e $\delta$ positivos tais que $|f(x)|\leq{M}$ para todo x em D, $0 <|x-p|<\delta$ e $\lim_{x\to p}g(x)=0,$
		      ent\~ao $\lim_{x\to p}(f \cdot g)(x) = 0.$
		\item[ii)] Se $\lim_{x\to p}f(x) = L_{f}$ e $\lim_{x\to p}g(x) = L_{g},$ ent\~ao $\lim_{x\to p}(f + \lambda g)(x) = L_{f} + \lambda L_{g}.$
		      e $\lim_{x\to p}(f \cdot g)(x) = L_{f}L_{g},$. Al\'em disso, se $L_{g}\neq0,$ ent\~ao $\lim_{x\to p}\frac{f}{g}(x) = \frac{L_{f}}{L_{g}}.$
	\end{itemize}
\end{theorem*}
\begin{theorem*}
	Seja $D\subseteq{\mathbb{R}}, f, g:D\rightarrow \mathbb{R}$ fun\c c\~oes e p um ponto de acumula\c c\~ao de D. O limite
	$\lim_{x\to p}f(x)$ existe se, e somente se, f \'e de Cauchy em p, i.e., dado $\varepsilon > 0$, existe $\delta > 0$ tal que
	para x, y em D, $0<|x-p|<\delta$ e $0<|y-p|<\delta$ implica em $|f(x)-f(y)|<\varepsilon.$
\end{theorem*}
\begin{proof*}
	\'E claro que se $\lim_{x\to p}f(x) = L$, ent\~ao f \'e de Cauchy em p. De fato, dado $\varepsilon > 0$, existe $\delta > 0$
	tal que se x \'e um elemento de D para o qual $0<|x-p|<\delta,$ ent\~ao $|f(x)-L|<\frac{\varepsilon}{2}.$ Assim, dados
	x, y em D satisfazendo $0<|x-p|<\delta, 0<|y-p|<\delta,$ ent\~ao
	$$
		|f(x)-f(y)| = |f(x)-L+L-f(y)|\leq{|f(x)-L| + |f(y)-L|}< \frac{\varepsilon}{2}+\frac{\varepsilon}{2} = \varepsilon.
	$$

	Reciprocamente, se f \'e de Cauchy em p e $\{x_{n}\}$ \'e uma sequ\^encia em $D/\{p\}$ que converge para p,
	$\{f(x_{n})\}$ \'e de Cauchy e portanto convergente. Detalhando, dado $\varepsilon > 0$, existe $\delta > 0$ tal que
	se x, y s\~ao elementos de D, $0 < |x-p| <\delta$ e $0 < |x-p| <\delta$ implicar em $|f(x)-f(y)|<\varepsilon$,
	seja $\{x_{n}\}$ uma sequ\^encia em $D/\{p\}, x_{n}\overbracket[0pt]{\longrightarrow}^{n\to \infty}p$. Dado $\varepsilon > 0$
	tome $\delta$ da defini\c c\~ao de ``f \'e de Cauchy em p'' e N tal que $0<|x_{n}-p|<\delta$ para todo $n\geq{N}.$ Ent\~ao,
	$$
		|f(x_{n})-f(x_{m})| < \varepsilon\quad \forall n, m\geq{N}.
	$$
	Portanto, $\{f(x_{n})\}$ converge. \qedsymbol
\end{proof*}
\begin{def*}
	Seja D um subconjunto ilimitado superiormente de $\mathbb{R}$ e $f:D\rightarrow \mathbb{R}$ uma fun\c c\~ao. Diremos que
	o limite de f(x) quando x tende para infinito \'e L em $\mathbb{R}$ se, dado $\varepsilon > 0$, existe $M = M(\varepsilon) > 0$
	tal que
	$$
		x\in D, x > M \Rightarrow |f(x)-L|<\varepsilon.
	$$
	Escreveremos
	$$
		\lim_{x\to\infty}f(x) = L.
	$$
	Analogamente, quando D \'e ilimitado inferiormente, definimos
	$$
		\lim_{x\to-\infty}f(x) = L.\square
	$$
\end{def*}
O limite da sequ\^encia \'e um caso particular de limite infinito, especificamente o caso $D = \mathbb{N}.$
\begin{def*}
	Seja D um subconjunto de $\mathbb{R}$ e $f:D\rightarrow \mathbb{R}$ uma fun\c c\~ao. Se p \'e um ponto de acumula\c c\~ao de D,
	diremos que f diverge para $+\infty$ quando x tende para p se, dado M positivo, existe $\varepsilon = \varepsilon(M) > 0$
	tal que
	$$
		x\in D, 0<|x-p|<\varepsilon \Rightarrow f(x) > M.
	$$
	Escreveremos $\lim_{x\to p}f(x) = +\infty$. Analogamente, define-se $\lim_{x\to p}f(x) =-\infty$.

	Se D \'e ilimitado superiormente(inferiormente), definimos tamb\'em
	$$
		\lim_{x\to+\infty}f(x) = \pm\infty\quad(\lim_{x\to-\infty}f(x)=\pm\infty.). \square
	$$
\end{def*}
\begin{def*}
	Seja D um subconjunto real e $f:D\rightarrow \mathbb{R}$. Se p \'e um ponto de acumula\c c\~ao de D, suponha que existe $\delta_{0} > 0$
	tal que
	$$
		\sup{\{f(x):x\in D, 0<|x-p|<\delta_{0}\}} < \infty.
	$$
	Ent\~ao, existe (ou diverge para $-\infty$) o limite
	$$
		\limsup_{x\to p}f(x)\coloneqq \limsup_{\delta\to0}\{f(x):x\in D, 0<|x-p|<\delta\}.
	$$
	Escrevemos $\limsup_{x\to p}f(x) = +\infty$ quando f n\~ao \'e limitada superiormente em nenhuma vizinhan\c ca de p.

	De maneira an\'aloga, se
	$$
		\inf{\{f(x):x\in D, 0<|x-p|<\delta\}} > -\infty,
	$$
	definimos (podendo ser $+\infty$)
	$$
		\liminf_{x\to p}f(x)\coloneqq \liminf_{\delta\to0}\{f(x):x\in D, 0<|x-p|<\delta\}.
	$$
	Escrevemos $\liminf_{x\to p}f(x) = -\infty$ quando f n\~ao for limitada inferiormente em uma vizinhan\c ca de p. $\square$
\end{def*}
\end{document}

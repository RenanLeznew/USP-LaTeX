\documentclass[analysis_notes.tex]{subfiles}
\begin{document}
\section{Aula 24 - 24/05/2023}
\begin{itemize}
	\item Lipschitz e diferenciabilidade.
\end{itemize}
\subsection{Usos do Recobrimento de Vitali}
\begin{lemma*}
	Se \(f:[a, b]\rightarrow \mathbb{R}\) é monótona, então f é diferenciável exceto
	possivelmente em um conjunto \(E\subseteq{[a, b]}\) com \(m^{*}(E) = 0\).
\end{lemma*}
\begin{proof*}
	Faremos apenas o caso f não-decrescente. Considere
	\begin{align*}
		 & \overline{d^{+}}f(x) = \limsup_{h\to 0^{+}} \frac{f(x+h)-f(x)}{h}\text{ e }\overline{d^{-}}f(x) = \limsup_{h\to 0^{+}}\frac{f(x) - f(x-h)}{h}   \\
		 & \underline{d^{+}}f(x) = \liminf_{h\to 0^{+}} \frac{f(x+h)-f(x)}{h}\text{ e }\underline{d^{-}}f(x) = \liminf_{h\to 0^{+}}\frac{f(x) - f(x-h)}{h} \\
	\end{align*}
	Provemos que o conjunto dos \(x\in[a, b]\) tais que \(\underline{d^{-}}f(x) < \overline{d^{+}}f(x)\)
	ou \(\overline{d^{-}}f(x) > \underline{d^{+}}f(x)\) tem medida exterior nula.
	Vamos apenas considerar o conjunto E dos pontos \(x\in[a, b]\) para os quais
	\(\overline{d^{+}}f(x) > \underline{d^{-}}f(x).\) O conjunto E é a união
	dos conjunto
	\[
		E_{u, v} = \{x: \overline{d^{+}}f(x) > u > v > \underline{d^{-}}f(x) \}
	\]
	para todos os racionais u e v. Logo, basta mostrar que \(m^{*}(E_{u, v}) = 0.\)
	Seja s = \(m^{*}(E_{u,v })\) e, escolhendo \(\varepsilon >0, E_{u, v}\) está contido
	em um aberto O tal que \(m^{*}(O) < s + \varepsilon .\)

	Para cada \(x\in E_{u, v},\) existe um intervalo \([x-h, x]\) contido em O tal que
	\[
		f(x) - f(x-h) < vh.
	\]
	Do \hyperlink{vitali_covering}{Lema de Vitali,} escolhemos uma cole\c cão \(\{I_{1}, \cdots, I_{N}\} \)
	disjunta desses intervalos cujos interiores cobre \(A\subseteq{E_{u, v}}\) com
	\(m^{*}(A) > s-\varepsilon .\) Somando \(f(x) - f(x-h) < vh\) para todos estes intervalos,
	\[
		\sum\limits_{n=1}^{N}[f(x_{n}) - f(x_{n}-h_{n})] < v \sum\limits_{n=1}^{N}h_{n} < vm^{*}(O) < v(s+\varepsilon )
	\]
	Agora, para cada y em A e k arbitrariamente pequeno, \([y, y+k]\subseteq{I_{n}}\) e
	\[
		f(y+k)-f(y) > uk.
	\]
	Novamente, usando o \hyperlink{vitali_covering}{Lema de Vitali,} temos uma cole\c cão
	disjunta \(\{J_{1}, \cdots, J_{M}\}\) desses intervalos cuja união contém um
	subconjunto de A com medida exterior maior que \(s-2\varepsilon .\) Assim,
	adicionando
	\[
		\sum\limits_{n=1}^{N}[f(x_{n}) - f(x_{n}-h_{n})] < v \sum\limits_{n=1}^{N}h_{n} < vm^{*}(O) < v(s+\varepsilon )
	\]
	para todos os invervalos, temos
	\[
		\sum\limits_{i=1}^{M}f(y_{i}-k_{i})-f(y_{i}) > u\sum k_{i} > u (s-2\varepsilon ).
	\]
	Cada intervalo \(J_{i}\) está contido em algum intervalo \(I_{n}\) e, como f é
	crescente, se somarmos para todos os i para os quais \(J_{i}\subseteq{I_{n}},\) temos
	\[
		\sum\limits_{}^{}f(y_{i}+k_{i}) - f(y_{i})\leq f(x_{n}) - f(x_{n}-h_{n}).
	\]
	Logo,
	\[
		\sum\limits_{n=1}^{N}f(x_{n})-f(x_{n}-h_{n})\geq \sum\limits_{i=1}^{M}f(y_{i}+k_{i}) - f(y_{i})
	\]
	e
	\[
		v(s+\varepsilon ) > u(s-2\varepsilon ).
	\]
	Como isso vale para todo \(\varepsilon >0, vs\geq us.\) Além disso, já que
	\(u > v, s=0.\) Portanto,
	\[
		\lim_{h\to 0}\frac{f(x+h)-f(x)}{h}
	\]
	existe exceto possivelmente em um conjunto E com \(m^{*}(E) = 0.\) \qedsymbol
\end{proof*}
\begin{crl*}
	Seja \(I\subseteq{\mathbb{R}}\) um intervalo aberto e \(f:I\rightarrow \mathbb{R}\) Lipschitz contínua em I.
	Então, f é diferenciável exceto possivelmente em um conjunto E com \(m^{*}(E) = 0.\)
\end{crl*}
\begin{crl*}
	Seja \(I\subseteq{\mathbb{R}}\) um intervalo aberto e \(f:I\rightarrow \mathbb{R}\) Lipschitz contínua em I. Então, f é diferenciável em um
	subconjunto denso de I.
\end{crl*}
\begin{theorem*}
	Seja I um intervalo aberto da reta e \(g:I\rightarrow \mathbb{R}\) Lipschitz contínua
	em I. Então, g é continuamente diferenciável se, e somente se, para cada \(x_{0}\in I,\)
	\[
		\biggl|\frac{g(x_{0}+s+h)-g(x_{0}+s)}{h}-\frac{g(x_{0}+h)+g(x_{0})}{h}\biggr|\overbracket[0pt]{\longrightarrow}^{|s|+|h|\to 0}0.
	\]
\end{theorem*}
\begin{proof*}
	Se f é \(C^{1}(I),\) existem \(\theta , \theta '\in(0, 1)\) tais que
	\[
		|\frac{g(x_{0}+s+h)-g(x_{0}+s)}{h}-\frac{g(x_{0}+h)+g(x_{0})}{h}| = |g'(x_{0}+s+\theta h)-g'(x_{0}+\theta 'h)|\overbracket[0pt]{\longrightarrow}^{|s|+|h|\to 0}0.
	\]
	Agora, mostraremos que se \(g:I\rightarrow \mathbb{R}\) é diferenciável, a equa\c cão
	no enunciado implica que \(g:I\rightarrow \mathbb{R}\) é continuamente diferenciável. Segue que
	dado \(\varepsilon >0\), existe \(\delta >0\) tal que, se \(|x-x_{0}| < \delta \) e \(|h| < \delta ,\)
	\[
		|\frac{g(x+h)-g(x)}{h}-\frac{g(x_{0}+h)+g(x_{0})}{h}| < \frac{\varepsilon }{2}.
	\]
	Segue que, para \(|x-x_{0}| < \delta ,\)
	\[
		|g(x)-g(x_{0})| = |\lim_{h\to 0}\biggl\{\frac{g(x+h)-g(x)}{h}-\frac{g(x_{0}+h)-g(x_{0})}{h}\biggr\}|\leq \frac{\varepsilon }{2} < \varepsilon
	\]
	e g' é contínua em \(x_{0}.\) Para concluir a prova, basta mostrar que \(g:I\rightarrow \mathbb{R}\) é diferenciável.

	Como g é Lipschitz contínua, ela é diferenciável em um conjunto denso de pontos.
	Para cada \(x_{0}\) em I, \(\varepsilon >0\), existe \(\delta >0\) tal que
	\[
		|g(x+h)-g(x)-g(x_{0}+h)+g(x_{0})|\leq \frac{\varepsilon }{4}|h|,\quad|x-x_{0}|+|h|<\delta
	\]
	e existe \(x^{*}\in(x_{0}-\delta ,x_{0}+\delta )\) tal que \(g'(x^{*})\) existe.
	Logo, para \(h\neq0\) suficientemente pequeno,
	\begin{align*}
		 & \biggl|\frac{g(x_{0}+h)-g(x_{0})}{h}-g'(x^{*})\biggr|\leq \frac{\varepsilon }{2},                      \\
		 & 0\leq \biggl\{\limsup_{h\to 0}-\liminf_{h\to 0}\biggr\}\frac{g(x_{0}+h)-g(x_{0})}{h}\leq \varepsilon .
	\end{align*}
	Portanto, como \(\varepsilon\) é arbitrário, \(g'(x_{0})\) existe. \qedsymbol
\end{proof*}
\end{document}

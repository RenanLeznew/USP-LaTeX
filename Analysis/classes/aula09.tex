\documentclass[Analysis/analysis_notes.tex]{subfiles}
\begin{document}
\section{Aula 09 - 31/03/2023}
\subsection{Motiva\c c\~oes}
\begin{itemize}
	\item Limites Infinitos;
	\item Sequ\^encias divergentes.
\end{itemize}

\subsection{Pontos de Ader\^encia e Limites Superiores/Inferiores.}
\begin{def*}
	Seja $\{a_{n}\} $ uma sequ\^encia. Um n\'umero real a \'e um valor de ader\^encia de $\{a_{n}\} $ se a sequ\^encia $\{a_{n}\}$ possui
	uma subsequ\^encia convergente para a.$\quad\square$
\end{def*}
\begin{def*}
	Seja $\{a_{n}\} $  uma sequ\^encia limitada. Definimos o limite superior $\displaystyle\limsup_{n\to\infty}a_{n}(\text{ inferior }\liminf_{n\to\infty}a_{n})$ da
	sequ\^encia $\{a_{n}\} $ por
	\begin{align*}
		 & \limsup_{n\to\infty}a_{n} = \lim_{n\to\infty}\sup_{k\geq{n}}a_{k} = \inf_{n\in \mathbb{N}}\sup_{k\geq{n}}a_{k}             \\
		 & \liminf_{n\to\infty}a_{n} = \lim_{n\to\infty}\inf_{k\geq{n}}a_{k} = \sup_{n\in \mathbb{N}}\inf_{k\geq{n}}a_{k}\quad\square
	\end{align*}
\end{def*}
Uma consequ\^encia direta do Teorema do Confronto que utiliza os conceitos acima nos permite dizer se uma sequ\^encia converge apenas
utilizando as ideias de limite superior e inferior:
\begin{theorem*}
	Se $\{a_{n}\} $ \'e uma sequ\^encia limitada, ent\~ao $a = \liminf_{n\to\infty}a_{n}$ e $b = \limsup_{n\to\infty}a_{n}$
	s\~ao valores de ader\^encia de $\{a_{n}\} $
\end{theorem*}
\begin{proof*}
	A prova se baseia em verificar que, dada uma vizinhan\c ca $V_{a}$ de a, temos $a_{n}\in V_{a}$ para infinitos \'indices n.
	Dado $\epsilon > 0$, existe N natural tal que, colocando $a =\displaystyle \liminf_{n\to\infty} = \lim_{n\to\infty}\inf_{k\geq{n}}a_{k} = \lim_{n\to\infty}i_{n},$
	$$
		a - \epsilon < i_{n} < a + \epsilon \quad \forall n \geq{N}.
	$$
	Assim, existe um $a_{\overline{k}}, \overline{k}\geq{N}$ em $(a-\epsilon, a+\epsilon).$ Assim, existe $\overline{n} > \overline{k}$ tal que
	$$
		a - \epsilon < i_{\overline{n}} < a + \epsilon.
	$$
	Repetindo o racioc\'icio, existe $a_{\overline{\overline{k}}}, \overline{\overline{k}} \geq{\overline{n}} > k$ em $(a - \epsilon, a + \epsilon).$
	Dando continuidade a este racioc\'inio ad infinitum, segue que $V_{a}$ cont\'em $a_{n}$ para infinitos ind\'ices n, tal que o teorema est\'a provado. \qedsymbol
\end{proof*}
\begin{crl*}
	(Exerc\'icio) Nas condi\c c\~oes do teorema, a \'e um valor de ader\^encia de $\{a_{n}\}$ e $b = \limsup_{n\to\infty}a_{n}$ tamb\'em \'e um valor de
	ader\^encia de $\{a_{n}\}$
\end{crl*}
\begin{theorem*}
	Se a \'e um valor de ader\^encia da sequ\^encia $\{a_{n}\} $, ent\~ao
	$$
		\liminf_{n\to\infty}a_{n}\leq{a}\leq{\limsup_{n\to\infty}a_{n}}.
	$$
	Al\'em disso, uma sequ\^encia \'e convergente se, e somente se, $\liminf_{n\to\infty}a_{n} = \limsup_{n\to\infty}a_{n}.$
\end{theorem*}
\begin{proof*}
	Defina $i_{n} = \inf_{k\geq{n}}a_{k}$. Segue que $i_{s(n)}\leq{a_{s(n)}}\overbracket[0pt]{\longrightarrow}^{\to }a$, pois o conjunto
	$\{a_{k}: k\geq{s(n)}\}$ cont\'em $a_{s(n)}$. Logo, como $i_{s(n)}$ converge para $\liminf_{n\to\infty}a_{n}$, segue do Teorema da compara\c c\~ao que
	$$
		\liminf_{n\to\infty}a_{n} \leq{\lim_{n\to\infty}a_{s(n)} = a}.
	$$
	Analogamente, como $a_{s(n)}\leq{\sup_{k\geq{s(n)}}{(a_{k})}} = s_{s(n)}$ e $\sup_{k\geq{s(n)}}{(a_{k})}\overbracket[0pt]{\longrightarrow}^{n\to\infty}a_{n}$,
	pelo teorema da compara\c c\~ao, chegamos novamente em
	$$
		\lim_{n\to\infty}a_{s(n)} = a \leq{\limsup_{n\to\infty}a_{n}}.
	$$
	Portanto, juntando ambos, segue o resultado. \qedsymbol
\end{proof*}

\subsection{Sequ\^encias Divergente para $\pm\infty$.}
Recorde-se que
\begin{def*}
	Diremos que uma sequ\^encia $\{a_{n}\}$ diverge para $+\infty(-\infty)$ se, dado $M >0,$ existe um natural N tal que $a_{n}\geq{M}(a_{n}\leq{-M})$
	para todo $n\geq{N}.$ Escreveremos $\lim_{n\to\infty}a_{n} = +\infty(-\infty),$ ou $a_{n}\overbracket[0pt]{\longrightarrow}^{n\to \infty}+\infty(-\infty)\square.$
\end{def*}
\begin{def*}
	Diremos que a sequ\^encia $\{a_{n}\}$ \'e eventualmente positiva (negativa) se existe um natural N tal que $a_{n} > 0 (a_{n} < 0)$
	para todo $n\geq{N}.\square$
\end{def*}
Vejamos a seguir algumas das propriedades dessas sequ\^encias.
\begin{theorem*}
	\begin{itemize}
		\item[a)] Se $a_{n}\overbracket[0pt]{\longrightarrow}^{n\to \infty}\infty$ e $\{b_{n}\}$ \'e limitada inferiormente, ent\~ao
		      $a_{n}+b_{n}\overbracket[0pt]{\longrightarrow}^{n\to \infty}\infty.$
		\item[b)] Se $a_{n}\overbracket[0pt]{\longrightarrow}^{n\to \infty}\infty$ e $\liminf_{n\to\infty}b_{n} > 0,$ ent\~ao
		      $\lim_{n\to\infty}a_{n}b_{m} = +\infty$
		\item[c)] Seja $\{a_{n}\}$ uma sequ\^encia com $a_{n}\neq0$ para todo n natural. $\{a_{n}\}$ \'e eventualmente negativa e
		      $a_{n}\overbracket[0pt]{\longrightarrow}^{n\to \infty}0$ se, e somente se, $\displaystyle \frac{1}{a_{n}}\overbracket[0pt]{\longrightarrow}^{n\to \infty}\infty.$
		\item[d)] Sejam $\{a_{n}\},\{b_{n}\} $ sequ\^encias eventualmente positivas, $b_{n}\neq 0$ para todo n natural.
		\item[d.1)]Se $\liminf_{n\to\infty}a_{n} > 0$ e $b_{n}\overbracket[0pt]{\longrightarrow}^{n\to \infty}0$, ent\~ao $\displaystyle\lim_{n\to\infty}\frac{a_{n}}{b_{n}} = +\infty.$
		\item[d.2)] Se $\{a_{n}\}$ \'e limitada e $b_{n}\overbracket[0pt]{\longrightarrow}^{n\to \infty}\infty$, ent\~ao $\displaystyle \frac{a_{n}}{b_{n}}\overbracket[0pt]{\longrightarrow}^{n\to \infty}0.$
	\end{itemize}
\end{theorem*}
\begin{proof*}
	(a) $\Rightarrow)$ Como $\{b_{n}\}$ \'e limitada inferiormente, existe um n\'umero real $l > 0$ tal que $b_{n} \geq{ -l}$ para todo
	n natural. Como $a_{n}\overbracket[0pt]{\longrightarrow}^{n\to \infty}\infty,$ dado M positivo, existe N natural tal que
	$a_{n}\geq{M}+l$ para todo $n\geq{N}.$ Logo,
	$$
		a_{n} + b_{n}\geq{M+l-l} = M,\quad \forall n\geq{N}.
	$$
	Portanto, $a_{n} + b_{n}\overbracket[0pt]{\longrightarrow}^{n\to \infty}\infty.$

	(b) $\Rightarrow)$ Como $a_{n}\overbracket[0pt]{\longrightarrow}^{n\to \infty}\infty$ e $\liminf_{n\to\infty}b_{n} = r > 0,$
	existe $N_{1}$ natural tal que $\liminf_{k\geq{n}}a_{n} \geq{\frac{r}{2}}$ para todo $n\geq{N_{1}.}$ Como $a_{n}\overbracket[0pt]{\longrightarrow}^{n\to \infty}\infty,$
	dado M positivo, existe $N_{2}$ natural tal que $a_{n} > \frac{2M}{r}$ para todo $n\geq{N_{2}}.$ Disto segue que, para $n\geq{N = \max\{N_{1}, N_{2}\}}$,
	$$
		a_{n}b_{n}\geq{\frac{2M}{r}\frac{r}{2}} = M,\quad \forall n\geq{N}.
	$$
	donde segue o que quer\'iamos.

	(c) $\Rightarrow)$ Se $\{a_{n}\}$ \'e infint\'esima e eventualmente positiva, dado M positivo, seja N natural tal que
	$0 < a_{n} < \frac{1}{M}$ para quaisquer $n\geq{N}.$ Logo, $\frac{1}{a_{n}} > M$ para todos os $n\geq{N},$ mostrando que
	$\frac{1}{a_{n}}\overbracket[0pt]{\longrightarrow}^{n\to \infty}\infty.$

	Reciprocamente, se $\frac{1}{a_{n}}\overbracket[0pt]{\longrightarrow}^{n\to \infty}\infty,$ dado $\epsilon > 0,$ seja N natural tal que
	$\frac{1}{a_{n}} > \frac{1}{\epsilon}$ para todo $n\geq{N}.$ Desta forma, $0 < a_{n} < \epsilon $ para todo $n\geq{N},$ provando o
	resultado.

	(d.1) $\Rightarrow)$ De fato, se $\liminf_{n\to\infty}a_{n} = r > 0,$ existe $N_{1}$ natural tal que $a_{n}\geq{\frac{r}{2}}$ para todo $n\geq{N_{1}.}$
	Dado M positivo, seja $N_{2}$ outro natural tal que $0 < a_{n} < \frac{r}{2M}$ para todo $n\geq{N_{2}}.$ Logo, $\frac{a_{n}}{b_{n}}>\frac{r}{2}\frac{2M}{r} = M$
	para todo $n\geq{N} = \max\{N_{1}, N_{2}\}$, tal que $\frac{a_{n}}{b_{n}}\overbracket[0pt]{\longrightarrow}^{n\to \infty}\infty$.

	(d.2) $\Rightarrow)$ Seja $L > 0 $ tal que $|a_{n}|\leq{L}$ para todo n natural. Como $b_{n}\overbracket[0pt]{\longrightarrow}^{n\to \infty}\infty,$
	dado $\epsilon > 0,$ existe N natural tal que $b_{n} > \frac{L}{\epsilon}(\frac{1}{b_{n}} < \frac{\epsilon}{L})$ para todo $n\geq{N}.$ Logo,
	$$
		\biggl|\frac{a_{n}}{b_{n}} - 0\biggr| < L \cdot \frac{\epsilon}{L} = \epsilon,\quad \forall n\geq{N},
	$$
	mostrando que $\frac{a_{n}}{b_{n}}\overbracket[0pt]{\longrightarrow}^{n\to \infty}0.$\qedsymbol
\end{proof*}
\'E importante notar que, se $a_{n}\overbracket[0pt]{\longrightarrow}^{n\to \infty}\infty$ e $b_{n}\overbracket[0pt]{\longrightarrow}^{n\to \infty}-\infty$, nada
podemos afirmar de $\lim_{n\to\infty}(a_{n}+b_{n}).$ Neste caso, tudo pode ocorrer! $\{a_{n} + b_{n}\} $ pode convergir para qualquer n\'umero real,
divergir para $+\infty, -\infty$ ou pode oscilar. Vamos ilustrar a situa\c c\~ao.
\begin{example}
	Se $a_{n} = \sqrt{n+1}, b_{n} = -\sqrt{n},$ para todo n natural, \'e f\'acil de ver que $a_{n}\overbracket[0pt]{\longrightarrow}^{n\to \infty}\infty$ e $b_{n}\overbracket[0pt]{\longrightarrow}^{n\to \infty}-\infty$.
	Para ver o que ocorre com a sequ\^encia $\{a_{n}+b_{n}\} $, observe que
	$$
		\sqrt{n+1} - \sqrt{n} = \frac{(\sqrt{n+1}-\sqrt{n})(\sqrt{n+1}+\sqrt{n})}{\sqrt{n+1}+\sqrt{n}} = \frac{1}{\sqrt{n+1}+\sqrt{n}}.
	$$
	Segue de (d.2) que $\{a_{n}+b_{n}\} $ \'e infinit\'esima.
\end{example}
\begin{example}
	Se $a > 1,$ ent\~ao a sequ\^encia $\{a_{n}\}$ dada por $a_{n} = \frac{a^{n}}{n}$ diverge para $+\infty.$ De fato, basta ver que
	a = 1 + h, com h positivo, e escrever
	$$
		\frac{a^{n}}{n} = \frac{(1+h)^{n}}{n} = \frac{1}{n} + h + (n-1)\frac{h^{2}}{2!} + s_{n}.
	$$
	O resultado segue aplicando (a).
\end{example}
\begin{example}
	Se $a > 1$, ent\~ao a sequ\^encia $\{a_{n}\}$ com $a_{n} = \frac{n!}{a^{n}}$ diverge para $+\infty.$ Com efeito, basta escolher
	$n_{0}$ tal que $\frac{n_{0}}{a} > 2$ e escrever, para $n\geq{n_{0}}, a_{n} = \frac{n_{0}!}{a^{n_{0}}}\frac{n!}{n_{0}!}\frac{1}{a^{n-n_{0}}}.$
	Se $r = \frac{n_{0}!}{a^{n_{0}}}\frac{n!}{n_{0}!},$ temos
	$$
		a_{n} = r \frac{n(n-1)\cdots(n_{0}+1)}{a^{n-n_{0}}} = r2^{n-n_{0}} + s_{n} = r(n+1 - n_{0}) + \tilde{s}_{n}
	$$
	Novamente, o resultado segue aplicando (a).
\end{example}
Por fim, algumas outras propriedades que n\~ao foram inclusas no teorema s\~ao deixadas como exerc\'icio ao leitor.
\begin{itemize}
	\item[a)] Se $a_{n}\overbracket[0pt]{\longrightarrow}^{n\to \infty}-\infty$ e $\{b_{n}\}$ \'e limitada superiormente, ent\~ao
	      $a_{n}+b_{n}\overbracket[0pt]{\longrightarrow}^{n\to \infty}-\infty.$
	\item[b)] Se $a_{n}\overbracket[0pt]{\longrightarrow}^{n\to \infty}-\infty$ e $\liminf_{n\to\infty}b_{n} > 0,$ ent\~ao $\lim_{n\to\infty}a_{n}b_{m} = -\infty.$
	\item[c)] Seja $\{a_{n}\}$ uma sequ\^encia com $a_{n}\neq 0$ para todo n natural. $\{a_{n}\}$ \'e eventualmente negativa e $a_{n}\overbracket[0pt]{\longrightarrow}^{n\to \infty}0$
	      se, e somente se, $\frac{1}{a_{n}}\overbracket[0pt]{\longrightarrow}^{n\to \infty}-\infty.$
	\item[d)] Sejam $\{a_{n}\},\{b_{n}\}$ duas sequ\^encias eventualmente negativas com $b_{n}\neq0$ para todo n natural.
	\item[d.1)] Se $\liminf_{n\to\infty}a_{n} < 0$ e $b_{n}\overbracket[0pt]{\longrightarrow}^{n\to \infty}0$, ent\~ao $\lim_{n\to\infty}\frac{a_{n}}{b_{n}} = +\infty.$
	\item[d.2)] Se $\{a_{n}\}$ \'e limitada e $b_{n}\overbracket[0pt]{\longrightarrow}^{n\to \infty}-\infty$, ent\~ao $\frac{a_{n}}{b_{n}}\overbracket[0pt]{\longrightarrow}^{n\to \infty}0.$
	\item[e)] No item (d), analise a situa\c c\~ao em que $\{a_{n}\}$ \'e eventualmente positiva e $\{b_{n}\}$ \'e eventualmente negativa.
\end{itemize}
\end{document}

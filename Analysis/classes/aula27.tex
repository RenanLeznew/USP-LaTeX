\documentclass[Analysis/analysis_notes.tex]{subfiles}
\begin{document}
\section{Aula 28 - 07/06/2023}
\subsection{Motivações}
\begin{itemize}
	\item Função Degrau;
	\item Séries como Integrais;
	\item Relação com a Integral de Riemann.
\end{itemize}
\begin{def*}
	A função degrau unitário I é definida por
	\[
		I(x) = \left\{\begin{array}{ll}
			0,\quad x\leq 0 \\
			1,\quad x > 0.
		\end{array}\right.\quad \square
	\]
\end{def*}
\begin{theorem*}
	Se \(a < s < b, f\in \mathcal{B}([a, b], \mathbb{R})\) é contínua em s e \(\alpha (x) = I(x-s)\), então
	\[
		\int_{a}^{b} f d\alpha = f(s)
	\]
\end{theorem*}
\begin{proof*}
	Considere as partições \(\mathcal{P} = \{x_{0}, x_1, x_2, x_3\},\) em que \(x_{0} = a\) e \(x_1 = s < x_2 < x_3 = b\). Então,
	\[
		U(\mathcal{P}, f, \alpha ) = M_{2}, \quad L(\mathcal{P}, f, \alpha ) = m_{2}.
	\]
	Como f é contínua em s, vemos que \(M_3\) e \(m_2\) convergem para f(s) quando \(x_2\to s.\) \qedsymbol
\end{proof*}
\begin{theorem*}
	Se \(c_{n}\geq 0, n = 1, 2, 3, \dotsc \), \(\sum\limits_{n=1}^{\infty}c_{n}\) é convergente, \(\{s_{n}\}\) é uma sequência de pontos distantes em (a, b),
	\[
		\alpha (x) = \sum\limits_{n=1}^{\infty}c_{n}I(x-s_{n})
	\]
	e f é contínua em [a, b], então
	\[
		\int_{a}^{b}f d\alpha  = \sum\limits_{n=1}^{\infty}c_{n}f(s_{n}).
	\]
\end{theorem*}
\begin{proof*}
	Por comparação, a série é convergente para cada x. Sua soma, \(\alpha (x)\), é monótona, \(\alpha (a) = 0\) e \(\beta (b) = \sum\limits_{n=1}^{\infty}c_{n}.\)

	Com isto em mente, dado \(\varepsilon > 0\), escolha \(N\in \mathbb{N}\) tal que \(\sum\limits_{N+1}^{\infty} < \varepsilon \). Faça
	\[
		\alpha _1(x) = \sum\limits_{n=1}^{N}c_{n}I(x-s_{n}),\quad \alpha _2(x) = \sum\limits_{N+1}^{\infty}c_{n}I(x-s_{n}).
	\]
	Dos teoremas anteriores,
	\[
		\int_{a}^{b}fd\alpha_1 = \sum\limits_{i=1}^{N}c_{n}f(s_{n}).
	\]
	Como \(\alpha _2(b) - \alpha _2(a) = \sum\limits_{N+1}^{\infty}c_{n} < \varepsilon \), segue que
	\[
		\biggl\vert \int_{a}^{b}f d\alpha_2 \biggr\vert \leq m\varepsilon ,
	\]
	sendo \(M = \sup{|f(x)|}\). Como \(\alpha = \alpha _1 + \alpha _2, \) segue que
	\[
		\biggl\vert \int_{a}^{b}fd\alpha  - \sum\limits_{i=1}^{N}c_{n}f(s_{n}) \biggr\vert\leq M\varepsilon .
	\]
	Se fazemos \(N\to \infty,\) portanto, segue o resultado que
	\[
		\int_{a}^{b}f d\alpha = \sum\limits_{n=0}^{\infty}c_{n}f(s_{n}) = f(s).\quad \text{\qedsymbol}
	\]
\end{proof*}
\begin{theorem*}
	Sejam \(\alpha :[a, b]\rightarrow \mathbb{R}\) não-decrescente e diferenciável com \(\alpha '\in \mathfrak{R}([a, b])\) e \(f\in \mathcal{B}([a, b], \mathbb{R})\). Então, \(f\in \mathfrak{R}(\alpha , [a, b])\) se, e somente se,
	\(f\alpha '\in \mathfrak{R}([a, b])\). Neste caso,
	\[
		\int_{a}^{b}fd\alpha = \int_{a}^{b}f(x)\alpha'(x) dx.
	\]
\end{theorem*}
\begin{proof*}
	Dado \(\varepsilon > 0\), existe \(\mathcal{P} = \{x_{0},\dotsc ,x_{n}\}\in \mathfrak{P}([a, b])\) tal que
	\[
		U(\mathcal{P}, \alpha ') - L(\mathcal{P}, \alpha') < \varepsilon .
	\]
	Do \hypertarget{mean_value}{Teorema do Valor Médio} existe \(t_{i}\in [x_{i-1}, x_{i}]\) tal que
	\[
		\Delta \alpha_{i} = \alpha'(t_{i})\Delta x_{i},\quad 1\leq i\leq n.
	\]
	Se \(s_{i}\in [x_{i-1}, x_{i}]\), então
	\[
		\sum\limits_{i=1}^{n}|\alpha '(s_{i}) - \alpha '(t_{i})|\Delta x_{i} < \varepsilon .
	\]
	Seja \(M = \sup|f(x)|\). Como
	\[
		\sum\limits_{i=1}^{n}f(s_{i})\Delta \alpha_{i} = \sum\limits_{i=1}^{n}f(s_{i})\alpha'(t_{i})\Delta x_{i},
	\]
	segue que
	\[
		\biggl\vert \sum\limits_{i=1}^{n}f(s_{i})\Delta \alpha_{i} - \sum\limits_{i=1}^{n}f(s_{i})\alpha'(s_{i})\Delta x_{i} \biggr\vert \leq M\varepsilon .
	\]
	Em particular,
	\[
		\sum\limits_{i=1}^{n}f(s_{i})\Delta \alpha_{i} \leq U(\mathcal{P}, f\alpha') + M\varepsilon
	\]
	para todas as escolhas \(s_{i}\in [x_{i-1}, x_{i}],\) de modo que
	\[
		U(\mathcal{P}, f, \alpha )\leq U(\mathcal{P}, fa') + M\varepsilon .
	\]
	O mesmo argumento de antes nos leva a
	\[
		U(\mathcal{P}, f\alpha') \leq U(\mathcal{P}, f, \alpha ) + M\varepsilon
	\]
	e, logo,
	\[
		|U(\mathcal{P}, f, \alpha ) - U(\mathcal{P}, f\alpha ')| \leq M\varepsilon .
	\]
	Agora, note que
	\[
		U(\mathcal{P}, f, \alpha ) - L(\mathcal{P}, f, \alpha ) < \varepsilon
	\]
	permanece válida se \(\mathcal{P}\) for substituída por um refinamento, ou seja, as relações derivadas disso
	também continuam válidas. Concluímos, então, que
	\[
		\biggl\vert \overline{\int_{a}^{b}}f d\alpha - \overline{\int_{a}^{b}}f(x)\alpha '(x)dx \biggr\vert\leq M\varepsilon .
	\]
	Como \(\varepsilon \) é arbitrário,
	\[
		\overline{\int_{a}^{b}}fd\alpha = \overline{\int_{a}^{b}}f(x)\alpha '(x)dx
	\]
	para qualquer \(f\in \mathcal{B}([a, b], \mathbb{R}).\) A igualdade para as integrais inferiores segue da mesma maneira. Portanto,
	\[
		\int_{a}^{b}fd\alpha = \int_{a}^{b}f(x)\alpha '(x)dx.\quad \text{\qedsymbol}
	\]
\end{proof*}
\end{document}

\documentclass[Analysis/analysis_notes.tex]{subfiles}
\begin{document}
\section{Aula 23 - 22/05/2023}
\subsection{O que esperar?}
\begin{itemize}
	\item Continuando medida exterior;
	\item Relacionando medida e derivada;
\end{itemize}
\subsection{Recobrimento de Vitali}
Dado um conjunto E, uma cobertura de Vitali dele é uma cole\c cão de intervalos
\(\mathcal{I}\) tal que para todo x em E e \(\varepsilon  > 0\), existe \(I\in \mathcal{I}\) tal que
\(x\in I\) e \(m^{*}(I) < \varepsilon .\)
\begin{lemma*}
	Seja \(E\subseteq{[a, b]}\), tal que \(m^{*}(E)\leq b-a.\) Se \(\mathcal{I}\)
	é uma cobertura de E por intervalos não degenerados (Não é um ponto só) e tal que, dados \(x\in E\) e
	\(\varepsilon >0\), existe \(I\in \mathcal{I}\) tal que \(x\in I\) e \(l(I) < \varepsilon .\)
	Então, dado \(\varepsilon  > 0\), exsite uma cole\c cão finita e disjunta \(\{I_{1}, \cdots, I_{N}\}\subseteq{\mathcal{I}}\)
	tal que
	\[
		m^{*}\biggl(E\biggl\backslash\biggr. \bigcup_{n=1}^{N}{I_{n}}\biggr) < \varepsilon .
	\]
\end{lemma*}
\begin{proof*}
	Basta considerar o caso com cada intervalo de \(\mathcal{I}\) fechado. Podemos assumir que
	\(\mathcal{I}\ni I \subseteq{\mathcal{O}=(b-1, b+1)}\) e que \(I\cap E \neq\emptyset\) para todo I
	de \(\mathcal{I}\).

	Escolhemos uma sequência \(\{I_{n}\}\) de intervalos disjuntos de \(\mathcal{I}\) da
	seguinte forma: Seja \(I_{1}\in \mathcal{I}\) qualquer e se \(I_{1}, \cdots, I_{n}\) já foram
	escolhidos, seja \(r_{n}\) o supremo dos comprimentos dos intervalos de \(\mathcal{I}\) que não interseptam
	nenhum dos \(I_{1}, \cdots, I_{n}\). Claramente, \(r_{n} <  l(\mathcal{O}).\) Se
	\(E\not\subseteq{\bigcup_{i=1}^{n}{I_{i}}},\) encontramos \(I_{n+1}\in \mathcal{I}\)
	disjunto de \(I_{1}, \cdots, I_{n}\) e tal que \(l(I_{n+1}) > \frac{1}{2}r_{n}\).

	Assim, \(\{I_{n}\}\) é uma sequência disjunta de intervalos em \(\mathcal{I}\) e, como
	\(\bigcup_{}^{}{I_{n}\subseteq{\mathcal{O}}}, \sum\limits_{}^{}l(I_{n})\leq l(\mathcal{O}).\) Logo,
	existe \(N\in \mathbb{N}\) tal que
	\[
		\sum\limits_{N+1}^{\infty} l(I_{n}) < \frac{\varepsilon }{5}.
	\]
	Seja
	\[
		R = E\biggl\backslash\bigcup_{n=1}^{N}{I_{n}}\biggr.
	\]
	Mostraremos que \(m^{*}(R) < \varepsilon .\) Se \(x\in R,\) como \(F = \bigcup_{n=1}^{N}{I_{n}}\) é
	fechado e \(x\not\in F,\) existe \(I\) em \(\mathcal{I}, x\in I\) e \(I\cap F=\emptyset.\)

	Agora, se \(I\cap I_{i} = \emptyset\) para \(i\leq \kappa \), temos \(l(I)\leq r_{\kappa } < 2l(I_{\kappa +1}.\)
	Como \(\lim_{\kappa \to \infty}l(I_{\kappa }) = 0,\) o intervalo I deve intersectar
	pelo menos um dos intervalos \(I_{\kappa }.\)

	Seja n o menor inteiro tal que \(I\cap I_{n} \neq\emptyset\). Claramente \(n > N\), e
	\(l(I)\leq r_{n-1} < 2l(I_{n}).\) Como \(x\in I\) e \(I\cap I_{n} \neq\emptyset\), a distância de
	x ao ponto médio de \(I_{n}\) é no máximo \(l(I) + \frac{1}{2}l(I_{n}) < \frac{5}{2} l(I_{n}).\)

	Logo, x pertence ao intervalo \(J_{n}\) tendo o mesmo ponto médio que \(I_{n}\) e
	o quíntuplo do comprimento. Desta forma,
	\[
		R\subseteq{\bigcup_{N+1}^{\infty}{J_{n}}}
	\]
	e, portanto,
	\[
		m^{*}(R)\leq \sum\limits_{n=N+1}^{\infty}l(J_{n}) = 5\sum\limits_{n=N+1}^{\infty}l(I_{n}) < \frac{5\varepsilon }{5} = \varepsilon .\text{ \qedsymbol}
	\]
\end{proof*}
\begin{lemma*}
	Se \(f:[a, b]\rightarrow \mathbb{R}\) é monótona, então f é diferenciável exceto
	possivelmente em um conjunto \(E\subseteq{[a, b]}\) com \(m^{*}(E) = 0\).
\end{lemma*}
\begin{proof*}
	Faremos apenas o caso f não-decrescente. Considere
	\begin{align*}
		 & \overline{d^{+}}f(x) = \limsup_{h\to 0^{+}} \frac{f(x+h)-f(x)}{h}\text{ e }\overline{d^{-}}f(x) = \limsup_{h\to 0^{+}}\frac{f(x) - f(x-h)}{h}   \\
		 & \underline{d^{+}}f(x) = \liminf_{h\to 0^{+}} \frac{f(x+h)-f(x)}{h}\text{ e }\underline{d^{-}}f(x) = \liminf_{h\to 0^{+}}\frac{f(x) - f(x-h)}{h} \\
	\end{align*}
	Provemos que o conjunto dos \(x\in[a, b]\) tais que \(\underline{d^{-}}f(x) < \overline{d^{+}}f(x)\)
	ou \(\overline{d^{-}}f(x) > \underline{d^{+}}f(x)\) tem medida exterior nula.
	Vamos apenas considerar o conjunto E dos pontos \(x\in[a, b]\) para os quais
	\(\overline{d^{+}}f(x) > \underline{d^{-}}f(x).\) O conjunto E é a união
	dos conjunto
	\[
		E_{u, v} = \{x: \overline{d^{+}}f(x) > u > v > \underline{d^{-}}f(x) \}
	\]
	para todos os racionais u e v. Logo, basta mostrar que \(m^{*}(E_{u, v}) = 0.\)
	Seja s = \(m^{*}(E_{u,v })\) e, escolhendo \(\varepsilon >0, E_{u, v}\) está contido
	em um aberto O tal que \(m^{*}(O) < s + \varepsilon .\)

	Para cada \(x\in E_{u, v},\) existe um intervalo \([x-h, x]\) contido em O tal que
	\[
		f(x) - f(x-h) < vh.
	\]
	Do \hyperlink{vitali_covering}{Lema de Vitali,} escolhemos uma cole\c cão \(\{I_{1}, \cdots, I_{N}\} \)
	disjunta desses intervalos cujos interiores cobre \(A\subseteq{E_{u, v}}\) com
	\(m^{*}(A) > s-\varepsilon .\) Somando \(f(x) - f(x-h) < vh\) para todos estes intervalos,
	\[
		\sum\limits_{n=1}^{N}[f(x_{n}) - f(x_{n}-h_{n})] < v \sum\limits_{n=1}^{N}h_{n} < vm^{*}(O) < v(s+\varepsilon )
	\]
	Agora, para cada y em A e k arbitrariamente pequeno, \([y, y+k]\subseteq{I_{n}}\) e
	\[
		f(y+k)-f(y) > uk.
	\]
	Novamente, usando o \hyperlink{vitali_covering}{Lema de Vitali,} temos uma cole\c cão
	disjunta \(\{J_{1}, \cdots, J_{M}\}\) desses intervalos cuja união contém um
	subconjunto de A com medida exterior maior que \(s-2\varepsilon .\) Assim,
	adicionando
	\[
		\sum\limits_{n=1}^{N}[f(x_{n}) - f(x_{n}-h_{n})] < v \sum\limits_{n=1}^{N}h_{n} < vm^{*}(O) < v(s+\varepsilon )
	\]
	para todos os invervalos, temos
	\[
		\sum\limits_{i=1}^{M}f(y_{i}-k_{i})-f(y_{i}) > u\sum k_{i} > u (s-2\varepsilon ).
	\]
	Cada intervalo \(J_{i}\) está contido em algum intervalo \(I_{n}\) e, como f é
	crescente, se somarmos para todos os i para os quais \(J_{i}\subseteq{I_{n}},\) temos
	\[
		\sum\limits_{}^{}f(y_{i}+k_{i}) - f(y_{i})\leq f(x_{n}) - f(x_{n}-h_{n}).
	\]
	Logo,
	\[
		\sum\limits_{n=1}^{N}f(x_{n})-f(x_{n}-h_{n})\geq \sum\limits_{i=1}^{M}f(y_{i}+k_{i}) - f(y_{i})
	\]
	e
	\[
		v(s+\varepsilon ) > u(s-2\varepsilon ).
	\]
	Como isso vale para todo \(\varepsilon >0, vs\geq us.\) Além disso, já que
	\(u > v, s=0.\) Portanto,
	\[
		\lim_{h\to 0}\frac{f(x+h)-f(x)}{h}
	\]
	existe exceto possivelmente em um conjunto E com \(m^{*}(E) = 0.\) \qedsymbol
\end{proof*}
\begin{crl*}
	Seja \(I\subseteq{\mathbb{R}}\) um intervalo aberto e \(f:I\rightarrow \mathbb{R}\) Lipschitz contínua em I.
	Então, f é diferenciável exceto possivelmente em um conjunto E com \(m^{*}(E) = 0.\)
\end{crl*}
\begin{crl*}
	Seja \(I\subseteq{\mathbb{R}}\) um intervalo aberto e \(f:I\rightarrow \mathbb{R}\) Lipschitz contínua em I. Então, f é diferenciável em um
	subconjunto denso de I.
\end{crl*}
\begin{theorem*}
	Seja I um intervalo aberto da reta e \(g:I\rightarrow \mathbb{R}\) Lipschitz contínua
	em I. Então, g é continuamente diferenciável se, e somente se, para cada \(x_{0}\in I,\)
	\[
		\biggl|\frac{g(x_{0}+s+h)-g(x_{0}+s)}{h}-\frac{g(x_{0}+h)+g(x_{0})}{h}\biggr|\overbracket[0pt]{\longrightarrow}^{|s|+|h|\to 0}0.
	\]
\end{theorem*}
\end{document}

\documentclass[Analysis/analysis_notes.tex]{subfiles}
\begin{document}
\section{Aula 02 - 15/03/2023}
\subsection{Motiva\c c\~oes}
\begin{itemize}
	\item Propriedades b\'asicas dos racionais;
	\item Constru\c c\~ao do corpo dos reais a partir dos racionais;
	\item Cortes de Dedekind.
\end{itemize}
\subsection{Propriedades de $\mathbb{Q}$ e sua Ordem}
Com as 9 propriedades de corpo, conseguimos obter novas regras nos racionais, como a famosa lei do cancelamento:
\begin{prop*}
	Em $\mathbb{Q},$ vale
	$$
		x + z = y + z \Rightarrow x = y
	$$
	e, se $z\neq{0}$,
	$$
		xz = yz \Rightarrow x = y
	$$
\end{prop*}
\begin{proof*}
	\begin{align*}
		 & x = x + 0 = x + (z + (-z)) = (x+z) + (-z) = (y + z) + (-z) = y + (z + (-z)) = y +0 = y                           \\
		 & x = x.1 = x(z.\frac{1}{z}) = (xz)\frac{1}{z} = (yz)\frac{1}{z} = y (z \frac{1}{z}) = y.1 = y. \text{ \qedsymbol}
	\end{align*}
\end{proof*}
\begin{prop*}
	Os elementos neutros da adi\c c\~ao e multiplica\c c\~ao s\~ao \'unicos. Os elementos oposto e inverso tamb\'em o s\~ao.
\end{prop*}
\begin{prop*}
	Para todo x racional, x.0 = 0.
\end{prop*}
\begin{prop*}
	Para todo x racional, -x = (-1)x.
\end{prop*}
A maioria desses resultados acima seguem diretamente da lei do cancelamento. Suas demonstra\c c\~oes ficam como exerc\'icio.
\begin{def*}
	Diremos que
	$$
		\frac{a}{b}\in \mathbb{Q} = \left\{\begin{array}{ll}
			\text{ n\~ao-negativo, } \quad ab\in \mathbb{N} \\
			\text{ positivo, } \quad ab\in \mathbb{N}, a\neq 0
		\end{array}\right.
	$$
	e diremos que
	$$
		\frac{a}{b}\in \mathbb{Q} = \left\{\begin{array}{ll}
			\text{ n\~ao-positivo, } \quad \frac{a}{b} \text{ n\~ao for postivo} \\
			\text{ negativo, } \quad \frac{a}{b} \text{ n\~ao for n\~ao-negativo.}
		\end{array}\right.\square
	$$
\end{def*}
\begin{def*}
	Sejam x, y racionais. Diremos que x \'e menor e que y e escrevemos ``$x < y$'' se existir t racional positivo tal que
	$$
		y = x + t.
	$$
	Neste mesmo caso, podemos dizer que y \'e maior que x, escrevendo ``$x > y$''. Em particular, temos $x > 0$ se x for positivo e
	$x < 0$ se x for negativo.

	Ademais, se $x < y$ ou x = y, escrevemos ``$x \leq{y}$'' se existir racional t n\~ao-negativo tal que
	$$
		y = x + t
	$$
	e, se $x > y$ ou x = y, escrevemos ``$x \geq{y}$'' caso exista racional t n\~ao-positivo com
	$$
		y = x + t. \square
	$$
\end{def*}
A qu\'adrupla ($\mathbb{Q}, +, \cdot, \leq{}$) satisfaz as propriedades de um corpo ordenado, i.e.,
\begin{align*}
	 & (O1) x \leq{x}\forall x\in \mathbb{Q};                                          \\
	 & (O2) x\leq{y} \text{ e } y \leq{x}\Rightarrow x = y \forall x, y\in \mathbb{Q}; \\
	 & (O3) x \leq{y}, y \leq{z}\Rightarrow x \leq{z}\forall x, y, z\in \mathbb{Q};    \\
	 & (O4)\forall x, y \in \mathbb{Q}, x \leq{y} \text{ ou } y \leq{x};               \\
	 & (OA) x \leq{y}\Rightarrow x + z \leq{y + z};                                    \\
	 & (OM) x \leq{y} \text{ e } z \geq{0}\Rightarrow xz \leq{yz}.
\end{align*}
\begin{prop*}
	Para quaisquer x, y, z, w no corpo ordenado dos racionais, valem
	\begin{align*}
		 & i.) x < y\Longleftrightarrow x + z < y + z                       \\
		 & ii.) z > 0\Longleftrightarrow \frac{1}{z} > 0                    \\
		 & iii.) z > 0\Longleftrightarrow -z < 0                            \\
		 & iv.) z > 0\Rightarrow x < y\Longleftrightarrow xz < yz           \\
		 & v.) z < 0\Rightarrow x < y\Longleftrightarrow xz > yz            \\
		 & vi.) xz < yw\Longleftrightarrow \left\{\begin{array}{ll}
			                                          0 \leq{x} < y \\
			                                          0 \leq{z} < w
		                                          \end{array}\right.        \\
		 & vii.) 0 < x < y\Longleftrightarrow 0 < \frac{1}{y} < \frac{1}{x} \\
		 & viii.) x < y \text{ ou } x =y \text{ ou } x > y                  \\
		 & ix.) xy = 0\Longleftrightarrow x = 0\text{ ou }y = 0.            \\
		 & x.) \left.\begin{array}{ll}
			             x \leq{y} \\
			             z \leq{w}
		             \end{array}\right\}\Rightarrow x + z \leq{y + w}       \\
		 & xi.) \left.\begin{array}{ll}
			              0 \leq{x} \leq{y} \\
			              0 \leq{z} \leq{w}
		              \end{array}\right\}\Rightarrow xz \leq{yw}.
	\end{align*}
\end{prop*}
\subsection{Incompletude de $\mathbb{Q}$}
Os n\'umeros racionais podem ser representados por pontos em uma reta horizontal ordenada, chamada reta real. Se P for a representa\c c\~ao
de um n\'umero racional x, diremos que x \'e a abscissa de P. Note que nem todo ponto da reta real \'e racional. Para isso, considere
um quadrado de lado 1 e diagonal d. Pelo Teorema de Pit\'agoras, $d^{2} = 1^2 + 1^2 = 2.$ Agora, seja P a intersec\c c\~ao do eixo
x com a circunfer\^encia de centro em 0 e raio d. Mostremos que P \'e um ponto da reta com abscissa $x\not\in \mathbb{Q}.$
\begin{prop*}
	Seja a um inteiro. Ent\~ao, se a for \'impar, seu quadrado tamb\'em ser\'a. Al\'em disso, se a for par, seu quadrado tamb\'em \'e par.
\end{prop*}
\begin{prop*}
	A equa\c c\~ao $x^2 = 2$ n\~ao admite solu\c c\~ao racional.
\end{prop*}
A ideia da prova \'e escrever um x na forma de fra\c c\~ao e chegar na contradi\c c\~ao de que tanto o numerador quanto o denominador
ser\~ao n\'umeros pares. Com isso, conclui-se que n\~ao existe racional irredut\'ivel com quadrado igual a 2, portanto n\~ao existe racional
satisfazendo a equa\c c\~ao.

Essa discuss\~ao mostra que existem v\~aos na ``reta'' dos racionais, requerindo a ado\c c\~ao de um novo corpo. Essa \'e a principal
motiva\c c\~ao por tr\'as dos n\'umeros reais, "preencher" os buracos deixados pelos racionais.
\begin{prop*}
	(Exerc\'icio.) Sejam $p_{1}, \ldots, p_{n}$ n\'umeros primos distintos. Ent\~ao, a equa\c c\~ao $x^{2} = p_{1}p_2\cdots p_{n}$ n\~ao
	tem solu\c c\~ao racional.
\end{prop*}
Vimos que os n\'umeros racionais com a sua adi\c c\~ao, multiplica\c c\~ao e rela\c c\~ao de ordem \'e um corpo ordenado. Nos interessamos,
tamb\'em, pelo corpo dos reais e dos racionais ($\mathbb{R}, \mathbb{C}$). De forma abstrata, um corpo \'e um conjunto n\~ao-vazio
$\mathbb{F}$ em que est\~ao definidas duas opera\c c\~oes bin\'arias
$$
	+:\mathbb{F}\times \mathbb{F}\rightarrow \mathbb{F}, \quad (x, y)\mapsto x + y
$$
e
$$
	\cdot: \mathbb{F}\times \mathbb{F}\rightarrow \mathbb{F}, \quad (x, y)\mapsto xy
$$
em que valem as oito propriedades vistas previamente para a defini\c c\~ao das opera\c c\~oes em $\mathbb{Q}$
Se, ainda por cima, no corpo $\mathbb{F}$ est\'a definida uma ordem com propriedades an\'alogas \`as vistas para a qu\'adrupla
($\mathbb{Q}, +, \cdot, \leq{}$), diremos que ($\mathbb{F}, +, \cdot, \leq{}$) \'e um corpo ordenado.
\begin{def*}
	Diremos que um subconjunto A de um corpo $\mathbb{F}$ ordenado \'e limitado superiormente se existe um L neste corpo tal que $a \leq{L}$ para todo
	a de A.

	Definimos para um subconjunto limitado superiormente um n\'umero $\sup(A)\in \mathbb{F}$ como o menor limitante superior de
	A, i.e., se $a \leq{\sup(A)}$ para todo a de A e se existe $f\in \mathbb{F}$ com $f < \sup(A),$ ent\~ao existe um a em A com $ f < a.$

	Por fim, diremos que um corpo ordenado \'e completo se todo subconjunto limitado superiormente possui supremo. $\square$
\end{def*}
Nem todo subconjunto limitado superiormente de $\mathbb{Q}$ tem supremo, ou seja, $\mathbb{Q}$ n\~ao \'e completo.

\subsection{Os N\'umeros Reais ($\mathbb{R}$)}
A ideia que iremos usar para construir o conjunto dos reais \'e que o conjunto dos n\'umeros reais preenche toda a reta real. Os Elementos
de $\mathbb{R}$ ser\~ao os subconjuntos de $\mathbb{Q}$ \`a esquerda de um ponto da reta real e ser\~ao chamados de cortes.
\begin{def*}
	Um corte \'e um subconjunto $\alpha\subsetneq \mathbb{Q}$ com as seguintes propriedades:
	\begin{itemize}
		\item[i)] $\alpha\neq \emptyset$ e $\alpha \neq \mathbb{Q};$
		\item[ii)] Se $p\in \alpha$ e q \'e um racional com $q < p$, ent\~ao $q\in \alpha$ (todos os racionais \`a esquerda de um elemento
		      de $\alpha$ est\~ao em $\alpha$);
		\item[iii)] Se $p\in \alpha$, existe um $r\in \alpha$ com $p < r$ ($\alpha$ n\~ao tem um maior elemento). $\square$
	\end{itemize}
\end{def*}
Essa ideia foi proposta inicialmente por Julius Wilhelm Richard Dedekind, um matem\'atico alem\~ao, em 1872, com o objetivo de encontrar uma explica\c c\~ao
e constru\c c\~ao elementar para os n\'umeros reais.
\begin{example}
	Se q \'e um racional, definimos $q^{*} = \{r\in \mathbb{Q}: r < q\}$.Ent\~ao, $q^{*}$ \'e um corte que chamamos de racional. Os
	cortes que n\~ao s\~ao desse tipo se chamam cortes irracionais.
\end{example}
\begin{example}
	$\sqrt{2} = \{q\in \mathbb{Q}: q^{2} < 2\}\cup \{q\in \mathbb{Q}: q < 0\}$ \'e um corte irracional.
\end{example}
Observe que se $\alpha$ \'e um corte, p \'e um ponto dele e q n\~ao \'e, ent\~ao $p < q$. Al\'em disso, se r pertence a $\alpha$
e $r < s,$ ent\~ao s n\~ao pertence ao corte.
\begin{def*}
	Diremos que $\alpha < \beta$, em que $\alpha$ e $\beta$ s\~ao cortes, se $\alpha\subsetneq \beta.\square$
\end{def*}
\begin{prop*}
	Se $\alpha, \beta, \gamma$ s\~ao cortes,
	\begin{itemize}
		\item[i)] $\alpha < \beta$ e $\beta < \gamma$ implica que $\alpha < \gamma$;
		\item[ii)] Exatamente uma das seguintes rela\c c\~oes \'e v\'alida: $\alpha < \beta$ ou $\alpha = \beta$ ou $\beta < \alpha$
		\item[iii)] Todo subconjunto n\~ao-vazio e limitado superiormente de $\mathbb{R}$ tem supremo.
	\end{itemize}
\end{prop*}
\end{document}


\documentclass[analysis_notes.tex]{subfiles}
\begin{document}
\section{Aula 28 - 12/06/2023}
\subsection{Motivação}
\begin{itemize}
	\item Mudança de Variáveis;
	\item Teorema Fundamental do Cálculo;
	\item Integração por Partes.
\end{itemize}
\subsection{Mudança de Variáveis}
Definmos como função degrau unitário a \(I(x)=0, x\leq 0\) e \(I(x)=1, x > 0\), ou seja,
\[
	I(x) = \left\{\begin{array}{ll}
		0, \quad x \leq 0- \\
		1, \quad x > 0
	\end{array}\right.
\]
\begin{theorem*}
	Se \(a < s < b\), \(f\in \mathcal{B}([a, b], \mathbb{R}) \) é contínua em s, e \(\alpha (x)=I(x-s)\), então
	\[
		\int_{a}^{b}f d \alpha = f(s).
	\]
	Se \(c_{n}\geq 0\), \(n=1, 2, 3, \dotsc \), \(\sum\limits_{n=1}^{\infty}c_{n}\) é convergente, \(\{s_{n}\}\) é uma sequência de pontos distintos em (a, b),
	\[
		\alpha (x)= \sum{n=1}{\infty}c_{n}I(x-s_{n})
	\]
	e f é contínua em [a, b], então
	\[
		\int_{a}^{b}f d \alpha = \sum\limits_{n=1}^{\infty}c_{n}f(s_{n})
	\]
\end{theorem*}
\begin{theorem*}
	Sejam \(\alpha : [a, b]\rightarrow \mathbb{R}\) não-decrescente e diferenciável com \(\alpha '\in \mathcal{R}([a, b])\) e \(f\in \mathcal{B}([a, b], \mathbb{R})\), Então \(f\in \mathcal{R}(\alpha , [a, b]) \) se, e só se, \(f \alpha '\in \mathcal{R}([a, b])\).
	Neste caso,
	\[
		\int_{a}^{b}f d \alpha = \int_{a}^{b} f(x)\alpha '(x)dx.
	\]
\end{theorem*}
\begin{proof*}
	Dado \(\varepsilon > 0\), existe \(\mathcal{P}=\{x_{0},\dotsc ,x_{n}\}\in \mathcal{P}([a,b]) \) tal que
	\[
		U(\mathcal{P}, \alpha ')-L(\mathcal{P}, \alpha ')<\varepsilon .
	\]
	Pelo \hyperlink{mean_value}{\textit{Teorema do Valor Médio}}, existe \(t_{i}\in[x_{i-1}, x_{i}]\) tal que
	\[
		\Delta \alpha_{i}=\alpha '(t_{i})\Delta x_{i},\quad 1\leq i\leq n.
	\]
	Se \(s_{i}\in [x_{i-1}, x_{i}]\), então
	\[
		\sum\limits_{i=1}^{n}|\alpha '(s_{i})-\alpha '(t_{i})|\Delta x_{i}<\varepsilon .
	\]
	Coloque \(M=\sup_{}|f(x)|\). Como
	\[
		\sum\limits_{i=1}^{n}f(s_{i})\Delta \alpha_{i}=\sum\limits_{i=1}^{n}f(s_{i})\alpha '(t_{i})\Delta x_{i}
	\]
	segue que
	\[
		\biggl\vert \sum\limits_{i=1}^{n}f(s_{i})\Delta \alpha_{i}-\sum\limits_{i=1}^{n}f(s_{i})\alpha '(s_{i})\Delta x_{i} \biggr\vert\leq M \varepsilon
	\]
	Em particular, para todas as escolhas \(s_{i}\in[x_{i-1},x_{i}]\),
	\[
		\sum\limits_{i=1}^{n}f(s_{i})\Delta \alpha_{i}\leq U(\mathcal{P}, f\alpha ')+M\varepsilon
	\]
	de modo que
	\[
		U(\mathcal{P}, f, \alpha )\leq U(\mathcal{P}, f\alpha ')+M\varepsilon .
	\]
	Por um argumento análogo, chegamos em
	\[
		U(\mathcal{P}, f\alpha ')\leq U(\mathcal{P}, f, \alpha )+M\varepsilon
	\]
	e, logo,
	\[
		-M\varepsilon \leq U(\mathcal{P}, f\alpha ')-U(\mathcal{P}, f, \alpha )\leq M\varepsilon .
	\]
	Dando continuidade ao argumento, perceba que a possibilidade de escolher a partição \(\mathcal{P}\) de forma que a soma superior e inferior de \(\alpha '\) com respeito a essa partição seja arbitrária continua, em particular,
	podendo ser arbitrária mesmo quando \(\mathcal{P}\) é substituída por um refinamento e, consequentemente, a estimativa em módulo para quão distintas são as somas com respeito a \(f\alpha '\) e de f estimada por \(\alpha \) (a última desigualdade com módulo
	obtida logo acima) também permanece válida. Concluímos, juntando tudo e refinando pela melhor partição de somas superiores possível, que
	\[
		\biggl\vert \overline{\int_{a}^{b}}fd\alpha - \overline{\int_{a}^{b}}f(x)\alpha '(x)dx \biggr\vert\leq M\varepsilon .
	\]
	Como \(\varepsilon \) foi escolhido de modo arbitrário,
	\[
		\overline{\int_{a}^{b}}fd\alpha = \overline{\int_{a}^{b}}f(x)\alpha '(x)dx
	\]
	para qualquer \(f\in \mathcal{B}([a, b], \mathbb{R})\). Fazendo o mesmo para as integrais inferiores, portanto,
	\[
		\int_{a}^{b}fd\alpha = \int_{a}^{b}f(x)\alpha '(x)dx.\quad \text{\qedsymbol}
	\]
\end{proof*}
Esses dois teoremas mostram a generalidade e a flexibilidade que o processo de integração de Stieltjes possui - se \(\alpha \) for uma função degrau pura, a integração se reduz a uma série, que pode ser finita ou infinita. Por outro lado, se a \(\alpha \) for uma função com derivada integrável, então
ela entra no caso de uma integral de Riemann. A beleza disso não está na semântica, mas sim em usos. Para isto, consideremos um exemplo da física - o momento de inércia de um fio reto de comprimento 1, em torno de um eixo que passa por uma extremidade, em ângulo reto com o fio, é dado por
\[
	\int_{0}^{1}x^{2}dm(x),
\]
em que \(m(x)\) é a densidade de massa no intervalo \([0, x]\). Se o fio tiver a densidade como uma função contínua do comprimento x, digamos \(\rho \), então \(\frac{dm}{dx}=\rho(x)\). Assim,
\[
	\int_{0}^{1}x^{2}dm(x)=\int_{0}^{1}x^{2}\rho(x)dx.
\]
Por outro lado, se a massa no fio na verdade for composta de varias massas \(m_{i}\) concentradas em lugares \(x_{i}\), então a integral torna-se
\[
	\int_{0}^{1}x^{2}dm(x)=\sum\limits_{i}^{}x_{i}^{2}m_{i}.
\]
Portanto, a integral de Stieltjes descreve todos os casos físicos, variando de acordo com a forma que a massa é distribuída ao longo do fio.
\end{document}

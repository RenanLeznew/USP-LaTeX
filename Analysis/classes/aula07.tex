\documentclass[analysis_notes.tex]{subfiles}
\begin{document}
\section{Aula 07 - 27/03/2023}
\subsection{Motiva\c c\~oes}
\begin{itemize}
	\item Exemplos de Sequ\^encias;
	\item Teorema da Compara\c c\~ao e do Sandu\'iche;
	\item Limites superior e inferior.
\end{itemize}
\subsection{Exemplos de Sequ\^encias}
Revisemos os exemplos da \'ultima aula, com um extra ao final.
\begin{example}
	Mostre que
	\begin{itemize}
		\item[i)] $\{a, a, a, \cdots\},a\in \mathbb{R}$ \'e convergente;
		\item[ii)] $\{0, 1, 0, 1\}$ n\~ao \'e convergente;
		\item[iii)] $\{n\}$ n\~ao \'e convergente.
	\end{itemize}
\end{example}
\begin{example}
	Se a \'e um n\'umero real mais ou igual a zero, ent\~ao a sequ\^encia $\{a^{n}\}$ \'e convergente se $0\leq{a}\leq{1}$ e divergente
	se $a > 1$. Com efeito, se $a > 1, a = 1 + h, h > 0$. Ent\~ao,
	$$
		a^{n} = (1+h)^{n} = \sum\limits_{k=0}^{n}\binom{n}{k}1^{n-k}h^{k} = 1 + nh + \cdots > 1 + nh.
	$$
	Mas, segue da Archimediana que $1 + nh$ sempre forma um conjunto ilimitado para n natural, ou seja, $a_{n}$ \'e ilimitada. Logo, a sequ\^encia
	diverge.

	Por outro lado, suponha que a pertence a (0, 1). Ent\~ao, $a^{n+1} = a a^{n} < a^{n}$, ou seja, \'e uma sequ\^encia decrescente e limitada inferiormente.
	Portanto $\{a_{n}\}$ \'e convergente.
\end{example}
\begin{example}
	Mostre que, se a \'e diferente de 1,
	$$
		\sum\limits_{i=0}^{n}a^{i} = \frac{1-a^{n+1}}{1-a}
	$$
	e que a sequ\^encia $\biggl\{\frac{1-a^{n+1}}{1-a}\biggr\}$ \'e convergente se $0\leq{a}<1$ e divergente se $a > 1$.
\end{example}
\begin{example}
	Mostre que a sequ\^encia $\{a_{n}\}$, com $a_{n} = \displaystyle \sum\limits_{i=0}^{n}\frac{1}{i!}$ \'e convergente para todo n natural. (Crescente e limitada por 3.)

	De fato, \'e claro que $\{a_{n}\} $ \'e crescente e que $\frac{1}{n!}\leq{\frac{1}{2^{n-1}}},$ para $n\geq{2}.$ Logo,
	$$
		a_{n}\leq{1+\sum\limits_{k=0}^{n}\frac{1}{2^{k}}} =
		1 + \frac{1-\frac{1}{2^{n+1}}}{1-\frac{1}{2}} < 3.
	$$
	Portanto, $\{a_{n}\}$ \'e convergente, e denotamos seu limite por e.
\end{example}
\begin{example}
	Mostre que as sequ\^encias $\biggl\{(1+\frac{1}{n}^{n})\biggr\}, \{n^{\frac{1}{n}}\}$ e $\{a^{\frac{1}{n}}\}$ com $a >0,$ s\~ao
	convergentes.
	\begin{align*}
		 & \circ (1+\frac{1}{n})^{n} = 1 + 1 + \frac{1}{2!}(1-\frac{1}{n}) + \cdots + \frac{1}{n!}(1-\frac{1}{n})(1-\frac{1}{n})(1-\frac{2}{n})\cdots(1-\frac{n-1}{n})        \\
		 & \circ n^{\frac{1}{n}} > (n+1)^{\frac{1}{n+1}}\Longleftrightarrow n^{n+1} > (n+1)^{n}\Longleftrightarrow n>(1+\frac{1}{n})^{n}                                      \\
		 & \circ x = a^{n} < 1\Rightarrow x < 1, x^{n} = a, x^{n+1} = a^{\frac{n+1}{n}},\text{ e } y^{n+1} = a \Rightarrow \biggl(\frac{x}{y}\biggr)^{n+1} = a^{\frac{1}{n}}.
	\end{align*}
	Ainda mais, uma delas t\^em como limite o n\'umero e definido no exemplo anterior. Para observar isso, considere o primeiro exemplo. Note que
	\begin{align*}
		b_{n} & = 1 + \binom{n}{1}n^{-1} + \binom{n}{2}n^{-2} + \cdots + \binom{n}{n-1}n^{-n+1} + \binom{n}{n}n^{-n}                              \\
		      & = 1 + 1 + \frac{1}{2!}(1-\frac{1}{n}) + \cdots + \frac{1}{n!}(1-\frac{1}{n})(1-\frac{1}{n})(1-\frac{2}{n})\cdots(1-\frac{n-1}{n}) \\
		      & \leq{1 + 1 + \frac{1}{2!} + \cdots + \frac{1}{n!} = a_{n} < e}
	\end{align*}
	Como cada termo da soma que define $b_{n}$ \'e crescente, obtemos que $b_{n}$ \'e crescente, tal que ela converge com limite $l = \sup{\{b_{n}:n\in \mathbb{N}\}}$.

	Com rela\c c\~ao ao \'ultimo item, resta elaborar como ele converge para 1. Lembre-se que $a^{\frac{1}{n}}$ \'e o \'unico n\'umero real positivo
	x tal que $x^{n} = a.$ Logo, se $x = a^{n} $ e $y = a^{\frac{1}{n+1}}$, temos $x^{n+1} = y^{n+1}x$ e, deste modo,
	\begin{align*}
		 & (a) 0 < a < 1 \Rightarrow x < 1 \text{ e } \biggl(\frac{x}{y}\biggr)^{n+1} = x < 1\text{ e, assim, } x < y. \\
		 & (b) a > 1 \Rightarrow x > 1\text{ e }\biggl(\frac{x}{y}\biggr)^{n+1} = x > 1\text{ tal que } x > y.
	\end{align*}
	Logo, se $a < 1,\{a^{\frac{1}{n}}\} $ \'e crescente e limitada superiormente por 1, mostrando que ela \'e convergente. Al\'em disto,
	se $a > 1,\{a^{\frac{1}{n}}\} $ \'e descrescente e limitada inferiormente por 1, tamb\'em sendo convergente. Por fim, segue de
	$a^{\frac{1}{n(n+1)}} = \frac{a^{\frac{1}{n}}}{a^{\frac{1}{n+1}}}$. Portanto, do item (a) do teorema junto com a regra para quociente de
	sequ\^encias, segue que l = 1 \'e o limite dela.
\end{example}\begin{example}
	Mostre que a sequ\^encia $\{c_{n}\}, c_{0} = 1, c_{n} = n^{\frac{1}{n}}, n\geq{1}$, \'e convergente. Com efeito, lembre-se que,
	para $n\geq{3}, n > b_{n} = (1+\frac{1}{n})^{n}.$ Logo, para $n\geq{3}, n^{n+1}>(n+1)^{n}$ e, consequentemente, $n^{\frac{1}{n}} > (n+1)^{\frac{1}{n+1}}.$\
	Disto segue de $\{n^{\frac{1}{n}}\} $ \'e limitada e, por (h), que $\{c_{n}\} $ \'e convergente com limite $l\geq{1}.$ Ainda mais,
	$(2n)^{\frac{1}{2n}}(2n)^{\frac{1}{2n}} = (2n)^{\frac{1}{n}} = 2^{\frac{1}{n}}n^{\frac{1}{n}}$ e, portanto, usamos (a) e o exemplo da \'ultima aula para
	mostrar que $l^{2} = l = 1.$\qedsymbol
\end{example}

\subsection{Teoremas da Compara\c c\~ao e do Sandu\'iche}
Nota\c c\~ao: Se uma sequ\^encia tem limite 0, ela \'e chamada infinit\'esima.
\begin{theorem*}
	Se $\{a_{n}\}$ \'e limitada e $\{b_{n}\}$ \'e infinit\'esima, ent\~ao $\{a_{n}\cdot b_{n}\} $ \'e infinit\'esima.
\end{theorem*}
\begin{proof*}
	Como $\{a_{n}\}$ \'e limitada, seja $M > 0$ tal que $|a_{n}|\leq{M}$ para todo n natural. Como $\{b_{n}\} $ \'e infinit\'esima, dado $\epsilon > 0$,
	seja N outro natural tal que $|b_{n}|<\frac{\epsilon}{M}$ para todo $n\geq{N}.$ Segue que
	$$
		|a_{n}b_{n}| \leq{M|b_{n}|} < M \frac{\epsilon}{M} = \epsilon,\quad \forall n\geq{N}.
	$$
	Portanto, $\{a_{n}b_{n}\}\overbracket[0pt]{\longrightarrow}^{n\to \infty}0$.\qedsymbol
\end{proof*}
\begin{example}
	Mostre que $\biggl\{\frac{n+\cos{(n)}}{n+1}\biggr\}$ converge.
\end{example}
Os resultados a seguir s\~ao os dois mencionados previamente, o teorema da compara\c c\~ao e o do sandu\'iche, respectivamente.
\begin{theorem*}
	Se $a_{n}\overbracket[0pt]{\longrightarrow}^{n\to \infty}a, b_{n}\overbracket[0pt]{\longrightarrow}^{n\to \infty}b$ e existe N natural tal que,
	para todo $n\geq{N}, a_{n}\leq{b_{n}}$, ent\~ao $a\leq{b}.$
\end{theorem*}
\begin{proof*}
	Dado $\epsilon > 0$, existe $N_{1}\leq{N}$ tal que, para todo $n\geq{N_{1}}$,
	$$
		a - \epsilon < a_{n} < a + \epsilon\quad\text{ e }\quad b-\epsilon < b_{n} < b + \epsilon.
	$$
	Logo, para todo $n\geq{N},$
	$$
		a-\epsilon < a_{n} \leq{b_{n}} < b + \epsilon.
	$$
	Desta forma, $a-b < \epsilon$ para todo $\epsilon > 0$ e, portanto, $a - b\leq{0}.$\qedsymbol
\end{proof*}

\begin{theorem*}
	Se $a_{n}\overbracket[0pt]{\longrightarrow}^{n\to \infty}l, c_{n}\overbracket[0pt]{\longrightarrow}^{n\to \infty}l$ e existe um
	N natural tal que, para todo $n\geq{N}, a_{n}\leq{b_{n}}\leq{c_{n}}$, ent\~ao $b_{n}\overbracket[0pt]{\longrightarrow}^{n\to \infty}l.$
\end{theorem*}
\begin{proof*}
	Dado $\epsilon>0$, existe $N_{1}\geq{N}$ tal que, para todo $n\geq{N_{1}},$
	$$
		l - \epsilon < a_{n} < l + \epsilon\quad\text{ e }\quad l-\epsilon < c_{n} < l+\epsilon.
	$$
	Logo, para todo $n\geq{N_{1}},$
	$$
		l - \epsilon < a_{n}\leq{b_{n}}\leq{c_{n}}<l + \epsilon.
	$$
	Disto segue que $|b_{n} - l|< \epsilon$ para todo $n\geq{N_{1}}$ e que, portanto, $\{b_{n}\} $ \'e convergente para l. \qedsymbol
\end{proof*}
\begin{example}
	Vamos mostrar que
	$$
		e = \lim_{n\to\infty}\overbrace{(1 + \frac{1}{1!} + \frac{1}{2!} + \cdots + \frac{1}{n!})}^{a_{n}} = \lim_{n\to\infty}\overbrace{\biggl(1 + \frac{1}{n}\biggr)^{n}}^{b_{n}} = l.
	$$
	De fato, como $a_{n} \geq{b_{n}}$ para todo n natural, segue do Teorema da Compara\c c\~ao que $e\geq{l}.$ Por outro lado, se
	$n\geq{p}\geq{2},$
	$$
		b_{n} > 1 + 1 + \frac{1}{2!}(1-\frac{1}{n})+\cdots+\frac{1}{p!}(1-\frac{1}{n})(1-\frac{2}{n})\cdots(1-\frac{p-1}{n}).
	$$
	Agora, novamente pelo Teorema da Compara\c c\~ao, $l = \displaystyle \lim_{n\to\infty}b_{n}\geq{a_{p}}$ para todo natural p.
	Portanto, $l = \displaystyle \lim_{n\to\infty}b_{n}\geq{\sup\{a_{n}:n\in \mathbb{N}\}} = \lim_{n\to\infty}a_{n} = e.$ \qedsymbol
\end{example}
\begin{def*}
	Seja $\{a_{n}\} $ uma sequ\^encia. Um n\'umero real a \'e um valor de ader\^encia de $\{a_{n}\} $ se a sequ\^encia $\{a_{n}\}$ possui
	uma subsequ\^encia convergente para a.$\quad\square$
\end{def*}
\begin{def*}
	Seja $\{a_{n}\} $  uma sequ\^encia limitada. Definimos o limite superior $\displaystyle\limsup_{n\to\infty}a_{n}(\text{ inferior}\liminf_{n\to\infty}a_{n})$ da
	sequ\^encia $\{a_{n}\} $ por
	\begin{align*}
		 & \limsup_{n\to\infty}a_{n} = \lim_{n\to\infty}\sup_{k\geq{n}}a_{k}             \\
		 & \liminf_{n\to\infty}a_{n} = \lim_{n\to\infty}\inf_{k\geq{n}}a_{k}\quad\square
	\end{align*}
\end{def*}
Uma consequ\^encia direta do Teorema do Confronto que utiliza os conceitos acima nos permite dizer se uma sequ\^encia converge apenas
utilizando as ideias de limite superior e inferior:
\begin{theorem*}
	Se a \'e um valor de ader\^encia da sequ\^encia $\{a_{n}\} $, ent\~ao
	$$
		\liminf_{n\to\infty}a_{n}\leq{a}\leq{\limsup_{n\to\infty}a_{n}}.
	$$
	Al\'em disso, uma sequ\^encia \'e convergente se, e somente se, $\liminf_{n\to\infty}a_{n} = \limsup_{n\to\infty}a_{n}.$
\end{theorem*}
\end{document}

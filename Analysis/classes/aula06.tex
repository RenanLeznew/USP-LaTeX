\documentclass[Analysis/analysis_notes.tex]{subfiles}
\begin{document}
\section{Aula 06 - 24/03/2023}
\subsection{Motiva\c c\~oes}
\begin{itemize}
	\item Provar o teorema da aula anterior;
	\item Exemplos.
\end{itemize}
\subsection{Propriedades de Sequ\^encias}
Recapitulemos o teorema da aula anterior:
\begin{theorem*}
	\begin{itemize}
		\item[a)]Uma sequ\^encia \'e convergente se, e somente se, toda subsequ\^encia dela converge para o mesmo limite.
		\item[b)] Toda sequ\^encia convergente \'e de Cauchy;
		\item[c)] Toda sequ\^encia limitada tem subsequ\^encia convergente;
		\item[d)] Toda sequ\^encia de Cauchy \'e limitada;
		\item[e)] Toda sequ\^encia de Cauchy que tem subsequ\^encia convergente \'e convergente.
		\item[f)] Toda sequ\^encia de Cauchy \'e convergente;
		\item[g)] Toda sequ\^encia crescente e limitada \'e convergente;
		\item[h)] Toda sequ\^encia decrescente e limitada \'e convergente.
	\end{itemize}
\end{theorem*}
\begin{proof*}
	a.) $\Leftarrow)$ Se toda subsequ\^encia de $\{a_{n}\}$ converge, ent\~ao $\{a_{n}\}$ converge, pois ela \'e uma subsequ\^encia de si mesma (basta tomar $s:\mathbb{N}\rightarrow \mathbb{N}, s(n) = n.$)'

	$\Rightarrow)$ Suponha que $a_{n}\overbracket[0pt]{\longrightarrow}^{n\to \infty}l$ e $\{b_{n}\}$ \'e uma subsequ\^encia de $\{a_{n}\}$, existe
	$s:\mathbb{N}\rightarrow \mathbb{N}$ estritamente crescente tal que $b_{k} = a_{s(k)}.$ Dado $\epsilon > 0$, seja N o natural tal que
	$|a_{n} - l| < \epsilon$ para todo $n\geq{N}.$ Note que $s(n)\geq{n},$ tal que se $n\geq{N},$ ent\~ao $s(n)\geq{N},$ de forma que
	$|a_{s(n)} - l| < \epsilon$. Portanto, qualquer subsequ\^encia de $\{a_{n}\}$ \'e convergente.

	b.) Se $a_{n}\overbracket[0pt]{\longrightarrow}^{n\to \infty}l,$ ent\~ao dado $\epsilon > 0$, existe N natural tal que
	$$
		|a_{n}-l|<\frac{\epsilon}{2},\quad\forall n\geq{N}.
	$$
	Logo, $|a_{n}-a_{m}| = |a_{n}-l + l-a_{m}| \leq{|a_{n}-l| + |l-a_{m}|} < \frac{\epsilon}{2} + \frac{\epsilon}{2} = \epsilon$ para todo $n, m \geq{N}.$

	c.) Suponha que $\{a_{n}\}$ \'e uma sequ\^encia limitada. Recorde que, do teorema de Bolzano-Weierstrass, todo conjunto inifinito e limitado
	possui um ponto de acumula\c c\~ao. Segue que a imagem I da sequ\^encia \'e finita ou infinita.

	No primeiro caso, se I \'e finito, um dos valores pertencentes a I \'e tal que $a_{n} = a$ para infinitos \'indices.
	Construiremos a sequ\^encia como segue - Coloque s(0) como o menor elemento do conjunto dos n's para os quais
	$a_{n} = a, i.e.,\{n\in \mathbb{N}: a_{n} = a\} = A. $ Al\'em disso, tome s(1) como  o menor elemento de A, com  exce\c c\~ao do
	s(0). Repetindo esse processo, obtemos uma subsequ\^encia constante at\'e que se obtenha s(n) = a, ou seja, ela ser\'a convergente.

	Agora, se I \'e infinito, segue de Bolzano-Weierstrass que I tem um ponto de acumula\c c\~ao, nomeie-o de a. Dado $\epsilon > 0, (a-\epsilon, a+\epsilon)$
	tem infinitos elementos do conjunto I. Analogamente ao anterior, coloque N = s(0) como o menor elemento de $\{n\in \mathbb{N}: a_{n}\neq a, a_{n}\in(a-\epsilon, a+\epsilon)\}$ e coloque, tamb\'em,
	$\epsilon_{1} = |a-a_{s(0)}|$. Em seguida, tome $s(1) = \{n\in \mathbb{N}: a_{n}\neq a, a_{n}\in(a-\frac{\epsilon}{2}, a+\frac{\epsilon}{2})\}$. Indutivamente,
	$b = a_{s(n)}$ \'e convergente para a.

	d.) Dado $\epsilon = 1,$ seja N um n\'umero natural tal que
	$$
		|a_{n}-a_{m}| < 1,\quad\forall n\geq{N}.
	$$
	Considere $M = \{|a_{0}|, |a_{1}|, \cdots, |a_{N-1}|, |a_{N}+1|, |a_{N}-1|\}.$ Assim, $a_{n}\in[-M, M]$ para todo n natural.

	e.) Seja $\{a_{n}\}$ de Cauchy e $\{a_{s(n)}\}$ convergente para l. Dado $\epsilon > 0$, existe um natural $N_{1}$ tal que
	$$
		|a_{n}-a_{m}| < \frac{\epsilon}{2},\quad\forall n\geq{N_{1}}.
	$$
	Al\'em disso, existe $N_{2}$ natural tal que
	$$
		|a_{s(n)} - l| < \frac{\epsilon}{2},\quad\forall s(n)\geq{N_{2}}.
	$$
	Seja $N=\max\{s(N_{2}), N_{1}\}$ e tome $n\geq{N}.$
	$$
		|a_{n}-l| = |a_{n} - a_{s(N_{2})} + a_{s(N_{2})} - l| \leq{|a_{n}-a_{s(N_{2})}| + |a_{s(N_{2})} - l|} < \frac{\epsilon}{2} + \frac{\epsilon}{2} = \epsilon.
	$$

	f.) Segue os itens (e), (d) e (c), visto que toda subsequ\^encia de Cauchy ter\'a subsequ\^encia convergente pelos itens (d) e (c).

	g.) Seja $\{a_{n}\}$ limitada e crescente, $l = \sup\{a_{n}:n\in \mathbb{N}\}.$ Ent\~ao, para todo $n \geq{N}$, em que N \'e tal que $a_{N}\in(l-\epsilon, l)$
	$$
		l-\epsilon < a_{N}\leq{a_{n}} \leq{l}.
	$$

	h.) An\'aloga ao g.
\end{proof*}
\begin{example}
	Mostre que
	\begin{itemize}
		\item[i)] $\{a, a, a, \cdots\},a\in \mathbb{R}$ \'e convergente;
		\item[ii)] $\{0, 1, 0, 1\}$ n\~ao \'e convergente;
		\item[iii)] $\{n\}$ n\~ao \'e convergente.
	\end{itemize}
\end{example}
\begin{example}
	Se a \'e um n\'umero real mais ou igual a zero, ent\~ao a sequ\^encia $\{a^{n}\}$ \'e convergente se $0\leq{a}\leq{1}$ e divergente
	se $a > 1$. Com efeito, se $a > 1, a = 1 + h, h > 0$. Ent\~ao,
	$$
		a^{n} = (1+h)^{n} = \sum\limits_{k=0}^{n}\binom{n}{k}1^{n-k}h^{k} = 1 + nh + \cdots > 1 + nh.
	$$
	Mas, segue da Archimediana que $1 + nh$ sempre forma um conjunto ilimitado para n natural, ou seja, $a_{n}$ \'e ilimitada. Logo, a sequ\^encia
	diverge.

	Por outro lado, suponha que a pertence a (0, 1). Ent\~ao, $a^{n+1} = a a^{n} < a^{n}$, ou seja, \'e uma sequ\^encia decrescente e limitada inferiormente.
	Portanto $\{a_{n}\}$ \'e convergente.
\end{example}
\begin{example}
	Mostre que, se a \'e diferente de 1,
	$$
		\sum\limits_{i=0}^{n}a^{i} = \frac{1-a^{n+1}}{1-a}
	$$
	e que a sequ\^encia $\biggl\{\frac{1-a^{n+1}}{1-a}\biggr\}$ \'e convergente se $0\leq{a}<1$ e divergente se $a > 1$.
\end{example}
\begin{example}
	Mostre que a sequ\^encia $\{a_{n}\}$, com $a_{n} = \displaystyle \sum\limits_{i=0}^{n}\frac{1}{i!}$ \'e convergente para todo n natural. (Crescente e limitada por 3.)
\end{example}
\begin{example}
	Mostre que as sequ\^encias $\biggl\{(1+\frac{1}{n}^{n})\biggr\}, \{n^{\frac{1}{n}}\}$ e $\{a^{\frac{1}{n}}\}$ com $a >0,$ s\~ao
	convergentes.
	\begin{align*}
		 & \circ (1+\frac{1}{n})^{n} = 1 + 1 + \frac{1}{2!}(1-\frac{1}{n}) + \cdots + \frac{1}{n!}(1-\frac{1}{n})(1-\frac{1}{n})(1-\frac{2}{n})\cdots(1-\frac{n-1}{n})        \\
		 & \circ n^{\frac{1}{n}} > (n+1)^{\frac{1}{n+1}}\Longleftrightarrow n^{n+1} > (n+1)^{n}\Longleftrightarrow n>(1+\frac{1}{n})^{n}                                      \\
		 & \circ x = a^{n} < 1\Rightarrow x < 1, x^{n} = a, x^{n+1} = a^{\frac{n+1}{n}},\text{ e } y^{n+1} = a \Rightarrow \biggl(\frac{x}{y}\biggr)^{n+1} = a^{\frac{1}{n}}.
	\end{align*}
\end{example}
\end{document}

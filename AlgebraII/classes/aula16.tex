\documentclass[algebraII_notes.tex]{subfiles}
\begin{document}
\section{Aula 16 - 23/10/2023}
\subsection{Motivações}
\begin{itemize}
	\item Domínios Euclideanos;
	\item Normas.
\end{itemize}
\subsection{Domínios Euclidianos}
\begin{def*}
	Um domínio A é dito \textbf{euclidiano} se existe uma função \(N:A\setminus{\{0\}}\rightarrow \mathbb{N}\cup\{0\}\)
	satisfazendo:
	\begin{itemize}
		\item[1)] \(N(a)\leq N(ab)\) para todos \(a, b\in A\setminus{\{0\}}\)
		\item[2)] (Algoritmo da Divisão) Para todos \(a, b\in A, b\neq0,\) existem \(q,r\in A\) tais que
		      \(a = bq + r,\) em que \(N(r) < N(b)\) ou \(r = 0\).
	\end{itemize}
	Neste caso, a função N é  chamada \textbf{norma} de A. \(\square\)
\end{def*}
\begin{example}
	\begin{itemize}
		\item[1)] Para \(A = \mathbb{Z},\) uma possível norma é
		      \begin{align*}
			       & N=|\cdot |:\mathbb{Z}\rightarrow \mathbb{N} \\
			       & n\mapsto |n|
		      \end{align*}
		\item[2)] Se F é um corpo e \(A = F[x]\), temos a norma
		      \begin{align*}
			       & N:F[x]\rightarrow \mathbb{N} \\
			       & p(x)\mapsto \deg{p(x)}.
		      \end{align*}
	\end{itemize}
\end{example}
\begin{theorem*}
	Domínios Euclidianos são D.I.P.'s
\end{theorem*}
\begin{proof*}
	Seja \(\mathfrak{i} \trianglelefteq{A}\) não nulo e seja \(0\neq a\in \mathfrak{i}\) elemento com
	menor norma, i.e., \(N(A) = \min\{N(x): x\in \mathfrak{i}\}\). Mostraremos que \(\mathfrak{i} = \langle a \rangle\).

	Por um lado, é claro que \(\langle a \rangle \subseteq{\mathfrak{i}}.\) Por outro, tome \(x\in \mathfrak{i}\).
	Pela segunda parte da definição, existem \(q, r\in A\) tais que \( x = qa + r\) e \(N(r) < N(a).\)
	Então, \(r=x-qa\in \mathfrak{i}.\) Se \(r\neq0, N(r) < N(a)\), o que é um absurdo pela definição de a como
	elemento com menor norma. Assim, \(r=0\), ou seja, \(x = qa\). Portanto, \(x\in \mathfrak{i}\) e \(\mathfrak{i} = \langle a \rangle\). \qedsymbol
\end{proof*}
\begin{example}
	\begin{itemize}
		\item[1)] Corpos são Domínios Euclidianos:
		      \begin{align*}
			       & N:F^{*}\rightarrow \mathbb{N} \\
			       & a\mapsto 1
		      \end{align*}
		\item[2)] Em \(A = K[x, y],\) o ideal \(\mathfrak{i} = \langle x, y \rangle\) não é principal. Logo, A não é Domínio Euclidiano.
	\end{itemize}
\end{example}
Vale observar que a norma num Domínio Euclidiano não é única, no sentido de que pode ter mais que uma norma. Por exemplo,
\[
	\left\{\begin{array}{ll}
		N_{1}: & \mathbb{Z}\setminus{\{0\}}\rightarrow \mathbb{N} \\
		       & a\mapsto |a|                                     \\
		N_{2}: & \mathbb{Z}\setminus{0}\rightarrow \mathbb{N}     \\
		       & a\mapsto |2a|.
	\end{array}\right.
\]
\begin{prop*}
	Seja \((A, N)\) um Domínio Euclidiano. Então, \(A^{*}=\{a\in A\setminus{\{0\}}: N(a) = N(1)\}.\)
\end{prop*}
\begin{proof*}
	Sejam \(a\in A^{*}\) e \(b\in A^{*}\) tais que \(ab = 1.\) Pela propriedade 1,
	\(N(a)\leq N(ab) = N(1).\) Além disso, vale também que \(N(1)\leq N(a \cdot 1) = N(a).\) Logo,
	\(N(a) = N(1).\) Com isso, provamos que \(A^{*}\subseteq{\{a\in A\setminus{\{0\}}: N(a) = N(1)\}}.\)

	Por outro lado, tome um elemento \(a\in A\setminus{\{0\}}\) tal que \(N(a) = N(1).\) Aplicando o algoritmo da divisão,
	existem \(q, r\in A\) tais que \(1 = aq + r, r = 0\) ou \(N(r) < N(a) = N(1).\)

	Se \(r\neq0, N(1)\leq N(1 \cdot r) = N(r)\) implica que \(N(1)\leq N(r).\) No entanto, por hipótese,
	\(N(r) < N(1)\), ou seja, temos um absurdo. Logo, \(r = 0\) e \(aq = 1\) implica que \(a\in A^{*},\) assim provando
	a outra relação de contenção.

	Portanto, \(A^{*}=\{a\in A\setminus{\{0\}}: N(a) = N(1)\}.\) \qedsymbol
\end{proof*}
\begin{theorem*}
	O anel \(\mathbb{Z}[i] = \{a + ib: a, b\in \mathbb{Z}\}\subseteq{\mathbb{C}}\) é um Domínio Euclidiano com a norma
	\begin{align*}
		 & N:\mathbb{Z}[i]\setminus{\{0\}}\rightarrow \mathbb{N} \\
		 & z = a + ib\mapsto |z|^{2} = a^{2} + b^{2}.
	\end{align*}
\end{theorem*}
\begin{proof*}
	Note que \(N(a+ib) = (a+ib)(a-ib)\), donde fica fácil ver que \(N((a+ib)(c+id)) = N(a+ib)N(c+id)\).
	Observe também que \(N(z) = 0\) se, e somente se, \(z = 0,\) tal que dados \(x, y\in \mathbb{Z}[i], N(xy) = N(x)N(y)
	\geq N(x)\) para todo \(y\neq0\).

	O corpo de frações \(\mathrm{Frac}(\mathbb{Z}[i]) = \mathbb{Q}[i] \subseteq{\mathbb{C}}\) pode receber a mesma norma
	induzida que a de \(\mathbb{Z}[i]\). Assim, sejam \(x, y\in \mathbb{Z}[i], y\neq0.\) Buscamos \(q, r\in \mathbb{Z}[i]\) tais que
	\(x = qy + r,\) em que \(r=0\) ou \(N(r) < N(y)\). Isso é equivalente a
	\begin{align*}
		N(x-qy) < N(y) & \Longleftrightarrow N \biggl(y \biggl(\frac{x}{y}\biggr) - qy\biggr) < N(y) \\
		               & \Longleftrightarrow N(y) N \biggl(\frac{x}{y}-q\biggr) < N(y)               \\
		               & \Longleftrightarrow N \biggl(\frac{x}{y} - q\biggr) < 1.
	\end{align*}
	Com isso, basta acharmos \(q\in \mathbb{Z}[i]\) tal que \(N(x/y - q) < 1.\)

	Dado \(x/y\in \mathbb{Q}[i],\) seja \(x/y = e + if.\) Existem \(g, h\in \mathbb{Z}\)
	tais que \(|e-g|\leq 1/2\) e \(|e-h|\leq 1/2\) através do seguinte processo:

	Se \(e\in \mathbb{Q},\) existe \(n\in \mathbb{Z}\) tal que \(n\leq e < n+1.\) Se \(0 < e-n\leq 1/2\),
	então \(|e-n|\leq 1/2\). Caso contrário, \(1/2 < e-n < 1,\) tal que \(-1/2 < e - n - 1 < 0\) e, assim,
	\(|e-(n+1)| < 1/2\).

	Com isso, se \(q\coloneqq g + ih,\) temos
	\begin{align*}
		N \biggl(\frac{x}{y} - q\biggr) & = N(e + if - g - ih)    \\
		                                & = N(e - q + (f-h)i)     \\
		                                & = (e-q)^{2} + (f-h)^{2} \\
		                                & =\frac{1}{2} < 1.
	\end{align*}
	Portanto, provamos o resultado. \qedsymbol
\end{proof*}
\begin{crl*}
	\(\mathbb{Z}[i]\) é um D.I.P.
\end{crl*}
\begin{example}
	Mostremos que o ideal \(\langle (3, 2 + \sqrt[]{-5}) \rangle \trianglelefteq{\mathbb{Z}[\sqrt[]{-5}]}\) não é principal. Logo, que
	\(\mathbb{Z}[\sqrt[]{-5}]\) não é um domínio euclidiano.

	Com efeito, seja \(\langle 3, 2 + \sqrt[]{-5} \rangle = \langle a + b\sqrt[]{-5} \rangle.\) Observe que
	\[
		N':\mathbb{Z}\sqrt[]{-5}\rightarrow \mathbb{N},\quad x = a+b\sqrt[]{-5}\mapsto a^{2}+5b^{2} = N(x)
	\]
	satisfaz a condição \(N(xy) = N(x)N(y)\geq N(x).\) Dado o elemento \(3\in \langle a + b\sqrt[]{-5}, \rangle\)
	vale que \(3 = (a+b\sqrt[]{-5})(c+d\sqrt[]{-5}),\) do que obtemos que
	\[
		9 = N(3) = N(a+b\sqrt[]{-5})N(c+d\sqrt[]{-5}) = (a^{2}+5b^{2})(c^{5}+5d^{5}).
	\]
	Com isso, \(a^{2}+5b^{2}\mid 9\) dá-nos três possibilidade - \(a^{2} + 5b^{2} = 1, 3,\text{ ou }9.\)
	Se \(a^{2} + 5b^{2} = 1\) ou \(a^{2} + 5b^{2} = 3,\) obtemos \(b = 0\), duas contradições. Fazendo mais contas,
	chegamos em ainda mais contradições, do que segue que a condição \(N(xy)\geq N(x)\) não pode ser satisfeito e, portanto,
	\(\mathbb{Z}[\sqrt[]{-5}]\) não é Domínio Euclidiano e nem \(\langle (3, 2+\sqrt[]{-5}) \rangle\) é um ideal principal.
\end{example}
\begin{example}[Exercício]
	Mostre que \(\mathbb{Z}[\sqrt[]{-2}] = \{a + b\sqrt[]{-2}: a, b\in \mathbb{Z}\}\) é um Domínio
	Euclidiano com a norma \(N(a+b\sqrt[]{-2}) = a^{2} + 2b^{2}.\)
\end{example}
\end{document}

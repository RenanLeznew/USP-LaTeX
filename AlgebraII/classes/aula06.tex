\documentclass[AlgebraII/algebraII_notes.tex]{subfiles}
\begin{document}
\section{Aula 06 - 28/08/2023}
\subsection{Motivações}
\begin{itemize}
	\item Relação Entre Anel e Quociente;
	\item Primeiro Teorema do Isomorfismo;
	\item Segundo Teorema do Isomorfismo.
\end{itemize}
\subsection{Relação entre Anéis e seus Quocientes}
\begin{prop*}
	Seja \(\mathfrak{i}\trianglelefteq{A}.\)
	\begin{itemize}
		\item[1)] \(\mathfrak{i}\) é primo se, e somente se, \(A/\mathfrak{i}\) é domínio;
		\item[2)] \(\mathfrak{i}\) é maximal se, e somente se, \(A/\mathfrak{i}\) é corpo.
	\end{itemize}
\end{prop*}
\begin{proof*}
	1. \(\Rightarrow ):\) Sejam \(a+\mathfrak{i}, b+\mathfrak{i}\in A/\mathfrak{i}\) tais que \((a+\mathfrak{i})(b+\mathfrak{i})=\mathfrak{i}=0_{A/\mathfrak{i}}.\) Assim,
	\(ab+\mathfrak{i} = \mathfrak{i}\) e \(ab\in \mathfrak{i}.\) como \(\mathfrak{i}\) é primo, \(a\in \mathfrak{i}\) ou \(b\in \mathfrak{i}\), o que implica que
	\(a+\mathfrak{i} = \mathfrak{i} = 0_{A/\mathfrak{i}}\) ou \(b+\mathfrak{i} = \mathfrak{i} = 0_{A/\mathfrak{i}}.\) Logo, \(A/\mathfrak{i}\) é domínio.

	\(\Leftarrow ):\) Suponha agora que \(A/\mathfrak{i}\) é domínio. Tome \(ab\in \mathfrak{i}\). Assim, \(ab + \mathfrak{i} = (a+\mathfrak{i})(b+\mathfrak{i}) = \mathfrak{i}.\)
	Como \(A/\mathfrak{i}\) é domínio, \(a+\mathfrak{i} = \mathfrak{i}\) ou \(b+\mathfrak{i} = \mathfrak{i}\), i.e., \(a\in \mathfrak{i}\) ou \(b\in \mathfrak{i}.\) Destarte, \(\mathfrak{i}\) é primo.

	2. \(\Rightarrow ):\) Suponha que \(\mathfrak{i}\) é maximal e seja \(\mathfrak{k}\trianglelefteq{A/\mathfrak{i}}.\) Pela proposição anterior,
	\(\mathfrak{k} = \mathfrak{j}/\mathfrak{i}\) para algum \(\mathfrak{i}\subseteq \mathfrak{j}\trianglelefteq{A}.\) Como \(\mathfrak{i}\) é maximal,
	\(\mathfrak{i}=\mathfrak{j}\) ou \(\mathfrak{j} = A.\) Com isso, \(\mathfrak{k} = \mathfrak{i}/\mathfrak{i} = (\mathfrak{i})\) ou \(\mathfrak{k} = A/\mathfrak{i}.\)
	Visto que \(\mathfrak{k}\) é genérico, \(A/\mathfrak{i}\) é um corpo.

	\(\Leftarrow ):\) Assuma que \(A/\mathfrak{i}\) seja um corpo. Seja \(\mathfrak{j}\trianglelefteq{A}\) com \(\mathfrak{i}\mathfrak{j}\).
	Considere o ideal no quociente \(\mathfrak{k} = \mathfrak{j}/\mathfrak{i}\trianglelefteq{A/\mathfrak{i}}\). Como \(A/\mathfrak{i}\) é um corpo,
	segue que \(\mathfrak{k} = (\mathfrak{i})\) ou \(\mathfrak{k} = A/\mathfrak{i}.\) Assim, \(\mathfrak{j} = \mathfrak{i}\) ou \(\mathfrak{j} = A.\)
	Portanto, \(\mathfrak{i}\) é maximal. \qedsymbol
\end{proof*}
\begin{example}
	\begin{itemize}
		\item[1)] Se p é um primo, então \(\mathbb{Z}/p \mathbb{Z}\) é um corpo, ou seja, \(p \mathbb{Z}\) é um ideal maximal de \(\mathbb{Z}.\)
		\item[2)] Seja \(\mathfrak{i} = \langle 2 \rangle - 2 \mathbb{Z}\subseteq \mathbb{Z}\) e considere o anel
		      \[
			      \mathfrak{i}[x] \trianglelefteq \mathbb{Z}[x],\quad  \mathfrak{i}[x]\coloneqq \{a_{0}+a_{1}x+\dotsc +a_{n}x^{n}\in \mathbb{Z}[x]: a_{i}\in \mathfrak{i}, n\in \mathbb{N}.\}
		      \]
		      Vamos verificar que o mapa
		      \[
			      \varphi : \frac{\mathbb{Z}[x]}{\mathfrak{i}[x]}\rightarrow \biggl(\frac{\mathbb{Z}}{\mathfrak{i}}\biggr)[x] = \mathbb{F}_{2}[x]
		      \]
		      dado por \(\varphi \biggl(\sum\limits_{i=0}^{n}a_{i}x^{i}\biggr) = \sum\limits_{i=0}^{n}\overline{a_{i}}x^{i}\) é um isomorfismo. Como \(\mathbb{F}_{2}[x]\) é um domínio,
		      pelo teorema anterior, \(\mathfrak{i}[x]\) é um ideal primo de \(\mathbb{Z}[x]\).
	\end{itemize}
\end{example}
\begin{prop*}[Exercício]
	Seja \(\mathfrak{p}\) um ideal primo de A. Mostre que
	\[
		\frac{A[x]}{\mathfrak{p}[x]}\cong{\biggl(\frac{A}{\mathfrak{p}}\biggr)}[x]
	\]
	e que \(\mathfrak{p}[x]\) é um ideal primo de A[x].
\end{prop*}
\begin{example}[Exercício]
	Seja p um número primo. Mostre que:
	\begin{itemize}
		\item[1)] \(\mathbb{Z}[1/p]\cong \mathbb{Z}[x]/(\langle px-1 \rangle)\);
		\item[2)] \(\mathbb{Z}[\sqrt[]{p}]\cong \mathbb{Z}[x]/(\langle x^{2}-p \rangle)\);
		\item[3)] Se \(\mathfrak{m}\) é o ideal maximal do anel local \(\mathbb{Z}_{(p)},\) então \(\mathbb{Z}_{(p)}/\mathfrak{m}\cong{\mathbb{Z}/p \mathbb{Z}}.\)
	\end{itemize}
\end{example}
\begin{prop*}[Exercício]
	Seja \(\mathfrak{i}\trianglelefteq{A}\) e \(\mathfrak{i}[x]\subseteq A[x]\) o conjunto de polinômios
	com coeficientes em \(\mathfrak{i}.\) Mostre que:
	\begin{itemize}
		\item[1)] \(\mathfrak{i}[x] \trianglelefteq{A[x]}\);
		\item[2)] \(A[x]/\mathfrak{i}[x]\cong{(A/\mathfrak{i})}[x];\)
		\item[3)] \(\mathfrak{i}\) é primo se, e somente se, \(\mathfrak{i}[x]\) é primo.
	\end{itemize}
\end{prop*}
\begin{prop*}[Exercício]
	Se \(\mathfrak{m}\in \mathrm{Specm}(A),\) mostre que \(A/\mathfrak{m}^{n}\) é um anel local para todo
	\(n \geq 1\), com ideal maximal \(\mathfrak{m}/\mathfrak{m}^{n}.\)
\end{prop*}
\begin{example}[Exercício]
	Mostre que \(\mathbb{R}[x]/(\langle x^{2}+1 \rangle)\cong{\mathbb{C}}, \mathbb{Q}[x]/(\langle x^{2}+1 \rangle)\cong{\mathbb{Q}[i]}\)
	e \(\mathbb{Z}[x]/(\langle x^{2}+1 \rangle)\cong{\mathbb{Z}[i]}.\)
\end{example}
\begin{prop*}[Exercício]
	Seja k um corpo e seja \(\mathfrak{m} = \{x_{1}-a_{1}, \dotsc, x_{n}-a_{n}\}\trianglelefteq{k[x_{1}, \dotsc, x_{n}]}\) para
	\(a_{i}\in k.\) Mostre que \(k[x_{1}, \dotsc, x_{n}]/\mathfrak{m}\cong{k}\), tal que \(\mathfrak{m}\) é maximal.
\end{prop*}
\subsection{Teoremas do Isomorfismo}
Enunciaremos a seguir dois resultados de extrema importância para a álgebra como um todo.
\hypertarget{first_isomorphism}{
	\begin{theorem*}[Primeiro Teorema do Isomorfismo]
		Seja \(f:A\rightarrow B\) um homomorfismo de anéis. O mapa \(\overline{f}:A/\ker{(f)}\rightarrow B\)
		é um morfismo injetor de anéis e \(A/\ker{(f)}\cong{\mathrm{Im}(f)}\)
	\end{theorem*}}
\hypertarget{second_isomorphism}{
	\begin{theorem*}[Segundo Teorema do Isomorfismo]
		Seja \(\mathfrak{i}\trianglelefteq{A}.\) Se \(\mathfrak{j}\subseteq \mathfrak{i}\) e \(\mathfrak{j}/\mathfrak{i}\trianglelefteq{A/\mathfrak{i}},\) então:
		\[
			\frac{A/\mathfrak{i}}{\mathfrak{j}/\mathfrak{i}}\cong{\frac{A}{\mathfrak{j}}}
		\]
	\end{theorem*}}
\begin{proof*}
	\textbf{\underline{Primeiro Teorema}:} Denotaremos por \(\mathfrak{i}\) o kernel \(\ker{(f)}.\)
	A priori, precisamos checar se \(\overline{f}\) está bem definida. De fato, se \(a+\mathfrak{i}=b+\mathfrak{i}, a-b\in \mathfrak{i}\) e, assim
	\(0 = f(a-b) = f(a)-f(b)\), tal que \(f(a) = f(b).\) Assim, \(\overline{f}(a+\mathfrak{i}) = f(a) = f(b) = \overline{f}(b+\mathfrak{i}).\)

	Agora, conferiremos que \(\overline{f}\) é um morfismo. Com efeito, sejam \(a+\mathfrak{i}, b+\mathfrak{i}\),
	\[
		\overline{f}(a+b+\mathfrak{i}) = f(a+b) = f(a)+f(b) = f(a+\mathfrak{i})+f(b+\mathfrak{i}).
	\]
	A mesma conta prova isto para o produto:
	\[
		\overline{f}(ab+\mathfrak{i}) = f(ab) = f(a)f(b) = f(a+\mathfrak{i})f(b+\mathfrak{i}).
	\]

	A posteriori, provemos a injetividade e o isomorfismo. Para o primeiro, observe que, se \(\overline{f}(a+\mathfrak{i}) = 0\), então
	\(f(a) = 0\) e \(a\in\ker{(f)}\). Assim, \(a+\ker{(f)} = \ker{(f)} = 0_{A/\mathfrak{i}}\). Por fim,
	para ver que \(\overline{f}\) é sobrejetora com a imagem (e, portanto, que há o isomorfismo com a imagem),
	seja \(b\in \mathrm{Im}(f).\) Assim, existe \(a\in A\) tal que \(f(a) = b\). Tome \(\overline{f}(a+\mathfrak{i}) = f(a)=b.\)

	Portanto, segue o teorema.

	\textbf{\underline{Segundo Teorema}:} Defina o mapa \(\theta :A/\mathfrak{i}\rightarrow A/\mathfrak{j},\) dado por \(\theta(a+\mathfrak{i}) = a + \mathfrak{j}.\)
	Deste modo, note que \(\theta \) está bem-definido, pois, se \(a+\mathfrak{i} = b+\mathfrak{i}\), então \(a-b\in \mathfrak{i}\). Como
	\(\mathfrak{i}\subseteq \mathfrak{j}, a-b\in \mathfrak{j}.\) Destarte, \(a+\mathfrak{j} = b+\mathfrak{j}\) e \(\theta(a+\mathfrak{i}) = \theta (b+\mathfrak{i}).\)

	A partir disto, faz sentido checarmos se \(\theta \) é epimorfismo injetor da forma desejada no enunciado. A parte do morfismo segue da conta
	\[
		\theta (a+b+\mathfrak{i}) = a + b + \mathfrak{j} = (a+\mathfrak{j}) + (b+\mathfrak{j}).
	\]
	A sobrejetividade vem de qualquer elemento \(a+\mathfrak{j}\in A/\mathfrak{j}\) ser escrito por um correspondente \(a+\mathfrak{i}\in A/\mathfrak{i}\).
	Finalmente, a injetividade para o isomorfismo desejado segue ao provarmos que \(\ker{(\theta )}=\mathfrak{j}/\mathfrak{i}.\) Com efeito,
	\(a+\mathfrak{i}\in\ker{(\theta )}\) equivale a dizer que \(a\in \mathfrak{j}\) e \(a+\mathfrak{i}\in \mathfrak{j}/\mathfrak{i}\).

	Portanto, podemos usar o primeiro teorema do isomorfismo e obtemos o resultado desejado. \qedsymbol
\end{proof*}
\begin{example}
	Seja \(\mathfrak{p}\) um primo e \(\mathbb{Z}_{(p)} = \biggl\{\frac{a}{b}\in \mathbb{Q}:a, b\in \mathbb{Z}, \text{ p não divide b.}\biggr\} \subseteq \mathbb{Q}\).
	Note que \(\mathbb{Z}_{(p)}\) é subanel de \(\mathbb{Q}.\) Além disso, o conjunto
	\[
		\mathfrak{i}_{(p)}=\biggl\{\frac{a}{b}\in \mathbb{Q}: p\mid a, p \text{ não divide b}\biggr\}
	\]
	é um ideal de \(\mathbb{Z}_{(p)}\). Vamos estudar este ideal por meio do Teorema do Isomorfismo. Defina a função
	\begin{align*}
		\varphi : & \mathbb{Z}_{(p)}\rightarrow \mathbb{F}_{p} = \mathbb{Z}/p \mathbb{Z}                  \\
		          & \frac{a}{b}\mapsto \frac{\overline{a}}{\overline{b}} = \overline{a}\overline{b}^{-1}.
	\end{align*}
	Em particular, como p não divide b, \(\mathrm{mdc}(p, b) = 1,\) tal que existem r, s em \(\mathbb{Z}\) que satisfazem
	\[
		rp + sb = 1 \Rightarrow \overline{rp + sb} = \overline{1}
	\]
	Como o anel quociente é \(\mathbb{Z}/p \mathbb{Z}\), isto quer dizer que
	\[
		\overline{r}\overline{0} + \overline{s}\overline{b} = \overline{1} \Rightarrow \overline{s}\overline{b} = \overline{1},
	\]
	tal que \(\overline{b}\) tem inverso multiplicativo no anel quociente. Além disso, \(\varphi \) é epimorfismo. Vamos mostrar que
	\(\ker{(\varphi )} = \mathfrak{i}_{(p)}.\)

	Com efeito, para ver que \(\mathfrak{i}_{(p)}\subseteq \ker{(\varphi )},\) dado \(\frac{a}{b}\in \mathfrak{i}_{(p)},\) segue que \(p\mid a\) e p não divide b, tal que
	\(\overline{a} = \overline{0}\) e \(\overline{b}\neq \overline{0}.\) Com isso,
	\[
		\varphi \biggl(\frac{a}{b}\biggr) = \frac{\overline{a}}{\overline{b}} = \frac{\overline{0}}{\overline{b}} = \overline{0}.
	\]
	Consequentemente, \(\frac{a}{b}\in \ker{(\varphi )}\) e \(\mathfrak{i}_{(p)}\subseteq \ker{(\varphi )}\).

	Por outro lado, suponha que \(\varphi \biggl(\frac{a}{b}\biggr)=\overline{0}\), tal que \(\frac{\overline{a}}{\overline{b}} = \overline{0}\). Em particular, isto significa que
	\(\overline{a} = \overline{0}\) e que, então, \(p\mid a\), ou seja, \(\frac{a}{b}\in \mathfrak{i}_{(p)},\) mostrando que \(\ker{(\varphi )}\subseteq \mathfrak{i}_(p)\).

	Assim, mostramos que \(\ker{(\varphi )} = \mathfrak{i}_{(p)}.\) Agora, pelo Teorema do Isomorfismo,
	\[
		\mathbb{Z}_{(p)}/\mathfrak{i}_{(p)} \cong{\mathbb{Z}_{(p)}/\ker{(\varphi )}} \cong{\mathbb{F}_{p}}.
	\]
	Logo, \(\mathfrak{i}_{(p)}\) é um ideal maximal de \(\mathbb{Z}_{(p)}\).

	Podemos levar isto um pouco mais a fundo e concluiremos que \(\mathbb{Z}_{(p)}\) tem apenas dois ideais primos, sendo eles \((0)\) e \(\mathfrak{i}_{(p)}=\langle \mathfrak{p}/\mathfrak{i} \rangle.\)
	Para a unicidade de \(\mathfrak{i}_{(p)}\) como ideal maximal, seja \(\mathfrak{m}\trianglelefteq \mathbb{Z}_{(p)}\) maximal tal que \(\mathfrak{m}\neq \mathfrak{i}_{(p)}\). Seja
	\(\frac{a}{b}\in \mathfrak{m}\setminus{\mathfrak{i}_{(p)}},\) de forma que p não divide a e nem b, sendo \(\frac{b}{a}\in \mathbb{Z}_{(p)}\). Temos:
	\[
		1 = \frac{b}{a}\cdot \frac{a}{b}\in \mathfrak{m}\Rightarrow \mathfrak{m}=\mathbb{Z}_{(p)}.
	\]
	Contradição. Logo, \(\mathbb{Z}_{(p)}\) não tem outro ideal maximal.

	Fica para o leitor checar os passos finais de que \(\mathfrak{i}_{(p)} = \langle \mathfrak{p}/\mathfrak{i} \rangle.\) Portanto, \(\mathbb{Z}_{(p)}\) tem apenas
	dois ideais primos, sendo eles \((0)\) e \(\mathfrak{i}_{(p)}.\)
\end{example}
\begin{prop*}[Exercício]
	Sejam \(f:A\rightarrow B\) um morfismo de anéis e \(\mathfrak{i}\trianglelefteq{A}.\)
	\begin{itemize}
		\item[1)] Se \(\mathfrak{i}\subseteq \ker{(f)}\), mostre que \(\overline{f}:A/\mathfrak{i}\rightarrow B,\) definido por \(\overline{f}(a+\mathfrak{i}) = f(a)\) é
		      um morfismo bem-definido;
		\item[2)] Se \(\mathfrak{j}\trianglelefteq{B}\), mostre que \(\overline{f}:A/f^{-1}(\mathfrak{j})\rightarrow B/\mathfrak{j},\) definido por
		      \(\overline{f}(a+f^{-1}(\mathfrak{j})) = f(a) + \mathfrak{j}\) é um morfismo bem definido e injetor;
		\item[3)] Se \(\mathfrak{j}\trianglelefteq{B}\) e \(\mathfrak{i}\trianglelefteq{A}\) tal que \(\mathfrak{i}\subseteq f^{-1}(\mathfrak{j}),\) então
		      \(\overline{f}:A/\mathfrak{i}\rightarrow B/\mathfrak{j}, \) definido por \(\overline{f}(a+\mathfrak{i})=f(a)+\mathfrak{j}\),
		      é um morfismo bem definido e \(\ker{(\overline{f})} = f^{-1}(\mathfrak{j})/\mathfrak{i}.\)
	\end{itemize}
\end{prop*}
\subsubsection{EXTRA: O Terceiro e o Quarto Teoremas do Isomorfismo}
Um pouco menos conhecidos, existem dois outros Teoremas do Isomorfismo em Teoria de Anéis.

\begin{theorem*}[Terceiro Teorema do Isomorfismo]
	Seja A um anel e \(\mathfrak{i} \trianglelefteq{A}\) um ideal de A. Então,
	\begin{itemize}
		\item[i)] Se B é um subanel de A, tal que \(\mathfrak{i}\subseteq B\subseteq A,\) então \(B/\mathfrak{i}\) é um
		      subanel de \(A/\mathfrak{i};\)
		\item[ii)] Todo subanel de \(A/\mathfrak{i}\) é da forma \(B/\mathfrak{i}\) para algum subanel B de A tal que
		      \(\mathfrak{i} \subseteq B \subseteq A\);
		\item[iii)] Se \(\mathfrak{j}\) é um ideal de A tal que \(\mathfrak{i}\subseteq \mathfrak{j}\subseteq A,\) então
		      \(\mathfrak{j}/\mathfrak{i}\) é um ideal de \(A/\mathfrak{i}.\)
		\item[iv)] Todo ideal de \(A/\mathfrak{i}\) é da forma \(\mathfrak{j}/\mathfrak{i}\) para algum ideal \(\mathfrak{j}\) tal que
		      \(\mathfrak{i}\subseteq \mathfrak{j}\subseteq R.\)
	\end{itemize}
\end{theorem*}
\begin{theorem*}[Quarto Teorema do Isomorfismo]
	Seja \(\mathfrak{i}\) um ideal de A. A correspondência \(A\longleftrightarrow A/\mathfrak{i}\) é uma bijeção, entre os subconjuntos B de A que contêm \(\mathfrak{i}\) e o
	conjunto de subanéis de \(A/\mathfrak{i}\), que preserva inclusões. Além disso, o subanel B com \(\mathfrak{i} \subseteq B\) é um ideal de A se, e somente se,
	\(B/\mathfrak{i}\) é um ideal de \(A/\mathfrak{i}\).
\end{theorem*}
Em algumas fontes é possível encontrar o segundo teorema do isomorfismo como outro resultado (O apresentado aqui seria parte do terceiro), relacionando o quociente da soma de ideais
com a interseção:
\begin{theorem*}[Segundo Teorema do Isomorfismo Alternativo]
	Seja A um anel, S um subanel de A e \(\mathfrak{i}\) um ideal de A. Então,
	\[
		\frac{(S+\mathfrak{i})}{\mathfrak{i}}\cong{\frac{S}{S\cap \mathfrak{i}}}.
	\]
\end{theorem*}
\end{document}

\documentclass[AlgebraII/algebraII_notes.tex]{subfiles}
\begin{document}
\section{Aula 11 - 20/09/2023}
\subsection{Motivações}
\begin{itemize}
	\item Corolários do Algoritmo da Divisão;
	\item Decomposição de polinômios.
\end{itemize}
\subsection{Corolários do Algoritmo da Divisão}
\begin{prop*}
	Se F é um corpo, então \(F[x]\) é um Domínio de Ideais Principais.
\end{prop*}
\begin{proof*}
	Seja \(\mathfrak{i} \trianglelefteq{F[x]}\) e seja
	\[
		n = \min\{\deg{f(x)}: f(x)\in \mathfrak{i}\setminus{\{0\}}\}.
	\]
	Se \(n=0,\) então existe um polinômio constante \(f(x) = a\neq 0, a\in F.\) Logo,
	\[
		1 = \frac{1}{a}\cdot a = \frac{1}{a}f(x)\in \mathfrak{I}.
	\]
	Assim,
	\[
		I = F[x] = \langle 1 \rangle.
	\]
	Então, podemos assumir que \(n > 0\). Seja \(g(x)\in \mathfrak{i}\) com \(\deg{g} = n\) e observe que
	\(\deg{g}\geq 1\) e \(\ell_{g}\in F^{*}\). Se \(f(x)\in \mathfrak{i},\) pelo algoritmo da divisão, existem
	\(g(x), r(x)\in F[x]\) tais que \(f(x) = q(x)g(x) + r(x),\) com \(\deg{r(x)} < \deg{g(x)}.\) Se \(r(x)\neq0, r(x) = f(x)
	- q(x)g(x)\in \mathfrak{i}\) e, junto com \(\deg{r} < \deg{g},\) obtemos um absurdo, pois g deveria ser o polinômio de
	grau mínimo.

	Logo, \(r(x) = 0\) e, assim, \(f(x) = q(x)g(x)\) para algum polinômio \(q(x)\in F[x]\), tal que \(f(x)\in \langle g(x) \rangle.\)
	Portanto, \(\mathfrak{i} = \langle g(x) \rangle.\) \qedsymbol
\end{proof*}
\begin{example}[Exercício]
	\begin{itemize}
		\item[1)] \(\mathbb{Z}[x]\) não é D.I.P: O ideal gerado \(\mathfrak{i} = \langle 2, x \rangle\) não é principal, mas é primo.
		\item[2)] Mostre que \(\langle 3, x^{2}-2 \rangle \trianglelefteq{\mathbb{Z}[x]}\) é ideal maximal e não é principal. (Dica:
		      mostre que
		      \[
			      \frac{\mathbb{Z}[x]}{\langle 3, x^{2}-2 \rangle}\cong{\frac{\mathbb{Z}_{3}[x]}{\langle x^{2}-2 \rangle}}.
		      \]
	\end{itemize}
\end{example}
\begin{crl*}
	Seja F um corpo e \(f, g, h\in F[x]\) tais que f(x) é mônico e irredutível.
	Se \(f(x)\mid g(x)h(x),\) então \(f(x)\mid g(x)\) ou \(f(x)\mid h(x).\)
\end{crl*}
\begin{proof*}
	Faremos indução no grau de f.

	Se \(\deg{f(x)} = 1,\) como f é mônico, \(f(x) = x-a\) para algum a em F. Por hipótese, existe \(p(x)\) tal que
	\(g(x)h(x) = (x-a)p(x).\) Aplicando em \(x=a,\) segue que \(g(a)h(a) = 0.\) Como f é domínio, é preciso que
	\(g(a) = 0\) ou \(h(a) = 0\). Em outras palavras, \((x-a)\mid g(x)\) ou \((x-a)\mid h(x),\) como desejado.

	Suponha agora que o resultado vale grau menor que n, \(\deg{f} = n\) e \(f\mid g\) e \(f\mid h.\) Assim, existem
	\(q, r, q', r'\in A[x]\) tais que \(g(x) = f(x)q(x) + r(x)\) e \(h(x) = q'(x)f(x) + r'(x),\) satisfazendo
	\(\deg{r'} < \deg{g'}, \deg{r} < \deg{g}\) e \(r, r'\neq 0.\) Assim,
	\begin{align*}
		g(x)h(x)    & = (qf + r)(q'f + r')                \\
		            & = qq'f^{2} + (qr' + q'r)f + rr'     \\
		\Rightarrow & rr' = gh - qq'f^{2} - (qr' + rq')f.
	\end{align*}
	Sabemos que f divide \(gh, qq'f^{2}\) e \((qr' + q'r)f.\) Com isso, \(f\mid rr'\) e, então,
	existe \(p(x)\in F[x]\) tal que \(fp = rr'.\) Note também que \(\deg{p} < \deg{r} \) e \(\deg{r'}.\) De fato,
	caso contrário, teríamos \(\deg{p}\geq \deg{r}\) e como \(\deg{f} > \deg{r'},\) temos
	\[
		\deg{p}\geq \deg{r} \Rightarrow \deg{f} + \deg{p} > \deg{r'} + \deg{r} \Rightarrow \deg{fp} > \deg{rr'},
	\]
	o que é um absurdo. O mesmo vale para \(\deg{p}\geq r'.\) Escrevamos \(p = \alpha p_{1}^{\alpha_{1}}\cdot \dotsc \cdot p_{r}^{\alpha_{r}},\)
	em que \(p_{1}, \cdots, p_{r}\) são irredutíveis (isso é possível pois F é corpo). Assim, \(p_{i}\mid fp = rr'.\)
	Como \(\deg{p} < \deg{r} < \deg{f}\) e \(\deg{p_{i}}\leq \deg{p},\) podemos aplicar a hipótese de indução em \(p_{i}\) e então
	\(p_{i}\mid r\) ou \(p_{i}\mid r'\) para cada \(i=1, \cdots, r\). Suponha, sem perda de generalidade, que \(p_{1}\mid r,\) tal que
	\(r=p_{1}r_{1}\). Consequentemente,
	\[
		fp = rr' \Rightarrow fp_{1}^{\alpha_{1}}\cdot \dotsc \cdot p_{r}^{\alpha_{r}} = p_{1}rr' \Rightarrow fp_{1}^{\alpha_{1} -1}\cdot \cdots \cdot p_{r}^{\alpha_{r}} = r_{1}r.
	\]
	Pode-se continuar esse processo até sobrarem os termos \(p_{i}\)'s, chegando em
	\(f = r_{t}r_{t}'\) com \(\deg{r_{t}}\geq 1\) e \(\deg{r_{t}'}\geq 1\) (pois \(\deg{p} < \deg{r}, \deg{r'}\)). Isso é uma contradição
	com o fato de r ser irredutível.
\end{proof*}
\begin{example}[Exercício]
	Seja A um domínio, \(a\in A\) e \((x-a)\mid f(x)g(x).\) Mostre que \((x-a)\mid f(x)\) e \((x-a)\mid g(x).\)
\end{example}
\begin{prop*}
	Seja F um corpo e \(f(x)\in F[x]\). Então, temos uma decomposição única
	\[
		f(x) = af_{1}^{\alpha_{1}}(x)\cdot \dotsc \cdot f_{r}^{\alpha_{r}}(x),
	\]
	com \(a\in F\) e \(f_{1}, \cdots, f_{r}\in F[x]\) polinômios irredutíveis e mônicos.
\end{prop*}
\begin{proof*}
	Já provamos a existência da decomposição para polinômios mônicos. Como F é corpo, existe \(b\in F\) tal que
	\(bf(x)\) é mônico (basta tomarmos \(b=a^{-1},\) em que a é o coeficiente líder de f). Então, \(bf(x) = f_{1}^{\alpha_{1}}\cdot \dotsc \cdot f_{r}^{\alpha_{r}}\)
	e assim \(f(x) =af_{1}^{\alpha_{1}}\cdot \dotsc \cdot f_{r}^{\alpha_{r}}.\)

	Provemos agora a unicidade. Seja
	\[
		f(x)=af_{1}^{\alpha_{1}} \cdot \dotsc \cdot f_{r}^{\alpha_{r}} = bg_{1}^{\beta_{1}} \cdot \cdots \cdot g_{s}^{\beta_{s}}.
	\]
	Mostraremos que \(\{f_{1}, \cdots, f_{r}\} = \{g_{1}, \cdots, g_{s}\}\) e \(\{\alpha_{1}, \cdots, \alpha_{r}\} = \{\beta_{1}, \cdots, \beta_{s}\}\).
	Se \(\deg{f} = 1, f = a(x-\alpha ) = b(x-\beta )\) e \(ax -a\alpha =bx - b\beta,\) tal que \(a=b\) e \(\alpha =\beta .\)

	Agora, suponha que o resultado vale para grau menor que n e \(\deg{f} = n.\) Então:
	\[
		f_{1}\mid f(x) \Rightarrow f_{1}\mid bg_{1}^{\beta_{1}}\cdot \dotsc g_{s}^{\beta_{s}} \Rightarrow \exists j: f_{1}\mid g_{j}.
	\]
	Como \(g_{j}\) é irredutível e mônico, se \(g_{j}=f_{1}\), existe \(h_{j}\) tal que \(g_{j} = h_{j}f_{1}.\) Se \(\deg{(h_{j})}\geq 1, g_{j}\) não seria
	irredutível e, então, \(h_{j} = a\in F.\) Mas, se \(a\neq1,\) como \(f_{1}\) é mônico, \(g_{j}\) não seria mônico. Logo, \(h_{j} = 1\) e \(g_{j} = f_{1}.\)
	Podemos reordenar tal que \(f_{1} = g_{1}. \) Assim,
	\[
		af_{2}^{\alpha_{2}}\cdot \dotsc \cdot f_{r}^{\alpha_{r}} = bg_{2}^{\beta_{2}} \cdot \dotsc \cdot g_{s}^{\beta_{s}}.
	\]
	Portanto, pela hipótese indutiva, o resultado segue. \qedsymbol
\end{proof*}
\end{document}

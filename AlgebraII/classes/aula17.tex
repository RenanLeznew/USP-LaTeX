\documentclass[algebraII_notes.tex]{subfiles}
\begin{document}
\section{Aula 17 - 25/10/2023}
\subsection{Motivações}
\begin{itemize}
	\item Corpos quadráticos;
	\item O anel dos inteiros de \(\mathbb{Q}[\sqrt[]{d}]\);
	\item Elementos irredutíveis;
	\item Anéis Noetherianos e Teorema da Base de Hilbert.
\end{itemize}
\subsection{Corpos Quadráticos}
\begin{def*}
	Seja \(\alpha \) raiz de um polinômio irredutível e mônico de grau 2, ou seja, \(x^{2}+ax+b\in \mathbb{Z}[x].\) Vamos denotar
	por \(\mathbb{Z}[\alpha ]\) o menor subanel de \(\mathbb{C}\) que contém \(\mathbb{Z}\) e \(\alpha \):
	\[
		\mathbb{Z}[\alpha ]\coloneqq \bigcap_{A\in \mathcal{A}}^{}{A},
	\]
	em que \(\mathcal{A} = \{A\text{ subanel de }\mathbb{C}: \mathbb{Z}\subseteq{A}\text{ e }\alpha \in A\}. \square\)
\end{def*}
\begin{prop*}
	Temos \(\mathbb{Z}[\alpha ] = \{a + b\alpha : a, b\in \mathbb{Z}\}.\)
\end{prop*}
\begin{proof*}
	De fato, coloque \(B\coloneqq \{c+d\alpha : c, d\in \mathbb{Z}\}.\) Note que B é subanel de \(\mathbb{C}\).
	Com isso, se \(a + b\alpha , c + d\alpha \in B,\) então \(a + b\alpha + c +d\alpha = a + c + (b+d)\alpha \in B\)
	e, a multiplicação:
	\[
		(a+b\alpha )(c+d\alpha ) = ac + bd\alpha^{2} + ad\alpha + bc\alpha
	\]
	e note que, já que \(\alpha \) é raiz de um polinômio, \(x^{2} + a'x + b'\), então \(\alpha ^{2} + a'\alpha  + b' = 0\) implica
	que \(\alpha^{2} = -a'\alpha  - b'\), isto é, \(\alpha^{2} = c'\alpha + d'\) para \(c', d'\in \mathbb{Z},\) donde segue que
	\[
		ac + bd(c'+d'\alpha ) + ad\alpha + bc\alpha = ac + bdc' + \alpha(bdd'+ad+bc)\in B.
	\]
	Logo, B é subanel de \(\mathbb{C}.\) Além disso, temos \(\mathbb{Z}\subseteq{B}, \alpha \in B\), tal que
	\(\mathbb{Z}[\alpha ]\subseteq{B}.\)

	Por outro lado, seja \(A\) um subanel de \(\mathbb{C}\) que contém \(\mathbb{Z}\) e \(\alpha .\) Caso \(c+d\alpha \in B, c, d,\in A\)
	e a combinação \(c+d\alpha \in A\). Logo, \(B \subseteq{A}.\) Portanto, \(B\subseteq{\mathbb{Z}[\alpha ]}\). \qedsymbol
\end{proof*}
\begin{example}[Exercícios]
	\begin{itemize}
		\item[1)] Seja \(\alpha \) raiz de \(x^{2} + ax + b\in \mathbb{Z}[x]\) irredutível. Então,
		      \[
			      \mathbb{Z}[\alpha ] \cong{\frac{\mathbb{Z}[x]}{\langle x^{2} + ax + b \rangle}}
		      \]
		\item[2)] Se \(\beta \) é outra raiz do mesmo polinômio, então \(\mathbb{Z}[\alpha ] = \mathbb{Z}[\beta ]\)
	\end{itemize}
\end{example}
Definimos
\begin{align*}
	 & N:\mathbb{Z}[\alpha ]\setminus{\{0\}}\rightarrow \mathbb{N} \\
	 & x+y\alpha \mapsto |x^{2} - axy + y^{2}b|.
\end{align*}
\begin{def*}
	Se \(\alpha \) é a raiz do polinômio mônico de grau 2, o \textbf{conjugado} de \(\theta = c+d\alpha \) é definido como \(\overline{\theta } = c + d\beta \), em que \(\beta \) é a outra raiz. \(\square\)
\end{def*}
Note que um elemento e seu conjugado satisfazem a seguinte relação
\begin{align*}
	\theta \overline{\theta } & = (x+y\alpha )(x+y\beta)                           \\
	                          & = x^{2} + xy\beta  + xy\alpha  + y^{2}\alpha \beta \\
	                          & = x^{2} + (\alpha +\beta )xy + \alpha \beta y^{2}  \\
	                          & = x^{2} + (-a)xy + (b)y^{2}                        \\
	                          & = N(\theta).
\end{align*}
Disto segue que
\begin{itemize}
	\item[1)] \(N(\theta \varphi )=N(\theta )N(\varphi )\);
	\item[2)] \(\mathbb{Zj[\alpha ]^{*} = \{\theta \in \mathbb{Z}[\alpha ]: N(\theta ) = 1\}}\).
\end{itemize}
\begin{def*}
	Seja \(d\in \mathbb{Z}\) livre de quadrados. O corpo \(\mathbb{Q}[\sqrt[]{d}] = \{a + b\sqrt[]{d}: a, b\in \mathbb{Q}\}\) é
	chamado \textbf{corpo quadrático.} \(\square\)
\end{def*}
\begin{example}
	\begin{itemize}
		\item[1)] Os inteiros de Gauss, \(\mathbb{Z}[i]\), com o polinômio correspondente \(x^{2} + 1 = 0\).
		\item[2)] Os inteiros de Eisenstein, \(\mathbb{Z}\biggl[\frac{1+\sqrt[]{-3}}{2}\biggr]\), com o polinômio \(x^{2} - x + 1 = 0\)
		\item[3)] Para o polinômio \(x^{2} - x - 1 = 0, \mathbb{Z}\biggl[\frac{1 + \sqrt[]{5}}{2}\biggr]\)
		\item[4)] Se \(d\in \mathbb{Z}\) é \textbf{livre de quadrados}, i.e., não existe \(l\in \mathbb{Z}\)
		      tal que \(l^{2}\mid d, \mathbb{Z}[\sqrt[]{d}]\) é um anel quadrático do polinômio \(x^{2} - d = 0.\)
	\end{itemize}
\end{example}
\begin{def*}
	Seja d livre de quadrados, denotamos por \(\mathcal{O}_{d}\) o conjunto
	\(\{\alpha \in \mathbb{Q}[\sqrt[]{d}]:\text{ existe }f(x)\in \mathbb{Z}[x], \deg{f(x)} = 2, f \text{ mônico irracionais, tais que } f(\alpha)=0\}.\quad\square\)
\end{def*}
\begin{theorem*}
	Se \(d\in \mathbb{Z}\) é livre de quadrados, então
	\[
		\mathcal{O}_{d} = \left\{\begin{array}{ll}
			\mathbb{Z}[\sqrt[]{d}],\quad\text{se } d\equiv 2, 3 \mod{4} \\
			\mathbb{Z}\biggl[\frac{1+\sqrt[]{d}}{2}\biggr],\quad\text{se } d\equiv 1\mod{4}.
		\end{array}\right.
	\]
\end{theorem*}
\begin{def*}
	O anel \(\mathcal{O}_{d}\) é chamado o \textbf{anel dos inteiros} do corpo \(\mathbb{Q}[\sqrt[]{d}].\) Nele,
	podemos definir a norma:
	\begin{align*}
		 & N_{d}:\mathcal{O}_{d}\rightarrow \mathbb{N}                                              \\
		 & x+y\theta \mapsto \left\{\begin{array}{ll}
			                            x^{2} + dy^{2},\quad \text{se } d\equiv2,3\mod{4} \\
			                            x^{2} - xy - \frac{d-1}{4}y^{2},\quad\text{se } d\equiv1\mod{4},
		                            \end{array}\right.
	\end{align*}
	em que \(\theta = \sqrt[]{d}\) se \(d\equiv 2, 3\mod{4}\) e \(\theta = \frac{1+\sqrt[]{d}}{2}\), se \(d\equiv 1\mod{4}.\square\)
\end{def*}
\begin{def*}
	Dizemos que \(\mathcal{O}_{d}\) é \textbf{norma euclidiano} se é um domínio euclidiano com a norma definida \(N_{d}\) acima. \(\square\)
\end{def*}
\begin{theorem*}
	\(\mathcal{O}_{d}\) é norma euclidiano se, e somente se,
	\[
		d\in\{-11, -7, -3, -2, -1, 2, 3, 5, 6, 7, 11, 13, 17, 19, 21, 29, 33, 37, 41, 57, 73\}.
	\]
\end{theorem*}
\begin{theorem*}
	\begin{itemize}
		\item[1)] \(\mathcal{O}_{69}\) é euclidiano (mas não é norma euclidiano)
		\item[2)] \(\mathcal{O}_{14}\) é euclidiano (mas não é norma euclidiano)
		\item[3)] Se \(d < 0\) e \(d\not\in\{-11, -7, -3, -2, 1\},\) então \(\mathcal{O}_{d}\) não é euclidiano
		\item[4)] Se \(d < 0\), então \(\mathcal{O}_{d}\) é D.I.P., se e somente se
		      \[
			      d\in\{-1, -2, -3, -7, -11, -19, -43, -67, -163\}.
		      \]
	\end{itemize}
\end{theorem*}
Relacionada a esses anéis dos inteiros, há a seguinte conjectura:

\textbf{Conjectura:} Classificar para quais \(d\in \mathbb{Z}\) o anel \(\mathcal{O}_{d}\) é D.I.P. e para quais
\(\mathcal{O}_{d}\) é euclidiano.

\begin{def*}
	Seja A um domínio. Dizemos que \(a\in A\) é \textbf{irredutível} se \(a\not\in A^{*}\) e se \(a = bc,\) então \(b\in A^{*}\) ou \(c\in A^{*}.\square\)
\end{def*}
\begin{example}
	\begin{itemize}
		\item[1)] Em \(\mathbb{Z}\), apenas primos positivos ou negativos são irredutíveis;
		\item[2)] Se F é um corpo, em \(F[x]\) apenas os polinômios irredutíveis são irredutíveis.
	\end{itemize}
\end{example}
\begin{theorem*}
	Se A é um D.I.P, então todo elemento não irredutível de A pode ser escrito como produto de elementos irredutíveis.
\end{theorem*}
\begin{proof*}
	Suponha que existe um \(a\in A\) que não é irredutível, porém não pode ser escrito como produto de
	elementos irredutíveis. Seja \(a=bc\) tais que \(b, c\not\in A^{*}.\)

	Note que um dos dois, b ou c, não pode ser produto de elementos irredutíveis. Suponha que este seja b.
	Disto, segue que \(\langle a \rangle\subsetneq{\langle b \rangle}\), já que, como \(a=bc, a\in \langle b \rangle\) e, assim,
	\(\langle a \rangle\subset{\langle b \rangle}\). Suponha, agora, que \(\langle a \rangle = \langle b \rangle.\) Então, \(b\in \langle a \rangle\) implica
	que \(b = ar\) para algum \(r\in A\), de forma que \(b = (bc)r.\) Como A é domínio, \(cr = 1\), ou seja, \(c\in A^{*},\)
	contrariando a hipótese antes feita.

	Seja \(a_{1}\coloneqq b.\) Segue que \(\langle a \rangle\subsetneq{\langle a_{1} \rangle}\) e \(a_{1}\) não é produto de elementos irredutíveis.
	Continuando este processo indutivamente, obteremos uma cadeia de ideais \(\langle a \rangle \subsetneq{\langle a_{1} \rangle}\subsetneq{\langle a_{2} \rangle}\subsetneq{\cdots}\)
	tal que cada \(a_{i}\) não é produto de elementos irredutíveis. Seja \(\mathfrak{i} = \bigcup_{i=1}^{\infty}{}\). Já que A é D.I.P., seja \(x\in A\) tal que \(\mathfrak{i} = \langle x \rangle\).

	Caso \(x\in \mathfrak{i},\) existe \(n\in \mathbb{N}\) tal que \(x\in \langle a_{n} \rangle\) e, então, \(\mathfrak{i} = \langle x \rangle\subseteq{\langle a_{n} \rangle}.\) Portanto,
	\(\langle a_{n} \rangle = \langle a_{n+1} \rangle = \cdots = \mathfrak{i}\), o que é um absurdo. Portanto, todo elemento irredutível de A
	pode ser escrito como produto de elementos irredutíveis. \qedsymbol
\end{proof*}
\end{document}

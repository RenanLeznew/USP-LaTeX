\documentclass{article}
\usepackage{bookmark}
\usepackage{amsmath}
\usepackage{amsthm}
\usepackage{amssymb}
\usepackage{pgfplots}
\usepackage[utf8]{inputenc}
\usepackage{amsfonts}
\usepackage[margin=2.5cm]{geometry}
\usepackage{graphicx}
\usepackage[export]{adjustbox}
\usepackage{fancyhdr}
\usepackage[portuguese]{babel}
\usepackage{hyperref}
\usepackage{multirow}
\usepackage{lastpage}
\usepackage{mathtools}
\usepackage{fouriernc}

\pagestyle{fancy}
\fancyhf{}

\pgfplotsset{compat = 1.18}

\hypersetup{
   colorlinks,
   citecolor=black,
   filecolor=black,
   linkcolor=black,
   urlcolor=black
}
\newtheorem*{def*}{\underline{Defini\c c\~ao}}
\newtheorem*{theorem*}{\underline{Teorema}}
\newtheorem*{lemma*}{\underline{Lema}}
\newtheorem*{prop*}{\underline{Proposi\c c\~ao}}
\newtheorem{example}{\underline{Exemplo}}
\newtheorem*{proof*}{\underline{Prova}}
\newtheorem*{crl*}{\underline{Corolário}}
\renewcommand\qedsymbol{$\blacksquare$}

\rfoot{P\'agina \thepage \hspace{1pt} de \pageref{LastPage}}

\begin{document}
\begin{figure}[ht]
\minipage{0.76\textwidth}
\includegraphics[width=4cm]{icmc.png}
\hspace{7cm}
\includegraphics[height=4.9cm,width=4cm]{brasao_usp_cor.jpg}
\endminipage  
\end{figure}

\begin{center}
\vspace{1cm}
\LARGE
UNIVERSIDADE DE S\~AO PAULO

\vspace{1.3cm}
\LARGE
INSTITUTO DE CI\^ENCIAS MATEM\'ATICAS E COMPUTACIONAIS - ICMC

\vspace{1.7cm}
\Large
\textbf{Notas de Álgebra II}

\vspace{1.3cm}
\large
\textbf{Renan Wenzel - 11169472}

\vspace{1.3cm}
\large
\textbf{Professor(a): Behrooz Miraii}

\textbf{E-mail: behrooz@icmc.usp.br}

\vspace{1.3cm}
\today
\end{center}

\newpage
\textbf{{\Huge Disclaimer}}
\vspace{5cm}

{\huge Essas notas não possuem relação com professor algum. 

Qualquer erro é responsabilidade solene do autor.

Caso julgue necessário, contatar: renan.wenzel.rw@gmail.com}
\tableofcontents

\newpage

\section{Aula 01 - 09/08/2023}
\subsection{Motivações}
\begin{itemize}
  \item Anéis e Anéis Comutativos;
  \item Exemplos de Anéis;
  \item Grupo Multiplicativo e Elemento Inversível.
\end{itemize}
\subsection{Anéis - Nosso Objeto de Estudo no Curso}
 \begin{def*}
  Um \textbf{anel} é um conjunto A munido de duas operações, \(+, \cdot \), tais que:
 \begin{itemize}
  \item[1)] \((A, +)\) é um grupo abeliano;
  \item[2)] \((A, \cdot )\) é associativo, com unidade \(1_{A}\);
  \item[3)] Para todos a, b, c em A, temos 
    \[
      (a+b)\cdot c = a \cdot c + b \cdot c \quad\&\quad a \cdot (b+c) = a \cdot b + a \cdot c.\quad\square
    \]
 \end{itemize}
 \end{def*}
\begin{def*}
   Dado um anel \((A, +, \cdot )\), diremos que A é um \textbf{anel comutativo} se o grupo \((A, \cdot )\) é abeliano. \(\square\)
\end{def*}
\begin{example}
 \begin{itemize}
  \item[1)] Com as operações usuais, \(\mathbb{Z}, \mathbb{R}, \mathbb{C}\) são todos anéis (comutativos).
  \item[2)] Os anéis de polinômios na variável X com coeficientes inteiros ou complexos são anéis, i.e., 
    \[
      \mathbb{Z}[x]\coloneqq \biggl\{\sum\limits_{i=0}^{n}a_{i}x^{i}: n\in \mathbb{N}, a_{i}\in \mathbb{Z}\biggr\}\quad\&\quad \mathbb{C}[x]\coloneqq \biggl\{\sum\limits_{i=0}^{n}a_{i}x^{i}:n\in \mathbb{N}, a_{i}\in \mathbb{C}\biggr\}
    \]
 \end{itemize}
\end{example}
\textbf{\underline{Observação 1}:} A unidade é única - Se \(1_{A}, e\) são duas unidades de \((A, \cdot ),\) então 
  \[
    1_{A}\cdot e = e = e \cdot 1_{A} = 1_{A} \Rightarrow e = 1_{A}.
  \]

 \textbf{\underline{Observação 2}:} Se A é um anel e \(0\in A\) é o elemento neutro da soma, temos, para todo \(a\in A\), 
  \[
    a0 = a(0+0) = a0 + a0 \Rightarrow a0 = 0.
  \]
\begin{def*}
  Seja \((A, +, \cdot )\) um anel. Um subconjunto \(B\subseteq{A}\) é um \textbf{subanel} de A se \((B, +, \cdot )\) é um anel e \(1_{A} = 1_{B}.\quad\square\)
\end{def*}
  A segunda condição é importante para evitar casos como o seguinte: Considere \(\mathbb{Z}\times \{0\}\subseteq{\mathbb{Z}\times \mathbb{Z}}.\) Note que a restrição
das operações de \(\mathbb{Z}\times \mathbb{Z}\) faria de \(\mathbb{Z}\times \{0\}\) um subanel, mas \(1_{\mathbb{Z}\times \mathbb{Z}} = (1, 1)\) e \(1_{\mathbb{Z}\times\{0\}}=(1, 0).\)
\begin{example}
\begin{itemize}
  \item[i)] Temos a seguinte cadeia de subanéis: \(\mathbb{Z}\subseteq{\mathbb{Q}}\subseteq{\mathbb{R}}\subseteq{\mathbb{C}}.\)
  \item[ii)] Analogamente, temos a seguinte cadeia de subanéis de polinômios: \(\mathbb{Z}[x]\subseteq{\mathbb{Q}[x]}\subseteq{\mathbb{R}[x]}\subseteq{\mathbb{C}[x]}\).
  \item[iii)]\texttt{(Anel zero)} Seja S um conjunto unitário \(S = \{a\}.\) Podemos definir as operações \(a + a\coloneqq a\) e \(a \cdot a\coloneqq a\). Com estas operações,
S torna-se um anel, o \textbf{Anel Zero}, no qual \(1_{S} = 0_{S}\). Este anel é especial não só porque podemos brincar que ``1 = 0'', como também porque ele é \textit{único}.

  Com efeito, seja A um anel qualquer no qual \(1_{A} = 0_{A}\). Para \(a\in A\), vale \(a = a \cdot 1_{A} = a \cdot 0_{A} = 0.\) Assim, \(a = 0\) e \(A = \{0\}\) é 
um anel zero. 

  Portanto, denotaremos o único anel zero por \(\{0_{A}\} = (0).\)
\end{itemize}
\end{example}
\begin{example}[Exercícios]
  Mostre que os seguintes conjuntos, com suas operações usuais, são anéis:
\begin{itemize}
  \item[a)] \(\mathbb{Z}[i] \subseteq{\mathbb{C}}, \mathbb{Z}[\sqrt[]{p}]\subseteq{\mathbb{R}},\) em que p é primo;
  \item[b)] \(\mathbb{Z}\biggl[\frac{1}{p}\biggr]\coloneqq \biggl\{\frac{a}{p^{r}}: a\in \mathbb{Z}, r\in \mathbb{N}\biggr\}\subseteq{\mathbb{Q}}\);
  \item[c)] \(\mathbb{Z}_{(p)}\coloneqq \biggl\{\frac{a}{b}: a, b\in \mathbb{Z}\quad \text{e}\quad \text{p não divide b} \biggr\}\subseteq{\mathbb{Q}}\).
\end{itemize}
\end{example}
\begin{def*}
  Seja A um anel. Um elemento \(a\in A\) é dito \textbf{invertível} (ou \textbf{unitário}) se existe \(b\in B\) tal que \(ab = 1_{A}.\) O \textbf{grupo multiplicativo} de A
é o conjunto 
  \[
    A^{*}\coloneqq \{a\in A: \text{a é invertível}\}.\quad\square
  \]
\end{def*}
  Vale observar que, caso exista o inverso, ele é único.
\begin{prop*}
  O grupo multiplicativo de um anel \((A, +, \cdot )\) é, de fato, um grupo.
\end{prop*}
\begin{proof*}
  Primeiro, note que \(1_{A}\in A^{*},\) pois \(1_{A} \cdot 1_{A} = 1_{A}.\) 

  Além disso, todo elemento \(a\in A^{*}\) possui inverso \(a^{-1}\in A^{*}.\) A operação herda a associatividade do anel. 

  Finalmente, basta provarmos que A é fechado por multiplicação. Com efeito, se \(a, b\in A^{*},\) considere \(x = b^{-1}a^{-1}\) e note que \(abx = abb^{-1}a^{-1} =
a 1_{A} a^{-1} = aa^{-1} = 1_{A}\) e, assim, \(ab\in A^{*}.\)

  Portanto, \(A^{*}\) é um grupo. \qedsymbol
\end{proof*}
\begin{example}[Exercício]
  Prove que os grupos multiplicativos dos anéis a seguir são estes:
\begin{itemize}
  \item[1)] \((\mathbb{Z}/n \mathbb{Z})^{*} = \{\overline{r}: (n, r) = 1\}\), em que \((n, r)\) denota o \textit{maior divisor comum} entre n e r;
  \item[2)] \((\mathbb{Z}[i])^{*}=\{\pm1, \pm i\}\);
  \item[3)] \((\mathbb{Z}[\sqrt[]{p}])^{*} = \{\pm1\}\);
  \item[4)] \((\mathbb{Z}[1/p])^{*} = \{\pm p^{r}: r\in \mathbb{Z}\}\);
  \item[5)] \((\mathbb{Z}_{(p)})^{*} = \biggl\{\frac{a}{b}: a, b\in \mathbb{Z}, \text{ em que p não divide a e nem b}\biggr\}\);
  \item[6)] Se A é um anel comutativo, \(M_{n}(A)^{*} = \{A\in M_{n}(A): \det{(A)}\in A^{*}\}\).
\end{itemize}
\end{example}
\newpage

\section{Aula 02 - 14/08/2023}
\subsection{Motivações}
\begin{itemize}
  \item Domínios;
  \item Ideais, Ideais Principais e Ideais Finitamente Gerados;
  \item Corpos;
  \item Operações com Ideais e Ideais em \(\mathbb{Z}\).
\end{itemize}
\subsection{Domínios e Ideais}
\begin{def*}
  Seja A um anel. Um elemento não nulo \(a\in A\) é dito \textbf{divisor de zero} se existe \(b\in A\setminus{\{0\}}\) tal que
 \(ab = 0.\) Caso A não possua divisores de zero, A é chamado \textbf{domínio.} \(\square\)
\end{def*}
\begin{example}
  O anel \(\mathbb{Z}/4\) não é um domínio, pois \(\overline{2}\cdot \overline{2} = \overline{4} = \overline{0}.\)
\end{example}
\begin{example}
 \begin{itemize}
  \item[1)] Se A é um anel, \(A\times A\) não é um domínio. Basta notar que \((1, 0)\cdot (0, 1) = (0, 0).\)
  \item[2)] O conjunto \(A = \mathbb{Q} + \mathbb{Q}x \subseteq{\mathbb{Q}[x]}\) (apenas como um subconjunto) com operação de soma usual e um
produto diferente definido por 
  \[
    (a+bx)(c+dx) = ac + (bc+ad)x.
  \]
  Neste caso, \(x \cdot x = 0.\) Essa construção será futuramente conhecida como o anel quociente \(\mathbb{Q}/(x^{2}).\)
 \end{itemize}
\end{example}
\begin{example}[Exercício]
 \begin{itemize}
  \item[1)] Temos \(\mathbb{Z}/n \mathbb{Z}\) é um domínio se, e somente se, n é primo;
  \item[2)] A é domínio se, e somente se, \(A[x]\) é domínio.
 \end{itemize}
\end{example}
\begin{def*}
  Seja A um anel. Um subconjunto \(\mathfrak{i}\subseteq{A}\) é um \textbf{ideal} se:
 \begin{itemize}
  \item[1)] Para todos \(a, b\in \mathfrak{i}, a + b\in \mathfrak{i}\), ou seja, \textbf{um ideal é fechado pela soma};
  \item[2)] Para \(a\in \mathfrak{i}\) e \(r\in A, r \cdot a\in \mathfrak{i}\), ou seja, \textbf{um ideal absorve produtos}.
 \end{itemize}
Ideais serão denotados por letras em \textit{Old English Script} - letras frescas, como \(\mathfrak{i}, \mathfrak{j}, \mathfrak{o}, \mathfrak{a}, \dotsc\).
Além disso, para dizer que um subconjunto é um ideal, escrevemos \(\mathfrak{i} \trianglelefteq{A}.\quad\square\)
\end{def*}
\begin{prop*}
  Todo ideal \(\mathfrak{i}\trianglelefteq{A}\) é um subgrupo pela soma de A, ou seja, \((\mathfrak{i}, +)\leq (A, +).\)
\end{prop*}
\begin{proof*}
  Já vimos que, por definição, os ideais são fechados pela soma. Além disso, como \(0\in A,\) 
tome \(a\in \mathfrak{i}\) qualquer. Teremos \(0 = 0a\in \mathfrak{i},\) fornecendo o elemento neutro do grupo.
Finalmente, \(-a = (-1)a\in \mathfrak{i}\). Portanto, \((\mathfrak{i}, +)\) é um subgrupo de \((A, +).\) \qedsymbol
\end{proof*}
\begin{def*}
  Se \((M, +)\) é um grupo abeliano munido de uma aplicação \(A\times \mathfrak{i}\longrightarrow \mathfrak{i}\) com \((r, a)\mapsto r \cdot a\)
e satisfazendo
\begin{itemize}
  \item \((r + r')a = ra + r'a\);
  \item \(r(a + a') = ra + ra'\);
  \item \(r(r'a) = (rr')a\);
  \item \(1 \cdot a = a,\)
\end{itemize}
dizemos que M é um \textbf{A-módulo.} \(\square\)
\end{def*}
  Em particular, se \(\mathfrak{i} \trianglelefteq{A},\) então \(\mathfrak{i}\) é um A-módulo.
\begin{example}
  \begin{itemize}
    \item[i)] Seja A um anel. Então, \(\{0\}\) e A são ideais de A. 
    \item[ii)] Se \(a\in A,\) então \(\langle a \rangle\coloneqq aA = \{ra: r\in A\}\trianglelefteq{A}.\) De fato,
verifiquemos os axiomas de ideal para este caso.

  Caso \(ra, r'a\in \langle a \rangle\), então \(ra + r'a = (r+r')a\in \langle a \rangle\), já que \(r + r'\in A.\)
Além disso, se \(r'\in A\) e \(ra\in \langle a \rangle\), temos \(r'(ra) = (r'r)a\in \langle a \rangle\), pois \(rr'\in A.\)

  Portanto, \(\langle a \rangle \trianglelefteq{A}\). Em particular, \((0) = \langle 0 \rangle\) e \(\langle 1 \rangle = A.\)
  \end{itemize}
\end{example}
  Esse último tipo do exemplo compões uma classe importante de ideais, definida a seguir
\begin{def*}
  Seja A um anel. Um ideal \(\mathfrak{i}\trianglelefteq{A}\) é \textbf{principal} se existe \(a\in A\) tal que \(\langle a \rangle = \mathfrak{i}.\quad\square\)
\end{def*}
\begin{lemma*}
  Seja \(\{\mathfrak{i}_{k}\}_{k\in K}\) uma família qualquer de ideais de um anel A. Então, 
  \[
    \bigcap_{k\in K}^{}{\mathfrak{i}_{k}}\trianglelefteq{A}.\quad \text{(Em outras palavras, a interseção de ideais é um ideal)}
  \]
\end{lemma*}
\begin{proof*}
  Para não carregar a notação, coloque \(\mathfrak{i} = \bigcap_{k\in K}^{}{\mathfrak{i_{k}}}.\) Seja \(a, a'\in \mathfrak{i}\) e \(r\in A\).
Para \(k\in K, a, a'\in \mathfrak{i}_{k}\). Como cada \(\mathfrak{i}_{k}\) é um ideal, \(a + a'\in \mathfrak{i}_{k}\) e, assim,
 \(a + a'\in \mathfrak{i}.\) Além disso, como \(ra\in \mathfrak{i}_{k}\) para todo \(k\in K\), temos também \(ra \in \mathfrak{i}.\)

 Portanto, \(\mathfrak{i}\) é um ideal. \qedsymbol
\end{proof*}
\begin{prop*}[Exercício]
  Seja A um anel. Se \(\mathfrak{i}, \mathfrak{j}\trianglelefteq{A},\) mostre que \(\mathfrak{i}\cup \mathfrak{j}\trianglelefteq{A}\) se, e somente se,
 \(\mathfrak{i}\subseteq{\mathfrak{j}}\) ou \(\mathfrak{j}\subseteq{\mathfrak{i}}.\)
\end{prop*}
\begin{def*}
  Se \(S\subseteq{A}\) é um subconjunto, o ideal gerado por S é o ideal 
  \[
    \langle S \rangle\coloneqq \bigcap_{S\subseteq{\mathfrak{i}}\trianglelefteq{A}}^{}{\mathfrak{i}} \trianglelefteq{A}.\quad\square
  \]
\end{def*}
  Esta definição passa a noção de \textit{menor ideal que contém o conjunto S.}
\begin{lemma*}
  Vale a igualdade \(\langle S \rangle = \{\sum\limits_{i=1}^{n}a_{i}s_{i}: a_{i}\in A, s_{i}\in S, n\in \mathbb{N}\}.\)
\end{lemma*}
\begin{proof*}
  Novamente, a fim de não carregar a notação, seja \(\mathfrak{i} = \{\sum\limits_{i=1}^{n}a_{i}s_{i}: a_{i}\in A, s_{i}\in S, n\in \mathbb{N}\}\). Note que
 \(\mathfrak{i}\) é um ideal, pois a soma de dois elementos de \(\mathfrak{i}\) está em \(\mathfrak{i}\) e o produto por um elemento do anel é absorvido.
A soma segue por definição, e a absorção segue de, se \(r\in A,\) 
  \[
    r \sum\limits_{i=1}^{n}a_{i}s_{i} = \sum\limits_{i=1}^{n}(ra_{i})s_{i}\in \mathfrak{i}.
  \]
  Para ver que \(\langle S \rangle \subseteq{\mathfrak{i}}.\) Para isso, note que \(S\subseteq{\mathfrak{i}},\) bastando tomar \(a_{1} = 1\) para todo
 \(s\in S\). Assim, \(\langle S \rangle\) é a interseção de todos os ideais que contém S e, como \(\mathfrak{i}\) é um deles, a inclusão segue.

  Por outro lado, mostremos que \(\mathfrak{i}\subseteq{\langle S \rangle}.\) Se \(\mathfrak{j}\) é um ideal que contém S, pela definição de ideal, 
temos \(s_{1}, \dotsc, s_{n}\in S\) implica que \(s_{1} + \dotsc + s_{n}\in \mathfrak{j}\) e, se \(s\in S, r\in A\), então \(rs\in \mathfrak{j}.\)
Temos, assim, que se \(r_{1}, \dotsc, r_{n}\in A\) e \(s_{1}, \dotsc, s_{n}\in S\), então \(r_{i}s_{i}\in \mathfrak{j}\) e \(\sum\limits_{}^{}r_{i}s_{i}\in \mathfrak{j}.\)
Com isso, \(\mathfrak{i}\subseteq{\mathfrak{j}}.\) Repetindo isso para todo \(\mathfrak{j}\) que contém S, temos \(\mathfrak{i}\subseteq{\langle S \rangle}.\)

  Portanto, \(\langle S \rangle = \mathfrak{i}.\) \qedsymbol
\end{proof*}
\begin{def*}
  Seja A um anel. Dizemos que \(\mathfrak{i} \trianglelefteq{A}\) é \textbf{finitamente gerado (f.g.)} se existe um conjunto finito
 \(S = \{s_{1}, \dotsc, s_{n}\}\) tal que \(\mathfrak{i} = \langle S \rangle.\quad\square\)
\end{def*}
  Em particular, todo ideal principal é finitamente gerado.
\begin{def*}
  Dizemos que um domínio A é um \textbf{domínio de ideais principais (D.I.P.)} se todo ideal é principal. \(\square\)  
\end{def*}
\begin{def*}
  Dizemos que um anel A é um \textbf{corpo} se \((0)\) e A são os únicos ideais principais. \(\square\)
\end{def*}
\begin{example}[Exercício]
\begin{itemize}
\item[1)] Se \(\mathfrak{i}\trianglelefteq{A}\) e \(\mathfrak{i}\cap A^{*}\neq\emptyset\), então \(\mathfrak{i} = A\);
\item[2)] Todo corpo é um domínio;
\item[3)] Todo domínio finito é um corpo;
\item[4)] A é um corpo se, e somente se, \(A^{*} = A \setminus{\{0\}}\);
\item[5)] \(\mathbb{Q}[i]\) é um corpo.
\end{itemize} 
\end{example}
\begin{lemma*}
  Vale que \(\mathbb{Z}\) e \(\mathbb{Z}/n \mathbb{Z}\) são domínio de ideais principais.
\end{lemma*}
\begin{proof*}
  Seja \(\mathfrak{i} \trianglelefteq{Z}.\) Então, \((\mathfrak{i}, +)\leq (\mathbb{Z}, +)\). Usando o resultado
de classificação de subgrupos de \((\mathbb{Z}, +)\), que afirma que todo subgrupo é da forma \(n \mathbb{Z}\) para
algum \(n\in \mathbb{N},\) segue que \(\mathfrak{i} = n \mathbb{Z}\) para algum n e, assim, \(\mathfrak{i} = \langle n \rangle\) 
é principal.

  O caso \(\mathbb{Z}/n \mathbb{Z}\) fica de exercício. \qedsymbol
\end{proof*}
\begin{def*}
  Seja A um anel. Se \(\mathfrak{i}, \mathfrak{j} \trianglelefteq{A},\) podemos definir os ideais \(\mathfrak{i} + \mathfrak{j}\coloneqq \{x+y:x\in \mathfrak{i}\text{ e } y\in \mathfrak{j}\}\)
e \(\mathfrak{i} \cdot \mathfrak{j}\coloneqq \{x_{1}y_{1}+\dotsc+x_{n}y_{n}: x_{i}\in \mathfrak{i}, y_{i}\in \mathfrak{j}, \text{ e } n\in \mathbb{N}\}.\quad\square\)
\end{def*}
  Vamos mostrar que esses conjuntos são, de fato, ideais. No primeiro caso, sejam \(a, a'\in \mathfrak{i} + \mathfrak{j}.\)
Assim, eles podem ser escritos como \(a = x + y\) e \(a'= x'+ y'\) para \(x, x'\in \mathfrak{i}\) e \(y, y'\in \mathfrak{j}.\)
Então, \(a + a' = (x + x') + (y + y')\in \mathfrak{i} + \mathfrak{j}.\) Considere agora \(r\in A\), tal que 
  \[
    ra = r(x'+y') = rx' + ry'.
  \]
  Como \(\mathfrak{i}, \mathfrak{j}\) são ideais, cada parcela destas está no seu ideal respectivo.
Logo, \(ra \in \mathfrak{i} + \mathfrak{j}.\)

  No segundo caso, se \(a, b\in \mathfrak{i}\cdot \mathfrak{j},\) temos 
  \[
    a = \sum\limits_{}^{}x_{i}y_{i}\quad\&\quad b = \sum\limits_{}^{}u_{j}v_{j},
  \]
com \(x_{i}, u_{j}\in \mathfrak{i}\) e \(y_{i}, v_{j}\in \mathfrak{j}.\) Reorganizando os termos, podemos ver
 \(a+b\) como a soma finita de termos da forma \(x_{k}y_{k}, x_{k}\in \mathfrak{i}, y_{k}\in \mathfrak{j}\), ou seja, 
 \(a+b\in \mathfrak{i}\cdot \mathfrak{j}\). Além disso, se \(r\in A,\) temos 
  \[
    ra = r \sum\limits_{}^{}x_{i}y_{i} = \sum\limits_{}^{}(rx_{i})y_{i}\in \mathfrak{i}\cdot \mathfrak{j},
  \]
pois \(rx_{i}\in \mathfrak{i},\) já que \(\mathfrak{i}\) é ideal, provando, assim, que ambos os conjuntos definidos são ideais.
\begin{lemma*}
  Se A é um anel e \(\mathfrak{i}, \mathfrak{j}\trianglelefteq{A},\) então \(\mathfrak{i}\cdot \mathfrak{j}=\langle \{xy: x\in \mathfrak{i}, y\in \mathfrak{j}\} \rangle\) e
 \(\mathfrak{i}+\mathfrak{j} = \langle \mathfrak{i}\cup \mathfrak{j} \rangle\).
\end{lemma*}
\begin{proof*}
  Denotemos por \(\mathfrak{b}\) o conjunto 
  \[
    \mathfrak{b}\coloneqq \langle \{xy:x\in \mathfrak{i}, y\in \mathfrak{j}\} \rangle.
  \]
Primeiramente, note que \(\mathfrak{i}\cdot \mathfrak{j}\subseteq{\mathfrak{b}}.\) De fato, seja \(\mathfrak{k}\) um ideal tal que \(\{xy:x\in \mathfrak{i}, y\in \mathfrak{j}\}\subseteq{\mathfrak{k}}.\)
Como \(\mathfrak{k}\) é fechado pela soma, se \(x_{1}, \dotsc, x_{n}\in \mathfrak{i}\) e \(y_{1},\dotsc,y_{n}\in \mathfrak{j},\) temos  
 \(\sum\limits_{}^{}x_{i}y_{i}\in \mathfrak{k}.\) Logo, \(\mathfrak{i}\cdot \mathfrak{j}\subseteq{\mathfrak{k}}\) e \(\mathfrak{i}\cdot \mathfrak{j}\) está
 na intersecção \(\langle \{xy: x\in \mathfrak{i}, y\in \mathfrak{j}\} \rangle.\)

 Por outro lado, por definição, elementos da forma \(xy\) tal que \(x\in \mathfrak{i}, y\in \mathfrak{j}\) são elementos
de \(\mathfrak{i}\cdot \mathfrak{j}.\) Assim, \(\{xy:x\in \mathfrak{i}, y\in \mathfrak{j}\}\subseteq{\mathfrak{i}\cdot \mathfrak{j}}.\) Como
 \(\mathfrak{i}\cdot \mathfrak{j}\) é ideal, a intersecção de ideais com ele está contido nele, ou seja, \(\mathfrak{b}\subseteq{\mathfrak{i}\cdot \mathfrak{j}}\).
 Logo, \(\mathfrak{b} = \mathfrak{i}\cdot \mathfrak{j}.\)

 Com relação à segunda parte, começamos por mostrar que \(\mathfrak{i}+\mathfrak{j}\subseteq{\langle \mathfrak{i}\cup \mathfrak{j} \rangle}.\) De fato,
seja \(\mathfrak{k}\) um ideal tal que \(\mathfrak{i}\cup \mathfrak{j}\subseteq{\mathfrak{k}}.\) Como \(\mathfrak{k}\) é fechado pela soma,
se \(x\in \mathfrak{i}\) e \(y\in \mathfrak{j},\) temos \(x+y\in \mathfrak{k}\) tal que, assim, \(\mathfrak{i} + \mathfrak{j}\subseteq{\mathfrak{k}}\).
Finalmente, como \(\mathfrak{k}\) é qualquer, \(\mathfrak{i}+\mathfrak{j}\subseteq{\langle \mathfrak{i}\cup \mathfrak{j} \rangle}.\)

  Em contrapartida, pela definição de \(\mathfrak{i} + \mathfrak{j},\) conseguimos 
que \(\mathfrak{i}\cup \mathfrak{j}\subseteq{\mathfrak{i}+\mathfrak{j}}\) (basta tomar 0 + y ou x + 0).
Como \(\mathfrak{i} + \mathfrak{j}\) é ideal e contém \(\mathfrak{i}\cup \mathfrak{j},\) segue que \(\langle \mathfrak{i}\cup \mathfrak{j} \rangle\subseteq{\mathfrak{i}+\mathfrak{j}}\).
Portanto, \(\mathfrak{i}+\mathfrak{j} = \langle \mathfrak{i}\cup \mathfrak{j} \rangle.\) \qedsymbol
\end{proof*}
\begin{prop*}
  Seja \(\{\mathfrak{i}_{k}\}_{k\in K}\) uma família de ideais de A. Então,
 \begin{itemize}
  \item[1)] O conjunto 
  \[
    \sum\limits_{k\in K}^{}\mathfrak{i}_{k}\coloneqq \biggl\{\sum\limits_{\text{finita}}^{}a_{i}: a_{i}\in \mathfrak{i}\biggr\},
  \]
  é um ideal e, além disso, 
  \[
    \sum\limits_{k\in K}^{}\mathfrak{i}_{k}\coloneqq \biggl\langle \bigcup_{k\in K}^{}{\mathfrak{i}_{k}} \biggr\rangle
  \]
  \item[2)] Se \((K,\leq )\) é um conjunto totalmente ordenado e a família satisfaz a propriedade que
\textbf{para todos k, k' em K tais que} \(k\leq k'\), \textbf{temos} \(\mathfrak{i}_{k}\subseteq{\mathfrak{i}_{k'}}\). Então, 
  \[
    \bigcup_{k\in K}^{}{\mathfrak{i}_{k}}\trianglelefteq{A}.
  \]
 \end{itemize}
 \begin{proof*}
  Provar que \(\sum\limits_{}^{}\mathfrak{i}_{k}\) é um ideal e que é igual ao gerado pela união é análogo ao caso finito.

  Para o item dois, sejam \(a, a'\in \cup \mathfrak{i}_{k}.\) Assim, existem \(k, k'\) tais que \(a\in \mathfrak{i}_{k}\) e \(a'\in \mathfrak{i}_{k'}\).
Como K é totalmente ordenado, podemos supor, sem perda de generalidade, que \(k\leq k'.\) Pela propriedade que K possui,
 \(\mathfrak{i}_{k}\subseteq{\mathfrak{i}_{k'}}\) e \(a, a'\in \mathfrak{i}_{k'}.\) Como \(\mathfrak{i}_{k'}\) é ideal, a soma
 \(a + a'\in \mathfrak{i}_{k}\) e, assim, \(a + a'\in\cup \mathfrak{i}_{k}.\) Agora, considere \(r\in A\)
 e \(a\in \cup \mathfrak{i}_{k}.\) Então, existe \(k\in K\) tal que \(a\in \mathfrak{i}_{k}\). Já que \(\mathfrak{i}_{k}\)
é ideal, \(ra\in \mathfrak{i}_{k}\), tal que \(ra\in \cup \mathfrak{i}_{k}.\) Portanto, \(\bigcup_{k\in K}^{}{\mathfrak{i}_{k}}\) é um
ideal de A. \qedsymbol
 \end{proof*}
\end{prop*}
\begin{example}[Exercício]
  Se \(n \mathbb{Z}\) e \(m \mathbb{Z}\) são ideais de \(\mathbb{Z},\) prove que:
 \begin{itemize}
  \item[1)] \(n \mathbb{Z}\cdot m \mathbb{Z} = nm \mathbb{Z};\)
  \item[2)] \(n \mathbb{Z} + m \mathbb{Z} = \mathrm{mdc}(n, m) \mathbb{Z};\)
  \item[3)] \(n \mathbb{Z}\cap m \mathbb{Z} = \mathrm{mmc}(n, m) \mathbb{Z}.\)
 \end{itemize}
\end{example}
\begin{def*}
  Seja A um anel e \(\mathfrak{i}, \mathfrak{j}\trianglelefteq{A}.\) Dizemos que \(\mathfrak{i}, \mathfrak{j}\) são \textbf{coprimos}
se \(\mathfrak{i} + \mathfrak{j} = A.\quad\square\)
\end{def*}
\begin{prop*}[Exercício]
  Sejam \(\mathfrak{i}, \mathfrak{j}\trianglelefteq{A}\). Se \(\mathfrak{i}, \mathfrak{j}\) são coprimos, então \(\mathfrak{i}\cdot \mathfrak{j} = \mathfrak{i}\cap \mathfrak{j}.\)
\end{prop*}
\newpage

\section{Aula 03 - 16/08/2023}
\subsection{Motivações}
\begin{itemize}
  \item Ideais Primos e Ideais Maximais;
  \item Espectro de um Anel;
  \item Anéis Locais e Elementos Nilpotentes.
\end{itemize}
\subsection{Ideais Primos e  Maximais}
\begin{def*}
  Uma ideal \(\mathfrak{i} \trianglelefteq{A}\) é dito \textbf{primo} se \(\mathfrak{i}\neq A\) e \(ab\in \mathfrak{i}\)
implica \(a\in \mathfrak{i}\) ou \(b\in \mathfrak{i}.\) 

  Um ideal \(\mathfrak{i}\trianglelefteq{A}\) é dito maximal se \(\mathfrak{i}\neq A\) e se, dado \(\mathfrak{j}\trianglelefteq{A}\)
com \(\mathfrak{i}\subseteq{\mathfrak{j}}\), então 
  \[
    \mathfrak{i} = \mathfrak{j}\text{ ou } \mathfrak{j} = A.\quad\square
  \]
\end{def*}
  Diremos que uma \textbf{cadeia} \(K\subseteq{S}\) de um conjunto parcialmente ordenado é um subconjunto totalmente
ordenado do mesmo.
\begin{prop*}
 \begin{itemize}
  \item[1)] Os ideais maximais são primos;
  \item[2)] Todo anel não nulo tem um ideal maximal e logo tem um ideal primo;
  \item[3)] Todo ideal próprio de A está contido em algum ideal maximal de A.
 \end{itemize}
\end{prop*}
\begin{proof*}
  1. Seja M maximal e \(ab\in M.\) Mostremos \(a\in M\) ou \(b\in M\). Suponha
 \(a\not\in M\). Considere o anel:
  \[
    M + \langle a \rangle = \{x+ra: x\in M\text{ e }r\in A\} \trianglelefteq{A}.
  \] 
  Note que \(M\subsetneq{M + \langle a \rangle},\) pois \(a\not\in M.\) Como M é maximal, segue que
 \(M + \langle a \rangle = A.\) Assim, \(1\in M + \langle a \rangle\) e existe \(x\in M\), \(r\in A\)
tais que \(x + ra = 1\) e, então, 
  \[
    b = b1 = b(x+ra) = bx + rab.
  \]
  Como M é ideal, \(ab\in M\), o que implica em \(rab\in M\), e \(x\in M\) implica em \(bx\in M\).
Assim, \(b = bx + rab\in M\). 

  2. Para obtermos o ideal maximal, usaremos o Lema de Zorn. Construiremos o conjunto parcialmente ordenado 
dado por \(S\coloneqq \{\mathfrak{i}\trianglelefteq{A}: \mathfrak{i}\neq A\}\), sendo a ordem dada por \(\subseteq{}.\)
Sabemos que \(S \neq\emptyset\), pois \((0)\in S.\) Considere, agora, uma cadeia \(\{\mathfrak{i}_{l}\}_{l\in L}\) de elementos
de S. Por indução, a união destes ideais é um ideal e, além disso, é um ideal próprio
  \[
    \mathfrak{i}\coloneqq \bigcup_{l\in L}^{}{\mathfrak{i}_{l}}\vartriangleleft{A},
  \]
pois para cada \(l\in L, 1\not\in \mathfrak{i}_{l}\), já que, se \(1\in \mathfrak{i}_{l}\) para algum \(l\in L\), teríamos
 \(\mathfrak{i} = A\). Então, a união é um elemento de S e é uma cota superior da cadeia, no sentido de \(\mathfrak{i}_{l}\subseteq{\mathfrak{i}}\)
para cada \(l\in L.\) Pelo Lema de Zorn, segue que existe um elemento maximal no conjunto S, que será o ideal maximal por definição.

  3. Considere \(\mathfrak{i}\) um ideal próprio de A e defina o conjunto \(S_{\mathfrak{i}}\coloneqq \{\mathfrak{j}\vartriangleleft{A}:\mathfrak{i}\subseteq{\mathfrak{j}}\}\).
Como \(\mathfrak{i}\in S_{\mathfrak{i}}, S_{\mathfrak{i}}\neq\emptyset.\) Seja \(\{\mathfrak{j}_{l}\}_{l\in L}\) uma cadeia de \(S_{\mathfrak{i}}.\)
Assim como no item 2, a união de elementos da cadeia é cota superior para ela e, aplicando o Lema de Zorn, temos um elemento
maximal dessa família - um anel maximal que contém \(\mathfrak{i}.\) 
\qedsymbol
\end{proof*}
\begin{def*}
  Seja A um anel. Definimos o \textbf{espectro primo} de A por \(\mathrm{Spec}(A) = \{\mathfrak{p}\trianglelefteq{A}: \mathfrak{p}\text{ é ideal primo de } A\}\)
e o \textbf{espectro maximal} de A como \(\mathrm{Specm}(A) = \{\mathfrak{m}\trianglelefteq{A}:\mathfrak{m} \text{ é ideal maximal de } A\}.\quad\square\)
\end{def*}
\begin{prop*}[Exercício]
 \begin{itemize}
  \item[1)] A é domínio se, e somente se, \((0)\in \mathrm{Spec}(A)\)
  \item[2)] Dado um ideal \(\mathfrak{i} = n \mathbb{Z},\) então \(\mathfrak{i}\in \mathrm{Spec}(\mathbb{Z})\) se, e somente se,
n é primo ou n = 0. Assim, \(\mathrm{Spec}(\mathbb{Z}) = \{p \mathbb{Z}: p \text{ é primo}\}\cup \{(0)\}\).
  \item[3)] Dado um ideal \(\mathfrak{i} = n \mathbb{Z}\), então \(\mathfrak{i}\in \mathrm{Specm}(\mathbb{Z})\) se, e somente se,
n é primo. Logo, \(\mathrm{Specm}(\mathbb{Z}) = \{p \mathbb{Z}:p \text{ é primo}\}\).
 \end{itemize}
\end{prop*}
\begin{def*}
  Um anel é dito \textbf{local} se tem só um ideal maximal. \(\square\)
\end{def*}
\begin{lemma*}
  Seja \(\mathfrak{m}\in \mathrm{Specm}(A)\). Então, A é local se, e somente se, \(A^{*} = A\setminus{\mathfrak{m}}\).
\end{lemma*}
\begin{proof*}
  \(\Rightarrow )\) Suponha que \(\mathfrak{m}\) seja o único ideal maximal de A. Como \(\mathfrak{m} \neq\emptyset, A^{*}\cap \mathfrak{m} = \emptyset\).
Assim, \(A^{*}\subseteq{A\setminus{\mathfrak{m}}}.\) Tome \(a\in A\setminus{\mathfrak{m}}\) e considere o ideal principal \(\langle a \rangle \trianglelefteq{A}.\)
Se \(\langle a \rangle\neq A,\) então \(\langle a \rangle \subseteq{\mathfrak{m}},\) pois todo ideal próprio está contido num maximal, que supomos ser unicamente \(\mathfrak{m}\).
Logo, \(a\in \mathfrak{m}\), o que é um absurdo. Então, \(\langle a \rangle = A\) e existe \(r\in A\) tal que \(ra = 1,\) i.e., \(a\in A^{*}.\)

  \(\Leftarrow )\) Suponha que existe outro ideal maximal \(\mathfrak{m}'\neq \mathfrak{m}.\) Então, existe \(a\in \mathfrak{m}'\)
tal que \(a\in A\setminus{\mathfrak{m}} = A^{*}.\) Portanto, \(\mathfrak{m}'\cap A^{*} \neq\emptyset\) e, assim, \(\mathfrak{m}' = A,\)
contrariando a maximalidade de \(\mathfrak{m}.\) \qedsymbol
\end{proof*}
\begin{example}[Exercício]
 \begin{itemize}
  \item[1)] Mostre que todo corpo é um anel local.
  \item[2)] Mostre que \(\mathbb{Z}_{(p)},\) para p primo, é um anel local.
  \item[3)] Seja \(\mathbb{K}\) um corpo e seja \(f(x)\in \mathbb{K}[x]\) um polinômio não nulo irredutível.
Defina o seguinte conjunto:
  \[
    \mathbb{K}[x]_{(f(x))} = \biggl\{\frac{h(x)}{g(x)}: h(x), g(x)\in \mathbb{K}[x] \text{ e f(x) não divide g(x)}\biggr\}.
  \]
  Mostre que este é um anel local de \(\mathbb{K}[x].\)
 \end{itemize}
\end{example}
\begin{def*}
  Para um anel A, os ideais a seguir:
  \[
    \mathrm{nil}(A) = \bigcap_{\mathfrak{p}\in \mathrm{Spec}(A)}^{}{\mathfrak{p}}\quad \& \quad J(A)\coloneqq \bigcap_{\mathfrak{m}\in \mathrm{Specm}(A)}^{}{\mathfrak{m}}
  \]
  são chamados nil-radical e radical Jacobson de A. \(\square\)
\end{def*}
\begin{prop*}
  Temos \(\mathrm{nil}(A) \subseteq{J(A)}.\)
\end{prop*}
\begin{proof*}
  De fato, como sabemos que os ideais maximais são primos, vale que \(\mathrm{Specm}(A)\subseteq{\mathrm{Spec}(A)}.\) Assim, se
 \(x\in \mathrm{nil}(A),x\in \mathfrak{p}\) para todo \(\mathfrak{p}\in \mathrm{Spec}(A).\) Em particular, \(x\in \mathfrak{p}\) para todo \(\mathfrak{p}\in \mathrm{Specm}(A)\)
 e então \(x\in J(A)\). Portanto, \(\mathrm{nil}(A) \subseteq{J(A)}.\) \qedsymbol
\end{proof*}
\begin{def*}
  Um elemento \(a\in A\) é dito \textbf{nilpotente} se existir um inteiro \(n\geq 1\) tal que \(a^{n} = 0.\) Um ideal
 \(\mathfrak{i}\) é dito \textbf{nilpotente} se existir um inteiro \(n\geq 1\) tal que \(\mathfrak{i}^{n} = (0).\)
\end{def*}
\begin{prop*}[Exercício]
 \begin{itemize}
  \item[1)] Mostre que \(\mathrm{nil}(A)\) é o conjunto de todos os elementos nilpotentes de A;
  \item[2)] Mostre que se \(\mathrm{nil}(A)\) é finitamente gerado, ele é um ideal nilpotente;
  \item[3)] Se A é domínio, \((0)\in \mathrm{Spec}(A)\) e, logo, \(\mathrm{nil}(A) = (0);\)
  \item[4)] Se A é local com ideal maximal \(\mathfrak{m}\), então \(J(A) = \mathfrak{m}.\)
 \end{itemize}  
\end{prop*}
\newpage

\section{Aula 04 - 21/08/2023}
\subsection{Motivações}
\begin{itemize}
  \item Morfismos entre Anéis
\end{itemize}
\subsection{Morfismos de Anéis}
\begin{def*}
  Sejam \((A, +_{A}, \cdot_{A})\) e \(B, +_{B}, \cdot_{B})\) anéis. Um mapa \(f:A\rightarrow B\) é chamado um \textbf{homomorfismo de anéis}
(ou, neste texto, apenas morfismo), se:
\begin{itemize}
  \item[i)] \(f(a +_{A} a') = f(a) +_{B} f(a')\) para todo \(a, a'\in A;\)
  \item[ii)] \(f(a \cdot_{A} a') = f(a)\cdot_{B}f(a')\) para todo \(a, a'\in A;\)
  \item[iii)] \(f(1_{A}) = 1_{B}.\quad\square\)
\end{itemize}
\end{def*}
\begin{def*}
  Dizemos que um homomorfismo de anéis \(f:A\rightarrow B\) é:
\begin{itemize}
  \item[1)] \textbf{monomorfismo} se f é um morfismo injetor;
  \item[2)] \textbf{epimorfismo} se f é um morfismo sobrejetor;
  \item[3)] \textbf{isomorfismo} se f é um morfismo bijetor.
\end{itemize}
  Caso exista um isomorfismo entre dois anéis, dizemos que eles são \textbf{isomorfos}. \(\square\)
\end{def*}
\begin{prop*}[Exercício]
\begin{itemize}
  \item[1)] A identidade \(id_{A}:A\rightarrow A\) é um isomorfismo;
  \item[2)] Se A é subanel de B, então a inclusão \(i:A\rightarrow B\), \(i(a) = a\), é
um monomorfismo.
\end{itemize}
\end{prop*}
\begin{example}[Exercício]
\begin{itemize}
  \item[1)] Se A é um anel qualquer, então:
  \begin{align*}
      \varphi:&\mathbb{Z}\rightarrow A\\
              &n\mapsto n1_{A},
  \end{align*}
  é um morfismo.
  \item[2)] Se \(B=(0), f:A\rightarrow (0)\) é um morfismo.
  \item[3)] Se A é um anel, então:
  \begin{align*}
    e:&A[x]\rightarrow A\\
      &f(x)\mapsto f(a).
  \end{align*}
\end{itemize}
\end{example}
\begin{def*}
  Seja \(f:A\rightarrow B\) um homomorfismo. O conjunto \(\ker{(f)}\coloneqq \{a\in A: f(a) = 0\}\)
é chamado \textbf{núcleo} ou \textbf{kernel} de f. O conjunto \(\mathrm{Im}(f)\coloneqq f(A)\) é chamado 
\textbf{imagem} de f. \(\square\)
\end{def*}
\begin{lemma*}
  O kernel de um homomorfismo é um ideal do Anel domínio e a imagem de um homomorfismo é subanel do codomínio. Em outras palavras,
  \[
    \ker{(f)}\trianglelefteq{A}\quad\&\quad \mathrm{Im}(f)\text{ é subanel de }B.
  \]
\end{lemma*}
\begin{proof*}
  1.) Da Álgebra I, sabemos que \((\ker{(f)}, +)\leq (A, +).\) Seja \(a\in A\) e 
considere \(x\in\ker{(f)}.\) Note que \(f(ax) = f(a)f(x) = f(a)0 = 0\) e, então, \(ax\in\ker{(f)}.\)

  2.) Para provar este, basta usarmos as propriedades de morfismo de anéis. Segue que
 \(f(a), f(b)\in \mathrm{Im}(f),\quad f(a) + f(b) = f(a+b)\in \mathrm{Im}(f).\) Além disso,
 \(f(a)f(b) = f(ab)\in \mathrm{Im}(f)\) e \(1_{A}f(a) = f(1)f(a) = f(1a) = f(a).\) \qedsymbol
\end{proof*}
\begin{lemma*}
  As seguintes propriedades são equivalentes:
\begin{itemize}
  \item[1)] f é monomorfismo;
  \item[2)] \(\ker{(f)} = (0);\)
  \item[3)] Se \(f(a) = 0\), então \(a=0.\)
\end{itemize}
\end{lemma*}
  A prova é análoga ao caso de grupos, então será omitida.
\begin{example}[Exercício]
 \begin{itemize}
  \item[1)] Os anéis com um elemento são todos isomorfos;
  \item[2)] Todo anel com p elementos, sendo p um primo, é isomorfo à \(\mathbb{Z}/p \mathbb{Z};\)
  \item[3)] Os anéis A e B são isomorfos se, e somente se, existirem homomorfismos
\(f:A\rightarrow B\) e \(g:A\rightarrow B\) tais que \(f\circ{g}=id_{B}\) e \(g\circ{f} = id_{A}.\)
 \end{itemize} 
\end{example}
\begin{prop*}
  Seja \(f:A\rightarrow B\) um homomorfismo de anéis.
\begin{itemize}
  \item[1)] Se \(\mathfrak{j}\trianglelefteq{B}\), então \(f^{-1}(\mathfrak{j})\trianglelefteq{A}.\) Em particular, \(\ker{(f)} = f^{-1}((0))\trianglelefteq{A};\)
  \item[2)] Se \(\mathfrak{q}\in \mathrm{Spec}(B)\), então \(f^{-1}(\mathfrak{q})\in \mathrm{Spec}(A).\)
  \item[3)] Se A é subanel de B e \(\mathfrak{j}\trianglelefteq{B},\) então \(\mathfrak{j}\cap A \trianglelefteq{A}.\) Se \(i:A\hookrightarrow B\),
então \(i^{-1}(\mathfrak{j})=\mathfrak{j}\cap A.\)
\end{itemize}
\begin{proof*}
  1.) De fato, já sabemos que \(f^{-1}(\mathfrak{j})\leq A\) como grupo. Porém, sejam \(x\in f^{-1}(\mathfrak{j})\)
e \(y\in A\) tais que \(f(x) = a\in \mathfrak{j}.\) Como \(\mathfrak{j}\) é ideal, \(af(y)\in \mathfrak{j},\)
ou seja, \(f(x)f(y) = f(xy)\in \mathfrak{j}\). Logo, \(xy\in f^{-1}(\mathfrak{j})\).

  2.) Se \(xy\in f^{-1}(\mathfrak{q}),\) temos \(f(xy) = f(x)f(y)\in \mathfrak{q}.\) Como \(\mathfrak{q}\) é primo, 
então \(f(x)\in \mathfrak{q}\) ou \(f(y)\in \mathfrak{q}.\) Assim, \(x\in f^{-1}(\mathfrak{q})\) ou \(y\in f^{-1}(\mathfrak{q}).\)

  3.) A primeira parte segue do fato de que subanel é um subconjunto fechado pelas operações restritas. Além disso,
 \(i^{-1}(\mathfrak{j}) = \mathfrak{j}\cap A.\) \qedsymbol
\end{proof*}
\end{prop*}
  Note que a imagem inversa de um ideal maximal não é necessariamente maximal.
Basta considerar a inclusão \(i:\mathbb{Z}\hookrightarrow \mathbb{Q}, (0)\in \mathrm{Specm}(\mathbb{Q}),\)
mas \(i^{-1}((0)) = (0)\not\in \mathrm{Specm}(\mathbb{Z}).\)
\begin{lemma*}
  Seja \(f:A\rightarrow B\) um epimorfismo de anéis.
\begin{itemize}
  \item[1)] Se \(\mathfrak{i}\trianglelefteq{A},\) então \(f(\mathfrak{i})\trianglelefteq{B}.\) De fato, sabemos que a imagem
de grupo é um grupo.
  \item[2)] Todo ideal \(\mathfrak{j}\) de B é da forma \(f(\mathfrak{i})\) para algum ideal \(\mathfrak{i}\trianglelefteq{A}.\)
\end{itemize}
\end{lemma*}
\begin{proof*}
  1.) Já sabemos que a imagem é grupo. Seja \(y\in B.\) Como f é epimorfismo, existe
 \(a\in A\) tal que \(f(a) = y.\) Assim, se \(f(x)\in f(\mathfrak{i}), x\in \mathfrak{i}\) e 
 \(ax\in \mathfrak{i}.\) Então, \(f(ax) = f(a)f(x) = y f(x)\in f(\mathfrak{i}).\)

  2.) Seja \(\mathfrak{j}\trianglelefteq{B}.\) Com isso, \(f^{-1}(\mathfrak{j})\) é ideal de A, pelo lema anterior.
Como f é epimorfismo, \(f(f^{-1}(\mathfrak{j})) = \mathfrak{j}.\) Assim, \(\mathfrak{j}\) é imagem de um ideal
 \(f^{-1}(\mathfrak{j})\) de A. \qedsymbol
\end{proof*}
\newpage

\section{Aula 05 - 23/08/2023}
\subsection{Motivações}
\begin{itemize}
  \item Anel Quociente.
\end{itemize}
\subsection{Anel Quociente}
\begin{def*}
  Seja A um anel e \(\mathfrak{i}\trianglelefteq{A}.\) Como \((\mathfrak{i}, +)\leq (A, +)\) como grupos e
como é um subgrupo normal, podemos considerar o \textbf{grupo quociente} \((A/\mathfrak{i}, +)\), em que 
  \[
    A/\mathfrak{i} = \{a + \mathfrak{i}: a \in A\}.\quad\square
  \]
\end{def*}
  A operação de soma no grupo quociente é definida pela regra 
  \[
    (a+\mathfrak{i})+(b+\mathfrak{i})\coloneqq (a+b)+\mathfrak{i}.
  \]
  Definamos, também, uma nova operação neste grupo - a multiplicação, dada por 
  \[
    (a+\mathfrak{i})\cdot (b+\mathfrak{i})\coloneqq ab + \mathfrak{i}. 
  \]
  Precisamos mostrar que esta operação está bem definida. Com efeito, suponha que
 \(a+\mathfrak{i} = a'+\mathfrak{i}\) e \(b+\mathfrak{i} = b'+\mathfrak{i},\) ou seja,
  \(a'-a\in \mathfrak{i}\) e \(b-b'\in \mathfrak{i}.\) Vamos provar que 
  \[
    (a+\mathfrak{i})(b+\mathfrak{i}) = (a'+\mathfrak{i})(b'+\mathfrak{i}) \Longleftrightarrow ab - a'b'\in \mathfrak{i}.
  \]
  Note que podemos somar \(0=a'b-a'b\), a fim de obter 
  \[
    ab - a'b + a'b - a'b' = b(a-a') + a'(b-b')\in \mathfrak{i}.
  \] 
  Como \(b(a-a') + a'(b-b') = ba - a'b',\) mostramos o que queríamos. Por fim,
defina a identidade do produto por \(1 + \mathfrak{i}.\) A partir dessa construção,
definimos o \textbf{anel quociente A por \(\mathfrak{i}\)} como \((A/\mathfrak{i}, +, \cdot )\)
  
  Vale observe que, junto dessa construção, obtemos acesso canônico à definição do \textbf{mapa quociente} \(\pi :A\rightarrow A/\mathfrak{i},\)
dado por \(\pi(a) = a + \mathfrak{i}.\) Este mapa é um morfismo de anéis sobrejetor - um epimorfismo.
Com efeito, a sobrejetividade ocorre pois, se \(a+\mathfrak{i}\in A/\mathfrak{i}\), sabemos que \(\pi(a)= a + \mathfrak{i}.\)
Sobre a parte do morfismo, se \(a, b\in A, \pi (a+b) = (a+b)+\mathfrak{i} = (a+\mathfrak{i})+(b+\mathfrak{i}) = \pi(a) + \pi(b)\).
Além disso, \(\pi (ab) = ab + \mathfrak{i} = (a+\mathfrak{i})(b+\mathfrak{i}) = \pi (a)\pi (b).\)
\begin{prop*}
  Seja \(\mathfrak{i}\trianglelefteq{A}\) um ideal de A.
 \begin{itemize}
  \item[1)] Se \(\mathfrak{j}\trianglelefteq{A}\) e \(\mathfrak{i}\subseteq{\mathfrak{j}}, \) então \(\mathfrak{j}/\mathfrak{i}\trianglelefteq{A/\mathfrak{i}},\) em que 
 \(\mathfrak{j}/\mathfrak{i}=\{a+\mathfrak{i}\}\);
  \item[2)] Se \(\mathfrak{k}\trianglelefteq{A/\mathfrak{i}},\) então \(\mathfrak{k}=\mathfrak{j}/\mathfrak{i}\) para algum \(\mathfrak{j}\trianglelefteq{A};\)
  \item[3)] Exite uma correspondência biunívoca entre os conjuntos:
    \[
      \biggl\{\mathfrak{j}\trianglelefteq{A}: \mathfrak{i}\subseteq{\mathfrak{j}}\biggr\}\longleftrightarrow \biggl\{\mathfrak{k}\trianglelefteq{A/\mathfrak{i}}\biggr\};
    \]
  \item[4)] Se \(\mathfrak{k}\in \mathrm{Spec}(A/\mathfrak{i}),\) então \(\mathfrak{k} = \mathfrak{p}/\mathfrak{i}\) para algum \(\mathfrak{p}\in \mathrm{Spec}(A).\)
Assim, \(\mathrm{Spec}(A/\mathfrak{i}) = \{\mathfrak{p}/\mathfrak{i}:\mathfrak{i}\subseteq{\mathfrak{p}}\in \mathrm{Spec}(A)\}\);
  \item[5)] Se \(\mathfrak{k}\in \mathrm{Specm}(A/\mathfrak{i}),\) então \(\mathfrak{k} = \mathfrak{m}/\mathfrak{i}\) para algum \(\mathfrak{m}\in \mathrm{Specm}(A).\)
Com isso, \(\mathrm{Specm}(A/\mathfrak{i}) = \{\mathfrak{m}/\mathfrak{i}: \mathfrak{i}\subseteq{\mathfrak{m}}\in \mathrm{Specm}(A)\}.\)
 \end{itemize}
\end{prop*}
\begin{proof*}
  1.) Se \(a+\mathfrak{i}, b+\mathfrak{i}\in \mathfrak{j}/\mathfrak{i}\) e \(r+\mathfrak{i}\in A/\mathfrak{i},\) temos
 \(a, b\in \mathfrak{j}.\) Como \(\mathfrak{j}\) é ideal, \(a+b\in \mathfrak{j}\) e assim \((a+\mathfrak{i}) + (b+\mathfrak{i}) = (a+b)
 +\mathfrak{i}\in \mathfrak{j}/\mathfrak{i}.\) Além disso, \(ar\in \mathfrak{j}\), tal que \((a+\mathfrak{i})(r+\mathfrak{i})=ar + \mathfrak{i}\in
 \mathfrak{j}/\mathfrak{i}.\)

 2.) Tome \(\mathfrak{j} = \pi^{-1}(\mathfrak{k})\trianglelefteq{A}.\) Se \(\mathfrak{k}\trianglelefteq{A/\mathfrak{i}}, 0_{A/\mathfrak{i}}\in \mathfrak{k}\)
implica que \(\mathfrak{i}\subseteq{\mathfrak{j}}.\) Como \(\pi \) é sobrejetora, podemos escrever
 \(\mathfrak{k} = \pi(\mathfrak{j}) = \{a+\mathfrak{i}: a\in \mathfrak{j}\} = \mathfrak{j}/\mathfrak{i}.\)

 3.) Segue usando 1 e 2.

 4.) Dado \(\mathfrak{k}\in \mathrm{Spec}(A/\mathfrak{i})\), temos \(\mathfrak{k} = \mathfrak{p}/\mathfrak{i},\)
para \(\mathfrak{i}\subseteq{\mathfrak{p}}\trianglelefteq{A}.\) Vamos, então, provar que \(\mathfrak{p}\) é primo. 
De fato, se \(ab\in \mathfrak{p},\) então \(\pi (ab)=\pi(a)\pi(b)\in \mathfrak{k}\). Como \(\mathfrak{k}\) é primo,
isto significa que \(\pi(a)\in \mathfrak{k}\) ou \(\pi(b)\in \mathfrak{k}\). Logo, \(a\in \mathfrak{p}\) ou \(b\in \mathfrak{p},\) do que
segue que \(\mathfrak{p}\) é primo.

  5.) Suponha que \(\mathfrak{k}\in \mathrm{Specm}(A/\mathfrak{i}).\) Com isso, \(\mathfrak{k} = \mathfrak{m}/\mathfrak{i}\)
para algum \(\mathfrak{m}\subseteq{\mathfrak{i}}.\) Assuma que \(\mathfrak{j}/\mathfrak{i}\) é outro ideal tal que \(\mathfrak{m}/\mathfrak{i}\subseteq{\mathfrak{j}/\mathfrak{i}}.\)
Então, \(\mathfrak{j}/\mathfrak{i} = \mathfrak{m}/\mathfrak{i}\) ou \(\mathfrak{j}/\mathfrak{i} = A/\mathfrak{i}\), isto é,
 \(\mathfrak{j} = \mathfrak{m}\) ou \(\mathfrak{j} = A\). Portanto, \(\mathfrak{m} \) é maximal. \qedsymbol
\end{proof*}
\newpage
 
\section{Aula 06 - 28/08/2023}
\subsection{Motivações}
\begin{itemize}
  \item Relação Entre Anel e Quociente;
  \item Primeiro Teorema do Isomorfismo;
  \item Segundo Teorema do Isomorfismo.
\end{itemize}
\subsection{Relação entre Anéis e seus Quocientes}
 \begin{prop*}
  Seja \(\mathfrak{i}\trianglelefteq{A}.\)
 \begin{itemize}
  \item[1)] \(\mathfrak{i}\) é primo se, e somente se, \(A/\mathfrak{i}\) é domínio;
  \item[2)] \(\mathfrak{i}\) é maximal se, e somente se, \(A/\mathfrak{i}\) é corpo.
 \end{itemize}
 \end{prop*}
 \begin{proof*}
  1. \(\Rightarrow ):\) Sejam \(a+\mathfrak{i}, b+\mathfrak{i}\in A/\mathfrak{i}\) tais que \((a+\mathfrak{i})(b+\mathfrak{i})=\mathfrak{i}=0_{A/\mathfrak{i}}.\) Assim,
 \(ab+\mathfrak{i} = \mathfrak{i}\) e \(ab\in \mathfrak{i}.\) como \(\mathfrak{i}\) é primo, \(a\in \mathfrak{i}\) ou \(b\in \mathfrak{i}\), o que implica que
 \(a+\mathfrak{i} = \mathfrak{i} = 0_{A/\mathfrak{i}}\) ou \(b+\mathfrak{i} = \mathfrak{i} = 0_{A/\mathfrak{i}}.\) Logo, \(A/\mathfrak{i}\) é domínio.

 \(\Leftarrow ):\) Suponha agora que \(A/\mathfrak{i}\) é domínio. Tome \(ab\in \mathfrak{i}\). Assim, \(ab + \mathfrak{i} = (a+\mathfrak{i})(b+\mathfrak{i}) = \mathfrak{i}.\)
Como \(A/\mathfrak{i}\) é domínio, \(a+\mathfrak{i} = \mathfrak{i}\) ou \(b+\mathfrak{i} = \mathfrak{i}\), i.e., \(a\in \mathfrak{i}\) ou \(b\in \mathfrak{i}.\) Destarte, \(\mathfrak{i}\) é primo.

  2. \(\Rightarrow ):\) Suponha que \(\mathfrak{i}\) é maximal e seja \(\mathfrak{k}\trianglelefteq{A/\mathfrak{i}}.\) Pela proposição anterior,
 \(\mathfrak{k} = \mathfrak{j}/\mathfrak{i}\) para algum \(\mathfrak{i}\subseteq \mathfrak{j}\trianglelefteq{A}.\) Como \(\mathfrak{i}\) é maximal,
 \(\mathfrak{i}=\mathfrak{j}\) ou \(\mathfrak{j} = A.\) Com isso, \(\mathfrak{k} = \mathfrak{i}/\mathfrak{i} = (\mathfrak{i})\) ou \(\mathfrak{k} = A/\mathfrak{i}.\) 
 Visto que \(\mathfrak{k}\) é genérico, \(A/\mathfrak{i}\) é um corpo.

 \(\Leftarrow ):\) Assuma que \(A/\mathfrak{i}\) seja um corpo. Seja \(\mathfrak{j}\trianglelefteq{A}\) com \(\mathfrak{i}\mathfrak{j}\).
Considere o ideal no quociente \(\mathfrak{k} = \mathfrak{j}/\mathfrak{i}\trianglelefteq{A/\mathfrak{i}}\). Como \(A/\mathfrak{i}\) é um corpo,
segue que \(\mathfrak{k} = (\mathfrak{i})\) ou \(\mathfrak{k} = A/\mathfrak{i}.\) Assim, \(\mathfrak{j} = \mathfrak{i}\) ou \(\mathfrak{j} = A.\)
Portanto, \(\mathfrak{i}\) é maximal. \qedsymbol
 \end{proof*}
 \begin{example}[Exercício]
  Seja p um número primo. Mostre que:
 \begin{itemize}
  \item[1)] \(\mathbb{Z}[1/p]\cong \mathbb{Z}[x]/(\langle px-1 \rangle)\);
  \item[2)] \(\mathbb{Z}[\sqrt[]{p}]\cong \mathbb{Z}[x]/(\langle x^{2}-p \rangle)\);
  \item[3)] Se \(\mathfrak{m}\) é o ideal maximal do anel local \(\mathbb{Z}_{(p)},\) então \(\mathbb{Z}_{(p)}/\mathfrak{m}\cong{\mathbb{Z}/p \mathbb{Z}}.\)
 \end{itemize}
\end{example}
\begin{prop*}[Exercício]
  Seja \(\mathfrak{i}\trianglelefteq{A}\) e \(\mathfrak{i}[x]\subseteq A[x]\) o conjunto de polinômios 
com coeficientes em \(\mathfrak{i}.\) Mostre que:
 \begin{itemize}
  \item[1)] \(\mathfrak{i}[x] \trianglelefteq{A[x]}\);
  \item[2)] \(A[x]/\mathfrak{i}[x]\cong{(A/\mathfrak{i})}[x];\)
  \item[3)] \(\mathfrak{i}\) é primo se, e somente se, \(\mathfrak{i}[x]\) é primo.
 \end{itemize}
\end{prop*}
\begin{prop*}[Exercício]
  Se \(\mathfrak{m}\in \mathrm{Specm}(A),\) mostre que \(A/\mathfrak{m}^{n}\) é um anel local para todo
 \(n \geq 1\), com ideal maximal \(\mathfrak{m}/\mathfrak{m}^{n}.\)
\end{prop*}
\begin{example}[Exercício]
  Mostre que \(\mathbb{R}[x]/(\langle x^{2}+1 \rangle)\cong{\mathbb{C}}, \mathbb{Q}[x]/(\langle x^{2}+1 \rangle)\cong{\mathbb{Q}[i]}\)
e \(\mathbb{Z}[x]/(\langle x^{2}+1 \rangle)\cong{\mathbb{Z}[i]}.\)
\end{example}
\begin{prop*}[Exercício]
  Seja k um corpo e seja \(\mathfrak{m} = \{x_{1}-a_{1}, \dotsc, x_{n}-a_{n}\}\trianglelefteq{k[x_{1}, \dotsc, x_{n}]}\) para
 \(a_{i}\in k.\) Mostre que \(k[x_{1}, \dotsc, x_{n}]/\mathfrak{m}\cong{k}\), tal que \(\mathfrak{m}\) é maximal. 
\end{prop*}
\subsection{Teoremas do Isomorfismo}
  Enunciaremos a seguir dois resultados de extrema importância para a álgebra como um todo.
 \hypertarget{first_isomorphism}{
  \begin{theorem*}[Primeiro Teorema do Isomorfismo]
  Seja \(f:A\rightarrow B\) um homomorfismo de anéis. O mapa \(\overline{f}:A/\ker{(f)}\rightarrow B\)
é um morfismo injetor de anéis e \(A/\ker{(f)}\cong{\mathrm{Im}(f)}\)
 \end{theorem*}}
  \hypertarget{second_isomorphism}{
  \begin{theorem*}[Segundo Teorema do Isomorfismo]
  Seja \(\mathfrak{i}\trianglelefteq{A}.\) Se \(\mathfrak{j}\subseteq \mathfrak{i}\) e \(\mathfrak{j}/\mathfrak{i}\trianglelefteq{A/\mathfrak{i}},\) então:
  \[
    \frac{A/\mathfrak{i}}{\mathfrak{j}/\mathfrak{i}}\cong{\frac{A}{\mathfrak{j}}}
  \]
 \end{theorem*}}
\begin{proof*}
  \textbf{\underline{Primeiro Teorema}:} Denotaremos por \(\mathfrak{i}\) o kernel \(\ker{(f)}.\)
A priori, precisamos checar se \(\overline{f}\) está bem definida. De fato, se \(a+\mathfrak{i}=b+\mathfrak{i}, a-b\in \mathfrak{i}\) e, assim
 \(0 = f(a-b) = f(a)-f(b)\), tal que \(f(a) = f(b).\) Assim, \(\overline{f}(a+\mathfrak{i}) = f(a) = f(b) = \overline{f}(b+\mathfrak{i}).\)

  Agora, conferiremos que \(\overline{f}\) é um morfismo. Com efeito, sejam \(a+\mathfrak{i}, b+\mathfrak{i}\),
  \[
    \overline{f}(a+b+\mathfrak{i}) = f(a+b) = f(a)+f(b) = f(a+\mathfrak{i})+f(b+\mathfrak{i}).
  \]
  A mesma conta prova isto para o produto:
  \[
    \overline{f}(ab+\mathfrak{i}) = f(ab) = f(a)f(b) = f(a+\mathfrak{i})f(b+\mathfrak{i}).
  \]

  A posteriori, provemos a injetividade e o isomorfismo. Para o primeiro, observe que, se \(\overline{f}(a+\mathfrak{i}) = 0\), então
\(f(a) = 0\) e \(a\in\ker{(f)}\). Assim, \(a+\ker{(f)} = \ker{(f)} = 0_{A/\mathfrak{i}}\). Por fim,
para ver que \(\overline{f}\) é sobrejetora com a imagem (e, portanto, que há o isomorfismo com a imagem),
seja \(b\in \mathrm{Im}(f).\) Assim, existe \(a\in A\) tal que \(f(a) = b\). Tome \(\overline{f}(a+\mathfrak{i}) = f(a)=b.\)

  Portanto, segue o teorema.

  \textbf{\underline{Segundo Teorema}:} Defina o mapa \(\theta :A/\mathfrak{i}\rightarrow A/\mathfrak{j},\) dado por \(\theta(a+\mathfrak{i}) = a + \mathfrak{j}.\)
Deste modo, note que \(\theta \) está bem-definido, pois, se \(a+\mathfrak{i} = b+\mathfrak{i}\), então \(a-b\in \mathfrak{i}\). Como
 \(\mathfrak{i}\subseteq \mathfrak{j}, a-b\in \mathfrak{j}.\) Destarte, \(a+\mathfrak{j} = b+\mathfrak{j}\) e \(\theta(a+\mathfrak{i}) = \theta (b+\mathfrak{i}).\)

  A partir disto, faz sentido checarmos se \(\theta \) é epimorfismo injetor da forma desejada no enunciado. A parte do morfismo segue da conta 
  \[
    \theta (a+b+\mathfrak{i}) = a + b + \mathfrak{j} = (a+\mathfrak{j}) + (b+\mathfrak{j}).
  \]
  A sobrejetividade vem de qualquer elemento \(a+\mathfrak{j}\in A/\mathfrak{j}\) ser escrito por um correspondente \(a+\mathfrak{i}\in A/\mathfrak{i}\).
Finalmente, a injetividade para o isomorfismo desejado segue ao provarmos que \(\ker{(\theta )}=\mathfrak{j}/\mathfrak{i}.\) Com efeito,
 \(a+\mathfrak{i}\in\ker{(\theta )}\) equivale a dizer que \(a\in \mathfrak{j}\) e \(a+\mathfrak{i}\in \mathfrak{j}/\mathfrak{i}\).

  Portanto, podemos usar o primeiro teorema do isomorfismo e obtemos o resultado desejado. \qedsymbol
\end{proof*}
\begin{prop*}[Exercício]
  Sejam \(f:A\rightarrow B\) um morfismo de anéis e \(\mathfrak{i}\trianglelefteq{A}.\)
\begin{itemize}
  \item[1)] Se \(\mathfrak{i}\subseteq \ker{(f)}\), mostre que \(\overline{f}:A/\mathfrak{i}\rightarrow B,\) definido por \(\overline{f}(a+\mathfrak{i}) = f(a)\) é 
um morfismo bem-definido;
  \item[2)] Se \(\mathfrak{j}\trianglelefteq{B}\), mostre que \(\overline{f}:A/f^{-1}(\mathfrak{j})\rightarrow B/\mathfrak{j},\) definido por
\(\overline{f}(a+f^{-1}(\mathfrak{j})) = f(a) + \mathfrak{j}\) é um morfismo bem definido e injetor;
  \item[3)] Se \(\mathfrak{j}\trianglelefteq{B}\) e \(\mathfrak{i}\trianglelefteq{A}\) tal que \(\mathfrak{i}\subseteq f^{-1}(\mathfrak{j}),\) então
 \(\overline{f}:A/\mathfrak{i}\rightarrow B/\mathfrak{j}, \) definido por \(\overline{f}(a+\mathfrak{i})=f(a)+\mathfrak{j}\),
é um morfismo bem definido e \(\ker{(\overline{f})} = f^{-1}(\mathfrak{j})/\mathfrak{i}.\)
\end{itemize}
\end{prop*}
\subsubsection{EXTRA: O Terceiro e o Quarto Teoremas do Isomorfismo}
\newpage

\section{Aula 07 - 30/08/2023}
\subsection{Motivações}
\begin{itemize}
  \item O Teorema Chinês dos Restos
\end{itemize}
\subsection{Teorema Chinês dos Restos}
  Antes de mais nada, relembramos a noção de ideais coprimos:
 \begin{quote}
  ``Dois ideais \(\mathfrak{i}, \mathfrak{j}\trianglelefteq{A}\) são ditos \textbf{coprimos} se \(\mathfrak{i}+\mathfrak{j} = A,\) ou seja,
se \(1\in \mathfrak{i}+\mathfrak{j}.\)''
 \end{quote}
\begin{example}
 \begin{itemize}
  \item[1)] Se m, n são números inteiros tais que \(\mathrm{mdc}(m, n) = 1,\) então \(m \mathbb{Z} + n \mathbb{Z} = \mathbb{Z}\).
  \item[2)] Seja k um corpo e \(f(x), g(x)\in k[X].\) Se \(\mathrm{mdc}(f(x), g(x)) = 1\), i.e.,
existe \(h(x)\in k[X]\) com \(\deg(h(x))\geq 1\) tal que \(h(x)\mid f(x)\) e \(h(x)\mid g(x)\), então os ideais
 \(\langle f(x) \rangle, \langle g(x) \rangle\) são coprimos.
  \item[3)] Seja A anel tal que \(\mathfrak{m}_{1}, \mathfrak{m}_{2}\in \mathrm{Specm}(A), \mathfrak{m}_{1}\neq \mathfrak{m}_{2}.\) Então,
 \(\mathfrak{m}_{1} + \mathfrak{m}_{2} \trianglelefteq{A},\) mas \(\mathfrak{m}_1\subseteq \mathfrak{m}_{1}+\mathfrak{m}_{2}.\) 
 Como \(\mathfrak{m}_{1}\) é maximal e \(\mathfrak{m}_{1}\neq \mathfrak{m}_{2},\) resta a opção \(\mathfrak{m}_{1} + \mathfrak{m}_{2} = A.\)
 \end{itemize}
\end{example}
\begin{prop*}
  Sejam \(\mathfrak{i}, \mathfrak{j}\trianglelefteq{A}\) com \(\mathfrak{i}+\mathfrak{j} = A.\) Então, para todo \(n\geq 1, m\geq 1,\) os ideais
 \(\mathfrak{i}^{m}\) e \(\mathfrak{j}^{n}\) são coprimos.
\end{prop*}
\begin{proof*}
  Se \(\mathfrak{i}, \mathfrak{j}\) são coprimos, existem \(x\in \mathfrak{i}, y\in \mathfrak{j}\) com \(x+y = 1.\)
Assim, \(1 = 1^{m+n} = (x+y)^{m+n}.\) Expandindo o binômio, obtemos
\begin{align*}
  &1 = \sum\limits_{k=0}^{m+n}\binom{m+n}{k}x^{k}y^{m+n-k}\\
  &1 = \sum\limits_{k=0}^{m}\binom{m+n}{k}x^{k}y^{m+n-k} + \sum\limits_{k=m+1}^{m+n}\binom{m+n}{k}x^{k}y^{m+n-k}\\
  &1 = \biggl(\sum\limits_{k=0}^{m}\binom{m+n}{k}x^{k}y^{m-k}\biggr)y^{n}+\biggl(\sum\limits_{k=m+1}^{m+n}\binom{m+n}{k}x^{k-m}y^{m+n-k}\biggr)x^{m.}
\end{align*}
  Assim, escrevemos \(1 = ay^{n}+bx^{m}\), isto é, como a soma de elementos de \(\mathfrak{i}^{m}\) e \(\mathfrak{j}^{n}.\) Portanto,
 \(\mathfrak{i}^{m}\) e \(\mathfrak{j}^{n}\) são coprimos. \qedsymbol
\end{proof*}
 \hypertarget{chinese_remainder}{
\begin{theorem*}[Teorema Chinês dos Restos]
  Seja A um anel e \(\mathfrak{i}_{1},\dotsc,\mathfrak{i}_{n}\trianglelefteq{A}\) ideais de A dois a dois coprimos, ou seja,
 \(\mathfrak{i}_{i}+\mathfrak{i}_{j}=A, i\neq j.\) Então:
\begin{itemize}
  \item[1)] \(\mathfrak{i}_{1}\cdot \dotsc \cdot \mathfrak{i}_{n} = \mathfrak{i}_{1}\cap \dotsc\cap \mathfrak{i}_{n};\)
  \item[2)] Existe um isomorfismo 
 \begin{align*}
  &\varphi :\frac{A}{\bigcap_{i=1}^{n}\mathfrak{i}_{i}}\rightarrow \frac{A}{\mathfrak{i}_{1}}\times \dotsc\times \frac{A}{\mathfrak{i}_{n}}\\
  &a+\bigcap_{i=1}^{n}\mathfrak{i}_{i}\mapsto(a+\mathfrak{i}_{1}, \dotsc, a+\mathfrak{i}_{n}).
 \end{align*}
\end{itemize} 
\end{theorem*}}
\begin{proof*}
  Provaremos os dois fatos juntos utilizando a indução a partir do primeiro caso n=2. 

  Já sabemos que 1 vale para este caso, então iremos encontrar o morfismo. Para tanto, considere \(\varphi :A/(\mathfrak{i}_{1}\cap \mathfrak{i}_{2})\rightarrow A/\mathfrak{i}_{1}\times A/\mathfrak{i}_{2}.\)
Observe que \(\varphi \) está bem-definido, dado que, se \(a+\mathfrak{i}_{1}\cap \mathfrak{i}_{2} = b + \mathfrak{i}_{1}\cap \mathfrak{i}_{2}, a-b\in \mathfrak{i}_{1}\cap \mathfrak{i}_2\), 
então \(a+\mathfrak{i}_{1} = b + \mathfrak{i}_{1}\) e \(a + \mathfrak{i}_{2} = b + \mathfrak{i}_{2}.\) Então, \(\varphi(a+\mathfrak{i}_{1}\cap \mathfrak{i}_{2}) = \varphi(b+\mathfrak{i}_{1}\cap \mathfrak{i}_{2}).\)
Resta provar a injetividade, que é homomorfismo e a sobrejetividade. Faremos nesta ordem

  Suponha \(a+\mathfrak{i}_{1}\cap \mathfrak{i}_{2}\in\ker{(\varphi )}\). Então, \(\varphi (a + \mathfrak{i}_{1}\cap \mathfrak{i}_{2}) = (\mathfrak{i}_{1}, \mathfrak{i}_{2})\),
isto é, \(a\in \mathfrak{i}_{1}\) e \(a\in \mathfrak{i}_{2}.\) Assim, \(a\in \mathfrak{i}_{1}\cap \mathfrak{i}_{2}\), tal que \(\ker{(\varphi )} = \{0_{A/(\mathfrak{i}_{1}\cap \mathfrak{i}_{2})\} = \mathfrak{i}_{1}\cap \mathfrak{i}_{2}}\)
e \(\varphi \) é injetora.

  Além disso, \(\varphi (a + b + \mathfrak{i}_{1}\cap \mathfrak{i}_{2}) = (a+b+\mathfrak{i}_{1}, a + b + \mathfrak{i}_{2}) =
(a + \mathfrak{i}_{1}, a + \mathfrak{i}_{2}) + (b+\mathfrak{i}_{1}, b + \mathfrak{i}_{2}) = \varphi(a + \mathfrak{i}_{1}\cap \mathfrak{i}_{2}) + \varphi(b + \mathfrak{i}_{1}\cap \mathfrak{i}_{2}).\)
Além disso, \(\varphi (ab + \mathfrak{i}_{1}\cap \mathfrak{i}_{2}) = (ab + \mathfrak{i}_{1}, ab+ \mathfrak{i}_{2}) = (a+\mathfrak{i}_{1}, a+\mathfrak{i}_{2}).\)

  Finalmente, para ver que \(\varphi \) é sobrejetora, tome um elemento genérico \((a+\mathfrak{i}_{1}, b+\mathfrak{i}_{2})\in A/\mathfrak{i}_{1}\times A/\mathfrak{i}_{2}.\)
Note que essa \(\varphi \) só atinge elementos da forma \((z + \mathfrak{i}_{1}, z + \mathfrak{i}_{2})\) e que 
\begin{align*}
  (a+\mathfrak{i}_{1}, b+\mathfrak{i}_{2}) &= (a+\mathfrak{i}_{1}, \mathfrak{i}_{2}) + (\mathfrak{i}_{1}, b + \mathfrak{i}_{2})\\
                                           &= (a+\mathfrak{i}_{1}, a + \mathfrak{i}_{2})(1+\mathfrak{i}_{1}, 0 + \mathfrak{i}_{2}) + (b+\mathfrak{i}_{1}, b+\mathfrak{i}_{2})(0+\mathfrak{i}_{1}, 1 + \mathfrak{i}_{2})\\
                                           &= \varphi(a)(1+\mathfrak{i}_{1}, 0 + \mathfrak{i}_{2}) + \varphi(b)(0+\mathfrak{i}_{1}, 1 + \mathfrak{i}_{2}).
\end{align*}
  Com isso, resolvemos nosso problema se encontrarmos \(x_{1}, x_{2}\) tais que \(\varphi (x_{1}+\mathfrak{i}_{1}\cap \mathfrak{i}_{2}) = (0 +\mathfrak{i}_{1}, 1 +\mathfrak{i}_{2})\) e 
 \(\varphi (x_{2} + \mathfrak{i}_{1}\cap \mathfrak{i}_{2}) = (1+\mathfrak{i}_{1}, 0+\mathfrak{i}_{2}).\) Considere
 \(x_{1}\in \mathfrak{i}_{1}, x_{2}\in \mathfrak{i}_{2}\) tais que \(x_{1} + x_{2} = 1.\) Temos:
\begin{align*}
  \varphi (x_{1}+\mathfrak{i}_{1}\cap \mathfrak{i}_{2}) &= (x_{1}+\mathfrak{i}_{1}, x_{1}+\mathfrak{i}_{2})\\
                                                        &= (\mathfrak{i}_{1}, 1 - x_{2} + \mathfrak{i}_{2})\\
                                                        &= (\mathfrak{i}_{1}, 1+\mathfrak{i}_{2}).
\end{align*}
  Usando o mesmo raciocínio, 
 \begin{align*}
   \varphi (x_{2} + \mathfrak{i}_{1}\cap \mathfrak{i}_{2}) &= (x_{2}+\mathfrak{i}_{1}, x_{2}+\mathfrak{i}_{2})\\
                                                           &= (1-x_{1}+\mathfrak{i}_{1}, \mathfrak{i}_{2})\\
                                                           &= (\mathfrak{i}_{1}, 1 + \mathfrak{i}_{2}).
 \end{align*}
  Temos todos os elementos que precisávamos. Basta tomarmos \(\varphi (bx_{1} + ax_{2} + \mathfrak{i}_{1}\cap \mathfrak{i}_{2}) = (a+\mathfrak{i}_{1}, b+\mathfrak{i}_{2})\) e, pelos resultados anteriores,
\(\varphi \) é um isomorfismo de anéis.

  Agora, utilizaremos este caso base para conseguir a hipótese indutiva. Suponha que o resultado vale para \(n-1\) e provemos para n.
Sejam \(\mathfrak{i}_{1}, \dotsc, \mathfrak{i}_{n}\) ideais de A dois a dois coprimos. Primeiro, note que,
se \(\mathfrak{j}\coloneqq \mathfrak{i}_{2}\cdot \dotsc \cdot \mathfrak{i}_{n}\), \(\mathfrak{i}_{1}\) e \(\mathfrak{j}\) são coprimos.

  De fato, por hipótese, vale \(\mathfrak{i}_{1}+\mathfrak{i}_{i} = A\) para \(i=2, \dotsc, n.\) Então, existem
 \(x_{1, i}\in \mathfrak{i}_{1}\) e \(y_{i}\in \mathfrak{i}_{i}\) tais que \(x_{1, i}+y_{i} = 1.\) Assim, o produto:
  \[
    1 = (x_{1, 2} + y_{2})(x_{1, 3}+y_{3})\dotsc(x_{1, n}+y_{n}) = y_{2}\cdot \dotsc \cdot y_{n}+z,
  \]
  em que z é o resto da expansão do produto e está em \(\mathfrak{i}_{1},\) porque será uma soma de produtos de termos \(x_{1, i}\in \mathfrak{i}_{1}\) com alguns 
termos \(y_{},\) que estão no anel, e \(\mathfrak{i}_{1}\) é ideal. Como \(y_{2}\cdot \dotsc \cdot y_{n}\in \mathfrak{j},\) segue que  \(\mathfrak{i}_{1} + \mathfrak{j} = A.\)
Supondo, por indução, que \(\mathfrak{i}_{2}\cdot \dotsc \cdot \mathfrak{i}_{n} = \mathfrak{i}_{2}\cap \dotsc \cap \mathfrak{i}_{n},\) donde tiramos
 \(\mathfrak{i}_{1}\cdot (\mathfrak{i}_{2}\cdot \dotsc \cdot \mathfrak{i}_{n}) = \mathfrak{i}_{1}\cap (\mathfrak{i}_{2}\cap \dotsc\cap \mathfrak{i}_{n})\) pelo caso \(n=2\).
 Ademais, para a segunda parte do teorema, note que 
\begin{align*}
 \frac{A}{\mathfrak{i}_{1}\cap(\bigcap_{i=2}^{n}\mathfrak{i}_{i})} = \frac{A}{\mathfrak{i}_{1}\cap \prod\limits_{i=2}^{n}\mathfrak{i}_{i}}\cong \frac{A}{\mathfrak{i}_{1}}\times \frac{A}{\prod\limits_{i=2}^{n}\mathfrak{i}_{i}}&=\frac{A}{\mathfrak{i}_{1}}\times \frac{A}{\bigcap_{i=2}^{n}\mathfrak{i}_{i}}\\  
                                                                                                                                                                                                                                 &\cong \frac{A}{\mathfrak{i}_{1}}\times \dotsc\times \frac{A}{\mathfrak{i}_{i}}.
\end{align*}
  Portanto, o isomorfismo definido existe. \qedsymbol
\end{proof*}
\begin{crl*}
  Sejam \(x_{1}, \dotsc, x_{r}\in \mathbb{Z}\) e \(n_{1}, \dotsc, n_{r}\in \mathbb{N}\) dois a dois coprimos. Então,
o sistema 
\begin{equation*}
    \left\{\begin{array}{ll}
      x\equiv x_{1}\mod n_{1}\\
      \vdots\\
      x\equiv x_{r}\mod n_{r}
    \end{array}\right.
\end{equation*}
  possui uma única solução mod \(n_{1}\cdot \dotsc n_{r}.\)
\end{crl*}
\begin{example}
  Seja \(f(x)\in \mathbb{C}[x].\) Pelo teorema fundamental da álgebra, podemos escrever \(f(x) = a(x-a_{1})^{r_{1}}\cdot \dotsc \cdot (x-a_{r})^{n_{r}},\)
em que \(a_{i}\) são raízes distintas de ordem \(r_{1}, \dotsc, r_{n}\) respectivamente, com \(\deg{(g)} = \sum\limits_{}^{}r_{i}.\) Temos
 \(\langle x-a_{i} \rangle \trianglelefteq{\mathbb{C}[x]}\) é ideal maximal para \(i=1, \dotsc, n.\) Além disso, se \(i\neq j\), os ideais
 \(\langle x-a_{i} \rangle^{r_{i}}\) e \(\langle x-a_{j} \rangle^{r_{j}}\) são coprimos. Aplicando o Teorema Chinês dos Restos, obtemos o isomorfismo:
  \[
    \frac{\mathbb{C}[x]}{\langle f(x) \rangle}\cong \frac{\mathbb{C}[x]}{\langle x-a_{1} \rangle^{r_{1}}}\times \dotsc\times \frac{\mathbb{C}[x]}{\langle x-a_{n} \rangle^{r_{n}}}.    
  \]
  Se temos o caso especial em que f(x) é irredutível, isto é, \(r_{1} = \dotsc = r_{n} = 1\), então:
  \[
    \frac{\mathbb{C}[x]}{\langle x-a_{i} \rangle}\cong{\mathbb{C}},
  \]
  e, assim, 
  \[
    \frac{\mathbb{C}[x]}{\langle f(x) \rangle}\cong{\mathbb{C}^{n}}.
  \]
\end{example}
\begin{example}[Exercícios]
  Sejam A, B anéis. Mostre que:
 \begin{itemize}
  \item[1)] \(\mathrm{Specm}(A\times B) = \{A\times \mathfrak{j}:\mathfrak{j}\in \mathrm{Specm}(B)\}\cup\{\mathfrak{i}\times B:\mathfrak{i}\in \mathrm{Specm}(A)\}.\)
  \item[2)] Encontre todos os ideais maximais de \(A_{1}\times \dotsc\times A_{n}\), em que cada
 \(A_{i}\) é um anel.
  \item[3)] Prove que \(\mathrm{Specm}(\mathbb{C}^{n}) = \mathrm{Spec}(\mathbb{C}^{n}).\) Anéis que satisfazem essa propriedade
são chamados \textbf{anéis artinianos.}
 \end{itemize}
\end{example}
\begin{def*}
  Um anel A é dito \textbf{semilocal} se \(\mathrm{Specm}(A)\) é finito. \(\square\)
\end{def*}
\begin{prop*}[Exercício]
  Mostre que para todo polinômio \(f(x)\in \mathbb{C}[x]\) tal que f(x) é não-constante e irredutível, o anel
 \(\mathbb{C}[x]/\langle f(x) \rangle\) é semilocal.
\end{prop*}
\newpage

\section{Aula 08 - 11/09/2023}
\subsection{Motivações}
\begin{itemize}
  \item Anéis Multiplicativos;
  \item Anel e mapa de Localização.
\end{itemize}
\subsection{Anéis Multiplicativos}
\begin{def*}
  Seja A um anel. Um subconjunto \(S\subseteq A\) é dito \textbf{multiplicativo} se:
\begin{itemize}
  \item[1)] \(1\in S;\)
  \item[2)] S é fechado pela multiplicação, i.e., \(s, s'\in S\) implica que \(s \cdot s'\in S.\)
\end{itemize}
\end{def*}
\begin{example}
\begin{itemize}
  \item[1)] Se \(a\in A,\) com \(a\neq 0,\) o conjunto \(S = \{a^{i}: i\geq 0\}\) é multiplicativo;
  \item[2)] Se \(\mathfrak{p}\in \mathrm{Spec}(A),\) então \(S\coloneqq A\setminus{\mathfrak{p}}\) é multiplicativo. De fato, como
 \(\mathfrak{p}\vartriangleleft A, 1\not\in \mathfrak{p}\) e \(1\in A\setminus{\mathfrak{p}}.\)
 Além disso, S é fechado pelo produto. De fato, suponha, por hipótese de absurdo, que existam \(s, s'\in S\) tais que \(s, s'\not\in S.\) Então,
 \(s \cdot s'\in \mathfrak{p}\) e, como \(\mathfrak{p}\) é primo, segue que \(s\in \mathfrak{p}\) ou \(s'\in \mathfrak{p}.\) Mas, então, \(s\not\in S\) ou
 \(s'\not\in S,\) o que é um absurdo. Logo, S é multiplicativo. Em particular, se A for um domínio, temos \((0)\in \mathrm{Spec}(A)\) e, assim,
 \(S=A\setminus{\{0\}}\) é multiplicativo.
  \item[3)] Se \(\mathfrak{p}_{1}, \dotsc , \mathfrak{p}_{n}\in \mathrm{Spec}(A)\) então o conjunto:
  \[
    S = A\setminus{\bigcup_{i=1}^{n}\mathfrak{p}_{i}}
  \]
  é multiplicativo.
  \item[4)] Se \(a, b\in A \setminus{\{0\}},\) então \(S = \{a^{i}\cdot b^{j}: i, j\geq 0\}\) é multiplicativo (o mesmo vale para número finito de
elementos do anel).
  \item[5)] Se \(f:A\rightarrow B\) é um morfismo de anéis e \(S\subseteq A\) e \(S\cap \ker{(f)} = \emptyset\) é um conjunto multiplicativo,
então \(f(S)\) é multiplicativo.

  De fato, como \(1_{A}\in S\) e f é morfismo, \(f(1_{A}) = 1_{B}\in f(S).\) Se \(x, y\in f(S),\) existem
 \(a, b\in S\) tais que \(x=f(a)\) e \(y=f(b).\) Como S é multiplicativo, \(a \cdot b\in S\) e, então, \(f(a \cdot b) = f(a)f(b)\in S.\) 
  \item[6)] O conjunto das unidades \(S = A^{*}\subseteq A\) é multiplicativo.
\end{itemize}
\end{example}
  Seja A um domínio e \(S\subseteq A\) um conjunto multiplicativo com \(0\not\in S.\) Definiremos uma relação de equivalência no conjunto
 \(A\times S:\) Se \((a, s), (a', s')\in A\times S\), então defina 
  \[
    (a, s)\sim (a', s') \Longleftrightarrow as' = sa'.
  \]
  Provemos que esta relação é de equivalência. É reflexiva, já que, para todo
 \((a, s)\in A\times S, as = sa,\) ou seja, \((a, s)\sim (a, s).\) É simétrica, pois,
 se \((a, s)\sim (a'\Sigma s')\) vale \(as' = sa'\) e, então, \(a's = s'a, \) por comutatividade. 
 Logo, \((a', s')\sim (a, s).\) A transitividade segue do fato de que, se \((a, b), (c, d), (x, y)\in A\times S\) são
 tais que \((a, b)\sim (c, d)\) e \((c, d)\sim (x, y)\), então valem \(ad = bc\) e \(cy = dx.\) Queremos provar
 que \((a, b)\sim (x, y)\). No entanto, note que:
  \[
    bcy = bcy \Rightarrow (da)y = b(xd) \Rightarrow ay = bx.
  \]
  Na última linha, foi utilizada a propriedade dos cancelamentos dos domínios para podermos cancelar d. De fato, se \(d\in S, d\neq 0.\)
Vale a seguinte propriedade: Se \(Ad = Bd,\) vale \(a = B.\) Caso contrário, teríamos \(A\neq B\) e \(A - B\neq 0,\) mas \(Ad = Bd\) implica que
 \(d(A-B) = 0\) e então d seria um divisor de zero, o que é um absurdo.

\begin{def*}
  A classe de equivalência de um elemento \((a, s)\in A\times S\) será denotada por 
  \[
    \frac{a}{s} = \{(a', s')\in A\times S: (a, s)\sim (a', s')\}.\quad\square
  \] 
  Denotaremos o conjunto de todas as classes \(S^{-1}A\) o conjunto de todas essas classes de equivalência. 
\end{def*}
  Podemos caracterizar as classes fazendo:
  \[
    \frac{a}{s} = \frac{a'}{s'} \Longleftrightarrow as'=a's.
  \]
  Em \(S^{-1}A\), podemos definir as operações soma e multiplicação como segue:
  \[
    \frac{a}{s}+\frac{b}{t}\coloneqq \frac{at+bs}{st}\quad\&\quad \frac{a}{s}\frac{b}{t}\coloneqq \frac{ab}{st}.
  \]
  Precisamos mostrar que essas operações estão bem definidas. Quanto à soma,
\begin{align*}
  \frac{a}{s} + \frac{b}{t} = \frac{a'}{s'} + \frac{b'}{t'} &\Longleftrightarrow \frac{at+bs}{st} = \frac{a't' + b's'}{s't'}\\
                                                            &\Longleftrightarrow (at+bs)(s't') = (a't'+b's')(st)\\
                                                            &\Longleftrightarrow as't t'+bt'ss' = a'st t' + b'tss'.
\end{align*}
  Note que \((as')t t' = (a's)t t'\) e \((bt')ss' = (b't)ss',\) o que faz com que valam as igualdades acima.

  Agora, para o produto, 
\begin{align*}
  \frac{a}{s}\frac{b}{t} = \frac{a'}{s'}\frac{b'}{t'} &\Longleftrightarrow \frac{ab}{st}=\frac{a'b'}{s't'}\\
                                                      &\Longleftrightarrow as'bt' = a'sb't. 
\end{align*}
  Portanto, as operações estão bem-definidas.
\begin{prop*}
  O conjunto \(S^{-1}A\) com as operações definidas acima é um anel comutativo e com unidade.
\end{prop*}
\begin{proof*}
  Note que a classe \(0/s = 0/1\) para todo \(s\in S\). Além disso, se \(a/s\in S^{-1}A,\) temos:
  \[
    \frac{a}{s}+\frac{0}{1} = \frac{a1 + 0s}{s1} = \frac{a}{s}.
  \]  
  Então, \(0/1\) é o elemento neutro da soma neste conjunto. Dado um elemento genérico \(a/s\in S^{-1}A,\) podemos tomar seu inverso \((-a)/s:\)
  \[
    \frac{a}{s}+\frac{-a}{s} = \frac{as + (-a)s}{s^{2}} = \frac{0}{s^{2}} = \frac{0}{1}.
  \]
  A associatividade fica como exercício, assim como a distributiva. A unidade \(1_{S^{-1}A} = 1/1 = s/s \) para todo \(s\in S\). De fato,
 \(a/b \cdot s/s = as/bs = a/b\). A comutatividade fica como exercício também. \qedsymbol
\end{proof*}
\begin{lemma*}
  O mapa \(\rho :A\rightarrow S^{-1}A,\) definido por \(\rho (a) = a/1\) é um morfismo de anéis e é injetor, chamado \textbf{mapa de localização}.
\end{lemma*}
\begin{proof*}
  Vamos provar, primeiro, que é morfismo. Se \(a, b\in A,\) então \(\rho (ab) = ab/1 = a/1 \cdot b/1 = \rho (a)\rho (b)\) e \(\rho (a+b) = (a1+b1)/1 = a/1 + b/1 = \rho (a) + \rho (b).\)
Além disso, \(\rho (1) = 1/1 = 1_{S^{-1}A}.\)

  Agora, se \(a\in\ker{(\rho )},\) segue que \(\rho (a) = a/1 = 0/1.\) Isso implica que \(a1 = 0\) e, então, \(a = 0.\) Portanto,
 \(\rho \) é um morfismo injetor. \qedsymbol
\end{proof*}
\begin{prop*}
  Prove que \(S^{-1}A\) é um domínio e calcule \(S^{-1}A\).  
\end{prop*}
\begin{proof*}
  Se \(a/s, b/t\in S^{-1}A\) tais que \(a/s \cdot b/t = 0_{S^{-1}A}\), temos \(ab/st = 0/1\) e, assim,
 \(ab = 0.\) Como A é domínio, \(a = 0\) ou \(b = 0\). Portanto,
  \[
    a/s = 0/1\quad\&\quad b/t = 0/1.\quad \text{\qedsymbol}
  \]
\end{proof*}
\newpage

\section{Aula 09 - 13/09/2023}
\subsection{Motivações}
\begin{itemize}
  \item Caracterizações da Localização;
  \item Corpo das Frações.
\end{itemize}
\subsection{Caracterizando a Localização}
\begin{prop*}
  Sejam A um domínio, S multiplicativo e \((S^{-1}A, \rho )\) localização de A por S.
\begin{itemize}
  \item[1)] Se \(\mathfrak{i}\trianglelefteq{A},\) então \(S^{-1}\mathfrak{i} = \{a/s: a\in \mathfrak{i}, s\in S\}\trianglelefteq{S^{-1}A};\) 
  \item[2)] Se \(\mathfrak{j}\trianglelefteq{S^{-1}A}\), existe \(\mathfrak{i}\trianglelefteq{A}\) tal que \(\mathfrak{j} = S^{-1}\mathfrak{i};\)
  \item[3)] Se \(\mathfrak{i}\trianglelefteq{A}\), então \(\mathfrak{i}\cap S \neq\emptyset\) se, e somente se, \(S^{-1}\mathfrak{i} = S^{-1}A;\)
  \item[4)] Vale que \(\mathrm{Spec}(S^{-1}A) = \{S^{-1}\mathfrak{p}:\mathfrak{p}\in \mathrm{Spec}(A) \text{ e } S\cap \mathfrak{p}=\emptyset\}\). Além disso, se considerarmos A como subanel de \(S^{-1}A\),
então \(S^{-1}\mathfrak{p}\cap A = \mathfrak{p}\) para todo \(\mathfrak{p}\in \mathrm{Spec}(A).\)
\end{itemize}
\end{prop*}
\begin{proof*}
  1.) Seja \(a/s\in S^{-1}A\) e \(b/t\in S^{-1}\mathfrak{i}.\) Assim, \(a\in A\) e \(b\in \mathfrak{i}\) implicam
 que \(ab\in \mathfrak{i},\) e S ser fechado por multiplicação implica que \(st\in S.\) Então, \(a/s \cdot b/t = ab/st\in S^{-1}\mathfrak{i}.\)
 Agora, considere \(b/t, b'/t'\in S^{-1}\mathfrak{i}.\) Como \(t, t'\in S\subseteq A\) e \(\mathfrak{i}\) é ideal, \(bt'+b't\in \mathfrak{i}\) e,
 com isso, \(b/t + b'/t' = (bt'+b't)/t t'\in S^{-1}\mathfrak{i}.\)

  2.) Seja \(\mathfrak{j}\trianglelefteq{S^{-1}A}.\) Tome \(\mathfrak{i} = \rho^{-1}(\mathfrak{j})\trianglelefteq{A}.\) Observe que o
 \(\mathfrak{i} = \{a\in A:\exists s\in S, a/s\in \mathfrak{j}\}\). Seja \(a/s\in S^{-1}\mathfrak{i},\) e então
 \(a\in \mathfrak{i}\) e \(\rho (a)\in \mathfrak{j}.\) Já que \(a/s = 1/s \cdot a/1, 1/s\in S^{-1}A\) e \(a/1\in \mathfrak{j}\),
 junto do fato de \(\mathfrak{j}\) ser ideal, conclui-se que \(a/s\in \mathfrak{j}.\)

 Por outro lado, considere \(x=a/s\in \mathfrak{j}.\) Como \(s/1\in S^{-1}A, a/1 = s/1 \cdot a/s\in \mathfrak{j}.\) Logo, 
como \(\rho (a) = a/1,\) temos \(a\in \mathfrak{i}\) e \(a/s\in S^{-1}\mathfrak{i}.\)

  3.) Suponha, primeiro, que \(S\cap \mathfrak{i} \neq\emptyset\) e seja \(s\in S\cap \mathfrak{i}.\) Então,
 \(1 = s/s\in S^{-1}\mathfrak{i}.\) Já que \(S^{-1}\mathfrak{i}\) é ideal e contém 1, \(S^{-1}\mathfrak{i} = S^{-1}A.\)

 Agora, suponha que \(S^{-1}\mathfrak{i} = S^{-1}A.\) Segue que \(s/s\in S^{-1}\mathfrak{i},\) o que implica que \(s\in \mathfrak{i}\) para algum
 \(s\in S\) ou \(1\in \mathfrak{i}.\) Portanto, \(1\in S\), e o resultado segue.

  4.) Se \(\mathfrak{p}\in \mathrm{Spec}(A)\) e \(\mathfrak{p}\cap S = \emptyset,\) então \(S^{-1}\mathfrak{p}\vartriangleleft S^{-1}A.\)
Considere elementos \(a/s, b/t\in S^{-1}A\) tais que \(a/s \cdot b/t\in S^{-1}\mathfrak{p}.\) Então, \(ab/st\in S^{-1}\mathfrak{p}\) e \(ab\in \mathfrak{p}.\)
Utilizando que \(\mathfrak{p}\) é primo, \(a\in \mathfrak{p}\) ou \(b\in \mathfrak{p}\) implica que \(a/s\in S^{-1}\mathfrak{p}\) ou \(b/t\in S^{-1}\mathfrak{p}.\)
Logo, \(S^{-1}\mathfrak{p}\in \mathrm{Spec}(S^{-1}A).\)

  Por outro lado, suponha que \(\mathfrak{q}\in \mathrm{Spec}(S^{-1}A).\) tome \(\mathfrak{p}\coloneqq \rho^{-1}(\mathfrak{q})\) e
sabemos que, como a imagem inversa de um morfismo de um ideal primo é um ideal primo, o item 2 fornece \(\mathfrak{q} = S^{-1}\mathfrak{p}.\) 
Quanto à última parte da afirmação, considere \(\rho :A\hookrightarrow S^{-1}A\) e então \(S^{-1}\mathfrak{p}\cap A = \rho^{-1}(S^{-1}\mathfrak{p}) = \mathfrak{p} \trianglelefteq{A}.\) \qedsymbol
\end{proof*}
\begin{prop*}[Exercício]
\begin{itemize}
  \item[1)] Se \(f:A\rightarrow B\) é morfismo de domínios, \(S\subseteq A\) tal que \(S\cap\ker{(f)} = \emptyset\), então
 \(f(S)\subseteq B\) é multiplicativo sem o 0, no modelo em que precisamos para realizar a localização. Além disso, \(f|_S:S\rightarrow S\) 
 é injetivo.
  \item[2)] Se \(\mathfrak{p}\in \mathrm{Spec}(A)\) e \(\mathfrak{p}\cap S = \emptyset,\) então \(\overline{S} = \pi (S)\) é multiplicativo sem o 0, com
morfismo quociente. Além disso, os morfismos
\begin{align*}
  &\varphi :\frac{S^{-1}A}{S^{-1}\mathfrak{p}}\rightarrow \overline{S}^{-1}\biggl(\frac{A}{\mathfrak{p}}\biggr)\\
  &\frac{a}{s}+S^{-1}\mathfrak{p}\mapsto \frac{\pi (a)}{\pi (s)}.
\end{align*}
e 
\begin{align*}
  &\psi:\overline{S}^{-1}\biggl(\frac{A}{\mathfrak{p}}\biggr)\rightarrow \frac{S^{-1}A}{S^{-1}\mathfrak{p}}\\
  &\frac{\pi (a)}{\pi (s)}\mapsto \frac{a}{s} + S^{-1}\mathfrak{p}
\end{align*}
estão bem definidos e são um o inverso do outro, ou seja, obtivemos um isomorfismo 
  \[
    \frac{S^{-1}A}{S^{-1}\mathfrak{p}}\cong{\overline{S}^{-1}\biggl(\frac{A}{\mathfrak{p}}\biggr)}.
  \]
\end{itemize}  
\end{prop*}
\begin{example}
\begin{itemize}
  \item[1)] Se \(t\in A, t\neq0,\) temos o conjunto multiplicativo \(S=\{1, t, t^{2}, \dotsc\}\).
Denotaremos \(S^{-1}A\coloneqq A[1/t] = \{a/t^{i}: a\in A \text{ e } i\geq 0\}\). Além disso, 
 \(\mathrm{Spec}(A[1/t]) = \{S^{-1}\mathfrak{p}:\mathfrak{p}\in \mathrm{Spec}(A) \text{ e }t\not\in \mathfrak{p}\}.\)
  \item[2)] Se \(t_{1}, \dotsc, t_{n}\in A\setminus{\{0\}},\) podemos considerar o conjunto multiplicativo \(S = \{t_{1}^{r_{1}}\cdot \dotsc \cdot t_{n}^{r_{n}}: r_{i}\geq 0\}.\)
Como \textbf{exercício}, encontre todos os ideais primos de \(S^{-1}A.\)
  \item[3)] Se \(\mathfrak{p}\in \mathrm{Spec}(A),\) considere \(S = A\setminus{\mathfrak{p}}.\) Neste caso, denotaremos 
 \(S^{-1}A\coloneqq A_{\mathfrak{p}} = \{a/s: a\in A, s\in A\setminus{\mathfrak{p}}.\}\) Em particular, se \(\mathfrak{p}=(0), A_{(0)}\)
é um corpo. De fato, se \(a/s\in A_{(0)},\) com \(a/s\neq0/1,\) temos \(a\neq0\) e \(s\neq0.\) Com isso, o elemento
 \(s/a\in A_{(0)}\) e é o inverso multiplicativo de \(a/s\). Chamamos \(A_{(0)}\) de \textbf{corpo de frações} de A,
denotado por \(\mathrm{Frac}(A).\) Em geral, se S é multiplicativo, \(S^{-1}A\hookrightarrow \mathrm{Frac}(A).\)
\end{itemize}
\end{example}
\begin{prop*}[Exercício]
 \begin{itemize}
  \item[1)] Se \(S\subseteq A^{*},\) mostre que o morfismo localização \(\rho :A\hookrightarrow S^{-1}A\) é um isomorfismo;
  \item[2)] Se A é corpo, mostre que \(\mathrm{Frac}(A)\cong{A;}\)
  \item[3)] Observe que \(\mathrm{Frac}(A)\) é o menor corpo que contém A;
 \end{itemize}  
\end{prop*}
\begin{example}[Exercício]
 \begin{itemize}
  \item[1)] Mostre que \(\mathbb{Z}\) não possui subanéis próprios;
  \item[2)] Seja K um corpo. Se K não possui subcorpo próprio, mostre que \(K\cong{\mathbb{Q}}\) ou
 \(K\cong{\mathbb{Z}/p \mathbb{Z}}\) para algum primo p.
 \end{itemize}  
\end{example}
\begin{prop*}
  Seja \(\mathfrak{p}\in \mathrm{Spec}(A)\). Então, o domínio \(A_{\mathfrak{p}}\) é um anel local
com ideal maximal \(\mathfrak{p}A_{\mathfrak{p}}\coloneqq \{a/s: a\in \mathfrak{p}\text{ e }s\not\in \mathfrak{p}\}.\) Além disso,
  \[
    \frac{A_{\mathfrak{p}}}{\mathfrak{p}A_{\mathfrak{p}}}\cong{\mathrm{Frac}\biggl(\frac{A}{\mathfrak{p}}\biggr).}
  \]
  Em particular, se \(\mathfrak{m}\in \mathrm{Specm}(A),\) então:
  \[
    \frac{A_{\mathfrak{m}}}{\mathfrak{m}A_{\mathfrak{m}}}\cong{\mathrm{Frac}\biggl(\frac{A}{\mathfrak{m}}\biggr)} \cong{\frac{A}{\mathfrak{m}}}
  \]
\end{prop*}
\begin{proof*}
  Observe que \(A_{\mathfrak{p}}=S^{-1}A,\) com \(S=A\setminus{\mathfrak{p}}.\) É claro que \(\mathfrak{p}A_{\mathfrak{p}} = S^{-1}\mathfrak{p}\), e então
 \(\mathfrak{p}A_{\mathfrak{p}}\in \mathrm{Spec}(A_{\mathfrak{p}}).\) Provemos que este ideal é maximal e é o único.

  Seja \(\mathfrak{p}a_{\mathfrak{p}}\subsetneq \mathfrak{j}\trianglelefteq{A_{\mathfrak{p}}}. \) Então, \(\mathfrak{j} = S^{-1}\mathfrak{i}\) para algum \(\mathfrak{i}\trianglelefteq{A}\) e 
\(\mathfrak{p}A_{\mathfrak{p}} = S^{-1}\mathfrak{p}\subsetneq S^{-1}\mathfrak{i}\subseteq S^{-1}A \), tal que \(\mathfrak{p}\subsetneq \mathfrak{i}\subseteq A. \)
Como \(\mathfrak{p}\neq \mathfrak{i}, \mathfrak{i}\cap A\setminus{\mathfrak{p}} = \mathfrak{i}\cap S \neq\emptyset\). Logo,
 \(S^{-1}\mathfrak{i} = \mathfrak{j} = A_{\mathfrak{p}}.\) Assim, provamos que \(\mathfrak{p}A_{\mathfrak{p}}\) é maximal em \(S^{-1}A.\) 

 Agora, considere \(\mathfrak{k}\in \mathrm{Specm}(S^{-1}A)\). Então, \(\mathfrak{k} = S^{-1}\mathfrak{i}\) para algum ideal 
 \(\mathfrak{i}\trianglelefteq{A}.\) Suponha que \(\mathfrak{k}\neq \mathfrak{p}A_{\mathfrak{p}}.\) Então, \(\mathfrak{i}\neq \mathfrak{p}\) e
 \(\mathfrak{i}\cap A\setminus{\mathfrak{p}}\neq\emptyset.\) Com isso, \(S^{-1}\mathfrak{i} = S^{-1}A = A_{\mathfrak{p}},\) o que é um absurdo com a
 suposição de que \(\mathfrak{k}\) é maximal e, então, próprio.

  Como \(\mathfrak{p}A_{\mathfrak{p}}\in \mathrm{Spec}(A_{\mathfrak{p}}), A_{\mathfrak{p}}/\mathfrak{p}A_{\mathfrak{p}}\) é um corpo. Além disso,
note que os mapas 
\begin{align*}
  &\rho :\frac{A_{\mathfrak{p}}}{\mathfrak{p}A_{\mathfrak{p}}}\rightarrow \mathrm{Frac}\biggl(\frac{A}{\mathfrak{p}}\biggr)\\
  &\frac{a}{s} + \mathfrak{p}A_{\mathfrak{p}}\mapsto \frac{\pi (a)}{\pi (s)}
\end{align*}
e
\begin{align*}
  &\psi:\mathrm{Frac}\biggl(\frac{A}{\mathfrak{p}}\biggr)\rightarrow \frac{A_{\mathfrak{p}}}{\mathfrak{p}A_{\mathfrak{p}}}\\
  &\frac{\pi (a)}{\pi (s)}\mapsto \frac{a}{s} + \mathfrak{p}A_{\mathfrak{p}}
\end{align*}
estão bem definido e são um o inverso do outro, provando o teorema (fica de exercício mostrar isso). \qedsymbol
\end{proof*}
\begin{prop*}[Exercício]
  Sejam \(\mathfrak{p}_{1}, \dotsc, \mathfrak{p}_{n}\in \mathrm{Spec}(A)\) tais que \(\mathfrak{p}_{i}\subsetneq \mathfrak{p}_{j} \) para todo
 \(i\neq j\). Se \(S = A\setminus{\bigcup_{}^{}\mathfrak{p}_{i}}\), então \(S^{-1}A\) é um anel semilocal, com \(\mathrm{Specm}(S^{-1}A) = \{S^{-1}\mathfrak{p}_{1}, \dotsc, S^{-1}\mathfrak{p}_{n}\}\).
\end{prop*}
\newpage

\section{Aula 10 - 18/09/2023}
\subsection{Motivações}
\begin{itemize}
\item Polinômios em Anéis;
\item Algoritmo de Divisão;
\item Exemplos.
\end{itemize}
\subsection{Anéis de Polinômios}
\begin{def*}
  Seja A um anel. O \textbf{anel de polinômios com coeficientes em A}, \(A[x]\), é definido como 
  \[
    A[x]\coloneqq \{a_{n}x^{n}+\cdots+a_{1}x + a_{0}: a_{i}\in A, i = 1, \cdots, n\},
  \]
em que x é chamado de variável. \(\square\)
\end{def*}
  Nestes estudos, assumiremos sempre algumas coisas:
 \begin{itemize}
   \item[i)] \(ax = xa\) para todo a em A;
   \item[ii)] \(a_{n}x^{n} + \cdots + a_{0} = 0\) se, e somente se, \(a_{i} = 0\) para todo \(i=1, \cdots, n.\)
   \item[iii)] \(a_{n}x^{n} + \cdots + a_{1}x + a_{0} = b_{m}x^{m} + \cdots + b_{1}x + b_{0}\) se, e somente se, \(n=m\) e \(a_{i} = b_{i}\) para todo \(i=1, \cdots, n\).
 \end{itemize}

\begin{def*}
  Seja \(f(x) = a_{n}x^{n} + \cdots + a_{0}\in A[x]\). Definimos
  \begin{itemize}
   \item[1)] Se \(a_{n}\neq0, a_{n}\) é chamado o \textbf{coeficiente líder} de f(x). Denotamos o coeficiente líder
  de f(x) por \(\ell_{f}\) ou \(\ell_{f(x)}\)
  \item[2)] Se \(a_{n}\neq0\), n é chamado o \textbf{grau} de f(x), denotado por \(\deg(f(x))\)
  \item[3)] \(a_{0}\) é chamado o \textbf{coeficiente constante} de f(x)
  \item[4)] Diz-se que \(\alpha\in A\) é uma raiz de \(f(x)\in A[x]\) se \(f(\alpha ) = 0.\quad\square\)
  \end{itemize}
\end{def*}
  Algumas propriedades simples que temos são:
\begin{itemize}
  \item[i)] \(\deg{(f(x) + g(x))}\leq \max\{\deg{f(x)}, \deg{g(x)}\}\). A igualdade ocorre se, e somente se,
   \(\deg{f(x)}\neq \deg{g(x)}\) ou \(\deg{f(x)}=\deg{g(x)},\) mas \(\ell_{f} + \ell_{g}\neq0.\)
  \item[2)] \(\deg{(f(x)\cdot g(x))}\leq \deg{f(x)} + \deg{g(x)}.\) A igualdade ocorre se, e somente se,
 \(\ell_{f}\cdot \ell_{g} \neq0\)
\end{itemize}
  Fica como um bom exercício prová-las.
 \begin{example}
   Se A é um domínio, então 
  \[
    \deg{(f \cdot g)} = \deg{f} + \deg{g},
  \]
para todo \(f, g\neq0.\)
 \end{example}
 \begin{example}
  Em \(\biggl(\mathbb{Z}/4\biggr)[x],\) temos 
  \[
    2x \cdot 2x = 0,
  \]
tal que \(\deg{2x \cdot 2x} = 0 < 2 = \deg{2x} + \deg{2x}.\)
 \end{example}
\begin{def*}
  Um polinômio \(f(x)\in A[x]\) é dito \textbf{mônico} se \(\ell_{f} = 1.\) Em outras palavras, 
  \[
    f(x) = x^{n} + a_{n-1}x^{n-1} + \cdots + a_{0}.
  \]
Um polinômio mônico \(f(x)\) é dito \textbf{irredutível} se não existem polinômios mônicos
 \(g(x), h(x)\) tais que 
 \[
   f(x) = g(x)\cdot h(x).\square
 \]
\end{def*}
\begin{example}
  Em \(\mathbb{R}[x]\), o polinômio mônico \(x^{2} + 1\) é irredutível. De fato, suponha que existam 
 \(h(x), g(x)\) tais que 
  \[
    x^{2} + 1 = h(x)g(x).
  \]
  Então,
  \[
    x^{2} + 1 = h(x)g(x) = (x-a)(x-b) = x^{2} - (a+b)x +ab.
  \]
  Podemos reescrever isso em forma de sistema de equações 
  \[
   \left\{\begin{array}{ll}
      a + b = 0\\
      ab = 1
    \end{array}\right. 
    \Rightarrow  
    \left\{\begin{array}{ll}
        a = -b\\
        a(-a) = 1
    \end{array}\right.
    \Rightarrow 
    a^{2} = -1,
  \]
  mas isso não é possível para coeficientes reais. Portanto, não existe tal redução de \(x^{2} + 1.\)
\end{example}
   Se \(f(x)\in \mathbb{R}[x]\) é irredutível, então 
  \[
    1\leq \deg{f(x)}\leq 2.
  \]
  Suponha, por outro lado, que \(f(x)\in \mathbb{R}[x]\) tem grau n, isto é, \(\deg{f(x)} = n.\) Então, \(f(x)\) tem n raízes complexas,
digamos \(\alpha_{1}, \cdots, \alpha_{n}\). Assim, 
  \[
    f(x) = (x-\alpha_{1})\cdots(x-\alpha_{n}).
  \]
  Vamos separar essas raízes nas puramente reais e nas puramente complexas. Sejam \(\alpha_{1}, \cdots, \alpha_{h}\in \mathbb{R}\) e
 \(\alpha_{h+1}, \cdots, \alpha_{n}\in \mathbb{C}\setminus{\mathbb{R}}.\) Com isso, 
  \[
    f(x) = g(x)h(x),
  \]
em que \(g(x) = (x-\alpha_{1})\cdots(x-\alpha_{h})\) e \(h(x) = (x-\alpha_{h+1})\cdots(x-\alpha_{n}).\)
Se \(z = a + ib\in \mathbb{C}\) é uma raiz de \(h(x),\) então \(\overline{z} = a - ib\) também é uma raiz de h, ou seja, 
  \[
    (x-z)(x-\overline{z})\mid h(x).
  \]
  Se \(h(x) = (x-z)(x-\overline{z})h_{1}(x) = (x^{2}-2a + (a^{2}+b^{2}))h_{1}(x)\), o termo 
  \(x^{2}-2a + (a^{2}+b^{2})\) é irredutível em \(\mathbb{R}[x]\). Continuando assim, podemos ver que 
  \[
    f(x) = (x-a)\cdots(x-a_{n})(x^{2}+a_{h+1}x + b_{h+1})(x^{2}+a_{h+2}x+b_{h+2})\cdots(x^{2}+a_{n}x + b_{n}),\quad n = h = 2m.
  \]
  Uma consequência deste raciocínio é que um polinômio complexo \(f(x)\in \mathbb{C}[x]\) é irredutível se, e somente se, \(\deg{f(x)} = 1.\)
 \begin{example}
   Em \(\mathbb{Q}[x]\), os polinômios irredutíveis podem ter qualquer grau. Polinômios irredutíveis em \(\mathbb{Q}[x]\)
são, por exemplo, 
\begin{align*}
  &x-a,\quad a\in \mathbb{Q}\\
  &x^{2}-2ax + (a^{2}+b^{2}),\quad a, b\in \mathbb{Q}\\
  &x^{2} - p,\quad p\in \mathbb{Z}\text{ livre de quadrados}\\
  &x^{n} - 2.
\end{align*}
 \end{example}
\begin{def*}
  Dados \(f, g\in A[x]\), dizemos que \textbf{f divide g,} denotado \(f\mid g\), se existe \(h\in A[x]\) tal que 
    \[
      g(x) = f(x)h(x).\quad\square
    \]
\end{def*}
\begin{lemma*}
  Se \(f(x)\in A[x]\) é mônico, então existem polinômios mônicos irredutíveis \(f_{1}, \cdots, f_{m}\in A[x]\) tais que 
    \[
      f(x) = f_{1}(x)^{r_{1}}\cdot \cdots \cdot f_{m}(x)^{r_{m}}.
    \]
\end{lemma*}
\begin{proof*}
  Exercício.
\end{proof*}
  Agora, vejamos o \textbf{Algoritmo da Divisão:}
 \begin{theorem*}
  Sejam \(f(x), g(x)\in A[x], \deg{g(x)}\geq 1\) e \(\ell_{g}\in A^{*}.\) Então,
  existem polinômios únicos \(q(x)\) e \(r(x)\) em \(A[x]\) satisfazendo 
  \[
    f(x) = q(x)g(x) + r(x),
  \]
com \(\deg{r(x)} < \deg{g(x)}.\)
 \end{theorem*}
 \begin{proof*}
  A prova é por indução no grau de f.

  Se \(\deg{f}=0,\) f é constante. Como \(\deg{g}\geq 1, 0 = \deg{f} < \deg{g},\) g não é constante.
  Tome \(q(x) = 0\) e \(r(x) = f(x).\) Então, \(f(x) = 0g(x) + f(x)\) e \(\deg{r} < \deg{g}.\)

  Suponha agora que o resultado seja válido para polinômios de grau menor que n e sejam \(f(x) = \sum\limits_{i=0}^{n+1}a_{i}x^{i}, g(x) = \sum\limits_{i=0}^{m}b_{i}x^{i}, b_{m}\in A^{*}\).
Se \(\deg{f(x)} < \deg{g(x)},\) basta copiarmos o caso em que f era constante, colocando \(q(x) = 0\) e \(r(x) = f(x)\).
Assim, podemos assumir que \(\deg{f(x)}\geq \deg{g(x)}.\) Neste caso, o polinômio 
  \[
    f_{1}(x)\coloneqq f(x) - a_{n+1}b_{m}^{-1}x^{n+1-m}g(x) = c_{n}x^{n} + \cdots + c_{0}
  \] 
  é de grau menor ou igual que n. Aplicando a hipótese de indução, existem \(q_{1}(x), r_{1}(x)\) 
com \(\deg{r_{1}(x)} < \deg{g(x)},\) de forma que \(f_{1} = gq_{1} + r_{1}.\) Temos:
\begin{align*}
  f(x) &= f(x)-a_{n}b_{m}^{1}x^{n+1-m}g(x) = q_{1}(x)g(x) + r_{1}(x)\\
       &= (a_{n}b_{m}^{-1}x^{n+1-m}+q_{1}(x))g(x) + r_{1}(x).
\end{align*}
  Definindo \(q(x) = a_{n}b_{m}^{-1}x^{n+1-m}+q_{1}(x), r = r_{1},\) temos o resultado. Pela hipótese
de indução, ele vale para todo \(n\in \mathbb{N},\) restando apenas provar que os polinômios são únicos.

Destarte, suponha que existem \(q_{1}, r_{1}, q_{2}, r_{2}\in A[x]\) tais que 
  \[
    f(x) = q_{1}g+r_{1} = q_{2}g+r_{2},
  \]
com \(\deg{r_{1}}, \deg{r_{2}} < \deg{g}.\) Temos \(r_{1}-r_{2} = (q_{2}-q_{1})g.\) Suponha que \(q_{2}\neq q_{1},\)]
de modo que \(q_{2}-q_{1}\neq0\) e 
  \[
    \deg{(r_{1}-r_{2})} = \deg{(q_{2}-q_{1})}g = \deg{(q_{2}-q_{1})} + \deg{g},
  \]
em que a igualdade do produto dos graus ocorre pois, se \(f, g\in A[x]\) e o coeficiente líder de g é \(b\in A^{*}\),
então \(\deg{fg} = \deg{f} + \deg{g}.\) De fato, \(f(x)g(x) = a_{n}b_{m}x^{n+m} + \cdots + a_{0}b_{0}\) com
 \(a_{n}b_{m}\neq0\), já que, se \(a_{n}b_{m} = 0, a_{n} = 0.\) Assim, o grau do produto é \(n+m\).

  Logo, da igualdade, segue que \(\deg{(r_{1}-r_{2})} = \deg{(q_{2}-q_{1})g}\geq \deg{g}.\) Como
 \(\deg{(r_{1}-r_{2})}\leq \max\{\deg{r_{1}}, \deg{r_{2}}\}\leq \deg{g},\) temos um absurdo. Portanto,
  \(q_{2} = q_{1}.\) \qedsymbol
 \end{proof*}
 \begin{example}
   Em \(\mathbb{Z}[x],\) considere \(f(x) = 2x^{3} + x^{2} + x + 1\)  e \(g(x) = -x + 1.\)
Podemos aplicar o algoritmo da divisão da seguinte forma:
\begin{align*}
  &2x^{3} + x^{2} + x + 1\quad \text{(Polinômio original)}\\
  &2x^{3} + x^{2} + x + 1 \underbrace{- 2x^{3} + 2x^{2}}_{-2x^{2}g(x)} \quad (-2x^{2}g(x)\text{ pois é o que falta para g ``alcançar'' o grau de f)}\\
  &3x^{2} + x + 1 - 3x^{2} \underbrace{- 3x^{2} + 3x}_{-3xg(x)} \quad (\text{Novamente, faltava multiplicar g(x) por } -3x \text{ a fim de cancelar).}\\
  &4x + 1 \underbrace{- 4x + 4}_{-4g(x)} \quad\text{(Mesma coisa, multiplicamos g(x) para cancelar o } 4x)\\
  &5\quad\text{(Finaliza-se com um polinômio de grau 0 (constante)).}
\end{align*}
  Coletamos os termos que não foram usados para cancelar os maiores graus, ou seja, \(2x^{2}, 3x\) e \(4\), multiplicamos eles por -1 e, assim, o algoritmo da divisão nos dá 
    \[
      f(x) = g(x)(-2x^{2} -3x -4) + 5 = (-x+1)(-2x^{2}-3x-4)+5.
    \]
\end{example}
\newpage

\section{Aula 11 - 20/09/2023}
\subsection{Motivações} 
\begin{itemize}
  \item Corolários do Algoritmo da Divisão;
  \item Decomposição de polinômios.
\end{itemize}
\subsection{Corolários do Algoritmo da Divisão}
 \begin{prop*}
  Se F é um corpo, então \(F[x]\) é um Domínio de Ideais Principais.
 \end{prop*}
 \begin{proof*}
  Seja \(\mathfrak{i} \trianglelefteq{F[x]}\) e seja 
  \[
    n = \min\{\deg{f(x)}: f(x)\in \mathfrak{i}\setminus{\{0\}}\}.
  \]
  Se \(n=0,\) então existe um polinômio constante \(f(x) = a\neq 0, a\in F.\) Logo, 
  \[
    1 = \frac{1}{a}\cdot a = \frac{1}{a}f(x)\in \mathfrak{I}.
  \]
  Assim, 
  \[
    I = F[x] = \langle 1 \rangle.
  \]
  Então, podemos assumir que \(n > 0\). Seja \(g(x)\in \mathfrak{i}\) com \(\deg{g} = n\) e observe que
 \(\deg{g}\geq 1\) e \(\ell_{g}\in F^{*}\). Se \(f(x)\in \mathfrak{i},\) pelo algoritmo da divisão, existem
  \(g(x), r(x)\in F[x]\) tais que \(f(x) = q(x)g(x) + r(x),\) com \(\deg{r(x)} < \deg{g(x)}.\) Se \(r(x)\neq0, r(x) = f(x) 
- q(x)g(x)\in \mathfrak{i}\) e, junto com \(\deg{r} < \deg{g},\) obtemos um absurdo, pois g deveria ser o polinômio de
grau mínimo.

  Logo, \(r(x) = 0\) e, assim, \(f(x) = q(x)g(x)\) para algum polinômio \(q(x)\in F[x]\), tal que \(f(x)\in \langle g(x) \rangle.\) 
Portanto, \(\mathfrak{i} = \langle g(x) \rangle.\) \qedsymbol
 \end{proof*}
 \begin{example}[Exercício]
 \begin{itemize}
   \item[1)] \(\mathbb{Z}[x]\) não é D.I.P: O ideal gerado \(\mathfrak{i} = \langle 2, x \rangle\) não é principal, mas é primo.
   \item[2)] Mostre que \(\langle 3, x^{2}-2 \rangle \trianglelefteq{\mathbb{Z}[x]}\) é ideal maximal e não é principal. (Dica:
mostre que 
  \[
    \frac{\mathbb{Z}[x]}{\langle 3, x^{2}-2 \rangle}\cong{\frac{\mathbb{Z}_{3}[x]}{\langle x^{2}-2 \rangle}}.
  \]
 \end{itemize} 
\end{example}
\begin{crl*}
  Seja F um corpo e \(f, g, h\in F[x]\) tais que f(x) é mônico e irredutível.
Se \(f(x)\mid g(x)h(x),\) então \(f(x)\mid g(x)\) ou \(f(x)\mid h(x).\)
\end{crl*}
\begin{proof*}
  Faremos indução no grau de f.
  
  Se \(\deg{f(x)} = 1,\) como f é mônico, \(f(x) = x-a\) para algum a em F. Por hipótese, existe \(p(x)\) tal que
 \(g(x)h(x) = (x-a)p(x).\) Aplicando em \(x=a,\) segue que \(g(a)h(a) = 0.\) Como f é domínio, é preciso que
  \(g(a) = 0\) ou \(h(a) = 0\). Em outras palavras, \((x-a)\mid g(x)\) ou \((x-a)\mid h(x),\) como desejado.

  Suponha agora que o resultado vale grau menor que n, \(\deg{f} = n\) e \(f\mid g\) e \(f\mid h.\) Assim, existem 
 \(q, r, q', r'\in A[x]\) tais que \(g(x) = f(x)q(x) + r(x)\) e \(h(x) = q'(x)f(x) + r'(x),\) satisfazendo
   \(\deg{r'} < \deg{g'}, \deg{r} < \deg{g}\) e \(r, r'\neq 0.\) Assim,
\begin{align*}
  g(x)h(x) &= (qf + r)(q'f + r')\\
           &= qq'f^{2} + (qr' + q'r)f + rr'\\
  \Rightarrow & rr' = gh - qq'f^{2} - (qr' + rq')f.
\end{align*}
  Sabemos que f divide \(gh, qq'f^{2}\) e \((qr' + q'r)f.\) Com isso, \(f\mid rr'\) e, então,
existe \(p(x)\in F[x]\) tal que \(fp = rr'.\) Note também que \(\deg{p} < \deg{r} \) e \(\deg{r'}.\) De fato,
caso contrário, teríamos \(\deg{p}\geq \deg{r}\) e como \(\deg{f} > \deg{r'},\) temos 
  \[
    \deg{p}\geq \deg{r} \Rightarrow \deg{f} + \deg{p} > \deg{r'} + \deg{r} \Rightarrow \deg{fp} > \deg{rr'},
  \]
  o que é um absurdo. O mesmo vale para \(\deg{p}\geq r'.\) Escrevamos \(p = \alpha p_{1}^{\alpha_{1}}\cdot \cdots \cdot p_{r}^{\alpha_{r}},\) 
em que \(p_{1}, \cdots, p_{r}\) são irredutíveis (isso é possível pois F é corpo). Assim, \(p_{i}\mid fp = rr'.\) 
Como \(\deg{p} < \deg{r} < \deg{f}\) e \(\deg{p_{i}}\leq \deg{p},\) podemos aplicar a hipótese de indução em \(p_{i}\) e então
 \(p_{i}\mid r\) ou \(p_{i}\mid r'\) para cada \(i=1, \cdots, r\). Suponha, sem perda de generalidade, que \(p_{1}\mid r,\) tal que 
 \(r=p_{1}r_{1}\). Consequentemente, 
 \[
   fp = rr' \Rightarrow fp_{1}^{\alpha_{1}}\cdot \cdots \cdot p_{r}^{\alpha_{r}} = p_{1}rr' \Rightarrow fp_{1}^{\alpha_{1} -1}\cdot \cdots \cdot p_{r}^{\alpha_{r}} = r_{1}r.
 \]
 Pode-se continuar esse processo até sobrarem os termos \(p_{i}\)'s, chegando em
\(f = r_{t}r_{t}'\) com \(\deg{r_{t}}\geq 1\) e \(\deg{r_{t}'}\geq 1\) (pois \(\deg{p} < \deg{r}, \deg{r'}\)). Isso é uma contradição
com o fato de r ser irredutível.
\end{proof*}
\begin{example}[Exercício]
  Seja A um domínio, \(a\in A\) e \((x-a)\mid f(x)g(x).\) Mostre que \((x-a)\mid f(x)\) e \((x-a)\mid g(x).\) 
\end{example}
\begin{prop*}
  Seja F um corpo e \(f(x)\in F[x]\). Então, temos uma decomposição única 
  \[
    f(x) = af_{1}^{\alpha_{1}}(x)\cdot \cdots \cdot f_{r}^{\alpha_{r}}(x),
  \]
  com \(a\in F\) e \(f_{1}, \cdots, f_{r}\in F[x]\) polinômios irredutíveis e mônicos.
\end{prop*}
\begin{proof*}
  Já provamos a existência da decomposição para polinômios mônicos. Como F é corpo, existe \(b\in F\) tal que 
 \(bf(x)\) é mônico (basta tomarmos \(b=a^{-1},\) em que a é o coeficiente líder de f). Então, \(bf(x) = f_{1}^{\alpha_{1}}\cdot \cdots \cdot f_{r}^{\alpha_{r}}\)
e assim \(f(x) =af_{1}^{\alpha_{1}}\cdot \cdots \cdot f_{r}^{\alpha_{r}}.\)

  Provemos agora a unicidade. Seja 
  \[
    f(x)=af_{1}^{\alpha_{1}} \cdot \cdots \cdot f_{r}^{\alpha_{r}} = bg_{1}^{\beta_{1}} \cdot \cdots \cdot g_{s}^{\beta_{s}}.
  \]
  Mostraremos que \(\{f_{1}, \cdots, f_{r}\} = \{g_{1}, \cdots, g_{s}\}\) e \(\{\alpha_{1}, \cdots, \alpha_{r}\} = \{\beta_{1}, \cdots, \beta_{s}\}\).
Se \(\deg{f} = 1, f = a(x-\alpha ) = b(x-\beta )\) e \(ax -a\alpha =bx - b\beta,\) tal que \(a=b\) e \(\alpha =\beta .\)

  Agora, suponha que o resultado vale para grau menor que n e \(\deg{f} = n.\) Então: 
  \[
    f_{1}\mid f(x) \Rightarrow f_{1}\mid bg_{1}^{\beta_{1}}\cdot \cdots g_{s}^{\beta_{s}} \Rightarrow \exists j: f_{1}\mid g_{j}.
  \]
  Como \(g_{j}\) é irredutível e mônico, se \(g_{j}=f_{1}\), existe \(h_{j}\) tal que \(g_{j} = h_{j}f_{1}.\) Se \(\deg{(h_{j})}\geq 1, g_{j}\) não seria
irredutível e, então, \(h_{j} = a\in F.\) Mas, se \(a\neq1,\) como \(f_{1}\) é mônico, \(g_{j}\) não seria mônico. Logo, \(h_{j} = 1\) e \(g_{j} = f_{1}.\) 
Podemos reordenar tal que \(f_{1} = g_{1}. \) Assim, 
  \[
    af_{2}^{\alpha_{2}}\cdot \cdots \cdot f_{r}^{\alpha_{r}} = bg_{2}^{\beta_{2}} \cdot \cdots \cdot g_{s}^{\beta_{s}}.
  \]
  Portanto, pela hipótese indutiva, o resultado segue. \qedsymbol
\end{proof*}
\newpage

\section{Aula 12 - 25/09/2023}
\subsection{Motivações}
\begin{itemize}
  \item Corpos de Elementos Finitos;
  \item Característica de um Corpo.
\end{itemize}
\subsection{Corpos Finitos}
  Seja F um corpo. Definimos o homomorfismo de anéis
\begin{align*}
  &\varphi :\mathbb{Z}\rightarrow F\\
  &n\mapsto n \cdot 1_{F}\coloneqq \underbrace{1_{F} + \cdots + 1_{F}}_{\text{n-vezes}}.
\end{align*}
Já vimos que \(\overline{\varphi }:\frac{\mathbb{Z}}{\ker{(\varphi )}}\rightarrow F, \overline{n}\mapsto \varphi (n)=n \cdot 1_{F}\) é um monomorfismo. Como F é um
domínio, \(\mathbb{Z}/\ker{(\varphi )}\) é um domínio. Logo, \(\ker{(\varphi )}\in \mathrm{Spec}(\mathbb{Z}).\) Então, 
  \[
    \ker{(\varphi )} = (0)\quad\text{ou}\quad \ker{(\varphi )}=\langle p \rangle,\text{ p primo.}
  \]
  Olhando mais atentamente a estes casos, se \(\ker{(\varphi )} = (0),\) então \(\varphi \) é um monomorfismo. Agora, 
o mapa 
 \begin{align*}
   &\overline{\varphi }:\mathbb{Q}\rightarrow F\\
   &\frac{m}{n}\mapsto m \cdot 1_{F} \cdot (n \cdot 1_{F})^{-1} = m \cdot n^{-1}
 \end{align*}
é um homomorfismo de corpos que é injetivo.

  Por outro lado, se \(\ker{(\varphi )} = \langle p \rangle,\) p um primo, então 
  \[
    \overline{\varphi }:\mathbb{F}_{p}=\mathbb{Z}_{p}\hookrightarrow F,
  \]
logo \(\overline{\varphi }:\mathbb{F}_{p}\rightarrow F\) é um homomorfismo de corpos que é injetivo.
\begin{def*}
  Dizemos que um corpo F \textbf{é de característica} 0 se pudermos mergulhar \(\mathbb{Q}\) em F e diremos que \textbf{é de característica positiva} p, p
um primo, se pudemos mergulhar \(\mathbb{F}_{p} em \mathbb{F}.\) Neste caso, 
  \[
    \mathrm{char}{(F)} = \left\{\begin{array}{ll}
        0,\quad \mathbb{Q}\hookrightarrow F\\
        p,\quad \mathbb{F}_{p}\hookrightarrow \mathbb{F}_{p}.
      \end{array}\right.\quad\square
  \]
\end{def*}
  Algumas observações devem ser feitas.
 \begin{itemize}
   \item[1)] \(\mathbb{Q}\) e \(\mathbb{F}_{p}\) não têm subcorpos próprios. Por isso, são chamados corpos primos.
   \item[2)] Se F é um corpo finito, então \(\mathrm{char}(F) = p\) por um primo p.
   \item[3)] Se F é um subcorpo do corpo E, então \(\mathrm{char}(E) = \mathrm{char}(F).\)
 \end{itemize}
\begin{example}
 \begin{itemize}
  \item[1)] \(\mathrm{char}(\mathbb{Q}) = 0,\quad \mathrm{char}(\mathbb{R})=0 = \mathrm{char}(\mathbb{C})\)
  \item[2)] Seja \(\mathbb{Q}(\sqrt[]{2})\coloneqq \{a + b\sqrt[]{2}: a, b\in \mathbb{Q}\}\supseteq{\mathbb{Q}}.\) Segue que 
  \[
    \mathbb{Q}(\sqrt[]{2})\cong{\frac{\mathbb{Q}[x]}{\langle x^{2}-2 \rangle}},
  \]
em que \(x^{2}-2\) é irracional. Assim, o mapa \(\mathbb{Q}\hookrightarrow \mathbb{Q}[x]/\langle x^{2}-2 \rangle,\quad a\mapsto \overline{a}\) é vazio.
Com isso, tomando \(f(x)\in \mathbb{Q}[x]\) irracional e colocando \(F = \frac{\mathbb{Q}[x]}{\langle f(x) \rangle},\)
segue que \(\mathrm{char}(F) = 0.\)
  \item[3)] Coloque \(A = \mathbb{Q}[x]\) e
  \[
    F = Q(A)\coloneqq \biggl\{\frac{f(x)}{g(x)}: f, g\in \mathbb{Q}[x], g(x)\neq0\biggr\}.
  \]
  Segue que \(\mathbb{Q}\subseteq{F}\), tal que \(\mathrm{char}(F) = 0\). Em particular, 
  \[
    E = Q(\mathbb{Z}[x]) = \biggl\{\frac{f(x)}{g(x)}:f, g\in \mathbb{Z}[x], g(x)\neq0\biggr\} = F.
  \]
  \item[4)] Colocando \(K = Q(\mathbb{R}[x]),\) segue que \(\mathbb{Q} \subseteq{\mathbb{R}}\subseteq{K}\) e, logo, \(\mathrm{char}(K) = 0.\)
  \item[5)] Temos \(\mathrm{char}(\mathbb{F}_{p}) = p > 0.\) Além disso, colocando 
  \[
    \mathbb{F}_{p}[\sqrt[]{2}]\coloneqq \frac{\mathbb{F}_{p}[x]}{\langle x^{2}-2 \rangle},\quad p\neq2,
  \]
se chamarmos de \(F = \mathbb{F}_{3}[\sqrt[]{2}]\) e \(K = \mathbb{F}_{5}[\sqrt[]{5}],\) então
F é um corpo com \(\mathrm{char}(F) = 3\) e K é um corpo com \(\mathrm{char}(F) = 5.\) No entanto, para p = 7,
\(x^{2}-2\) tem uma raiz em \(\mathbb{F}_{7}\) e, logo, não é irracional. Portanto, \(\mathbb{F}_{7}[\sqrt[]{2}]\) não é um
corpo.
  \item[6)] Defina 
  \[
    \mathbb{Q}(\mathbb{F}_{p}[x])\coloneqq \biggl\{\frac{f}{g}: f, g\in \mathbb{F}_{p}[x], g\neq 0\biggr\}.
  \]
  Então, \(\mathbb{F}_{p}[x]\) é infinito, logo F é infinito. Apesar disso, \(\mathrm{char}(F) = p > 0.\)
 \end{itemize}
\end{example}
 Seja K um corpo finito tal que \(\mathrm{char}(K) = p > 0.\) Então, \(\mathbb{F}_{p}\hookrightarrow K\) e 
  \[
    n = [K:\mathbb{F}_{p}] = \dim_{\mathbb{F}_{p}}K < \infty.
  \]
  Isto implica que \(|K| = p^{n}\). Como \(\mathbb{F}_{p}\)-espaço vetorial, temos 
  \[
    K\cong{\mathbb{F}_{p}^{n}} = \underbrace{\mathbb{F}_{p}\times \cdots\times \mathbb{F}_{p}}_{\text{n-vezes}}
  \]
  Como um fato geral, se F é um corpo e olhamos para V como um F-espaço vetorial com dimensão \(n = \dim_{F}V < \infty\), então 
  \[
    V\cong{F^{n}}.
  \]
 \begin{theorem*}
\begin{itemize}
    \item[1)] Para todo primo p e todo número natural \(n > 0,\) temos um corpo com \(p^{n}\) elementos;
    \item[2)] Quaisquer dois corpos com \(p^{n}\) elementos são isomorfos.
\end{itemize}
 \end{theorem*}
\begin{def*}
  Denotamos um corpo finito com \(p^{n}\) elementos por \(\mathbb{F}_{p^{n}}.\square\)
\end{def*}
\begin{theorem*}
  Seja p um primo e \(m, n\in \mathbb{N}\) com \(m\leq n.\) Então, podemos mergulhar \(\mathbb{F}_{p^{m}}\) em \(\mathbb{F}_{p^{n}}\) se, e somente se, \(m\mid n.\)
\end{theorem*}
\newpage

\section{Aula 13 - 27/09/2023}
\subsection{Motivações}
\begin{itemize}
  \item Espectro e Maximais de um Corpo de Polinômios;
  \item Extensões e Quocientes.
\end{itemize}
\subsection{Espectro de Polinômios}
\begin{prop*}
  Se F é um corpo, então o espectro de \(F[x]\) é o conjunto \(\mathrm{Spec}(F[x]) = \{0\}\cup\{\langle f(x) \rangle: f\text{ é irredutível}\}\) e o espectro maximal de 
 \(F[x]\) é \(\mathrm{Specm}(F[x]) = \{\langle f(x) \rangle: f\text{ é irredutível}\}\).
\end{prop*}
\begin{proof*}
  Como \(F[x]\) é um domínio, \((0)\in \mathrm{Spec(F[x])}.\) Seja \( \mathfrak{p} = \langle f(x) \rangle\in \mathrm{Spec}(F[x])\) e seja
 f(x) não irredutível. Então, \(f(x) = g(x)h(x),\) em que \(\deg{g}, \deg{h} < \deg{f}.\) Temos 
  \[
    gh = f\in \langle f(x) \rangle = \mathfrak{p} \Rightarrow g\in \mathfrak{p}\text{ ou } h\in \mathfrak{p}.
  \]
  Contradição. Logo, f(x) é irredutível. Agora, suponhamos que \(f(x)\) é irredutível, de forma que nosso objetivo será mostrar que 
 \(\langle f(x) \rangle\) é primo. Sejam \(g, h \in \langle f(x) \rangle,\) tal que \(f\mid gh.\) Como f é irredutível, vale que 
 ou \(f\mid g\), ou \(f\mid h\), ou seja, \(g\in \langle f(x) \rangle\) ou \(h\in \langle f \rangle\), o que significa, exatamente,
 que \(\langle f \rangle\) é primo, isto é, \(\langle f \rangle\in \mathrm{Spec}(F[x]).\)

  Agora, tome \(\mathfrak{m}\in \mathrm{Specm}(F[x]).\) Duas possibilidades surgem - ou \(\mathfrak{m}\) é \(\langle f(x) \rangle\) com f irredutível,
ou \((0)\), ou seja, \(\mathrm{Specm}(F[x]) \subseteq{\{\langle f(x) \rangle: f\text{ é irredutível}\}}\). Por outro lado,
seja f(x) irredutível. Pela primeira parte, \(\langle f(x) \rangle\) é primo. Considere 
  \[
    \langle f(x) \rangle \subseteq{\mathfrak{j}}\subsetneq{F[x]}
  \]
  e coloque \(\mathfrak{j} = \langle g(x) \rangle.\) Temos 
  \[
    \langle f(x) \rangle \subseteq{\langle g(x) \rangle} \Rightarrow f(x)\in \langle g(x) \rangle,
  \]
ou seja, \(g\mid f,\) mas f é irredutível, donde segue que \(g=1\) ou \(g = f.\) Se \(g=1,\) então \(\mathfrak{j} = F[x],\) 
uma contradição. Logo, \(g=f\) e \(\langle f \rangle = \mathfrak{j}.\) Portanto, \(\langle f \rangle\) é maximal, provando que 
 \(\mathrm{Specm}(F[x]) = \{\langle f(x) \rangle: f\text{ é irredutível}\}\). \qedsymbol
\end{proof*}
\subsection{Corpos}
\begin{def*}
  Sejam F e K dois corpos tais que \(F\subseteq{K}.\) Neste caso, dizemos que K é uma \textbf{extensão} de F e denotaremos por
 \(K/F.\) Se \(K/F\) é uma extensão de corpos, K pode ser visto como espaço vetorial sobre F, e a dimensão de K
como F-espaço vetorial (\(\dim_{F}K\) é denotada por \([K:F]\) e chamada o \textbf{grau} da extensão. \(\square\)
\end{def*}
\begin{example}
 \begin{itemize}
   \item[1)] \([\mathbb{C}:\mathbb{R}] = 2,\) com base \(\{1, i\}\);
   \item[2)] \([\mathbb{R}:\mathbb{Q}] = \infty\) (não enumerável);
   \item[3)] Se F é um corpo, \([F:f]=1\);
   \item[4)] \([\mathbb{Q}[\sqrt[]{2}]:\mathbb{Q}] = 2,\) com base \(\{1, \sqrt[]{p}\}\).
 \end{itemize}
\end{example}
\begin{crl*}
  Se \(f(x)\in F[x]\) é irredutível e mônico, então \(K\coloneqq \frac{F[x]}{\langle f(x) \rangle}\) é um corpo.
\end{crl*}
\begin{proof*}
  Segue automaticamente de \(\langle f(x) \rangle \) ser maximal e o quociente de um corpo por um maximal ser outro corpo. \qedsymbol
\end{proof*}
  Podemos definir o mapa 
 \begin{align*}
   &\varphi :F\rightarrow K=\frac{F[x]}{\langle f(x) \rangle}\\
   &a\mapsto \overline{a} = a + \langle f(x) \rangle.
 \end{align*}
 Observe que \(\varphi \) é um morfismo de corpos injetor, já que \(\ker{\varphi } \trianglelefteq{F}\) e,
sendo F um corpo, os únicos ideais são \((0)\) ou \(F\). Sabendo que \(\varphi(1) = \overline{1}\neq\overline{0}, \ker{\varphi }\neq F\) e,
assim, \(\ker{\varphi } = (0).\)

  Com isso, podemos pensar em F como subcorpo de K usando a injeção dada por \(\varphi \), e denotaremos \(\overline{a}\) simplesmente por a.
\begin{prop*}
  Seja \(f(x)\in F[x]\) irredutível e mônico e \(K = \frac{F[x]}{\langle f(x) \rangle}.\) Então, \([K:F] = \deg{f(x)}.\)
\end{prop*}
\begin{proof*}
  Seja \(\alpha  = \overline{x}\), tal que \(K = F(\alpha ) = \{g(\alpha ): g(x)\in F[x]\}.\) Tome \(\beta \in K.\) Então,
 \(\beta = \overline{g(x)} = g(\overline{x}) = g(\alpha ).\) Mostraremos que \(B = \{1, \alpha , \cdots, \alpha ^{n-1},\}\) em que
 \(n=\deg{f}\), é base de \(K/F.\)

  Começamos provando que B gera todo o espaço. Pelo algoritmo da divisão, \(g(x) = q(x)f(x) + r(x),\) com \(\deg{r} < \deg{g}.\)
Assim, \(\beta  = g(\alpha ) = \underbrace{\overline{q(x)f(x)}}_{=\overline{0}} + \overline{r(x)} = \overline{r(x)}\).
Então, se \(r(x) = b_{m}x^{m} + \cdots + b_{0}, m < n,\) segue que 
  \[
    \beta = 0\alpha ^{n-1} + \cdots + 0\alpha ^{m+1} + b_{m}\alpha ^{m} + \cdots + b_{0}.
  \] 

  Agora, vamos mostrar que B é linearmente independente. Seja \(c_{0}1 + c_{1}\alpha + \cdots + c_{n-1}\alpha ^{n-1} = 0\) com coeficientes
 \(c_{i}\in F.\) Se \(h(x) = c_{n-1}x^{n-1} + \cdots + c_{1}x + c_{0},\) então \(\overline{h(x)} = \overline{0}\in K\) e, assim, 
 \(h(x)\in \langle f(x) \rangle,\) ou seja, \(f(x)\mid h(x).\) Como \(\deg{h} < \deg{f}, \) isso é possível se, e somente se, \(h(x) = 0,\) o
 que implica que \(c_{i} = 0\) para todo \(i=0, \cdots, n-1\).

  Portanto, B é um conjunto gerador linearmente independente de K, ou seja, uma base. \qedsymbol
\end{proof*}
\begin{example}
 \begin{itemize}
   \item[1)] Como \([\mathbb{C}:\mathbb{R}]=2, \mathbb{C}\cong{\frac{\mathbb{R}}{\langle x^{2} + 1 \rangle}}\)
   \item[2)] Seja F um corpo qualquer. Como \([F:F] = 1, F\cong{\frac{F[x]}{\langle x-a \rangle}}\)
   \item[3)] Para \(\mathbb{Q}[\sqrt[]{p}],\) tal que \([\mathbb{Q}[\sqrt[]{p}]:\mathbb{Q}]=2, \mathbb{Q}[\sqrt[]{p}]\cong{\mathbb{Q}/\langle x^{2}-p \rangle}\)
   \item[4)] Seja \(f(x) = x^{2} + x + 1\in \mathbb{F}_{2}[x].\) Note que \(f(0) = 1\neq0\) e \(f(1) = 1^{2} + 1 + 1 = 1\neq0\), ou seja,
f não tem raízes em \(\mathbb{F}_{2}\), o que significa que ele é irredutível. Assim, \(K = \mathbb{F}_{2}[x]/\langle x^{2}+x+1 \rangle = \mathbb{F}_{2}(\alpha )\) 
e, pelo resultado acima, \([K:\mathbb{F}_{2}] = \deg{f} = 2,\) tal que \(K = \{\overline{0}, \overline{1}, \overline{x}, \overline{x+1}: \overline{x}^{2} = \overline{x} + \overline{1}\}\)
e \(|K| = 4.\) Construímos, assim, um corpo com 4 elementos.
   \item[5)] Se \(K = \frac{\mathbb{F}_{2}[x]}{\langle x^{3} + x + 1 \rangle}\) e \(L = \mathbb{F}_{2}[x]/\langle x^{3}+x^{2}+1 \rangle,\) então \([K:\mathbb{F}_{2}] = 3, [L:\mathbb{F}_{2}]=3\).
Assim, \(|K| = 2^{3} = 8\) e \(|L| = 2^{3} = 8\).
 \end{itemize}
\end{example}
\begin{example}[Exercícios]
 \begin{itemize}
   \item[1)] Se E é um corpo com 4 elementos, mostre que \(E\cong{\frac{\mathbb{F}_{2}[x]}{\langle x^{2}+x+1 \rangle}}\)
   \item[2)] Se E é um corpo com 8 elementos, então \(E\cong{K}\cong{L}.\) Denotaremos um corpo com 8 elementos por \(\mathbb{F}_{8}.\)
   \item[3)] Mostre que não existe monomorfismo \(\mathbb{F}_{4}\hookrightarrow \mathbb{F}_{8}.\)
   \item[4)] Seja F corpo e \(f(x)\in F[x]\) tal que \(2\leq \deg{f}\leq 3\). Mostre que f é irredutível se, e somente se, f noa possui raízes em F.
   \item[5)] Mostre que \(\mathbb{F}_{4}^{*}\cong{\mathbb{Z}/3}, \mathbb{F}_{8}^{*}\)
 \end{itemize}
\end{example}
\newpage

\section{Aula 14 - 09/10/2023}
\subsection{Motivações}
\begin{itemize}
  \item Números Algébricos e Transcendentes;
  \item Extensões Algébricas;
\end{itemize}
\subsection{Números Algébricos e Transcendentes}
\begin{def*}
  Seja \(K/F\) uma extensão de corpos e seja \(a\in K\). Definimos o \textbf{mapa de evaluação em} \(\alpha \) por:
 \begin{align*}
   &e_{\alpha }:F[x]\rightarrow K\\
   &f(x)\mapsto f(\alpha ).\quad\square
 \end{align*}
 Note que \(e_{\alpha }\) é um morfismo de anéis. Dizemos que um elemento \(\alpha \in K\) é
\begin{itemize}
  \item[1)] \textbf{algébrico} sobre F se \(\ker{(e_{\alpha })}\neq(0)\)
  \item[2)] \textbf{transcendente} sobre F se \(\ker{(e_{\alpha })}=(0).\)
\end{itemize}
\end{def*}
\begin{example}[Exercícios]
 \begin{itemize}
   \item[1)] Mostre que \(\alpha \in K\) é algébrico sobre F se, e somente se, existe \(f(x)\neq 0\) em \(F[x]\) tal que
 \(f(\alpha ) = 0.\)
   \item[2)] Mostre que \(\alpha \in K\) é transcendente sobre F se, e somente se, para todo
 \(f(x)\in F[x]\) não nulo, \(f(\alpha)\neq0.\)
 \end{itemize}
\end{example}
  Podemos observar algumas coisas. A primeira delas é que \(\ker{e_{\alpha }}\neq(0).\) Como
F[x] é D.I.P., existe f(x) tal que \(\ker{e_{\alpha }} = \langle f(x) \rangle.\) Já que \((0)\in \mathrm{Spec(K)}, \ker{e_{\alpha }}=e_{\alpha }^{-1}((0))\in \mathrm{Spec(F[x])}.\)
Assim, \(f(x)\) é irredutível e \(\ker{e_{a}}\in \mathrm{Specm(F[x])},\) donde obtemos a injeção de corpos:
  \[
    \overline{e}_{\alpha }:\frac{F[x]}{\langle f(x) \rangle}\rightarrow K.
  \]
  Denotaremos \(Im(e_{\alpha }) = F(\alpha ).\) Desta forma, \(F(\alpha )\subseteq{K}\) e \([F(\alpha ): F]=\deg{f}\) pela última aula.

  Além disso, note que se \(e_{\alpha }\) é injetor, então \(F[x]\cong{F[\alpha ]}\) (isomorfismo de anéis).
 \begin{def*}
   Dizemos que uma extensão \(K/F\) é \textbf{algébrica} se todo elemento de K é algébrico. \(\square\)
 \end{def*}
 \begin{example}
  \begin{itemize}
    \item[1)] Se \([K:F]\) é finita, então \(K/F\) é algébrica. De fato, seja \(n=[K:F]\) e \(\alpha \in K\) um elemento qualquer.
Considere o conjunto \(S = \{1, \cdots, \alpha ^{n}\}\subseteq{K};\) Como \(\dim_{F}K = n,\) S é necessariamente um conjunto linearmente
dependente. Então, existe \(a_{0}, \cdots, a_{n}\in F\) não todos nulos tais que \(a_{0} + a_{1}\alpha + \cdots + a_{n}\alpha^{n} = 0.\)

  Considere o polinômio \(f(x) = a_{0}+a_{1}x + \cdots + a_{n}x^{n}.\) Ele é não nulo, pois nem todos os \(a_{i}\) são nulos,
e \(f(\alpha ) = 0\). Logo, \(\alpha \) é algébrico sobre F.

  \item[2)] Seja \(K/\mathbb{C}\) uma extensão algébrica sobre \(\mathbb{C}.\) Então, \(K = \mathbb{C}.\) Com efeito, seja \(\alpha \in K\)
e \(f(x)\in \mathbb{C}[x]\) não nulo e mônico tal que \(f(\alpha ) = 0\). Pelo Teorema Fundamental da Álgebra, podemos escrever 
  \[
    f(x) = a(x-\alpha_{1})\cdot \cdots \cdot (x-\alpha_{n}),
  \]
  para algum \(a\in \mathbb{C}\setminus{\{0\}},\alpha_{i}\in \mathbb{C}\) e \(n = \deg{f}.\) Assim, \(0 = f(\alpha ) = a(\alpha -\alpha_{1})\cdot \cdots(\alpha -\alpha_{n})\)
e, como K é domínio, existe i tal que \((\alpha -\alpha_{i})=0,\) o que significa que \(\alpha=\alpha_{i}\in \mathbb{C}.\) Logo, \(\ker{e_{\alpha }}=\langle x-a \rangle\) e \([K:\mathbb{C}] = 1.\)
\end{itemize}
\end{example}
\begin{prop*}
  Seja \(K/F\) uma extensão de corpos. Suponha que F é não enumerável e \([K:F]=\dim_{F}K\) é enumerável. Então,
 \(K/F\) é algébrica.
\end{prop*}
\begin{proof*}
  Suponha que \(K/F\) não seja algébrica. Escolha \(\alpha \in K/F\) que não é
algébrico sobre F. Mostraremos que 
  \[
    S = \biggl\{\frac{1}{\alpha - a}: a\in F\biggr\}\subseteq{K}
  \]
é linearmente independente, e assim teremos um absurdo, pois \(|S| = |F|\) e teríamos 
 \(\dim_{F}K\geq |S|.\)

 Com efeito, seja uma combinação linear finita:
\begin{align*}
  &\frac{c_{1}}{\alpha - a_{1}} + \cdots + \frac{c_{n}}{\alpha - a_{n}} = 0\\
  \underbrace{\Rightarrow}_{\times (\alpha -a_{1})}&c_{1} + \frac{c_{2}(\alpha - a_{1})}{\alpha - a_{2}} + \cdots + \frac{c_{n}(\alpha - a_{1})}{\alpha - a_{n}} = 0.
\end{align*}
  Como \(\alpha \) é transcendente, temos \(F[x]\cong{F(\alpha )}\) e então \(\mathrm{Frac}(F[x])\cong{\mathrm{Frac}(F(\alpha ))}\).
Assim, aplicando o isomorfismo, substituindo \(\alpha \) por x, obtemos a mesma igualdade com 0: 
  \[
    c_{1} + \frac{c_{2}(x-a_{1})}{x-a_{2}} + \cdots + \frac{c_{n}(x-a_{1})}{x-a_{n}} = 0.
  \]
  Como a igualdade vale para todo x, substituindo \(x=a_{1}, c_{1} = 0.\) Podemos fazer esse processo para
todo \(i=1, \cdots, n\) e conseguindo, assim, \(c_{i} = 0\) para \(i=1, \cdots, n.\) Então, o conjunto é linearmente independente
e temos o que queríamos demonstrar. \qedsymbol
\end{proof*}
\begin{theorem*}
  Seja \(A = F[x_{1}, \cdots, x_{n}],\) F corpo. Então, \(\mathfrak{i} = \langle x_{1}-a_{1}, \cdots, x_{n}-a_{n}\rangle \trianglelefteq{A}\) 
é maximal, para \(a_{1}, \cdots, a_{n}\in F\).
\end{theorem*}
\begin{proof*}

 A prova segue da indução sobre o número de variáveis

\textbf{\underline{Caso Base}:} Sejam \(\mathfrak{i} = \langle x - a_{1} \rangle, 0\neq \overline{f(x)}\in A/\mathfrak{i}.\)
Existem \(q(x), r(x)\) tais que \(f(x) = q(x)(x-a_{1})+r(x),\) com \(\deg{r}\leq \deg{(x-a)}=1.\) Com isso,
r(x) é constante, digamos \(r(x) = a\). Então, \(\overline{f(x)}=\overline{a}\) é constante e
 \(g(x) = \overline{a^{-1}}\) é o inverso de f(x) em \(A/\mathfrak{i}.\) Assim, \(A/\mathfrak{i}\) é um corpo isomorfo a F por meio
 de \(f(x)\mapsto a.\)

\textbf{\underline{Hipótese Indutiva}:} Suponha que o resultado vale para \(k < n\). Agora, \(\mathfrak{i} = \langle x-a_{1}, x-a_{2}, \cdots, x-a_{n} \rangle\) e tomemos
 \(0\neq \overline{f(x_{1}, \cdots, x_{n})}\in A/\mathfrak{i}\). Podemos escrever
 \(f(x_{1}, \cdots, x_{n})\) como um elemento de \(F[x_{2}, \cdots, x_{n}][x_{1}]\) da seguinte forma:
  \[
    f(x_{1}, \cdots, f_{n}) = f_{m}x_{1}^{m} + f_{m-1}x_{1}^{m-1} + \cdots + f_{0},
  \]
em que \(f_{i}\in F[x_{2}, \cdots, x_{n}].\) Pelo algoritmo de Euclides, existem \(q, r\in A\) com 
  \[
    f(x_{1}, \cdots, x_{n}) = q(x_{1} - a_{1} ) + r,
  \]
  com \(r\neq0\) e \(\deg_{x_{1}}{r} = 0, r \in F[x_{2}, \cdots, x_{n}].\) Por indução, \(\frac{F[x_{2}, \cdots, x_{n}]}{\langle x_{2}-a_{2}, \cdots, x_{n}-a_{n} \rangle}\cong{F}\) e
então \(\overline{r} = \overline{b}\in F, r = b+g, g\in \langle x_{2}-a_{2}, \cdots, x_{n}-a_{n} \rangle\). Logo, 
  \[
    f(x_{1}, \cdots, x_{n}) = q(x_{1}-a_{1}) + g + b \Rightarrow \overline{f(x_{1}, \cdots, x_{n})} = \overline{b}.
  \]
  Assim, \(\overline{f(x_{1}, \cdots, x_{n})}\) corresponde a um polinômio constante \(\overline{b}\in F.\)
Portanto, \(\mathfrak{i}\) é maximal. \qedsymbol
\end{proof*}
\newpage

\section{Aula 15 - 18/10/2023}
\subsection{Motivações}
\begin{itemize}
  \item Nullstelensatz de Hilbert;
  \item Fecho Algébrico.
\end{itemize}
\subsection{Nullstelensatzs}
\begin{theorem*}[Nullstelensatz Algébrico]
  Seja K um corpo algebricamente fechado. Então, 
  \[
    \mathrm{Specm}(K[x_{1}, \cdots, x_{n}]) = \{\langle x-a_{1}, \cdots, x-a_{n} \rangle: a_{1}, \cdots, a_{n}\in K\}
  \] 
\end{theorem*}
\begin{proof*}
  Vamos provar para o caso \(K = \mathbb{C}.\) Tome \(M\in \mathrm{Specm}(K[x_{1}, \cdots, x_{n}])\) e considere o quociente \(F \coloneq K[x_{1}, \cdots, x_{n}]/M\). Como M é maximal
este quociente é um corpo. É claro que \(F\hookrightarrow K\) e que \(S=\{\overline{x_{1}}^{i_{1}}\cdot \dotsc \cdot \overline{x_{n}}^{i_{n}}: i_{j}\in \mathbb{N}\} \subseteq{F}\)
gera F como K-espaço vetorial e é enumerável. Em outras palavras, \(\dim_{K}F\) é enumerável e, como K é não enumerável, a extensão
 \(K/F\) é algébrico, do que segue que \(K\cong{F}.\)

  Seja \(\overline{x_{i}} = \overline{a_{i}}, a_{i}\in \mathbb{C}\). Então, \(\overline{x_{i} - a_{i}} = \overline{0}\) e, assim, \(x_{i}-a_{i}\in M\) para cada \(i=1, \cdots, n.\)
Sabemos que \(\langle x_{1}-a_{1}, \cdots, x_{n}-a_{n} \rangle\) é maximal e \(\langle x_{1}-a_{1}, \cdots, x_{n}-a_{n} \rangle \subseteq{M}\) implica, portanto, que
 \(M = \langle x_{1}-a_{1}, \cdots, x_{n}-a_{n} \rangle\). \qedsymbol
\end{proof*}
\begin{theorem*}[Nullstelensatz Geométrico]
  Seja \(S \subseteq{K[x_{1}, \cdots, x_{n}]}\) tal que \(\langle S \rangle\neq K[x_{1}, \cdots, x_{n}].\) Então, existe \(a=(a_{1}, \cdots, a_{n})\in K^{n}\)
tal que, para todo \(f(x_{1}, \cdots, x_{n})\in K[x_{1}, \cdots, x_{n}], f(a_{1}, \cdots, a_{n}) = 0.\)
\end{theorem*}
\begin{proof*}
  Tome \(M\in \mathrm{Specm}(K[x_{1}, \cdots, x_{n}])\) tal que \(\langle S \rangle \subseteq{M}.\) Pelo Teorema
Nullstelensatz Algébrico, sabemos que \(M = \langle x_{1}-a_{1}, \cdots, x_{n}-a_{n} \rangle\), para \(a_{i}\in \mathbb{C}.\)

  Pelo isomorfismo \(\varphi :K[x_{1}, \cdots, x_{n}]/M\rightarrow \mathbb{C},\) definido por \(\varphi (\overline{x_{i}}) = a_{i}\) temos
um polinômio genérico \(\varphi (\overline{g(x_{1}, \cdots, x_{n})}) = g(a_{1}, \cdots, a_{n}).\) Como \(\langle S \rangle \subseteq{M},\)
dado \(f(x_{1}, \cdots, x_{n})\in S, \varphi (\overline{f(x_{1}, \cdots, x_{n})})=\varphi (\overline{0}) = 0\) e, por outro, \(\varphi (\overline{f(x_{1}, \cdots, x_{n})}) =
f(a_{1}, \cdots, a_{n}).\) Assim, o ponto \((a_{1}, \cdots, a_{n})\) é raiz de todo polinômio \(f\in M\). Portanto,
é raiz de \(f\in S.\) \qedsymbol
\end{proof*}
\subsection{Fecho Algébrico}
\begin{def*}
  Um corpo K é dito \textbf{algebricamente fechado} se todo polinômio não constante \(f(x)\in K[x]\) tem
uma raiz em K. \(\square\)
\end{def*}
\begin{lemma*}
  Se K é algebricamente fechado, então todo \(f(x)\in K[x]\) fatora-se na forma 
    \[
      f(x) = (x-a_{1})^{r_{1}}\cdot \cdots \cdot (x-a_{n})^{r_{n}},
    \]
    em que \(a_{1}, \cdots, a_{n}\in K\).
\end{lemma*}
\begin{example}
 \begin{itemize}
   \item[1)] \(\mathbb{C}\) é algebricamente fechado pelo Teorema Fundamental da Álgebra;
   \item[2)] \(\overline{\mathbb{Q}} = \{\alpha \in \mathbb{C}: f(\alpha ) = 0, f(x)\in \mathbb{Q}[x]\}\) é algebricamente fechado
 \end{itemize}
\end{example}
  Todo corpo pode ser mergulhado em um outro algebricamente fechado.
\begin{def*}
  Seja F um corpo e e K um corpo algebricamente fechado que é uma extensão de F. Chamamos de
\textbf{fecho algébrico de F} o conjunto 
  \[
    \overline{F} = \{\alpha \in K: \exists f(x)\in F[x], f(\alpha ) = 0\}.\quad\square
  \]
\end{def*}
\begin{prop*}
  Seja F um corpo e K um corpo algebricamente fechado tal que \(F \subseteq{K}.\) Então, 
  \[
    \overline{F} = \{\alpha \in K: \exists f(x)\in F[x], f(\alpha ) = 0\}
  \]
  é um corpo algebricamente fechado.
\end{prop*}
  Vale notar que o fecho algébrico de um corpo é único a menos de isomorfismo.
\subsection{Parte Extra - Construindo a Topologia de Zariski sobre \(\mathbb{C}\)}
  Para \(\mathfrak{i}\trianglelefteq{\mathbb{C}[x_{1}, \cdots, x_{n}]},\) defina \(V(I) = \{a\in \mathbb{C}^{n}: f(a) = 0 \forall f\in \mathfrak{i}\}\). 
Mostre que:
\begin{itemize}
  \item[1)] \(\bigcap_{k\in K}^{}{V(\mathfrak{i}_{k})} = V(\sum\limits_{k\in K}^{}\mathfrak{i}_{k})\);
  \item[2)] \(\bigcup_{i=1}^{n}{V(\mathfrak{i_{k}})} = V(\bigcap_{i=1}^{n}{\mathfrak{i}_{k}}) = V(\mathfrak{i}_{1}\cdot \dotsc \mathfrak{i}_{k})\);
  \item[3)] \(V(\mathbb{C}[x_{1}, \cdots, x_{n}]) = \emptyset\) e \(V((0)) = \mathbb{C}^{n}\);
  \item[4)] \(\mathfrak{i}\subseteq{\mathfrak{j}} \Rightarrow V(\mathfrak{j}) \subseteq{} V(\mathfrak{i})\);
  \item[5)] \(V(\mathfrak{i}) = V(\sqrt[]{\mathfrak{i}}),\) em que \(\sqrt[]{\mathfrak{i}}\coloneqq \{g\in A: \exists n, g^{n}\in \mathfrak{i}\}\)
\end{itemize}
  As propriedade 1, 2 e 3 fornece-nos uma topologia sobre \(\mathbb{C}^{n}\) gerada pelos fechados \(V(\mathfrak{i}),\) chamada \textbf{topologia de Zariski.} Se
 \(f\in \mathbb{C}[x_{1}, \cdots, x_{n}],\) definimos \(V(f) = V(\langle f \rangle)\).
\begin{itemize}
  \item[1)] Mostre que \(S^{1}\coloneqq \{z=(a, b)\in \mathbb{C}^{2}: |a|^{2} + |b|^{2}=1\}\) é fechado na topologia usual em
 \(\mathbb{C}^{2},\) mas não é fechado na topologia de Zariski.
  \item[2)] Mostre que \(\{V(f): f\in \mathbb{C}[x_{1}, \cdots, x_{n}]\}\) é uma base para a topologia de Zariski em \(\mathbb{C}^{n}.\)
  \item[3)] Mostre que \(\mathbb{C}^{n}\) é compacto com a topologia de Zariski.
\end{itemize}
\newpage

\section{Aula 16 - 23/10/2023}
\subsection{Motivações} 
\begin{itemize}
  \item Domínios Euclideanos;
  \item Normas.
\end{itemize}
\subsection{Domínios Euclidianos}
 \begin{def*}
  Um domínio A é dito \textbf{euclidiano} se existe uma função \(N:A\setminus{\{0\}}\rightarrow \mathbb{N}\cup\{0\}\)
satisfazendo:
\begin{itemize}
  \item[1)] \(N(a)\leq N(ab)\) para todos \(a, b\in A\setminus{\{0\}}\)
  \item[2)] (Algoritmo da Divisão) Para todos \(a, b\in A, b\neq0,\) existem \(q,r\in A\) tais que
 \(a = bq + r,\) em que \(N(r) < N(b)\) ou \(r = 0\).
\end{itemize}
Neste caso, a função N é  chamada \textbf{norma} de A. \(\square\)
 \end{def*}
\begin{example}
 \begin{itemize}
  \item[1)] Para \(A = \mathbb{Z},\) uma possível norma é 
  \begin{align*}
      &N=|\cdot |:\mathbb{Z}\rightarrow \mathbb{N}\\
      &n\mapsto |n|
  \end{align*}
  \item[2)] Se F é um corpo e \(A = F[x]\), temos a norma 
  \begin{align*}
    &N:F[x]\rightarrow \mathbb{N}\\
    &p(x)\mapsto \deg{p(x)}.
  \end{align*}
 \end{itemize}
\end{example}
\begin{theorem*}
  Domínios Euclidianos são D.I.P.'s
\end{theorem*}
\begin{proof*}
  Seja \(\mathfrak{i} \trianglelefteq{A}\) não nulo e seja \(0\neq a\in \mathfrak{i}\) elemento com
menor norma, i.e., \(N(A) = \min\{N(x): x\in \mathfrak{i}\}\). Mostraremos que \(\mathfrak{i} = \langle a \rangle\).

  Por um lado, é claro que \(\langle a \rangle \subseteq{\mathfrak{i}}.\) Por outro, tome \(x\in \mathfrak{i}\).
Pela segunda parte da definição, existem \(q, r\in A\) tais que \( x = qa + r\) e \(N(r) < N(a).\)
Então, \(r=x-qa\in \mathfrak{i}.\) Se \(r\neq0, N(r) < N(a)\), o que é um absurdo pela definição de a como 
elemento com menor norma. Assim, \(r=0\), ou seja, \(x = qa\). Portanto, \(x\in \mathfrak{i}\) e \(\mathfrak{i} = \langle a \rangle\). \qedsymbol
\end{proof*}
\begin{example}
 \begin{itemize}
  \item[1)] Corpos são Domínios Euclidianos:
  \begin{align*}
    &N:F^{*}\rightarrow \mathbb{N}\\
    &a\mapsto 1
  \end{align*}
  \item[2)] Em \(A = K[x, y],\) o ideal \(\mathfrak{i} = \langle x, y \rangle\) não é principal. Logo, A não é Domínio Euclidiano.
 \end{itemize}
\end{example}
  Vale observar que a norma num Domínio Euclidiano não é única, no sentido de que pode ter mais que uma norma. Por exemplo,
  \[
    \left\{\begin{array}{ll}
      N_{1}:&\mathbb{Z}\setminus{\{0\}}\rightarrow \mathbb{N}\\
            &a\mapsto |a|\\
      N_{2}:&\mathbb{Z}\setminus{0}\rightarrow \mathbb{N}\\
            &a\mapsto |2a|.
    \end{array}\right.
  \]
 \begin{prop*}
  Seja \((A, N)\) um Domínio Euclidiano. Então, \(A^{*}=\{a\in A\setminus{\{0\}}: N(a) = N(1)\}.\)
 \end{prop*}
 \begin{proof*}
  Sejam \(a\in A^{*}\) e \(b\in A^{*}\) tais que \(ab = 1.\) Pela propriedade 1,
 \(N(a)\leq N(ab) = N(1).\) Além disso, vale também que \(N(1)\leq N(a \cdot 1) = N(a).\) Logo,
 \(N(a) = N(1).\) Com isso, provamos que \(A^{*}\subseteq{\{a\in A\setminus{\{0\}}: N(a) = N(1)\}}.\)

  Por outro lado, tome um elemento \(a\in A\setminus{\{0\}}\) tal que \(N(a) = N(1).\) Aplicando o algoritmo da divisão,
existem \(q, r\in A\) tais que \(1 = aq + r, r = 0\) ou \(N(r) < N(a) = N(1).\)

  Se \(r\neq0, N(1)\leq N(1 \cdot r) = N(r)\) implica que \(N(1)\leq N(r).\) No entanto, por hipótese,
 \(N(r) < N(1)\), ou seja, temos um absurdo. Logo, \(r = 0\) e \(aq = 1\) implica que \(a\in A^{*},\) assim provando
a outra relação de contenção.

  Portanto, \(A^{*}=\{a\in A\setminus{\{0\}}: N(a) = N(1)\}.\) \qedsymbol
 \end{proof*}
 \begin{theorem*}
  O anel \(\mathbb{Z}[i] = \{a + ib: a, b\in \mathbb{Z}\}\subseteq{\mathbb{C}}\) é um Domínio Euclidiano com a norma
 \begin{align*}
   &N:\mathbb{Z[i]}\setminus{\{0\}}\rightarrow \mathbb{N}\\
   &z = a + ib\mapsto |z|^{2} = a^{2} + b^{2}.
 \end{align*}
 \end{theorem*}
 \begin{proof*}
  Note que \(N(a+ib) = (a+ib)(a-ib)\), donde fica fácil ver que \(N((a+ib)(c+id)) = N(a+ib)N(c+id)\).
Observe também que \(N(z) = 0\) se, e somente se, \(z = 0,\) tal que dados \(x, y\in \mathbb{Z}[i], N(xy) = N(x)N(y)
\geq N(x)\) para todo \(y\neq0\).

  O corpo de frações \(\mathrm{Frac}(\mathbb{Z}[i]) = \mathbb{Q}[i] \subseteq{\mathbb{C}}\) pode receber a mesma norma
induzida que a de \(\mathbb{Z}[i]\). Assim, sejam \(x, y\in \mathbb{Z}[i], y\neq0.\) Buscamos \(q, r\in \mathbb{Z}[i]\) tais que
 \(x = qy + r,\) em que \(r=0\) ou \(N(r) < N(y)\). Isso é equivalente a
\begin{align*}
  N(x-qy) < N(y) &\Longleftrightarrow N \biggl(y \biggl(\frac{x}{y}\biggr) - qy\biggr) < N(y)\\
                 &\Longleftrightarrow N(y) N \biggl(\frac{x}{y}-q\biggr) < N(y)\\
                 &\Longleftrightarrow N \biggl(\frac{x}{y} - q\biggr) < 1.
\end{align*}
  Com isso, basta acharmos \(q\in \mathbb{Z}[i]\) tal que \(N(x/y - q) < 1.\)

  Dado \(x/y\in \mathbb{Q}[i],\) seja \(x/y = e + if.\) Existem \(g, h\in \mathbb{Z}\)
tais que \(|e-g|\leq 1/2\) e \(|e-h|\leq 1/2\) através do seguinte processo:

  Se \(e\in \mathbb{Q},\) existe \(n\in \mathbb{Z}\) tal que \(n\leq e < n+1.\) Se \(0 < e-n\leq 1/2\),
então \(|e-n|\leq 1/2\). Caso contrário, \(1/2 < e-n < 1,\) tal que \(-1/2 < e - n - 1 < 0\) e, assim,
 \(|e-(n+1)| < 1/2\).

  Com isso, se \(q\coloneqq g + ih,\) temos 
 \begin{align*}
   N \biggl(\frac{x}{y} - q\biggr) &= N(e + if - g - ih)\\
                                   &= N(e - q + (f-h)i)\\
                                   &= (e-q)^{2} + (f-h)^{2}\\
                                   &=\frac{1}{2} < 1.
 \end{align*}
  Portanto, provamos o resultado. \qedsymbol
 \end{proof*}
\begin{crl*}
  \(\mathbb{Z}[i]\) é um D.I.P.
\end{crl*}
\begin{example}
  Mostremos que o ideal \(\langle (3, 2 + \sqrt[]{-5}) \rangle \trianglelefteq{\mathbb{Z}[\sqrt[]{-5}]}\) não é principal. Logo, que
 \(\mathbb{Z}[\sqrt[]{-5}]\) não é um domínio euclidiano.

  Com efeito, seja \(\langle 3, 2 + \sqrt[]{-5} \rangle = \langle a + b\sqrt[]{-5} \rangle.\) Observe que 
    \[
      N':\mathbb{Z}\sqrt[]{-5}\rightarrow \mathbb{N},\quad x = a+b\sqrt[]{-5}\mapsto a^{2}+5b^{2} = N(x)
    \]
satisfaz a condição \(N(xy) = N(x)N(y)\geq N(x).\) Dado o elemento \(3\in \langle a + b\sqrt[]{-5}, \rangle\)
vale que \(3 = (a+b\sqrt[]{-5})(c+d\sqrt[]{-5}),\) do que obtemos que 
  \[
    9 = N(3) = N(a+b\sqrt[]{-5})N(c+d\sqrt[]{-5}) = (a^{2}+5b^{2})(c^{5}+5d^{5}).
  \]
Com isso, \(a^{2}+5b^{2}\mid 9\) dá-nos três possibilidade - \(a^{2} + 5b^{2} = 1, 3,\text{ ou }9.\)
Se \(a^{2} + 5b^{2} = 1\) ou \(a^{2} + 5b^{2} = 3,\) obtemos \(b = 0\), duas contradições. Fazendo mais contas,
chegamos em ainda mais contradições, do que segue que a condição \(N(xy)\geq N(x)\) não pode ser satisfeito e, portanto,
 \(\mathbb{Z}[\sqrt[]{-5}]\) não é Domínio Euclidiano e nem \(\langle (3, 2+\sqrt[]{-5}) \rangle\) é um ideal principal.
\end{example}
 \begin{example}[Exercício]
  Mostre que \(\mathbb{Z}[\sqrt[]{-2}] = \{a + b\sqrt[]{-2}: a, b\in \mathbb{Z}\}\) é um Domínio
Euclidiano com a norma \(N(a+b\sqrt[]{-2}) = a^{2} + 2b^{2}.\)
\end{example}
 \newpage

\section{Aula 17 - 25/10/2023}
\subsection{Motivações} 
\begin{itemize}
  \item Corpos quadráticos;
  \item O anel dos inteiros de \(\mathbb{Q}[\sqrt[]{d}]\);
  \item Elementos irredutíveis;
  \item Anéis Noetherianos e Teorema da Base de Hilbert.
\end{itemize}
\subsection{Corpos Quadráticos}
\begin{def*}
  Seja \(\alpha \) raiz de um polinômio irredutível e mônico de grau 2, ou seja, \(x^{2}+ax+b\in \mathbb{Z}[x].\) Vamos denotar
por \(\mathbb{Z}[\alpha ]\) o menor subanel de \(\mathbb{C}\) que contém \(\mathbb{Z}\) e \(\alpha \): 
  \[
    \mathbb{Z}[\alpha ]\coloneqq \bigcap_{A\in \mathcal{A}}^{}{A},
  \]
em que \(\mathcal{A} = \{A\text{ subanel de }\mathbb{C}: \mathbb{Z}\subseteq{A}\text{ e }\alpha \in A\}. \square\)
\end{def*}
\begin{prop*}
  Temos \(\mathbb{Z}[\alpha ] = \{a + b\alpha : a, b\in \mathbb{Z}\}.\)
\end{prop*}
\begin{proof*}
  De fato, coloque \(B\coloneqq \{c+d\alpha : c, d\in \mathbb{Z}\}.\) Note que B é subanel de \(\mathbb{C}\).
Com isso, se \(a + b\alpha , c + d\alpha \in B,\) então \(a + b\alpha + c +d\alpha = a + c + (b+d)\alpha \in B\)
e, a multiplicação: 
  \[
    (a+b\alpha )(c+d\alpha ) = ac + bd\alpha^{2} + ad\alpha + bc\alpha 
  \]
e note que, já que \(\alpha \) é raiz de um polinômio, \(x^{2} + a'x + b'\), então \(\alpha ^{2} + a'\alpha  + b' = 0\) implica 
que \(\alpha^{2} = -a'\alpha  - b'\), isto é, \(\alpha^{2} = c'\alpha + d'\) para \(c', d'\in \mathbb{Z},\) donde segue que 
  \[
    ac + bd(c'+d'\alpha ) + ad\alpha + bc\alpha = ac + bdc' + \alpha(bdd'+ad+bc)\in B.
  \]
  Logo, B é subanel de \(\mathbb{C}.\) Além disso, temos \(\mathbb{Z}\subseteq{B}, \alpha \in B\), tal que
 \(\mathbb{Z}[\alpha ]\subseteq{B}.\)

  Por outro lado, seja \(A\) um subanel de \(\mathbb{C}\) que contém \(\mathbb{Z}\) e \(\alpha .\) Caso \(c+d\alpha \in B, c, d,\in A\)
  e a combinação \(c+d\alpha \in A\). Logo, \(B \subseteq{A}.\) Portanto, \(B\subseteq{\mathbb{Z}[\alpha ]}\). \qedsymbol
\end{proof*}
\begin{example}[Exercícios]
 \begin{itemize}
 \item[1)] Seja \(\alpha \) raiz de \(x^{2} + ax + b\in \mathbb{Z}[x]\) irredutível. Então, 
   \[
     \mathbb{Z}[\alpha ] \cong{\frac{\mathbb{Z}[x]}{\langle x^{2} + ax + b \rangle}}
   \]
  \item[2)] Se \(\beta \) é outra raiz do mesmo polinômio, então \(\mathbb{Z}[\alpha ] = \mathbb{Z}[\beta ]\)
 \end{itemize}
\end{example}
  Definimos 
\begin{align*}
  &N:\mathbb{Z}[\alpha ]\setminus{\{0\}}\rightarrow \mathbb{N}\\
  &x+y\alpha \mapsto |x^{2} - axy + y^{2}b|.
\end{align*}
\begin{def*}
  Se \(\alpha \) é a raiz do polinômio mônico de grau 2, o \textbf{conjugado} de \(\theta = c+d\alpha \) é definido como \(\overline{\theta } = c + d\beta \), em que \(\beta \) é a outra raiz. \(\square\)
\end{def*}
  Note que um elemento e seu conjugado satisfazem a seguinte relação
\begin{align*}
  \theta \overline{\theta } &= (x+y\alpha )(x+y\beta)\\
                            &= x^{2} + xy\beta  + xy\alpha  + y^{2}\alpha \beta \\
                            &= x^{2} + (\alpha +\beta )xy + \alpha \beta y^{2}\\
                            &= x^{2} + (-a)xy + (b)y^{2}\\
                            &= N(\theta).
\end{align*}
  Disto segue que 
 \begin{itemize}
  \item[1)] \(N(\theta \varphi )=N(\theta )N(\varphi )\);
  \item[2)] \(\mathbb{Zj[\alpha ]^{*} = \{\theta \in \mathbb{Z}[\alpha ]: N(\theta ) = 1\}}\).
 \end{itemize}
\begin{def*}
  Seja \(d\in \mathbb{Z}\) livre de quadrados. O corpo \(\mathbb{Q}[\sqrt[]{d}] = \{a + b\sqrt[]{d}: a, b\in \mathbb{Q}\}\) é 
chamado \textbf{corpo quadrático.} \(\square\)
\end{def*}
 \begin{example}
  \begin{itemize}
    \item[1)] Os inteiros de Gauss, \(\mathbb{Z}[i]\), com o polinômio correspondente \(x^{2} + 1 = 0\).
    \item[2)] Os inteiros de Eisenstein, \(\mathbb{Z}\biggl[\frac{1+\sqrt[]{-3}}{2}\biggr]\), com o polinômio \(x^{2} - x + 1 = 0\)
    \item[3)] Para o polinômio \(x^{2} - x - 1 = 0, \mathbb{Z}\biggl[\frac{1 + \sqrt[]{5}}{2}\biggr]\)
    \item[4)] Se \(d\in \mathbb{Z}\) é \textbf{livre de quadrados}, i.e., não existe \(l\in \mathbb{Z}\)
      tal que \(l^{2}\mid d, \mathbb{Z}[\sqrt[]{d}]\) é um anel quadrático do polinômio \(x^{2} - d = 0.\)
  \end{itemize}
 \end{example}
\begin{def*}
  Seja d livre de quadrados, denotamos por \(\mathcal{O}_{d}\) o conjunto
\(\{\alpha \in \mathbb{Q}[\sqrt[]{d}]:\text{ existe }f(x)\in \mathbb{Z}[x], \deg{f(x)} = 2, f \text{ mônico irracionais, tais que } f(\alpha)=0\}.\quad\square\)
\end{def*}
\begin{theorem*}
  Se \(d\in \mathbb{Z}\) é livre de quadrados, então 
  \[
    \mathcal{O}_{d} = \left\{\begin{array}{ll}
        \mathbb{Z}[\sqrt[]{d}],\quad\text{se } d\equiv 2, 3 \mod{4}\\
        \mathbb{Z}\biggl[\frac{1+\sqrt[]{d}}{2}\biggr],\quad\text{se } d\equiv 1\mod{4}.
      \end{array}\right.
  \]
\end{theorem*}
\begin{def*}
  O anel \(\mathcal{O}_{d}\) é chamado o \textbf{anel dos inteiros} do corpo \(\mathbb{Q}[\sqrt[]{d}].\) Nele,
podemos definir a norma:
\begin{align*}
  &N_{d}:\mathcal{O}_{d}\rightarrow \mathbb{N}\\
  &x+y\theta \mapsto \left\{\begin{array}{ll}
      x^{2} + dy^{2},\quad \text{se } d\equiv2,3\mod{4}\\
      x^{2} - xy - \frac{d-1}{4}y^{2},\quad\text{se } d\equiv1\mod{4},
    \end{array}\right.
\end{align*}
em que \(\theta = \sqrt[]{d}\) se \(d\equiv 2, 3\mod{4}\) e \(\theta = \frac{1+\sqrt[]{d}}{2}\), se \(d\equiv 1\mod{4}.\square\)
\end{def*}
\begin{def*}
  Dizemos que \(\mathcal{O}_{d}\) é \textbf{norma euclidiano} se é um domínio euclidiano com a norma definida \(N_{d}\) acima. \(\square\)
\end{def*}
\begin{theorem*}
  \(\mathcal{O}_{d}\) é norma euclidiano se, e somente se, 
  \[
    d\in\{-11, -7, -3, -2, -1, 2, 3, 5, 6, 7, 11, 13, 17, 19, 21, 29, 33, 37, 41, 57, 73\}.
  \]
\end{theorem*}
\begin{theorem*}
 \begin{itemize}
  \item[1)] \(\mathcal{O}_{69}\) é euclidiano (mas não é norma euclidiano)
  \item[2)] \(\mathcal{O}_{14}\) é euclidiano (mas não é norma euclidiano)
  \item[3)] Se \(d < 0\) e \(d\not\in\{-11, -7, -3, -2, 1\},\) então \(\mathcal{O}_{d}\) não é euclidiano
  \item[4)] Se \(d < 0\), então \(\mathcal{O}_{d}\) é D.I.P., se e somente se 
    \[
      d\in\{-1, -2, -3, -7, -11, -19, -43, -67, -163\}.
    \]
 \end{itemize}
\end{theorem*}
  Relacionada a esses anéis dos inteiros, há a seguinte conjectura:

\textbf{Conjectura:} Classificar para quais \(d\in \mathbb{Z}\) o anel \(\mathcal{O}_{d}\) é D.I.P. e para quais
 \(\mathcal{O}_{d}\) é euclidiano.

\begin{def*}
  Seja A um domínio. Dizemos que \(a\in A\) é \textbf{irredutível} se \(a\not\in A^{*}\) e se \(a = bc,\) então \(b\in A^{*}\) ou \(c\in A^{*}.\square\)
\end{def*}
\begin{example}
 \begin{itemize}
  \item[1)] Em \(\mathbb{Z}\), apenas primos positivos ou negativos são irredutíveis;
  \item[2)] Se F é um corpo, em \(F[x]\) apenas os polinômios irredutíveis são irredutíveis.
 \end{itemize}
\end{example}
\begin{theorem*}
  Se A é um D.I.P, então todo elemento não irredutível de A pode ser escrito como produto de elementos irredutíveis.
\end{theorem*}
\begin{proof*}
  Suponha que existe um \(a\in A\) que não é irredutível, porém não pode ser escrito como produto de
elementos irredutíveis. Seja \(a=bc\) tais que \(b, c\not\in A^{*}.\)

  Note que um dos dois, b ou c, não pode ser produto de elementos irredutíveis. Suponha que este seja b.
Disto, segue que \(\langle a \rangle\subsetneq{\langle b \rangle}\), já que, como \(a=bc, a\in \langle b \rangle\) e, assim,
 \(\langle a \rangle\subset{\langle b \rangle}\). Suponha, agora, que \(\langle a \rangle = \langle b \rangle.\) Então, \(b\in \langle a \rangle\) implica
 que \(b = ar\) para algum \(r\in A\), de forma que \(b = (bc)r.\) Como A é domínio, \(cr = 1\), ou seja, \(c\in A^{*},\)
contrariando a hipótese antes feita.

  Seja \(a_{1}\coloneqq b.\) Segue que \(\langle a \rangle\subsetneq{\langle a_{1} \rangle}\) e \(a_{1}\) não é produto de elementos irredutíveis.
Continuando este processo indutivamente, obteremos uma cadeia de ideais \(\langle a \rangle \subsetneq{\langle a_{1} \rangle}\subsetneq{\langle a_{2} \rangle}\subsetneq{\cdots}\)
tal que cada \(a_{i}\) não é produto de elementos irredutíveis. Seja \(\mathfrak{i} = \bigcup_{i=1}^{\infty}{}\). Já que A é D.I.P., seja \(x\in A\) tal que \(\mathfrak{i} = \langle x \rangle\).

  Caso \(x\in \mathfrak{i},\) existe \(n\in \mathbb{N}\) tal que \(x\in \langle a_{n} \rangle\) e, então, \(\mathfrak{i} = \langle x \rangle\subseteq{\langle a_{n} \rangle}.\) Portanto,
 \(\langle a_{n} \rangle = \langle a_{n+1} \rangle = \cdots = \mathfrak{i}\), o que é um absurdo. Portanto, todo elemento irredutível de A
pode ser escrito como produto de elementos irredutíveis. \qedsymbol
\end{proof*}
\newpage

\section{Aula 18 - 30/10/2023}
\subsection{Motivações}
\begin{itemize}
  \item Aneis Noetherianos;
  \item Anel de Inteiros Algébricos.
\end{itemize}
\subsection{Aneis Noetherianos}
\begin{prop*}
  Todo elemento de um anel quadrático é produto de elementos irredutíveis.
\end{prop*}
\begin{proof*}
  Seja \(\mathbb{Z}[\alpha ] = \{c + d\alpha : c, d\in \mathbb{Z}\}, \alpha \) raiz de \(x^{2} + ax + b\in \mathbb{Z}[x]\) irredutível.
Tome \(x\in \mathbb{Z}[\alpha ]\) tal que x não é o produto dos elementos irredutível com menor norma \(N(c+d\alpha ) = [c^{2}-acd + bd^{2}]\).
Como x não é irredutível, \(x=yz\), com \(y,z\not\in \mathbb{Z}[\alpha ]^{*},\) em que um deles não é o produto de elementos
irredutíveis.

  Agora, note que \(N(y)\neq1\) e \(N(z)\neq1\). De fato, se \(N(y)=1, y\in \mathbb{Z}[\alpha ]^{*}\) e então \(x=yz\) seria irredutível.
Pela propriedade \(N(x) = N(y)N(z)\) e \(N(y) > 1, N(z) > 1\), temos \(N(y) < N(x)\) e \(N(z) < N(x)\). 
Pelo menos um deles não é produto de irredutíveis, contrariando a minimalidade de x. Portanto, x deve ser o produto
de elementos irredutíveis. \qedsymbol
\end{proof*}
\begin{example}
  2 é elemento irredutível em \(\mathbb{Z}[\sqrt[]{-5}].\)

  De fato, suponha que 2 possa ser escrito como \(2 = (a + b\sqrt[]{-5})(c+d\sqrt[]{-5}).\) Então,
 \begin{align*}
   &N(2) = N(a+b\sqrt[]{-5})N(c+d\sqrt[]{-5})\\
   &4 = (a^{2} + 5b^{2})(c^{2}+5d^{2}).
 \end{align*}
 Então, \(c^{2} + 5d^{2}\mid 4\), tal que \(c^{2} + 5d^{2} = 1, 2, 4.\). Caso \(a^{2}+5b^{2} = 1, b = 0 e a = \pm 1\in \mathbb{Z}[\sqrt[]{-5}]^{*}.\) 
Com isso, \(a^{2} + 5b^{2} = 4\) e \(a = \pm 2\) e \(d=0.\) Assim, \(2=(\pm1)(\pm2).\)
O caso \(c^{2} + 5d^{2} = 2\) não tem soluções e o caso \(c^{2} + 5d^{2} = 4\) é simétrico ao primeiro caso.
\end{example}
\begin{def*}
  Dizemos que um anel A é \textbf{noetheriano} se todo ideal de A é finitamente gerado. \(\square\)
\end{def*}
\begin{prop*}[Exercício]
  As propriedades a seguir são equivalentes:
 \begin{itemize}
  \item[1)] A é noetheriano;
  \item[2)] Toda cadeia de ideais de A: \(\mathfrak{i}_{1}\subseteq{}\mathfrak{i}_{2}\subseteq{}\mathfrak{i}_{3}\subseteq{\cdots}\) estabiliza, i.e.,
existe \(n\in \mathbb{N}\) tal que \(\mathfrak{i}_{n} = \mathfrak{i}_{n+1} = \cdots.\)
  \item[3)] Todo conjunto S de ideias de A tem um elemento maximal.
 \end{itemize}
\end{prop*}
\begin{theorem*}[Base de Hilbert]
  Se A é noetheriano, então \(A[x]\) também é noetheriano.
\end{theorem*}
\begin{prop*}[Exercícios]
 \begin{itemize}
  \item[1)] Seja A noetheriano e \(\mathfrak{i}\trianglelefteq{A}.\) Então, \(A/\mathfrak{i}\) é noetheriano.
  \item[2)] Seja A um domínio noetheriano. Então, todo elemento de A é produto de irredutíveis
 \end{itemize}
\end{prop*}
\begin{example}
 \begin{itemize}
  \item[1)] Mostre que \(\mathbb{Z}[\sqrt[]{-3}]\) não é D.I.P. (Mostre que \(\langle 2, 1 + \sqrt[]{-3} \rangle\) não é principal);
  \item[2)] Seja \(F\subseteq{\mathbb{C}}\) subcorpo. Então, \(\mathbb{Q}\subseteq{F}.\);
  \item[3)] Seja F um subcorpo de \(\mathbb{C}\) tal que \([F:\mathbb{Q}]<\infty\). (Chamamos estes corpo de \textbf{corpos globais});
  \item[4)] Seja \(\mathcal{O}_{F}\coloneqq \{\alpha \in F: \exists f(x)\in \mathbb{Z}[x]\} \text{ mônico irredutível e } f(\alpha ) = 0\). Mostre que
 \(\mathcal{O}_{F}\) é um anel. (Chamamos este anel de \textbf{anel de inteiros algébricos} de F). Além disso, \(\mathrm{Frac}(\mathcal{O}_{F}) = F.\);
  \item[5)] Se \(F = \mathbb{Q}[\sqrt[]{d}],\) d livre de quadrados, então \(\mathcal{O}_{F} = \mathcal{O}_{d}.\)
 \end{itemize}
\end{example}
\begin{example}
 \begin{itemize}
  \item[1)] \(\mathcal{O}_{-3} = \{\pm1, \pm \frac{1\pm \sqrt[]{-3}}{2}\}\);
  \item[2)] \(\mathcal{O}_{-1} = \{\pm1, \pm i\}\);
  \item[3)] \(2, 3\in \mathcal{O}_{-19}\) são irredutíveis;
  \item[4)] Seja p primo. Se \(p\equiv 1 \mod 4,\) então \(p\in \mathbb{Z}[i]\) não é irredutível.
 \end{itemize}
\end{example}
\newpage

\section{Aula 19 - 01/11/2023}
\subsection{Motivações} 
\begin{itemize}
  \item Teorema de Dedekind;
  \item Anéis quasi-euclidianos;
  \item Domínios de Fatoração Única.
\end{itemize}
\subsection{O Teorema de Dedekind}
\begin{theorem*}
  \(\mathcal{O}_{-19}\) não é euclidiano.
\end{theorem*}
\begin{proof*}
  Suponha que exista \(N':\mathcal{O}_{-19}\setminus{\{0\}}\rightarrow \mathbb{N}\) tal que \(\mathcal{O}_{-19}\) é um domínio euclidiano com norma \(N'.\) Seja \(m\in \mathcal{O}_{-19}, m\neq \pm1, 0,\)
tal que 
  \[
    N'(m) = \min\{N'(a): a\in \mathcal{O}_{-19}, a\neq \pm1, 0\}.
  \]
  Pelo algoritmo da divisão, existem \(q, r\) tais que \(2 = mq + r\), com \(N'(r) < N'(m)\)
ou \(r=0\).

  Como m tem menor norma dentre os elementos diferentes de \(0\) e \(\pm 1\), então \(r=0, \pm1\). Se \(r=1\), temos
 \(mq = 1\) implica que \(m\in \mathcal{O}_{-19}^{*},\) um absurdo. Assim, \(r=0, -1\) e,
 logo, \(mq = 2, 3\) e, já que \(2, 3\) são irredutíveis, \(q\in \mathcal{O}_{-19}^{*} = \{\pm1\}.\) Com isso,
 \(m=\pm2, \pm 3\in \mathbb{Z}.\)
  
  Agora, seja \(\theta = \frac{1 + \sqrt[]{-19}}{2}.\) Então, existem \(q', r'\) tais que \(\theta  = mq'+r'.\) Usando o mesmo
argumento que o feito acima, temos \(r'=0, \pm1.\) Destarte, \(mq'=\theta - r', r'=0, \pm1.\) Seja \(q'= a + b\theta ,\)
em que \(a, b\in \mathbb{Z}.\) Logo, 
  \[
    m(a + b\theta ) = -r + \theta  \Rightarrow ma + mb\theta = -r + \theta  \Rightarrow mb = 1,
  \]
mas, note que, como \(m = \pm2, \pm3\), temos \(b\not\in \mathbb{Z}\), ou seja, não existe \(q'\) que satisfaça \(\theta  = mq' + r'\) e, portanto,
 \(\mathcal{O}_{-19}\) não é euclidiano. \qedsymbol
\end{proof*}
\begin{def*}
  Um domínio A é dito \textbf{quasi-euclidiano} se existe uma aplicação \(d:A\setminus{\{0\}}\rightarrow \mathbb{N}\) tal que para todos \(a, b\in A, b\neq0\),
existem \(p, q, r\in A\) com \(p\neq0\) tais que: 
  \[
    ap = bq + r\quad\&\quad d(r) < d(b) \text{ ou } (p=1 e r=0).\quad\square
  \]
\end{def*}
\begin{lemma*}
  Todo domínio quasi-euclidiano é D.I.P.
\end{lemma*}
\begin{proof*}
  Seja \((0)\neq \mathfrak{i}\trianglelefteq{A}\) e \(b\in \mathfrak{i}\) tal que \(d(b) = \min\{d(x):x\in \mathfrak{i}\setminus{\{\}}\}\).
Tome \(a\in \mathfrak{i}.\) Então, existem \(p, q, r\in A\) tais que \(p\neq 0\) e \(pa = bq + r\), com \(d(r) < d(b)\) ou \(p=1, r=0.\)
Caso \(r\neq 0,\) então \(r = pa-bq\in \mathfrak{i}\) e, assim, \(d(r) < d(b),\) um absurdo com a minimalidade de b em \(\mathfrak{i}.\)

  Com isso, \(r=0\) e, então, \(a = ap = bq\), tal que \(a\in \langle b \rangle.\) Além disso, segue que \(\mathfrak{i}\subseteq{\langle b \rangle}.\)
Por outro lado, como temos \(b\in \mathfrak{i}, \mathfrak{I} = \langle b \rangle\). Portanto, todo ideal de A é principal, ou seja, A é D.I.P. \qedsymbol
\end{proof*}
\begin{theorem*}
  \(\mathcal{O}_{-19}\) é quasi-euclidiano. (Em particular, é D.I.P).
\end{theorem*}
\begin{proof*}
  Considere a norma \(N:\mathcal{O}_{-19}\rightarrow \mathbb{N}\) definida por \(N(a+b\theta ) = a^{2} + ab + 5b^{2}.\) Essa norma é a
restrição dos complexos em \(\mathcal{O}_{-19}.\) De fato,
\begin{align*}
  a + b\theta &= a +b \biggl(\frac{1+\sqrt[]{-19}}{2}\biggr) = \biggl(a + \frac{b}{2}\biggr) + \frac{\sqrt[]{19}}{2}ib.\\
              &= a' + b'i\in \mathbb{C}.
\end{align*}
  Representando \(a + b\theta \) desta forma, conseguimos calcular a norma como 
 \begin{align*}
   |a+b\theta |^{2} &= \biggl(a + \frac{b}{2}\biggr)^{2} + \biggl(\frac{\sqrt[]{19}}{b}\biggr)^{2}\\
                    &= a^{2} + ab + \frac{b^{2}}{4} + \frac{19}{4}b^{2}\\
                    &= a^{2} + ab + 5b^{2} = N(a + b\theta ).
 \end{align*}
 Sejam \(a, b\in a, b\neq0.\) Se \(a\in \langle b \rangle, a = bq\) para algum \(q\in A\) (ou seja, \(p=1, r = 0\)).
Caso \(a\not\in \langle b \rangle\), a existência de \(p, q, r\in A\) com \(p\neq 1\) e \(r\neq 0\) depende da condição 
a seguir:
\begin{align*}
  N(r) &= N(ap-bq) < N(p)\\
      \Longleftrightarrow & N(b)N \biggl(\frac{a}{b}p - q\biggr) < N(b)\\
      \Longleftrightarrow & N \biggl(\frac{a}{b}p - q\biggr) < 1\\
      \Longleftrightarrow & \biggl|\frac{a}{b}p - q\biggr| < 1. \quad\text{\qedsymbol}
\end{align*} 
\end{proof*}
\begin{lemma*}
  Para todo \(a, b\in A = \mathcal{O}_{-19},\) com \(b\neq0, a\not\in \langle b \rangle,\) existem \(p, q\in A\) tais que 
 \(\biggl|\frac{a}{b}p - q\biggr| < 1.\)
\end{lemma*}
\subsection{Domínios de Fatoração Única}
\begin{def*}
  Seja A um domínio e sejam \(a, b\in A\).
\begin{itemize}
  \item[1)] Dizemos que a \textbf{divide} b (ou b é \textbf{múltiplo} de x), e escrevemos \(a\mid b,\) se existe \(c\in A\) tal que \(b = ac.\)
  \item[2)] Dizemos que a, b são \textbf{associados} se existe \(c\in A^{*}\) tal que a = bc. \(\square\)
\end{itemize}
\end{def*}
\begin{prop*}[Exercício]
\begin{itemize}
  \item[1)] \(a\mid b \Longleftrightarrow \langle b \rangle \subseteq{\langle a \rangle}\);
  \item[2)] a, b são associados se, e somente se, \(\langle b \rangle = \langle a \rangle\);
  \item[3)] a é irredutível se, e somente se, todo divisor de a é associado, ou a 1, ou a a.
\end{itemize}
\end{prop*}
\begin{def*}
  Dizemos que \(0\neq a\in A\setminus{A^{*}}\) é \textbf{primo} se \(a\mid xy \Rightarrow a\mid x\) ou \(a\mid y.\quad\square\)
\end{def*}
\begin{prop*}[Exercício]
  Um elemento \(a\in A\) é primo se, e somente se, \(\langle a \rangle\in \mathrm{Spec}(A).\)
\end{prop*}
\begin{lemma*}
  Elementos primos são irredutíveis.
\end{lemma*}
\begin{proof*}
  Seja \(a\in A\) e \(a = xy, x, y\in A.\) Como \(a\mid a, a\mid xy\) e \(a\mid x\) ou \(a\mid y\).

  Suponha, sem perda de generalidade, que \(a\mid x\). Então, existe \(c\in A\) tal que \(x=ac\) e, assim, \(a=xy=acy\), de forma que \(cy = 1\), ou seja,
 \(y\in A^{*}.\) Portanto, a é irredutível. \qedsymbol
\end{proof*}
\begin{example}[Exercício]
  Seja \(A = \mathbb{Z}\) ou \(A = F[x]\), em que F é um corpo. Mostre que os elementos irredutíveis de A são primos.
\end{example}
\begin{def*}
  Um domínio A é chamado \textbf{Domínio de Fatoração Única (D.F.U.)} ou \textbf{Domínio Fatorial} se 
todo elemento não nulo \(a\in A\) pode ser escrito como produto de elementos irredutíveis de maneira única a menos 
da ordem dos fatores e de elementos associados. Mais explicitamente, dado \(a\in A\setminus{\{0\}},\)
\begin{itemize}
  \item[1)] Existem \(\pi_{1}, \cdots, \pi_{r}\) irredutíveis em A tais que \(a = \pi_{1}\cdot \dotsc \cdot \pi_{r}.\)
  \item[2)] Se \(a = \rho_{1}\cdot\dotsc \cdot \rho_{m} = \pi_{1}\cdot \dotsc \cdot \pi_{r}\), com \(\pi_{i}, \rho_{j}\) irredutíves em A, então \(r = m\)
e existe \(b\in S_{r}\) permutação tal que \(\pi_{i}\) é associado com \(\rho_{b(i)}\) para todo \(i=1, \cdots, r. \quad\square\)
\end{itemize}
\end{def*}
\begin{prop*}
  Num D.F.U., elementos irredutíveis são primos.
\end{prop*}
\begin{proof*}
  Seja a irredutível e \(a\mid xy\). Então, existe \(c\in A\) tal que \(ac = xy.\) Coloque 
  \[
    x = \pi_{1} \cdot \dotsc \cdot \pi_{n},\quad y = \pi_{n+1}\cdot \dotsc \cdot \pi_{n+m}\quad\&\quad c=\rho_{1}\cdot \dotsc \rho_{r}.
  \]
  Temos \(a\mid ac = xy = \pi_{1}\cdot \dotsc \cdot \pi_{n+m},\) tal qual existe \(d\in A\) satisfazendo \(a = d\pi_{1}\cdot \dotsc \cdot \pi_{n+m}.\) Como a é 
irredutível e cada \(\pi_{i}\) é irredutível, \(d\in A^{*}.\) Assim, a é associado a \(\pi_{i}\) para algum \(i.\) Se \(i\leq n, a\mid x\) e, se \(n < i\leq m + n, a \mid y.\)
Portanto, a é primo. \qedsymbol
\end{proof*}
\begin{prop*}
  Se A é um domínio Noetheriano e todo elemento irredutível é primo, então A é D.F.U.
\end{prop*}
\begin{proof*}
  Sabemos que, se A é noetheriano, todo elemento de A é produto de elementos irredutíveis. Precisamos, então, provar apenas que essa representação é única.

  Seja \(a\in A\setminus{\{0\}}\) e \(a = \pi_{1}\cdot \dotsc \cdot \pi_{n} = \rho_{1}\cdot \dotsc \cdot \rho_{m},\) em que \(\rho_{j}, \pi_{i}\) são elementos
irredutíveis em A. Faremos essa prova por indução. Se n = 1, então \(a = \pi_{1}\) e, como \(a\mid a\), \(\pi_{1}\mid\rho_{1}\cdot \dotsc \cdot \rho_{m}\). Já que 
 \(\pi_{1}\) é primo, temos \(\pi_{1}\mid\rho_{j}\) para algum \(1\leq j\leq m.\) Sem perda de generalidade, podemos supor \(j=1\) (basta reordenar via permutação). Então, existe \(c\in A\) tal que \(\rho _{1} = c\pi_{1}.\) Como \(\rho_{j}\) é irredutível,
 \(c\in A^{*},\) pois \(\pi_{1}\) é irredutível, ou seja, \(\pi_{1}\not\in A^{*}\).

  Suponha que \(m > 1,\) então 
  \[
    \pi_{1} = \rho_{1}\cdot \dotsc \cdot \rho_{m} \Rightarrow c\pi_{1}\cdot \rho_{2}\cdot \dotsc \cdot \rho_{m} = \pi_{1} \Rightarrow c\rho_{2} \cdot \dotsc \cdot \rho_{m} = 1.
  \]
  Assim, temos \(\rho_{2}\in A^{*},\) um absurdo. Logo, \(m=1\) e \(\pi_{1} = \rho_{1}\). 

  Agora, seja \(n > 1\). Segue que \(\pi_{1}\mid \rho_{1}\cdot \dotsc \cdot \rho_{m}\) e existe j tal que \(\pi_{1}\mid\rho_{j}.\) Podemos supor novamente
que j=1, tal que, pelo mesmo argumento anterior, existe \(c\in A^{*}\) tal que \(\pi_{1} = c\rho_{1}\) e, assim, 
  \[
    \pi_{1}\cdot \dotsc \cdot \pi_{n} = c\rho_{1}\cdot \rho_{2} \cdot \dotsc \cdot \rho_{m} \Rightarrow \pi_{2}\cdot \dotsc \pi_{n} = \rho_{2}\cdot \dotsc \cdot \rho_{m}
  \]
  Renumerando os termos da última igualdade tal que eles vão de 1 até n-1 do lado esquerdo e de 1 até m-1 do lado direito, pela hipótese de indução, existe \(b\in S_{m-1}\)
tal que \(\pi_{i} = c_{i}\rho_{b(i)}\) e \(n-1 = m-1.\) Portanto, n = m e os termos são associados. \qedsymbol
\end{proof*}
\newpage

\section{Aula 20 - 06/11/2023}
\subsection{Motivações}
\begin{itemize}
  \item Domínios de Fatoração e primos;
  \item Primos de \(\mathcal{O}_{-3}\);
  \item M.D.C. e M.M.C.
\end{itemize}
\subsection{D.F.U.'s e Exemplos}
 \begin{theorem*}
  Se A é um D.I.P., então A é D.F.U.
 \end{theorem*}
\begin{proof*}
  Como um D.I.P. é noetheriano, basta mostrar que elementos irredutíveis são primos.
Seja \(\pi \in A\) irredutível e seja \(\pi \mid ab.\) Suponha que \(\pi\) não divide a e considere
o ideal \(\langle \pi , a \rangle.\) Seja 
  \[
    \langle \pi , a \rangle = \langle d \rangle.
  \]
  Temos 
  \[
    \pi \in \langle d \rangle \Rightarrow \exists d'\in A: \pi  = dd'.
  \]
  Assim, \(d\in A^{*},\) ou \(d'\in A^{*}\), tal que \(d'^{-1}\pi  = d\), do que segue que \(\pi \mid d\) e,
como \(d\mid a,\) temos \(\pi \mid a.\) Contradição. Logo, a única possibilidade é \(d\in A^{*}\), donde segue que 
  \[
    \langle \pi , a \rangle = \langle d \rangle = \langle 1 \rangle = A.
  \]
  Já que \(1\in A, 1\in \langle \pi , a \rangle\), ou seja, existem r e s tais que 
  \[
    1 = r\pi +sa.
  \]
  Desta forma,
 \begin{align*}
   b = b \cdot 1 &= b(r\pi + sa)\\
   \Rightarrow & b = br\pi + sab.
 \end{align*}
 Como \(\pi \mid ab\) e \(\pi \mid \pi ,\) portanto, 
  \[
    \pi \mid(rb\pi +sab) = b.\quad \text{\qedsymbol}
  \]
\end{proof*}
\begin{example}
 \begin{itemize}
  \item[1)] Sabe-se que \(\mathbb{Z}\) é D.F.U;
  \item[2)] Se F é um corpo, \(F[x]\) é um D.F.U.;
  \item[3)] Domínio Euclidianos são D.F.U.;
  \item[4)] Domínios quasi-euclidianos são D.F.U. (em particular, \(\mathcal{O}_{-19})\);
  \item[5)] Seja A um D.I.P. Se \(0\neq\mathfrak{p}\in \mathrm{Spec}(A)\), então \(A/\mathfrak{p}\) é D.I.P. e é um corpo;
  \item[6)] Se A é um D.I.P. e \(S\subseteq{A}\) é um conjunto multiplicativo, então \(S^{-1}A\) é um D.I.P. e, assim, um D.F.U. Seja \(\mathfrak{p}\in \mathrm{Spec}(A), \mathfrak{p} = \langle \pi  \rangle, \pi\) primo. Considere o conjunto 
  \[
    A_{\mathfrak{p}} = \biggl\{\frac{a}{b}: a, b \in A, \pi \text{ não divide } b\biggr\}.\quad \text{(\textbf{``Discrete Valuation Ring'' - Anel de Avaliação Discreta})}
  \]
  Então, se \(\mathfrak{p} = (0),\) segue que \(A_{\mathfrak{p}} = \mathrm{Frac}(A)\) e, se \(\mathfrak{p}\neq (0),\) então
 \(A_{\mathfrak{p}}\) é anel local com ideal maximal 
 \[
   \mathfrak{p}A_{\mathfrak{p}} = \biggl\{\frac{a}{b}: a,b\in A, \pi \mid a, \pi \text{ não divide } b\biggr\}
 \]
 Assim, \(\mathrm{Spec}(A_{\mathfrak{p}}) = \{(0), \mathfrak{p}A_{\mathfrak{p}}\}\) e os elementos irredutíveis primos de \(A_{\mathfrak{p}}\)
são associados com \(\pi \). Além disso, dado \(\mathfrak{j}\trianglelefteq{A_{\mathfrak{p}}}, \mathfrak{j} = \langle x \rangle, x\in A,\) então
 \(x = \pi ^{n}\) para algum n, pois só temos um elemento irredutível. Com isso, \(\mathfrak{j} = \langle \pi ^{n} \rangle = \langle \pi  \rangle^{n} = (\mathfrak{p}A_{\mathfrak{p}})^{n} = \mathfrak{p}^{n}A_{\mathfrak{p}}\).
 \item[7)] Tome \(p\in A\) primo, A D.I.P. e coloque 
  \[
    S = \{p^{n}: n\in \mathbb{N}\cup \{0\}\} = \{1, p, p^{2}, p^{3}, \cdots\}.
  \]
  O conjunto \(A \biggl[\frac{1}{p}\biggr] = S^{-1}A\) é um D.I.P. e, além disso, 
  \[
    S^{-1}A = \biggl\{\frac{a}{p^{n}}:a\in A, n\in \mathbb{N}\cup\{0\}\biggr\}.
  \]
  Se \(\mathcal{P}\) é o conjunto dos primos de A, então o conjunto dos primos de \(A \biggl[\frac{1}{p}\biggr]\) é 
 \(\mathcal{P}\setminus{\{p\}}\).
 \end{itemize}
\end{example}
  É importante notar que \textbf{nem todo D.F.U.} é D.I.P. Antes disso, porém, vamos ver como encontrar
todos os ideais primos dos inteiros de Gauss:
\begin{example}
  Considere \(\mathcal{O}_{-1} = \mathbb{Z}[i] = \{a + ib: a, b\in \mathbb{Z}\}\subseteq{\mathbb{C}}\) e considere o elemento \(13\in \mathbb{Z}\subseteq{\mathbb{Z}[i]}\).
  Em \(\mathbb{Z},\) 13 é primo, mas em \(\mathbb{Z}[i]\), temos 
  \[
    13 = 9 + 4 = (3+2i)(3-2i).
  \]
  Como \(\mathbb{Z}[i]^{*} = \{\pm1, \pm i,\}\) temos \(3-2i, 2+2i\not\in \mathbb{Z}[i]^{*}\) e, assim,
13 é o produto de dois elementos não invertíveis, fazendo com que 13 não seja primo nos inteiros de Gauss.
\end{example}
\begin{example}
  Seja \(\mathcal{O}_{-3} = \{a + b\theta : a, b\in \mathbb{Z}\} \subseteq{\mathbb{C}}, \theta  = \frac{1 + \sqrt[]{-3}}{2}\). Vimos que 
  \[
    \mathcal{O}_{-3} = \biggl\{\pm 1, \frac{\pm1 \pm \sqrt[]{-3}}{2}\biggr\}.
  \]
Então, 

\begin{center}
  \begin{table}[h!]
  \caption{Casos dos elementos}
  \centering
    \begin{tabular}{| c | c |}
      \hline
      Produtos de \(\mathcal{O}_{-3}^{*}\) & Resultado\\
      \hline
      \(\theta ^{2} - \theta  + 1 = 0\) & \(\mathbb{Z} \subseteq{}\mathcal{O}_{-3}\)\\
      \hline
      \(\theta (\theta -1)=-1\) & \(-2\in \mathbb{Z}\) é primo\\
      \hline
      \(-\theta (\theta -1) = 1\) & \(-2\in \mathcal{O}_{-3}, 2 = (1-\sqrt[]{-3})\underbrace{\biggl(\frac{1+\sqrt[]{-3}}{2}\biggr)}_{\in \mathcal{O}_{-3}^{*}} = (1+\sqrt[]{-3})\underbrace{\biggl(\frac{1-\sqrt[]{-3}}{2}\biggr)}_{\in \mathcal{O}_{-3}^{*}}\)\\
      \hline
    \end{tabular}
  \end{table}
\end{center}

  Assim, vemos que \(2\) e \((1\pm\sqrt[]{-3})\) são associados. Como \(2\in \mathcal{O}_{-3}\) é primo, \(1\pm\sqrt[]{-3}\) são primos também. Logo,
 \(\langle 2 \rangle = \langle 1 + \sqrt[]{-3} \rangle = \langle 1 - \sqrt[]{-3} \rangle.\) Para ver isso,
 utiliza-se a norma induzida dos complexos em \(\mathcal{O}_{-3}\) e escreve-se 2 e 4 como elementos de \(\mathcal{O}_{-3}\).
\end{example}
\subsection{Máximo Divisor Comum e Mínimo Múltiplo Comum}
\begin{def*}
  Sejam A um domínio e \(a_{1}, \cdots, a_{n}\in A\setminus{\{0\}}\). O \textbf{maior divisor comum (m.d.c)} de \(a_{1}, \cdots, a_{n}\)
é um elemento \(d = \mathrm{mdc}(a_{1}, \cdots, a_{n})\in A\setminus{\{0\}}\) tal que 
\begin{itemize}
  \item[i)] \(d\mid a_{i}\) para todo \(i = 1, \cdots, n;\)
  \item[ii)] Se \(a\in A\setminus{\{0\}}\) e \(a\mid a_{i}\) para todo \(i=1, \cdots, n\), então \(a\mid d.\quad\square\)
\end{itemize}
\end{def*}
\begin{def*}
  Sejam A um domínio e \(a_{1}, . ., a_{n}\in A\setminus{\{0\}}\). O \textbf{menor múltiplo comum} de \(a_{1}, \cdots, a_{n}\) é um elemento
 \(m = \mathrm{mmc}(a_{1}, \cdots, a_{n})\in A\setminus{\{0\}}\) tal que
\begin{itemize}
  \item[i)] \(a_{i}\mid m\) para todo \(i=1, \cdots, n\);
  \item[ii)] Se \(m'\in A\setminus{\{0\}}\) e \(a_{i}\mid m'\) para todo \(i=1, \cdots, n\), então \(m\mid m'.\quad\square\)
\end{itemize}
\end{def*}
\begin{prop*}
 \begin{itemize}
  \item[1)] Se \(d, d'\) são maiores divisores comuns de \(a_{1}, \cdots, a_{n}\in A\), então eles são associados (existe \(s\in A^{*}: d = d's\));
  \item[2)] Se \(m, m'\) são menores múltiplos comuns de \(a_{1}, \cdots, a_{n}\in A\), então eles são associados (existe \(s\in A^{*}: m = m's\)).
 \end{itemize}
\end{prop*}
\begin{example}
  Considere \(2, 3\in \mathbb{Z}.\) Então, \(\mathrm{mdc}(2, 3) = \pm1\)
\end{example}
\begin{theorem*}[Exercício]
  Sejam A um D.F.U. e \(a_{1}, \cdots, a_{n}\in A\setminus{\{0\}}.\) Suponha que 
  \[
    a_{j} = a_{j}\pi_{1}^{e_{j1}}\cdot\dotsc \cdot \pi_{n}^{e_{jn}} \quad (e_{ji} = 0 \text{ se } \pi_{j} \text{ não divide } a_{j}),
  \]
  em que \(\pi_{i}\) são irredutíveis e dois-a-dois distintos. Com isso, 
  \begin{align*}
    &i)\quad\mathrm{mdc}(a_{1},\cdots,a_{n}) = \prod\limits_{i=1}^{n}\pi_{i}^{\min\{e_{1i}, \cdots, e_{nj}\}};\\
    &ii)\quad \mathrm{mmc}(a_{1}, \cdots, a_{n}) = \prod\limits_{i=1}^{n}\pi_{i}^{\max\{e_{1i}, \cdots, e_{ni}\}};\\
    &iii)\quad a_{1}\cdot \dotsc \cdot  a_{n} = \mathrm{mdc}(a_{1}, \cdots, a_{n})\cdot \mathrm{mmc}(a_{1}, \cdots, a_{n});\\
    &iv)\quad \text{Se A é um D.I.P., então }  \langle a_{1}, \cdots, a_{n} \rangle = \langle \mathrm{mdc}(a_{1}, \cdots, a_{n}) \rangle.
  \end{align*}
\end{theorem*}
\begin{prop*}
  Os primos de \(\mathcal{O}_{-1} = \mathbb{Z}[i]\), a menos de associação, são:
 \begin{itemize}
  \item[1)] Primo \(p\in \mathbb{Z}\) tal que \(p\equiv \beta  \mod 4\)
  \item[2)] \(a+ib\in \mathbb{Z}[i]\) tais que \(a^{2} + b^{2} = p, p\) primo com \(p = 2\) ou \(p\equiv 1 \mod 4\)
 \end{itemize}
\end{prop*}
\begin{proof*}
  Seja \(p\equiv 3\mod 4\) e seja 
  \[
    p = \alpha \cdot \beta , \alpha , \beta \in \mathbb{Z}[i].
  \]
  Temos \(N(p) = N(\alpha )\cdot N(\beta )\), tal que \(p^{2} = N(\alpha )\cdot N(\beta )\). Logo,
 \(N(\alpha ) = 1, p, ^{2},\) a partir donde analisamos casos.
\begin{itemize}
  \item Se \(N(\alpha ) = 1\), então \(\alpha \in \mathbb{Z}[i]^{*}\);
  \item Se \(N(\alpha ) = p^{2},\) então \(N(\beta ) = 1\) e \(\beta \in \mathbb{Z}[i]^{*}\);
\end{itemize}
  Resta o caso \(N(\alpha ) = N(\beta ) = p.\) Com este caso, segue que 
  \[
    \alpha = a + ib \Rightarrow N(\alpha ) = a^{2} + b^{2} = p,
  \]
  tal que \(\overline{a}^{2} + \overline{b}^{2} = 0\) em \(\mathbb{F}_{p} = \mathbb{Z}/\{p\}\). Consequentemente,
 \(p^{2}\mid a^{2}+b^{2} = p\), um absurdo. Logo, \(\overline{a}\neq \overline{0}\) e \(\overline{b}\neq \overline{0}\), o que implica que 
  \[
    \biggl(\frac{\overline{a}}{\overline{b}}\biggr)^{2} + 1 = 0 \text{ em } \mathbb{F}_{p}.
  \]
  Logo, \(x^{2} + 1 = 0\) tem uma raiz em \(\mathbb{F}_{p},\) o que ocorre apenas se \(p\equiv 1 \mod 4,\) que também
é um absurdo. Conclui-se, assim, que o caso \(N(\alpha ) = p\) é impossível, restando apenas as outras opções.

  Agora, seja \(a+ib\in \mathbb{Z}[i]\) primo tal que \(a^{2} + b^{2} = p, p\) um primo de \(\mathbb{Z}.\)
Vamos mostrar que \(p = 2\) ou \(p\equiv 1\mod 4\). Com efeito, considerando \(\pi  = \alpha \cdot \beta \), vale 
  \[
    N(\pi ) = N(\alpha )\cdot N(\beta ).
  \]
  Com isto, 
  \[
    p = a^{2} + b^{2} = N(\pi) = N(\alpha )\cdot N(\beta ) \Rightarrow N(\alpha ) = 1 \text{ ou } N(\beta ) = 1,
  \]
do que segue o fato de \(\pi \) ser um primo. Analogamente ao argumento acima, se \(N(\alpha ) = p\), então \(x^{2} + 1\) tem raiz em \(\mathbb{F}_{p}\),
tal que \(p\equiv 1 \mod 4\) ou \(p = 2\).
  
  Fica como exercício mostrar que não há outros primos além destes. \qedsymbol
\end{proof*}
\begin{example}
  Os elementos \(3, 7, 11, 19, 23, \dotsc\) são primos, assim como \(1\pm i, 1\pm 2i, 2\pm i, 3\pm 2i, 4\pm i, 5\pm 2i, \dotsc\) 
\end{example}
\newpage

\section{Aula 21 - 08/11/2023}
\end{document}

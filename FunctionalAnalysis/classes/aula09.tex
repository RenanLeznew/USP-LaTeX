\documentclass[../functional_analysis.tex]{subfiles}
\begin{document}
\section{Aula 09 - 09 de Setembro, 2025}
\subsection{Motivações}
\begin{itemize}
	\item Conjuntos Ortonormais;
	\item Bases de Hilbert;
	\item Transformações Lineares Ilimitadas.
\end{itemize}
\subsection{Conjuntos Ortonormais e Bases de Hilbert}
\begin{theorem*}
	Se \(\{u_{\alpha }\}_{\alpha \in A}\) for um conjunto ortonormal em H, as seguintes afirmações são equivalentes:
	\begin{itemize}
		\item[a)] (Completamento): se \(\langle u, u_{\alpha } \rangle=0\) para todo \(\alpha \in A\), então \(u=0\);
		\item[b)] \hypertarget{parseval_identity}{(Identidade de Parseval):} para todo u de H,
		      \[
			      \Vert u \Vert^{2}=\sum\limits_{\alpha \in A}^{}| \langle u, u_{\alpha } \rangle |^{2}.
		      \]
		\item[c)] Todo elemento u de H pode ser escrito como a soma de cada projeção dele sobre os elementos do conjunto ortonormal; isto é,
		      \[
			      u=\sum\limits_{\alpha \in A}^{} \langle u, u_{\alpha } \rangle u_{\alpha },
		      \]
		      onde a soma tem apenas um número enumerável de termos não nulos e converge independente da ordem dos termos.
	\end{itemize}
\end{theorem*}
\begin{proof*}
	\(a \Rightarrow c):\) se x pertence a H, seja \(\alpha_1,\alpha_2,\dotsc \) uma enumeração qualquer dos \(\alpha \)'s para os quais \(\langle u, u_{\alpha } \rangle\neq 0\). Pela \hyperlink{bessel_inequality}{\textit{Desigualdade de Bessel,}} garantimos a convergência da série
	\[
		\sum\limits_{j=1}^{\infty}| \langle u, u_{\alpha_{j}} \rangle |^{2}
	\]
	e, pelo \hyperlink{pythagorean_theorem}{\textit{Teorema de Pitágoras}},
	\[
		\biggl\Vert \sum\limits_{j=m}^{n}\langle u, u_{\alpha_{j}} \rangle u_{\alpha_{j}} \biggr\Vert^{2}=\sum\limits_{j=m}^{n}| \langle u, u_{\alpha_{j}} \rangle |^{2} \substack{ \\ \longrightarrow \\ m,n\to \infty}0.
	\]
	Por H ser completo, a série \(\sum\limits_{j=0}^{\infty}\langle u, u_{\alpha_{j}} \rangle u_{\alpha_{j}}\) é convergente. Assim, se
	\[
		v= u - \sum\limits_{j=0}^{\infty}\langle u, u_{\alpha_{j}} \rangle u_{\alpha_{j}},
	\]
	temos \(\langle v, u_{\alpha } \rangle =0\) para todo \(\alpha \). Logo, pelo item (a), v deve ser nulo.

	\(c \Rightarrow b):\) com a notação acima, assim como na prova da \hyperlink{bessel_inequality}{\textit{Desigualdade de Bessel}}, temos
	\[
		\Vert u \Vert^{2} -\sum\limits_{j=1}^{n}| \langle u, u_{\alpha_{j}} \rangle |^{2}=\biggl\Vert u - \sum\limits_{j=1}^{n}\langle u, u_{\alpha_{j}} \rangle u_{\alpha_{j}} \biggr\Vert^{2}\substack{ \\ \longrightarrow \\ n\to \infty}0.
	\]

	\(b \Rightarrow a):\) automático. \qedsymbol
\end{proof*}
\begin{def*}
	Um conjunto ortonormal tendo as propriedades do Teorema acima é chamado uma \textbf{base ortonormal de H}. \(\square\)
\end{def*}

Para cada \(\alpha \in A\), defina \(e_{\alpha }\in l^{2}(A)\) por:
\[
	e_{\alpha }(\alpha ) =1 \quad\&\quad e_{\alpha }(\beta )=0,\; \alpha \neq \beta .
\]
A família \(\{e_{\alpha }\}_{\alpha \in A}\) é ortonormal e, para qualquer \(f\in l^{2}(A)\), temos
\[
	\langle f, e_{\alpha } \rangle = f(\alpha ),
\]
donde segue que \(\{e_{\alpha }\}_{\alpha \in A}\) é uma base ortonormal.
\begin{theorem*}
	Todo espaço de Hilbert tem uma base ortonormal.
\end{theorem*}
\begin{proof*}
	Basta aplicar o \hyperlink{zorn_lemma}{\textit{Lema de Zorn}} para mostrar que a coleção de todos os conjuntos ortonormais, ordenados pela inclusão, tem um elemento maximal, que equivale à propriedade (a) da definição de uma base ortonormal. \qedsymbol
\end{proof*}

\begin{theorem*}
	Um espaço de Hilbert H é separável se, e somente se, tem uma base ortonormal enumerável. Neste caso, toda base ortonormal de H é enumerável.
\end{theorem*}
\begin{proof*}
	Se \(\{x_{n}\}\) é um conjunto enumerável denso em H, descartando recursivamente qualquer \(x_{i}\) que possa ser uma combinação linear de \(x_1,\dotsc ,x_{i-1},\) obtemos uma sequência linearmente independente \(\{y_{n}\}\) e, aplicando o processo de \hyperlink{gram_schmidt}{\textit{Ortogonalização}}, obtemos uma sequência ortonormal \(\{u_{n}\}\) que gera um subespaço denso em H, o qual consequentemente será uma base.

	Reciprocamente, se \(\{u_{n}\}\) é uma base ortonormal, as combinações lineares finitas dos \(u_{n}\)'s com coeficientes em um subconjunto enumerável e denso em \(\mathbb{C}\) forma um subconjunto enumerável e denso em H. Além disso, se \(\{v_{\alpha }\}_{\alpha \in A}\) é outra base ortonormal, para cada n o conjunto
	\[
		A_{n}=\{\alpha \in A:\; \langle u_{n}, v_{\alpha } \rangle\neq 0\}
	\]
	é enumerável. Portanto, pelo completamento de \(\{u_{n}\}\),
	\[
		A = \bigcup_{n=1}^{\infty}A_{n},
	\]
	mostrando que ele é enumerável. \qedsymbol
\end{proof*}
Em um espaço com produto interno H, a identidade de polarização expressa o produto interno em função da norma por meio de
\[
	\langle u, v \rangle = \frac{1}{4}[\Vert u+v \Vert^{2} - \Vert u-v \Vert^{2} + i \Vert u+iv \Vert^{2} - i \Vert u-iv \Vert^{2}],\quad \forall u, v\in H.
\]

\begin{def*}
	Se \(H_1\) e \(H_2\) são espaços de Hilbert com produtos escalares \(\langle \cdot , \cdot  \rangle_1\) e \(\langle \cdot , \cdot  \rangle_2\), uma \textbf{transformação unitária de \(H_1\) sobre \(H_2\)} é uma transformação linear sobrejetora \(U:H_1\rightarrow H_2\) que preserva o produto escalar, ou seja,
	\[
		\langle Ux, Uy \rangle_2 = \langle x, y \rangle_1. \quad \square
	\]
\end{def*}
\begin{exr}
	Mostre que toda transformação unitária é uma isometria e, reciprocamente, que toda isometria de \(H_1\) sobre \(H_2\) é uma transformação unitária.
\end{exr}
\begin{prop*}
	Seja \(\{e_{\alpha }\}_{\alpha \in A}\) uma base ortonormal de X. Então, a correspondência
	\[
		x\mapsto \hat{x},\quad \hat{x}(\alpha )\coloneqq \langle x, u_{\alpha } \rangle
	\]
	é uma transformação unitária em \(l^{2}(A).\)
\end{prop*}
\begin{proof*}
	A linearidade da transformação segue da linearidade do próprio produto interno, e o fato de ser uma isometria de H em \(l^{2}(A)\) é consequência da \hyperlink{parseval_identity}{\textit{Identidade de Parseval}}:
	\[
		\Vert x \Vert^{2} = \sum\limits_{\alpha \in A}^{}| \hat{x}(\alpha ) |^{2}.
	\]

	Agora, se \(f\in l^{2}(A)\), então
	\[
		\sum\limits_{\alpha \in A}^{} | f(\alpha ) |^{2}<\infty,
	\]
	e pelo \hyperlink{pythagorean_theorem}{\textit{Teorema de Pitágoras}}, as somas parciais da série \(\sum\limits_{\alpha \in A}^{}f(\alpha )u_{\alpha }\), da qual apenas um número enumerável de termos são não nulos, formam uma sequência de Caucho. Portanto,
	\[
		x = \sum\limits_{\alpha \in A}^{}f(\alpha )u_{\alpha }
	\]
	existe em H e \(\hat{x}=f.\) \qedsymbol
\end{proof*}
\subsection{Transformações Lineares Ilimitadas}

Nosso próximo tópico de estudo será alguns fatos elementares sobre as transformações lineares, não necessariamente limitadas.

Sejam X, Y espaços vetoriais normados sobre um mesmo corpo \(\mathbb{K}\) e \(A:D(A)\subseteq X\rightarrow Y\), uma transformação linear, onde D é um subespaço vetorial de X.

Definiremos alguns conceitos relacionados às transformações lineares que usaremos de notação:

\begin{def*}
	Para a transformação linear \(A:X\rightarrow Y\), diremos que:
	\begin{itemize}
		\item \(D(A)\) denotará o \textbf{domínio de A};
		\item \(\mathrm{Graf}(A) = \{(u, Au)\in X\times Y:\; u\in D(A)\}\subseteq X\times Y\) é o \textbf{gráfico de A};
		\item \(\mathrm{Im}(A)=\{Ax\in Y:\; x\in D(A)\}\) é a \textbf{imagem de A}; e
		\item \(N(A) = \mathrm{ker}(A) = \{x\in D(A):\; Ax = 0\}\) é o \textbf{núcleo de A}.
	\end{itemize}
	Além disso, se \(D(A)\) é denso em X, diremos que A é \textbf{densamente definida}. \(\square\)
\end{def*}

\begin{example}
	Seja \(X = \mathcal{C}([0, 1], \mathbb{R})\) com a norma usual, \(D(A) = \mathcal{C}^{1}([0, 1], \mathbb{R})\) e defina
	\begin{align*}
		A: & D(A)\subseteq X\rightarrow X                             \\
		   & u\longmapsto (Au)(s) = u'(s), \quad \forall s\in [0, 1].
	\end{align*}
	Note que esta transformação linear não é limitada!
\end{example}

\begin{def*}
	Uma transformação linear A é dita \textbf{fechada} se seu gráfico
	\[
		\mathrm{Graf}(A) = \{(x, A(x)): x\in X,\; A(x)\in Y\} \subseteq X\times Y
	\]
	é fechado em \(X\times Y.\; \square\)
\end{def*}
Uma consequência da definição é que
\begin{prop*}
	Seja \(A:D(A)\subseteq X\rightarrow X\) uma transformação linear. Então, A é fechada se, e somente se, para toda sequência \(\{(u_{n}, Au_{n})\}\) em \(D(A)\times Y\) que é convergente em \(X\times Y\) para algum \((u, v)\in X\times Y\), temos
	\[
		u\in D(A) \quad\&\quad Au = v.
	\]
\end{prop*}


Lembre-se que uma transformação linear \(A:D(A)\subseteq X\rightarrow Y\) é limitada se existe uma constante \(c\geq 0\) tal que
\[
	\Vert Au \Vert\leq c\Vert u \Vert,\quad \forall u\in D(A).
\]
Se A é limitada e densamente definida, podemos estendê-la a uma transformação linear limitada definida em X; neste caso, veremos, como consequência do \textit{Teorema do Gráfico Fechado,} que A não é fechada, a não ser que \(D(A) = X.\)

Além disso, provaremos que, como outra consequência do \textit{Teorema do Gráfico Fechado} que se X e Y forem espaços de Banach e \(A:X\rightarrow Y\) for uma transformação linear fechada, então A é limitada. Já temos um contraexemplo do caso em que X e Y não são de Banach, pois a transformação do exemplo anterior é fechada, mas ilimitada.

Apesar disso, em geral, estamos apenas interessados em estudar transformações lineares fechadas definidas entre espaços de Banach X, Y. Assim, as únicas transformações lineares fechadas \(A:D(A)\subseteq X\rightarrow Y\) que não são limitados estão definidos em um subespaço vetorial \(D(A)\subsetneq X\).

\begin{def*}
	Diremos que uma transformação linear A é \textbf{fechável} se \(\overline{\mathrm{Graf}(A)}\) é gráfico de uma transformação linear; neste caso, \(\overline{\mathrm{Graf}(A)}\) define uma transformação linear \(\overline{A}: D(\overline{A})\subseteq X\rightarrow Y\) e \(D(A)\subseteq D(\overline{A})\), com
	\[
		Au = \overline{A}u,\quad \forall u\in D(A).\; \square
	\]
\end{def*}
Da definição acima, segue que \(\overline{A}\) em si é fechada e que ela é a menor extensão fechada de A.

De forma análoga à proposição acima,
\begin{prop*}
	Seja \(A:D(A)\subseteq X\rightarrow Y\) uma transformação linear; então, A é fechável se, e somente se, para toda sequência \(\{(u_{n}, Au_{n})\}\) em \(D(A)\times Y\) que é convergente para \((0, v)\in X\times Y\), temos \(v=0.\)
\end{prop*}
\begin{proof*}
	Denotemos por \(\overline{A}\) a transformação linear tal que
	\[
		\mathrm{Graf}(\overline{A}) = \overline{\mathrm{Graf}(A)}
	\]
	e suponha que A é fechável com fecho \(\overline{A}\). Se \(\{u_{n}\}\) é uma sequência em \(D(A)\) que converge para zero e tal que \(\{Au_{n}\}\) é convergente com limite v, devemos ter
	\[
		0=\overline{A}0 = v.
	\]

	Por outro lado, suponha que, para toda sequência \(\{(u_{n}, Au_{n})\}\) em \(D(A)\times Y\) que é convergente para \((0, v)\) em \(X\times Y\), temos \(v = 0\). Se \((u, v),\; (u, \overline{v})\in \overline{\mathrm{Graf}(A)}\), existem sequências \(\{(u_{n}, Au_{n})\}\) e \(\{(\overline{u_{n}}, A \overline{u_{n}})\}\) tais que
	\[
		(u_{n}, Au_{n})\rightarrow (u, v) \quad\&\quad (\overline{u_{n}}, A \overline{u_{n}})\rightarrow (u, \overline{v})
	\]
	em \(X\times Y.\) Desta forma, \(u_{n}-\overline{u_{n}}\) é um ponto no domínio de A, e
	\[
		(u_{n}-\overline{u_{n}}, A(u_{n}-\overline{u_{n}}))\rightarrow (0, v-\overline{v}),
	\]
	levando à conclusão de que \(v=\overline{v}.\) Logo, \(\overline{\mathrm{Graf}(A)}\) é gráfico de um transformação linear. Portanto, o resultado segue. \qedsymbol
\end{proof*}
Em todas as aplicações de interesse, A é fechável com domínio denso.

\end{document}

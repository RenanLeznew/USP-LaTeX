\documentclass[../functional_analysis.tex]{subfiles}
\begin{document}
\section{Aula 04 - 14 de Agosto, 2025}
\subsection{Motivações}
\begin{itemize}
	\item Formas Geométricas de Hahn-Banach
\end{itemize}
\subsection{Teoremas de Hahn-Banach: forma geométrica}
\begin{theorem*}
	Seja X um espaço vetorial normado.
	\begin{itemize}
		\item[a)] Se M é um subespaço vetorial de X, fechado, e x é um ponto fora de M, então existe f no dual \(X^{*}\) de X tal que
		      \[
			      f(x)\neq 0 \quad\&\quad f|_{M}=0.
		      \]
		      Mais ainda, se \(\delta =\inf_{y\in M}\Vert x-y \Vert\), f pode ser tomada tal que
		      \[
			      \Vert f \Vert=1 \quad\&\quad f(x)=\delta ;
		      \]
		\item[b)] Se \(x\neq 0,\) existe \(f\in X^{*}\) tal que
		      \[
			      \Vert f \Vert=1\quad\&\quad f(x)=\Vert x \Vert;
		      \]
		\item[c)] Os funcionais lineares limitados em X separam pontos;
		\item[d)] Se x em X define um funcional \(\hat{x}:X^{*}\rightarrow \mathbb{C}\) por
		      \[
			      \hat{x}(f)=f(x),\quad \forall f\in X^{*},
		      \]
		      então a transformação
		      \[
			      x \substack{T \\ \longrightarrow \\ }\hat{x}
		      \]
		      é uma isometria linear de X em \(X^{**}.\)
	\end{itemize}
\end{theorem*}
\begin{proof*}
	a) Defina f em \(M+\mathbb{C}x\) por \(f(y+\lambda x)=\lambda \delta \), onde y é um elemento de M e \( \lambda \) é um número complexo. Então, \(f(x)=\delta ,\; f|_{M}=0\) e, para \(\lambda \) diferente de zero,
	\[
		|f(y+\lambda x)|=|\lambda |\Vert \lambda^{-1}y + x \Vert=\Vert y+\lambda x \Vert,
	\]
	e o resultado segue do \hyperlink{complex_hahn_banach}{\textit{Teorema de Hahn Banach}} com \(p(x)=\Vert x \Vert\) e M substituído por \(M + \mathbb{C}x\).

	b) É um caso especial de (a) com \(M=0.\)

	c) Se \(x\neq y\), existe f em \(X^{*}\) com \(f(x-y)\neq 0\), ou seja, \(f(x)\neq f(y)\).

	d) A transformação \(\hat{x}\), conforme fora definida, já é linear de \(X^{*}\) em \(\mathbb{K}.\) Com relação à transformação
	\[
		x\substack{T \\ \longrightarrow \\ }\hat{x},
	\]
	podemos ver sua linearidade a partir de
	\begin{align*}
		T(\alpha x+\beta y)(f)=(\hat{\alpha x+\beta y})(f) & =f(\alpha x+\beta y)                \\
		                                                   & =\alpha f(x)+ \beta f(y)            \\
		                                                   & =\alpha \hat{x}(f)+\beta \hat{y}(f) \\
		                                                   & =\alpha T(x)(f)+\beta T(y)(f),
	\end{align*}
	que é válido para toda f no dual de X. Note que
	\[
		|\hat{x}(f)|=|f(x)|\leq \Vert f \Vert \Vert x \Vert \Rightarrow \Vert \hat{x} \Vert \leq \Vert x \Vert.
	\]

	Por outro lado, de b, existe f em \(X^{*}\) tal que \(f(x)=\Vert x \Vert,\; \Vert f \Vert=1\) e, portanto,
	\[
		|\hat{x}(f)| = f(x) = \Vert x \Vert \quad\&\quad \Vert \hat{x} \Vert\geq \Vert x \Vert.\quad \text{\qedsymbol}
	\]
\end{proof*}

Com a notação do teorema acima, seja \(\hat{X}=\{\hat{x}:\; x\in X\}.\) Como \(X^{**}\) é Banach, segue que \([\hat{X}]^{-}\) é Banach e, consequentemente, o mapeamento
\[
	x\ni X \mapsto \hat{x}\in \hat{X}
\]
é uma imersão densa de X em \([\hat{X}]^{-}\). O espaço \([\hat{X}]^{-}\) é chamado \textbf{completamento} de X, e, se X é Banach, \([\hat{X}]^{-}=\hat{X}.\) Observe que, se X tem dimensão finita , então \(\hat{X}=X^{**}\), já que ambos têm a mesma dimensão, mas se ela for infinita nem sempre temos esta sorte; para os casos onde a igualdade se mantém, X é dito \textbf{reflexivo}. Em geral, identificamos X com \(\hat{X}\) e consideramos X um subespaço de \(X^{**}\), tal que a propriedade de X ser reflexivo torna-se \(X=X^{**}.\)
\begin{exr}
	Seja X um espaço vetorial e M um subespaço vetorial de X. Suponha que todo funcional linear limitado que se anula em M é identicamente nulo e mostre que \(\overline{M}=X.\)
\end{exr}

\begin{def*}
	Um \textbf{hiperplano afim} é um conjunto da forma
	\[
		H=\{x\in X:\; f(x)=\alpha \},
	\]
	onde \(f:X\rightarrow \mathbb{R}\) é um funcional linear não identicamente nulo e \(\alpha \in \mathbb{R}.\) Diremos que H é o hiperplano da equação \([f=\alpha ].\; \square\)
\end{def*}

\begin{prop*}
	O hiperplano da equação \([f=\alpha ]\) é fechado se, e somente se, f é contínua.
\end{prop*}
\begin{proof*}
	Um dos lados, de f ser continua resultar em H ser fechado, é esperado. Por outro lado, suponha que H é fechado, seja \(x_{0}\) um ponto fora de H e suponha que \(f(x_{0})<\alpha \); tome \(r>0\) tal que
	\[
		B_r(x_{0})\subseteq H ^{\complement}.
	\]
	Então, \(f(B_r(x_{0}))\) é convexo e não intercepta \(\{\alpha \}\) (Verifique!), tal que
	\[
		f(B_r(x_{0}))\subseteq \{x\in \mathbb{R}:\; x<\alpha \},
	\]
	do que segue que
	\[
		f(x_{0}+rz)<\alpha , \quad \forall z\in B_1(0)
	\]
	e
	\[
		f(z)<\frac{\alpha -f(x_{0})}{r}.
	\]
	Portanto,
	\[
		\Vert f \Vert\leq \frac{\alpha - f(x_{0})}{r}. \text{ \qedsymbol}
	\]
\end{proof*}

\end{document}

\documentclass[../functional_analysis.tex]{subfiles}
\begin{document}
\section{Aula 04 - 14 de Agosto, 2025}
\subsection{Motivações}
\begin{itemize}
	\item Formas Geométricas de Hahn-Banach
\end{itemize}
\subsection{Teoremas de Hahn-Banach: forma geométrica}
\begin{theorem*}
	Seja X um espaço vetorial normado.
	\begin{itemize}
		\item[a)] Se M é um subespaço vetorial de X, fechado, e x é um ponto fora de M, então existe f no dual \(X^{*}\) de X tal que
		      \[
			      f(x)\neq 0 \quad\&\quad f|_{M}=0.
		      \]
		      Mais ainda, se \(\delta =\inf_{y\in M}\Vert x-y \Vert\), f pode ser tomada tal que
		      \[
			      \Vert f \Vert=1 \quad\&\quad f(x)=\delta ;
		      \]
		\item[b)] Se \(x\neq 0,\) existe \(f\in X^{*}\) tal que
		      \[
			      \Vert f \Vert=1\quad\&\quad f(x)=\Vert x \Vert;
		      \]
		\item[c)] Os funcionais lineares limitados em X separam pontos;
		\item[d)] Se x em X define um funcional \(\hat{x}:X^{*}\rightarrow \mathbb{C}\) por
		      \[
			      \hat{x}(f)=f(x),\quad \forall f\in X^{*},
		      \]
		      então a transformação
		      \[
			      x \substack{T \\ \longrightarrow \\ }\hat{x}
		      \]
		      é uma isometria linear de X em \(X^{**}.\)
	\end{itemize}
\end{theorem*}
\begin{proof*}
	a) Defina f em \(M+\mathbb{C}x\) por \(f(y+\lambda x)=\lambda \delta \), onde y é um elemento de M e \( \lambda \) é um número complexo. Então, \(f(x)=\delta ,\; f|_{M}=0\) e, para \(\lambda \) diferente de zero,
	\[
		|f(y+\lambda x)|=|\lambda |\Vert \lambda^{-1}y + x \Vert=\Vert y+\lambda x \Vert,
	\]
	e o resultado segue do \hyperlink{complex_hahn_banach}{\textit{Teorema de Hahn Banach}} com \(p(x)=\Vert x \Vert\) e M substituído por \(M + \mathbb{C}x\).

	b) É um caso especial de (a) com \(M=0.\)

	c) Se \(x\neq y\), existe f em \(X^{*}\) com \(f(x-y)\neq 0\), ou seja, \(f(x)\neq f(y)\).

	d) A transformação \(\hat{x}\), conforme fora definida, já é linear de \(X^{*}\) em \(\mathbb{K}.\) Com relação à transformação
	\[
		x\substack{T \\ \longrightarrow \\ }\hat{x},
	\]
	podemos ver sua linearidade a partir de
	\begin{align*}
		T(\alpha x+\beta y)(f)=(\hat{\alpha x+\beta y})(f) & =f(\alpha x+\beta y)                \\
		                                                   & =\alpha f(x)+ \beta f(y)            \\
		                                                   & =\alpha \hat{x}(f)+\beta \hat{y}(f) \\
		                                                   & =\alpha T(x)(f)+\beta T(y)(f),
	\end{align*}
	que é válido para toda f no dual de X. Note que
	\[
		|\hat{x}(f)|=|f(x)|\leq \Vert f \Vert \Vert x \Vert \Rightarrow \Vert \hat{x} \Vert \leq \Vert x \Vert.
	\]

	Por outro lado, de b, existe f em \(X^{*}\) tal que \(f(x)=\Vert x \Vert,\; \Vert f \Vert=1\) e, portanto,
	\[
		|\hat{x}(f)| = f(x) = \Vert x \Vert \quad\&\quad \Vert \hat{x} \Vert\geq \Vert x \Vert.\quad \text{\qedsymbol}
	\]
\end{proof*}

Com a notação do teorema acima, seja \(\hat{X}=\{\hat{x}:\; x\in X\}.\) Como \(X^{**}\) é Banach, segue que \([\hat{X}]^{-}\) é Banach e, consequentemente, o mapeamento
\[
	x\ni X \mapsto \hat{x}\in \hat{X}
\]
é uma imersão densa de X em \([\hat{X}]^{-}\). O espaço \([\hat{X}]^{-}\) é chamado \textbf{completamento} de X, e, se X é Banach, \([\hat{X}]^{-}=\hat{X}.\) Observe que, se X tem dimensão finita , então \(\hat{X}=X^{**}\), já que ambos têm a mesma dimensão, mas se ela for infinita nem sempre temos esta sorte; para os casos onde a igualdade se mantém, X é dito \textbf{reflexivo}. Em geral, identificamos X com \(\hat{X}\) e consideramos X um subespaço de \(X^{**}\), tal que a propriedade de X ser reflexivo torna-se \(X=X^{**}.\)
\begin{exr}
	Seja X um espaço vetorial e M um subespaço vetorial de X. Suponha que todo funcional linear limitado que se anula em M é identicamente nulo e mostre que \(\overline{M}=X.\)
\end{exr}

\begin{def*}
	Um \textbf{hiperplano afim} é um conjunto da forma
	\[
		H=\{x\in X:\; f(x)=\alpha \},
	\]
	onde \(f:X\rightarrow \mathbb{R}\) é um funcional linear não identicamente nulo e \(\alpha \in \mathbb{R}.\) Diremos que H é o hiperplano da equação \([f=\alpha ].\; \square\)
\end{def*}
\begin{def*}
	Seja X um espaço vetorial sobre \(\mathbb{K}. \) Diremos que um subconjunto C de X é \textbf{convexo} se
	\[
		tx+(1-t)y\in C
	\]
	sempre que \(t\in [0,1]\) e \(x,y\in C.\; \square\)
\end{def*}
\begin{prop*}
	O hiperplano da equação \([f=\alpha ]\) é fechado se, e somente se, f é contínua.
\end{prop*}
\begin{proof*}
	Um dos lados, de f ser continua resultar em H ser fechado, é esperado. Por outro lado, suponha que H é fechado, seja \(x_{0}\) um ponto fora de H e suponha que \(f(x_{0})<\alpha \); tome \(r>0\) tal que
	\[
		B_r(x_{0})\subseteq H ^{\complement}.
	\]
	Então, \(f(B_r(x_{0}))\) é convexo e não intercepta \(\{\alpha \}\) (Verifique!), tal que
	\[
		f(B_r(x_{0}))\subseteq \{x\in \mathbb{R}:\; x<\alpha \},
	\]
	do que segue que
	\[
		f(x_{0}+rz)<\alpha , \quad \forall z\in B_1(0)
	\]
	e
	\[
		f(z)<\frac{\alpha -f(x_{0})}{r}.
	\]
	Portanto,
	\[
		\Vert f \Vert\leq \frac{\alpha - f(x_{0})}{r}. \text{ \qedsymbol}
	\]
\end{proof*}
\begin{def*}
	Se A, B são subconjuntos de X dizemos que o hiperplano de equação \([f=\alpha ]\) \textbf{separa A e B no sentido fraco} se
	\begin{align*}
		 & f(x)\leq \alpha ,\quad \forall x\in A  \\
		 & f(x)\geq \alpha ,\quad \forall x\in B.
	\end{align*}
	Analogamente, diremos que o hiperplano de equação \([f=\alpha ]\) \textbf{separa A e B no sentido forte} se existe \(\varepsilon >0\) tal que
	\begin{align*}
		 & f(x)\leq \alpha -\varepsilon ,\quad \forall x\in A               \\
		 & f(x)\geq \alpha+\varepsilon  ,\quad \forall x\in B.\quad \square
	\end{align*}
\end{def*}

\hypertarget{first_geometric_hahn_banach}{\begin{theorem*}[Hahn-Banach Geométrico: primeira forma]
		Seja X um espaço vetorial normado real e sejam A, B dois subconjuntos convexos, não vazios e disjuntos de X. Se A é aberto, existe um hiperplano fechado que separa A e B no sentido fraco.
	\end{theorem*}}
Antes de prová-lo, precisaremos dos seguintes lemas:
\hypertarget{minkowski_functional}{
	\begin{lemma*}[Funcional de Minkowski de um Convexo]
		Seja X um espaço vetorial normado sobre \(\mathbb{R}\) e C um aberto convexo de X contendo a origem. Para todo x em X, defina o \textbf{funcional de Minkowski} por
		\[
			p(x)\coloneqq \inf\{\alpha >0:\; \alpha^{-1}x\in C\}.
		\]
		Então, p é um funcional sub-linear e existe um número positivo M tal que
		\begin{align*}
			 & 0\leq p(x)\leq M\Vert x \Vert,\quad \forall x\in X \\
			 & C=\{x\in X:\; p(x)<1\}.
		\end{align*}
	\end{lemma*}
}
\begin{proof*}
	Seja r um número positivo tal que
	\[
		\overline{B}_r(0)\subseteq C
	\]
	e note que, para todo x em X,
	\[
		r \frac{x}{\Vert x \Vert}\in \overline{B}_r(0)\subseteq C,
	\]
	o que implica em
	\[
		p(x)\leq \frac{1}{r}\Vert x \Vert,
	\]
	fazendo a primeira parte das propriedades de p ser verdadeira a partir do momento que colocamos \(M=1/r\). Além disso, para a segunda propriedade, se x for um elemento de C qualquer, existe \(\varepsilon >0\), tal que \((1+\varepsilon )x\in C\); assim,
	\[
		p(x)\leq \frac{1}{1+\varepsilon }<1.
	\]

	Reciprocamente, se \(p(x)<1\), existe \(\alpha\in (0, 1) \) tal que \(\alpha^{-1}x\in C\) e, consequentemente,
	\[
		x=\alpha (\alpha ^{-1}x)+(1-\alpha )0\in C.
	\]

	Vamos verificar, para finalizar, que p é um funcional sub-linear. Primeiramente, observe que
	\[
		p(\lambda x)=\lambda p(x),\quad \lambda >0.
	\]
	Ademais, sejam x e y elementos de X e \(\varepsilon >0\); então, para todo t em \([0,1]\)
	\[
		\frac{x}{p(x)+\varepsilon }\in C \;\&\; \frac{y}{p(y)+\varepsilon }\in C \Rightarrow \frac{tx}{p(x)+\varepsilon } + \frac{(1-t)y}{p(y)+\varepsilon }\in C.
	\]
	Em particular, pondo
	\[
		t=\frac{p(x)+\varepsilon }{p(x)+p(y)+2\varepsilon },
	\]
	temos
	\[
		\frac{x+y}{p(x)+p(y)+2\varepsilon }\in C.
	\]
	Portanto, para todo \(\varepsilon \) positivo,
	\[
		p(x+y)\leq p(x)+p(y)+2\varepsilon  \Rightarrow p(x+y)\leq p(x)+p(y).\text{ \qedsymbol}
	\]
\end{proof*}

\begin{lemma*}
	Seja C um subconjunto aberto, convexo e não vazio de X tal que \(x_{0}\) \textit{\textbf{não}} pertence a C. Então, existe um funcional f em \(X^{*}\) tal que \(f(x)<f(x_{0})\) para todo x em C. Em particular, o hiperplano fechado da equação \([f=f(x_{0})]\) separa C de \(x_{0}\) no sentido fraco.
\end{lemma*}
\begin{proof*}
	Por translação, sempre podemos supor que C contém a origem; dito isso, sejam p o \hyperlink{minkowski_functional}{\textit{funcional de Minkowski}} de C, G o conjunto \(\mathbb{R}x_{0}\) e \(g:G\rightarrow \mathbb{R}\) dada por
	\[
		g(tx_{0})=t,\quad t\in \mathbb{R}.
	\]
	Com isso,
	\[
		g(x)=g(tx_{0}) = \left\{\begin{array}{ll}
			t\leq t p(x_{0})=p(tx_{0}), & \quad t>0     \\
			t\leq 0\leq p(tx_{0}),      & \quad t\leq 0
		\end{array}\right..
	\]
	Logo, pelo \hyperlink{hahn_banach}{\textit{Teorema de Hahn-Banach real}}, existe \(f:X\rightarrow \mathbb{R}\) tal que \(f|_{G}=g\) e, para todo x em X,
	\[
		f(x)\leq p(x).
	\]
	Em particular, \(f(x_{0})=1\) e f é contínua por conta de \(p(x)\leq M\Vert x \Vert.\) Além disso, \(f(x)<1\) para todo x em C, provando o resultado. \qedsymbol
\end{proof*}
Finalmente, podemos ir para a
\begin{proof*}[\hyperlink{first_geometric_hahn_banach}{\textit{Teorema de Hahn-Banach: primeira forma geométrica}}]
	Seja \(C\coloneqq A-B\), o qual é convexo, aberto e não contém a origem, já que \(A\cap B = \emptyset \). Pelo lema acima, existe um funcional \(f\in X^{*}\) tal que
	\[
		f(z)<0,\quad \forall z\in C,
	\]
	donde segue que
	\[
		f(a)<f(b),\quad \forall a\in A,\; b\in B.
	\]
	Portanto, escolhendo \(\alpha \) tal que
	\[
		\sup_{a\in A}f(x)\leq \alpha \leq \inf_{b\in B}f(b),
	\]
	segue que o hiperplano de equação \([f=\alpha ]\) separa A e B no sentido fraco. \qedsymbol
\end{proof*}
\hypertarget{second_geometric_hahn_banach}{
	\begin{theorem*}[Segunda Forma Geométrica do Hahn-Banach]
		Seja X um espaço vetorial normado real, A e B convexos, não vazio e disjuntos em X. Suponha que A é fechado e B é compacto; então, existe um hiperplano fechado que separa A e B no sentido forte.
	\end{theorem*}
}
\begin{proof*}
	Dado \(\varepsilon > 0\), definamos os conjuntos
	\begin{align*}
		 & A_\varepsilon = A + B_\varepsilon(0)   \\
		 & B_\varepsilon = B + B_\varepsilon (0).
	\end{align*}
	Desta forma, \(A_\varepsilon \) e \(B_\varepsilon \) são abertos, convexos e não vazios; em particular, para \(\varepsilon > 0\) pequeno, \(A_\varepsilon \) e \(B_\varepsilon \) são disjuntos, tendo em vista que, por A ser fechado,
	\[
		d(b, A) > 0, \quad \forall b\in B
	\]
	e, como B é compacto,
	\[
		\inf_{b\in B}d(b, A) = d(B, A) > 0,
	\]
	confirmando a propriedade de serem disjuntos.

	Pela \hyperlink{first_geometric_hahn_banach}{\textit{Primeira Versão,}} existe um hiperplano fechado que separa \(A_\varepsilon \) e \(B_\varepsilon \) no sentido fraco; logo, para todo x em A, y em B e z numa bola \(B_1(0)\),
	\[
		f(x+\varepsilon z)\leq \alpha \leq f(y+\varepsilon z),
	\]
	donde obtemos
	\[
		f(x)-\varepsilon \Vert f \Vert\leq \alpha \leq f(y)+\varepsilon \Vert f \Vert,\quad \forall x\in A,\; y\in B.
	\]
	Portanto, como f é diferente de uma função nula, segue o resultado. \qedsymbol
\end{proof*}
\begin{crl*}
	Seja X um espaço vetorial normado sobre \(\mathbb{K}\) e F um subespaço vetorial próprio de X, \textit{i.e.}, \(\overline{F}\subsetneq X \). Então, existe um funcional não nulo \(f\in X^{*}\) tal que
	\[
		f(x) = 0, \quad \forall x\in F.
	\]
\end{crl*}
\begin{proof*}
	Sabemos que, se X é um espaço vetorial normado sobre \(\mathbb{R},\) dado \(x_{0}\not\in \overline{F},\) existe um funcional linear \(f:X\rightarrow \mathbb{R}\) contínuo tal que
	\[
		f(x)\leq f(x_{0}),\quad \forall x\in F,
	\]
	que leva diretamente à conclusão de \(f(x) = 0\) para todo x de F.

	Para o caso complexo, basta tomar \(g:X\rightarrow \mathbb{C}\) dada por
	\[
		g(x) = f(x)-if(ix),
	\]
	finalizando a prova. \qedsymbol
\end{proof*}
\begin{tcolorbox}[
		skin=enhanced,
		title=Observação,
		fonttitle=\bfseries,
		colframe=black,
		colbacktitle=cyan!75!white,
		colback=cyan!15,
		colbacklower=black,
		coltitle=black,
		drop fuzzy shadow,
		%drop large lifted shadow
	]
	Um lugar de uso do corolário acima é quando temos que mostrar que F é denso, tal que, pela sua contraposição, basta mostrar que
	\[
		f(x) = 0,\quad \forall x\in F \Rightarrow f = 0.
	\]
\end{tcolorbox}
\end{document}

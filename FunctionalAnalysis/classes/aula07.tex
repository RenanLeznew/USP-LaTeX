\documentclass[../functional_analysis.tex]{subfiles}
\begin{document}
\section{Aula 07 - 26 de Agosto, 2025}
\subsection{Motivações}
\begin{itemize}
	\item Espaços com Produto Interno;
	\item Espaços de Hilbert;
\end{itemize}
\subsection{Espaços com Produto Interno e Espaços de Hilbert}
\begin{def*}
	Seja H um espaço vetorial sobre \(\mathbb{K}.\) Um \textbf{produto escalar} é uma função \(\left< \cdot , \cdot  \right>: H\times H\rightarrow \mathbb{R}\) tal que, para todo u, v e w de H,
	\begin{align*}
		 & (a)\; \left< u, v \right> = \overline{\left< v, u \right>}                                                                                \\
		 & (b)\; \left< \alpha u + \beta v, w \right> = \alpha \left< u, w \right> + \beta \left< v, w \right>, \quad \alpha,\; \beta \in \mathbb{K} \\
		 & (c)\; \left< u, u \right>\geq 0                                                                                                           \\
		 & (d)\; \left< u, u \right> = 0 \Longleftrightarrow u = 0.
	\end{align*}
	Um espaço vetorial H juntamente com um produto interno é dito um \textbf{espaço com produto interno.} \(\square\)
\end{def*}

Dessas propriedades, segue que, para todos u, v e w em H, e \(\alpha , \beta \) em \(\mathbb{K}\),
\[
	\left< u, \alpha v + \beta w \right> = \overline{\alpha }\left< u, v \right> + \overline{\beta }\left< u, w \right>
\]
e vale a \hypertarget{cauchy_schwarz}{desigualdade de Cauchy-Schwarz}:
\[
	| \left< u, v \right> | \leq \left< u, u \right>^{\frac{1}{2}}\left< v, v \right>^{\frac{1}{2}}.
\]
De fato, para t reais,
\[
	0\leq \left< u+tw, u+tw \right> = \left< u, u \right> + 2t \mathrm{Re}\left< u, w \right> + t^{2}\left< w, w \right>.
\]
Como a expressão do lado direito é uma função quadrática de t com uma ou nenhuma raiz real
\[
	0\geq 4(\mathrm{Re}\left< u, w \right>)^{2} - 4 \left< u, u \right>\left< w, w \right>
\]
e, se \(w = e^{i \mathrm{arg}\left< u, v \right>}v\), temos
\[
	0\geq 4 | \left< u, v \right> |^{2} - 4 \left< u, u \right>\left< v, v \right>,
\]
provando a desigualdade por manipulação algébrica:
\begin{align*}
	                    & 0\geq 4 | \left< u, v \right> |^{2} - 4 \left< u, u \right>\left< v, v \right>,      \\
	\Longleftrightarrow & 4 | \left< u, v \right> |^{2} \geq 4 \left< u, u \right>\left< v, v \right>          \\
	\Longleftrightarrow & | \left< u, v \right> | \geq (\left< u, u \right>\left< v, v \right>)^{\frac{1}{2}}.
\end{align*}

Quando temos um espaço com produto interno, ganhamos naturalmente uma norma dada por
\begin{align*}
	\Vert \cdot  \Vert: & H\rightarrow \mathbb{R}                                         \\
	                    & u\longmapsto \Vert u \Vert = \left< u, u \right>^{\frac{1}{2}}.
\end{align*}
Para ver que isto é verdade, precisamos mostrar apenas que \(\Vert u + v \Vert\leq \Vert u \Vert + \Vert v \Vert\) para todo u, v em H, que segue da \hyperlink{cauchy_schwarz}{\textit{desigualdade de Cauchy-Schwarz}} e de
\begin{align*}
	\Vert u+v \Vert^{2} & = \Vert u \Vert^{2} + 2 \mathrm{Re}\left< u, v \right> + \Vert v \Vert^{2} \\
	                    & \leq \Vert u \Vert^{2} + 2 | \left< u, v \right> | + \Vert v \Vert^{2}     \\
	                    & \leq \Vert u \Vert^{2} + 2 \Vert u \Vert\Vert v \Vert + \Vert v \Vert^{2}  \\
	                    & = (\Vert u \Vert + \Vert v \Vert)^{2}.
\end{align*}

\hypertarget{parallelogram_identity}{\begin{lemma*}[identidade do Paralelogramo]
		Em um espaço com produto interno \((H, \left< \cdot , \cdot  \right>)\), vale a identidade do paralelogramo:
		\[
			\Vert u + v \Vert^{2} + \Vert u-v \Vert^{2} = 2(\Vert u \Vert^{2} + \Vert v \Vert^{2}),\quad \forall u, v\in H.
		\]
	\end{lemma*}}
\begin{def*}
	Se um espaço com produto interno \(H\) é completo, dizemos que H é um \textbf{espaço de Hilbert.} \(\square\)
\end{def*}
\begin{def*}
	Dois vetores u e v em um espaço com produto interno H são ditos \textbf{ortogonais}, denotado por \(u\perp v\), se \(\left< u, v \right> = 0.\; \square\)
\end{def*}
\hypertarget{pythagorean_theorem}{\begin{theorem*}[Teorema de Pitágoras]
		Dados dois vetores ortogonais u e v num espaço com produto interno H, vale o Teorema de Pitágoras:
		\[
			\Vert u + v \Vert^{2} = \Vert u \Vert^{2} + \Vert v \Vert^{2}.
		\]
		Mais geralmente, se \(u_1, \dotsc , u_{n}\) são vetores dois a dois ortogonais em um espaço com produto interno H, então
		\[
			\biggl\vert \sum\limits_{i=1}^{n}u_{i} \biggr\vert^{2} = \sum\limits_{i=1}^{n}\Vert u_{i} \Vert^{2}.
		\]
	\end{theorem*}}
\begin{tcolorbox}[
		skin=enhanced,
		title=Lembrete!,
		after title={\hfill Conjuntos Convexos},
		fonttitle=\bfseries,
		sharp corners=downhill,
		colframe=black,
		colbacktitle=yellow!75!white,
		colback=yellow!30,
		colbacklower=black,
		coltitle=black,
		%drop fuzzy shadow,
		drop large lifted shadow
	]
	Lembre-se que um subconjunto C de um espaço vetorial X é dito \textbf{convexo} se, para todos x, y em C e \(t\in [0, 1]\),
	\[
		tx + (1-t)y\in C,
	\]
	ou seja, qualquer segmento de reta entre pontos de C continua inteiramente em C.
\end{tcolorbox}

\begin{lemma*}
	Se K é um subconjunto fechado e convexo de um espaço de Hilbert H e \(u_{0}\in H\), existe um único \(v_{0}\in K\) tal que
	\[
		\Vert u_{0}-v_{0} \Vert = \inf_{v\in K}\Vert u_{0}-v \Vert.
	\]
	Escrevemos \(v_{0} = P_{K}u_{0}\) e dizemos que \(P_{K}\) é a \textbf{projeção sobre o convexo K}.
\end{lemma*}

Se M é um subespaço vetorial de H, então
\[
	M^{\perp } \coloneqq \{u\in H:\; u\perp v,\; \forall v\in M\}.
\]
\begin{exr}
	Prove que \(M^{\perp }\) é sempre um subespaço vetorial fechado de \(H.\)
\end{exr}
\begin{prop*}
	Seja H um espaço de Hilbert, K um subconjunto fechado de H e convexo, e \(u_{0}\in H\); então,
	\[
		\mathrm{Re}\langle u_{0} - P_{K}u_{0}, w-P_{K}u_{0} \rangle \leq 0,\quad \forall w\in K.
	\]
	Além disso, se M é um subespaço vetorial fechado de H, então
	\[
		\langle u_{0}-P_{M}u_{0}, w \rangle = 0,\quad \forall w\in M,
	\]
	isto é,
	\[
		(I-P_{M})u_{0}\in M^{\perp }.
	\]
\end{prop*}

Vale a volta da proposição acima:
\begin{prop*}
	Sejam H um espaço de Hilbert e K um subconjunto convexo, não vazio e fechado de H. Se, dado \(u_{0}\in H\), existe \(v_{0}\in K\) tal que
	\[
		\mathrm{Re} \langle u_{0}-v_{0}, w-v_{0} \rangle \leq 0,\quad \forall w\in K,
	\]
	então \(v_{0} = P_{K}U_{0}\).
\end{prop*}
\begin{proof*}
	Se \(v_{0}\in K\) cumpre 1 e w é um ponto de K, temos
	\begin{align*}
		\Vert v_{0}-u_{0} \Vert^{2} - \Vert w-u_{0} \Vert^{2} & = \Vert v_{0}-u_{0} \Vert^{2}-\langle v_{0}-u_{0}+w-v_{0}, v_{0}-u_{0}+w-v_{0} \rangle \\
		                                                      & = 2 \mathrm{Re}\langle u_{0}-v_{0}, w-v_{0} \rangle - \Vert w-v_{0} \Vert^{2}          \\
		                                                      & \leq 0.
	\end{align*}
	Portanto,
	\[
		\Vert u_{0}-v_{0} \Vert = \inf_{w\in K}\Vert u_{0}-w \Vert. \text{ \qedsymbol}
	\]
\end{proof*}
\begin{crl*}
	Se H é um espaço de Hilbert E M é um subespaço vetorial fechado de H, então \(P_M:H\rightarrow H\) é caracterizada por \(v = P_{M}u\) se, e somente se ,
	\[
		\mathrm{Re}\langle u-v, w \rangle = 0,\quad \forall w\in M.
	\]
	Além disso, \(P_{M}\) é linear e idempotente, isto é, \(P_{M}^{2} = P_{M}.\)
\end{crl*}
\begin{def*}
	Uma transformação linear \(P:H\rightarrow H\) que satisfaz \(P^{2} = P\) é chamada de \textbf{projeção}. Caso \(P\in \mathcal{L}(H)\) seja uma projeção, M denote sua imagem e \(M^{\perp } = \mathrm{ker}(P)\), diremos que P é uma \textbf{projeção ortogonal sobre M.} \(\square\)
\end{def*}
\begin{exr}
	Mostre que uma projeção ortogonal P é continua se, e somente se, \(M = \mathrm{Im}(P)\) é fechado.
\end{exr}
\begin{theorem*}
	Seja H um espaço de Hilbert e M um subespaço vetorial fechado de H; então, \(M \oplus M^{\perp } = H\), ou seja, cada u em H pode ser expresso unicamente na forma \(u= w + v\) com w sendo um ponto de M e v um ponto de \(M^{\perp }.\) Ademais, os vetores v e w são os únicos elementos de M e \(M^{\perp } \) cuja distância a u é mínima; em outras palavras,
	\[
		w = P_{M}u \quad\&\quad v = P_{M^{\perp }}u.
	\]
	Finalmente, \(P_{M}\) e \(P_{M^{\perp }}\) são projeções lineares limitadas com \(\Vert P_{M} \Vert = \Vert P_{M^{\perp }} \Vert = 1\), exceto quando \(M=0\) ou \(M=H.\)
\end{theorem*}
\begin{proof*}
	Para u em H, \(P_{M}u\) é o único elemento de M que minimiza a distância de u a M. Note que, para todo \(u\in H\), temos
	\[
		M\cap M^{\perp } = \{0\}\;\&\; u = P_{M}u + (I-P_{M})u\in M + M^{\perp }.
	\]

	Além disso, se z é um ponto perpendicular aos pontos de M, vale
	\begin{align*}
		\Vert u-z \Vert^{2} & = \Vert P_{M}u + (I-P_{M})u - z \Vert^{2}                 \\
		                    & = \Vert P_{M}u \Vert^{2} + \Vert (I-P_{M})u - z \Vert^{2} \\
		                    & \geq \Vert P_{M}u \Vert^{2},
	\end{align*}
	mostrando que, também, \((I-P_{M})u\) é o único ponto de \(M^{\perp }\) que minimiza a distância de u a \(M^{\perp },\) donde segue que, como não podem existir dois pontos únicos e distintos que minimizem a distância de u a \( M^{\perp }\), deve-se ter
	\[
		(I-P_{M})u = P_{M^{\perp }}u.
	\]
	Agora, vamos mostrar a continuidade da projeção \(P_{M}\): primeiramente, observe que \(P_{M}\) é linear pois, se \(u, v\in H\) e \(\alpha \in \mathbb{K}\), então
	\[
		P_{M}(\alpha u+v) = z
	\]
	é o elemento de M que minimiza a distância a \(\alpha u + z\) e
	\[
		\Vert \alpha u + v - m \Vert = \Vert \alpha P_{M}u + P_{M}v - m \Vert^{2} + \Vert \alpha (u-P_{M}u) + v - P_{M}v \Vert^{2},
	\]
	com o mínimo ocorrendo quando \(m = \alpha P_{M}u + P_{M}v\). Logo, pela proposição provada,
	\[
		\Vert P_{M}u \Vert \leq \Vert u \Vert
	\]
	e, do fato que \(P_{M}u = u \) para todo ponto de M, temos portanto \(\Vert P_{M} \Vert =1.\) \qedsymbol
\end{proof*}

Se H é um espaço de Hilbert e y é um ponto em H, segue da \hyperlink{cauchy_schwarz}{\textit{Desigualdade de Cauchy-Schwarz}} que
\[
	f_y(x)=\langle x, y \rangle
\]
define um funcional linear contínuo com a propriedade de
\[
	\Vert f_{y} \Vert_{H^{*}}=\Vert y \Vert_{H},
\]
ou seja, a transformação mapeando y de H para o funcional \(f_y\) é uma isometria linear-conjugada entre H e seu dual. Mais do que isso,
\hypertarget{riesz_representation}{\begin{theorem*}[Representação de Riesz]
		Se f é um funcional de H, então existe um único ponto y em H tal que, para todos \(x\in H\),
		\[
			f(x)=\langle x, y \rangle.
		\]
	\end{theorem*}}
\begin{proof*}
	A unicidade fica como exercício.

	Se f é o funcional nulo, basta tomar y = 0. Caso contrário, seja M o conjunto dos elementos x de H nos quais o funcional se anula; então, M é fechado e um subconjunto próprio de H (pois pedimos que f não fosse o funcional nulo), donde concluímos que \(M^{\perp }\neq \{0\}.\) Seja z um elemento de \(M^{\perp }\) tal que \(\Vert z \Vert=1\); se \(x\in H\) e \(u=f(x)z-f(z)x\), então
	\[
		f(u) = f(f(x)z-f(z)x) = f(x)f(z)-f(z)f(x) = 0,
	\]
	mostrando que u pertence a M. Mais do que isso,
	\[
		0=\langle u, z \rangle = f(x)\Vert z \Vert^{2} - f(z)\langle x, z \rangle = f(x)- \langle x, \overline{f(z)}z \rangle.
	\]
	Portanto, pondo \(y=\overline{f(z)}z\),
	\[
		f(x)=\langle x, y \rangle. \text{ \qedsymbol}
	\]
\end{proof*}
\begin{exr}
	Prove a unicidade do y no \hyperlink{riesz_representation}{\textit{Teorema Acima}}.
\end{exr}
O que conseguimos com isso é que os espaços de Hilbert são reflexivos em um sentido \textit{muito} forte, pois possuem uma identificação isomórfica natural tanto com \(H^{**}\) quanto uma com \(H^{*}\) através de uma transformação linear-conjugada.
\begin{def*}
	Um subconjunto \(\{u_{\alpha }\}_{\alpha \in A}\) de H é chamado um \textbf{conjunto ortonormal} se \(\Vert u_{\alpha } \Vert = 1\) para todo \(\alpha \in A\) e, além disso,
	\[
		\alpha \neq \beta \Rightarrow u_{\alpha }\perp u_{\beta }.\quad \square
	\]
\end{def*}

Dada uma sequência linearmente independente de vetores em H, digamos \(\{v_{n}\}_{n=1}^{\infty}\), existe um procedimento usual e já conhecido para convertê-la em uma sequência ortonormal \(\{u_{n}\}_{n=1}^{\infty}\) tal que o espaço gerado por \(\{v_{1}, \dotsc , v_{N}\}\) coincide com aquele gerado por \(\{u_{1},\dotsc , u_{N}\}\): a \hypertarget{gram_schmidt}{Ortogonalização de Gram-Schmidt}, consistindo em tomar primeiro \(u_1 = \frac{v_1}{\Vert v_1 \Vert}\) e, por indução, determinados \(u_1,\dotsc ,u_{N}\), colocar
\[
	u_{N} = \frac{ \biggl[v_{N}-\sum\limits_{n=1}^{N-1}{\color{Chartreuse4}\langle v_{N}, u_{n} \rangle u_{n}}\biggr]}{{\color{Turquoise4}\Vert v_{N} -\sum\limits_{n=1}^{N-1}\langle V_{N}, u_{n} \rangle u_{n}\Vert}},
\]
onde, entendendo melhor os termos, o que está {\color{Chartreuse4}em verde é a projeção dos vetores \(v_{N}\) sobre \(u_{i}\)} e o termo {\color{Turquoise4}em azul representa a normalização}, formando, juntos, a projeção ortogonal de \(v_{N}\) sobre \(u_{i}\)
\end{document}

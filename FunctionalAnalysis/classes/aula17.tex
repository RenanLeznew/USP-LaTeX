\documentclass[../functional_analysis.tex]{subfiles}
\begin{document}
\section{Aula 17 - 23 de Outubro, 2025}
\subsection{Motivações}
\begin{itemize}
	\item Espaços de Dimensão Infinita;
	\item Teorema de Riesz.
\end{itemize}
\subsection{Espaços de Dimensão Infinita}

Antes do tópico novo, vamos finalizar uma aplicação do \hyperlink{friedrichs}{\textit{Teorema de Friedrichs}}:
\begin{example}
	Seja \(X = L^{2}(0,  \pi )\) e \(D(A_{0}) = \mathcal{C}_{0}^{2}(0, \pi )\) o conjunto das funções duas vezes continuamente diferenciáveis e que têm suporte compacto em
	\((0, \pi )\). Defina \(A_{0}:D(A_{0})\subseteq X\rightarrow X\) por
	\[
		(A_{0})(x)=-\varphi''(x),\quad x\in (0, \pi ).
	\]

	O operador \(A_{0}\) é simétrico e, para \(\varphi \in D(A_{0})\),
	\[
		\langle A_{0}\varphi , \varphi  \rangle_{2} \geq \frac{2}{\pi^{2}}\Vert \varphi \Vert_{X}^{2}.
	\]
	Consequentemente, \(A_{0}\) possui uma extensão auto adjunta A que satisfaz \(\langle A\varphi , \varphi  \rangle_{2}\geq \frac{2}{\pi^{2} }\Vert \varphi  \Vert_{X}^{2}\) para todo \(\varphi \) em \(D(A)\).

	Observe que o espaço \(X^{\frac{1}{2}}\) do \hyperlink{friedrichs}{\textit{Teorema de Friedrichs}} é, neste exemplo, o fecho de \(D(A)\) na norma \(\Vert \cdot  \Vert_{\frac{1}{2}}\), sendo denotado por \(H_{0}^{1}(0, \pi )\). tal que
	\(H_{0}^{1}(0, \pi ) \subseteq H^{1}(0, \pi )\) com normas equivalentes. Com isso, se \(\varphi \in H^{1}(0, \pi )\), então \(\varphi \) é Hölder contínua com expoente \(\frac{1}{2}\) e, se \(\varphi \in H^{1}(0, \pi )\),
	\[
		\varphi (0) = \varphi (\pi )=0.
	\]

	Por outro lado, \(D(A^{*})\) é caracterizado por
	\[
		D(A_{0}^{*})=\{\varphi \in X:\; \exists \varphi^{*}\in X,\; \langle -u'', \varphi  \rangle = \langle u, \varphi^{*} \rangle,\; \forall u\in D(A_{0})\},
	\]
	tal que
	\[
		A_{0}^{*}\varphi = -\varphi '' = \varphi^{*}, \quad \forall u\in D(A_{0}^{*}).
	\]
	Assim, \(D(A)=H^{2}(0, \pi )\cap H_{0}^{1}(0, \pi )\) e \(Au = -u''\) para todo u em \(D(A)\), sendo, então, uma derivada no sentido fraco previamente definido.
\end{example}

\end{document}

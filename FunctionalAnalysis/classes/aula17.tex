\documentclass[../functional_analysis.tex]{subfiles}
\begin{document}
\section{Aula 17 - 23 de Outubro, 2025}
\subsection{Motivações}
\begin{itemize}
	\item Espaços de Dimensão Infinita;
	\item Teorema de Riesz.
\end{itemize}
\subsection{Espaços de Dimensão Infinita}

Antes do tópico novo, vamos finalizar uma aplicação do \hyperlink{friedrichs}{\textit{Teorema de Friedrichs}}:
\begin{example}
	Seja \(X = L^{2}(0,  \pi )\) e \(D(A_{0}) = \mathcal{C}_{0}^{2}(0, \pi )\) o conjunto das funções duas vezes continuamente diferenciáveis e que têm suporte compacto em
	\((0, \pi )\). Defina \(A_{0}:D(A_{0})\subseteq X\rightarrow X\) por
	\[
		(A_{0})(x)=-\varphi''(x),\quad x\in (0, \pi ).
	\]

	O operador \(A_{0}\) é simétrico e, para \(\varphi \in D(A_{0})\),
	\[
		\langle A_{0}\varphi , \varphi  \rangle_{2} \geq \frac{2}{\pi^{2}}\Vert \varphi \Vert_{X}^{2}.
	\]
	Consequentemente, \(A_{0}\) possui uma extensão auto adjunta A que satisfaz \(\langle A\varphi , \varphi  \rangle_{2}\geq \frac{2}{\pi^{2} }\Vert \varphi  \Vert_{X}^{2}\) para todo \(\varphi \) em \(D(A)\).

	Observe que o espaço \(X^{\frac{1}{2}}\) do \hyperlink{friedrichs}{\textit{Teorema de Friedrichs}} é, neste exemplo, o fecho de \(D(A)\) na norma \(\Vert \cdot  \Vert_{\frac{1}{2}}\), sendo denotado por \(H_{0}^{1}(0, \pi )\). tal que
	\(H_{0}^{1}(0, \pi ) \subseteq H^{1}(0, \pi )\) com normas equivalentes. Com isso, se \(\varphi \in H^{1}(0, \pi )\), então \(\varphi \) é Hölder contínua com expoente \(\frac{1}{2}\) e, se \(\varphi \in H^{1}(0, \pi )\),
	\[
		\varphi (0) = \varphi (\pi )=0.
	\]

	Por outro lado, \(D(A^{*})\) é caracterizado por
	\[
		D(A_{0}^{*})=\{\varphi \in X:\; \exists \varphi^{*}\in X,\; \langle -u'', \varphi  \rangle = \langle u, \varphi^{*} \rangle,\; \forall u\in D(A_{0})\},
	\]
	tal que
	\[
		A_{0}^{*}\varphi = -\varphi '' = \varphi^{*}, \quad \forall u\in D(A_{0}^{*}).
	\]
	Assim, \(D(A)=H^{2}(0, \pi )\cap H_{0}^{1}(0, \pi )\) e \(Au = -u''\) para todo u em \(D(A)\), sendo, então, uma derivada no sentido fraco previamente definido.
\end{example}
Com o exemplo finalizado, podemos dar continuidade às discussões teóricas.

Muitas das soluções dos problemas matemáticos importantes são obtidas como mínimos, máximos, ou pontos fixos de funções definidas em espaços de dimensão infinita, e a noção de compacidade desempenha um papel
fundamental para a solução de inúmeras questões importantes na matemática. Para introduzir nosso estudo aprofundados desse assunto, considere um espaço vetorial normado X e um conjunto linearmente independente de vetores de X, denotado por
\(\{f_1, \dotsc , f_{n}\}\).
\begin{lemma*}
	Se F é gerado pelo conjunto de vetores, i.e.,  \(F = [f_1, \dotsc , f_{n}]\), então F é completo.
\end{lemma*}
\begin{proof*}
	Mostraremos que existe c positivo tal que
	\[
		\sup_{}\biggl\{\biggl\Vert \sum\limits_{i=1}^{n}\alpha_{i}f_{i} \biggr\Vert_{X}:\; (\alpha_1, \dotsc , \alpha_{n})\in \mathbb{K}^{n},\; \sum\limits_{i=1}^{n}| \alpha_{i} | = 1\biggr\}\geq c.
	\]
	De fato, se o caso não fosse esse, existiria uma sequência \(\{(\alpha_{1}^{k}, \dotsc , \alpha_{n}^{k})\}_{k\in \mathbb{N}}\) com
	\[
		\sum\limits_{i=1}^{n}| \alpha_{i}^{k} | = 1
	\]
	e tal que
	\[
		\biggl\Vert \sum\limits_{i=1}^{n}\alpha_{i}^{k}f_{i} \biggr\Vert_{X}\substack{k\to \infty \\ \longrightarrow \\ }0.
	\]
	Assim, tomando subsequências, poderíamos assumir que \(\alpha_{i}^{k}\) converge para \(\alpha_{i}\) conforme k tende a infinito; consequentemente,
	\[
		\sum\limits_{i=1}^{n}| \alpha_{i} | = 1 \quad\&\quad \sum\limits_{i=1}^{n}\alpha_{i}f_{i} =0,
	\]
	contradizendo a independências linear de \(\{f_1, \dotsc , f_{n}\}\). Com isso, revertendo a desigualdade do supremo utilizada, segue a completude. \qedsymbol
\end{proof*}
\begin{exr}
	Se F é um espaço vetorial normado de dimensão natural \(n\) sobre K,  mostre que:
	\begin{itemize}
		\item Existe uma transformação linear contínua de F sobre \((\mathbb{K}^{n}, \Vert \cdot  \Vert_1)\);
		\item Todas as normas em F são equivalente;
		\item Todas as normas em \(\mathbb{K}^{n}\) são equivalentes; e
		\item Um subconjunto K de F é compacto se, e somente se, K for fechado e limitado.
	\end{itemize}
\end{exr}

Para as próximas considerações, pede-se que X seja um espaço vetorial normado com dimensão infinita. Veremos, a partir disso, que nenhum subconjunto de X com interior não vazio é compacto na
topologia induzida pela norma, donde decorre que os compactos de X não podem conter qualquer bola.

O resultado acima torna muito mais complicada a análise de problemas matemáticos no contexto mencionado, então deveremos procurar alguma topologia para X com menos abertos, já que isso resultaria em mais compactos,
facilitando a solução de possíveis problemas -- tais topologias serão conhecidas como \textit{topologia fraca em X} e \textit{topologia fraca-*} em \(X^{*}\).

Sendo assim, começando pela introdução do problema, vamos mostrar que a bola unitária de um espaço vetorial normado é compacta se, e somente se, X tem dimensão finita, e disso decorrerá que nenhum compacto em dimensão infinita poderá conter uma bola de raio qualquer.

\hypertarget{riesz_lemma}{
	\begin{lemma*}[Lema de Riesz]
		Seja X um espaço vetorial normado sobre \(\mathbb{K}\) e \(M\) um subespaço vetorial fechado próprio de X. Então, para cada \(\varepsilon \) positivo, existe um vetor u de X tal que
		\[
			\Vert u \Vert = 1 \quad\&\quad \mathrm{dist}(u, M) \geq 1-\varepsilon .
		\]
	\end{lemma*}
}
\begin{proof*}
	Sem perda de generalidade, podemos fazer o caso em que \(0 < \varepsilon <1\). Para tanto, seja v um vetor de X, mas não de M; como M é fechado, a distância \(d = \mathrm{dist}(v, M)\) é não nula (e positiva). Com isso, escolha \(m_{0}\) em M tal que
	\[
		d < \Vert v-m_{0} \Vert < \frac{d}{1-\varepsilon },
	\]
	e sejam \(u=\frac{v-m_{0}}{\Vert v-m_{0} \Vert}\) e \(m\in M\); então, \(m_{0}+\Vert v-m_{0} \Vert m\) pertence a M e u satisfaz
	\begin{align*}
		\Vert u-m \Vert = \biggl\Vert \frac{v-m_{0}}{\Vert v-m_{0} \Vert} - m \biggr\Vert & = \biggl\Vert \frac{v-m_{0}-m \Vert v-m_{0} \Vert}{\Vert v-m_{0} \Vert} \biggr\Vert \\
		                                                                                  & \geq \frac{d}{\Vert v-m_{0} \Vert}                                                  \\
		                                                                                  & \geq 1-\varepsilon. \text{ \qedsymbol}
	\end{align*}
\end{proof*}
Com o Lema de Riesz, estamos aptos a provar o resultado mencionado:
\hypertarget{riesz_theorem}{
	\begin{theorem*}[Teorema de Riesz]
		Seja X um espaço vetorial normado sobre \(\mathbb{K}\) tal que \(\overline{B}_{1}(0) = \{x\in X:\; \Vert x \Vert\leq 1\}\) é compacta. Então, X tem dimensão finita.
	\end{theorem*}
}
\begin{proof*}
	Suponha, por absurdo, que X tem dimensão infinita. Com isso, existem subespaços \(M_{n}\) de X, com n natural, tais que \(\mathrm{dim}(M_{n}) = n\) e \(M_{n}\subseteq M_{n+1}\) para todo n natural.
	Como \(M_{n}\) é fechado, segue do \hyperlink{riesz_lemma}{\textit{Lema de Riesz}} que existe \(u_{n}\) em \(M_{n}\) com norma 1 e tal que
	\[
		\mathrm{dist}(u_{n}, M_{n-1})\geq \frac{1}{2}.
	\]
	Portanto, \(\{u_{n}\}\) não tem subsequência convergente, o que contradiz a compacidade de \(\overline{B}_{1}(0)\). \qedsymbol
\end{proof*}
\begin{crl*}
	Seja X um espaço vetorial normado de dimensão infinita. Se K é um compacto de X, então \(K^{\circ } = \emptyset.\)
\end{crl*}

Para podermos continuar o assunto e chegar nas topologias fraca e fraca-*, recordaremos algumas noções elementares de topologia geral que são indispensáveis para a apresentação delas.

\begin{def*}
	Seja X um conjunto não vazio. Uma \textbf{topologia em X} é uma família \(\tau \) de subconjuntos de X que satisfazem:
	\begin{itemize}
		\item \(\tau \) contém em si \(\emptyset \) e X;
		\item A família \(\tau \) é fechada por interseções finitas; e
		\item A família \(\tau \) é fechada por uniões arbitrárias.
	\end{itemize}
	Os elementos de \(\tau \) são chamados \textbf{abertos de X}, e X dotado de uma topologia \(\tau \) é chamado um \textbf{espaço topológico}, o qual denotaremos por \((X, \tau )\). \(\square\)
\end{def*}

Quando estiver claro qual a topologia envolvida no contexto do espaço topológico, ao invés de escrevermos \((X, \tau )\), escreveremos simplesmente X.

\begin{def*}
	Seja \((X, \tau )\) um espaço topológico. Um subconjunto de X é dito \textbf{fechado} se o seu complementar é aberto. \(\square\)
\end{def*}

\begin{tcolorbox}[
		skin=enhanced,
		title=Observação,
		fonttitle=\bfseries,
		colframe=black,
		colbacktitle=cyan!75!white,
		colback=cyan!15,
		colbacklower=black,
		coltitle=black,
		drop fuzzy shadow,
		%drop large lifted shadow
	]
	Note que a família dos subconjuntos fechados de X é fechada por uniões finitas (complementar das interseções finitas) e por interseções quaisquer (complementar das uniões quaisquer).
\end{tcolorbox}

\begin{def*}
	O \textbf{interior} de um subconjunto A de X, denotado por \(A^{\circ }\), é o maior aberto contido em A. Por outro lado, o \textbf{fecho} de A, denotado por \(\overline{A}\), é o menor fechado que contém A. \(\square\)
\end{def*}

\begin{def*}
	Uma família de subconjuntos
	\[
		\mathcal{C} = \{C_{\lambda }\in 2^{X}:\; \lambda \in \Lambda \}
	\]
	é dita uma \textbf{cobertura de A} se A está contido na união de todos os membros da família:
	\[
		A\subseteq \bigcup_{\lambda \in \Lambda }^{}C_{\lambda },
	\]
	e dizemos que \(\mathcal{C}\) \textbf{cobre A}. Qualquer subconjunto de \(\mathcal{C}\) que ainda cobre A é chamado uma \textbf{subcobertura} e, se todos os conjuntos \(C_{\lambda }\) são abertos, dizemos que \(\mathcal{C}\) é uma \textbf{cobertura aberta.} \(\square\)
\end{def*}

\begin{def*}
	Um subconjunto K de X é \textbf{compacto} se toda cobertura aberta de K possui subcobertura finita. \(\square\)
\end{def*}

\begin{def*}
	Se X é um conjunto não vazio e \(\tau_{1},\ \tau_{2}\) são topologias em X, dizemos que \(\tau_2 \) é \textbf{mais fina} que \(\tau_1\) se \(\tau_2 \supseteq \tau_1\). \(\square\)
\end{def*}

\end{document}

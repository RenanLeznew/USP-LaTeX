\documentclass[../functional_analysis.tex]{subfiles}
\begin{document}
\section{Aula 02 - 07 de Agosto, 2025}
\subsection{Motivações}
\begin{itemize}
 \item Espaços de Banach;
 \item Transformações Lineares;
 \item Funcional Sublinear;
 \item Teorema de Hahn-Banach.
\end{itemize}
\subsection{Espaços de Banach e o Terema de Hahn-Banach}
 \begin{def*}
   Um espaço vetorial normado que é completo com a métrica induzida pela norma, isto é, 
     \[
       \rho (x, y) = \Vert x-y \Vert,
     \]
     é dito \textbf{Espaço de Banach.} \(\square\)
 \end{def*}
 Veremos mais tarde que todo espaço vetorial normado pode ser imerso em um espaço de Banach.

\begin{def*}
  Duas normas \(\Vert \cdot  \Vert_{1}\) e \(\Vert \cdot \Vert_{2}\) em X são \textbf{equivalentes} se existem constantes positivas \(c_1\) e \(c_2\) tais que 
    \[
      c_1 \Vert x \Vert_1 \leq \Vert x \Vert_2 \leq c_2\Vert x \Vert_1,\quad \forall x\in X.\; \square
    \]
\end{def*}
\begin{def*}
  Uma série \(\sum\limits_{j=1}^{\infty}x_{j}\) é dita \textbf{convergente} em X se 
    \[
      \sum\limits_{j=1}^{n}x_{j}\substack{n\to \infty \\ \longrightarrow \\ }x
    \]
    e \textbf{absolutamente convergente} se 
      \[
        \sum\limits_{j=1}^{\infty}\Vert x_{j} \Vert
      \]
      converge. \(\square\)
\end{def*}
\begin{theorem*}
  Um espaço vetorial normado é completo se, e somente se, toda série absolutamente convergente é convergente.
\end{theorem*}
\begin{proof*}
  Se X é um espaço de Banach e 
    \[
      \sum\limits_{j=1}^{\infty}\Vert x_{j} \Vert<\infty,
    \]
  é fácil ver que \(\biggl\{ \sum\limits_{j=1}^{n}x_{j}\biggr\}\) é uma sequência de Cauchy, portanto convergente por definição. 

  Por outro lado, se X é um espaço vetorial normado X onde toda série absolutamente convergente é convergente e \(\{x_{n}\}\) é uma sequência de Cauchy, então existem \(n_1<n_2<\dotsc \) naturais tais que 
    \[
      \Vert x_{n}-x_{m} \Vert\leq 2^{-j},\quad n, m\geq j,
    \]
    então escolhemos \(y_1=x_{n_1},\) e, para \(j\geq 2\), \(y_{j}=x_{n_{j}}-x_{n_{j-1}}\). 
    Logo,  
      \[
        \sum\limits_{j=1}^{k}y_{j}=x_{n_{k}}\;\&\; \sum\limits_{j=1}^{k}\Vert y_{j} \Vert\leq \Vert y_1 \Vert +0\sum\limits_{j=1}^{k}2^{-j}<\Vert y_1 \Vert + 1 < \infty.
      \]
    Portanto, \(\{x_{n_{k}}\}\) é convergente e \(\{x_{n}\}\) é convergente. \qedsymbol
\end{proof*}
\begin{def*}
  Uma transformação \(T:X\rightarrow Y\) linear entre dois espaços vetoriais normados é \textbf{limitada} se existe uma constante \(c\geq 0\) tal que 
    \[
      \Vert Tx \Vert_{Y}\leq c \Vert x \Vert_{X}, \quad \forall x\in X.\; \square
    \]
\end{def*}
\begin{prop*}
  Se X, Y são espaços vetoriais normados \(T:X\rightarrow Y\) é linear, são equivalentes: 
 \begin{itemize}
   \item[a)] T é contínua; 
     \item[b)] T é contínua em 0; 
       \item[c)] T é limitada.
 \end{itemize}
\end{prop*}
\begin{proof*}
  Provaremos esta proposição em ciclo de \(a \Rightarrow b \Rightarrow c \Rightarrow a.\) A primeira delas, no entanto, é trivial; para ver que b implica c, tomando \(\varepsilon =1\), existe \( \delta >0\) tal que 
    \[
      T([B_\delta (0)]^{-})\subseteq T(B_{2\delta }(0)) \subseteq \{y\in Y:\; \Vert y \Vert<1\}.
    \]
    Como \(\Vert Tx \Vert\leq 1\), quando \(\Vert x \Vert\leq \delta \), temos 
      \[
        \biggl\Vert T \frac{\delta x}{\Vert x \Vert} \biggr\Vert\leq 1
      \]
      para x não nulo. Com isso, 
        \[
          \Vert Tx \Vert \leq \delta^{-1}\Vert x \Vert,\quad \forall x\in X.
        \]

        Finalmente, para ver como c implica em a, note que se existe \(c>0\) tal que \(\Vert Tx - Ty \Vert\leq c \Vert x-y \Vert\) para todo x, y em X, então dado \(\varepsilon >0\), escolhemos 
          \[
            \delta = \frac{\varepsilon }{c},
          \]
          donde segue, portanto, que
            \[
              \Vert x-y \Vert<\delta  \Rightarrow \Vert Tx - Ty \Vert < c \frac{\varepsilon }{c} = \varepsilon .\quad \text{\qedsymbol}
            \]
\end{proof*}

\begin{example}
  Daqui adiante, denotaremos por \(L(X, Y)\) o conjunto das transformações lineares e contínuas de X em Y. Defina a seguinte norma: 
 \begin{align*}
   \Vert T \Vert&\coloneqq \inf_{x\in X}\{c\geq 0:\; \Vert Tx \Vert\leq c \Vert x \Vert\}\\
                & = \sup_{\substack{\Vert x \Vert\in X \\ x\neq0}} \frac{\Vert Tx \Vert}{\Vert x \Vert}\\ 
                & = \sup_{\Vert x \Vert=1} \Vert Tx \Vert.
 \end{align*}
\end{example}
 
\begin{prop*}
  Se Y é completo, então \(L(X, Y)\) é completo. 
\end{prop*}
\begin{proof*}
  Seja \(\{T_{n}\}\) uma sequência de Cauchy em \(L(X, Y)\). Então, \(\{T_{n}x\}\) é de Cauchy em Y; defina 
    \[
      Tx = \lim_{n\to \infty}T_{n}x,
    \]
    tal que T é linear e que 
      \[
        \Vert Tx \Vert=\lim_{n\to \infty}\Vert T_{n}x \Vert \leq \limsup_{n\geq 1}\Vert T_{n} \Vert \Vert x \Vert.
      \]

      Logo, T é uma transformação linear de X a Y. Além disso, dado \(\varepsilon >0\), existe N natural tal que, para todos \(m,n>N\), 
        \[
          \Vert T_{n}x - T_{m}x \Vert = \Vert T_{n}-T_{m} \Vert \Vert x \Vert < \varepsilon \Vert x \Vert,\quad \forall x\in X.
        \]
        Passando o limite, quando m tende a infinito, obtemos 
          \[
            \Vert T_{n}x - Tx \Vert \leq \varepsilon \Vert x \Vert,\quad \forall n\geq N\;\&\; \forall x\in X.
          \]
          Portanto, para todo \(n\geq N\) e \(T_{n}\rightarrow T\),
            \[
              \Vert T_{n}-T \Vert<\varepsilon 
            \]
            em \(L(X, Y)\). \qedsymbol
\end{proof*}
  Em particular, se \(L(X, Y)\) é completo, então Y é completo também. Nele, vale a propriedade de que, dadas duas transformações \(T\in L(X,Y)\) e \(S\in L(Y, Z)\), então \(S\circ T\in L(X, Z)\) e 
    \[
      \Vert S\circ T \Vert\leq \Vert S \Vert \Vert T \Vert.
    \]

   \begin{def*}
  Uma transformação \(T\in L(X, Y)\) é \textbf{inversível} ou um \textbf{isomorfismo} se T é bijetora e \(T^{-1}\in L(Y, X)\), isto é, para algum \(c>0\),
    \[
      \Vert Tx \Vert_{Y} \geq c \Vert x \Vert_{X}.
    \]    
    Diremos que T é uma \textbf{isometria} se 
      \[
        \Vert Tx \Vert_{Y} = \Vert x \Vert_{X},\quad \forall x\in X.
      \]
   \end{def*}
  \begin{def*}
    Seja X um espaço vetorial sobre \(\mathbb{K}.\) Uma função linear \(f:X\rightarrow \mathbb{K}\) é chamada um \textbf{funcional linear.}  Se X é um espaço vetorial \(L(X, \mathbb{K})\) é um espaço de Banach chamado \textbf{espaço dual de X}, denotado por \(X^{*}. \;\square\)
  \end{def*}
   \begin{tcolorbox}[
   skin=enhanced,
   title=Observação,
   fonttitle=\bfseries,
 colframe=black,
   colbacktitle=cyan!75!white, 
   colback=cyan!15,
   colbacklower=black,
 coltitle=black,
   drop fuzzy shadow,
   %drop large lifted shadow
   ]
   Se X é um espaço vetorial sobre \(\mathbb{C}\), ele também é um sobre \(\mathbb{R}\), e podemos considerar funcionais lineares reais \(f:X\rightarrow \mathbb{R}\) ou complexos \(f:X\rightarrow \mathbb{C}.\)
   \end{tcolorbox}

  \begin{prop*}
    Seja X um espaço vetorial sobre \(\mathbb{C}.\) Se \(f:X\rightarrow \mathbb{C}\) é um funcional linear e \(u= \mathcal{R}f\) (onde \(\mathcal{R}\) denota a parte real de f), então u é um funcional linear real e 
      \[
        f(x)=u(x)-iu(x),\quad \forall x\in X.
      \]

      Reciprocamente, se \(u:X\rightarrow \mathbb{R}\) é um funcional linear real e \(f:X\rightarrow \mathbb{C}\) é definido por 
        \[
          f(x)=u(x)-iu(ix),
        \]
        então f é um funcional linear complexo. Além disso, se X é normado, f é limitado se, e somente se, u é limitado; neste caso, \(\Vert f \Vert = \Vert u \Vert.\)
  \end{prop*}
 \begin{proof*}
   Se \(f:X\rightarrow \mathbb{C}\) é linear, então \(u=\mathcal{R}f\) é linear e 
     \[
       \mathcal{I}f(x)=-\mathcal{R}if(x)=-\mathcal{R}f(ix)=-u(ix),
     \]
     onde \(\mathcal{I}\) denota a parte imaginária.

     Por outro lado, se u é um funcional linear real, 
       \[
         f(x)=u(x)-iu(ix)
       \]
       é linear. Se X é normado e f é limitado, \(|u(x)|=|\mathcal{R}f(x)|\leq |f(x)|.\) Logo, u é limitado e \(\Vert u \Vert\leq \Vert f \Vert.\)

       Por outro lado, se u é limitado, 
         \[
           |f(x)|=\underbrace{e^{\mathrm{arg}(f(x))}f(x)}_{\alpha }=f(\alpha x)=u(\alpha x)\in \mathbb{R}.
         \]
         Logo, 
           \[
             |f(x)|\leq \Vert u \Vert \Vert \alpha x \Vert = \Vert u \Vert \Vert x \Vert
           \]
           e f é limitado com \(\Vert f \Vert\leq \Vert u \Vert.\) Portanto, 
             \[
               \Vert f \Vert=\Vert u \Vert.\quad \text{\qedsymbol}
             \]
 \end{proof*}

 \begin{def*}
   Se X é normado, um \textbf{funcional sublinear} é uma função \(p:X\rightarrow \mathbb{R}\) tal que, para todo x, y em X e \(\lambda \geq 0\),
    \begin{align*}
      & p(x+y)\leq p(x)+p(y)\\ 
      & p(\lambda x) = \lambda p(x).\quad \square
    \end{align*}
 \end{def*}
  \hypertarget{hahn_banach}{ 
    \begin{theorem*}[Hahn-Banach]
   Seja X um espaço vetorial real, p um funcional sublinear em X, M um subespaço vetorial de X e f um funcional linear em M tal que \(f(x)\leq p(x)\) para todo x em M. Então, existe um funcional linear F em X tal que 
     \[
       F(x)\leq p(x),\quad \forall x\in X
     \]
  e F estende f, isto é, \(F|_{M}=f.\)
 \end{theorem*}}
\begin{proof*}
  Começamos mostrando que, se x é um ponto de X fora de M, podemos estender f a um funcional linear g definido sobre \(M+\mathbb{R}x\) e satisfazendo 
    \[
      g(y)\leq p(y),\quad \forall y\in M + \mathbb{R}x.
    \]
    Se \(y_1, y_2\) são pontos de M, temos 
      \[
        f(y_1)+f(y_2) = f(y_1+y_2)\leq p(y_1+y_2)\leq p(y_1-x)+p(x+y_2),
      \]
      ou 
        \[
          f(y_1)-p(y_1-x)\leq p(x+y_2)-f(y_2).
        \]

        Logo, 
          \[
            r_1=\sup_{y\in M}\{f(y)-p(y-x)\}\leq \inf_{y\in M}\{p(x+y)-f(y)\}=r_2.
          \]
          Seja \(\alpha \) tal que \(r_1\leq \alpha \leq r_2\) e defina \(g:M+\mathbb{R}x\rightarrow \mathbb{R}\) por 
            \[
              g(y+\lambda x)=f(y)+\lambda \alpha .
            \]
            Assim, g é linear e \(g|_{M}=f,\) o que implica, para todo y em M, 
              \[
                g(y)\leq p(y).
              \]
              Adicionalmente, se \(\lambda > 0\) e y pertence a M,  
             \begin{align*}
               g(y+\lambda x)&=\lambda \biggl[f \biggl(\frac{y}{\lambda }\biggr)+\alpha \biggr]\\ 
                             &\leq \lambda \biggl[f\biggl(\frac{y}{\lambda }\biggr) + p \biggl(x + \frac{y}{\lambda }\biggr) - f \biggl(\frac{y}{\lambda }\biggr)\biggr]\\ 
                             & p(y+\lambda x),
             \end{align*}
             enquanto que, no caso em que \(\lambda =-\mu <0\), 
              \begin{align*}
               g(y+\lambda x)&=\mu \biggl[f \biggl(\frac{y}{\mu }\biggr)-\alpha \biggr]\\ 
                             &\leq \mu \biggl[f\biggl(\frac{y}{\mu }\biggr) + p \biggl(x + \frac{y}{\mu }\biggr) - f \biggl(\frac{y}{\mu }\biggr)\biggr]\\ 
                             & = p(y - \mu x)= p(y+\mu x).
x            \end{align*}
             Logo, para todo z em \(M + \mathbb{R}x\), 
               \[
                 g(z)\leq p(z),
               \]
               mostrando que o domínio de uma extensão linear maximal de f satisfazendo \(f\leq p\) deve necessariamente ser o espaço todo.

                Para finalizar, seja \(\mathcal{F}\) a família de todas as extensões de f satisfazendo \(f\leq p\) e parcialmente ordenado pela inclusão nos gráficos. Sendo um conjunto linearmente ordenado de extensões, ele tem a união como limitante superior, então podemos usar o Lema de Zorn que \(\mathcal{F}\) para obter um elemento maximal. \qedsymbol
\end{proof*}
  Se p é uma seminorma (ou seja, 0 não é necessariamente o único ponto onde ela se anula) e \(f:X\rightarrow \mathbb{R}\), a desigualdade \(f\leq p\) é equivalente a \(|f|\leq p\), pois 
    \[
      |f(x)| = \pm f(x)= f(\pm x)< p(\pm x) = p(x).
    \]

    \hypertarget{complex_hahn_banach}{
     \begin{theorem*}
       Seja X um espaço vetorial complexo, p uma seminorma em X, M um subespaço vetorial de X e \(f:M\rightarrow \mathbb{C}\) linear com 
         \[
           |f(x)|\leq p(x),\quad x\in M.
         \]
         Então, existe \(F:X\rightarrow \mathbb{C}\) linear tal que, para todo x em X, 
           \[
             |F(x)|\leq p(x)
           \]
           e \(F|_{M}=f.\)
     \end{theorem*}
    }
   \begin{proof*}
     Seja u a parte real de f. Pelo \hyperlink{hahn_banach}{\textit{Teorema Anterior}}, existe uma extensão linear U de u a X tal que 
       \[
         |U(x)|\leq p(x),\quad \forall x\in X.
       \]
       Seja \(F(x)=U(x)-iU(ix)\); então, F é uma extensão linear complexa de f e, para cada x em X, se \(\alpha =e^{-i \mathrm{arg}(F(x))}\), temos 
         \[
           |F(x)|=\alpha F(x)=F(\alpha x)=U(\alpha x)\leq p(\alpha x)=p(x).\quad \text{\qedsymbol}.\quad \text{\qedsymbol}
         \]
   \end{proof*}

  \begin{theorem*}
    Seja X um espaço vetorial normado. 
   \begin{itemize}
     \item[a)] Se M é um subespaço vetorial de X, fechado, e x é um ponto fora de M, então existe f no dual \(X^{*}\) de X tal que 
       \[
         f(x)\neq 0 \quad\&\quad f|_{M}=0.
       \]
       Mais ainda, se \(\delta =\inf_{y\in M}\Vert x-y \Vert\), f pode ser tomada tal que 
         \[
           \Vert f \Vert=1 \quad\&\quad f(x)=\delta ;
         \]
         \item[b)] Se \(x\neq 0,\) existe \(f\in X^{*}\) tal que 
           \[
             \Vert f \Vert=1\quad\&\quad f(x)=\Vert x \Vert;
           \]
        \item[c)] Os funcionais lineares limitados em X separam pontos; 
          \item[d)] Se x em X define um funcional \(\hat{x}:X^{*}\rightarrow \mathbb{C}\) por 
            \[
              \hat{x}(f)=f(x),\quad \forall f\in X^{*},
            \]
            então a transformação 
              \[
                x \substack{T \\ \longrightarrow \\ }\hat{x}
              \]
              é uma isometria linear de X em \(X^{**}.\)
   \end{itemize}
  \end{theorem*}
 \begin{proof*}
  a) Defina f em \(M+\mathbb{C}x\) por \(f(y+\lambda x)=\lambda \delta \), onde y é um elemento de M e \( \lambda \) é um número complexo. Então, \(f(x)=\delta ,\; f|_{M}=0\) e, para \(\lambda \) diferente de zero, 
    \[
      |f(y+\lambda x)|=|\lambda |\Vert \lambda^{-1}y + x \Vert=\Vert y+\lambda x \Vert,
    \]
    e o resultado segue do \hyperlink{complex_hahn_banach}{\textit{Teorema de Hahn Banach}} com \(p(x)=\Vert x \Vert\) e M substituído por \(M + \mathbb{C}x\).

  b) É um caso especial de (a) com \(M=0.\)

  c) Se \(x\neq y\), existe f em \(X^{*}\) com \(f(x-y)\neq 0\), ou seja, \(f(x)\neq f(y)\). 

  d) A transformação \(\hat{x}\), conforme fora definida, já é linear de \(X^{*}\) em \(\mathbb{K}.\) Com relação à transformação 
    \[
      x\substack{T \\ \longrightarrow \\ }\hat{x},
    \]
    podemos ver sua linearidade a partir de 
   \begin{align*}
     T(\alpha x+\beta y)(f)=(\hat{\alpha x+\beta y})(f)&=f(\alpha x+\beta y)\\ 
                                                       &=\alpha f(x)+ \beta f(y)\\ 
                                                       &=\alpha \hat{x}(f)+\beta \hat{y}(f)\\ 
                                                       &=\alpha T(x)(f)+\beta T(y)(f),
   \end{align*}
   que é válido para toda f no dual de X. Note que 
     \[
       |\hat{x}(f)|=|f(x)|\leq \Vert f \Vert \Vert x \Vert \Rightarrow \Vert \hat{x} \Vert \leq \Vert x \Vert.
     \]

     Por outro lado, de b, existe f em \(X^{*}\) tal que \(f(x)=\Vert x \Vert,\; \Vert f \Vert=1\) e, portanto,
       \[
         |\hat{x}(f)| = f(x) = \Vert x \Vert \quad\&\quad \Vert \hat{x} \Vert\geq \Vert x \Vert.\quad \text{\qedsymbol}
       \] 
 \end{proof*}

 Com a notação do teorema acima, seja \(\hat{X}=\{\hat{x}:\; x\in X\}.\) Como \(X^{**}\) é Banach, segue que \([\hat{X}]^{-}\) é Banach e, consequentemente, o mapeamento 
   \[
     x\ni X \mapsto \hat{x}\in \hat{X}
   \]
   é uma imersão densa de X em \([\hat{X}]^{-}\). O espaço \([\hat{X}]^{-}\) é chamado \textbf{completamento} de X, e, se X é Banach, \([\hat{X}]^{-}=\hat{X}.\) Observe que, se X tem dimensão finita , então \(\hat{X}=X^{**}\), já que ambos têm a mesma dimensão, mas se ela for infinita nem sempre temos esta sorte; para os casos onde a igualdade se mantém, X é dito \textbf{reflexivo}. Em geral, identificamos X com \(\hat{X}\) e consideramos X um subespaço de \(X^{**}\), tal que a propriedade de X ser reflexivo torna-se \(X=X^{**}.\)

\end{document}

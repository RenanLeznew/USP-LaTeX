\documentclass[../functional_analysis.tex]{subfiles}
\begin{document}
\section{Aula 02 - 07 de Agosto, 2025}
\subsection{Motivações}
\begin{itemize}
	\item Espaços de Banach;
	\item Transformações Lineares.
\end{itemize}
\subsection{Espaços de Banach e Transformações Lineares}
\begin{def*}
	Um espaço vetorial normado que é completo com a métrica induzida pela norma, isto é,
	\[
		\rho (x, y) = \Vert x-y \Vert,
	\]
	é dito \textbf{Espaço de Banach.} \(\square\)
\end{def*}
Veremos mais tarde que todo espaço vetorial normado pode ser imerso em um espaço de Banach.

\begin{def*}
	Duas normas \(\Vert \cdot  \Vert_{1}\) e \(\Vert \cdot \Vert_{2}\) em X são \textbf{equivalentes} se existem constantes positivas \(c_1\) e \(c_2\) tais que
	\[
		c_1 \Vert x \Vert_1 \leq \Vert x \Vert_2 \leq c_2\Vert x \Vert_1,\quad \forall x\in X.\; \square
	\]
\end{def*}
\begin{def*}
	Uma série \(\sum\limits_{j=1}^{\infty}x_{j}\) é dita \textbf{convergente} em X se
	\[
		\sum\limits_{j=1}^{n}x_{j}\substack{n\to \infty \\ \longrightarrow \\ }x
	\]
	e \textbf{absolutamente convergente} se
	\[
		\sum\limits_{j=1}^{\infty}\Vert x_{j} \Vert
	\]
	converge. \(\square\)
\end{def*}
\begin{theorem*}
	Um espaço vetorial normado é completo se, e somente se, toda série absolutamente convergente é convergente.
\end{theorem*}
\begin{proof*}
	Se X é um espaço de Banach e
	\[
		\sum\limits_{j=1}^{\infty}\Vert x_{j} \Vert<\infty,
	\]
	é fácil ver que \(\biggl\{ \sum\limits_{j=1}^{n}x_{j}\biggr\}\) é uma sequência de Cauchy, portanto convergente por definição.

	Por outro lado, se X é um espaço vetorial normado X onde toda série absolutamente convergente é convergente e \(\{x_{n}\}\) é uma sequência de Cauchy, então existem \(n_1<n_2<\dotsc \) naturais tais que
	\[
		\Vert x_{n}-x_{m} \Vert\leq 2^{-j},\quad n, m\geq j,
	\]
	então escolhemos \(y_1=x_{n_1},\) e, para \(j\geq 2\), \(y_{j}=x_{n_{j}}-x_{n_{j-1}}\).
	Logo,
	\[
		\sum\limits_{j=1}^{k}y_{j}=x_{n_{k}}\;\&\; \sum\limits_{j=1}^{k}\Vert y_{j} \Vert\leq \Vert y_1 \Vert +0\sum\limits_{j=1}^{k}2^{-j}<\Vert y_1 \Vert + 1 < \infty.
	\]
	Portanto, \(\{x_{n_{k}}\}\) é convergente e \(\{x_{n}\}\) é convergente. \qedsymbol
\end{proof*}
\begin{def*}
	Uma transformação \(T:X\rightarrow Y\) linear entre dois espaços vetoriais normados é \textbf{limitada} se existe uma constante \(c\geq 0\) tal que
	\[
		\Vert Tx \Vert_{Y}\leq c \Vert x \Vert_{X}, \quad \forall x\in X.\; \square
	\]
\end{def*}
\begin{prop*}
	Se X, Y são espaços vetoriais normados \(T:X\rightarrow Y\) é linear, são equivalentes:
	\begin{itemize}
		\item[a)] T é contínua;
		\item[b)] T é contínua em 0;
		\item[c)] T é limitada.
	\end{itemize}
\end{prop*}
\begin{proof*}
	Provaremos esta proposição em ciclo de \(a \Rightarrow b \Rightarrow c \Rightarrow a.\) A primeira delas, no entanto, é trivial; para ver que b implica c, tomando \(\varepsilon =1\), existe \( \delta >0\) tal que
	\[
		T([B_\delta (0)]^{-})\subseteq T(B_{2\delta }(0)) \subseteq \{y\in Y:\; \Vert y \Vert<1\}.
	\]
	Como \(\Vert Tx \Vert\leq 1\), quando \(\Vert x \Vert\leq \delta \), temos
	\[
		\biggl\Vert T \frac{\delta x}{\Vert x \Vert} \biggr\Vert\leq 1
	\]
	para x não nulo. Com isso,
	\[
		\Vert Tx \Vert \leq \delta^{-1}\Vert x \Vert,\quad \forall x\in X.
	\]

	Finalmente, para ver como c implica em a, note que se existe \(c>0\) tal que \(\Vert Tx - Ty \Vert\leq c \Vert x-y \Vert\) para todo x, y em X, então dado \(\varepsilon >0\), escolhemos
	\[
		\delta = \frac{\varepsilon }{c},
	\]
	donde segue, portanto, que
	\[
		\Vert x-y \Vert<\delta  \Rightarrow \Vert Tx - Ty \Vert < c \frac{\varepsilon }{c} = \varepsilon .\quad \text{\qedsymbol}
	\]
\end{proof*}

\begin{example}
	Daqui adiante, denotaremos por \(L(X, Y)\) o conjunto das transformações lineares e contínuas de X em Y. Defina a seguinte norma:
	\begin{align*}
		\Vert T \Vert & \coloneqq \inf_{x\in X}\{c\geq 0:\; \Vert Tx \Vert\leq c \Vert x \Vert\} \\
		              & = \sup_{\substack{\Vert x \Vert\in X                                     \\ x\neq0}} \frac{\Vert Tx \Vert}{\Vert x \Vert}\\
		              & = \sup_{\Vert x \Vert=1} \Vert Tx \Vert.
	\end{align*}
\end{example}

\begin{prop*}
	Se Y é completo, então \(L(X, Y)\) é completo.
\end{prop*}
\begin{proof*}
	Seja \(\{T_{n}\}\) uma sequência de Cauchy em \(L(X, Y)\). Então, \(\{T_{n}x\}\) é de Cauchy em Y; defina
	\[
		Tx = \lim_{n\to \infty}T_{n}x,
	\]
	tal que T é linear e que
	\[
		\Vert Tx \Vert=\lim_{n\to \infty}\Vert T_{n}x \Vert \leq \limsup_{n\geq 1}\Vert T_{n} \Vert \Vert x \Vert.
	\]

	Logo, T é uma transformação linear de X a Y. Além disso, dado \(\varepsilon >0\), existe N natural tal que, para todos \(m,n>N\),
	\[
		\Vert T_{n}x - T_{m}x \Vert = \Vert T_{n}-T_{m} \Vert \Vert x \Vert < \varepsilon \Vert x \Vert,\quad \forall x\in X.
	\]
	Passando o limite, quando m tende a infinito, obtemos
	\[
		\Vert T_{n}x - Tx \Vert \leq \varepsilon \Vert x \Vert,\quad \forall n\geq N\;\&\; \forall x\in X.
	\]
	Portanto, para todo \(n\geq N\) e \(T_{n}\rightarrow T\),
	\[
		\Vert T_{n}-T \Vert<\varepsilon
	\]
	em \(L(X, Y)\). \qedsymbol
\end{proof*}
Em particular, se \(L(X, Y)\) é completo, então Y é completo também. Nele, vale a propriedade de que, dadas duas transformações \(T\in L(X,Y)\) e \(S\in L(Y, Z)\), então \(S\circ T\in L(X, Z)\) e
\[
	\Vert S\circ T \Vert\leq \Vert S \Vert \Vert T \Vert.
\]

\begin{def*}
	Uma transformação \(T\in L(X, Y)\) é \textbf{inversível} ou um \textbf{isomorfismo} se T é bijetora e \(T^{-1}\in L(Y, X)\), isto é, para algum \(c>0\),
	\[
		\Vert Tx \Vert_{Y} \geq c \Vert x \Vert_{X}.
	\]
	Diremos que T é uma \textbf{isometria} se
	\[
		\Vert Tx \Vert_{Y} = \Vert x \Vert_{X},\quad \forall x\in X.
	\]
\end{def*}
\begin{def*}
	Seja X um espaço vetorial sobre \(\mathbb{K}.\) Uma função linear \(f:X\rightarrow \mathbb{K}\) é chamada um \textbf{funcional linear.}  Se X é um espaço vetorial \(L(X, \mathbb{K})\) é um espaço de Banach chamado \textbf{espaço dual de X}, denotado por \(X^{*}. \;\square\)
\end{def*}
\begin{tcolorbox}[
		skin=enhanced,
		title=Observação,
		fonttitle=\bfseries,
		colframe=black,
		colbacktitle=cyan!75!white,
		colback=cyan!15,
		colbacklower=black,
		coltitle=black,
		drop fuzzy shadow,
		%drop large lifted shadow
	]
	Se X é um espaço vetorial sobre \(\mathbb{C}\), ele também é um sobre \(\mathbb{R}\), e podemos considerar funcionais lineares reais \(f:X\rightarrow \mathbb{R}\) ou complexos \(f:X\rightarrow \mathbb{C}.\)
\end{tcolorbox}

\begin{prop*}
	Seja X um espaço vetorial sobre \(\mathbb{C}.\) Se \(f:X\rightarrow \mathbb{C}\) é um funcional linear e \(u= \mathcal{R}f\) (onde \(\mathcal{R}\) denota a parte real de f), então u é um funcional linear real e
	\[
		f(x)=u(x)-iu(x),\quad \forall x\in X.
	\]

	Reciprocamente, se \(u:X\rightarrow \mathbb{R}\) é um funcional linear real e \(f:X\rightarrow \mathbb{C}\) é definido por
	\[
		f(x)=u(x)-iu(ix),
	\]
	então f é um funcional linear complexo. Além disso, se X é normado, f é limitado se, e somente se, u é limitado; neste caso, \(\Vert f \Vert = \Vert u \Vert.\)
\end{prop*}
\begin{proof*}
	Se \(f:X\rightarrow \mathbb{C}\) é linear, então \(u=\mathcal{R}f\) é linear e
	\[
		\mathcal{I}f(x)=-\mathcal{R}if(x)=-\mathcal{R}f(ix)=-u(ix),
	\]
	onde \(\mathcal{I}\) denota a parte imaginária.

	Por outro lado, se u é um funcional linear real,
	\[
		f(x)=u(x)-iu(ix)
	\]
	é linear. Se X é normado e f é limitado, \(|u(x)|=|\mathcal{R}f(x)|\leq |f(x)|.\) Logo, u é limitado e \(\Vert u \Vert\leq \Vert f \Vert.\)

	Por outro lado, se u é limitado,
	\[
		|f(x)|=\underbrace{e^{\mathrm{arg}(f(x))}f(x)}_{\alpha }=f(\alpha x)=u(\alpha x)\in \mathbb{R}.
	\]
	Logo,
	\[
		|f(x)|\leq \Vert u \Vert \Vert \alpha x \Vert = \Vert u \Vert \Vert x \Vert
	\]
	e f é limitado com \(\Vert f \Vert\leq \Vert u \Vert.\) Portanto,
	\[
		\Vert f \Vert=\Vert u \Vert.\quad \text{\qedsymbol}
	\]
\end{proof*}

\end{document}

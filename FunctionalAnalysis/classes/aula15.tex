\documentclass[../functional_analysis.tex]{subfiles}
\begin{document}
\section{Aula 15 - 25 de Setembro, 2025}
\subsection{Motivações}
\begin{itemize}
	\item Operadores Simétrico;
	\item Operadores Auto-Adjuntos.
\end{itemize}
\subsection{Operadores Simétricos e Auto-Adjuntos}
\begin{def*}
	Seja H um espaço de Hilbert com produto interno \(\langle \cdot , \cdot  \rangle_{H}\). Se \(A:D(A)\subseteq H\rightarrow H\) é um operador densamente definido, o \textbf{adjunto de A} é definido por
	\(A^{\tikz[baseline=-0.5ex]\draw[black, fill=black, radius=2pt](0, 0)circle;}:D(A^{\tikz[baseline=-0.5ex]\draw[black, fill=black, radius=2pt](0, 0)circle;})\subseteq H\rightarrow H\)
	tal que
	\[
		D(A^{\tikz[baseline=-0.5ex]\draw[black, fill=black, radius=2pt](0, 0)circle;}) = \{u\in H:\; v\mapsto \langle Av, u \rangle_{H}:D(A)\rightarrow \mathbb{K} \text{ é limitado}\}.
	\]
\end{def*}
Caso u pertença a \(D(A^{\tikz[baseline=-0.5ex]\draw[black, fill=black, radius=2pt](0, 0)circle;})\), o \hyperlink{riesz_representation}{\textit{Teorema da Representação de Riesz}} garante que \(A^{\tikz[baseline=-0.5ex]\draw[black, fill=black, radius=2pt](0, 0)circle;}u\) é o
único elemento de H tal que
\[
	\langle v, A^{\tikz[baseline=-0.5ex]\draw[black, fill=black, radius=2pt](0, 0)circle;}u \rangle_{H} = \langle Av, u \rangle_{H},\; \forall v\in D(A).
\]
\begin{tcolorbox}[
		skin=enhanced,
		title=Observação,
		fonttitle=\bfseries,
		colframe=black,
		colbacktitle=cyan!75!white,
		colback=cyan!15,
		colbacklower=black,
		coltitle=black,
		drop fuzzy shadow,
		%drop large lifted shadow
	]
	Se H é um espaço de Hilbert sobre \(\mathbb{C}\), o mapa \(E:H\rightarrow H^{*}\) é definido por \(Eu(v) = \langle v, u \rangle_{H}\) é uma isometria
	linear-conjugada entre H e \(H^{*}\), tal que podemos identificar H e \(H^{*}\) identificando u com Eu.

	De fato, se \(A^{*}:D(A^{*})\subseteq X^{*}\rightarrow X^{*}\) é o dual de A, então
	\[
		A^{\tikz[baseline=-0.5ex]\draw[black, fill=black, radius=2pt](0, 0)circle;} = E^{-1}\circ A^{*}\circ E,
	\]
	e embora \(E\) e \(E^{-1}\) sejam operadores lineares-conjugados, a dupla conjugação que ocorre em \(A^{\tikz[baseline=-0.5ex]\draw[black, fill=black, radius=2pt](0, 0)circle;}\) torna-o
	apenas linear.

	Com base nisso, chamaremos ambos \(A^{\tikz[baseline=-0.5ex]\draw[black, fill=black, radius=2pt](0, 0)circle;}\) e \(A^{*}\) de adjunto de A, e ambos serão denotados por \(A^{*}\), porém é importante
	observar que, se \(A = \alpha B\), então \(A^{\tikz[baseline=-0.5ex]\draw[black, fill=black, radius=2pt](0, 0)circle;} = \overline{\alpha }B^{\tikz[baseline=-0.5ex]\draw[black, fill=black, radius=2pt](0, 0)circle;}\),
	enquanto que \(A^{*} = \alpha B^{*}\). Desta forma,
	\[
		(\lambda I - A)^{\tikz[baseline=-0.5ex]\draw[black, fill=black, radius=2pt](0, 0)circle;} = \overline{\lambda }I - A^{\tikz[baseline=-0.5ex]\draw[black, fill=black, radius=2pt](0, 0)circle;},
	\]
	enquanto que
	\[
		(\lambda I - A)^{*} = \lambda I^{*} - A^{*}.
	\]
\end{tcolorbox}

Com base na observação acima, escreveremos \(A^{*}\) para denotar os operadores dual e adjunto indistintamente.

\begin{def*}
	Seja H um espaço de Hilbert sobre \(\mathbb{K}\) com produto interno \(\langle \cdot , \cdot  \rangle_{H}\). Diremos que um operador
	\(A:D(A)\subseteq H\rightarrow H\) é \textbf{simétrico}, também chamado de \textbf{Hermitiano} quando \(\mathbb{K} = \mathbb{C}\), se
	\(\overline{D(A)} = H\) e \(A\subseteq A^{*}\), isto é, se para todo x, y em \(D(A)\),
	\[
		\langle Ax, y \rangle_{H} = \langle x, Ay \rangle_{H}.
	\]
	Caso \(A = A^{*},\) diremos que A é \textbf{auto-adjunto.} \(\square\)
\end{def*}
\begin{exr}
	Seja H um espaço de Hilbert. Se \(A:D(A)\subseteq H\rightarrow H\) é um operador densamente definido, prove que \(A^{*}:D(A^{*})\subseteq H\rightarrow H\) é fechado e que, se A é fechado, então \(A^{*}\) é
	densamente definido.
\end{exr}
\begin{exr}
	Seja H um espaço de Hilbert sobre \(\mathbb{K}\). Mostre que, se \(A:D(A)\subseteq H\rightarrow H\) é simétrico e \(\lambda \in \mathbb{K}\) é um auto-valor de A, então \(\lambda \in \mathbb{R}\). Além disso, prove que
	\[
		\inf_{\Vert x \Vert_{H} = 1} \langle Ax, x \rangle_{H} \leq \lambda \leq \sup_{\Vert x \Vert_{H} = 1}\langle Ax, x \rangle_{H}.
	\]
\end{exr}
\begin{exr}
	Seja \(H = \mathbb{C}^{n}\) com o produto usual. Se \(A = (a_{ij})_{i, j=1}^{n}\) é uma matriz com coeficientes complexos que representa um operador linear em \(A\in \mathcal{L}(H)\),
	determine \(A^{\tikz[baseline=-0.5ex]\draw[black, fill=black, radius=2pt](0, 0)circle;}\) e \(A^{*}.\)
\end{exr}
\begin{exr}
	Seja H um espaço de Hilbert sobre \(\mathbb{K}\) com produto interno \(\langle \cdot , \cdot  \rangle_{H}\) e
	\(A:D(A)\subseteq H\rightarrow H\) um operador densamente definido. Prove que
	\[
		\mathrm{Graf}(A^{*}) = \{(-Ax, x):\; x\in D(A)\}^{\perp }.
	\]
\end{exr}
\begin{prop*}
	Seja H um espaço de Hilbert sobre \(\mathbb{K}\) com produto interno \(\langle \cdot , \cdot  \rangle_{H}\). Se
	\(A:D(A)\subseteq H\rightarrow H\) é um operador auto adjunto, injetor e com imagem densa. Então, \(A^{-1}\) é
	auto adjunto.
\end{prop*}
\begin{proof*}
	Como A é auto adjunto, vale que
	\[
		\{(x, -Ax):\; x\in D(A)\}^{\perp } = \{(Ax, x):\; x\in D(A)\} = \mathrm{Graf}(A^{-1}).
	\]

	Como A é injetor e tem imagem densa, segue do exercício anterior que
	\[
		\mathrm{Graf}((A^{-1})^{*}) = \{(-A^{-1}x, x):\; x\in \mathrm{Im}(A)\}^{\perp } = \mathrm{Graf}(A^{-1}).
	\]
	Portanto,
	\[
		A^{-1} = (A^{-1})^{*}. \text{ \qedsymbol}
	\]
\end{proof*}
\end{document}

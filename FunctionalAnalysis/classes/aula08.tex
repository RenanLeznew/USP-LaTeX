\documentclass[../functional_analysis.tex]{subfiles}
\begin{document}
\section{Aula 08 - 28 de Agosto, 2025}
\subsection{Motivações}
\begin{itemize}
	\item Desigualdade de Bessel;
	\item Completamento e Identidade de Parseval;
	\item Bases Ortonormais em Espaços de Hilbert.
\end{itemize}
\subsection{Bases Ortonormais em Espaços de Hilbert}
\hypertarget{bessel_inequality}{
	\begin{theorem*}[Desigualdade de Bessel]
		Se \(\{u_{\alpha }\}_{\alpha \in A}\) for um conjunto ortonormal em H, então para \(u\in H\),
		\[
			\sum\limits_{\alpha \in A}^{}| \langle u, u_{\alpha } \rangle |^{2}\leq \Vert u \Vert^{2}.
		\]
		Em particular, o conjunto
		\[
			\{\alpha \in A:\; \langle u, u_{\alpha } \rangle\neq 0\}
		\]
		é enumerável.
	\end{theorem*}
}
\begin{proof*}
	É suficiente provar que, para todo conjunto \(F\subseteq A\) finito,
	\[
		\sum\limits_{\alpha \in F}^{}| \langle u, u_{\alpha } \rangle |^{2} \leq \Vert u \Vert.
	\]
	Com efeito,
	\begin{align*}
		0 & \leq \biggl\Vert u-\sum\limits_{\alpha \in F}^{}\langle u, u_{\alpha } \rangle u_{\alpha } \biggr\Vert^{2}                                                                                                                                    \\
		  & = \Vert u \Vert^{2}-2 \mathrm{Re}\biggl\langle u, \sum\limits_{\alpha \in F}^{}\langle u, u_{\alpha } \rangle u_{\alpha } \biggr\rangle + \biggl\Vert \sum\limits_{\alpha \in F}^{}\langle u, u_{\alpha } \rangle u_{\alpha } \biggr\Vert^{2} \\
		  & = \Vert u \Vert^{2}-2 \sum\limits_{\alpha \in F}^{}| \langle u, u_{\alpha } \rangle |^{2} + \sum\limits_{\alpha \in F}^{}| \langle u, u_{\alpha } \rangle |^{2}                                                                               \\
		  & = \Vert u \Vert^{2} - \sum\limits_{\alpha \in F}^{}| \langle u, u_{\alpha } \rangle |^{2},
	\end{align*}
	onde utilizamos o \hyperlink{pythagorean_theorem}{\textit{Teorema de Pitágoras}}. \qedsymbol
\end{proof*}

\end{document}

\documentclass[../functional_analysis.tex]{subfiles}
\begin{document}
\section{Aula 08 - 28 de Agosto, 2025}
\subsection{Motivações}
\begin{itemize}
	\item Desigualdade de Bessel;
	\item Completamento e Identidade de Parseval;
	\item Bases Ortonormais em Espaços de Hilbert.
\end{itemize}
\subsection{Bases Ortonormais em Espaços de Hilbert}
\hypertarget{bessel_inequality}{
	\begin{theorem*}[Desigualdade de Bessel]
		Se \(\{u_{\alpha }\}_{\alpha \in A}\) for um conjunto ortonormal em H, então para \(u\in H\),
		\[
			\sum\limits_{\alpha \in A}^{}| \langle u, u_{\alpha } \rangle |^{2}\leq \Vert u \Vert^{2}.
		\]
		Em particular, o conjunto
		\[
			\{\alpha \in A:\; \langle u, u_{\alpha } \rangle\neq 0\}
		\]
		é enumerável.
	\end{theorem*}
}
\begin{proof*}
	É suficiente provar que, para todo conjunto \(F\subseteq A\) finito,
	\[
		\sum\limits_{\alpha \in F}^{}| \langle u, u_{\alpha } \rangle |^{2} \leq \Vert u \Vert.
	\]
	Com efeito,
	\begin{align*}
		0 & \leq \biggl\Vert u-\sum\limits_{\alpha \in F}^{}\langle u, u_{\alpha } \rangle u_{\alpha } \biggr\Vert^{2}                                                                                                                                    \\
		  & = \Vert u \Vert^{2}-2 \mathrm{Re}\biggl\langle u, \sum\limits_{\alpha \in F}^{}\langle u, u_{\alpha } \rangle u_{\alpha } \biggr\rangle + \biggl\Vert \sum\limits_{\alpha \in F}^{}\langle u, u_{\alpha } \rangle u_{\alpha } \biggr\Vert^{2} \\
		  & = \Vert u \Vert^{2}-2 \sum\limits_{\alpha \in F}^{}| \langle u, u_{\alpha } \rangle |^{2} + \sum\limits_{\alpha \in F}^{}| \langle u, u_{\alpha } \rangle |^{2}                                                                               \\
		  & = \Vert u \Vert^{2} - \sum\limits_{\alpha \in F}^{}| \langle u, u_{\alpha } \rangle |^{2},
	\end{align*}
	onde utilizamos o \hyperlink{pythagorean_theorem}{\textit{Teorema de Pitágoras}}. \qedsymbol
\end{proof*}
\begin{theorem*}
	Se \(\{u_{\alpha }\}_{\alpha \in A}\) for um conjunto ortonormal em H, as seguintes afirmações são equivalentes:
	\begin{itemize}
		\item[a)] (Completamento): se \(\langle u, u_{\alpha } \rangle=0\) para todo \(\alpha \in A\), então \(u=0\);
		\item[b)] \hypertarget{parseval_identity}{(Identidade de Parseval):} para todo u de H,
		      \[
			      \Vert u \Vert^{2}=\sum\limits_{\alpha \in A}^{}| \langle u, u_{\alpha } \rangle |^{2}.
		      \]
		\item[c)] Todo elemento u de H pode ser escrito como a soma de cada projeção dele sobre os elementos do conjunto ortonormal; isto é,
		      \[
			      u=\sum\limits_{\alpha \in A}^{} \langle u, u_{\alpha } \rangle u_{\alpha },
		      \]
		      onde a soma tem apenas um número enumerável de termos não nulos e converge independente da ordem dos termos.
	\end{itemize}
\end{theorem*}
\begin{proof*}
	\(a \Rightarrow c):\) se x pertence a H, seja \(\alpha_1,\alpha_2,\dotsc \) uma enumeração qualquer dos \(\alpha \)'s para os quais \(\langle u, u_{\alpha } \rangle\neq 0\). Pela \hyperlink{bessel_inequality}{\textit{Desigualdade de Bessel,}} garantimos a convergência da série
	\[
		\sum\limits_{j=1}^{\infty}| \langle u, u_{\alpha_{j}} \rangle |^{2}
	\]
	e, pelo \hyperlink{pythagorean_theorem}{\textit{Teorema de Pitágoras}},
	\[
		\biggl\Vert \sum\limits_{j=m}^{n}\langle u, u_{\alpha_{j}} \rangle u_{\alpha_{j}} \biggr\Vert^{2}=\sum\limits_{j=m}^{n}| \langle u, u_{\alpha_{j}} \rangle |^{2} \substack{ \\ \longrightarrow \\ m,n\to \infty}0.
	\]
	Por H ser completo, a série \(\sum\limits_{j=0}^{\infty}\langle u, u_{\alpha_{j}} \rangle u_{\alpha_{j}}\) é convergente. Assim, se
	\[
		v= u - \sum\limits_{j=0}^{\infty}\langle u, u_{\alpha_{j}} \rangle u_{\alpha_{j}},
	\]
	temos \(\langle v, u_{\alpha } \rangle =0\) para todo \(\alpha \). Logo, pelo item (a), v deve ser nulo.

	\(c \Rightarrow b):\) com a notação acima, assim como na prova da \hyperlink{bessel_inequality}{\textit{Desigualdade de Bessel}}, temos
	\[
		\Vert u \Vert^{2} -\sum\limits_{j=1}^{n}| \langle u, u_{\alpha_{j}} \rangle |^{2}=\biggl\Vert u - \sum\limits_{j=1}^{n}\langle u, u_{\alpha_{j}} \rangle u_{\alpha_{j}} \biggr\Vert^{2}\substack{ \\ \longrightarrow \\ n\to \infty}0.
	\]

	\(b \Rightarrow a):\) automático. \qedsymbol
\end{proof*}
\begin{def*}
	Um conjunto ortonormal tendo as propriedades do Teorema acima é chamado uma \textbf{base ortonormal de H}. \(\square\)
\end{def*}

Para cada \(\alpha \in A\), defina \(e_{\alpha }\in l^{2}(A)\) por:
\[
	e_{\alpha }(\alpha ) =1 \quad\&\quad e_{\alpha }(\beta )=0,\; \alpha \neq \beta .
\]
A família \(\{e_{\alpha }\}_{\alpha \in A}\) é ortonormal e, para qualquer \(f\in l^{2}(A)\), temos
\[
	\langle f, e_{\alpha } \rangle = f(\alpha ),
\]
donde segue que \(\{e_{\alpha }\}_{\alpha \in A}\) é uma base ortonormal.
\begin{theorem*}
	Todo espaço de Hilbert tem uma base ortonormal.
\end{theorem*}
\begin{proof*}
	Basta aplicar o \hyperlink{zorn_lemma}{\textit{Lema de Zorn}} para mostrar que a coleção de todos os conjuntos ortonormais, ordenados pela inclusão, tem um elemento maximal, que equivale à propriedade (a) da definição de uma base ortonormal. \qedsymbol
\end{proof*}

\begin{theorem*}
	Um espaço de Hilbert H é separável se, e somente se, tem uma base ortonormal enumerável. Neste caso, toda base ortonormal de H é enumerável.
\end{theorem*}
\begin{proof*}
	Se \(\{x_{n}\}\) é um conjunto enumerável denso em H, descartando recursivamente qualquer \(x_{i}\) que possa ser uma combinação linear de \(x_1,\dotsc ,x_{i-1},\) obtemos uma sequência linearmente independente \(\{y_{n}\}\) e, aplicando o processo de \hyperlink{gram_schmidt}{\textit{Ortogonalização}}, obtemos uma sequência ortonormal \(\{u_{n}\}\) que gera um subespaço denso em H, o qual consequentemente será uma base.

	Reciprocamente, se \(\{u_{n}\}\) é uma base ortonormal, as combinações lineares finitas dos \(u_{n}\)'s com coeficientes em um subconjunto enumerável e denso em \(\mathbb{C}\) forma um subconjunto enumerável e denso em H. Além disso, se \(\{v_{\alpha }\}_{\alpha \in A}\) é outra base ortonormal, para cada n o conjunto
	\[
		A_{n}=\{\alpha \in A:\; \langle u_{n}, v_{\alpha } \rangle\neq 0\}
	\]
	é enumerável. Portanto, pelo completamento de \(\{u_{n}\}\),
	\[
		A = \bigcup_{n=1}^{\infty}A_{n},
	\]
	mostrando que ele é enumerável. \qedsymbol
\end{proof*}
\begin{def*}
	Se \(H_1\) e \(H_2\) são espaços de Hilbert com produtos escalares \(\langle \cdot , \cdot  \rangle_1\) e \(\langle \cdot , \cdot  \rangle_2\), uma \textbf{transformação unitária de \(H_1\) sobre \(H_2\)} é uma transformação linear sobrejetora \(U:H_1\rightarrow H_2\) que preserva o produto escalar, ou seja,
	\[
		\langle Ux, Uy \rangle_2 = \langle x, y \rangle_1. \quad \square
	\]
\end{def*}
\begin{exr}
	Mostre que toda transformação unitária é uma isometria e, reciprocamente, que toda isometria de \(H_1\) sobre \(H_2\) é uma transformação unitária.
\end{exr}
\begin{prop*}
	Seja \(\{e_{\alpha }\}_{\alpha \in A}\) uma base ortonormal de X. Então, a correspondência
	\[
		x\mapsto \hat{x},\quad \hat{x}(\alpha )\coloneqq \langle x, u_{\alpha } \rangle
	\]
	é uma transformação unitária em \(l^{2}(A).\)
\end{prop*}
\begin{proof*}
	A linearidade da transformação segue da linearidade do próprio produto interno, e o fato de ser uma isometria de H em \(l^{2}(A)\) é consequência da \hyperlink{parseval_identity}{\textit{Identidade de Parseval}}:
	\[
		\Vert x \Vert^{2} = \sum\limits_{\alpha \in A}^{}| \hat{x}(\alpha ) |^{2}.
	\]

	Agora, se \(f\in l^{2}(A)\), então
	\[
		\sum\limits_{\alpha \in A}^{} | f(\alpha ) |^{2}<\infty,
	\]
	e pelo \hyperlink{pythagorean_theorem}{\textit{Teorema de Pitágoras}}, as somas parciais da série \(\sum\limits_{\alpha \in A}^{}f(\alpha )u_{\alpha }\), da qual apenas um número enumerável de termos são não nulos, formam uma sequência de Caucho. Portanto,
	\[
		x = \sum\limits_{\alpha \in A}^{}f(\alpha )u_{\alpha }
	\]
	existe em H e \(\hat{x}=f.\) \qedsymbol
\end{proof*}

\end{document}

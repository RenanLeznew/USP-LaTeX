\documentclass[../functional_analysis.tex]{subfiles}
\begin{document}
\section{Aula 13 - 18 de Setembro, 2025}
\subsection{Motivações}
\begin{itemize}
	\item Operadores Duais;
	\item Anuladores;
	\item Operadores Compactos.
\end{itemize}
\subsection{Operadores Duais e Anuladores}
Nesta seção, recordaremos a definição de operadores duais e trabalharemos a teoria deles.
Para isso, sejam X e Y espaços vetoriais normados sobre um corpo \(\mathbb{K}\) com duais \(X^{*}\) e \(Y^{*}\).

Se \(x^{*}\in X^{*}\) e \(y^{*}\in Y^{*}\), denotaremos o seu valor em um vetor x ou y por \(\langle x, x^{*} \rangle\) ou \(\langle y, y^{*} \rangle\) (Note que \((X\times Y)^{*} = X^{*}\times Y^{*}\)).
\begin{def*}
	Se \(A:D(A)\subseteq X\rightarrow Y\) é um operador linear densamente definido, \textbf{o dual} \(A^{*}:D(A^{*})\subseteq Y^{*}\rightarrow X^{*}\) de A é o operador linear definido da seguinte forma:
	\[
		D(A^{*}) = \{y^{*}\in Y^{*}:\; \exists z^{*}\in X^{*} \text{ tal que } \langle Ax, y^{*} \rangle = \langle x, z^{*} \rangle, \forall x\in D(A)\}.
	\]
	Se \(y^{*}\in D(A^{*})\), definimos \(A^{*}y^{*}\coloneqq z^{*}\), onde \(z^{*}\) é o único elemento de \(X^{*}\) satisfazendo
	\[
		\langle Ax, y^{*} \rangle = \langle x, A^{*}y^{*} \rangle, \quad \forall x\in D(A)\;\&\; \forall y^{*}\in D(A^{*}).\; \square
	\]
\end{def*}

Note que, se \(y^{*}\circ A:D(A)\subseteq X\rightarrow \mathbb{K}\) é limitado e \(\overline{D(A)}=X\), dado x em X e uma sequência
\(\{x_{n}\}\) em D(A) convergindo para x, então \(\{\langle x_{n}, y^{*}\circ A \rangle\}\) é de Cauchy em \(\mathbb{K}\) e, portanto,
convergente.

Observe que este limite independe da sequência \(\{x_{n}\}\) escolhida, então podemos definir o funcional
\[
	z^{*}(x)=\lim_{n\to \infty}\langle x_{n}, y^{*}\circ A \rangle.
\]

Para alguns resultados básicos, temos
\begin{prop*}
	Se X e Y são espaços vetoriais e \(A:D(A)\subseteq X\rightarrow Y\) é um operador linear densamente definido, então
	\[
		A^{*}:D(A^{*})\subseteq Y^{*}\rightarrow X^{*}
	\]
	é um operador linear fechado.
\end{prop*}
\begin{proof*}
	Considerando as sequências
	\[
		y_{n}^{*}\ensurestackMath{\stackon[0em]{\longrightarrow}{n\to \infty}}y^{*}
	\]
	e

	\[
		A^{*}y_{n}^{*}\ensurestackMath{\stackon[0em]{\longrightarrow}{n\to \infty}}x^{*}
	\]
	para todo x no domínio de A, temos
	\[
		\langle Ax, y^{*} \rangle \ensurestackMath{\stackon[0em]{\longleftarrow}{n\to \infty}} \langle Ax, y_{n}^{*} \rangle = \langle x, A^{*}y_{n}^{*} \rangle\ensurestackMath{\stackon[0em]{\longrightarrow}{n\to \infty}} \langle x, x^{*} \rangle.
	\]
	Portanto, \(y^{*}\in D(A^{*})\) e \(A^{*}y^{*} = x^{*}.\) \qedsymbol
\end{proof*}
\begin{lemma*}
	Sejam X e Y espaços de Banach sobre \(\mathbb{K}\) e \(A\in \mathcal{L}(X, Y)\); então, \(A^{*}\in \mathcal{L}(Y^{*}, X^{*})\) e
	\[
		\Vert A \Vert_{\mathcal{L}(X, Y)} = \Vert A^{*} \Vert_{\mathcal{L}(Y^{*}, X^{*})}.
	\]
\end{lemma*}
\begin{proof*}
	Para todo funcional \(y^{*}\in Y^{*},\; y^{*}\circ A\) é um funcional linear contínuo e assim determina um único elemento \(x^{*}\in X^{*}\) para o qual
	\[
		\langle x, x^{*} \rangle = \langle Ax, y^{*} \rangle,\quad \forall x\in X.
	\]
	Segue que \(D(A^{*})=Y^{*}.\) Portanto,
	\begin{align*}
		\Vert A^{*} \Vert_{\mathcal{L}(Y^{*}, X^{*})} = \sup_{\Vert y^{*} \Vert_{Y^{*}}\leq 1}\Vert A^{*}y^{*} \Vert_{X^{*}} & = \sup_{\Vert y^{*} \Vert_{Y^{*}}\leq 1}\sup_{\Vert x \Vert_{X}\leq 1}| \langle x, A^{*}y^{*} \rangle | \\
		                                                                                                                     & = \sup_{\Vert x \Vert_{X}\leq 1}\sup_{\Vert y^{*} \Vert_{X^{*}}\leq 1} | \langle Ax, y^{*} \rangle |    \\
		                                                                                                                     & = \sup_{\Vert x \Vert_{X}\leq 1}\Vert Ax \Vert_{Y}                                                      \\
		                                                                                                                     & = \Vert A \Vert_{\mathcal{L}(X, Y)}. \text{ \qedsymbol}
	\end{align*}
\end{proof*}
\begin{lemma*}
	Seja Y um espaço de Banach reflexivo sobre \(\mathbb{K}.\) Se \(A:D(A)\subseteq X\rightarrow Y\) é fechado e densamente definido, então \(D(A^{*})\) é denso em \(Y^{*}.\)
\end{lemma*}
\begin{proof*}
	Se \(D(A^{*})\) não é denso em \(Y^{*}\), existe um elemento \(\hat{y}\in Y^{**}\)
	\[
		\hat{y}=Jy \neq  0
	\]
	tal que \(\langle Jy, y^{*} \rangle=\langle y, y^{*} \rangle=0\) para todo \(y^{*}\in D(A^{*})\). Como A é fechado, seu gráfico é fechado em \(X\times Y\)
	e não contém \((0, y)\); agora, do \hyperlink{hahn_banach}{\textit{Teorema de Hahn-Banach}}, existem \(x^{*}\in X^{*}\) e \(y^{*}\in Y^{*}\) tais que, para
	todo \(z\in D(A)\),
	\[
		\langle z, x^{*} \rangle-\langle z, y^{*} \rangle=0
	\]
	e
	\[
		\langle 0, x^{*} \rangle - \langle y, y^{*} \rangle\neq 0.
	\]

	Logo, \(y^{*}\neq 0,\; \langle y, y^{*} \rangle\neq 0,\; y^{*}\in D(A^{*})\) e \(A^{*}y^{*}=x^{*},\) que implica em \(\langle y, y^{*} \rangle=0\), uma contradição. Portanto,
	\[
		\overline{D(A^{*})} = Y^{*}. \text{ \qedsymbol}
	\]
\end{proof*}
\begin{theorem*}
	Se \(S:D(S)\subseteq X\rightarrow X\) é um operador linear injetivo, densamente definido e com imagem densa, então \(S^{*}:D(S^{*})\subseteq X^{*}\rightarrow X^{*}\) está bem definido, é injetivo e
	\[
		(S^{*})^{-1}=(S^{-1})^{*}.
	\]
	Além disso, \(S^{-1}\in \mathcal{L}(X)\) se, e somente se, S é fechado e \((S^{*})^{-1}\in \mathcal{L}(X^{*})\).
\end{theorem*}
\begin{proof*}
	Pela densidade de \(D(S)\) e de sua imagem, vale que \(S^{*}\) e \((S^{-1})^{*}\) são bem definidos; por outro lado, se \(x^{*}\) for um ponto no domínio de \(D(S^{*})\) tal que
	\[
		S^{*}x^{*} = 0\;\& \; x\in D(S),
	\]
	então
	\[
		0 = \langle x, S^{*}x^{*} \rangle = \langle Sx, x^{*} \rangle,
	\]
	e, como a imagem de S é densa por hipótese, concluímos que \(x^{*} = 0\) e que \(S^{*}\) é injetivo, consequentemente evando à conclusão de que \((S^{*})^{-1}\) também está bem definido.

	Finalmente, seja \(x\in \mathrm{Im}(S)\) e \(x^{*}\in D(S^{*})\), e note que
	\[
		\langle x, x^{*} \rangle = \langle S\circ S^{-1}x, x^{*} \rangle = \langle S^{-1}x, S^{*}x^{*} \rangle,
	\]
	donde segue que \(D((S^{*})^{-1}) = \mathrm{Im}(S^{*}) \subseteq D((S^{-1})^{*})\) e que
	\[
		\langle S\circ S^{-1}x, x^{*} \rangle = \langle x, (S^{-1})^{*}\circ S^{*}x^{*} \rangle, \quad \forall x^{*}\in D(S^{*})\;\&\; \forall x\in \mathrm{Im}(S).
	\]
	Assim, para todo \(x^{*}\in D(S^{*})\),
	\[
		(S^{-1})^{*}\circ S^{*}x^{*} = x^{*}.
	\]

	Reciprocamente, dados \(x\in D(S)\) e \(X^{*}\in D((S^{-1})^{*})\), temos
	\[
		\langle x, x^{*} \rangle = \langle S^{-1}\circ Sx, x^{*} \rangle = \langle Sx, (S^{-1})^{*}x^{*} \rangle,
	\]
	resultando em \(\mathrm{Im}((S^{-1})^{*})\subseteq D(S^{*})\) e
	\[
		\langle S^{-1}\circ Sx, x^{*} \rangle = \langle x, S^{*}\circ (S^{-1})^{*}x^{*} \rangle, \quad \forall x^{*}\in D((S^{-1})^{*})\;\&\; \forall x\in D(S).
	\]
	Portanto, para todo \(x^{*}\in D((S^{-1})^{*})\),
	\[
		S^{*}\circ (S^{-1})^{*}x^{*} = x^{*}.\text{ \qedsymbol} \]
\end{proof*}
\end{document}

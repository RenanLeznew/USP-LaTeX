\documentclass[../functional_analysis.tex]{subfiles}
\begin{document}
\section{Aula 11 - 16 de Setembro, 2025}
\subsection{Motivações}
\begin{itemize}
	\item Teorema da Aplicação Aberta e Consequências;
	\item Princípio da Limitação Uniforme;
	\item Teorema do Gráfico Fechado
\end{itemize}
\subsection{Teoremas da Aplicação Aberta, da Limitação Uniforme e do Gráfico Fechado}
Na aula passada, enunciamos o \hyperlink{open_application}{\textit{Teorema da Aplicação Aberta}}; hoje, daremos sua prova e veremos alguns de seus resultados conseguintes e alguns outros grandes teoremas da área. Começamos por
\begin{lemma*}
	Sejam X e Y espaços vetoriais normados e \(T:X\rightarrow Y\) uma transformação linear; então, são equivalentes:
	\begin{itemize}
		\item[a)] T é uma aplicação aberta;
		\item[b)] Existe r positivo tal que \(T(B_{1}^{X}(0)\) contém \(B_{r}^{Y}(0)\).
	\end{itemize}
\end{lemma*}
\begin{proof*}
	A primeira parte segue simplesmente da definição de aberto num espaço métrico.

	Por outro lado, assuma que b ocorre e mostremos que, se \(U\) for um subconjunto aberto de X, então \(Tx\in (T(U))^{\circ }\) para todo x de U.

	Com efeito, dado x em U, tome \(s> 0\) tal que \(B_{s}^{X}(x)\subseteq U\); segue que
	\begin{align*}
		T(U)\supseteq T(B_{s}^{X}(x)) & = T(x + sB_{1}^{X}(0))       \\
		                              & = Tx + sT(B_{1}^{X}(0))      \\
		                              & \supseteq Tx + sB_{r}^{Y}(0) \\
		                              & = Tx + B_{sr}^{Y}(0)         \\
		                              & = B_{sr}^{Y}(Tx),
	\end{align*}
	provando portanto que \(Tx\) é interior a \(T(U)\). \qedsymbol
\end{proof*}

\begin{lemma*}
	Se X for Banach, Y for um espaço vetorial normado e \(T\in \mathcal{L}(X, Y)\) for tal que, para algum \(r> 0\),
	\[
		B_{r}^{Y}(0)\subseteq [T(B_{1}^{X}(0))].
	\]
	Então,
	\[
		B_{\frac{r}{2}}^{Y}(0)\subseteq T(B_{1}^{X}(0)).
	\]
\end{lemma*}
\begin{proof*}
	Como T é linear, se \(\Vert y \Vert < r2^{-n}\), então \(y\in \overline{T(B_{2^{-n}}^{X}(0))}.\)

	Agora, se \(\Vert y \Vert < r/2\), considere \(x_1\in B_{1/2}^{X}(0)\) tal que
	\[
		\Vert y-Tx_1 \Vert < \frac{r}{4};
	\]
	indutivamente, quando \(x_{n}\in B_{2^{-n}}^{X}(0)\), ele é tal que
	\[
		\biggl\Vert y- \sum\limits_{j=1}^{n}T_{j} \biggr\Vert < r2^{-n-1}.
	\]
	Como X é completo, a série \(\sum x_{n}\) é convergente; denote sua soma por x. Então,
	\[
		\Vert x \Vert < \sum\limits_{n=1}^{\infty}2^{-n}=1\;\&\; y = Tx.
	\]
	Portanto, se \(\Vert y \Vert < r/2\),
	\[
		y\in T(B_{1}^{X}(0)).\text{ \qedsymbol}
	\]

\end{proof*}
\begin{proof*}[Prova da \hyperlink{open_application}{\textit{Aplicação Aberta}}]
	Como podemos escrever
	\[
		X = \bigcup_{n=1}^{\infty}B_{n}^{X}(0),
	\]
	o conjunto
	\[
		T(X) = \bigcup_{n=1}^{\infty}T(B_{n}^{X}(0))
	\]
	é de segunda categoria em y, donde segue que o mapa de Y em Y dado por
	\[
		y\mapsto ny
	\]
	é um homeomorfismo levando \(T(B_{1}^{X}(0))\) em \(T(B_{n}^{X}(0))\). Com isso, pelo \hyperlink{baire_theorem}{\textit{Teorema de Baire}}, o conjunto
	\(T(B_{1}^{X}(0))\) não pode ser raro, ou seja, deve existir um ponto \(y_{0}\in Y\) e \(r > 0\) tais que
	\[
		B_{2r}^{Y}(y_{0})\subseteq \overline{T(B_{1}^{X}(0))}.
	\]
	Logo,
	\[
		B_{2r}^{Y}(-y_{0})\subseteq \overline{T(B_{1}^{X}(0))}\;\&\; B_{2r}^{Y}(0)\subseteq \overline{T(B_{1}^{X}(0))}.
	\]
	Portanto, pelos lemas anteriores,
	\[
		B_{r}^{Y}(0)\subseteq T(B_{1}^{X}(0)),
	\]
	provando que T é aberta. \qedsymbol
\end{proof*}
\hypertarget{uniform_limitation}{
	\begin{theorem*}[Princípio da Limitação Uniforme]
		Sejam X e Y espaços vetoriais normados e A um subconjunto de \(\mathcal{L}(X, Y)\).
		\begin{itemize}
			\item[a)] Se \(\{x\in X:\; \sup_{}\{\Vert Tx \Vert:\; T\in A\} < \infty\}\) for de segunda categoria, então
			      \[
				      \sup_{}\{\Vert T \Vert:\; T\in A\} < \infty.
			      \]
			\item[b)] Se X for Banach e \(\{x\in X:\; \sup_{}\{\Vert Tx \Vert:\; T\in A\} < \infty\} = X\), então
			      \[
				      \sup_{}\{\Vert T \Vert:\; T\in A\} < \infty;
			      \]
			\item[c)] Se X for Banach, \(\{T_{n}:\; n\in \mathbb{N}\} \subseteq \mathcal{L}(X, Y),\; \{T_{n}x\}\) for convergente para cada x em X, e definirmos
			      \begin{align*}
				      T: & X\rightarrow Y                              \\
				         & x\longmapsto Tx = \lim_{n\to \infty}T_{n}x,
			      \end{align*}
			      então \(T\in \mathcal{L}(X, Y)\) e \(\Vert T \Vert \leq \liminf \Vert T_{n} \Vert.\)
		\end{itemize}
	\end{theorem*}
}
\begin{proof*}
	Por conta do \hyperlink{baire_theorem}{\textit{Teorema de Baire}}, basta provarmos (a), e a mistura dos dois resultará nos itens (b) e (c). Para isso, seja
	\[
		E_{n} = \{x\in X: \sup_{T\in A}\Vert Tx \Vert\leq n\} = \bigcap_{T\in A}^{}\{x\in X:\; \Vert Tx \Vert \leq n\},
	\]
	tal que os \(E_{n}\)'s são fechados e, como sua união contém um conjunto de segunda categoria, devemos ter uma bola \(\overline{B_r}(x_{0}),\; r > 0\), contido em algum \(E_{n}\).
	Consequentemente, como \(\Vert x \Vert \leq r\), temos \(x_{0}-x\in \overline{B_r(x_{0})}\subseteq E_{n}\) e
	\[
		\Vert Tx \Vert = \Vert T(x-x_{0}) \Vert + \Vert Tx_{0} \Vert \leq n+n = 2n, \quad \forall T\in A,
	\]
	donde concluímos que \(\overline{B_r(0)}\subseteq E_{2n}\).

	Logo, \(\Vert Tx \Vert \leq 2n\) sempre que \(\Vert x \Vert \leq r\) e para todo T em A. Portanto,
	\[
		\Vert T \Vert \leq \frac{2n}{r}, \quad \forall T\in A. \text{ \qedsymbol}
	\]
\end{proof*}
\begin{exr}
	Tente utilizar o \hyperlink{baire_theorem}{\textit{Teorema de Baire}} para provar os itens restantes.
\end{exr}

\hypertarget{closed_graphic}{
	\begin{theorem*}[Gráfico Fechado]
		Se X e Y forem espaços de Banach e \(T:X\rightarrow Y\) for fechada, então T será limitada.
	\end{theorem*}
}
\begin{proof*}
	Sejam \(\pi_1\) e \(\pi_2\) as projeções de \(\mathrm{Graf}(T)\) em X e Y, \textit{i.e.},
	\[
		\pi_1(x, Tx) = x \quad\&\quad \pi_2(x, Tx) = Tx,
	\]
	sendo ambas lineares de \(\mathrm{Graf}(T)\) em seus respectivos espaços X e Y.

	Como X e Y são completos, \(X\times Y\) também o é, mostrando que \(\mathrm{Graf}(T)\) é completo, por ser um subconjunto fechado de um espaço completo; além disso, como \(\pi_1\) é uma bijeção de \(\mathrm{Graf}(T)\) em X, \(\pi_{1}^{-1}\) é limitado. Portanto,
	\[
		T = \pi_2 \circ \pi_{1}^{-1}
	\]
	é limitado. \qedsymbol
\end{proof*}

\end{document}

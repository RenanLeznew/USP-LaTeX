\documentclass[../functional_analysis.tex]{subfiles}
\begin{document}
\section{Aula 11 - 16 de Setembro, 2025}
\subsection{Motivações}
\begin{itemize}
	\item Teorema da Aplicação Aberta e Consequências;
	\item Princípio da Limitação Uniforme;
	\item Teorema do Gráfico Fechado
\end{itemize}
\subsection{Teoremas da Aplicação Aberta, da Limitação Uniforme e do Gráfico Fechado}
Na aula passada, enunciamos o \hyperlink{open_application}{\textit{Teorema da Aplicação Aberta}}; hoje, daremos sua prova e veremos alguns de seus resultados conseguintes e alguns outros grandes teoremas da área. Começamos por
\begin{lemma*}
	Sejam X e Y espaços vetoriais normados e \(T:X\rightarrow Y\) uma transformação linear; então, são equivalentes:
	\begin{itemize}
		\item[a)] T é uma aplicação aberta;
		\item[b)] Existe r positivo tal que \(T(B_{1}^{X}(0))\) contém \(B_{r}^{Y}(0)\).
	\end{itemize}
\end{lemma*}
\begin{proof*}
	A primeira parte segue simplesmente da definição de aberto num espaço métrico.

	Por outro lado, assuma que b ocorre e mostremos que, se \(U\) for um subconjunto aberto de X, então \(Tx\in (T(U))^{\circ }\) para todo x de U.

	Com efeito, dado x em U, tome \(s> 0\) tal que \(B_{s}^{X}(x)\subseteq U\); segue que
	\begin{align*}
		T(U)\supseteq T(B_{s}^{X}(x)) & = T(x + sB_{1}^{X}(0))       \\
		                              & = Tx + sT(B_{1}^{X}(0))      \\
		                              & \supseteq Tx + sB_{r}^{Y}(0) \\
		                              & = Tx + B_{sr}^{Y}(0)         \\
		                              & = B_{sr}^{Y}(Tx),
	\end{align*}
	provando portanto que \(Tx\) é interior a \(T(U)\). \qedsymbol
\end{proof*}

\begin{lemma*}
	Se X for Banach, Y for um espaço vetorial normado e \(T\in \mathcal{L}(X, Y)\) for tal que, para algum \(r> 0\),
	\[
		B_{r}^{Y}(0)\subseteq \overline{[T(B_{1}^{X}(0))]}.
	\]
	Então,
	\[
		B_{\frac{r}{2}}^{Y}(0)\subseteq T(B_{1}^{X}(0)).
	\]
\end{lemma*}
\begin{proof*}
	Como T é linear, se \(\Vert y \Vert < r2^{-n}\), então \(y\in \overline{T(B_{2^{-n}}^{X}(0))}.\)

	Agora, se \(\Vert y \Vert < r/2\), considere \(x_1\in B_{1/2}^{X}(0)\) tal que
	\[
		\Vert y-Tx_1 \Vert < \frac{r}{4};
	\]
	indutivamente, quando \(x_{n}\in B_{2^{-n}}^{X}(0)\), ele é tal que
	\[
		\biggl\Vert y- \sum\limits_{j=1}^{n}T_{j} \biggr\Vert < r2^{-n-1}.
	\]
	Como X é completo, a série \(\sum x_{n}\) é convergente; denote sua soma por x. Então,
	\[
		\Vert x \Vert < \sum\limits_{n=1}^{\infty}2^{-n}=1\;\&\; y = Tx.
	\]
	Portanto, se \(\Vert y \Vert < r/2\),
	\[
		y\in T(B_{1}^{X}(0)).\text{ \qedsymbol}
	\]

\end{proof*}
\end{document}

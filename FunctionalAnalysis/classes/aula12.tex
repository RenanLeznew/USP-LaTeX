\documentclass[../functional_analysis.tex]{subfiles}
\begin{document}
\section{Aula 12 - 16 de Setembro, 2025}
\subsection{Motivações}
\begin{itemize}
	\item Consequências do Teorema Categórico;
	\item Princípio da Limitação Uniforme;
	\item Operadores Duais.
\end{itemize}
\subsection{Consequências do Teorema Categórico}
\begin{theorem*}[Teorema da Aplicação Aberta]
	Seja X um espaço de Banach e Y um espaço vetorial normado. Se \(T\in \mathcal{L}(X, Y)\) e \(T(X)\) for de segunda categoria em Y, entãp:
	\begin{itemize}
		\item[a)] T será sobrejetora;
		\item[b)] T será aberta; e
		\item[c)] Y será de segunda categoria.
	\end{itemize}
\end{theorem*}
\begin{proof*}
	Como podemos escrever
	\[
		X = \bigcup_{n=1}^{\infty}B_{n}^{X}(0),
	\]
	o conjunto
	\[
		T(X) = \bigcup_{n=1}^{\infty}T(B_{n}^{X}(0))
	\]
	é de segunda categoria em y, donde segue que o mapa de Y em Y dado por
	\[
		y\mapsto ny
	\]
	é um homeomorfismo levando \(T(B_{1}^{X}(0))\) em \(T(B_{n}^{X}(0))\). Com isso, pelo \hyperlink{baire_theorem}{\textit{Teorema de Baire}}, o conjunto
	\(T(B_{1}^{X}(0))\) não pode ser raro, ou seja, deve existir um ponto \(y_{0}\in Y\) e \(r > 0\) tais que
	\[
		B_{2r}^{Y}(y_{0})\subseteq \overline{T(B_{1}^{X}(0))}.
	\]
	Logo,
	\[
		B_{2r}^{Y}(-y_{0})\subseteq \overline{T(B_{1}^{X}(0))}\;\&\; B_{2r}^{Y}(0)\subseteq \overline{T(B_{1}^{X}(0))}.
	\]
	Portanto, pelos lemas anteriores,
	\[
		B_{r}^{Y}(0)\subseteq T(B_{1}^{X}(0)),
	\]
	provando que T é aberta. \qedsymbol
\end{proof*}
\begin{crl*}
	Se X e Y são espaços de Banach e \(T\in \mathcal{L}(X, Y)\) é bijetora, então T é um isomorfismo, ou seja, \(T^{-1}\in \mathcal{L}(Y, X)\).
\end{crl*}
\hypertarget{uniform_limitation}{
	\begin{theorem*}[Princípio da Limitação Uniforme]
		Sejam X e Y espaços vetoriais normados e A um subconjunto de \(\mathcal{L}(X, Y)\).
		\begin{itemize}
			\item[a)] Se \(\{x\in X:\; \sup_{}\{\Vert Tx \Vert:\; T\in A\} < \infty\}\) for de segunda categoria, então
			      \[
				      \sup_{}\{\Vert T \Vert:\; T\in A\} < \infty.
			      \]
			\item[b)] Se X for Banach e \(\{x\in X:\; \sup_{}\{\Vert Tx \Vert:\; T\in A\} < \infty\} = X\), então
			      \[
				      \sup_{}\{\Vert T \Vert:\; T\in A\} < \infty;
			      \]
			\item[c)] Se X for Banach, \(\{T_{n}:\; n\in \mathbb{N}\} \subseteq \mathcal{L}(X, Y),\; \{T_{n}x\}\) for convergente para cada x em X, e definirmos
			      \begin{align*}
				      T: & X\rightarrow Y                              \\
				         & x\longmapsto Tx = \lim_{n\to \infty}T_{n}x,
			      \end{align*}
			      então \(T\in \mathcal{L}(X, Y)\) e \(\Vert T \Vert \leq \liminf \Vert T_{n} \Vert.\)
		\end{itemize}
	\end{theorem*}
}
\begin{proof*}
	Por conta do \hyperlink{baire_theorem}{\textit{Teorema de Baire}}, basta provarmos (a), e a mistura dos dois resultará nos itens (b) e (c). Para isso, seja
	\[
		E_{n} = \{x\in X: \sup_{T\in A}\Vert Tx \Vert\leq n\} = \bigcap_{T\in A}^{}\{x\in X:\; \Vert Tx \Vert \leq n\},
	\]
	tal que os \(E_{n}\)'s são fechados e, como sua união contém um conjunto de segunda categoria, devemos ter uma bola \(\overline{B_r}(x_{0}),\; r > 0\), contido em algum \(E_{n}\).
	Consequentemente, como \(\Vert x \Vert \leq r\), temos \(x_{0}-x\in \overline{B_r(x_{0})}\subseteq E_{n}\) e
	\[
		\Vert Tx \Vert = \Vert T(x-x_{0}) \Vert + \Vert Tx_{0} \Vert \leq n+n = 2n, \quad \forall T\in A,
	\]
	donde concluímos que \(\overline{B_r(0)}\subseteq E_{2n}\).

	Logo, \(\Vert Tx \Vert \leq 2n\) sempre que \(\Vert x \Vert \leq r\) e para todo T em A, donde temos
	\[
		\Vert T \Vert \leq \frac{2n}{r}, \quad \forall T\in A.
	\]
	Agora, se \(T_{n}\) converge para \(T\), qualquer subsequência de \(\{T_{n}\}\) também irá convergir; assim, existe uma subsequência \(\{T_{n_k}\}\) para a qual
	\[
		\lim_{k\to \infty}\Vert T_{n_k}\Vert = \liminf_{n\to \infty}\Vert T_{n}\Vert,
	\]
	mas
	\[
		\lim_{k\to \infty}T_{n_k} = Tx.
	\]
	Portanto,
	\[
		\Vert T \Vert \leq \limsup_{n\to \infty}\Vert T_{n_k} \Vert = \liminf_{n\to \infty} \Vert T_{n} \Vert. \text{ \qedsymbol}
	\]
\end{proof*}
\begin{exr}
	Tente utilizar o \hyperlink{baire_theorem}{\textit{Teorema de Baire}} para provar os itens restantes.
\end{exr}
\begin{crl*}
	Se X é um espaço de Banach, B é um subconjunto de X e \(f(B)=\{f(b):\; b\in B\}\) é limitado para toda f no dual de X, então B é limitado.
\end{crl*}
\begin{proof*}
	Para cada b em B, defina \(\hat{b}:X^{*}\rightarrow \mathbb{K}\) por
	\[
		\hat{b}(f)=f(b),\quad \forall f\in X^{*}.
	\]
	Então, para cada f no dual de X,
	\[
		\sup_{b\in B}| \hat{b}(f) | = \sup_{b\in B}| f(b) | < \infty
	\]
	segue do \hyperlink{uniform_limitation}{\textit{Princípio da Limitação Uniforme}} que
	\[
		\sup_{b\in B} \Vert \hat{b} \Vert = \sup_{b\in B}\Vert b \Vert < \infty. \text{ \qedsymbol}
	\]
\end{proof*}
\begin{crl*}
	Seja X um espaço de Banach e \(B^{*}\subseteq X^{*}\); suponhamos que, para todo x de X, o conjunto
	\[
		B^{*}(X)=\{b^{*}(x):\; b^{*}\in B^{*}\}
	\]
	é limitado. Nestas condições, segue que \(B^{*}\) é limitado.
\end{crl*}
\begin{proof*}
	Por hipótese, para cada x em X,
	\[
		\sup_{b\in B^{*}} | b^{*}(x) | \leq c_{x}.
	\]
	Segue do \hyperlink{uniform_limitation}{\textit{Princípio da Limitação Uniforme}} que
	\[
		\sup_{b^{*}\in B^{*}}\Vert b^{*} \Vert<\infty. \text{ \qedsymbol}
	\]
\end{proof*}

\hypertarget{closed_graphic}{
	\begin{theorem*}[Gráfico Fechado]
		Se X e Y forem espaços de Banach e \(T:X\rightarrow Y\) for fechada, então T será limitada.
	\end{theorem*}
}
\begin{proof*}
	Sejam \(\pi_1\) e \(\pi_2\) as projeções de \(\mathrm{Graf}(T)\) em X e Y, \textit{i.e.},
	\[
		\pi_1(x, Tx) = x \quad\&\quad \pi_2(x, Tx) = Tx,
	\]
	sendo ambas lineares de \(\mathrm{Graf}(T)\) em seus respectivos espaços X e Y.

	Como X e Y são completos, \(X\times Y\) também o é, mostrando que \(\mathrm{Graf}(T)\) é completo, por ser um subconjunto fechado de um espaço completo; além disso, como \(\pi_1\) é uma bijeção de \(\mathrm{Graf}(T)\) em X, \(\pi_{1}^{-1}\) é limitado. Portanto,
	\[
		T = \pi_2 \circ \pi_{1}^{-1}
	\]
	é limitado. \qedsymbol
\end{proof*}
\end{document}

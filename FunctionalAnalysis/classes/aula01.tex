 \documentclass[../functional_analysis.tex]{subfiles}
\begin{document}
\section{Aula 01 - 05 de Agosto, 2025}
\subsection{Motivações}
\begin{itemize}
	\item Introdução à disciplina;
	\item Desigualdades Importantes.
\end{itemize}
\subsection{Espaços Vetoriais Normados}
\begin{def*}
	Um espaço vetorial é um conjunto não vazio V sobre um corpo \(\mathbb{K}\) com duas operações + e \(\cdot \), sendo a primeira um grupo abeliano com elemento neutro 0, a multiplicação é associativa, com elemento neutro 1 e ambas satisfazem a distribuição. \(\square\)
\end{def*}
\begin{def*}
	Seja V um espaço vetorial sobre \(\mathbb{K}\). Uma \textbf{norma} em V é uma função \(\Vert \cdot  \Vert_{V}:V\rightarrow \mathbb{R}^{+}\) que satisfaz:
	\begin{align*}
		 & \Vert v \Vert_{V} = 0 \Longleftrightarrow v = 0                                         \\
		 & \Vert \lambda v \Vert_{V} = |\lambda | \Vert v \Vert_{V}                                \\
		 & \Vert v+w \Vert_{V} \leq \Vert v \Vert_{V} + \Vert w \Vert_{V},\quad \forall v, w\in V.
	\end{align*}
	Um espaço vetorial com uma norma é chamado \textbf{espaço vetorial normado}, denotado por \((V, \Vert \cdot  \Vert_{V}).\; \square\)
\end{def*}
\begin{def*}
	Seja X um conjunto não vazio. Uma \textbf{métrica} em X é uma função \(\rho :X \times X\rightarrow [0, \infty)\) tal que
	\begin{align*}
		 & \rho (x, y) = 0,\quad \forall x, y\in X                               \\
		 & \rho (x, y) = \rho (y, x),\quad \forall x, y\in X                     \\
		 & \rho (x,z) \leq \rho (x, y) + \rho (y, z),\quad \forall x, y, z\in X,
	\end{align*}
	onde \(\rho \) é chamada distância ou métrica. \(\square\)
\end{def*}
Segue que, se \((V, \Vert \cdot  \Vert_{V})\) é normado e \(\rho_{V}:V\times V\rightarrow \mathbb{R}^{+}\) definida por
\[
	\rho_{V}(v, w) = \Vert v-w \Vert,
\]
então \((V, \rho_{V})\) é métrico. Neste caso, diremos que a métrica é proveniente da norma e todo espaço vetorial normado é um espaço métrico com esta norma.
\begin{exr}
	Prova a afirmação de que \((V, \rho_V)\) é métrico.
\end{exr}
No curso de métricos, provamos que todo espaço métrico é isometricamente isomorfo a um subconjunto de um espaço vetorial normado.

Começaremos introduzindo o assunto com o seguinte exemplo: para \(p \in [1, \infty]\), seja \((\mathbb{R}^{N}, \Vert \cdot  \Vert_{p})\) com \(\Vert \cdot  \Vert_{p}:\mathbb{R}^{N}\rightarrow \mathbb{R}^{+}\) dada por
\[
	\Vert \xi  \Vert_{p} = \biggl(\sum\limits_{i=1}^{N}|\xi |^{p}\biggr)^{\frac{1}{p}},\quad \xi  = (\xi_1, \dotsc , \xi_n),\; 1\leq p < \infty
\]
e
\[
	\Vert \xi  \Vert_{\infty} = \max\limits_{1\leq i\leq N}\{\xi_{i}\}.
\]

Vamos fazer apenas os casos \(1 < p < \infty\). Por definição da norma,
\[
	\Vert x \Vert_{p} = 0 \Longleftrightarrow x =0
\]
e que
\[
	\Vert \lambda x \Vert_p = \biggl(\sum\limits_{i=1}^{N}|\lambda \xi_{i}|_p\biggr)^{\frac{1}{p}} = |\lambda |\biggl(\sum\limits_{i=1}^{N}|\xi_{i}|^{p}\biggr)^{\frac{1}{p}} = |\lambda | \Vert x \Vert_p, \quad \forall x\in \mathbb{R}^{N},\;\&\; \lambda \in \mathbb{R}.
\]
O trabalho todo está em mostrar a desigualdade triangular.

Com efeito, precisaremos da chamada \hyperlink{young_inequality}{\textit{desigualdade de Young}}
\hypertarget{young_inequality}{
	\begin{lemma*}[Desigualdade de Young]
		Se \(p, q\in (1, \infty)\) são conjugaods, i.e., \(\frac{1}{p} + \frac{1}{q} = 1\), e \(a, b\in [0, \infty]\), então
		\[
			a^{\frac{1}{p}}b^{\frac{1}{q}} \leq \frac{a}{p} + \frac{b}{q}.
		\]
	\end{lemma*}
}
\begin{proof*}
	A ideia da prova da desigualdade de Young é transformar isto num problema de cálculo 1. Para isso, fazemos
	\[
		\frac{a^{\frac{1}{p}}b^{\frac{1}{q}}}{b^{\frac{1}{p}+\frac{1}{q}}} \leq \biggl(\frac{a}{b}\biggr)\frac{1}{p} + \frac{1}{q},
	\]
	ou seja,
	\[
		\biggl(\frac{a}{b}\biggr)^{\frac{1}{p}}\leq \biggl(\frac{a}{b}\biggr)\frac{1}{p} + \frac{1}{q},
	\]
	e chamaremos a primeira expressão de t:
	\[
		t^{\frac{1}{p}} \leq t \frac{1}{p} + \frac{1}{q},
	\]
	então é isso que precisamos checar se é verdade, mostrando que a função
	\begin{align*}
		f: & \mathbb{R}^{+}\rightarrow \mathbb{R}          \\
		   & t \mapsto \alpha t - t^{\alpha } + 1 - \alpha
	\end{align*}
	satisfaz
	\[
		f(t)\geq 0, \forall t\in \mathbb{R}^{+}.
	\]
	Derivando esta função, obtemos
	\[
		f'(t) = \alpha (1-t^{\alpha -1})  \left\{\begin{array}{ll}
			<0,  & \quad t\in [0, 1)      \\
			= 0, & \quad t=1              \\
			>0,  & \quad t\in (1, \infty)
		\end{array}\right.,
	\]
	mostrando que \(t=1\) é um ponto de mínimo global para f e \(f(1) = 0\). Portanto, \(f(t)\geq 0\), concluindo a prova. \qedsymbol
\end{proof*}
\hypertarget{holder_inequality}{
	\begin{lemma*}[Desigualdade de Hölder]
		Se \(p, q\in (1, \infty)\) são conjugados, então
		\[
			\sum\limits_{i=1}^{N}|x_{i}y_{i}| \leq \biggl[\sum\limits_{i=1}^{N}|x_{i}|^{p}\biggr]^{\frac{1}{p}}\biggl[\sum\limits_{i=1}^{N}|y_{i}|^{q}\biggr]^{\frac{1}{q}},\; \forall x=(x_1, \dotsc , x_{N}),\; y = (y_1, \dotsc , y_N)\in \mathbb{R}^{N}.
		\]
	\end{lemma*}
}
\begin{proof*}
	Se \(x=0\) ou \(y=0\), é automático, então suponha ambos não nulos. Fazendo
	\[
		a_{j} = \frac{|x_{j}|^{p}}{\sum\limits_{i=1}^{N}|x_{i}|^{p}} \quad\&\quad b_{j} = \frac{|y_{j}|^{q}}{\sum\limits_{i=1}^{N}|y_{i}|^{q}}.
	\]
	Aplicando a \hyperlink{young_inequality}{\textit{Desigualdade de Young}}, obtemos
	\[
		a_{j}^{\frac{1}{p}}b_{j}^{\frac{1}{q}} = \frac{|x_{j}||y_{j}|}{\biggl[\sum\limits_{i=1}^{N}|x_{i}|^{p}\biggr]^{\frac{1}{p}}\biggl[\sum\limits_{i=1}^{N}|y_{i}|^{q}\biggr]^{\frac{1}{q}}}\leq \frac{1}{p}a_{j} + \frac{1}{q}b_{j}, \quad 1\leq j\leq N.
	\]
	Note que a expressão em fração é o mesmo que
	\[
		\frac{|x_{j}y_{j}|}{\Vert x \Vert_p \Vert y \Vert_q} \leq \frac{1}{p}\frac{|x_{j}|^{p}}{\Vert x \Vert_{p}^{p}} + \frac{1}{q} \frac{|y_{j}|^{q}}{\Vert y \Vert_{q}^{q}}
	\]
	e, somando em j,
	\begin{align*}
		 & \sum\limits_{j=1}^{N}\frac{|x_{j}y_{j}|}{\Vert x \Vert_p \Vert y \Vert_q} \leq \sum\limits_{j=1}^{N} \frac{1}{p}\frac{|x_{j}|^{p}}{\Vert x \Vert_{p}^{p}} + \frac{1}{q} \frac{|y_{j}|^{q}}{\Vert y \Vert_{q}^{q}}                             \\
		 & \Longleftrightarrow \frac{1}{\Vert x \Vert_p \Vert y \Vert_q}\sum\limits_{j=1}^{N}|x_{j}y_{j}| \leq \frac{1}{p}\sum\limits_{j=1}^{N} \frac{|x_{j}|}{\Vert x \Vert_p} + \frac{1}{q} \sum\limits_{j=1}^{N} \frac{|y_{j}|}{\Vert y \Vert_q}      \\
		 & \Longleftrightarrow \frac{1}{\Vert x \Vert_p \Vert y \Vert_q}\sum\limits_{j=1}^{N}|x_{j}y_{j}| \leq \frac{1}{p}\frac{\Vert x \Vert_p}{\Vert x \Vert_p} + \frac{1}{q} \frac{\Vert y \Vert_q}{\Vert y \Vert_q} = \frac{1}{p} + \frac{1}{q} = 1.
	\end{align*}
	Portanto,
	\[
		\sum\limits_{j=1}^{N}|x_{j}y_{j}|\leq \Vert x \Vert_p \Vert y \Vert_q. \quad \text{\qedsymbol}
	\]
\end{proof*}
Note que, para o caso \(p=q=2\), isto é a Desigualdade de Cauchy-Schwarz.
\hypertarget{minkowski_inequality}{
	\begin{lemma*}[Desigualdade de Minkowski]
		Se \(p\in [1, \infty]\), então
		\[
			\Vert x + y \Vert_p \leq \Vert x \Vert_p + \Vert y \Vert_p.
		\]
	\end{lemma*}
}
\begin{proof*}
	Os casos onde \(p=1, \infty\) são exercícios. Se \(p\in (1, \infty)\), temos
	\[
		\biggl[\sum\limits_{i=1}^{N}|x_{i}+y_{i}|^{p}\biggr]^{\frac{1}{p}} \leq \biggl[\sum\limits_{i=1}^{N}(|x_{i}|+|y_{i}|)^{p}\biggr]^{\frac{1}{p}}.
	\]
	Agora,
	\[
		(|x_{i}|+|y_{i}|)^{p} = (|x_{i}|+|y_{i}|)^{p-1}|x_{i}| + (|x_{i}|+|y_{i}|)^{p-1}|y_{i}|
	\]
	e, somando as igualdades acima,
	\[
		\sum\limits_{i=1}^{N}(|x_{i}|+|y_{i}|)^{p} = \sum\limits_{i=1}^{N}(|x_{i}|+|y_{i}|)^{p-1}|x_{i}| + \sum\limits_{i=1}^{N}(|x_{i}|+|y_{i}|)^{p-1}|y_{i}|.
	\]
	Utilizando a \hyperlink{holder_inequality}{\textit{Desigualdade de Hölder}} em cada um dos somatórios acima e notando que \((p-1)q = p\),
	\begin{align*}
		\sum\limits_{i=1}^{N}(|x_{i}|+|y_{i}|)^{p-1}|x_{i}| & \leq \biggl[\sum\limits_{i=1}^{N}|x_{i}|^{p}\biggr]^{\frac{1}{p}}\biggl[\sum\limits_{i=1}^{N}(|x_{i}|+|y_{i}|)^{(p-1)q}\biggr]^{\frac{1}{q}} \\
		                                                    & = \biggl[\sum\limits_{i=1}^{N}|x_{i}|^{p}\biggr]^{\frac{1}{p}}\biggl[\sum\limits_{i=1}^{N}(|x_{i}|+|y_{i}|)^{p}\biggr]^{\frac{1}{q}},
	\end{align*}
	e, analogamente,
	\[
		\sum\limits_{i=1}^{N}(|x_{i}|+|y_{i}|)^{p-1}|y_{i}| \leq \biggl[\sum\limits_{i=1}^{N}|y_{i}|^{p}\biggr]^{\frac{1}{p}}\biggl[\sum\limits_{i=1}^{N}(|x_{i}|+|y_{i}|)^{p}\biggr]^{\frac{1}{q}}.
	\]
	Consequentemente,
	\begin{align*}
		\sum\limits_{i=1}^{N}(|x_{i}|+|y_{i}|)^{p} & = \sum\limits_{i=1}^{N}(|x_{i}|+|y_{i}|)^{p-1}|x_{i}| + \sum\limits_{i=1}^{N}(|x_{i}|+|y_{i}|)^{p-1}|y_{i}|                                                                                                          \\
		                                           & \leq \biggl[\sum\limits_{i=1}^{N}(|x_{i}|+|y_{i}|)^{p}\biggr]^{\frac{1}{q}}\biggl(\biggl[\sum\limits_{i=1}^{N}|x_{i}|^{p}\biggr]^{\frac{1}{p}} + \biggl[\sum\limits_{i=1}^{N}|y_{i}|^{p}\biggr]^{\frac{1}{p}}\biggr) \\
		                                           & = (\Vert x \Vert_p + \Vert y \Vert_p)\biggl[\sum\limits_{i=1}^{N}(|x_{i}|+|y_{i}|)^{p}\biggr]^{\frac{1}{q}}.
	\end{align*}
	Portanto,
	\[
		\biggl[\sum\limits_{i=1}^{N}(|x_{i}|+|y_{i}|)^{p}\biggr]^{\frac{1}{p}} \leq \biggl[\sum\limits_{i=1}^{N}(|x_{i}|+|y_{i}|)^{p}\biggr]^{\frac{1}{p} + \frac{1}{q}} \leq (\Vert x \Vert_p + \Vert y \Vert_p)\biggl[\sum\limits_{i=1}^{N}(|x_{i}|+|y_{i}|)^{p}\biggr]^{\frac{1}{q}}. \text{ \qedsymbol}
	\]
\end{proof*}
Com a desigualdade de Minkowski, provamos a triangular e, portanto, que a norma-p é de fato uma norma.
\begin{exr}
	Mostre os casos \(p=1\) e \( p=\infty\) na demonstração da \hyperlink{minkowski_inequality}{\textit{desigualdade de Minkowski}}.
\end{exr}

\begin{example}
	O espaço \(\mathcal{C}[a, b] = \{f:[a, b]\rightarrow \mathbb{R}:\; f \text{ é contínua}\}\) é um espaço vetorial sobre \(\mathbb{R}\), sendo cada ponto nele uma função contínua definida em \([a, b]\) a valores reais. As operações de adição e multiplicação por escalar são definidas ponto a ponto: para todas funções \(f, g\) em \(\mathcal{C}[a, b]\) e \(\alpha \) real,
	\begin{align*}
		 & (f+g)(t) = f(t) + g(t)       \\
		 & (\alpha f)(t) = \alpha f(t).
	\end{align*}
	Uma das normas que podemos pôr é
	\[
		\Vert f \Vert_{\mathcal{C}[a, b]} = \max\limits_{x\in [a, b]}\{|f(x)|\},
	\]
	ou, outra ainda, seria
	\[
		\Vert f \Vert_{L^{1}(a, b)} = \int_{a}^{b}|f(t)| \mathrm{dt}.
	\]
	O que determina o uso de uma sobre a outra, neste caso, é a \textit{completude} -- a norma do sup torna este espaço num completo, mas a norma \(L^{1}(a, b)\) não! Basta tomar
	\[
		\int_{0}^{1}x^{n} \mathrm{dx}= \frac{1}{n+1}x^{n+1}\biggl|_{0}^{1}\biggr. = \frac{1}{n+1},
	\]
	mas acrescentando um degrau nela:
	\[
		f(x) = \left\{\begin{array}{ll}
			x^{n}, & x\in [0, 1] \\
			1,     & x\in (1, 2)
		\end{array}\right.,
	\]
	formando uma sequência de Cauchy que não converge na norma \(L^{1}(a, b)\).

	Voltando às possíveis normas, podemos generalizar a segunda para outra norma nesse espaço, a norma \(L^{p}(a, b)\) dada como
	\[
		\Vert f \Vert_{L^{p}(a, b)} = \biggl(\int_{a}^{b}|f(t)|^{p} \mathrm{dt}\biggr)^{\frac{1}{p}}
	\]
\end{example}

\end{document}

\documentclass[../functional_analysis.tex]{subfiles}
\begin{document}
\section{Aula 05 - 19 de Agosto, 2025}
\subsection{Motivações}
\begin{itemize}
	\item Construindo Funcionais e Espaços;
	\item Complexificação;
\end{itemize}
\subsection{Construindo Funcionais e Espaços}

Vamos começar esta aula retomando o tema de hiperplanos.


\begin{def*}
	Um \textbf{hiperplano afim} é um conjunto da forma
	\[
		H=\{x\in X:\; f(x)=\alpha \},
	\]
	onde \(f:X\rightarrow \mathbb{R}\) é um funcional linear não identicamente nulo e \(\alpha \in \mathbb{R}.\) Diremos que H é o hiperplano da equação \([f=\alpha ].\; \square\)
\end{def*}

\begin{prop*}
	O hiperplano da equação \([f=\alpha ]\) é fechado se, e somente se, f é contínua.
\end{prop*}

Note que, se X é um espaço vetorial e \(f:X\rightarrow \mathbb{K}\) é um funcional linear não nulo, existe \(x_f\in X\) tal que \(f(x_f) = 1\) -- basta notar que, como f é justamente diferente de 0,
\[
	x_f = \frac{x}{f(x)} \Rightarrow f(x_f) = f \biggl(\frac{x}{f(x)}\biggr) = 1.
\]
Em particular, levando em conta o hiperplano
\[
	[x_f] = \{sx_f:\; s\in \mathbb{R}\},
\]
temos a expressão \(X = N(f)\oplus [x_f]\), onde \(N(f)\) é o núcleo de f; de fato, podemos escrever
\[
	x=x-\underbrace{f(x)\cdot x_f}_{\mathclap{z}} + \underbrace{f(x)\cdot x_f}_{\mathclap{w}},
\]
pois f é um funcional linear. Porém, seria esta a \textit{única} representação de x nestes termos? Vamos tomar outra combinação para x, digamos
\[
	x = z+w \quad\&\quad x = z'+w',
\]
de forma que
\[
	z-z' = w'-w = \alpha x_f.
\]
Logo,
\[
	0 = \alpha f(x_f) = \alpha ,
\]
mostrando que as duas representações são as mesmas. Mais ainda, pela forma como o hiperplano gerado por \(x_f\) se expressa, temos
\[
	f(sx_f) = sf(x_f) = s \Rightarrow f(\alpha x_f) = \alpha \quad\&\quad f(y+\alpha x_\alpha ) = \alpha \Longleftrightarrow y = 0, \text{ i.e. }, y\in N(f).
\]
Por isso chamamos de hiperplano -- ele divide o espaço em dois pedaços pelo espaço afim \(\alpha x_\alpha \oplus N(f).\) Isso nos dá uma ferramenta para mostrar que todo espaço finito-dimensional possui um \textit{complemento algébrico}, um subespaço cuja soma direta com ele dá o espaço inteiro... Mas como fazemos para achá-lo?

Dados um espaço vetorial X sobre \(\mathbb{K}\), n natural não nulo, F um subespaço vetorial de X com dimensão finita e um conjunto \(\{x_1, \dotsc , x_{n}\}\) linearmente independente de vetores de X que gere F, sejam \(f_{i}:X\rightarrow \mathbb{K}\) funcionais tais que \(f_{i}(x_{i}) = 1,\; 1\leq i\leq n\). Além disso, tome um vetor x em X e escreva-o como
\[
	x = \underbrace{x - \sum\limits_{i=1}^{n} f(x)x_{i}}_{\in G} + \underbrace{\sum\limits_{i=1}^{n}f_{i}(x)x_{i}}_{\in F},
\]
onde o termo mais à direita determina exatamente um produto de escalares por elementos que geram F e, a partir disso, temos \(G = \bigcap_{i=1}^{n}N(f_{i})\), pois
\[
	f_{i} \bigl( x-\sum\limits_{i=1}^{n}f_{i}(x)x_{i}\bigr) = f_{j}(x) - \sum\limits_{i=1}^{n}f_{i}(x)f_{j}(x_{i}) = f_{j}(x) - f_{j}(x) = 0,
\]
tal que \(G=\bigcap_{i=1}^{n}N(f_{i})\) e \(X = G+F\). Afirmamos que esta soma é direta; com efeito, dado x em X,
\begin{align*}
	                    & x = g+f = g'+f'                                                           \\
	\Longleftrightarrow & g-g' = f'-f = \sum\limits_{i=1}^{n}\alpha_{i}x_{i}                        \\
	\Longleftrightarrow & 0=f_{j}(g-g') = \sum\limits_{i=1}^{n}\alpha_{i}f_{j}(x_{i}) = \alpha_{j}.
\end{align*}
Para podermos tomar estes funcionais todos como contínuos, tornando G fechado, lembre-se que \(F = [x_1, \dotsc , x_{n}]\) e associemos a cada vetor da base
\[
	f_1(x_1) = \delta _1 \Rightarrow \frac{f_1(x_1)}{\delta_1} = 1 \quad\&\quad \frac{f_1(x_2)}{\delta_1}=\dotsc = \frac{f_{1}(x_{n})}{\delta_1} = 0;
\]
então, achamos um funcional limitado e contínuo, mas perdemos a norma unitária. Este será nosso novo \(f_1\). Continuando a construção, temos uma n-upla
\begin{align*}
	            & f_1(x_1) = 1,\quad f_1(x_2) = \dotsc = f_1(x_{n}) = 0      \\
	            & f_2(x_2) = 1,\quad f_2(x_1) = \dotsc = f_2(x_{n}) = 0      \\
	            & \vdots                                                     \\
	            & f_n(x_n) = 1,\quad f_n(x_1) = \dotsc = f_n(x_{n-1}) = 0    \\
	\Rightarrow & \bigcap_{i=1}^{n}N(f_{i})\oplus [x_1, \dotsc , x_{n}] = X,
\end{align*}
tal que exibimos quais eram os tais funcionais que geram o espaço por soma direta. Portanto, num espaço vetorial qualquer, todo subespaço de dimensão finita tem complementar fechado que, em soma direta com ele, completa o espaço todo, permitindo que construamos uma projeção contínua sobre este espaço fechado.
\begin{def*}
	Seja X um espaço vetorial sobre \(\mathbb{K}. \) Diremos que um subconjunto C de X é \textbf{convexo} se
	\[
		tx+(1-t)y\in C
	\]
	sempre que \(t\in [0,1]\) e \(x,y\in C.\; \square\)
\end{def*}

\begin{def*}
	Se A, B são subconjuntos de X dizemos que o hiperplano de equação \([f=\alpha ]\) \textbf{separa A e B no sentido fraco} se
	\begin{align*}
		 & f(x)\leq \alpha ,\quad \forall x\in A  \\
		 & f(x)\geq \alpha ,\quad \forall x\in B.
	\end{align*}
	Analogamente, diremos que o hiperplano de equação \([f=\alpha ]\) \textbf{separa A e B no sentido forte} se existe \(\varepsilon >0\) tal que
	\begin{align*}
		 & f(x)\leq \alpha -\varepsilon ,\quad \forall x\in A               \\
		 & f(x)\geq \alpha+\varepsilon  ,\quad \forall x\in B.\quad \square
	\end{align*}
\end{def*}

\hypertarget{first_geometric_hahn_banach}{\begin{theorem*}[Hahn-Banach Geométrico: primeira forma]
		Seja X um espaço vetorial normado real e sejam A, B dois subconjuntos convexos, não vazios e disjuntos de X. Se A é aberto, existe um hiperplano fechado que separa A e B no sentido fraco.
	\end{theorem*}}
Antes de prová-lo, precisaremos dos seguintes lemas:
\hypertarget{minkowski_functional}{
	\begin{lemma*}[Funcional de Minkowski de um Convexo]
		Seja X um espaço vetorial normado sobre \(\mathbb{R}\) e C um aberto convexo de X contendo a origem. Para todo x em X, defina o \textbf{funcional de Minkowski} por
		\[
			p(x)\coloneqq \inf\{\alpha >0:\; \alpha^{-1}x\in C\}.
		\]
		Então, p é um funcional sub-linear e existe um número positivo M tal que
		\begin{align*}
			 & 0\leq p(x)\leq M\Vert x \Vert,\quad \forall x\in X \\
			 & C=\{x\in X:\; p(x)<1\}.
		\end{align*}
	\end{lemma*}
}
\begin{proof*}
	Seja r um número positivo tal que
	\[
		\overline{B}_r(0)\subseteq C
	\]
	e note que, para todo x em X,
	\[
		r \frac{x}{\Vert x \Vert}\in \overline{B}_r(0)\subseteq C,
	\]
	o que implica em
	\[
		p(x)\leq \frac{1}{r}\Vert x \Vert,
	\]
	fazendo a primeira parte das propriedades de p ser verdadeira a partir do momento que colocamos \(M=1/r\). Além disso, para a segunda propriedade, se x for um elemento de C qualquer, existe \(\varepsilon >0\), tal que \((1+\varepsilon )x\in C\); assim,
	\[
		p(x)\leq \frac{1}{1+\varepsilon }<1.
	\]

	Reciprocamente, se \(p(x)<1\), existe \(\alpha\in (0, 1) \) tal que \(\alpha^{-1}x\in C\) e, consequentemente,
	\[
		x=\alpha (\alpha ^{-1}x)+(1-\alpha )0\in C.
	\]

	Vamos verificar, para finalizar, que p é um funcional sub-linear. Primeiramente, observe que
	\[
		p(\lambda x)=\lambda p(x),\quad \lambda >0.
	\]
	Ademais, sejam x e y elementos de X e \(\varepsilon >0\); então, para todo t em \([0,1]\)
	\[
		\frac{x}{p(x)+\varepsilon }\in C \;\&\; \frac{y}{p(y)+\varepsilon }\in C \Rightarrow \frac{tx}{p(x)+\varepsilon } + \frac{(1-t)y}{p(y)+\varepsilon }\in C.
	\]
	Em particular, pondo
	\[
		t=\frac{p(x)+\varepsilon }{p(x)+p(y)+2\varepsilon },
	\]
	temos
	\[
		\frac{x+y}{p(x)+p(y)+2\varepsilon }\in C.
	\]
	Portanto, para todo \(\varepsilon \) positivo,
	\[
		p(x+y)\leq p(x)+p(y)+2\varepsilon  \Rightarrow p(x+y)\leq p(x)+p(y).\text{ \qedsymbol}
	\]
\end{proof*}

\begin{lemma*}
	Seja C um subconjunto aberto, convexo e não vazio de X tal que \(x_{0}\) \textit{\textbf{não}} pertence a C. Então, existe um funcional f em \(X^{*}\) tal que \(f(x)<f(x_{0})\) para todo x em C. Em particular, o hiperplano fechado da equação \([f=f(x_{0})]\) separa C de \(x_{0}\) no sentido fraco.
\end{lemma*}
\begin{proof*}
	Por translação, sempre podemos supor que C contém a origem; dito isso, sejam p o \hyperlink{minkowski_functional}{\textit{funcional de Minkowski}} de C, G o conjunto \(\mathbb{R}x_{0}\) e \(g:G\rightarrow \mathbb{R}\) dada por
	\[
		g(tx_{0})=t,\quad t\in \mathbb{R}.
	\]
	Com isso,
	\[
		g(x)=g(tx_{0}) = \left\{\begin{array}{ll}
			t\leq t p(x_{0})=p(tx_{0}), & \quad t>0     \\
			t\leq 0\leq p(tx_{0}),      & \quad t\leq 0
		\end{array}\right..
	\]
	Logo, pelo \hyperlink{hahn_banach}{\textit{Teorema de Hahn-Banach real}}, existe \(f:X\rightarrow \mathbb{R}\) tal que \(f|_{G}=g\) e, para todo x em X,
	\[
		f(x)\leq p(x).
	\]
	Em particular, \(f(x_{0})=1\) e f é contínua por conta de \(p(x)\leq M\Vert x \Vert.\) Além disso, \(f(x)<1\) para todo x em C, provando o resultado. \qedsymbol
\end{proof*}
Finalmente, podemos ir para a
\begin{proof*}[\hyperlink{first_geometric_hahn_banach}{\textit{Teorema de Hahn-Banach: primeira forma geométrica}}]
	Seja \(C\coloneqq A-B\), o qual é convexo, aberto e não contém a origem, já que \(A\cap B = \emptyset \). Pelo lema acima, existe um funcional \(f\in X^{*}\) tal que
	\[
		f(z)<0,\quad \forall z\in C,
	\]
	donde segue que
	\[
		f(a)<f(b),\quad \forall a\in A,\; b\in B.
	\]
	Portanto, escolhendo \(\alpha \) tal que
	\[
		\sup_{a\in A}f(x)\leq \alpha \leq \inf_{b\in B}f(b),
	\]
	segue que o hiperplano de equação \([f=\alpha ]\) separa A e B no sentido fraco. \qedsymbol
\end{proof*}
\hypertarget{second_geometric_hahn_banach}{
	\begin{theorem*}[Segunda Forma Geométrica do Hahn-Banach]
		Seja X um espaço vetorial normado real, A e B convexos, não vazio e disjuntos em X. Suponha que A é fechado e B é compacto; então, existe um hiperplano fechado que separa A e B no sentido forte.
	\end{theorem*}
}
\begin{proof*}
	Dado \(\varepsilon > 0\), definamos os conjuntos
	\begin{align*}
		 & A_\varepsilon = A + B_\varepsilon(0)   \\
		 & B_\varepsilon = B + B_\varepsilon (0).
	\end{align*}
	Desta forma, \(A_\varepsilon \) e \(B_\varepsilon \) são abertos, convexos e não vazios; em particular, para \(\varepsilon > 0\) pequeno, \(A_\varepsilon \) e \(B_\varepsilon \) são disjuntos, tendo em vista que, por A ser fechado,
	\[
		d(b, A) > 0, \quad \forall b\in B
	\]
	e, como B é compacto,
	\[
		\inf_{b\in B}d(b, A) = d(B, A) > 0,
	\]
	confirmando a propriedade de serem disjuntos.

	Pela \hyperlink{first_geometric_hahn_banach}{\textit{Primeira Versão,}} existe um hiperplano fechado que separa \(A_\varepsilon \) e \(B_\varepsilon \) no sentido fraco; logo, para todo x em A, y em B e z numa bola \(B_1(0)\),
	\[
		f(x+\varepsilon z)\leq \alpha \leq f(y+\varepsilon z),
	\]
	donde obtemos
	\[
		f(x)-\varepsilon \Vert f \Vert\leq \alpha \leq f(y)+\varepsilon \Vert f \Vert,\quad \forall x\in A,\; y\in B.
	\]
	Portanto, como f é diferente de uma função nula, segue o resultado. \qedsymbol
\end{proof*}
\begin{crl*}
	Seja X um espaço vetorial normado sobre \(\mathbb{K}\) e F um subespaço vetorial próprio de X, \textit{i.e.}, \(\overline{F}\subsetneq X \). Então, existe um funcional não nulo \(f\in X^{*}\) tal que
	\[
		f(x) = 0, \quad \forall x\in F.
	\]
\end{crl*}
\begin{proof*}
	Sabemos que, se X é um espaço vetorial normado sobre \(\mathbb{R},\) dado \(x_{0}\not\in \overline{F},\) existe um funcional linear \(f:X\rightarrow \mathbb{R}\) contínuo tal que
	\[
		f(x)\leq f(x_{0}),\quad \forall x\in F,
	\]
	que leva diretamente à conclusão de \(f(x) = 0\) para todo x de F.

	Para o caso complexo, basta tomar \(g:X\rightarrow \mathbb{C}\) dada por
	\[
		g(x) = f(x)-if(ix),
	\]
	finalizando a prova. \qedsymbol
\end{proof*}
\begin{tcolorbox}[
		skin=enhanced,
		title=Observação,
		fonttitle=\bfseries,
		colframe=black,
		colbacktitle=cyan!75!white,
		colback=cyan!15,
		colbacklower=black,
		coltitle=black,
		drop fuzzy shadow,
		%drop large lifted shadow
	]
	Um lugar de uso do corolário acima é quando temos que mostrar que F é denso, tal que, pela sua contraposição, basta mostrar que
	\[
		f(x) = 0,\quad \forall x\in F \Rightarrow f = 0.
	\]
\end{tcolorbox}

\end{document}

\documentclass[../functional_analysis.tex]{subfiles}
\begin{document}
\section{Aula 14 - 23 de Setembro, 2025}
\subsection{Motivações}
\begin{itemize}
	\item Anuladores e Operadores Duais
\end{itemize}
\subsection{Anuladores}
\begin{def*}
	Seja X um espaço vetorial normado sobre \(\mathbb{K},\) M um subespaço vetorial de X e \(N^{*}\) um subespaço vetorial de \(X^{*}.\) Definimos os conjuntos
	\begin{align*}
		 & M^{\perp } = \{f\in X^{*}:\; f(x) = 0,\; \forall x\in M\}        \\
		 & (N^{*})^{\perp } = \{x\in X:\; f(x) = 0,\; \forall f\in N^{*}\}.
	\end{align*}
	Dizemos que \(M^{\perp }\) é o \textbf{anulador de M} e que \((N^{*})^{\perp }\) é o \textbf{anulador de \(N^{*}\).} \(\square\)
\end{def*}
Observe que \(M^{\perp }\) é um subespaço vetorial \textit{fechado} de \(X^{*}\); de fato, para provar isto, note que se \(f\in \overline{(M^{\perp })}\), então existe uma sequência \(\{f_{n}\}\) em \(M^{\perp }\) com
\[
	f_{n}\substack{n\to \infty \\ \longrightarrow \\ }f,
\]
donde segue que
\[
	0 = f_{n}(x)\substack{n\to \infty \\ \longrightarrow \\ }f(x) \Rightarrow f(x) = 0,\; \forall x\in M,
\]
mostrando que \(f\in M^{\perp }\). Analogamente, \((N^{*})^{\perp }\) é um subespaço vetorial fechado de X, bastando observar que
\[
	\bigcap_{f\in N^{*}}^{}f^{-1}(0) = (N^{*})^{\perp}
\]
e, sendo a intersecção de fechados, \((N^{*})^{\perp }\) é fechado.

\begin{tcolorbox}[
		skin=enhanced,
		title=Observação,
		fonttitle=\bfseries,
		colframe=black,
		colbacktitle=cyan!75!white,
		colback=cyan!15,
		colbacklower=black,
		coltitle=black,
		drop fuzzy shadow,
		%drop large lifted shadow
	]
	O nome \textit{anulador} para \((N^{*})^{\perp }\) não é tão bom, conforme defendeu o professor -- afinal, ele nada mais é que o núcleo do produto interno \(\langle \cdot , n^{*} \rangle\), tanto é
	que foi esse o argumento usado para mostrar que ele	é um subespaço vetorial fechado.
\end{tcolorbox}

\begin{prop*}
	Seja X um espaço vetorial normado sobre \(\mathbb{K}.\)
	\begin{itemize}
		\item[1)] Se M é subespaço vetorial de X, então \((M^{\perp })^{\perp }=\overline{M}\); e
		\item[2)] Se \(N^{*}\) é um subespaço vetorial de \(X^{*}\), então \(((N^{*})^{\perp })^{\perp }\supseteq \overline{(N^{*})}\)
	\end{itemize}
\end{prop*}
\begin{proof*}
	Para o item 1, começamos determinando o que exatamente é o espaço estudado; segue que
	\[
		(M^{\perp })^{\perp }=\{x\in M:\; f(x)=0,\; \forall f\in M^{\perp }\}.
	\]

	Se x é um ponto de M, então \(f(x) = 0\) para todo f em \(M^{\perp }\), donde segue que \(x\in (M^{\perp })^{\perp }\), mostrando a inclusão \(M\subseteq (M^{\perp })^{\perp }\). Finalmente, como \((M^{\perp })^{\perp }\) é fechado, segue que
	\[
		\overline{M}\subseteq (M^{\perp })^{\perp }.
	\]

	Por outro lado, se \(x_{0}\) pertence ao complementar de \(\overline{M}\) relativo a \((M^{\perp })^{\perp }\), então segue do \hyperlink{hahn_banach}{\textit{Teorema de Hahn-Banach}} que existe um funcional f tal que \(f(x)=0\) para todo ponto de \(\overline{M}\), mas \(f(x_{0})\neq 0\), indicando que \(f\in M^{\perp }\) e, como \(x_{0}\in (M^{\perp })^{\perp },\) temos \(f(x_{0})=0\), que é um absurdo. Logo, \((M^{\perp })^{\perp }\subseteq \overline{M}\), seguindo a igualdade.

	Relativo ao item 2, o conjunto é
	\[
		((N^{*})^{\perp })^{\perp }=\{f\in X^{*}:\; f(x)=0,\; \forall x\in (N^{*})^{\perp }\}.
	\]
	Assim, se f é um funcional em \(N^{*}\), então \(f(x)=0\) para todo x em \((N^{*})^{\perp }\), donde segue que f pertence a \(((N^{*})^{\perp })^{\perp }\) e que \(N^{*}\subseteq ((N^{*})^{\perp })^{\perp }.\) Finalmente, do fato que \(((N^{*})^{\perp })^{\perp }\) é fechado, provando que
	\[
		\overline{N}\subseteq ((N^{*})^{\perp })^{\perp }. \text{ \qedsymbol}
	\]
\end{proof*}
\begin{lemma*}
	Seja X um espaço vetorial normado sobre \(\mathbb{K},\; M_1,\; M_2\) subespaços vetoriais de X com \(M_1 \subseteq M_2\) e \(N_{1}^{*},\; N_{2}^{*}\) subespaços vetoriais de \(X^{*}\) com \(N_{1}^{*}\subseteq N_{2}^{*}.\)
	Então,
	\[
		M_{2}^{\perp }\subseteq M_{1}^{\perp }\;\&\; (N_{2}^{*})^{\perp }\subseteq (N_{1}^{*})^{\perp }.
	\]
\end{lemma*}
\begin{proof*}
	Se f pertence a \(M_{2}^{\perp }\), então
	\[
		f(x) = 0,\; \forall x\in M_2.
	\]
	Como \(M_1 \subseteq M_2\), segue que, para todo ponto x de \(M_1\),
	\[
		f(x) = 0
	\]
	e, portanto,
	\[
		f\in M_{1}^{\perp }. \text{ \qedsymbol}
	\]
\end{proof*}
\begin{exr}
	Prove a parte de \((N_{2}^{*})^{\perp } \subseteq (N_{1}^{*})^{\perp }\).
\end{exr}
\begin{def*}
	Um subconjunto \(M^{*}\subseteq X^{*}\) é dito \textbf{total} se \((M^{*})^{\perp }=\{0\}.\; \square\)
\end{def*}
Note que um conjunto total é um composto por coordenadas suficientes para descrever qualquer vetor de X, isto é, dados dois vetores
\(x, y\in X\) cujas coordenadas em \((M^{*})^{\perp }\) coincidem, ou seja, tais que
\[
	\langle x, x^{*} \rangle=\langle y, x^{*} \rangle,\quad \forall x^{*}\in M^{*},
\]
acontecerá que \(x=y\). De certa forma, isto implica que um conjunto total \(M^{*}\) em \(X^{*}\) é grande, embora não precise ser denso em \(X^{*}\).
\begin{prop*}
	Seja \(A:D(A)\subseteq X\rightarrow X\) um operador linear densamente definido; o gráfico de \(A^{*}\)
	\[
		\mathrm{Graf}(A^{*})=\{(x^{*}, A^{*}x^{*}):\; x^{*}\in D(A^{*})\}
	\]
	é o anulador em \(X^{*}\times X^{*}\) do conjunto
	\[
		\{(-Ax, x):\; x\in D(A)\}.
	\]
\end{prop*}
\begin{proof*}
	De fato, se \(x^{*}\in D(A^{*})\), seguirá que, para todo x em \(D(A)\),
	\[
		\langle (-Ax, x), (x^{*}, A^{*}x^{*}) \rangle = 0,
	\]
	tal que
	\[
		(x^{*}, A^{*}x^{*})\in \{(-Ax, x):\; x\in D(A)\}^{\perp },
	\]
	provando que \(\mathrm{Graf}(A^{*})\subseteq \{(-Ax, x):\; x\in D(A)\}^{\perp }\).

	Por outro lado, tomando um par
	\[
		(x^{*}, y^{*})\in \{(-Ax, x):\; x\in D(A)\}^{\perp },
	\]
	temos

	\[
		\langle (-Ax, x), (x^{*}, y^{*}) \rangle=-\langle Ax, x^{*} \rangle + \langle x, ^{*} \rangle =0
	\]
	para todo x em \(D(A)\); disto, segue que \(x^{*}\in D(A^{*})\) e que \(A^{*}x^{*}=y^{*}\). Portanto,
	\[
		(x^{*}, y^{*})\in \mathrm{Graf}(A^{*}) \Rightarrow \{(-Ax, x):\; x\in D(A)\}^{\perp }\subseteq \mathrm{Graf}(A^{*}).\text{ \qedsymbol}
	\]
\end{proof*}
Essa proposição nos dá quase as condições necessárias para dois elementos que coincidem no gráfico de um operador serem os mesmos, mas falta uma parte:
\begin{prop*}
	Se \(A:D(A)\subseteq X\rightarrow X\) é fechado e densamente definido, então \(D(A^{*})\) é total.
\end{prop*}
\begin{proof*}
	Seja x um ponto de X tal que, para todo \(x^{*}\in D(A^{*})\),
	\[
		\langle x, x^{*} \rangle=0.
	\]

	Nosso objetivo é mostrar que x vale 0; como
	\[
		\mathrm{Graf}(A^{*})=\{(x^{*}, A^{*}x^{*}):\; x^{*}\in D(A^{*})\}= \{(-Ax, x):\; x\in D(A)\}^{\perp },
	\]
	vale que, para todo \(x^{*}\in D(A^{*})\),
	\[
		\langle (x, 0), (x^{*}, A^{*}x^{*}) \rangle = 0.
	\]
	Portanto,
	\[
		(x, 0)\in \mathrm{Graf}(A^{*})^{\perp }=\{(-Ax, x):\; x\in D(A)\}^{\perp }\;\&\; x=0. \text{ \qedsymbol}
	\]
\end{proof*}
\subsection{Operadores Compactos}
\begin{def*}
	Sejam X, Y espaços de Banach sobre \(\mathbb{K}\). Diremos que um operador linear \(K:X\rightarrow Y\) é \textbf{compacto} se \(K(B_{1}^{X}(0))\) é um subconjunto \hyperlink{relatively_compact}{\textit{relativamente compacto}} de Y.

	O conjunto dos operadores lineares compactos \(K:X\rightarrow Y\) será denotado por \(\mathcal{K}(X, Y)\). \(\square\)
\end{def*}
\begin{tcolorbox}[
		skin=enhanced,
		title=Lembrete!,
		after title={\hfill Conjunto Relativamente Compacto},
		fonttitle=\bfseries,
		sharp corners=downhill,
		colframe=black,
		colbacktitle=yellow!75!white,
		colback=yellow!30,
		colbacklower=black,
		coltitle=black,
		%drop fuzzy shadow,
		drop large lifted shadow
	]
	Um subconjunto \hypertarget{relatively_compact}{relativamente compacto} Y de um espaço topológico X é um subconjunto cujo fecho \(\overline{Y}\) é compacto, ou seja, no caso de espaços métricos como aqui, diremos que Y é um subconjunto relativamente compacto de X se toda sequência de Y tem uma subsequência convergente em X.

	Essa segunda definição, para o caso de espaços de funções, deveria lembrar o leitor de um \textit{teorema um tanto famoso}\footnote{Arzelá-Ascoli.} na análise.
\end{tcolorbox}
\begin{exr}
	Seja \(X=\mathcal{C}([a, b]; \mathbb{C})\) e \(k\in\mathcal{C}([a, b]\times [a, b]; \mathbb{C})\).
	Defina \(K\in \mathcal{L}(X)\) por
	\[
		(Kx)(t)=\int_{a}^{b}k(t, s)x(s) \mathrm{ds}.
	\]
	Mostre que \(K\) é de fato um operador linear em X e, usando o Teorema de Arzelá-Ascoli, mostre que K é um operador compacto.
\end{exr}
\begin{theorem*}
	Sejam X, Y espaços de Banach sobre \(\mathbb{K}.\) Então, \(\mathcal{K}(X, Y)\) é um subespaço fechado de \(\mathcal{L}(X, Y)\).
\end{theorem*}
\begin{proof*}
	Suponha que \(\{K_{n}\}\) é uma sequência em \(\mathcal{K}(X, Y)\) que converge para \(K\in \mathcal{L}(X, Y)\) na topologia de \(\mathcal{L}(X, Y)\);
	em outras palavras, dado \(\varepsilon >0\), existe \(n_{\varepsilon }\in \mathbb{N}\) tal que
	\[
		K(B_{1}^{X}(0))\subseteq K_{n_\varepsilon }(B_{1}^{X}(0)) + B_{\varepsilon }^{Y}(0).
	\]
	Portanto, \(K(B_{1}^{X}(0))\) é totalmente limitado em Y, consequentemente é relativamente compacto também. \qedsymbol
\end{proof*}
\begin{exr}
	Seja \(X = \ell^{2}(\mathbb{C})\) e \(A:X\rightarrow X\) dado por
	\[
		A\{x_{n}\}=\biggl\{\frac{x_{n}}{n+1}\biggr\},
	\]
	o qual sabemos ser limitado e \(0\in \sigma_{C}(A)\). Mostre que A é um operador compacto.
\end{exr}
\begin{tcolorbox}[
		skin=enhanced,
		title=Lembrete!,
		after title={\hfill Espaços \(\ell^{p}\)},
		fonttitle=\bfseries,
		sharp corners=downhill,
		colframe=black,
		colbacktitle=yellow!75!white,
		colback=yellow!30,
		colbacklower=black,
		coltitle=black,
		%drop fuzzy shadow,
		drop large lifted shadow
	]
	Os espaços \(\ell^{p}\) são subconjunto dos espaços de sequências que consistem nas sequências \(\{x_{n}\}_{n}\) que satisfazem
	\[
		\sum\limits_{n}^{}| x_{n} |^{p}<\infty.
	\]
	Se \(p\geq 1\), podemos definir a norma \(\Vert \cdot  \Vert_{p}\) em \(\ell^{p}\) por
	\[
		\Vert x \Vert_{p}=\biggl(\sum\limits_{n}^{}| x_{n} |^{p}\biggr)^{\frac{1}{p}},\quad \forall x\in \ell^{p},
	\]
	tornando-o um espaço de Banach relativo a esta norma.

	Em particular, para \(p=2\), \(\ell^{2}\) é um espaço de Hilbert com produto interno dado por
	\[
		\langle x, y \rangle=\sum\limits_{n}^{}\overline{x_{n}}y_{n},
	\]
	onde \(x, y\in \ell^{2}\) são as sequências \(\{x_{n}\}_{n}\) e \(\{y_{n}\}_{n}\)
\end{tcolorbox}
\begin{tcolorbox}[
		skin=enhanced,
		title=Lembrete!,
		after title={\hfill Espectro Contínuo},
		fonttitle=\bfseries,
		sharp corners=downhill,
		colframe=black,
		colbacktitle=yellow!75!white,
		colback=yellow!30,
		colbacklower=black,
		coltitle=black,
		%drop fuzzy shadow,
		drop large lifted shadow
	]
	O conjunto de \(\lambda\in \mathbb{K}\) para os quais \(A-\lambda I\) é injetivo e tem imagem densa, mas não é sobrejetor, é chamado \textbf{espectro contínuo de A}
	e denotado por \(\sigma_{c}(A)\).

	O espectro contínuo consiste dos escalares que são aproximadamente autovalores, mas não estão no espectro residual.

	Para o caso de \(A:\ell^{2}(\mathbb{N})\rightarrow \ell^{2}(N)\) dado por \(e_{j}\mapsto e_{j}/j\), segue que \(A\) é injetivo e tem imagem densa, mas \(\mathrm{Im}(A)\subsetneq \ell^{2}(\mathbb{N})\), pois se
	\[
		x=\sum\limits_{j\in \mathbb{N}}^{}c_{j}e_{j}\in \ell^{2}(\mathbb{N})
	\]
	com \(c_{j}\in \mathbb{C}\) tais que \(\sum\limits_{j\in \mathbb{N}}^{}| c_{j} |^{2}<\infty\), não necessariamente é verdade que
	\[
		\sum\limits_{j\in \mathbb{N}}^{}| jc_{j} |^{2}<\infty.
	\]
	Logo,
	\[
		\sum\limits_{j\in \mathbb{N}}^{}jc_{j}e_{j}\not\in \ell^{2}(\mathbb{N}).
	\]
\end{tcolorbox}
\begin{theorem*}
	Sejam \(X, Y, Z \) espaços de Banach sobre \(\mathbb{K},\) A uma transformação linear de X em Y e B uma de Y em Z; então,
	\begin{itemize}
		\item[a)] se A ou B forem um operador compacto, então \(B\circ A\in \mathcal{K}(X, Z)\);
		\item[b)] se A for um operador compacto, então \(A^{*}\in \mathcal{K}(Y^{*}, X^{*})\); e
		\item[c)] se A for um operador compacto e \(\mathrm{Im}(A)\) for um subespaço fechado de Y, então \(\mathrm{Im}(A)\) tem dimensão finita.
	\end{itemize}
\end{theorem*}
\begin{proof*}
	Para provar (b), mostraremos que se \(\{x_{n}^{*}\}\) for uma sequência em \(A^{*}(B_{1}^{Y^{*}}(0))\), então ela possui uma subsequência convergente.

	Para tanto, considere o espaço \(\mathcal{C}(\overline{A(B_{1}^{X}(0))}, \mathbb{K})\) e note que, dados \(y^{*}\in B_{1}^{Y^{*}}(0)\) e \(z\in A(B_{1}^{X}(0))\),
	existe \(x\in B_{1}^{X}(0)\) tal que \(z = Ax\). Consequentemente,
	\[
		| y^{*}(z) | = | y^{*}(Ax) |\leq \Vert A \Vert_{\mathcal{L}(X, Y)}.
	\]
	Além disso, se \(z_1, z_2\in \overline{A(B_{1}^{X}(0))}\), então
	\[
		| y^{*}(z_1) - y^{*}(z_2) | \leq \Vert z_1 - z_2 \Vert_{Y}.
	\]

	Desta forma, a família
	\[
		\mathcal{F} = \biggl\{y^{*}|_{\overline{A(B_{1}^{X}(0))}}:\; y^{*}\in B_{1}^{Y^{*}}(0)\biggr\}
	\]
	é uniformemente limitada e equicontínua em \(\mathcal{C}(\overline{A(B_{1}^{X}(0))}, \mathbb{K})\). Assim, segue do teorema de Arzelá-Ascoli que,
	se \(x_{n}^{*} = y_{n}^{*}\circ A\) com \(y_{n}^{*}\in B_{1}^{Y^{*}}(0)\), existe uma subsequência \(y_{n_k}^{*}\) de \(\{y_{n}^{*}\}\) tal que
	\begin{align*}
		\sup_{x\in B_{1}^{X}(0)}| x_{n_k}^{*}(x) - x_{n_{\ell}}^{*}(x) | & = \sup_{x\in B_{1}^{X}(0)}| y_{n_{k}}^{*}\circ A(x)-y_{n_{\ell}}^{*}\circ A(x) |                    \\
		                                                                 & = \sup_{z\in A(B_{1}^{X}(0))}| y_{n_k}^{*}(z) - y_{n_{\ell}}^{*}(z) | \substack{ k, \ell \to \infty \\ \longrightarrow \\ }0.
	\end{align*}
	Portanto, \(\{x_{n}^{*}\}\) tam subsequência convergente para algum \(x^{*}\in X^{*}\), provando (b). \qedsymbol
\end{proof*}
\begin{exr}
	Prove os itens (a) e (c) do teorema acima.
\end{exr}
\end{document}

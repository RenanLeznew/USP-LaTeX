\documentclass[algebra_notes.tex]{subfiles}
\begin{document}
\section{Aula 09 - 25/04/2023}
\subsection{O que esperar}
\begin{itemize}
	\item Transformações que preservam distâncias - Isometrias;
	\item Caracterizações de Isometrias.
\end{itemize}

\subsection{Motivação para Aula}
Ações de grupos são um conceito importante na álgebra, que estuda a interação entre grupos e estruturas algébricas.
Elas descrevem como os elementos de um grupo atuam em outro conjunto, preservando a estrutura desse conjunto.

Formalmente, uma ação de grupo é uma função que mapeia cada par formado por um elemento do grupo e um elemento do
conjunto em outro elemento do conjunto, obedecendo às seguintes propriedades:
\begin{itemize}
	\item[i)] Identidade: A ação do elemento neutro do grupo (normalmente denotado por "e") em qualquer elemento do conjunto é o próprio elemento, ou seja, $e \cdot x = x.$
	\item[ii)] Compatibilidade: A ação de um produto de elementos do grupo é a mesma que a ação dos elementos
	      individuais na sequência, ou seja, $(gh)x = g(h \cdot x)$, para todos g, h pertencentes ao grupo e x pertencente ao conjunto.
\end{itemize}
Num contexto mais aplicado, as ações de grupos surgem em áreas como física, química, computação, etc.
Na forma de
\begin{itemize}
	\item[1)] Simetrias: Em geometria, ações de grupos são usadas para analisar simetrias de formas geométricas, como rotações e reflexões.
	\item[2)] Teoria dos Grafos: Ações de grupos ajudam a estudar a simetria e as propriedades de grafos em teoria dos grafos.
	\item[3)] Química: Ações de grupos são aplicadas na análise das simetrias moleculares, especialmente na teoria dos grupos de pontos, que é fundamental para a compreensão da espectroscopia e da estrutura molecular.
	\item[4)] Criptografia: Em ciência da computação, as ações de grupos são utilizadas em algoritmos criptográficos e na teoria da codificação.
\end{itemize}
Uma analogia que podemos utilizar para compreender a ideia é a de crianças com um brinquedo: Imagina que
você tem um grupo de amiguinhos e alguns brinquedos. Os amiguinhos são como os elementos do grupo, e os brinquedos
são como os elementos do conjunto. Quando seus amiguinhos brincam com os brinquedos, eles podem trocar os brinquedos
de lugar, girá-los ou virá-los de cabeça para baixo. Essa brincadeira segue algumas regrinhas, e é isso que chamamos de "ações de grupos".
Essa ideia nos ajuda a entender padrões e simetrias em coisas do nosso mundo, como formas geométricas e até moléculas!

De maneira geral, as ações de grupos são uma maneira de descrever como um grupo de elementos interage com outro conjunto de elementos. Essa
interação segue algumas regras e ajuda a entender padrões, simetrias e propriedades em várias áreas do conhecimento, como geometria, química e ciência da computação.

Começaremos introduzindo este assunto por meio das simetrias, partindo do ponto das isometrias.
\subsection{Isometrias}
Aqui, considere $\left< \cdot , \cdot  \right>:\mathbb{R}^{n}\times \mathbb{R}^{n}\rightarrow \mathbb{R}$ o produto interno usual.
\begin{def*}
	Uma isometria $f:\mathbb{R}^{n}\rightarrow \mathbb{R}^{n}$ é uma função que preserva distância, isto é,
	dados u, v em $\mathbb{R}^{n},$
	$$
		|f(u)-f(v)| = |u-v|.
	$$
\end{def*}
\begin{example*}
	O operador linear ortogonal $\varphi :\mathbb{R}^{n}\rightarrow \mathbb{R}^{n}$ é uma isometria.
\end{example*}
\begin{example*}
	Se $a\in \mathbb{R}^{n}, t_{a}:\mathbb{R}^{n}\rightarrow \mathbb{R}^{n}, v\mapsto v+a$ é uma isometria chamada
	de translação.
\end{example*}
\begin{example*}
	A função $\varphi :\mathbb{R}^{2}\rightarrow \mathbb{R}^{2}, \varphi (x,y)=(x\cos{\theta }-y\sin{\theta }, x\sin{\theta }+y\cos{\theta })$
	é uma isometria.
\end{example*}
\begin{example*}
	A composta de isometrias é uma isometria.
\end{example*}
\begin{lemma*}
	Sejam $x,y\in \mathbb{R}^{n}$ tais que $\left< x, x \right> = \left< x, y \right> = \left< y, y \right>.$ Então,
	$x = y.$
\end{lemma*}
\begin{proof*}
	Segue que
	$$
		\left< x-y, x-y \right> = \left< x, x \right>-2\left< x, y \right>+\left< y, y \right> = 0.
	$$
	Portanto, $x-y =0$, ou seja, $x=y.$ \qedsymbol
\end{proof*}
\begin{theorem*}
	Seja $\varphi :\mathbb{R}^{n}\rightarrow \mathbb{R}^{n}.$ São equivalentes:
	\begin{itemize}
		\item[i)] $\varphi $ é isometria e $\varphi (0)=0;$
		\item[ii)] $\left< \varphi (u), \varphi (v) \right> = \left< u, v \right>\forall u, v\in \mathbb{R}^{n};$
		\item[iii)]$\varphi $ é um operador linear ortogonal.
	\end{itemize}
\end{theorem*}
\begin{proof*}
	$c)\Rightarrow a)$ Ok;

	$a)\Rightarrow b)$ Pela definição de isometria, temos
	$$
		|\varphi (u)-\varphi (v)| = |u-v| \Longleftrightarrow \left< \varphi (u)-\varphi (v), \varphi (u)-\varphi (v) \right> =
		\left< u-v, u-v \right>.
	$$
	Fazendo u = 0, temos
	$$
		\left< \varphi (v), \varphi (v) \right> = \left< v, v \right>.
	$$
	Analogamente, para v =0, temos
	$$
		\left< \varphi (u), \varphi (u) \right> = \left< u, u \right>.
	$$
	Como o produto interno é bilinear,
	\begin{align*}
		 & \left< \varphi (u), \varphi (u) \right> - 2\left< \varphi (u), \varphi (v) \right> + \left< \varphi (v), \varphi (v) \right> \\
		 & =\left< u, u \right> - 2\left< u, v \right> + \left< v, v \right>.
	\end{align*}
	Logo,
	$$
		\left< \varphi (u), \varphi (v) \right> = \left< u, v \right>.
	$$

	$b)\Rightarrow c)$ Basta mostrar que $\varphi (u+v)=\varphi (u)+\varphi (v)$ e $\varphi (\alpha u) = \alpha \varphi (u).$
	para todos u, v em $\mathbb{R}^{n}, \alpha \in \mathbb{R}.$

	Sejam $x=\varphi (u+v), y=\varphi (u)+\varphi (v).$ Observe que
	\begin{align*}
		 & \circ{}\left< x, x \right>=\left< \varphi (u+v), \varphi (u+v) \right> = \left< u+v, u+v \right>                                                      \\
		 & \circ{}\left< y, y \right> = \left< \varphi (a)+\varphi (v), \varphi (a)+\varphi (v) \right> =                                                        \\
		 & =\left< \varphi (u), \varphi (u) \right> + 2\left< \varphi (u),\varphi (v) \right> + \left< \varphi (v), \varphi (v) \right>                          \\
		 & =\left< u, v \right> +2\left< u, v \right> + \left< v, v \right> = \left< u+v, u+v \right>.                                                           \\
		 & \circ{}\left< x, y \right>=\left< \varphi (u+v), \varphi (u)+\varphi (v) \right>=\left< u+v, u \right>+\left< u+v, v \right>=\left< u+v, u+v. \right>
	\end{align*}
	Assim, pelo lema anterior, x=y. Analogamente, se $x=\varphi (\alpha u)$ e $y=\alpha \varphi (u),$ temos x=y (exercício). \qedsymbol
\end{proof*}
\begin{crl*}
	Todo isomorfismo f de $\mathbb{R}^{n}$ é uma composição de um operador ortogonal com uma translação. Mais precisamente,
	$f=t_{a}\varphi,$ sendo $\varphi $ ortogonal a $a=f(0).$ Além disso, essa decomposição é única.
\end{crl*}
\begin{proof*}
	Observe que $t_{a}^{-1}f(0)=(t_{a}^{-1})(a)=0.$ Logo, $t_{a}^{-1}f$ é uma isometria que leva 0 em 0. Pelo teorema, $\varphi =t_{a}^{-1}f$
	é um operador linear ortogonal. Deste modo, $f=t_{a}\varphi .$ \qedsymbol
\end{proof*}
\begin{prop*}
	\begin{itemize}
		\item[1)] Se $\varphi $ é um operador linear ortogonal, então $\varphi ^{-1}$ é um operador linear ortogonal.
		\item[2)] Se $\varphi , \psi$ são operadores ortogonais, netão $\varphi\circ{\psi}$ é operador ortogonal.
		\item[3)] $t_{a}\circ{t_{b}}=t_{a+b};$
		\item[4)] $\varphi\circ{t_{a}}=t_{a'}\circ{\varphi },$ em que $a'=\varphi (a)$ e $\varphi $ é operador linear.
	\end{itemize}
\end{prop*}
\begin{proof*}
	Exercício.
\end{proof*}
\begin{crl*}
	Corolário 1: Seja $\mathcal{O}_{n}\coloneqq\{\varphi :\mathbb{R}^{n}\rightarrow \mathbb{R}^{n}: \varphi \text{ é operador linear}\}$.
	Então, $(\mathcal{O}_{n}, \circ)$ é um grupo.
\end{crl*}
\begin{crl*}
	Corolário 2: Seja $T=\{ta:a\in \mathbb{R}^{n}\}$. Então $(T, \circ)$ é um grupo.
\end{crl*}
\begin{crl*}
	Corolário 3: Se $M_{n}$ é o conjunto de todas as isometrias de $\mathbb{R}^{n}$, então $(M_{n}, \circ)$ é um grupo.
\end{crl*}
\begin{proof*}
	Já sabemos que a composta de isometrias é uma isometria e que a operação de composição de funções é associativa.
	Observe que $Id_{\mathbb{R}^{n}}=Id\in M_{n}$, logo dado $f\in M_{n},$
	$$
		f\circ{Id}=Id\circ{f}=f.
	$$
	Agora, se $f\in M_{n}$, do teorema anterior, $f = t_{a}\varphi ,$ sendo a um vetor de $\mathbb{R}^{n}$ e $\varphi $
	um operador linear ortogonal. Netão,
	$$
		f^{-1}\coloneqq\varphi ^{-1}\circ{t_{-a}}=t_{a'}\circ{\varphi ^{-1}},\quad a'=\varphi ^{-1}(-a).
	$$
	Logo, $f^{-1}\in M_{n}.$ Ainda mais,
	$$
		f\circ{f^{-1}} = t_{a}\varphi \varphi ^{-1}t_{-a}=t_{a}t_{-a}=t_{a A}=t_{0}=Id.
	$$
	e, portanto, $f^{-1}\circ{f}=Id.$ \qedsymbol
\end{proof*}
\begin{prop*}
	Seja $a\in \mathbb{R}^{n}$ e $\pi :M_{n}\rightarrow \mathcal{O}_{n}, f=t_{a}p\mapsto \varphi $, então $\pi $
	é um morfismo sobrejetor cujo núcleo é T. Em particular, $T\trianglelefteq{M_{n}}.$
\end{prop*}
\begin{proof*}
	Sejam $f, g\in M_{n}.$ Então, existem $a, b\in \mathbb{R}^{n}$ e $\varphi , \psi\in \mathcal{O}_{n}$ tais que
	$f=t_{a}\varphi $ e $g=t_{b}\psi.$ Desta forma,
	\begin{align*}
		\pi (fg) & = \pi (t_{a}\varphi t_{b}\psi)=\pi (t_{a}t_{b'}\varphi\circ{\psi}) & \\
		         & =\pi (t_{a+b'}(\varphi\circ{\psi}))                                  \\
		         & =\varphi \psi = \pi (f)\pi (g).\text{ \qedsymbol}
	\end{align*}
\end{proof*}
Em particular, pelo Teorema do Isomorfismo, uma consequência desta proposição é que
$$
	\frac{M_{n}}{T}\cong{\mathcal{O}_{n}}.
$$
\end{document}

\documentclass[Algebra/algebra_notes.tex]{subfiles}
\begin{document}
\section{Aula 06 - 11/04/2023}
\subsection{Motivações}
\begin{itemize}
	\item Classes de equivalência;
	\item índice de um grupo;
	\item Teorema de Lagrange.
\end{itemize}

\subsection{Classe de Equivalência de Partições}
\begin{theorem*}
	Uma partição em um conjunto S define uma relação de equivalência em S. Reciprocamente, uma relação de
	equivalência define uma partição em S.
\end{theorem*}
\begin{proof*}
	Seja $S = \bigsqcup_{i\in I}S$ uma partição, tal que
	$$
		R = \{(a,b)\in S\times S: \exists i\in I, a ,b\in S_{i}\}
	$$
	é uma definição de equivalência em S.

	Reciprocamente, dada uma relação de equivalência em S, temos $[a]$ como a classe de equivalência de um elemento a de S.
	Por reflexividade, $a\in[a],$ ou seja, $[a]$ é não-vazio e
	$$
		S = \bigcup_{a\in S}[a].
	$$
	\textbf{Afirmação:} Se $[a]\cap[b]\neq\emptyset,$ então $[a]=[b].$
	Com efeito, seja c um elemento na intersecção das classes [a] e [b]. Segue que $a\sim c$ e $b\sim c$. Por simetria,
	$c\sim b$, logo, por transitividade, $a\sim c\sim b$, ou seja, $c\sim b$. Tome x em [b], ou seja, $b\sim x.$ Novamente,
	por transitividade, $a\sim x,$ isto é, $x\in[a].$ Provamos, então, que $[a]\subseteq{[b]}$ e $[b]\subseteq{[a]}$, ou seja,
	$[a]=[b].$ \qedsymbol
\end{proof*}
\begin{example*}
	A ordem de um elemento define uma relação de equivalência em um grupo. Particularmente, no caso de $S_{3},$ as classes
	de equivalência são
	\begin{align*}
		 & (i)\quad [(12)]  = \{(12), (13), (23)\} \\
		 & (ii)\quad [(123)] = \{(123), (132)\}    \\
		 & (iii)\quad [id] = \{id\}.
	\end{align*}
	Assim, $S_{3}=[(12)]\bigsqcup{}[(123)]\bigcup{[id]}.$ \qedsymbol
\end{example*}
\begin{def*}
	Se $S = \bigsqcup{S_{i}}_{i\in I}$ é uma partição, denotamos $\overline{S}\coloneqq\{[S_{i}]\} $ o conjunto das classes
	de equivalência dadas por $S_{i}.$
\end{def*}
\begin{example*}
	Se $\mathbb{Z}=\{\text{números pares}\}\bigsqcup{\{\text{números ímpares}\}}$, então $\overline{\mathbb{Z}} =\{[\text{pares}], [\text{ímpares}]\}
		\{\overline{0}, \overline{1}\}.$
\end{example*}
Observe que, se S tem uma relação de equivalência, é possível ``projetarmos'' um elemento de S em sua classe de equivalência através de
$$
	\pi:S\rightarrow \overline{S},\quad a\mapsto[a]=\overline{a}.
$$
\begin{example*}
	Fixe um inteiro n. Dados a, b também inteiros, dizemos que $a\sim b$ módulo n (ou $a\equiv b\mod n$) se $n|a-b.$ Mostre que
	a congruência mod n é de fato uma relação de equivalência em $\mathbb{Z}.$ Além disso, $\overline{\mathbb{Z}}^{n} = \mathbb{Z}/n\mathbb{Z} = \{\overline{0}, \cdots, \overline{n-1}\}.$
\end{example*}
\begin{def*}
	Se $\varphi:S\rightarrow T$ é um mapa entre conjuntos, então $\varphi$ define uma relação de equivalência em S dada por
	$$
		a\sim b\quad \Longleftrightarrow \varphi(a) = \varphi(b).
	$$
	Além disso, [a] é definida como a fibra por $\varphi$ de $\varphi(a) = t,$ ou seja,
	$$
		\varphi^{-1}(t) = \{s\in S: \varphi(s) = t\}.
	$$
	Em outras palavras, se $\varphi(a) = t$, então $\overline{a} = [a] = \varphi^{-1}(t).$ Em partícular,
	$$
		S = \bigsqcup_{t\in Im\varphi}{\varphi^{-1}(t)}.
	$$
\end{def*}

\subsection{Classes Laterias e índices.}
\begin{prop*}
	Sejam $\varphi:G\rightarrow G'$ morfismos de grupos e K = $\ker{\varphi}$. Então, a fibr de $\varphi$ que contém o elemento a
	a de G é a classe lateral $aK$. Além disso, essas classes particionam G e correspondem aos elementos da imagem de $\varphi.$
\end{prop*}
\begin{proof*}
	Segue que, dado b em aK, $\varphi(a) = \varphi(b) = t$. Disto segue que a, b pertencem \`a pré-imagem de $\varphi, \varphi^{-1}(t).$
\end{proof*}
\begin{prop*}
	As classes laterais \`a esquerda de H em G são as classes de equivalência da seguinte relação
	$$
		a\cong{b} \Longleftrightarrow b = ah, \quad h\in H.
	$$
	Denotamos $a\cong{b}$ por $a\equiv b.$
\end{prop*}
\begin{proof*}
	Observe que a = 1a, $1\in H.$ Logo, $a\cong{a}.$ Se $a\cong{b},$ existe h em H tal que b = ah, ou seja, $a = bh^{-1}, h^{-1}\in H$
	de modo que $b\cong{a}.$

	Se $a\cong{b}, b\cong{c},$ existem $h_{1}, h_{2}\in H$ tais que $b = ah_{1}, c = bh_{2},$ de forma que $c = ah_{1}h_{2},$ i.e.,
	$a\cong{c}.$

	Por fim,
	\begin{align*}
		[a] & = \{b\in G: a\cong{b}\}                            \\
		    & = \{b\in G: \exists h\in H\text{ tal que }b = ah\} \\
		    & = \{ah: h \in H\} = aH.\text{ \qedsymbol}
	\end{align*}
\end{proof*}
\begin{crl*}
	G = $\bigsqcup_{a\in G\text{distintos}}^{}{aH}.$
\end{crl*}
\begin{example*}
	Em $S_{3} = <(12), (123)>$, seja $H = <(12)>.$ Temos
	\begin{align*}
		 & (i)id H = H = \{id, (12)\} = (12)H                               \\
		 & (ii)(123)H = \{(123), (123)(12)\} = (123)(12)H                   \\
		 & (iii)(123)^{2}H = \{(123)^{2}, (123)^{2}(12)\} = (123)^{2}(12)H.
	\end{align*}
	De fato, $S_{3} = H\bigsqcup_{}^{}{(123)H}\bigsqcup_{}^{}{(123)^{2}H}$ \qedsymbol
\end{example*}
\begin{def*}
	O número de classes laterais (\`a esquerda) de H em G é chamado o índice de H em G, denotado por $[G:H].\square$
\end{def*}
\begin{example*}
	No exemplo anterior, $[S_{3} : H]=3, |H| = 2.$
\end{example*}
\begin{lemma*}
	Todas as classes laterais aH de H em G têm mesma ordem.
\end{lemma*}
\begin{proof*}
	O mapa $m_{a}:H\rightarrow H_{a}, h\mapsto ah$ é um mapa bijetor com inversa $m_{a^{-1}}.$ \qedsymbol
\end{proof*}
Este lema será usado para demonstrar um resultado extremamente importante em Teoria de Grupos. Essencialmente falando,
ele afirma que a ordem de um grupo é um múltiplo da ordem dos seus subgrupos. Mas por que isso importa? Além de ser um
resultado intrigante por si só, ele pode ser usado para mostrar que grupos não são isomórficos, quantidade de elementos de
uma dada ordem dentro de um grupo, entre outras coisas. Segue seu enunciado.
\begin{theorem*}
	Se G é um grupo finito e H um subgrupo de G, então
	$$
		|G| = |H|[G:H].
	$$
	Em particular, $|H|$ divide $|G|.$
\end{theorem*}
\begin{proof*}
	Da proposição anterior, sabemos que $G = \bigsqcup_{a\in{G}\text{diferentes}}^{}{aH}.$ Portanto,
	$$
		|G| = \sum\limits_{a\in{G}\text{diferentes}}^{}|aH| = |H|[G:H]\text{ \qedsymbol}
	$$
\end{proof*}
\begin{crl*}
	Se a pertence a G, então $|a|\biggl|\biggr.|G|$
\end{crl*}
\begin{proof*}
	Segue do Teorema de Lagrange que
	$$
		|a|=|<a>|\biggl|_{}^{}\biggr.|G|\text{ \qedsymbol}
	$$
\end{proof*}
\begin{crl*}
	Se $|G|=p$ número primo e a é um elemento de G diferente do elemento neutro, então
	$$
		G = <a>.
	$$
\end{crl*}
\begin{proof*}
	Se |a| divide p, então $|a|=1$ ou p, mas, como $a\neq 1, |a| = p,$ segue que
	$$
		G = <a>\text{ \qedsymbol}
	$$
\end{proof*}
\begin{crl*}
	Se $\varphi:G\rightarrow G'$ é um morfismo de grupos finitos, segue que
	\begin{align*}
		 & 1)\quad |G| = |\ker{\varphi}||im\varphi|              \\
		 & 2)\quad |\ker{\varphi}| \biggl|_{}^{}\biggr. |G|      \\
		 & 3)\quad |im\varphi|\text{ divide } |G|\text{ e }|G'|.
	\end{align*}
\end{crl*}
\begin{proof*}
	$1\Rightarrow)$ Temos
	\begin{align*}
		G & = \bigsqcup_{}^{}{a\ker{\varphi}}                                              \\
		  & \Rightarrow |G| = |im\varphi||\ker{\varphi}|                                   \\
		  & \Rightarrow |G| = |\ker\varphi|[G:\ker{\varphi}] = |\ker{\varphi}||im\varphi|.
	\end{align*}
	Os outros itens ficam como exercício para os estudantes. \qedsymbol
\end{proof*}
\begin{example*}
	Considere $sgn:S_{n}\rightarrow \{\pm 1\}.$ Temos
	$$
		|im(sgn)| = 2,
	$$
	de modo que
	$$
		|A_{n}| = |\ker{sgn}| = \frac{|S_{n}|}{2} = \frac{n!}{2}.
	$$
\end{example*}
\begin{prop*}
	Se $K\leq{H}\leq{G},$ então
	$$
		[G:K] = [G:H][H:K]
	$$
\end{prop*}
\begin{proof*}
	Suponha que $[G:H] = n$ e $[H:K]=m$. Então,
	$$
		G = g_{1}H\bigsqcup_{}^{}{\cdots}\bigsqcup_{}^{}{g_{n}H} = \bigsqcup_{i=1}^{n}{g_{i}H}\quad\text{e}\quad H = \bigsqcup_{j=1}^{m}{h_{j}K}.
	$$
	A multliplicação por $g_{i}$ é uma bijeção
	$$
		m_{g_{i}}:h_{j}K\rightarrow g_{i}h_{j}K,\quad x\mapsto g_{i}x.
	$$
	Assim, $g_{i}H = \bigsqcup_{j=1}^{m}{g_{i}h_{j}K}.$ Portanto,
	$$
		G = \bigsqcup_{i=1}^{n}{\biggl[\bigsqcup_{j=1}^{m}{g_{i}g_{j}K}\biggr]},
	$$
	de onde concluí-se que $[G:K]=mn.$ \qedsymbol
\end{proof*}
\end{document}

\documentclass[algebra_notes.tex]{subfiles}
\begin{document}
\section{Aula 16 - 25/05/2023}
\subsection{O que esperar?}
\begin{itemize}
	\item Grupos Simples;
	\item Normalizadores;
	\item Teoremas de Sylow.
\end{itemize}
\subsection{Grupos Simples}
\begin{def*}
	Um grupo G é simples se \(G\neq\{1\}\) e não tem subgrupo normal próprio (Não
	existe subgrupo normal de G diferente de \(\{1\}\) ou G.) \(\square\)
\end{def*}
\begin{example*}
	\begin{itemize}
		\item Todo p-grupo é um grupo simples;
		\item \(A_{2}=\{1\}\), ou seja, não é simples;
		\item \(A_{3} = \mathbb{Z}/3 \mathbb{Z}\) é cíclico de ordem 3, logo simples;
		\item \(A_{4}\) não é simples, pois \(H = \{Id, (12)(34), (13)(24), (14)(23)\}\)
		      é um subgrupo normal de \(S_{4}\) e \(H\leq A_{4}.\)
		\item \(A_{n}\) é simples para \(n\geq 5\)
	\end{itemize}
\end{example*}
\begin{def*}
	Seja X o conjunto de todos os subgrupos de um grupo G. Então,
	\[
		G\times X\rightarrow X,\quad (g, H)\mapsto gHg^{-1}
	\]
	é a ação por conjugação e, dado \(H\in X\), seu estabilizador é
	\[
		N(H) = \{g\in G: gHg^{-1}=H\}
	\]
	chamado de normalizador de H. \(\square\)
\end{def*}
Observe que \(N(H)\leq G\) e \(H\leq N(G)\). Além disso, pelo Teorema da Órbita-Estabilizador,
\begin{align*}
	|G| & = |N(H)|\cdot (\#\text{subgrupos de G conjugados por H}) \\
	    & = |N(H)|[G:N(H)].
\end{align*}
\begin{prop*}
	Sejam \(H\leq G, N = N(H)\) normalizador.
	\begin{itemize}
		\item[1)] \(H \trianglelefteq N\);
		\item[2)] \(H \trianglelefteq{G} \Longleftrightarrow N = G.\)
		\item[3)] \(|H|\bigl|\bigr.|G|\) e \(|N|\bigl|\bigr.|G|\)
	\end{itemize}
\end{prop*}
\begin{proof*}
	1) Dado g em N, então \(gHg^{-1}=H\) por definição de \(N = N(H)\), logo
	\(H \trianglelefteq{N}.\)

	2) Para todo g de G, \(H \trianglelefteq{G} \Longleftrightarrow gHg^{-1} = H\) .
	Consequentemente, \(H \trianglelefteq{G}\) se, e somente se, \(g\in N\) para todo g em G.
	Logo, se, e somente se, \(N = G.\)

	3) Ok. \qedsymbol
\end{proof*}
\begin{example*}
	Em \(S_{3}, |C((123))| = 2,\) logo, se \(H = \langle(123)\rangle\), então \(|\mathcal{O}(H)| = 2\) e
	\(|N(H)| = \frac{6}{2} = 3.\)
\end{example*}
\subsection{Teoremas de Sylow}
Adotaremos a seguinte convenção: Se G é um grupo de ordem \(n=p^{varepsilon}m\),
sendo p um número primo e \(mdc(p^{varepsilon}, m) = 1.\)
\begin{def*}
	Se \(|G| = p^{varepsilon}m = n\), um subgrupo \(H\leq G\) é um p-subgrupo de Sylow (ou p-Sylow) se
	\(|H| = p^{varepsilon}.\square\)
\end{def*}
Observe que H é p-subgrupo de Sylow se \(p\nmid [G:H].\)
\begin{example*}
	Em \(S_{3}, \langle (12) \rangle\) é um 2-Sylow e \(\langle (123) \rangle\) é um 3-Sylow.
\end{example*}
Uma pergunta é: quando existem p-subgrupos de Sylow? Por exemplo, \(A_{4}\) tem
ordem 12, mas não tem subgrupo de ordem 6.
\begin{lemma*}
	Seja X o conjunto de todos os subconjuntos de um grupo G. Então, \(G \curvearrowright X\) por multiplicação
	à esquerda e com respeito a esta ação, se \(\mathcal{U}\in X,\)
	\[
		|E(\mathcal{U})|\biggl|\biggr.|G|\quad\text{ e } |E(\mathcal{U})|\biggl|\biggr.|\mathcal{U}|
	\]
\end{lemma*}
\begin{proof*}
	Se \(H\leq G\), temos
	\[
		H\times G\rightarrow G\quad (h, g)\mapsto hg.
	\]
	Observe que \(\mathcal{O}(g) = Hg\). Agora, se \(H = E(\mathcal{U}), \mathcal{U}\in X,\) então
	\[
		H\times \mathcal{U}\rightarrow \mathcal{U} \quad (h, u)\mapsto hu
	\]
	é uma ação de \(\mathcal{U}\) bem-definida. Logo, todas as órbitas dessa ação particionam \(\mathcal{U},\) ou seja,
	existem \(u_{1}, \cdots, u_{n}\in \mathcal{U}\) tais que
	\[
		\mathcal{U} = \bigsqcup_{i=1}^{n}{Hu_{i}}.
	\]
	Logo, \(|\mathcal{U}| = \sum\limits_{i=1}^{n}|Hu_{i}| = \sum\limits_{}^{n}|H| = n|H|.\)
	Portanto, \(|H|\biggl|\biggr.|\mathcal{U}|.\) \qedsymbol
\end{proof*}
\begin{lemma*}
	Seja \(n=p^{\varepsilon }m, \varepsilon >0\), p primo, \(p\nmid m.\) Então,
	o número N de subconjuntos de ordem \(p^{\varepsilon }\) de um conjunto de ordem n não
	é divisível por p, isto é, \(p\nmid N\).
\end{lemma*}
\begin{proof*}
	Exercício. \(\biggl(N = \begin{pmatrix}
		n \\
		p^{\varepsilon }
	\end{pmatrix}\biggr)\)
\end{proof*}
O resultado a seguir é o primeiro dos três teoremas de Sylow.
\hypertarget{sylow_one}{
	\begin{theorem*}
		Seja G um grupo finito e \(p\mid |G|,\) p primo. Então, existe um p-subgrupo de Sylow.
	\end{theorem*}
}
\begin{proof*}
	Seja \(|G| = n = p^{\varepsilon }m, (p^{\varepsilon }, m) = 1\) e considere
	\[
		X = \{\mathcal{U}\subseteq{G}: |\mathcal{U}| = p^{\varepsilon }\}.
	\]
	Temos \(G \curvearrowright X\) por multiplicação à esquerda. Logo, existem
	\(\mathcal{U}_{1}, \cdots, \mathcal{U}_{n}\) em X tais que \(X = \bigsqcup_{}^{}{\mathcal{O}(\mathcal{U}_{i})}\).
	Logo, \(N = |X| = \sum\limits_{}^{}|\mathcal{O}(\mathcal{U}_{i})|\), tal que do segundo lema,
	\(p\nmid N.\) Então, existe j tal que \(p\nmid|\mathcal{O}(\mathcal{U}_{j})|\).
	Agora, considere \(H = E(\mathcal{U}_{j}).\) Do lema 1, também,
	\[
		|H|\biggl|\biggr. |\mathcal{U}_{j}| = p^{\varepsilon }.
	\]
	Então, \(|H| = p^{r}\) para algum \(r\geq 0\). Pelo Teorema Órbita-Estabilizador,
	\[
		|G| = p^{\varepsilon }m = |H||\mathcal{O}(\mathcal{U}_{j})|,
	\]
	mas, como \(p\nmid|\mathcal{O}(\mathcal{U}_{j})|,\) segue que \(|H| = p^{\varepsilon }\)(i.e. \(r=\varepsilon \)). \qedsymbol
\end{proof*}
\begin{crl*}
	Se \(p\mid |G|,\) então existe g em G tal que \(|g| = p.\)
\end{crl*}
\begin{proof*}
	Escreva \(|G| = p^{\varepsilon }m\) \((p^{\varepsilon }, m) = 1\) e seja \(H\leq G\),
	\(|H| = p^{\varepsilon }\) (Aqui, foi usado o \hyperlink{sylow_one}{Primeiro Teorema de Sylow})
	Seja h em H diferente da identidade. Logo, \(|h| = p^{r}.\) Assim, \(g = h^{p^{r}-1}\)
	tem ordem p, pois
	\[
		g^{p} = (h^{p^{r}-1})^{p} = h^{p^{r}} = 1.\quad\text{ \qedsymbol}
	\]
\end{proof*}
Em seguida, temos o segundo dos três teoremas de Sylow.
\hypertarget{sylow_two}{
	\begin{theorem*}
		Seja \(|G| = p^{\varepsilon }m, \varepsilon >0\).
		\begin{itemize}
			\item[1)] Os p-subgrupos de Sylow são conjugados entre si.
			\item[2)] Todo p-subgrupo de G está contido em algum p-subgrupo de Sylow.
		\end{itemize}
	\end{theorem*}}
Observação - No item (1), se H é p-subgrupo de Sylow, então qualquer outro é um elemento de
\(\mathcal{O}(H) = \{gHg^{-1}: g\in G\}\).
\begin{proof*}
	Seja \(X\coloneqq \{xH: x\in G\},\) em que H é um p-subgrupo de Sylow de G. Observe que
	\(|X| = m.\) Agora, considere \(G \curvearrowright X\) por
	\[
		G\times X\rightarrow X,\quad (g, xH)\mapsto gxH.
	\]
	Seja \(K\leq G\) um p-subgrupo e restrinja a ação anterior a K, isto é,
	\[
		K\times X\rightarrow X,\quad (k, xH)\mapsto kxH.
	\]
	Como K é um p-grupo e p não divide \(|X|,\) do \hyperlink{fixed_point}{Teorema do Ponto Fixo,} existe
	\(x\in G\) tal que \(k \cdot xH = xH\) para todo k de K. Logo, dados \(k\in K\) e
	\(h_{1}\in H,\) existe \(h_{2}\in H\) tal que
	\[
		kxh_{1} = xh_{2} \Longleftrightarrow k = xh_{2}h_{1}^{-1}x^{-1}\in xHx^{-1}.
	\]
	Portanto, \(K\subseteq{xHx^{-1}}\) e (2) está provado. Em particular, se K é um p-subgrupo de Sylow,
	então \(K\leq xHx^{-1}, |K|^{\varepsilon } = |xHx^{-1}|\). Portanto, \(K = xHx^{-1}.\) \qedsymbol
\end{proof*}
\begin{crl*}
	Se \(n_{p}\) é o número de p-subgrupos de Sylow distintos, então \(n_{p} = [G:N(H)]\), em
	que H é um algum p-Sylow.
\end{crl*}
\begin{proof*}
	Segue diretamente de
	\[
		n_{p} = |\mathcal{O}(H)| = \frac{|G|}{|N(H)|} = [G:N(G)].\quad\text{ \qedsymbol}
	\]
\end{proof*}
\end{document}

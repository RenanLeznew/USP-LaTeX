\documentclass{article}
\usepackage{amsmath}
\usepackage{amsthm}
\usepackage{amssymb}
\usepackage{amsfonts}
\usepackage{hyperref}

\newtheorem*{exercise*}{Enunciado}

\hypersetup{
    colorlinks,
    citecolor=black,
    filecolor=black,
    linkcolor=black,
    urlcolor=black
}

\title{Exerc\'icios do Elon}
\author{Renan Wenzel}
\date{\today}

\begin{document}
    \maketitle

    \newpage
    \tableofcontents
    \newpage
    
    \section{Cap\'itulo 3}
    \subsection{Exerc\'icio 1}
    \begin{exercise*}
        Seja $f:X\rightarrow{Y}$ uma fun\c c\~ao. A imagem inversa por f de um conjunto $B\subset{Y}$
        \'e o conjunto $f^{-1}(B) = {x\in{X}: f(x)\in{B}}$. Prove que vale sempre $A\subset{f^{-1}(f(A))}$
        para todo $A\subset{X}$ e $f(f^{-1}(B)) \subset{B}$ para todo $B\subset{Y}$. Prove tamb\'em que
        f \'e injetora se, e somente se, $f^{-1}(f(A)) = A$ para todo $A\subset{X}$. Analogamente, mostre
        que f \'e sobrejetora se, e somente se, $f(f^{-1}(B)) = B$ para todo $B\subset{Y}$.       
    \end{exercise*}

    \qedsymbol
    \newpage
    
    \subsection{Exerc\'icio 2}
    \begin{exercise*}
        $f:X\rightarrow{Y}$ \'e injetora se, e somente se, existe uma fun\c c\~ao $g:Y\rightarrow{X}$ tal que g(f(x)) = x para todo $x\in{X}$.
    \end{exercise*}
    
    \paragraph{} Comecemos pela implica\c c\~ao 
    \textit{\begin{quote}"    
    Se $f:X\rightarrow{Y}$ \'e injetora, ent\~ao
    existe uma fun\c c\~ao $g:Y\rightarrow{X}$ tal que g(f(x)) = x para todo $x\in{X}$.
    "\end{quote}} 
    A ideia
    aqui \'e que n\'os definamos uma "coisa" $g:Y\rightarrow{X}$ tal que g(f(x)) = x para todo $x\in{X}$
    que n\'os ainda n\~ao sabemos se \'e uma fun\c c\~ao ou n\~ao. A partir disso, vamos mostrar que,
    de fato, essa g \'e uma fun\c c\~ao. Em outras palavras, mostrar que ela \'e bem-definida (o que
    significa que se $y_1 = y_2$, ent\~ao $g(y_1) = g(y_2)$).

    A priori, suponha que $y_1 = y_2$, mas $g(y_1) \neq g(y_2)$. Suponha tamb\'em que $y_1, y_2$ 
    pertencem \`a imagem da fun\c c\~ao f. Em outras palavras, $y_1 = f(x_1), y_2 = f(x_2)$ para 
    algum $x_1, x_2\in{X}$. Neste caso, se $g(y_1) \neq g(y_2)$, ent\~ao $g(f(x_1)) = x_1 \neq 
    x_2 = g(f(x_2))$. No entanto, isso \'e uma contradi\c c\~ao, pois f \'e injetora, tal que
    $y_1 = y_2$ implica que $f(x_1) = f(x_2)$. 
    
    Agora, lidemos com o caso em que $y_1, y_2$ n\~ao pertencem \`a imagem de f. Com isso, podemos
    definir a fun\c c\~ao g da maneira que desejarmos, pois o caso que importa \'e quando ela \'e
    aplicada a algum elemento da imagem de f. Assim, definindo, por exemplo, $g(y) = 1$ para todo
    y fora da imagem de f. Ent\~ao, se $y_1 = y_2,$ segue que $g(y_1) = 1 = g(y_2)$, tal que a fun\c c\~ao
    est\'a, de fato, bem-definida.

    Resta lidar com a outra implica\c c\~ao, isto \'e,
    \begin{quote}
        Se existe uma fun\c c\~ao $g:Y\rightarrow{X}$ tal que g(f(x)) = x para todo $x\in{X}$, 
        ent\~ao f \'e injetora.
    \end{quote}
    Explicitamente, precisamos mostrar que se $f(x_1) = f(x_2)$, ent\~ao $x_1 = x_2$. De fato,
    suponha que $f(x_1) = f(x_2)$. Aplicando g, segue que:
    $$
        x_2 = g(f(x_2)) = g(f(x_1)) = x_1.
    $$
    Portanto, a fun\c c\~ao \'e injetora.
    \qedsymbol
    \newpage
    
    \subsection{Exerc\'icio 3}
    \begin{exercise*}
        Se $f:X\rightarrow{Y}$ \'e sobrejetora, ent\~ao
        existe uma fun\c c\~ao $g:Y\rightarrow{X}$ tal que f(g(y)) = y para todo $y\in{Y}$.            
    \end{exercise*}

    Vamos mostrar que se $f:X\rightarrow{Y}$ \'e sobrejetora, existe uma fun\c c\~ao $g:Y\rightarrow{X}$ 
    tal que f(g(x)) = x para todo $x\in{X}$ de maneira an\'aloga ao exerc\'icio anterior. Com efeito,
    suponha que $g(y_1) \neq g(y_2)$. Ent\~ao,$y_1 = f(g(y_1)) \neq f(g(y_2)) = y_2,$ tal que g est\'a
    bem-definida para todo y em Y, pois ambos eram arbitr\'arios, completando a prova.
    
    Por outro lado, suponha que existe uma fun\c c\~ao $g:Y\rightarrow{X}$ tal que f(g(y)) = y para todo $y\in{Y}$.
    Note que g(y) \'e, por defini\c c\~ao de g, um elemento de X. Em outras palavras, dado $y\in{Y}$, 
    podemos escrever y como f(g(y)), ou seja, todo elemento de Y pode ser escrito como a f aplicada a
    um elemento de X. Portanto, f \'e sobrejetora. 
    \qedsymbol
    \newpage

    \subsection{Exerc\'icio 4}
    \begin{exercise*}
        Dada uma fun\c c\~ao $f:X\rightarrow{Y}$ e $g, h:Y\rightarrow{X}$ tais que g(f(x)) = x 
        para todo $x\in{X}$ e $f(h(y)) = y$ para todo $y\in{X}$. Prove que g = h.
    \end{exercise*}

    Seja $y\in{Y}$. Como h \'e tal que $f(h(y)) = y$, segue que f \'e sobrejetora. Analogamente,
    como g \'e tal que $g(f(x)) = x$, f \'e injetora. Assim, temos: 
    $$
        h(y) = g(f(h(y))) = g(y).
    $$
    Portanto, como y era um elemento qualquer, segue que h = g.
    \qedsymbol

    \section{Cap\'itulo 7}
    \begin{exercise*}
        Diz-se que o n\'umero real $\alpha$ \'e uma ra\'iz de multiplicidade m do polin\^omio p(x) quando se    
        tem p(x) = $(x - \alpha)^mq(x),$ com $q(\alpha) \neq 0$. Se m = 1 ou m = 2, ela \'e chamada ra\'iz simples
        ou dupla, respectivamente. Prove que $\alpha$ \'e uma ra\'iz simples se, e somente se, $p(\alpha) = 0$ e $p^{'}(\alpha) \neq 0$
        e que $\alpha$ \'e uma ra\'iz dupla se, e somente se, $p(\alpha) = p^{'}(\alpha) = 0$ e $p^{''}(\alpha) \neq 0$. Generalize.
    \end{exercise*}
        Suponha que \'e v\'alido para n-1, isto \'e, $\alpha$ \'e uma ra\'iz de multiplicidade n-1. Explicitamente,
        isso significa que
        $$
            p(x) = (x - \alpha)^{n-1}q(x)
        $$
        implica em $p^{(n-1)}(\alpha) = (n-1)!q^{(n-1)}(\alpha)\neq 0$. Suponha que $\alpha$ \'e uma ra\'iz de 
        multiplicidade n. Ent\~ao, $p^{i}(x) = i!(x - \alpha)^{n-1-i}q^{i}(\alpha)$, tal que 
        $$
        p^{i}(\alpha) = i!(\alpha - \alpha)^{n-1-i}q^{i}(\alpha) = 0. 
        $$
        No entanto, em i = n, temos:
        $$
        p^{n}(\alpha) = n!q^{n}(\alpha) \neq 0,
        $$
        mostrando o que quer\'iamos. Por outro lado, suponha que $p^{i}(\alpha) = 0$ para todo $0\leq{i}<n$ e 
        $p^{n}(\alpha) \neq 0$. Ent\~ao, disto segue:
        $$
        p(x) = (x - \alpha)^m.
        $$ 
        Definindo q(x) = 1, temos $p(x) = (x - \alpha)^mq(x)$ com $q(\alpha) \neq 0$. Portanto, $\alpha$ \'e uma 
        ra\'iz de multiplicidade n do polin\^omio p(x).
\end{document}
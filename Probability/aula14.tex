\documentclass[main.tex]{subfiles}
\begin{document}

\section{Aula 14 - 23/11/2023}
\subsection{Motivações}
\begin{itemize}
	\item Aproximação da Distribuição Normal
\end{itemize}
\subsection{Outras Distribuições}
Antes de dar continuidade, vale a pena montar uma tabela com as distribuições e seus valores de esperança, variância, etc. Vistos até agora:

\begin{center}
	\begin{table}[h!]
		\caption{Distribuições Vistas até o Momento}
		\centering
		\begin{tabular}{| c | c | c | c | c |}
			\hline
			Distribuição                                   & Função de Probabilidade                  & FGM \((M_{X}(t))\)             & Esperança \((\mathbb{E}(X))\) & Variância \((\mathrm{Var}(X))\) \\
			\hline
			Bernoulli \((X\sim \mathrm{Be}(p)\))           & \(p^{x}(1-p)^{1-x}\)                     & \(e^{t}p + 1 - p\)             & p                             & (1-p)p                          \\
			\hline
			Binomial \((X\sim \mathrm{Bin(n; p)})\)        & \(\binom{n}{x}p^{x}(1-p)^{n-x}\)         & \([e^{t}p + 1 - p]^{n}\)       & np                            & np(1-p)                         \\
			\hline
			Geométrica \((X\sim \mathrm{Geo}(p))\)         & \(p(1-p)^{x}\)                           & \(\frac{e^{t}}{1-(1-p)e^{t}}\) & \(\frac{1-p}{p}\)             & \(\frac{1-p}{p^{2}}\)           \\
			\hline
			Poisson \((X\sim \mathrm{Poisson}(\lambda ))\) & \(\frac{e^{-\lambda }\lambda ^{x}}{x!}\) & \(e^{(e^{t}-1)\lambda }\)      & \(\lambda \)                  & \(\lambda \)                    \\
			\hline
		\end{tabular}
	\end{table}
\end{center}

Hoje, iremos acrescentar mais elementos a essa tabela, as \textbf{distribuições contínuas} - mais especificmamente, as distribuições Uniforme, Gama e Exponencial. Nesta ordem,
\subsection{Distribuição Uniforme}
\begin{def*}
	O \textit{modelo uniforme} descreve uma variável aleatória X uniformemente distribuída num intervalo de números reais \(a, b\in \mathbb{R}\) da
	forma \([a, b]\). Denotamos ele por
	\[
		X\sim \mathrm{Uniforme}(a, b).\quad\square
	\]
\end{def*}
A função de distribuição deste modelo é dada por
\[
	f(x) = \frac{1}{b-a}.
\]
A partir disso, podemos calcular a FGM dele como
\begin{align*}
	M_{X}(t) & = \mathbb{E}(e^{tX})                                  \\
	         & = \int_{a}^{b}e^{tx}\frac{1}{b-a}dx                   \\
	         & = \frac{1}{b-a}\frac{e^{tx}}{t}\biggl|_{a}^{b}\biggr. \\
	         & = \frac{1}{t(b-a)}(e^{tb} - e^{ta}).
\end{align*}
Assim, a esperança (lembre-se: dada pela derivada da FGM calculada em t=0), então, será
\begin{align*}
	M_{X}'(0) & = \frac{d}{dt}\biggl(\frac{1}{t(b-a)}(e^{tb}-e^{ta})\biggr)\biggl|_{t=0}^{}\biggr.                    \\
	          & = \frac{t(b-a)be^{tb}-t(b-a)ae^{ta} - (b-a)e^{tb}+(b-a)e^{ta}}{t^{2}(b-a)^{2}}\biggl|_{t=0}^{}\biggr. \\
	          & = \frac{a + b}{2}.
\end{align*}
Ou seja, a esperança deste modelo de probabilidade é \(\mathbb{E}(X) = \frac{a + b}{2}.\) Conseguimos, também,
encontrar a variância, que, após algumas contas, vimos que vale
\[
	\mathrm{Var}(X) = \frac{(b-a)^{2}}{12}.
\]
\subsection{Distribuição Exponencial}
Vejamos, agora, o segundo modelo motivado, a distribuição exponencial
\begin{def*}
	O \textit{modelo exponencial} de uma variável aleatória contínua \(X\geq 0\) descreve
	o tempo entre eventos num processo contínuo e independente que ocorre com frequência média constante
	\(\lambda \). Denotamos este modelo por
	\[
		X\sim \mathrm{Exp}(\lambda ).\quad\square
	\]
\end{def*}
A função de distribuição de probabilidade para este modelo é
\[
	f(x) = \lambda e^{-\lambda x},\quad \lambda > 0.
\]
Faremos um processo análogo ao úlitmo, começando pela FGM:
\begin{align*}
	M_{X}(t) = \mathbb{E}(e^{tX}) & = \int_{0}^{\infty}e^{tx}\lambda e^{-\lambda x}dx                         \\
	                              & = \lambda \int_{0}^{\infty}e^{x(\lambda - t)}dx                           \\
	                              & = \lambda \frac{e^{x(\lambda -t)}}{\lambda -t}\biggl|_{0}^{\infty}\biggr. \\
	                              & = \frac{\lambda }{\lambda -t}.
\end{align*}
Derivando uma vez e calculando em t = 0, obtemos a esperança
\begin{align*}
	\mathbb{E}(X) & = \frac{d}{dt}\frac{\lambda }{\lambda -t}\biggl|_{t=0}^{}\biggr. \\
	              & = \frac{\lambda }{(\lambda -t)^{2}}\biggl|_{t=0}^{}\biggr.       \\
	              & = \frac{1}{\lambda }.
\end{align*}
Além disso, o segundo momento dessa distribuição é
\begin{align*}
	\mathbb{E}(X^{2}) = M_{X}''(t)\biggl|_{t=0}^{}\biggr. & = \biggl[\lambda(\lambda -t)^{-2}\biggr]'\biggl|_{t=0}^{}\biggr. \\
	                                                      & = 2\lambda(\lambda -t)^{-3}\biggl|_{t=0}^{}\biggr.               \\
	                                                      & = \frac{2\lambda }{(\lambda-t)^{3}}\biggl|_{t=0}^{}\biggr.       \\
	                                                      & = \frac{2}{\lambda^{2}}.
\end{align*}
Logo, a variância da variável aleatória X no modelo exponencial vale
\[
	\mathrm{Var}(X) = \mathbb{E}(X^{2}) - (\mathbb{E}(X))^{2} = \frac{2}{\lambda^{2}}-\frac{1}{\lambda^{2}} = \frac{1}{\lambda^{2}}.
\]
\subsection{Distribuição Gama}
O último modelo de hoje é a distribuição com base na função gama.
\begin{def*}
	O \textit{modelo gama} de uma variável aleatória contínua \(X > 0\) modela o tempo entre eventos independentes
	que ocorrem a uma taxa média constante, levando, então, dois parâmetros, \(\alpha\) e \(\beta \).
	Denotamos ela por
	\[
		X\sim \mathrm{Gama}(\alpha , \beta ).\quad\square
	\]
\end{def*}
Para este modelo, a função de distribuição de probabilidade é da forma
\[
	f(x) = \frac{\beta^{\alpha }}{\Gamma (\alpha )}x^{\alpha -1}e^{-\beta x},\quad \alpha , \beta > 0,
\]
em que
\[
	\Gamma (\alpha ) = \int_{0}^{\infty}x^{\alpha -1}e^{-x }dx.
\]
Assim, sua FGM é dada pela forma
\begin{align*}
	M_{X}(t) & = \mathbb{E}(e^{tx})                                                                          \\
	         & = \int_{0}^{\infty}e^{tx}\frac{\beta ^{\alpha }}{\Gamma (\alpha )}e^{-\beta x}x^{\alpha -1}dx \\
	         & = \frac{\beta ^{\alpha }}{\Gamma (\alpha )}\int_{0}^{\infty}x^{\alpha -1}e^{-(\beta -t)x}dx.
\end{align*}
No entanto, note que,
\[
	\int_{0}^{\infty} \frac{\beta ^{\alpha }}{\Gamma (\alpha )}X^{\alpha -1}e^{-\beta x}dx = 1,
\]
ou seja,
\[
	\int_{0}^{\infty}e^{-\beta x}x^{\alpha -1}dx = \frac{\Gamma (\alpha )}{\beta^{\alpha } }.
\]
Aplicando esse raciocínio à fórmula até então encontrada da FGM, chegamos em
\begin{align*}
	M_{X}(t) & = \frac{\beta ^{\alpha }}{\Gamma (\alpha )}\frac{\Gamma (\alpha )}{(\beta - t)^{\alpha }} \\
	         & = \biggl(\frac{\beta }{\beta -t}\biggr)^{\alpha }.
\end{align*}
\subsection{A Nova Tabela}
Agora, podemos acrescentar as novas distribuições à tabela do começo da aula:

\begin{center}
	\begin{table}[h!]
		\caption{Distribuições Vistas até o Momento}
		\centering
		\begin{tabular}{| c | c | c | c | c |}
			\hline
			Distribuição                                    & Função de Probabilidade                                                & FGM \((M_{X}(t))\)                                  & Esperança \((\mathbb{E}(X))\) & Variância \((\mathrm{Var}(X))\) \\
			\hline
			Bernoulli \((X\sim \mathrm{Be}(p)\))            & \(p^{x}(1-p)^{1-x}\)                                                   & \(e^{t}p + 1 - p\)                                  & p                             & (1-p)p                          \\
			\hline
			Binomial \((X\sim \mathrm{Bin(n; p)})\)         & \(\binom{n}{x}p^{x}(1-p)^{n-x}\)                                       & \([e^{t}p + 1 - p]^{n}\)                            & np                            & np(1-p)                         \\
			\hline
			Geométrica \((X\sim \mathrm{Geo}(p))\)          & \(p(1-p)^{x}\)                                                         & \(\frac{e^{t}}{1-(1-p)e^{t}}\)                      & \(\frac{1-p}{p}\)             & \(\frac{1-p}{p^{2}}\)           \\
			\hline
			Poisson \((X\sim \mathrm{Poisson}(\lambda ))\)  & \(\frac{e^{-\lambda }\lambda ^{x}}{x!}\)                               & \(e^{(e^{t}-1)\lambda }\)                           & \(\lambda \)                  & \(\lambda \)                    \\
			\hline
			Uniforme \((X\sim \mathrm{Uniforme}(a, b))\)    & \(\frac{1}{b-a}\)                                                      & \(\frac{1}{t(b-a)}(e^{tb}-e^{ta})\)                 & \(\frac{a + b}{2}\)           & \(\frac{(b-a)^{2}}{12}\)        \\
			\hline
			Exponencial \((X\sim \mathrm{Exp}(\lambda ))\)  & \(\lambda e^{-\lambda x}\)                                             & \(\frac{\lambda }{\lambda -t}\)                     & \(\frac{1}{\lambda }\)        & \(\frac{1}{\lambda^{2}}\)       \\
			\hline
			Gama \((X\sim \mathrm{Gama}(\alpha , \beta ))\) & \(\frac{\beta ^{\alpha }}{\Gamma (\alpha )}X^{\alpha -1}e^{-\beta X}\) & \(\biggl(\frac{\beta }{\beta -t}\biggr)^{\alpha }\) & \(\frac{\alpha }{\beta }\)    & \(\frac{\alpha }{\beta ^{2}}\)  \\
			\hline
		\end{tabular}
	\end{table}
\end{center}
\end{document}

\documentclass[./probability_notes.tex]{subfiles}
\begin{document}

\section{Aula 15 - 30/11/2023}
\subsection{Motivações}
\begin{itemize}
	\item Distribuição Normal;
	\item Variável t-Student;
	\item Distribuição Chi-quadrado.
\end{itemize}
\subsection{O Modelo Normal}
\begin{def*}
	O \textit{modelo normal} de uma variável aleatória X que assume os valores no intervalo \((-\infty, \infty)\), com densidade de probabilidade dada por
	\(f(x) = \frac{1}{\sqrt[]{2\pi \sigma^{2}}}e^{-\frac{1}{2}\biggl(\frac{x-\mu }{\sigma }\biggr)^{2}}\). Denotamos ele por
	\[
		X\sim \mathrm{Normal}(\mu, \sigma )\quad \text{ou}\quad X\sim \mathrm{NO}(0, 1). \quad \square
	\]
\end{def*}
Essa distribuição é a \textbf{mais importante} na estatística, pois ela é a base para o futuo Teorema Central do Limite. Estudemos ela em um pouco mais de detalhe a seguir.
O parâmetro \(\mu \) pode ser interpretado como o centro da curva de sino que essa distribuição forma, enquanto que o valor \(\sigma \) dita o quão longe do centro os valores caem.
\begin{center}
	\begin{tikzpicture}
		\message{Cumulative probability^^J}

		\def\B{11};
		\def\Bs{3.0};
		\def\xmax{\B+3.2*\Bs};
		\def\ymin{{-0.1*gauss(\B,\B,\Bs)}};
		\def\h{0.07*gauss(\B,\B,\Bs)};
		\def\a{\B-0.8*\Bs};

		\begin{axis}[every axis plot post/.append style={
			mark=none,domain={-0.05*(\xmax)}:{1.08*\xmax},samples=\N,smooth},
			xmin={-0.1*(\xmax)}, xmax=\xmax,
			ymin=\ymin, ymax={1.1*gauss(\B,\B,\Bs)},
			axis lines=middle,
			axis line style=thick,
			enlargelimits=upper, % extend the axes a bit to the right and top
			ticks=none,
			xlabel=$x$,
			every axis x label/.style={at={(current axis.right of origin)},anchor=north},
			width=0.7*\textwidth, height=0.55*\textwidth,
			y=700pt,
			clip=false
			]
			% PLOTS
			\addplot[blue,thick,name path=B] {gauss(x,\B,\Bs)};
			% FILL
			\path[name path=xaxis]
			(0,0) -- (\pgfkeysvalueof{/pgfplots/xmax},0);
			\addplot[blue!25] fill between[of=xaxis and B, soft clip={domain=-1:{\a}}];
			% LINES
			\addplot[mydarkblue,dashed,thick]
			coordinates {({\a},{1.2*gauss(\a,\B,\Bs)}) ({\a},{-\h})}
			node[mydarkblue,below=-2pt] {$a$};
			\node[mydarkblue,above right] at ({\B+\Bs},{1.2*gauss(\B+\Bs,\B,\Bs)}) {$f(x)$};
			\node[blue!60!black,above left] at ({0.85*(\a)},{1.0*gauss(0.85*(\a),\B,\Bs)}) {$P(X\leq a)$};

			\addplot[black,dashed,thick]
			coordinates {({1.22*\a},{1.38*gauss(\a,\B,\Bs)}) ({1.22*\a},{-1.22*\h})}
			node[mydarkblue,below=-2pt] {$\mu$};
		\end{axis}
	\end{tikzpicture}
\end{center}
Com relação à FGM, já fizemos ela previamente para \(\mu = 0, \sigma =1\) (vide: ). Assim, seja \(Y = \sigma X + \mu.\)
Temos a regra
\[
	M_{Y}(t) = M_{\sigma X + \mu }(t) = e^{\mu t}M_{X}(\sigma t) = e^{\mu t}e^{\frac{(\sigma t)^{2}}{2}} = e^{\mu t + \frac{\sigma^{2}t^{2}}{2}}.
\]
Com isso, podemos descobrir a esperança e outras informações da distribuição normal:
\begin{itemize}
	\item[i)] Esperança: \(\mathbb{E}(Y) = \mu \)
	\item[ii)] Variância: \(\mathrm{Var}(Y) = \sigma ^{2}\)
	\item[iii)] Assimetria: \(\mathrm{Assim}(Y) = 0\)
	\item[iv)] Curtose: \(\mathrm{Curtose}(Y) = 3\)
\end{itemize}
\subsection{Distribuições Além da Normal}
\begin{def*}
	Dizemos que uma variável aleatória X que assume os valores no intervalo \((-\infty, \infty)\) é \textbf{t-Student} se sua função de probabilidade é dada pela função
	\[
		f(x) = \frac{\Gamma \biggl(\frac{(\nu+1)}{2}\biggr)}{\sqrt[]{\nu\pi }\Gamma \biggl(\frac{\nu}{2}\biggr)}\biggl(1 + \frac{x^{2}}{\nu}\biggr)^{-\frac{(\nu+1)}{2}}, \quad \nu > 0.
	\]
	Denotamos
	\[
		X\sim t(\nu ). \quad \square
	\]
\end{def*}
Não calcula-se a FGM dessa função de distribuição, porque ela é muito complicada. A esperança é nula para \(\nu > 1\),
mas indefinida caso contrário. A variância é:
\[
	\mathrm{Var}(X) = \left\{\begin{array}{ll}
		\frac{\nu}{\nu - 2},\quad \nu > 2 \\
		\infty,\quad 1 < \nu              \\leq 2.
	\end{array}\right.
\]
Caso contrário, é indefinida. A assimetria é dada por
\[
	\mathrm{Assim}(X) = 0,\quad \nu > 3
\]
e indefinida caso contrário. Finalmente, a curtose é
\[
	\mathrm{Curt}(X) = \left\{\begin{array}{ll}
		\frac{6}{\nu-4},\quad \nu > 4 \\
		\infty,\quad 2 < \nu          \\leq 4.
	\end{array}\right.
\]
e indefinida, caso não esteja nessas situações.
\begin{def*}
	Uma variável aleatória X que assume os valores no intervalo \((0, \infty)\) é dita \textit{Chi-quadrado} se sua função de densidade de probabilidade é dada por
	\[
		f(x) = \frac{1}{2^{\frac{k}{2}}\Gamma \biggl(\frac{k}{2}\biggr)}x^{\frac{k}{2}-1}e^{-\frac{x}{2}},\quad k = 1, 2, 3, \dotsc.
	\]
	Denotamos
	\[
		X\sim \chi^{2}(k). \quad \square
	\]
\end{def*}
Observe que a distribuição Chi-quadrado é, na verdade, \(X\sim \mathrm{Gama}\biggl(\frac{k}{2}, \frac{1}{2}\biggr)\)
Assim, para este modelo, a FGM é
\[
	M_{X}(t) = \biggl(\frac{\frac{1}{2}}{\frac{1}{2}-t}\biggr)^{\frac{k}{2}}.
\]
Além disso, a esperança e a variância serão
\[
	\mathbb{E}(X) = \frac{\frac{k}{2}}{\frac{1}{2}} = k \quad\&\quad \mathrm{Var}(X) = \frac{\frac{k}{2}}{\frac{1}{4}} = 2k.
\]
Curiosamente, suponha que temos um conjunto de variáveis aleatórias com distribuições normais \(X_{1}, \dotsc , X_{n}\). Então,
\[
	X_{1}^{2} + \dotsc + X_{n}^{2}\sim \chi^{2}(n)
\]
\begin{def*}
	Uma variável aleatória X que tem valores no intervalo (0, 1) tem distribuição beta se a função de densidade de probabilidade é
	\[
		f(x) = \frac{\Gamma (\alpha +\beta )}{\Gamma (\alpha )\Gamma (\beta )}x^{\alpha -1}(1-x)^{\beta -1}\quad \alpha , \beta > 0.
	\]
	Denotamos
	\[
		X\sim \mathrm{Beta}(\alpha , \beta )\quad \text{ou}\quad X\sim \mathrm{BE}(\alpha , \beta ). \quad \square
	\]
\end{def*}
\end{document}

\documentclass[./probability_notes.tex]{subfiles}
\begin{document}
\section{Aula 12 - 16/11/2023}
\subsection{Motivações}
\begin{itemize}
  \item Modelo de Probabilidade Associado a uma FGM;
  \item Distribuição e Variável Aleatória de Bernoulli;
  \item Distribuição Binomial.
\end{itemize}
\subsection{Modelo de Probabilidade}
\begin{example}
  Considere a variável aleatória X com função massa de probabilidade dada por 
  \[
    \mathbb{P}(X_{i}=x) = \frac{e^{-\lambda_{i} }\lambda_{i}^{x}}{x!},\quad \lambda_{i} > 0, x = 0, 1, 2, \dotsc.
  \]
  Tendo em mente que 
  \[
    M_{X_{i}} = e^{\lambda_{i} (e^{t}-1)},
  \]
  qual é a função geradora de momentos associada a \(Y = X_{1} + X_{2} + X_{3}\)? Além disso, qual modelo de probabilidade ela está associada?

  Podemos aplicar a proposição da FGM da soma de variáveis para lidar com a primeira questão. Assim,
  \begin{align*}
    M_{X_{1}+X_{2}+X_{3}}(t) &= M_{X_{1}}(t)\cdot M_{X_{2}}(t)\cdot M_{X_{3}}(t)\\
                             &= e^{\lambda_{1}(e^{t}-1)}e^{\lambda_{2}(e^{t}-1)}e^{\lambda_{3}(e^{t}-1)}\\
                             &= e^{(e^{t}-1)(\lambda_{1}+\lambda_{2}+\lambda_{3})}.
  \end{align*}
  Além disso, o modelo de probabilidade ao qual a FGM está associada é 
  \[
    \mathbb{P}(Y=y) = \frac{e^{-(\lambda_{1}+\lambda_{2}+\lambda_{3})}(\lambda_{1}+\lambda_{2}+\lambda_{3})^{y}}{y!}
  \]
\end{example}
Este modelo é remetente a uma importante distribuição, muito comum na natureza, que estudaremos logo em seguida.
\subsection{Distribuição de Bernoulli}
\begin{def*}
  Em um experimento aleatório com apenas dois resultados possíveis, definimos uma variável aleatória discreta X, que assume os valores 0 ou 1. Denotamos ela por 
  \[
    X = \mathrm{Bernoulli}(p)\quad\text{ou}\quad X = \mathrm{Ber}(p). \quad\square
  \]
\end{def*}
\begin{example}
  Se X é uma variável aleatória discreta associada ao lançamento de uma moeda e o valor de cara ou coroa, com distribuição de Bernoulli \(X\sim \biggl(\frac{1}{2}\biggr)\), então as probabilidades de X podem ser calculadas como
  \begin{align*}
   &\mathbb{P}(X=0) = \frac{1}{2}\\
   &\mathbb{P}(X=1) = \frac{1}{2}\\
   &\mathbb{P}(X\leq 0) = \frac{1}{2}\\
   &\mathbb{P}(X\leq 1) = 1.
  \end{align*}
  Podemos calcular a esperança associada a X como sendo 
  \[
    \mathbb{E}(X) = 0 \cdot \frac{1}{2} + 1 \cdot \frac{1}{2} = \frac{1}{2},
  \]
  o que também nos dá acesso à Função Geradora de Momentos para esta variável aleatória:
  \[
    M_{X}(t) = \mathbb{E}(e^{tX}) = \frac{e^{t0}}{2} + \frac{e^{1t}}{2} = \frac{e^{t}}{2} + \frac{1}{2}.
  \]
  Vamos ver alguns momentos dessa função, agora.
  \begin{itemize}
    \item[\({1}^{\mathrm{o}}\)):] Segue que 
      \[
        M_{X}'(t)\biggl|_{t=0}^{}\biggr. = \frac{e^{t}}{2}\biggl|_{t=0}^{}\biggr. = \frac{1}{2};
      \]
    \item[\({2}^{\mathrm{o}}\)):] Analogamente, 
      \[
        M_{X}''(t)\biggl|_{t=0}^{}\biggr. = \frac{e^{t}}{2}\biggl|_{t=0}^{}\biggr. = \frac{1}{2};
      \]
    \item[\({3}^{\mathrm{o}}\)):] Seguindo a mesma lógica,
      \[
        M_{X}^{(3)}(t)\biggl|_{t=0}^{}\biggr. = \frac{e^{t}}{2}\biggl|_{t=0}^{}\biggr. = \frac{1}{2}.
      \]
  \end{itemize}
  Com essas informações, segue que a variância de X é 
  \[
    \mathrm{Var}(X) = \frac{1}{2} - \biggl(\frac{1}{2}\biggr)^{2} = \frac{1}{4}.
  \]
  Podemos generalizar esse exemplo para um \(p < 1\) qualquer ao invés de apenas \(\frac{1}{2}.\) Com essa generalização, 
  \[
    \mathbb{E}(X) = p\quad\&\quad \mathrm{Var}(X) = p - p^{2} = p(p-1)\quad\&\quad M_{X}(t) = e^{t}p + 1 - p.
  \]
  O cálculo dos momentos, como esperado pela analogia, resulta em sempre os mesmos momentos - \(M_{X}'(t) = M_{X}''(t) = M_{X}^{(3)}(t)=\dotsc = p.\)
\end{example}
\begin{def*}
  Definimos a função massa de probabilidade para variável aleatória X que assume os valores 0 ou 1 como
  \[
    \mathbb{P}(X=x) = p(1-p)^{x},\quad x=0, 1, \quad 0\leq p\leq 1.\quad \square
  \]
\end{def*}

\subsection{Distribuição Binomial}
\begin{def*}
  Considere um conjunto \(n=1, 2, \dotsc.\) experimentos independentes de Bernoulli, representados pelas variáveis aleatórias \(X_{1}, \dotsc, X_{n}\). A contagem de número de ``sucessos''
  define uma nova variável aleatória \(Y = X_{1} + X_{2} + \dotsc + X_{n}\) que assume os valores \(0, 1, 2, \dotsc, n\). Denotamos:
  \[
    Y \sim \mathrm{Binomial}(n; p)\quad\&\quad Y \sim \mathrm{Bin}(n; p).\quad\square
  \]
\end{def*}
Para uma função com distribuição binomial, \(Y\sim \mathrm{Bin}(n; p)\), podemos repetir o raciocínio feito para a de Bernoulli, encontrando a FGM associada a esta distribuição esperança, a esperança e a variância.
Nesta ordem respectiva, 
\[
  M_{Y}(t) = (1+p(e^{t}-1))^{n},\quad \mathbb{E}(Y) = np,\quad\&\quad \mathrm{Var}(Y) = np(1-p).
\]
Para obter esses resultados, utilizamos o modelo de probabilidade dado por 
\[
  \mathbb{P}(Y=y) = \binom{n}{y}p^{y}(1-p)^{n-y}.
\]
\end{document}

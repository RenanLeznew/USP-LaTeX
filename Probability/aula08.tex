\documentclass[Probability/main.tex]{subfiles}
\begin{document}
\section{Aula 08 - 19/10/2023}
\subsection{Motivações}
\begin{itemize}
	\item Variância;
\end{itemize}
\subsection{Variância}
\paragraph{} Continuando do ponto de partida, buscamos responder o significado da frase ``Associada à variável
aleatória X, temos \(\mathbb{E}(X) = 2\).''

Ao considerarmos uma quantidade muito grande de realizações de X, quando calcula-se a média desses valores,
eles estarão, em média, próximos a 2. No entanto, podem estar muito distantes dele.
\begin{example}
	Suponhamos que X represente a duração da vida de lâmpadas que estejam sendo recebidas de um fabricante e que \(\mathbb{E}(X) = 1000\)horas.
	Isto pode significar uma dentre \textbf{muitas coisas}. Por exemplo, que algumas delas vão estar um pouco acima de \(1000\) horas,
	ou um pouco abaixo, mas giram em torno disso. Por outro lado, pode ser que todas estejam muito acima de mil e muito abaixo de mil.
	Esse tipo de informação não é possível obter apenas com base na esperança.
\end{example}
\begin{def*}
	A \textbf{mediana} é o ponto a partir do qual todos os valores podem ser separados em 50\% acima ou 50\% abaixo. A \textbf{moda} é
	o argumento máximo da distribuição de probabilidade. Matematicamente,
	\[
		\text{Med}(X) \coloneqq \biggl\{x: F(x) = \int_{-\infty}^{x}f(x)dx = \frac{1}{2}\biggr\} ,\quad Mo(x) \coloneqq \text{arg}\max{f(x)}
	\]
	A \textbf{variância} é definida como
	\[
		Var(X)\coloneqq \mathbb{E}\biggl[(X-\mathbb{E}(x))^{2}\biggr] = \mathbb{E}(X^{2}) - [\mathbb{E}(X)]^{2},
	\]
	em que \(\mathbb{E}(X^{2}) = \sum\limits_{i=1}^{\infty}x_{i}^{2}p(x_{i})\) é chamado \textbf{segundo momento central}.
	O \textbf{desvio} é definido por \((X-\mathbb{E}(X))^{2}\). Além disso, definimos como \textbf{desvio padrão} a
	raiz da variância
	\[
		DP(X) \coloneqq  \sqrt[]{Var(X)} = \mathbb{E}\biggl[|X-\mathbb{E}(X)|\biggr].\quad\square
	\]
\end{def*}
No caso contínuo, a esperança assume a forma
\[
	Var(X) = \underbrace{\int_{-\infty}^{\infty}x^{2}f(x)dx}_{\mathbb{E}(X^{2})} - \underbrace{\biggl[\int_{-\infty}^{\infty}xf(x)dx\biggr]^{2}}_{[\mathbb{E}(X)]^{2}}
\]
Algumas propriedades da variância:
\begin{lemma*}
	Sejam X, \(X_{1}, \cdots, X_{n}\) variáveis aleatórias e c uma constante real. Então,
	\begin{itemize}
		\item[a)] Var(c) = 0;
		\item[b)] Var(c+X) = Var(X);
		\item[c)] \(Var(cx) =c^{2}Var(x).\)
		\item[d)] Se \(X_{1}, X_{2}, \cdots, X_{n}\) são independentes,
		      \[
			      Var(X_{1}+X_{2}+\cdots+X_{n}) = Var(X_{1})+Var(X_{2})+\cdots+Var(X_{n}).
		      \]
	\end{itemize}
\end{lemma*}
\newpage
\begin{example}
	Considere a variável aleatória X, que denota o tempo em minutos para o processamento de um produto. A função de probabilidade de X é

	\begin{center}
		\begin{table}[h!]
			\centering
			\begin{tabular}{| c | c c c c c c |}
				\hline
				\(x_{i}\)                 & 2   & 3   & 4   & 5   & 6   & 7   \\
				\hline
				\(\mathbb{P}(X = x_{i})\) & 0.1 & 0.1 & 0.3 & 0.2 & 0.2 & 0.1 \\
				\hline
			\end{tabular}
		\end{table}
	\end{center}

	a) Calcule \(Var(X)\) e \(DP(X)\), sabendo que \(\mathbb{E}(X) = 4.6.\)

	Como \(Var(X) = \mathbb{E}(X^{2}) - [\mathbb{E}(X)]^{2} = \mathbb{E}(X^{2}) - (4,6)^{2},\) podemos reescrever como
	\begin{align*}
		Var(X) & = \sum\limits_{x=2}^{7}x^{2}\mathbb{P}(X=x) - (4,6)^{2} = 2^{2}\cdot 0,1 + 3^{2}\cdot 0,1 + 4^{2}\cdot 0,3 + 5^{2}\cdot 0,2 + 6^{2}\cdot 0,2 + 7^{2}\cdot 0,1 - [4,6]^{2} \\
		       & = 23,2 - [4,6]^{2} = 2,04.
	\end{align*}
	Com isso, o desvio padrão é
	\[
		DP(X) = \sqrt[]{Var(X)} = \sqrt[]{2,04}\approx 1,42\text{min.}
	\]

	b) Encontre Var(2X) e DP(2X), sabendo que \(\mathbb{E}(X) = 4,6.\)

	Segue que \(Var(2X) = 2^{2}Var(X) = 4Var(X) = 4 \cdot 2,04 = 8,16\) e que \(DP(2X) = \sqrt[]{Var(2X)} = 2\sqrt[]{Var(X)}\approx 2,84.\)
\end{example}
\begin{example}
	A demanda diária de um supermercado (em centenas de quilos) pode ser descrita pela variável aleatória X com função densidade
	\[
		f(x) = \left\{\begin{array}{ll}
			\frac{2x}{3},\quad 0\leq x <1,    \\
			\frac{-x}{3} + 1,\quad 1\leq x <3 \\
			0,\quad \text{caso contrário}.
		\end{array}\right.
	\]
	Calcule Var(X) e DP(X), sabendo que \(\mathbb{E}(X)\approx 1,3333.\)

	Temos \(Var(X) = \mathbb{E}(X^{2}) - [\mathbb{E}(X)]^{2}\), ou seja,
	\begin{align*}
		Var(X) & = \int_{-\infty}^{\infty}x^{2}f(x)dx - [\mathbb{E}(X)]^{2}                                                                           \\
		       & = \int_{0}^{1}x^{2}f(x)dx + \int_{1}^{3}x^{2}f(x)dx                                                                                  \\
		       & = \int_{0}^{1}\frac{2}{3}x^{3}dx + \int_{1}^{3}x^{2}\biggl(-\frac{x}{3}+1\biggr)dx - [\mathbb{E}(X)]^{2}                             \\
		       & = \frac{x^{4}}{6}\biggl|_{0}^{1}\biggr. + \biggl[\frac{x^{3}}{3}-\frac{x^{4}}{12}\biggr]\biggl|_{1}^{3}\biggr. - [\mathbb{E}(X)]^{2} \\
		       & = \frac{2}{12} + 9 - \frac{81}{12} - \frac{1}{3} + \frac{1}{12} - [1,3333]^{2}\approx 0,38
	\end{align*}
	Além disso, o desvio padrão vale \(DP(X) = \sqrt[]{0,38}\approx 0,62.\)
\end{example}
\begin{example}
	Suponha a variável aleatória X tal que
	\[
		\mathbb{P}(X=1) = p\quad\text{e}\quad \mathbb{P}(X=0) = 1-p,\quad p\in [0, 1].
	\]
	Como podemos interpretar \(\mathbb{E}(X)\) e Var(X)?

	Sendo essa variável aleatória categórica, segue que, para p = 0,8,
	\[
		\mathbb{E}(X) = \sum\limits_{x=0}^{1}x \mathbb{P}(X=x) = 0 \cdot 0,2 + 1 \cdot 0,8 = 0,8 = p
	\]
	e a variância será
	\[
		\mathbb{E}((X-\mathbb{E}(X))^{2}) = \sum\limits_{x=0}^{1}(x-p)^{2}\mathbb{P}(X=x) = p^{2}\cdot 1 + (1-p)^{2}p = p - p^{2} = p(1-p).
	\]
	que é um modelo cuja variância é máxima quando p vale \(\frac{1}{2}\).
\end{example}
\end{document}

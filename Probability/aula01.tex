\documentclass[main.tex]{subfiles}
\begin{document}

\section{Informações (Possivelmente) Úteis}
\subsection*{Monitoria}
\begin{itemize}
	\item[Monitora:] Patrícia Stülp
	\item[E-mail:] \textit{patriciastulp2@gmail.com}
\end{itemize}
\subsection*{Datas das Provas}
\subsubsection*{Mini provas}
\begin{itemize}
	\item[i)] 29/08/2023;
	\item[ii)] 12/09/2023; 26/09/2023;
	\item[iii)] 10/10/2023; 24/10/2023;
	\item[iv)] 07/11/2023; 21/11/2023;
	\item[v)] 05/12/2023
\end{itemize}
\subsubsection*{(Talvez) P3}
Pode não ocorrer, mas, a depender dos resultados das mini-provas, será dia 12/12/2023.
\subsection*{Bibliografia}
\begin{itemize}
	\item[Principal:] MEYER, P. L. ``Probabilidade: Aplicações à Estatística'', 2a edição, LTC, Rio de Janeiro, 2009.
	\item[Complementar:] ROSS, S. A. ``First Course in Probability", 8th edition, Pearson, 2010.
\end{itemize}

\newpage
\section{Aula 01 - 22/08/2023}
\subsection{Motivações}
\begin{itemize}
	\item O que é aleatoriedade e probabilidade?
	\item Conceitos Fundamentais;
\end{itemize}
\subsection{Aleatoriedade, probabilidade e conceitos fundamentais}
A probabilidade está nos conceitos bases da ciência atual, sendo resultado de uma revolução na ciência há um século atrás.
A noção de aleatoriedade, no entanto, é muito mais difícil de obter uma resposta.

Começamos com um experimento E - um mecanismo gerado. Dele, obtemos um resultado \(\omega \). A estes resultados,
associamos um conjunto, chamado de evento \(A = \{\omega_{1}, \cdots, \omega_{k}\}\). Com estas ideias, associa-se um valor
a um evento, chamado probabilidade \(\mathbb{P}(A).\) Vamos compreender estas ideias mais a fundo.

\subsubsection{Experimentos}
Experimentos podem ser distinguidos em alguns tipos. O primeiro deles é o determinístico. Nele,
o resultado obtido é determinado pelas condições sob as quais o experimento foi executado. A outra forma
é conhecida como experimento aleatório, que engloba experimentos cujos resultados não sabemos \textit{a priori}, isto é,
ainda que as condições iniciais sejam fixas, os resultados não podem ser previstos. Eles possuem as seguintes características:
\begin{itemize}
	\item[a)] Mesmo repetindo várias vezes com as mesmas condições iniciais, o resultado pode mudar;
	\item[b)] Apesar da falta de exatidão, é possível descrever o conjunto de todos os resultados possíveis;
	\item[c)] Há uma regularidade nos resultados após ele ser repetido muitas vezes, permitindo uma modelagem matemática dele.
	      \subsubsection{Espaço Amostral}
	      O espaço amostral denota todos os resultados que podem ocorrer ao realizar um experimento aleatório.

	      \textbf{Observação:} O mecanismo gerador está limitado a um determinado conjunto de possibilidades de saídas.
\end{itemize}
\begin{example}
	Se considerarmos características sócio demográficas de um grupo de pessoas, poderíamos ter
	\begin{itemize}
		\item[Sexo:)] \{Masculino, Feminino, Intersexo\}
		\item[Idade:)] \{0, 1, 2, ...\}
		\item[Estado civil:)] \{Solteiro, Casado, Viúvo, outros.\}
		\item[Renda familiar:)] \(\{x: x\in \mathbb{R}^{+}\}\)
	\end{itemize}
\end{example}
\begin{example}
	Dados os experimentos aleatórios, quais são os espaços amostrais?
	\begin{itemize}
		\item[\(E_{1}\))] Lançar uma moeda 2 vezes e observar as faces obtidas; \(\Omega =\{(C, C), (C, K), (K, C), (K, K)\}\)
		\item[\(E_{2}\))] Retirar uma carta de um barulho comum e observar o naipe; \(\Omega\)=\{``ouros'', ``copas'', ``paus'', ``espadas''\}
		\item[\(E_{3}\))] Duração de lâmpadas, deixando-as acesas até que queimem; \(\Omega = \{t: t\geq 0\}\)
		\item[\(E_{4}\))] Número de mensagens por dia entre uma empresa e um determinado cliente. \(\Omega = \{t: t\geq 0\}\)
	\end{itemize}
\end{example}
\subsubsection{Evento}
Um evento representa qualquer dúvida que possa surgir sobre o resultado de um experimento. Em particular, podem ser visualizados como subconjuntos do espaço amostral. Com isso,
o próprio amostral é um evento, chamado evento certo, assim como o vazio é um evento, dito evento impossível.
\begin{example}
	Considere o resultado obtido com o lançamento de um dado de seis faces, equilibrado. O espaço amostral é \(\{1, 2, 3, 4, 5, 6\}\). Descrevamos os seguintes eventos:
	\begin{itemize}
		\item[1)] A = ``ocorrência do número 3'' = \{3\};
		\item[2)] B = ``sair a face de número 7'' = \(\emptyset\);
		\item[3)] C = ``sair um número menor ou igual a 6'' = \(\Omega \).
	\end{itemize}
\end{example}
\begin{example}
	Podemos considerar experimentos cujos resultados podem ser vetores. A exemplo, o preço de fechamento de determinadas
	ações em uma data específica, como \(\omega = \)(PETR4, ITUB4, ..., AZUL4). Nesse contexto, um evento possível poderia ser
	a situação em que a média desse vetor de preços seja maior do que a média do dia anterior.
\end{example}
\subsubsection{Operações com Eventos}
Veja que um experimento aleatório pode ser tão complicado quanto necessário. Buscamos, porém, buscar simplificações confiáveis para descrever com probabilidades
estes comportamentos, ou seja, estabelecer as propriedades básicas da função de probabilidade. Antes de mais nada, porém, torna-se necessário
saber como operar os eventos.

Como eventos são subconjuntos do espaço amostral, a operação entre eles é dada por meio das operações entre conjuntos -
união, interseção, complemento, etc. Nesta lógica, compreendemos como
\begin{itemize}
	\item[] União de eventos - a capacidade do evento A OU do evento B ocorram;
	\item[] Interseção de eventos - a capacidade do evento A E do evento B ocorrerem simultaneamente.
	\item[] Complementar - a capacidade do evento A não acontecer
	\item[] Eventos mutuamente exclusivos - A e B não ocorrem simultaneamente, isto é, \(A\cap B = \emptyset\)
\end{itemize}
\begin{example}
	Um número entre 1 e 10 é selecionado ao acado. Considere A como o evento em que o número selecionado é múltiplo de 3 e B o conjunto em que o número selecionado é par. Então,
	\(\Omega = \{1, 2, \cdots, 10\}, A = \{3, 6, 9\}, B = \{2, 4, 6, 8, 10\}\). O evento que ocorre nos dois é \(A\cap B=\{6\}\). O evento representando que o número seja um múltiplo de 3 ou
	um número par é \(A\cup B = \{2, 3, 4, 6, 8, 9, 10\}\). Alguns outros são: \(A\cap B^{c} = \{3, 9\}, A^{c}\cap B = \{2, 4, 8, 10\}.\)
\end{example}
\end{document}

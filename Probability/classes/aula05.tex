\documentclass[../main.tex]{subfiles}
\begin{document}

\section{Aula 05 - 19/09/2023}
\subsection{Motivações}
\begin{itemize}
	\item Variáveis Aleatórias;
	\item Distribuições discretas;
	\item Massa de probabilidade.
\end{itemize}
\subsection{Observação}
A aula 04 foi, na verdade, uma aula sobre contabilidade, que irei adicionar futuramente, pois não consegui copiar ela na hora.
\subsection{Variáveis aleatórias}
Na descrição de um experimento aleatório, é conveniente descrever numericamente os resultados.
Vimos, ao longo do curso, exemplos de experimentos que já vinham atrelados a números - o tempo de duração
de uma lâmpada, um número telefônico que chega a uma central, até mesmo os dados e suas faces. No entanto,
em várias situações, esse tipo de informação não está disponível. Também tivemos exemplos
dessas situações, tais quais o sexo de um filho que nasceu e o resultado do lançamento de duas moedas.
Para estes casos, como podemos lidar com eles numericamente? Uma forma natural de fazer
isso é contar quantas vezes uma coisa aparece.
\begin{example}
	Considere uma moeda lançada duas vezes. Seja X a função definida no espaço e que é
	igual ao número de caras nos lançamentos. Vale que:
	\begin{center}
		\begin{table*}[h!]
			\caption{Variável Aleatória de Caras}
			\centering
			\begin{tabular}{| c | c |}
				\hline
				Espaço amostral                   & Valores de X \\
				\hline
				C C                               & 2            \\
				C \(\overline{C}\)                & 1            \\
				\(\overline{C}\) C                & 1            \\
				\(\overline{C}\) \(\overline{C}\) & 0            \\
				\hline
			\end{tabular}
		\end{table*}
	\end{center}
	Observa-se desta tabela que a variável aleatória associada ao evento ``moeda caiu cara" assume os valores
	\(X(CC) = 2, X(C\overline{C})=1, X(\overline{C}C)=1), X(\overline{CC}) = 0.\)
\end{example}
\begin{example}
	Dado um lote de 4 peças, das quais 2 são defeituosas, retiram-se peças até que as duas defeituosas
	sejam retiradas. Coloque como X o número de peças retiradas. Temos:
	\begin{center}
		\begin{table}[h]
			\caption{Número de Peças Retiradas}
			\centering
			\begin{tabular}{| c | c |}
				\hline
				Espaço Amostral & Valores de X      \\
				\hline
				D D             & X(D, D) = 2       \\
				D P D           & X(D, P, D) = 3    \\
				P D D           & X (P, D, D) = 3   \\
				P P D D         & X(P, P, D, D) = 4 \\
				P D P D         & X(P, D, P, D) = 4 \\
				D P P D         & X(D, P, P, D) = 4 \\
				\hline
			\end{tabular}
		\end{table}
	\end{center}
\end{example}
Em outras palavras, é interessante associar, a cada ponto do espaço amostral, um número real. Essa associação é
chamada \textit{variável aleatória}. Formalmente, podemos escrever a seguinte definição:
\begin{def*}
	Uma variável aleatória é uma função \(X:\Omega \rightarrow \mathbb{R}\) definida num espaço amostral e que assume valores reais.
	Sua imagem será denotada por \(R_{X}\). \(\square\)
\end{def*}
Também é importante transmitir a noção de finitude, ou de quantidades não contínuas de valores.
\begin{def*}
	As variáveis aleatórias que assumem valores em um conjunto enumerável serão denominadas discretas. Variáveis aleatórias que assumem valores
	num intervalo da reta real serão denominadas contínuas.\(\square\)
\end{def*}
Os eventos associados a \(\Omega \) são ``relacionados'' a eventos associados com \(R_{X}\) a partir
da seguinte definição:
\begin{def*}
	Seja o espaço amostral \(\Omega \). Seja X uma variável aleatória com imagem \(R_{X}\). Tome A
	um evento em \(\Omega \) e B um evento em \(R_{X}\). Diremos que os eventos A e B são equivalentes se
	\[
		A = \{\omega \in \Omega : X(\omega )\in B\}.\quad\square
	\]
\end{def*}
\begin{def*}
	Seja A um evento em \(\Omega \) e B um evento em \(R_{X}\). Definimos a probabilidade de B como
	\(\mathbb{P}_{X}(B) = \mathbb{P}(A),\) em que \(A = \{\omega \in \Omega : X(\omega )\in B\}.\quad\square\)
\end{def*}
Denotaremos por \(\mathbb{P}_{X}\) a medida de probabilidade induzida por X em \(R_{X}\), tal que \((R_{X}, \cdot , \mathbb{P}_{X})\) será o espaço
de probabilidade induzido pela variável aleatória.
\subsection{Distribuição de Probabilidade}
Vamos nos restringir às variáveis aleatórias discretas. Para conhecermos uma variável aleatória,
além de seus valores, precisamos ter em mente as probabilidades associadas a elas. Nisto, entra a ideia
de distribuição de probabilidade.
\begin{def*}
	A distribuição de probabilidade de uma variável aleatória discreta X, definida em um espaço amostral S,
	é uma tabela que associa a cada valor de X sua probabilidade. Em outras palavras,
	\[
		F(x) = \mathbb{P}(X\in (-\infty, x]) = \mathbb{P}(X\leq x),
	\]
	em que x percorre todos os reais.\(\square\)
\end{def*}
\begin{example}
	Vamos olhar para a distribuição de probabilidade da variável aleatória do caso das moedas.
	Para cada valor de X, determinamos os pontos amostrais nos quais X é igual a tal valor, ou seja,
	a imagem inversa de X. Vamos observar na tabela:
	\begin{center}
		\begin{table}[h]
			\caption{Valores de X, probabilidades e pontos amostrais}

			\centering
			\begin{tabular}{| c | c | c |}
				\hline
				Valores de X & Pontos Amostrais                 & Probabilidades                              \\
				\hline
				0            & \(\overline{C}\overline{C}\)     & \(\frac{1}{4}\)                             \\
				1            & \(\overline{C}C, C\overline{C}\) & \(\frac{1}{2} = \frac{1}{4} + \frac{1}{4}\) \\
				2            & CC                               & \(\frac{1}{4}\)                             \\
				\hline
			\end{tabular}
		\end{table}
	\end{center}
	Os valores das probabilidades são calculados da seguinte forma:
	\begin{align*}
		 & \mathbb{P}[X = 0] = \mathbb{P}(\overline{CC}) = \frac{1}{4} \\
		 & \mathbb{P}[X = 1] = \mathbb{P}(\overline{C}C) = \frac{1}{2} \\
		 & \mathbb{P}[X = 2] = \mathbb{P}(CC) = \frac{1}{4}.
	\end{align*}
\end{example}
\begin{example}
	Consideremos o experimento de lançar um dado sucessivamente sobre uma superfície plana.
	Analisemos quantos lançamentos o número 6 ocorre pela primeira vez - evento que denotaremos por X.
	Temos, para todo \(n\geq 1\),
	\[
		\mathbb{P}(X = n) = \biggl(\frac{5}{6}\biggr)^{(n-1)}\frac{1}{6}.
	\]
	De fato, pelos lançamentos serem independentes, a probabilidade de que não ocorra 6 nos
	(n-1) primeiros lançamentos, mas que ocorra no n-ésimo lançamento, é dada pela fórmula dada acima.
\end{example}
\begin{lemma*}
	A função de distribuição de uma variável aleatória X satisfaz as seguintes condições:
	\begin{itemize}
		\item[a)] \(0\leq F(x)\leq 1\)
		\item[b)] F(x) é não decrescente e é contínua à direita
		\item[c)] \(\lim_{x\to \infty}F(x) = 0\) e \(\lim_{x\to \infty}F(x) = 1.\)
	\end{itemize}
\end{lemma*}
\begin{proof*}
	Isso será, essencialmente, um comentário sobre a ideia da prova.

	a) Segue de que F(x) representa uma probabilidade, ou seja, \(0\leq F(x)\leq 1.\)

	b) Se \(x_{1}\leq x_{2},\) então \(\{\omega \in \Omega : X(\omega ) < x_{1}\}\subseteq{\{\omega \in \Omega : X(\omega )\leq x_{2}\}}\) e, assim, \(\mathbb{P}(\{\omega \in \Omega : X(\omega )[2]\leq x_{1}\})\leq
	\mathbb{P}(\omega \in \Omega : X(\omega )\leq x_{2}),\) ou seja, \(F(x_{1})\leq F(x_{2}).\)

	Para provar a continuidade à direita, seria preciso o seguinte resultado que sai do escopo do curso:
	\begin{quote}
		``Se uma sequência de eventos \(A_{n}\) é decrescente e se aproxima de um evento A, então a sequência
		das probabilidades \(\mathbb{P}(A_{n})\) também é decrescente e tem limite \(\mathbb{P}(A).\)''
	\end{quote}
	Assumindo esse resultado, consideramos uma sequência \(\{x_{n}\}\) de números reais que seja decrescente e que
	\(\lim_{n\to \infty}x_{n} = x.\)  Assim, a sequência de eventos \([X\leq x_{n}]\) é decrescente e se aproxima de \([X\leq x].\)
	Pela propriedade que citada, segue a continuidade à direita.

	c) Para uma sequência de eventos \(A_{n}\) que cresce para um evento A, vale uma propriedade análoga -
	\(\mathbb{P}(A_{n})\) converge para \(\mathbb{P}(A)\). Assim, tome \(\{x_{n}\}\) uma sequência de números reais
	que tende a infinito. Assim, \([X\leq x_{n}]\) é uma sequência que tende ao evento \([X < \infty]\), tal que \(\mathbb{P}[X\leq x_{n}]\)
	converge para \(\mathbb{P}[X <\infty]\), que é trivialmente o espaço todo, cuja probabilidade é 1. Para demonstrar a segunda parte,
	considera-se uma sequência \(\{x_{n}\}\) que tende a \(-\infty\), ou seja, \([X\leq x_{n}]\) tende ao vazio \(\emptyset\) e, portanto,
	\(F(x_{n}) = \mathbb{P}[X\leq x_{n}].\) \qedsymbol
\end{proof*}
\begin{example}
	Seja
	\[
		F(x)  = \left\{\begin{array}{ll}
			0,\quad x < 0;                 \\
			\frac{1}{2},\quad 0\leq x < 1; \\
			1,\quad x\geq 1.
		\end{array}\right.
	\]
	Essa F é uma função de distribuição?

	Ela é, de fato. Para ver isso, note que, como \(F(x) = 0\) para \(x < 0\) e \(F(x) = 1\) para
	\(x\geq 1\),
	\[
		\lim_{x\to -\infty} F(x) = 0\quad\&\quad \lim_{x\to +\infty}F(x) = 1
	\]
	Além disso, a continuidade nos reais entre 0 e 1 é simples. Para os extremos, temos
	\[
		F(0) = \lim_{x\to 0^{+}}F(x) = \frac{1}{2}\quad\&\quad F(1) = \lim_{x\to 1^{+}}F(x) = 1.
	\]
	Por fim, F é não decrescente pela sua definição.
\end{example}
\begin{def*}
	A função de probabilidade de uma variável aleatória discreta X é uma que atribui probabilidade
	a cada um dos possíveis valores \(x_{i}\) assumidos pela variável aleatória X, isto é,
	\[
		p(x_{i}) = \mathbb{P}(X = x_{i}) = \mathbb{P}(\{\omega \in \Omega (\omega ): X(\omega ) = x_{i}\}),\quad i = 1, 2, \cdots, n.
	\]
	Além disso, \(p(x_{i})\) deve satisfazer
	\[
		0\leq p(x_{i})\leq 1,\quad i = 1, \cdots, n
	\]
	e
	\[
		\sum\limits_{i=1}^{n}p(x_{i}) = 1.\quad \square
	\]
\end{def*}

\subsection{EXTRA: Densidade de Probabilidade}
As densidades de probabilidade surgem para tratar das variáveis aleatórias contínuas. Um dos problemas
que surgem é que a soma dos valores em quantidades não enumeráveis de números positivos não pode ser igual a um.
Com isso, definimos a densidade de probabilidade como uma função não negativa tal que sua integral, avaliada
num dado intervalo, equivale à probabilidade da variável pertencer a este intervalo. Além disso, para ser condizente com
a probabilidade total ser 1, impõe-se que a integral estendida à reta toda seja um.
\begin{def*}
	A densidade de probabilidade de uma variável aleatória contínua é uma função \(f(x)\geq 0\) tal que
	\[
		\int_{-\infty}^{+\infty}f(x)dx = 1.
	\]
	Além disso, a probabilidade de uma variável aleatória X pertencer a um intervalo da forma \((a, b]\) é dada por
	\[
		\mathbb{P}[a < X\leq b] = \int_{a}^{b}f(x)dx.\quad \square
	\]
\end{def*}
\begin{example}
	Seja \(f(x) = x\) para \(0\leq x\leq 1\) e \(f(x) = 2 - x\) para \(1\leq x\leq 2\). No complementar desses dois,
	coloque \(f(x)=0.\) Note que
	\[
		\int_{0}^{2}f(x)dx = \int_{0}^{1}xdx + \int_{1}^{2}2-xdx = 1.
	\]
	Vamos calcular, também, as probabilidades \(\mathbb{P}[0\leq X\leq 0.8].\) Segue que
	\[
		\mathbb{P}[0\leq X\leq 0.8]=\int_{0}^{0.8}xdx = \frac{1}{2}x^{2}\biggl|_{0}^{0.8}\biggr. = \frac{1}{2}(0.8)^{2} = 0.32.
	\]
	e
	\[
		\mathbb{P}[0.3\leq X\leq 1.5] = \int_{0.3}^{1}xdx + \int_{1}^{1.5}2-xdx = 0.83.
	\]
\end{example}
\end{document}

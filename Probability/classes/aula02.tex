\documentclass[../probability_notes.tex]{subfiles}
\begin{document}

\section{Aula 02 - 24/08/2023}
\subsection{Motivações}
\begin{itemize}
	\item As definições de probabilidade;
	\item A probabilidade segundo Kolmogorov.
\end{itemize}
\subsection{Conceitos}
\paragraph{}As ideias desenvolvidas até aqui são as que Laplace desenvolveu, contendo toda a parte do cálculo
de probabilidade por meio da contagem de casos favoráveis e dos possíveis. No entanto, foram desenvolvidas outras
abordagens para lidar com as limitações da dependência na uniformidade das saídas, no número finito de resultados possíveis, etc.
A definição clássica de probabilidade é a razão entre o número de casos favoráveis e o de possíveis, ou seja,
\begin{def*}
	Seja o evento A, associado a um espaço amostral finito e equiprovável, \(\Omega \). Definimos a probabilidade de ocorrência do evento A por:
	\[
		\mathbb{P}(A) = \frac{\#(A)}{\#(\Omega )}.\quad \square
	\]
\end{def*}
Richard von Mises buscou enxergar a probabilidade como algo que pode ser definido apenas para um gerador aleatório capaz de produzir uma sequência infinita de resultados.
A probabilidade, aqui, será a frequência limite do resultado nessa sequência. Ele baseou isso na ideia de que a aleatoriedade é um processo que produz resultados imprevisíveis
e não claramente determinados.
\begin{def*}
	Seja \(n_{A}\) o número de vezes que o evento A ocorre em n repetições independente de um mesmo experimento. Então,
	\[
		\mathbb{P}(A) = \lim_{n\to \infty}\frac{n_{A}}{n},
	\]
	desde que o limite exista. \(\square\)
\end{def*}
Essa definição mostrou-se importante até durante a contemporaneidade. No entanto, não é muito prático para fazer contas e encontrar probabilidades,
sua força está em exibir a noção que a aplicabilidade traz. Além disso, ela funciona como uma ponte entre a primeira definição e a próxima que será vista.
Outro problema com essa é que, quando quantidades enormes de saídas surgem, torna-se impossível de usá-lo, pois não dá para saber o total de possibilidades.

Assim, Kolmogorov estende a definição de probabilidade para espaços mais gerais, conhecido como a pessoa que formalizou a teoria da probabilidade. A ideia dele permite que
propostas mais flexíveis de probabilidades sejam usadas, saindo do limite de contabilidade finita e equiprovável da ideia de Laplace.
\begin{def*}
	Seja E um experimento aleatório e \(\Omega \) o espaço amostral associado. A cada evento A associamos um número real
	\(\mathbb{P}(A)\), denominado probabilidade de A, que satisfaz
	\begin{itemize}
		\item[P1)] \(\mathbb{P}(\Omega )=1;\)
		\item[P2)] \(0\leq \mathbb{P}(A)\leq 1,\) para todo A decorrente de \(\Omega \);
		\item[P3)] Se A e B forem eventos mutuamente exclusivos, então
		      \[
			      \mathbb{P}(A\cup B) = \mathbb{P}(A) + \mathbb{P}(B);
		      \]
		\item[P4)] Para qualquer sequência de eventos disjuntos dois-a-dois, \(A_{1}, A_{2}, \cdots, A_{n}\), tem-se
		      \[
			      \mathbb{P}\biggl(\bigcup_{i=1}^{n}A_{i}\biggr) = \sum\limits_{i=1}^{n}\mathbb{P}(A_{i}).\quad \square
		      \]
	\end{itemize}
\end{def*}
Seguem algumas propriedades:
\begin{theorem*}
	Se \(\emptyset\) é um evento impossível, então \(\mathbb{P}(\emptyset) = 0.\)
\end{theorem*}
\begin{theorem*}
	Se \(A^{c}\) for o complementar de A, então \(\mathbb{P}(A^{c}) = 1 - \mathbb{P}(A).\)
\end{theorem*}
\begin{theorem*}
	Se A e B são dois eventos quaisquer de \(\Omega \), então
	\[
		\mathbb{P}(A\cup B) = \mathbb{P}(A) + \mathbb{P}(B) - \mathbb{P}(A\cap B).
	\]
\end{theorem*}
\begin{theorem*}
	Se \(A_{1}, A_{2}, \cdots, A_{i}\) são eventos dois-a-dois disjuntos, então
	\begin{align*}
		\mathbb{P}(A_{1}\cup \cdots\cup A_{i}) & = \sum\limits_{i=1}^{n}\mathbb{P}(A_{i}) - \sum\limits_{i < j}^{n} \mathbb{P}(A_{i}\cap A_{j}) + \sum\limits_{i < j < r}^{n} \mathbb{P}(A_{i}\cap A_{j}\cap A_{r}) + \cdots \\
		                                       & \cdots + (-1)^{n-1}\mathbb{P}(A_{1}\cap \cdots A_{n})
	\end{align*}
\end{theorem*}
\begin{example}
	Um lote é formado por 10 peças boas, 4 com defeitos menores e 2 com defeitos graves. Uma peça é escolhida ao acaso. Calcule a probabilidade de que
	\begin{itemize}
		\item[a)] A peça não tenha defeito grave;
		\item[b)] A peça não tenha defeito;
		\item[c)] A peça seja boa ou tenha defeito grave.
	\end{itemize}
	a) Considerando A o conjunto de peças boas, B o de peças levemente defeituosas e C o de peças gravemente defeituosas, esses conjuntos são dois-a-dois disjuntos. Assim,
	\[
		\mathbb{P}(C^{c}) = 1 - \mathbb{P}(C) = 1 - \frac{2}{16} = \frac{7}{8}.
	\]

	b) Com a notação anterior,
	\[
		\mathbb{P}((B\cup C)^{c}) = \mathbb{P}(B^{c}\cap C^{c}) = \mathbb{P}(B^{c}) + \mathbb{P}(C^{c}) - \mathbb{P}(C^{c}\cup B^{c}) = \mathbb{P}(A) = \frac{10}{16} = \frac{5}{8}.
	\]

	c) Novamente, mantendo a anotação,
	\[
		\mathbb{P}(A\cup C) = \mathbb{P}(A) + \mathbb{P}(C) = \frac{10}{16} + \frac{2}{16} = \frac{12}{16} = \frac{3}{4}.
	\]
\end{example}
\end{document}

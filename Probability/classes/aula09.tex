\documentclass[../main.tex]{subfiles}
\begin{document}

\section{Aula 09 - 31/10/2023}
\subsection{Motivações}
\begin{itemize}
	\item Desigualdade de Tchebycheff;
	\item Momentos de Ordem Superior.
\end{itemize}
\subsection{Desigualdade de Thebycheff}
O conhecimento da função de probabilidade de uma variável aleatória possibilita-nos calcular a esperança e a variância
dela, desde que elas existam. No entanto, não somos capazes de fazer o oposto - ou seja, conhecer \(\mathbb{E}(X)\)
e Var(X) não permite-nos reconstruir a função densidade nem a função de probabilidade de X.
Embora isso seja impossível, o que podemos fazer é estabelecer limites superiores e inferiores muito úteis para estimá-las.
Uma dessas formas está contida no que é conhecido como Desigualdade de Tchebycheff.

A desigualdade de Tchebycheff é a seguinte:
\begin{theorem*}
	Suponha que X é uma variável aleatória
	com esperança \(\mathbb{E}(X) = \mu\) e seja c um número real qualquer. Então, se \(\mathbb{E}((X-c)^{2})\)
	for finita e \(\varepsilon >0\) for um número positivo qualquer, teremos
	\[
		\mathbb{P}(|X-c|\geq \varepsilon )\leq \frac{1}{\varepsilon ^{2}}\mathbb{E}((X-c)^{2}).
	\]
\end{theorem*}
\begin{proof*}
	Começamos analisando a primeira parte da expressão. Temos
	\begin{align*}
		\mathbb{P}(|X-c|\geq \varepsilon ) & = 1 - \mathbb{P}(|X-c| < \varepsilon )                                                    \\
		                                   & = 1 - \mathbb{P}(-\varepsilon + c < X < \varepsilon + c)                                  \\
		                                   & = 1 - \int_{c-\varepsilon }^{c+\varepsilon }f(x)dx\geq 1 - \int_{-\infty}^{\infty}f(x)dx.
	\end{align*}
	Note que
	\[
		\mathbb{P}(|X-c|\geq \varepsilon ) = \int_{-\infty}^{c-\varepsilon }f(x)dx + \int_{c+\varepsilon }^{\infty}f(x)dx\leq \int_{-\infty}^{c-\varepsilon }f(x)dx + \int_{c+\varepsilon }^{\infty}\frac{(X-c)^{2}}{\varepsilon ^{2}}f(x)dx
	\]
	O mesmo vale para a integral imprópria de \(-\infty\) a \(c-\varepsilon \), pois elevamos \(-\varepsilon \) ao quadrado e chegamos no seguinte:
	\begin{align*}
		\mathbb{P}(|X-c|\geq \varepsilon ) & \leq \int_{-\infty}^{c-\varepsilon }\frac{(X-c)^{2}}{\varepsilon ^{2}}f(x)dx + \int_{c+\varepsilon }^{\infty}\frac{(X-c)^{2}}{\varepsilon ^{2}}f(x)dx \\
		                                   & \leq \int_{-\infty}^{\infty}\frac{(X-c)^{2}}{\varepsilon ^{2}}f(x)dx = \frac{1}{\varepsilon ^{2}}\int_{-\infty}^{\infty}(X-c)^{2}f(x)dx               \\
		                                   & = \frac{1}{\varepsilon ^{2}}\mathbb{E}[(X-c)^{2}].
	\end{align*}
	Portanto, obtemos a desigualdade desejada. \qedsymbol
\end{proof*}
Há algumas formas equivalentes de interpretarmos isso.
Considerando o evento complementar, temos
\[
	\mathbb{P}(|X-c|< \varepsilon )\geq 1 - \frac{1}{\varepsilon ^{2}}\mathbb{E}((X-c)^{2})
\]
Se \(c = \mu,\) temos
\[
	\mathbb{P}(|X-\mu|\geq \varepsilon )\leq \frac{\mathrm{Var}(X)}{\varepsilon ^{2}}.
\]
Se \(c=\mu\) e \(\varepsilon = k\sigma \), em que \(\sigma ^{2}=\mathrm{Var}(X) > 0\), obtemos
\[
	\mathbb{P}(|X-\mu|\geq k\sigma )\leq k^{-2}.
\]

Uma das coisas importantes que são notáveis é que ele \textbf{não depende do modelo de probabilidade}. Em momento algum colocamos
alguma hipótese sobre \(\mathbb{P}(\cdot )\)! Afinal, caso o modelo fosse conhecido, valeria mais a pena simplesmente fazer a conta
\begin{example}
	Considere a terceira forma equivalente da desigualdade de Tchebycheff, isto é,
	\[
		\mathbb{P}(|X-\mu|\geq k\sigma )\leq k^{-2}.
	\]
	Para \(k=\frac{3}{2},\) o que podemos dizer sobre X? E se soubermos que X é uniformemente
	distribuída em \(\biggl(1-\frac{1}{\sqrt[]{3}}, 1 + \frac{1}{\sqrt[]{3}}\biggr)\)?

	Para a primeira parte, segue que
	\[
		\mathbb{P}(|X-\mu|\geq 1,5\sigma )\leq \frac{4}{9}.
	\]
	Quanto à segunda, dado que X é uma variável aleatória distribuída uniformemente em \(\biggl(1-\frac{1}{\sqrt[]{3}}, 1 + \frac{1}{\sqrt[]{3}}\biggr)\),
	sabemos descrever a densidade de probabilidade dela como
	\[
		f(X) = \frac{1}{1+\frac{1}{\sqrt[]{3}} - 1 + \frac{1}{\sqrt[]{3}}} = \frac{\sqrt[]{3}}{2}.
	\]
	Com isso, somos capazes de calcular a esperança de X
	\[
		\mathbb{E}(X) = \int_{1-\frac{1}{\sqrt[]{3}}}^{1 + \frac{1}{\sqrt[]{3}}} \frac{\sqrt[]{3}}{2}xdx = \frac{\sqrt[]{3}}{2}\frac{x^{2}}{2}\biggl|_{1-\frac{1}{\sqrt[]{3}}}^{1+\frac{1}{\sqrt[]{3}}}\biggr. = \frac{\sqrt[]{3}}{4} \frac{4}{\sqrt[]{3}} = 1.
	\]
	Além disso,
	\[
		\mathbb{E}(X^{2}) + \int_{1-\frac{1}{\sqrt[]{3}}}^{1 + \frac{1}{\sqrt[]{3}}} \frac{\sqrt[]{3}}{2}x^{2}dx = \frac{\sqrt[]{3}}{2}\frac{x^{3}}{3}\biggl|_{1-\frac{1}{\sqrt[]{3}}}^{1+\frac{1}{\sqrt[]{3}}}\biggr. = \frac{\sqrt[]{3}}{6}3,8 (????)
	\]
	Depois da confusão em aula, foi mostrada que a resposta é
	\begin{align*}
		\mathbb{P}\biggl(|X-1|\geq \frac{3}{2}\sqrt[]{\frac{1}{9}}\biggr) & = \mathbb{P}\biggl(|X-1|\geq \frac{1}{2}\biggr) \\
		                                                                  & = 1 - \frac{\sqrt[]{3}}{2} \approx 0,134.
	\end{align*}

\end{example}
\subsection{Momentos de Ordem Superior}
Podemos introduzir uma noção mais geral sobre as questões até o momento vistas introduzindo outros ``momentos'' - informações importantes que usamos
para caracterizar distribuições de probabilidade. Já vimos alguns deles - o primeiro momento é a esperança, o segundo é a dispersão e o terceiro, ainda não
visto, fornece um coeficiente de assimetria da distribuição e o quarto diz respeito ao ``achatamento'' dela.
\begin{def*}
	Se X é uma variável aleatória discreta, com função de probabilidade \(p(x_{i}) = \mathbb{P}(X = x_{i}),\) então o \textbf{momento de ordem k}
	é definido como
	\[
		\mu_{k} = \mathbb{E}(X^{k}),
	\]
	em que
	\[
		\mathbb{E}(X^{k}) = \sum\limits_{i=1}^{\infty}x_{i}^{k}p(x_{i}).
	\]
	Caso X seja uma variável aleatória contínua, a definição segue a mesma, mas
	\[
		\mathbb{E}(X^{k}) = \int_{-\infty}^{\infty}X^{k}f(x)dx.\quad\square
	\]
\end{def*}
Além dessa noção, podemos introduzir o momento central de ordem k como
\begin{def*}
	Se X for uma variável aleatória, com esperança finita, então o \textbf{momento central de ordem k} é definido como
	\[
		\mu_{k}^{*} = \mathbb{E}\biggl([X-\mathbb{E}(X)]^{k}\biggr).\quad\square
	\]
\end{def*}
Para evitar a expansão de altas ordens na integral, definimos a seguinte função:
\begin{def*}
	A \textbf{função geradora de momentos} da variável aleatória X é definida para \(t\in \mathbb{R}\) por
	\[
		M_{X}(t) = \mathbb{E}(e^{tX}),
	\]
	desde que a esperança seja finita para \(t\in \mathbb{R}\) em algum intervalo \(-t_{0} < t < t_{0},\) com \(t_{0} > 0.\square\)
\end{def*}
É preciso mostrarmos que isso de fato coincide com a definição.
\end{document}

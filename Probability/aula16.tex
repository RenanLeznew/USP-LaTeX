\documentclass[./probability_notes.tex]{subfiles}
\begin{document}
  
\section{Aula 16 - 05/12/2023}
\subsection{Motivações}
\begin{itemize}
  \item Multivariáveis Aleatórias;
  \item Funções de Densidade de Probabilidade em Multivariáveis.
\end{itemize}
\subsection{Multivariáveis Aleatórias}
\begin{def*}
  Seja E um experimento e \(\Omega \) um espaço amostral associado. Considere \(X = X(\Omega )\) e \(Y = Y(\Omega )\) duas variáveis aleatórias,
  cada uma associando um número real a cada \(\omega \in \Omega \). Uma \textbf{variável aleatória bidimensional} é o \textbf{vetor aleatório}.
  Analogamente, se \(X_{1} = X_{1}(\omega ), X_{2} = X_{2}(\omega ), \dotsc , X_{n} = X_{n}(\omega )\) são n variáveis aleatórias, uma \textbf{variável aleatória n-dimensional}
  é um \textbf{vetor aleatório n-dimensional} \((X_{1}, \dotsc , X_{n}). \quad \square\)
\end{def*}
A natureza funcional de \(X(\omega )\) e \(Y(\omega )\) continua sendo omitida, tal que escrevemos apenas 
\[
  \mathbb{P}(X \leq a, Y\leq b)
\]
ao invés de 
\[
  \mathbb{P}(X(\omega )\leq a, Y(\omega )\leq b).
\]
Sendo assim, o contradomínio de \((X, Y)\) é o conjunto todos os possíveis valores de (X, Y) em um subconjunto do plano euclidiano, \(\mathbb{R}_{X\times Y}\).
\begin{def*}
  O par (X, Y) será uma \textbf{variável aleatória discreta bidimensional} se os valores de (X, Y) forem finitos ou infinitos enumeráveis (números naturais). Isso é naturalmente generalizado para n-vetores
  aleatórios discretos ao pedir que cada uma delas seja finita ou infinita enumerável. \(\quad \square\)
\end{def*}
\begin{def*}
  O par (X, Y) será uma \textbf{variável aleatória contínua bidimensional} se os valores de (X, Y) forem contínuos. Isso é generalizado para n-vetores
  aleatórios discretos ao pedir que cada uma delas seja contínua. \(\quad \square\)
\end{def*}
Pode ocorrer que um ou mais dos componentes do vetor aleatório \((X_{1}, \dotsc , X_{n})\) seja discreto, enquanto os outros sejam contínuos, e vice-versa.
\begin{def*}
  Considere o vetor aleatório discreto e n-dimensional \((X_{1}, \dotsc , X_{n})\). A cada possível resultado \((x_{1}, \dotsc , x_{n})\) (em que \(X_{i},\) com \(i=1, \dotsc , n\), assume valores possível \(x_{1}, x_{2}, \dotsc \)), associamos um número 
  \[
    p(x_{1}, \dotsc , x_{n}) = \mathbb{P}(X_{1} = x_{1}, \dotsc , X_{n}=x_{n}),
  \]
  que satisfaz as condições 
  \begin{itemize}
    \item[i)] \(p(x_{1}, \dotsc , x_{n}) \geq 0, 1 \leq i \leq n\)
    \item[ii)] \(\sum\limits_{k_{1} = 1}^{\infty}\dotsc \sum\limits_{k_{n}=1}^{\infty}p(x_{1}, \dotsc , x_{n}) = 1\)
  \end{itemize}
\end{def*}
\begin{example}
  Duas linhas de produção fabricam peças com uma capacidade em um dia de 5 e 3 peças nal inha I e II, respect.

  O vetor aleatório bidimensional represnta o número de preças produzidas pelas linhas I e II, respec. A função massa de probabilididade conjunta
  de (X, Y) é dada por
  \begin{center}
    \begin{table}[h!]
      \centering
      \begin{tabular}{| c | c | c | c | c | c | c |}
        \hline
        y\(\setminus\)x & 0 & 1 & 2 & 3 & 4 & 5\\
        \hline
        0 & 0 & 0,01 & 0,03 & 0,05 & 0,07 & 0,09\\
        1 & 0,01 & 0,02 & 0,04 & 0,05 & 0,06 & 0,08\\
        2 & 0,01 & 0,03 & 0,05 & 0,05 & 0,05 & 0,06\\
        3 & 0,01 & 0,02 & 0,04 & 0,06 & 0,06 & 0,05\\
        \hline
      \end{tabular}
    \end{table}
  \end{center}
  Podemos ter interesse em saber a probabilidade da linha I produzir 2 e linha II produzir 3. Neste caso, 
  \[
    p(2, 3) = \mathbb{P}(X=2, Y = 3) = 0,04.
  \]
  Se definirmos 
  \[
    B = \{\text{Mais peças são produzidas pela linha I do que pela linha II}\},
  \]
  então estamos pegando a diagonal triangular superior da tabela, tal que 
  \[
    \mathbb{P}(B) = 0,75.
  \]
\end{example}

\begin{def*}
  Considere o vetor aleatório contínuo e n-dimensional \((X_{1}, \dotsc , X_{n})\). A cada possível resultado \((x_{1}, \dotsc , x_{n})\) (em que \(X_{i},\) com \(i=1, \dotsc , n\), assume valores possível \(x_{1}, x_{2}, \dotsc \)), associamos um número 
  \[
    p(x_{1}, \dotsc , x_{n}) = \mathbb{P}(X_{1} = x_{1}, \dotsc , X_{n}=x_{n}),
  \]
  que satisfaz as condições 
  \begin{itemize}
    \item[i)] \(p(x_{1}, \dotsc , x_{n}) \geq 0,\) para todo \((x_{1}, \dotsc , x_{n})\in \mathbb{R}^{n}\)
    \item[ii)] \(\int\limits_{-\infty}^{\infty}\dotsc \int\limits_{-\infty}^{\infty}f(x_{1}, \dotsc , x_{n})dx_{1}\dotsc dx_{n} = 1\)
  \end{itemize}
\end{def*}
\begin{example}
  Um fabricante de lâmpadas está interessado no número de lâmpadas encoendadas durante o mês de Janeiro e Fevereiro. Considere o par aleatório (X, Y) que representa o número de
  lâmpadas encomendadas durante esses dois meses. (X, Y) terá a função densidade de probabilidade conjunta dada por 
  \[
    f(x, y)  = \left\{\begin{array}{ll}
        c\quad 5000 \leq x \leq 10.000\\ 
        c\quad 4000 \leq y\leq 9000\\
        0\quad \text{caso contrário.}
    \end{array}\right.
  \]
  Determinamos c tendo em mente que a integral é 1. Temos:
  \[
    \int_{5000}^{10000}\int_{4000}^{9000}cdydx = 1 \Rightarrow c(5000)\int_{5000}^{10000}xdx = 1
  \]
  Logo, 
  \[
    c \cdot 5000^{2} = 1 \Rightarrow c = \frac{1}{5000^{2}}
  \]
  Se definirmos \(B = \{X \leq Y\}\) e \(\overline{B} = \{X \geq Y\}\), podemos calcular a probabilidade de vender mais em 
  Janeiro do que em Fevereiro por meio de qualquer um dos conjuntos, ou seja, temos a liberdade de escolher a região mais fácil de integração,
  tal como visto em Cálculo II. Com isso, segue que 
  \begin{align*}
    \mathbb{P}(B) &= 1 - \mathbb{P}(\overline{B}) \\
                  &= 1 - \int_{4.000}^{9.000} \int_{5.000}^{y}(5.000)^{-2}dxdy\\
                  &= 1 - \int_{4.000}^{9.000} (5000)^{-2}(y - 5.000)dy\\
                  &= 1 - (5.000)^{-2}\biggl[\frac{y^{2}}{2} - 5.000-y\biggr]\biggl|_{4.000}^{9.000}\biggr.\\
                  &= 1 - (5.000)^{-2}\biggl[\frac{9.000^{2}}{2} - 45.000.000 - \frac{4.000^{2}}{2} + 20.000\biggr]\\
                  &= 1 - \frac{97.500.00}{25.000.000} \approx 1 - 0.32 \approx 0.68 \approx 68\%.
  \end{align*}
\end{example}
\begin{example}
  Suponhamos que a variável aleatória contínua bidimensional (X, Y) tenha função densidade de probabilidade conjunta dada por 
  \[
    f(x, y) = \left\{\begin{array}{ll}
        x^{2} + \frac{xy}{3},\quad 0 \leq x\leq 1\\
        x^{2} + \frac{xy}{3},\quad 0\leq y\leq 2\\
        0,\quad \text{caso contrário.}
    \end{array}\right.
  \]
  Primeiramente, note que 
  \begin{align*}
    \int_{0}^{1}\int_{0}^{2}x^{2}+\frac{xy}{3}dydx &= \int_{0}^{1}\biggl[x^{2}y + \frac{xy^{2}}{6}\biggr]\biggl|_{0}^{2}\biggr.dx\\
                                                   &= \int_{0}^{1}2x^{2}+\frac{2x}{3}dx \\
                                                   &= \biggl(\frac{2x^{3}}{3}+\frac{x^{2}}{3}\biggr)\biggl|_{0}^{1}\biggr.\\
                                                   &= 1.
  \end{align*}
  Como podemos descobrir a probabilidade de obter \(X + Y \geq 1\), ou seja, o conjunto \(B = \{(X, Y): X + Y \geq 1\}?\)

  Para isso, calculamos
  \begin{align*}
    \int_{0}^{1}\int_{1 - x}^{2}\biggl(x^{2} + \frac{xy}{3}\biggr)dydx &= \int_{0}^{1}\biggl(x^{2}(2-1+x) + \frac{x}{6}(2^{2}-(1-x)^{2}\biggr)dx\\
                                                                       &= \int_{0}^{1}x^{2} + x^{3} + \frac{x}{6}(4-1+2x-x^{2})dx\\
                                                                       &= \int_{0}^{1}\frac{5}{6}x^{3} + \frac{4}{3}x^{2} + \frac{1}{2}xdx\\
                                                                       &= \biggl[\frac{5}{24}x^{4} + \frac{4x^{3}}{9} + \frac{x^{2}}{4}\biggr]\biggl|_{0}^{1}\biggr.\\
                                                                       &= \frac{5}{24} + \frac{4}{9} + \frac{1}{4} = \frac{65}{72}\approx 0,9
  \end{align*}
\end{example}
\begin{def*}
  Considere o vetor aleatório n-dimensional \((X_{1}, \dotsc , X_{n})\). A função de distribuição, F, é definida por 
  \[
    F(x_{1}, \dotsc , x_{n}) = \mathbb{P}(X_{1} \leq x_{1},\dotsc ,X_{n}\leq x_{n}).
  \]
  Assim, a n-ésima derivada na variável n retorna a densidade de probabilidade:
  \[
    \frac{\partial^{n}}{\partial x_{1}\dotsc \partial x_{n}} F(x_{1},\dotsc ,x_{n})\footnotemark[2] = f(x_{1}, \dotsc , x_{n}).\quad \square
  \]
  \footnotetext[2]{Vale lembrar essa definição: A n-ésima derivada parcial com relação a cada variável é dada por \[\frac{\partial^{n}}{\partial x_{1}\dotsc \partial x_{n}} = \frac{\partial^{}}{\partial x_{1}^{}}\biggl(\frac{\partial^{n-1}}{\partial x_{2}\dotsc \partial x_{n}}\biggr).\]
  Por exemplo, \[\frac{\partial^{2}}{\partial x \partial y}f(x, y) = \frac{\partial^{}}{\partial x^{}}\biggl(\frac{\partial^{}}{\partial y^{}}f(x, y)\biggr).\]}
\end{def*}
A cada variável aleatória n-dimensional \((X_{1}, \dotsc , X_{n})\), a soma por linha e por coluna delas fornece a distribuição marginal/individual pra cada 
variável. Este nome vem pois normalmente é um valor representado na margem de uma tabela contendo as probabilidade. Para a tabela de antes, por exemplo, a distribuição
marginal da probabilidade \(Y = 1\) e \(X = 3\) serão, respectivamente, 
\[
  \mathbb{P}(Y=1) = 0,08 + 0,06 + 0,05 + 0,04 + 0,02 + 0,01 = 0,26\quad\&\quad \mathbb{P}(X = 3) = 0,21.
\]

Para o caso n-dimensional, 
\[
  p(x_{i}) = \mathbb{P}(X_{i} = x_{i}) = \sum\limits_{k_{1}=1}^{\infty}\dotsc \sum\limits_{k_{i-1}=1}^{\infty}\sum\limits_{k_{i+1}=1}^{\infty}\dotsc \sum\limits_{k_{n}=1}^{\infty}p(x_{k_{1}}, \dotsc , x_{k_{i-1}},x_{k_{i+1}},\dotsc ,x_{k_{n}})
\]
Essas funções correspondem à função básica da variável aleatória unidimensional \(X_{i}\). Por exemplo, 
\begin{align*}
  \mathbb{P}(x \leq X_{i}\leq d) &= \mathbb{P}(-\infty<X_{1}<\infty, \dotsc , c \leq X_{i}\leq d, \dotsc , -\infty<X_{n} < \infty)\\
                                 &= \underbrace{\int_{-\infty}^{\infty}\dotsc \int_{-\infty}^{\infty}}_{\text{n-1 vezes}}f(x_{1},\dotsc ,x_{i-1},x_{i},x_{i+1},\dotsc ,x_{n})dx_{1}\dotsc dx_{i-1}dx_{i+1}\dotsc dx_{n}.
\end{align*}
\begin{example}
  Duas características do desempenho de um motor de um foguete são o empuxo (X) e a taxa de mistura (Y). Suponha que (X, Y) seja um vetor aleatório bidimensional com 
  função de densidade de probabilidade 
  \[
    f(x, y)  = \left\{\begin{array}{ll}
        2(x+y-2xy),\quad 0 \leq x \leq 1\\
        2(x+y-2xy),\quad 0 \leq y \leq 1\\
        0,\quad \text{caso contrário}.
    \end{array}\right.
  \]
\end{example}

\textbf{[Continua...]}.
\end{document}

\documentclass[./probability_notes.tex]{subfiles}
\begin{document}
\section{Aula 03 - 31/08/2023}
\subsection{Motivações}
\begin{itemize}
	\item Probabilidade Condicional;
	\item Dependência de Eventos.
\end{itemize}
\subsection{A Probabilidade Condicional}
\begin{def*}
	Sejam dois eventos A e B associados ao mesmo espaço amostral \(\Omega \). A probabilidade condicional de A
	dado que ocorreu B é representada por \(\mathbb{P}(A|B\) e dada por
	\[
		\mathbb{P}(A|B) = \frac{\mathbb{P}(A\cap B)}{\mathbb{P}(B)},\quad \mathbb{P}(B) > 0\quad \square.
	\]
\end{def*}
Em particular, seguem duas representações para a probabilidade de dois eventos ocorrerem simultaneamente, sendo elas
\begin{align*}
	 & \mathbb{P}(A|B) = \frac{\mathbb{P}(A\cap B)}{\mathbb{P}(B)} \Rightarrow \mathbb{P}(A\cap B) = \mathbb{P}(A|B)\mathbb{P}(B) \\
	 & \mathbb{P}(B|A) = \frac{\mathbb{P}(A\cap B)}{\mathbb{P}(A)} \Rightarrow \mathbb{P}(A\cap B) = \mathbb{P}(B|A)\mathbb{P}(A) \\
\end{align*}
\begin{example}
	Um dado de seis faces, equilibrado, é lançado e o número voltado para cima é obsecrado.
	\begin{itemize}
		\item[(a)] Se o resultado obtido for par, qual a probabilidade dele ser maior ou igual a 5?

		\item[(b)] Se o resultado obtido for maior ou igual a 5, qual a probabilidade dele ser par?
	\end{itemize}

	Para o item a, o espaço amostral é \(\Omega = \{1, 2, 3, 4, 5, 6\}\). Considere os eventos
	A como o resultado de ser par e B o resultado obtido sendo maior ou igual a 5. Então,
	\[
		\mathbb{P}(B|A) = \frac{\mathbb{P}(A\cap B)}{\mathbb{P}(A)} = \frac{\mathbb{P}(\{6\})}{\mathbb{P}(\{2, 4, 6\})} = \frac{1}{3};
	\]

	Para o item b, temos
	\[
		\mathbb{P}(A|B) = \frac{\mathbb{P}(A\cap B)}{\mathbb{P}(B)} = \frac{\mathbb{P}(\{6\})}{\mathbb{P}(\{5, 6\})} = \frac{\frac{1}{6}}{\frac{2}{6}} = \frac{1}{2}.
	\]
\end{example}
\begin{def*}
	Os eventos A e B são independentes se a informação da ocorrência de B não altera a probabilidade atribuída ao evento A, isto é,
	\[
		\mathbb{P}(A|B) = \mathbb{P}(A),
	\]
	ou, equivalentemente,
	\[
		\mathbb{P}(A\cap B) = \mathbb{P}(A)\mathbb{P}(B).\quad\square
	\]
\end{def*}
\begin{example}
	Uma moeda é viciada, de modo que a chance de sair cara é o dobro da de sair coroa.
	\begin{itemize}
		\item[(a)] Dê o espaço amostral

		\item[(b)] Calcule a probabilidade de ocorrer cara no lançamento desta moeda.
	\end{itemize}

	(a) O espaço amostral desse evento é \(\Omega = \{\text{Cara, Coroa}\}\). Seja A o evento que cai cara e B o que cai coroa.

	(b) Sabemos que \(\mathbb{P}(A) + \mathbb{P}(B) = 1\) e que \(\mathbb{P}(A) = 2 \mathbb{P}(B)\), ou seja,
	\[
		\mathbb{P}(A) + \frac{\mathbb{P}(A)}{2} = 1 \Rightarrow \mathbb{P}(A) = \frac{2}{3}.
	\]
\end{example}
\begin{example}
	Duas lâmpadas queimadas foram acidentalmente misturadas com seis boas. Se vamos testar as lâmpadas, uma por uma, até
	encontrar duas defeituosas, qual é a probabilidade de que a última defeituosa seja encontrada no quarto teste?

	Estamos interessados em calcular a probabilidade do seguinte evento:
	\begin{align*}
		(\overline{D_{1}}\cap \overline{D_{2}}\cap D_{3}\cap D_{4})\cup(\overline{D_{1}}\cap D_{2}\cap \overline{D_{3}}\cap D_{4})\cup(D_{1}\cap \overline{D_{2}}\cap \overline{D_{3}}\cap D_{4})                                   \\
		 & \Rightarrow \mathbb{P}(\overline{D_{1}}\cap \overline{D_{2}}\cap D_{3}\cap D_{4})\cup(\overline{D_{1}}\cap D_{2}\cap \overline{D_{3}}\cap D_{4})\cup(D_{1}\cap \overline{D_{2}}\cap \overline{D_{3}}\cap D_{4})          \\
		 & = \mathbb{P}(\overline{D_{1}}\cap \overline{D_{2}}\cap D_{3}\cap D_{4}) + \mathbb{P}(\overline{D_{1}}\cap D_{2}\cap \overline{D_{3}}\cap D_{4}) + \mathbb{P}(D_{1}\cap \overline{D_{2}}\cap \overline{D_{3}}\cap D_{4}).
	\end{align*}
	Após manipulações algébricas e contas, chegamos em
	\[
		\frac{1}{28} + \frac{1}{28} + \frac{1}{28} = \frac{3}{28}\approx 0,1071.
	\]
\end{example}
\begin{def*}
	Dizemos que \(A_{1}, A_{2}, \cdots, A_{n}\) formam uma partição de \(\Omega \) se eles são dois-a-dois disjuntos e a sua união é \(\Omega.\quad\square \)
\end{def*}
\begin{theorem*}
	Suponha que os eventos \(A_{1}, A_{2}, \cdots, A_{n}\) formam uma partição de \(\Omega \) e que todos têm probabilidade positiva. Então, para qualquer evento B,
	\[
		\mathbb{P}(B) = \sum\limits_{i=1}^{n}\mathbb{P}(B|A_{i})\mathbb{P}(A_{i})
	\]
\end{theorem*}
\begin{example}
	Uma companhia produz circuitos integrados em três fábricas, sendo elas A, B e C. A fábrica A produz 40\% deles, e as outras produzem 30\% cada. As probabilidades de que um circuito integrado produzido por essas fábricas não funcione são
	\(0,01; 0,04; 0,03\) respectivamente. Escolhido um circuito da produção conjunta das três fábricas, qual a probabilidade dele não funcionar?

	Sendo D o evento em que o circuito é defeituoso e A, B, C os eventos de cada fábrica, sabemos que
	\[
		\mathbb{P}(A) = 0,40,\quad \mathbb{P}(B) = 0,30,\quad \mathbb{P}(C) = 0,30.
	\]
	Além disso, sabemos que
	\[
		\mathbb{P}(D|A) = 0,02,\quad \mathbb{P}(D|B) = 0,04,\quad \& \mathbb{P}(D|C) = 0,03
	\]
	Segue do teorema que
	\[
		\mathbb{P}(D) = \sum\limits_{i=1}^{n}\mathbb{P}(D|A_{i})\mathbb{P}(A_{i}) = 0,025.
	\]

	Determine a probabilidade do defeituoso ter sido produzido pela empresa A.
	\[
		\mathbb{P}(A|D) = \frac{\mathbb{P}(A\cap D)}{\mathbb{P}(D)} = \frac{\mathbb{P}(D|A)\mathbb{P}(A)}{\mathbb{P}(D)} = \frac{0,01\times 0,4}{0,025} = \frac{0,004}{0,025} = 0,16.
	\]
\end{example}
\end{document}

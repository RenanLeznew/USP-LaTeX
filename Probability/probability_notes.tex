\documentclass{article}
\usepackage{amsmath}
\usepackage{xcolor}
\usepackage{amsthm}
\usepackage{amssymb}
\usepackage{pgfplots}
\usepackage[utf8]{inputenc}
\usepackage{amsfonts}
\usepackage[margin=2.5cm]{geometry}
\usepackage{graphicx}
\usepackage[export]{adjustbox}
\usepackage{fancyhdr}
\usepackage[portuguese]{babel}
\usepackage{hyperref}
\usepackage{lastpage}
\usepackage{mathtools}
\usepackage[light, math]{iwona}
\usepackage[T1]{fontenc}
\usepackage{subfiles}
\setcounter{section}{-1}

\pagestyle{fancy}
\fancyhf{}

\pgfplotsset{compat = 1.18}

\tikzset{>=latex} % for LaTeX arrow head
\usepackage[outline]{contour} % halo around text
\contourlength{1.2pt}
\usetikzlibrary{positioning,calc}
\usetikzlibrary{backgrounds}% required for 'inner frame sep'
\pgfmathdeclarefunction{gauss}{3}{%
	\pgfmathparse{1/(#3*sqrt(2*pi))*exp(-((#1-#2)^2)/(2*#3^2))}%
}
\pgfmathdeclarefunction{cdf}{3}{%
	\pgfmathparse{1/(1+exp(-0.07056*((#1-#2)/#3)^3 - 1.5976*(#1-#2)/#3))}%
}
\pgfmathdeclarefunction{fq}{3}{%
	\pgfmathparse{1/(sqrt(2*pi*#1))*exp(-(sqrt(#1)-#2/#3)^2/2)}%
}
\pgfmathdeclarefunction{fq0}{1}{%
	\pgfmathparse{1/(sqrt(2*pi*#1))*exp(-#1/2))}%
}

\colorlet{mydarkblue}{blue!30!black}
\usepgfplotslibrary{fillbetween}
\usetikzlibrary{patterns}

\def\N{50}
\hypersetup{
	colorlinks,
	citecolor=black,
	filecolor=black,
	linkcolor=black,
	urlcolor=black
}
\newtheorem*{def*}{\underline{Definição}}
\newtheorem*{theorem*}{\underline{Teorema}}
\newtheorem*{lemma*}{\underline{Lema}}
\newtheorem*{prop*}{\underline{Proposição}}
\newtheorem{example}{\underline{Exemplo}}
\newtheorem*{proof*}{\underline{Prova}}
\renewcommand\qedsymbol{$\blacksquare$}

\rfoot{P\'agina \thepage \hspace{1pt} de \pageref{LastPage}}

\begin{document}
\begin{figure}[ht]
	\minipage{0.76\textwidth}
	\includegraphics[width=4cm]{icmc.png}
	\hspace{7cm}
	\includegraphics[height=4.9cm,width=4cm]{brasao_usp_cor.jpg}
	\endminipage
\end{figure}

\begin{center}
	\vspace{1cm}
	\LARGE
	UNIVERSIDADE DE S\~AO PAULO

	\vspace{1.3cm}
	\LARGE
	INSTITUTO DE CI\^ENCIAS MATEM\'ATICAS E COMPUTACIONAIS - ICMC

	\vspace{1.7cm}
	\Large
	\textbf{Introdução à Probabilidade}

	\vspace{1.3cm}
	\large
	\textbf{Renan Wenzel - 11169472}

	\vspace{1.3cm}
	\large
	\textbf{Professor: Oilson Alberto Gonzatto Junior}

	\textbf{E-mail: oilson.agjr@icmc.usp.br}

	\vspace{5cm}
	\today
\end{center}
\newpage

\tableofcontents

\newpage

\subfile{Probability/classes/aula01}
\newpage

\subfile{Probability/classes/aula02}
\newpage

\subfile{Probability/classes/aula03}
\newpage

\subfile{Probability/classes/aula04}
\newpage

\subfile{Probability/classes/aula05}
\newpage

\subfile{Probability/classes/aula06}
\newpage

\subfile{Probability/classes/aula07}
\newpage

\subfile{Probability/classes/aula08}
\newpage

\subfile{Probability/classes/aula09}
\newpage

\subfile{Probability/classes/aula10}
\newpage

\subfile{Probability/classes/aula11}
\newpage

\subfile{Probability/classes/aula12}
\newpage

\subfile{Probability/classes/aula13}
\newpage

\subfile{Probability/classes/aula14}
\newpage

\subfile{Probability/classes/aula15}
\newpage

\subfile{Probability/classes/aula16}
\newpage

\subfile{Probability/classes/aula17}
\newpage

\subfile{Probability/classes/aula18}
\end{document}

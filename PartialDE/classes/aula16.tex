\documentclass[../pde_notes.tex]{subfiles}
\begin{document}
\section{Aula 16 - 07 de Maio, 2025}
\subsection{Motivações}
\begin{itemize}
	\item Calor e Onda em Mais Dimensões;
	\item Transição para o Estudo da Equação de Laplace.
\end{itemize}
\subsection{Equação de Laplace - O Começo a partir do que Temos.}
Até o momento, estudamos as equações da onda e do calor com a variável ``espaço'' (x) em uma única dimensão. A ideia aqui é explorarmos um pouco mais do que acontece quando aumentamos, e manteremos em \(\mathbb{R}^{3}\). Assim, considere um
aberto \(\Omega \subseteq \mathbb{R}^{n}\), com \(n=1, 2, 3\) e o problema
\[
	\left\{\begin{array}{ll}
		\frac{\partial^{}u}{\partial t^{}} = \Delta u, & \quad t > 0, x\in \Omega   \\
		u = 0,                                         & \quad x\in \partial \Omega \\
		u(x, 0) = u_{0}(x),                            & \quad x\in \Omega
	\end{array}\right..
\]
Usando a solução normalmente encontrada no passo 1, concluímos que, para \(x=(x_1, x_2, \dotsc , x_{n})\) em um conjunto \(\Omega \) de \(\mathbb{R}^{n}\),
\[
	u(x, t) = X(x)T(t) \Rightarrow \frac{X''(x)}{X(x)}=\frac{T'(t)}{T(t)}\dotsc ,
\]
mas não. Aqui, não faz sentido falar na derivada de X em x, porque x é uma variável em \(\mathbb{R}^{n}\)! A forma que encontramos, na verdade,
\[
	u(x, t) = X(x)T(t) \Rightarrow \frac{\Delta X(x)}{X(x)} = \frac{T'(t)}{T(t)} = \lambda .
\]
O processo da EDO é o mesmo, mas temos que levar em conta a multidimensionalidade da X!

O passo 2 também muda: partindo de que \(u(x, t) = 0\) quando x é um ponto na fronteira de \(\Omega \), segue que \(X(x) = 0 \) na fronteira também. Assim,
\begin{align*}
	 & \Delta X(x) = \lambda X(x), \quad x\in \Omega \\
	 & X(x) = 0, \quad x\in \partial \Omega          \\
	 & T''(t) = \lambda T(t).
\end{align*}

Para a onda, a ideia é parecida:
\[
	\left\{\begin{array}{ll}
		\frac{\partial^{2}u}{\partial t^{2}} = \Delta u,                           & \quad t > 0, x\in \Omega   \\
		u = 0,                                                                     & \quad x\in \partial \Omega \\
		u(x, 0) = u_{0}(x),\; \frac{\partial^{}u}{\partial t^{}}(x, 0) = v_{0}(x), & \quad x\in \Omega
	\end{array}\right..
\]
tal que o Passo 1 fornece
\[
	u(x,t) = T(t)X(x) \Rightarrow \frac{\Delta X(x)}{X(x)} = \frac{T''(t)}{T(t)} = \lambda ,
\]
e obtemos a EDO no Passo 2
\begin{align*}
	 & \Delta X(x) = \lambda X(x), \quad x\in \Omega \\
	 & X(x) = 0, \quad x\in \partial \Omega          \\
	 & T''(t) = \lambda T(t).
\end{align*}

Ambos os passos 2 das duas equações acabam lidando com o problema de autovalores do Laplaciano com condições de Dirichlet, que pode ser atacado por meio do teorema abaixo
\begin{theorem*}
	Se \(\Omega \) for um subconjunto aberto, convexo, limitado e com fronteira regular de \(\mathbb{R}^{n}\), então existe uma sequência de funções \(\{\varphi_{n}\}_{n\in \mathbb{N}}\) de classe \(\mathcal{C}^{\infty}(\overline{\Omega })\) e uma sequência de escalares \(\{\lambda_{n}\}_{n\in \mathbb{N}}\) tais que
	\begin{align*}
		 & 1) \int_{\Omega }^{}\varphi_{n}\varphi_{m} \mathrm{dx} = \delta_{nm};                                                                     \\
		 & 2) \Delta \varphi_{n} = \lambda_{n}\varphi_{n},\quad \varphi_{n}|_{\partial \Omega } = 0;                                                 \\
		 & 3) \lambda_{n} < 0, \quad \lambda_{n}\overbracket[0pt]{\longrightarrow}^{n\to \infty}-\infty                                              \\
		 & 4) \forall u_{0}\in \mathcal{C}(\overline{\Omega }),\quad u_{0} = \sum\limits_{n=1}^{\infty}\left< u_{0}, \varphi_{n} \right>\varphi_{n}.
	\end{align*}
\end{theorem*}
Note que a expressão para \(u_{0}\) no item 4 pode ser reescrita, usando os outros itens, como
\[
	\int_{\Omega }^{}\biggl\vert u_{0} - \sum\limits_{n=1}^{N}\left< u_{0}, \varphi_{n} \right>\varphi_{n} \biggr\vert^{2} \mathrm{dx}\overbracket[0pt]{\longrightarrow}^{N\to \infty}0.
\]

\begin{example}
	Considere o caso em que n vale 1 e \(\Omega \) é o intervalo aberto \(\Omega = (0, \pi )\). Aqui, tomemos
	\[
		\varphi_{n}(x) = \sqrt[]{\frac{2}{\pi }}\sin^{}{(nx)},\quad \lambda_{n} = -n^{2}
	\]
	e
	\[
		u_{0} = \sum\limits_{n=1}^{\infty}\left< u_{0}, \varphi_{n} \right>\varphi_{n} = \sum\limits_{n=1}^{\infty}\biggl(\frac{2}{\pi }\int_{0}^{\pi }u_{0}(y)\sin^{}{(ny)} \mathrm{dy}\biggr)\sin^{}{(nx)},
	\]
	que nada mais é que a série seno! Com isso, quando atacamos o passo da condição inicial no calor e na onda, conseguimos
	\begin{align*}
		 & u(x,t) = \sum\limits_{n=1}^{\infty}T_{n}(t)X_{n}(x) = \sum\limits_{n=1}^{\infty}c_{n}e^{\lambda_{n}t}\varphi_{n}(x)                                                                                                               \\
		 & u_{0}(x) = u(x, 0) = \sum\limits_{n=1}^{\infty}c_{n}\varphi_{n}(x) \Rightarrow c_{n}=\left< u_{0}, \varphi_{n} \right> \Rightarrow u(x,t)=\sum\limits_{n=1}^{\infty}\left< u_{0}, \varphi_{n} \right>e^{\lambda_{n}t}\varphi_{n}.
	\end{align*}

	Para a onda, também, obtemos a solução com esse mesmo processo:
	\begin{align*}
		 & u(x,t) = \sum\limits_{n=1}^{\infty}T_{n}(t)X_{n}(x) = \sum\limits_{n=1}^{\infty}[b_{n}\sin^{}{(\sqrt[]{-\lambda_{n}}t)} + a_{n}\cos^{}{(\sqrt[]{-\lambda_{n}}t)}]\varphi_{n}                                                       \\
		 & u_{0}(x) = \sum\limits_{n=1}^{\infty}a_{n}\varphi_{n} \Rightarrow a_{n} = \left< u_{0}, \varphi_{n} \right>                                                                                                                        \\
		 & v_{0}(x) = \frac{\partial^{}u}{\partial t^{}}(x, 0) = \sum\limits_{n=1}^{\infty}b_{n}\sqrt[]{-\lambda_{n}}\varphi_{n} \Rightarrow b_{n} = \frac{1}{\sqrt[]{-\lambda_{n}}}\left< v_{0}, \varphi_{n} \right>                         \\
		 & u(x,t) = \sum\limits_{n=1}^{\infty}\biggl[\left< u_{0}, \varphi_{n} \right>\cos^{}{(\sqrt[]{-\lambda_{n}}t)} + \frac{1}{\sqrt[]{-\lambda_{n}}}\left<v_{0}, \varphi_{n} \right>\sin^{}{(\sqrt[]{-\lambda_{n}}t)}\biggr]\varphi_{n}.
	\end{align*}
\end{example}

O que aprendemos com este exemplo é que já temos um método para resolver as EDPs que não depende tanto da dimensão. Testaremos ele com o seguinte problema de autovalores em \(n=2\): seja \(\Omega \) o retângulo de base \(L_1\) e altura \(L_2\) encostado na origem e considere a EDP
\[
	\left\{\begin{array}{ll}
		\frac{\partial^{2}u}{\partial x^{2}}(x,y) + \frac{\partial^{2}u}{\partial y^{2}}(x,y) = \lambda u(x, y), & \quad x\in(0, L_1),\; y\in(0, L_2)    \\
		u(0, y) = u(L_1, y) = u(x, 0) = u(x, L_2) = 0,                                                           & \quad x, y\in (0, L_1)\times (0, L_2)
	\end{array}\right..
\]
Temos que determinar a função u e o escalar \(\lambda \).

Nosso passo 1 será considerar a solução \(u(x, y) = X(x) Y(y)\), o que indica que
\[
	X''(x)Y(y) + X(x)Y''(y) = \lambda X(x)Y(y).
\]
Dividindo tudo por \(X(x)Y(y)\), temos a equação
\[
	\frac{X''(x)}{X(x)} = \frac{Y''(y)}{Y(y)} = \lambda ,
\]
tal que isto torna-se o conjunto de equações
\[
	\left\{\begin{array}{ll}
		X''(x) = \lambda_1 X(x) \\
		Y''(x) = \lambda_2 Y(y) \\
		\lambda_1 + \lambda_2 = \lambda .
	\end{array}\right.
\]

Em seguida, no passo 2, usamos as condições de contorno para obtermos
\[
	u(0, y) = u(L_1, y) = 0 \Leftrightarrow X(0)Y(y) = X(L_1)Y(y) = 0 \Leftrightarrow X(0) = X(L_1) = 0
\]
e, analogamente,
\[
	u(x, 0) = u(x, L_2) = 0 \Leftrightarrow X(x)Y(0) = X(x)Y(0) = 0 \Leftrightarrow Y(0) = Y(L_2) = 0,
\]
que nos fornece as soluções para as EDOs dadas por
\begin{align*}
	 & X''(x) = \lambda_1 X(x),\; X(0) = X(L_1) = 0 \Rightarrow X_{n}(x) = \sin^{}{\biggl(\frac{n\pi x}{L_1}\biggr)}, \lambda_1 = -\biggl(\frac{n\pi }{L_1}\biggr)^{2}  \\
	 & Y''(x) = \lambda_2 Y(x),\; Y(0) = Y(L_2) = 0 \Rightarrow Y_{n}(y) = \sin^{}{\biggl(\frac{n\pi y}{L_2}\biggr)}, \lambda_1 = -\biggl(\frac{n\pi }{L_2}\biggr)^{2},
\end{align*}
tal que as soluções da EDP terão a cara
\begin{align*}
	\varphi_{mn} & (x,y) = \frac{2}{\sqrt[]{L_1L_2}}\sin^{}{\biggl(\frac{n\pi x}{L_1}\biggr)}\sin^{}{\biggl(\frac{m\pi x}{L_2}\biggr)} \\
	             & \lambda_{mn}= -\biggl[\biggl(\frac{n\pi }{L_1}\biggr)^{2}+\biggl(\frac{m\pi }{L_2}\biggr)^{2}\biggr].
\end{align*}

Vamos ver o mesmo problema quando \(\Omega \) é uma bola unitária:
\[
	\Omega = \{(x, y)\in \mathbb{R}^{2}: x^{2}+y^{2}<1\}, \quad \partial \Omega  = \{(x, y)\in \mathbb{R}^{2}: x^{2}+y^{2}=1\},
\]
com EDP
\[
	\left\{\begin{array}{ll}
		\frac{\partial^{2}u}{\partial x^{2}}(x,y) + \frac{\partial^{2}u}{\partial y^{2}}(x,y) = \lambda u(x, y), & \quad (x, y)\in \Omega          \\
		u = 0,                                                                                                   & \quad (x, y)\in \partial \Omega
	\end{array}\right..
\]

Como retornaremos a isto quando virmos as EDPs de Laplace e Poisson, por agora vamos apenas jogar as soluções com uma transformação de variáveis, aceitando que fazer isso é possível:
\[
	\left.\begin{array}{ll}
		x = r \cos^{}{(\theta )} \\
		y = r \sin^{}{(\theta )}
	\end{array}\right\} \Delta u = \frac{\partial^{2}u}{\partial r^{2}} + \frac{1}{r}\frac{\partial^{}u}{\partial r^{}} + \frac{1}{r^{2}}\frac{\partial^{2}u}{\partial \theta ^{2}},
\]
com a vantagem disso tornar a EDP original
\[
	\left\{\begin{array}{ll}
		\frac{\partial^{2}u}{\partial r^{2}} + \frac{1}{r}\frac{\partial^{}u}{\partial r^{}} + \frac{1}{r^{2}}\frac{\partial^{2}u}{\partial \theta ^{2}}, = \lambda u(r, \theta ), & \quad (r, \theta )\in [0, 1]\times [0, 2\pi ] \\
		u(1, \theta ) = 0,                                                                                                                                                         & \quad \theta \in [0, 2\pi ]
	\end{array}\right..
\]
Assim, o passo 1 torna-se definir as soluções como \(u(r, \theta ) = R(r)\Theta(\theta )\), tal que
\[
	\Theta R'' + \frac{1}{r}\Theta R' + \frac{1}{r^{2}}R\Theta '' = \lambda R\Theta.
\]
Dividindo tudo por \(\dfrac{r^{2}}{R\Theta}\), isto equivale a
\[
	\frac{r^{2}R'' + rR'}{R} + \frac{\Theta''}{\Theta} = \lambda r^{2} \Longleftrightarrow \frac{r^{2}R'' + rR'-\lambda r^{2} R}{R} + \frac{\Theta''}{\Theta} = 0.
\]
Vamos denotar por
\[
	\lambda_1\coloneqq \frac{r^{2}R'' + rR'-\lambda r^{2} R}{R} \quad\&\quad \lambda_2 \coloneqq \frac{\Theta''}{\Theta},
\]
tal que podemos escrever em forma de EDOs
\[
	r^{2}R'' + rR' - \lambda r^{2}R = \lambda_1 R \quad\&\quad \Theta'' = \lambda_2 \Theta.
\]

Com isso, podemos ir ao passo 2! Comece observando que, em geral, na coordenada polar, os pontos que correspondem ao 0 também correspondem ao \(2\pi \), que nos fornece uma condição inicial geométrica na forma
\[
	\Theta(0) = \Theta (2\pi ) \quad\&\quad \frac{\mathrm{d}\Theta}{\mathrm{d}\theta }(0) = \frac{\mathrm{d}\Theta}{\mathrm{d}\theta }(2\pi ) = 0
\]
e, em R, teremos a condição pela própria EDP:
\[
	R(1)\Theta (\theta ) = 0 \Rightarrow R(1) = 0 \quad\&\quad \lim_{r\to 0}R(r) < \infty,
\]
em que a segunda condição é pedida por parcimônia mesmo. Logo,
\[
	\Theta''(\theta ) = \lambda_{2}\Theta(\theta ),\;\&\;\Theta(0) =\Theta(2\pi ),\;\&\;\Theta'(0) =\Theta'(2\pi ),
\]
que nos dá como solução
\[
	\Theta(\theta ) = \left\{\begin{array}{ll}
		1,             & \quad \lambda_2 = 0       \\
		\sin^{}{(nx)}, & \quad \lambda_2 = -n^{2}  \\
		\cos^{}{(nx)}, & \quad \lambda_2 = -n^{2}.
	\end{array}\right..
\]

A da EDO em R é a Equação de Bessel, então vamos deixar ela pra depois, porque a cara dessa aqui é feia:
\[
	J_{v}(x) = \sum\limits_{k=0}^{\infty}\frac{(-1)^{k}}{k!\Gamma (k+r-1)}\biggl(\frac{x}{2}\biggr)^{2k+v},\quad v^{2} = \lambda_{1}.
\]
Na próxima aula, olharemos para um novo tipo de EDP, a \textit{Equação de Laplace}.

\end{document}

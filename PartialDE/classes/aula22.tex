\documentclass[../pde_notes.tex]{subfiles}
\begin{document}
\section{Aula 22 - 28 de Maio, 2025}
\subsection{Motivações}
\begin{itemize}
	\item Transformada de Fourier.
\end{itemize}
\subsection{Transformada de Fourier: propriedades.}
Dado \(f:\mathbb{R}\rightarrow \mathbb{C}\), definimos
\[
	\mathcal{F}[f(\xi )] = \hat{f}(\xi ) = \int_{-\infty}^{\infty}e^{-ix\xi }f(x) \mathrm{dx}
\]
com a Transformada de Fourier\footnote{Às vezes, será escrito \(\mathcal{F}[f]\); outras, \(\mathcal{F}f\), ou \(\mathcal{F}(f)\), ou apenas \(\hat{f}\). Todas são a mesma coisa, mas dependem um pouco (bastante) da minha preguiça de escrever.} da função f. Veremos algumas propriedades sobre elas agora. Para isso, determinemos as seguintes notações:
\begin{itemize}
	\item \(L^{1}(\mathbb{R}):\) o espaço das funções integráveis; e
	\item \(L^{\infty}(\mathbb{R}):\) o espaço das funções ``limitadas'',
\end{itemize}
de tal forma que nosso primeira propriedade possa ser vista como dizer que \(\mathcal{F}\) é uma função de \(L^{1}\) em \(L^{\infty}\).
\begin{prop*}
	Se \(f:\mathbb{R}\rightarrow \mathbb{C}\) é integrável, então \(\mathcal{F}f\) é limitada e
	\[
		\Vert \hat{f} \Vert_{L^{\infty}}\leq \Vert f \Vert_{L^{1}} \Longleftrightarrow \sup_{\xi \in \mathbb{R}}|\hat{f}(\xi )| \leq \int_{-\infty}^{\infty}|f(x)| \mathrm{dx}.
	\]
\end{prop*}
\begin{proof*}
	Dado \(\xi \) um número real qualquer, segue que
	\[
		|\hat{f}(\xi )| = \biggl\vert \int_{-\infty}^{\infty}e^{-ix\xi }f(x) \mathrm{dx} \biggr\vert\leq \int_{-\infty}^{\infty}|f(x)| \mathrm{dx}.
	\]
	Portanto, como \(\xi \) fora escolhida arbitrariamente,
	\[
		\sup_{\xi \in \mathbb{R}}|\hat{f}(\xi )| \leq \int_{-\infty}^{\infty}|f(x)| \mathrm{dx}.\quad \text{\qedsymbol}
	\]
\end{proof*}
Em segundo lugar, vemos que a Transformada de Fourier é uma transformação linear
\begin{prop*}
	A função \(\mathcal{F}:L^{1}(\mathbb{R})\rightarrow L^{\infty}(\mathbb{R})\) é uma transformação linear.
\end{prop*}
\begin{proof*}
	Com efeito, basta notar que
	\begin{align*}
		\mathcal{F}(\alpha f + \beta g)(\xi ) & =\int_{-\infty}^{\infty}e^{-ix\xi }(\alpha f(x)+\beta g(x)) \mathrm{dx}                                                \\
		                                      & = \alpha \int_{-\infty}^{\infty}e^{-ix\xi }f(x) \mathrm{dx} + \beta \int_{-\infty}^{\infty}e^{-ix\xi }g(x) \mathrm{dx} \\
		                                      & = \alpha \mathcal{F}(f) + \beta \mathcal{F}(g).
	\end{align*}
\end{proof*}
\begin{prop*}
	Abaixo, estão listadas outras propriedades úteis:
	\begin{itemize}
		\item[i)] \(\mathcal{F}(f(x-a))=e^{-ia\xi }\hat{f}(\xi )\) e \(\mathcal{F}(e^{iax}f(x))=\hat{f}(f-a)\);
		\item[ii)] Para qualquer \(a > 0,\)
		      \[
			      \mathcal{F}(f(ax)) = \frac{1}{a}\hat{f}\biggl(\frac{\xi }{a}\biggr);
		      \]
		\item[iii)](\textbf{MUITO IMPORTANTE}): se f é uma função de classe \(\mathcal{C}^{1}\) e integrável (\(f\in \mathcal{C}^{1}\cap L^{1}\)) tal que \(f(x)\) tende a 0 quando x tende a \(\pm \infty\), então a transformada de Fourier leva derivadas em polinômios!
		      \[
			      \mathcal{F}(f'(x)) = i\xi \hat{f}(\xi ).
		      \]
		\item[iv)] De forma análoga à propriedade acima, se xf e f são integráveis,
		      \[
			      \mathcal{F}(xf(x)) = i \frac{\mathrm{d}\hat{f}(\xi )}{\mathrm{d}\xi }
		      \]
	\end{itemize}
\end{prop*}
\begin{proof*}
	i): basta fazer a transformação de variáveis \(y = x-a\) e \(dy = dx\) no meio das seguintes etapas:
	\[
		\mathcal{F}(f(x-a)) = \int_{-\infty}^{\infty}e^{-ix\xi }f(x-a) \mathrm{dx} = \int_{-\infty}^{\infty}e^{-iy\xi }e^{-ia\xi }f(y) \mathrm{dy} = e^{-ia\xi }\hat{f}(\xi ).
	\]
	Por outro lado,
	\[
		\mathcal{F}(e^{iax}f(x)) = \int_{-\infty}^{\infty}e^{-ix\xi }e^{iax}f(x) \mathrm{dx} = \int_{-\infty}^{\infty}e^{-ix(\xi -a)}f(x) \mathrm{dx} = \mathcal{F}f(\xi -a).
	\]

	ii): dessa vez, transformamos \(y = ax\) e \(dy = adx\), tal que
	\[
		\mathcal{F}(f(ax)) = \int_{-\infty}^{\infty}e^{-ix\xi }f(ax) \mathrm{dx} = \frac{1}{a}\int_{-\infty}^{\infty}e^{-iy \frac{\xi }{a}}f(y) \mathrm{dx} = \frac{1}{a}\hat{f}\biggl(\frac{\xi }{a}\biggr).
	\]

	iii): aqui, faremos uma aplicação da integração por partes:
	\begin{align*}
		\mathcal{F}(f') = \int_{-\infty}^{\infty}e^{-ix\xi }f'(x) \mathrm{dx} & = e^{ix\xi }f(x)\biggl|_{-\infty}^{\infty}\biggr. - \int_{-\infty}^{\infty}(-i\xi )e^{-ix\xi }f(x) \mathrm{dx} \\
		                                                                      & = i\xi \int_{-\infty}^{\infty}e^{-ix\xi }f(x) \mathrm{dx}.
	\end{align*}

	iv): primeiramente, note que
	\[
		i \frac{\mathrm{d}}{\mathrm{d}\xi }(e^{-ix\xi }) = i(-ix)e^{-ix\xi }=xe^{-ix\xi }.
	\]
	Com isso,
	\begin{align*}
		\mathcal{F}(xf(x)) = \int_{-\infty}^{\infty}e^{-ix\xi }xf(x) \mathrm{dx} & = i \int_{-\infty}^{\infty}\frac{\mathrm{d}}{\mathrm{d}\xi }(e^{-ix\xi })f(x) \mathrm{dx} \\
		                                                                         & = i \frac{\mathrm{d}}{\mathrm{d}\xi }\int_{-\infty}^{\infty}e^{-ix\xi }f(x) \mathrm{dx}   \\
		                                                                         & = i \frac{\mathrm{d}\hat{f}(\xi )}{\mathrm{d}\xi }
	\end{align*}
\end{proof*}
Em geral, vale que, se f é de classe \(\mathcal{C}^{k}\) com todas as derivadas integráveis e com todas as derivadas convergindo a 0 conforme \(|x|\) tende a \(\infty\), então
\[
	\mathcal{F}\biggl(\frac{\mathrm{d}^{j}f}{\mathrm{d}x^{j}}\biggr)= (i\xi )^{j}\mathcal{F}(f),\quad j< k
\]
Além disso, se \(x^{j}f\) é integrável para todo \(j\leq k\),
\[
	\mathcal{F}(x^{j}f) = i^{j}\frac{\mathrm{d}^{j}\hat{f}}{\mathrm{d}\xi^{j}}(\xi ).
\]

Antes de continuar, vamos terminar o \hyperlink{last_class_21}{\textit{exemplo da aula passada}}
\begin{example}
	Como calcular a integral de
	\[
		\int_{-\infty}^{\infty}e^{-ax^{2}} \mathrm{dx},\quad a>0?
	\]
	Para isso, seja
	\[
		I \coloneqq \int_{-\infty}^{\infty}e^{-ax^{2}} \mathrm{dx} = \int_{-\infty}^{\infty}e^{-ay^{2}} \mathrm{dx},
	\]
	tal que
	\[
		I^{2} = \biggl(\int_{-\infty}^{\infty}e^{-ax^{2}} \mathrm{dx}\biggr)\biggl(\int_{-\infty}^{\infty}e^{-ay^{2}} \mathrm{dy}\biggr) = \int_{-\infty}^{\infty}\int_{-\infty}^{\infty}e^{-a(x^{2}+y^{2})} \mathrm{dx} \mathrm{dy}.
	\]
	Aqui, aplicamos a transformada
	\[
		\left.\begin{array}{ll}
			x = r\cos^{}{(\theta )} \\
			y = r\sin^{}{(\theta )}
		\end{array}\right\}dxdy = r drd\theta ,
	\]
	que torna a expressão de \(I^{2}\) em
	\begin{align*}
		I^{2} = \int_{0}^{2\pi }\int_{0}^{\infty}e^{-ar^{2}}r \mathrm{dr} \mathrm{d\theta } = 2\pi \int_{0}^{\infty}re^{-ar^{2}} \mathrm{dr} & \underbrace{=}_{\mathclap{\substack{t=ar^{2}         \\ dt = 2ardr}}} 2\pi \int_{0}^{\infty}\frac{1}{2a}e^{-t} \mathrm{dt}\\
		                                                                                                                                     & = \frac{\pi }{a}\int_{0}^{\infty}e^{-t} \mathrm{dt}  \\
		                                                                                                                                     & = \frac{\pi }{a}(-e^{-t})\biggl|_{0}^{\infty}\biggr. \\
		                                                                                                                                     & = \frac{\pi }{a}.
	\end{align*}
	Portanto,
	\[
		I^{2} = \frac{\pi }{a} \Rightarrow I = \sqrt[]{\frac{\pi }{a}},
	\]
	donde concluímos que
	\[
		\mathcal{F}(e^{-ax^{2}}) = \sqrt[]{\frac{\pi }{a}}e^{-\frac{\xi^{2}}{4a}}.
	\]
\end{example}

Nossa etapa seguinte será estudar a Transformada Inversa. Mais especificamente, se f tem transformada de Fourier dada por
\[
	\hat{f}(\xi ) = \int_{-\infty}^{\infty}e^{-ix\xi }f(x) \mathrm{dx},
\]
então
\[
	f(x) = \frac{1}{2\pi }\int_{-\infty}^{\infty}e^{ix\xi }\hat{f}(\xi ) \mathrm{d\xi }.
\]

\begin{theorem*}
	Se f é uma função de classe \(\mathcal{C}(\mathbb{R})\cap L^{1}(\mathbb{R})\) e \(\hat{f}\) é integrável, então
	\[
		f(x) = \frac{1}{2\pi }\int_{-\infty}^{\infty}e^{ix\xi }\hat{f}(\xi ) \mathrm{d\xi }
	\]
\end{theorem*}
\begin{proof*}
	Basta fazer a conta e ver que
	\begin{align*}
		\frac{1}{2\pi }\int_{-\infty}^{\infty}e^{ix\xi }\hat{f}(\xi ) \mathrm{d\xi } & = \frac{1}{2\pi }\int_{-\infty}^{\infty}e^{ix\xi }\underbrace{\biggl(\int_{-\infty}^{\infty}e^{-i\xi y}f(y) \mathrm{dy}\biggr)}_{\hat{f}(\xi )} \mathrm{d\xi } \\
		                                                                             & =\frac{1}{2\pi }\int_{-\infty}^{\infty}\biggl(\int_{-\infty}^{\infty}e^{i(x-y)\xi } \mathrm{d\xi }\biggr)f(y)\mathrm{dy},
	\end{align*}
	exceto que essa igualdade não vale! Afinal,
	\[
		(y, \xi )\mapsto e^{i(x-y)\xi }f(y)
	\]
	\textbf{não} é integrável! Como resolver isso? Com limites!
	\begin{align*}
		\frac{1}{2\pi }\int_{-\infty}^{\infty}e^{ix\xi }\hat{f}(\xi ) \mathrm{d\xi } & = \frac{1}{2\pi }\int_{-\infty}^{\infty}\lim_{\varepsilon \to 0}\biggl(e^{ix\xi }e^{-\frac{\varepsilon^{2}}{2}\xi^{2}}\hat{f}(\xi )\biggr) \mathrm{d\xi }                                      \\
		                                                                             & = \frac{1}{2\pi }\lim_{\varepsilon \to 0}\int_{-\infty}^{\infty}e^{ix\xi }e^{-\frac{\varepsilon^{2} }{2}\xi^{2}}\hat{f}(\xi ) \mathrm{d\xi }                                                   \\
		                                                                             & = \frac{1}{2\pi }\lim_{\varepsilon \to 0}\int_{-\infty}^{\infty}e^{ix\xi }e^{-\frac{\varepsilon^{2}}{2}\xi^{2}}\biggl(\int_{-\infty}^{\infty}e^{-i\xi y}f(y) \mathrm{dy}\biggr) \mathrm{d\xi } \\
		                                                                             & = \frac{1}{2\pi }\lim_{\varepsilon \to 0}\int_{-\infty}^{\infty}\biggl(\int_{-\infty}^{\infty}e^{-i(y-x)\xi }e^{-\frac{\varepsilon^{2} }{2}\xi^{2}} \mathrm{d\xi }\biggr)f(y) \mathrm{dy}.
	\end{align*}
	Note que
	\[
		\int_{-\infty}^{\infty}e^{-i(y-x)\xi }e^{-\frac{\varepsilon^{2}}{2}\xi^{2}} \mathrm{d\xi } = \mathcal{F}(e^{-\frac{\varepsilon^{2}}{2}\xi^{2}})(y-x),
	\]
	que, como vimos nas propriedades da transformada de Fourier, pode ser reescrita como
	\[
		\mathcal{F}(e^{-\frac{\varepsilon^{2}}{2}\xi ^{2}}) = \sqrt[]{\frac{2\pi }{\varepsilon^{2}}}e^{-\frac{2}{4\varepsilon^{2}}(y-x)^{2}} = \frac{2\pi }{\varepsilon }e^{-\frac{(y-x)^{2}}{2\varepsilon^{2}}}.
	\]
	Com isso, substituindo
	\[
		z = \frac{y-x}{\varepsilon } \quad\&\quad dz = \frac{1}{\varepsilon }dy,
	\]
	podemos continuar a conta acima como
	\begin{align*}
		\frac{1}{2\pi }\int_{-\infty}^{\infty}e^{ix\xi }\hat{f}(\xi ) \mathrm{d\xi } & = \frac{1}{2\pi }\lim_{\varepsilon \to 0}\int_{-\infty}^{\infty}\biggl(\int_{-\infty}^{\infty}e^{-i(y-x)\xi }e^{-\frac{\varepsilon^{2} }{2}\xi^{2}} \mathrm{d\xi }\biggr)f(y) \mathrm{dy} \\
		                                                                             & = \lim_{\varepsilon \to 0}\int_{-\infty}^{\infty}\frac{1}{\sqrt[]{2\pi }\varepsilon }e^{-\frac{(y-x)^{2}}{2\varepsilon^{2}}}f(y) \mathrm{dy}                                              \\
		                                                                             & = \lim_{\varepsilon \to 0}\frac{1}{\sqrt[]{2\pi }}\int_{-\infty}^{\infty}e^{-\frac{z^{2}}{2}}f(x-\varepsilon z) \mathrm{dz}                                                               \\
		                                                                             & = \frac{1}{\sqrt[]{2\pi }}\int_{-\infty}^{\infty}\lim_{\varepsilon \to 0}e^{-\frac{z^{2}}{2}}f(x+\varepsilon z) \mathrm{dz}                                                               \\
		                                                                             & = \biggl(\frac{1}{\sqrt[]{2\pi }}\int_{-\infty}^{\infty}e^{-\frac{z^{2}}{2}} \mathrm{dz}\biggr)f(x) = f(x). \quad \text{\qedsymbol}
	\end{align*}
\end{proof*}

Pelo Teorema acima, nossa definição da transformada de Fourier inversa como
\[
	\mathcal{F}^{-1}(g)(x) = \frac{1}{2\pi }\int_{-\infty}^{\infty}e^{ix\xi }g(\xi ) \mathrm{d\xi }
\]
ganhou sentido. Nesta linha, ganhamos a dupla de transformadas invertíveis
\begin{align*}
	 & \mathcal{F}^{-1}(\mathcal{F}(f)) = f \text{ se }f\in \mathcal{C}(\mathbb{R})\cap L^{1}(\mathbb{R})\;\&\;\hat{f}\in L^{1}(\mathbb{R})               \\
	 & \mathcal{F}(\mathcal{F}^{-1}(f)) = f \text{ se }f\in \mathcal{C}(\mathbb{R})\cap L^{1}(\mathbb{R})\;\&\; \mathcal{F}^{-1}(f)\in L^{1}(\mathbb{R}).
\end{align*}-1
\begin{example}
	Ache as transformadas de Fourier
	\[
		\mathcal{F}\biggl(\frac{1}{a^{2}+x^{2}}\biggr)\quad\&\quad \mathcal{F}\biggl(\frac{\sin^{}{(ax)}}{x}\biggr).
	\]
	Sabemos que
	\[
		\mathcal{F}(e^{-a|x|}) = \frac{2a}{a^{2}+\xi^{2}},
	\]
	tal que
	\[
		e^{-a|x|} = \mathcal{F}^{-1}\biggl(\frac{2a}{a^{2}+\xi ^{2}}\biggr).
	\]
	Sendo assim,
	\[
		\frac{1}{2\pi }\int_{-\infty}^{\infty}e^{ix\xi }\frac{2a}{a^{2}+\xi^{2}} \mathrm{d\xi } = e^{-a|x|}\underbrace{\rightsquigarrow}_{\mathclap{\substack{x\mapsto -\xi  \\ \xi \mapsto x}}} \frac{1}{2\pi }\int_{-\infty}^{\infty}e^{-ix\xi }\frac{2a}{a^{2}+x^{2}} \mathrm{dx}.
	\]
	Consequentemente,
	\[
		\mathcal{F}\biggl(\frac{1}{a^{2}+x^{2}}\biggr) = \frac{\pi }{a}e^{-a|\xi |}.
	\]

	Para a de seno, vimos na aula passada que
	\[
		\mathcal{F}(\chi_{[-a, a]}) = \frac{2\sin^{}{(a\xi )}}{\xi },
	\]
	donde concluímos que
	\[
		\chi_{[-a, a]}(x) = \mathcal{F}^{-1}\biggl(\frac{2\sin^{}{(a\xi )}}{\xi }\biggr).
	\]
	Logo,
	\begin{align*}
		                    & \frac{1}{2\pi }\int_{-\infty}^{\infty}e^{ix\xi }\frac{2\sin^{}{(a\xi )}}{\xi } \mathrm{d\xi } = \chi_{[-a, a]}(x) \\
		\Longleftrightarrow & \frac{1}{2\pi }\int_{-\infty}^{\infty}e^{-ix\xi }\frac{2\sin^{}{(ax)}}{x} \mathrm{dx} =\chi_{[-a, a]}(\xi )       \\
		\Longleftrightarrow & \int_{-\infty}^{\infty}e^{-ix\xi }\frac{\sin^{}{(ax)}}{x} \mathrm{dx} = \pi \chi_{[-a, a]}(\xi ).
	\end{align*}
	Portanto,
	\[
		\mathcal{F}\biggl(\frac{\sin^{}{(ax)}}{x}\biggr) = \pi \chi_{[-a, a]}(\xi ).
	\]

\end{example}

\end{document}

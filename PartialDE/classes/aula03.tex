\documentclass[../pde_notes.tex]{subfiles}
\begin{document}
\section{Aula 03 - 10 de Março, 2025}
\subsection{Motivações}
\begin{itemize}
	\item Continuando os exemplos
\end{itemize}
\subsection{Resolvendo a Equação do Transporte}
Começamos a aula introduzindo a noção de curva integral, a qual será utilizada durante nossa solução para a primeira EDP deste curso!
\begin{def*}
	Uma \textbf{curva integral}, ou \textbf{curva característica}, de V é uma função \(\gamma :[a, b]\rightarrow \Omega \subseteq \mathbb{R}^{n}\) que satisfaz
	\[
		\gamma'(t) = V(\gamma (t)).\quad \square
	\]
\end{def*}
\begin{example}[Equação do Transporte]
	A equação do transporte leva em consideração um aberto \(\Omega \subseteq \mathbb{R}^{n}\), ao qual x pertence, um coeficiente vetorial de n valores \(b\in \mathbb{R}^{n}\), um valor de tempo \(t\in (0, T)\) e a função \(u:\Omega \times [0, T)\rightarrow \mathbb{R}\), sendo sua forma dada por
	\[
		\frac{\partial^{}u}{\partial t^{}} + b \cdot \nabla u = f.
	\]
	Relembrando a aula anterior,
	\begin{itemize}
		\item Se f = 0, chamamos ela de equação linear homogênea;
		\item Se f = f(x) \(\neq\) 0, chamamos ela de equação linear não homogênea;
		\item Se f = f(x, u(x)), chamamos ela de equação semilinear.
	\end{itemize}
	Vamos estudar o problema
	\[
		\left\{\begin{array}{ll}
			\sum\limits_{i=1}^{n}a_{j}(x)\frac{\partial^{}u}{\partial x_{j}^{}}(x) = f(x, u(x)) \\
			u(x_{0}, 0) = g(x_{0}),
		\end{array}\right.
	\]
	em que \(x = (x_1, x_2, \dotsc , x_{n-1}, x_{n})\). A ideia para abordar esta questão é definir \textit{curvas integrais (características)}; começamos isto por meio da definição do campo vetorial
	\begin{align*}
		V: & \Omega \subseteq \mathbb{R}^{n}\rightarrow \mathbb{R}^{n}           \\
		   & V(x) = (a_1(x), \dotsc , a_{n}(x)),\quad x = (x_1, \dotsc , x_{n}).
	\end{align*}

	Assim, se V é de classe \(\mathcal{C}^{1}\), dada uma curva \(\gamma (t) = (x_1(t), \dotsc , x_{n}(t))\), então \(\gamma'(t) = V(\gamma (t))\) se, e somente se,
	\[
		\left\{\begin{array}{ll}
			x_{1}^{'}(t) = a_1(x_1(t), \dotsc , x_{n}(t)) \\
			\quad \vdots                                  \\
			x_{n}^{'}(t) = a_n(x_1(t), \dotsc , x_{n}(t)) \\
		\end{array}\right.
	\]
	Utilizando a proposição a seguir, iremos dizer os passos que seriam usados para resolver, quase como uma receita de bolo
	\begin{prop*}
		Se \(\gamma :[a, b]\rightarrow \Omega \) é uma curva integral e u é solução de uma equação semilinear, então aplicar u à curva integral também resolve uma equação semilinear, ou seja, \(v\coloneqq u\circ \gamma \) satisfaz
		\[
			\frac{dv}{dt}(t) = f(\gamma (t), v(t))
		\]
	\end{prop*}
	\begin{proof*}
		Coloque \(v = u(\gamma (t))\). Segue que, pela regra da cadeia,
		\[
			\frac{dv}{dt}(t) = \nabla u(\gamma (t)) \gamma '(t) = \sum\limits_{j=1}^{n}\frac{\partial^{}u}{\partial x_{j}^{}}(\gamma (t))x_{j}^{'}(t).
		\]
		Como \(\gamma \) é curva integral, cada \(x_{j}\) satisfaz \(x_{j}'(t) = a_{j}(\gamma (t))\), tal que
		\[
			\sum\limits_{j=1}^{n}\frac{\partial^{}u}{\partial x_{j}^{}}(\gamma (t))x_{j}^{'}(t) = \sum\limits_{j=1}^{n}\frac{\partial^{}u}{\partial x_{j}^{}}(\gamma (t))a_{j}(\gamma (t)) = f(\gamma (t), u((\gamma (t)))),
		\]
		pois u é solução da equação semilinear. Portanto, observando que o termo \(u(\gamma (t))\) é exatamente a definição de v, obtivemos
		\[
			\frac{dv}{dt}(t) = f(\gamma (t), v(t)).\quad \text{\qedsymbol}
		\]
	\end{proof*}

	Para aplicar esta proposição, o que temos é o conhecimento de um dos pontos, a condição inicial \(u(x', 0) = g(x')\), e a possibilidade de calcular v a partir do conhecimento de u ao longo da curva que passa por x', a qual denotamos \(\gamma_{x'}.\)
	A partir disso, para acharmos a solução, pegamos um ponto \((x', x_{n})\), tentamos encontrar a curva integral que passa por este ponto e calculamos v.

	Em outras palavras, isto nos leva à seguinte receita:
	\hypertarget{recipe}{\begin{itemize}
			\item[Passo 1)] Dado um vetor \(r\in \mathbb{R}^{n-1}\), calculamos \(\gamma_{r}^{'}(s)=V(\gamma_{r}(s))\), com condição inicial \(\gamma_{r}(0) = (r, 0)\);
			\item[Passo 2)] Dado um vetor \(r\in \mathbb{R}^{n-1}\), calculamos \(v_{r}^{'}(s)=f(\gamma_{r}(s), v_{r}(s))\), com condição inicial \(v_{r}(0) = g(r)\);
			\item[Passo 3)] Seja \(G:U\subseteq \mathbb{R}^{n}\rightarrow \mathbb{R}^{n}\) tal que \(G(r, s) = \gamma_{r}(s)\). A solução deve, então, ser dada por
			      \[
				      u(x) = v\circ G^{-1}(x),
			      \]
			      pois
			      \[
				      v = u\circ \gamma_{r}(s) = u\circ G(r, s) \Longleftrightarrow u = v\circ G^{-1}.
			      \]
		\end{itemize}}
\end{example}
A receita encontrada no exemplo anterior pode ser usada para outros exemplos também! Vamos vê-los.
\begin{example}
	Considere o problema de condição inicial
	\[
		\left\{\begin{array}{ll}
			x \frac{\partial^{}u}{\partial x^{}}(x, y) + 1 \frac{\partial^{}u}{\partial y^{}}(x, y) = x^{2} \\
			u(x, 0) = \sin^{}{(x)}.
		\end{array}\right.
	\]
	Seguindo os passos propostos na \hyperlink{recipe}{\textit{receita}}, o x cumpre o papel de \(a_{1}(x, y)\), o 1 cumpre de \(a_{2}(x, y)\) e, assim, podemos ir ao primeiro passo: encontrar a \(\gamma_{r}\). Note que
	\[
		x_{r}'(s) = x(s),\; x_{r}(0) = r\quad\&\quad y_{r}^{'}(s) = 1,\; y_{r}(0) = 0.
	\]
	Com isso, como são apenas EDOs comuns, podemos resolvê-las afim de encontrarmos o seguinte resultado:
	\begin{align*}
		 & x_{r}(s) = Ae^{s},\quad x_{r}(0) = r \Rightarrow x_{r}(s) = re^{s} \\
		 & y_{r}(s) = s + A,\quad y_{r}(0) = 0 \Rightarrow y_{r}(s) = s.
	\end{align*}
	Logo,
	\[
		\gamma_{r}(s) = (re^{s}, s).
	\]

	O segundo passo, o de calcular o v, é feito por meio de
	\[
		\left\{\begin{array}{ll}
			v_{r}^{'}(s) = r^{2}e^{2s} \\
			v_{r}(0) = \sin^{}{(r)}
		\end{array}\right.\Rightarrow v_{r}(s) = A + \frac{r^{2}}{2}(e^{2s}-1).
	\]
	Como \(v_{r}(0) = A\), temos \(A = \sin^{}{(r)}\) e, consequentemente,
	\[
		v_{r}(s) = \sin^{}{(r)} + \frac{r^{2}}{2}(e^{2s}-1).
	\]

	Por fim, o terceiro passo é encontrar \(G^{-1}\), que permitirá a nós resolvermos a EDP inicial. Neste sentido, temos
	\[
		G(r, s) = \gamma_{r}(s) = (re^{s}, s) = (x, y),
	\]
	ou seja, temos uma transformação de coordenadas em que s torna-se y e r assume a forma \(xe^{-y}\). Destarte,
	\[
		G^{-1}(x, y) = (xe^{-y}, y).
	\]

	Portanto, a solução é dada por
	\begin{align*}
		u(x, y) = v\circ G^{-1}(x, y) & = v(xe^{-y}, y)                                                            \\
		                              & = \sin^{}{\bigl(xe^{-y}\bigr)} + \frac{(xe^{-y})^{2}}{2}(e^{2y}-1)         \\
		                              & = \sin^{}{\bigl(xe^{-y}\bigr)} + \frac{x^{2}}{2} - \frac{x^{2}e^{-2y}}{2}.
	\end{align*}
\end{example}
\begin{example}
	Agora, considere a EDP com problema de condição inicial
	\[
		\left\{\begin{array}{ll}
			x \frac{\partial^{}u}{\partial x^{}} + \frac{\partial^{}u}{\partial y^{}} = u^{2} \\
			u(x, 0) = \sin^{}{(x)}.
		\end{array}\right.
	\]
	No primeiro passo, chegamos à conclusão de que
	\[
		x_{r}'(s) = x_{r}(s),\: x_{r}(0) = r \quad\&\quad y_{r}'(s) = 1,\: y_{r}(0) = 0,
	\]
	ou seja,
	\[
		\gamma_{r}(s) = (re^{s}, s).
	\]

	Encontrada a gama, podemos partir ao passo dois:
	\[
		v_{r}'(s) = v_{r}(s)^{2},\: v_{r}(0) = \sin^{}{(r)}.
	\]
	Com isso, obtemos
	\begin{align*}
		\frac{v_{r}'}{v_{r}^{2}} = 1 & \Longleftrightarrow \int_{0}^{s}\frac{v_{r}'(\tau )}{v_{r}(\tau )^{2}}d\tau = \int_{0}^{s}d\tau                          \\
		                             & \Longleftrightarrow \int_{v_{r}(0)}^{v_{r}(s)}\frac{du}{u^{2}} = s, \quad u = v_{r}(\tau )\:\&\: du = v_{r}'(\tau )d\tau \\
		                             & \Rightarrow -\frac{1}{u}\biggl|_{v_{r(0)}}^{v_{r}(s)}\biggr. = \frac{1}{v_{r(0)}} - \frac{1}{v_{r}(s)} = s               \\
		                             & \Rightarrow \frac{1}{v_{r}(s)} = \frac{1}{v_{r}(0)}-s.
	\end{align*}
	Logo,
	\[
		v_{r}(s) = \frac{v_{r}(0)}{1-sv_{r}(0)},
	\]
	tal que, sabendo o valor de \(v_{r}\) em 0, podemos reescrever
	\[
		v_{r}(s) = \frac{\sin^{}{(r)}}{1-s \sin^{}{(r)}}.
	\]
	Assim como antes, podemos agora encontrar G e resolver a equação:
	\[
		G(r, s) = (re^{s}, s) \Rightarrow G^{-1}(x, y) = (xe^{-y}, y),
	\]
	donde segue, portanto, que
	\[
		u(x, y) = v\circ G^{-1}(x, y) = v(xe^{-y}, y) = \frac{\sin^{}{(xe^{-y})}}{1-y\sin^{}{(xe^{-y})}}.
	\]
\end{example}
\begin{example}
	Em seguida, olhemos para o problema de condição inicial
	\[
		\left\{\begin{array}{ll}
			x \frac{\partial^{}u}{\partial x^{}}(x, y, z) + y \frac{\partial^{}u}{\partial y^{}}(x, y, z) + \frac{\partial^{}u}{\partial z^{}}(x, y, z) = u(x, y, z) \\
			u(x, y, 0) = x^{2} + y^{2}.
		\end{array}\right.
	\]
	Na primeira etapa, temos
	\begin{align*}
		 & x_{r}'(s) = x_{r}(s),\: x_{r}(0) = r_{1} \\
		 & y_{r}'(s) = y_{r}(s),\: y_{r}(0) = r_{2} \\
		 & z_{r}'(s) = 1.
	\end{align*}
	em que \(r = (r_{1}, r_{2})\) é um elemento de \(\mathbb{R}^{2}\). Resolvendo as EDOs, obtemos
	\[
		x_{r}(s) = r_{1}e^{s},\: y_{r}(s) = r_{2}e^{s},\; \& z_{r}(s) = s.
	\]
	Logo,
	\[
		\gamma_{r}(s) = (r_{1}e^{s}, r_{2}e^{s}, s).
	\]

	Agora, podemos aplicar a mudança para v através de
	\[
		\left.\begin{array}{ll}
			v_{r}'(s) = v_{r}(s) \\
			v_{r}(0) = r_{1}^{2} + r_{2}^{2}
		\end{array}\right\} v_{r}(s) = (r_{1}^{2} + r_{2}^{2})e^{s}.
	\]

	Finalmente, vamos encontrar a G e reverter seu efeito:
	\[
		G(r_{1},r_{2}, s) = (r_{1}e^{s}, r_{2}e^{s}, s) = (x, y, z),
	\]
	o que significa que passamos de s para z, de \(r_1\) para \(xe^{-s}\) e de \(r_{2}\) para \(ye^{-s}\). Com isso, notamos que
	\[
		G^{-1}(x, y, z) = (xe^{-z}, ye^{-z}, z)
	\]
	e, portanto,
	\begin{align*}
		u(x, y, z) = v\circ G^{-1}(x, y, z) & = v(xe^{-z}, ye^{-z}, z)    \\
		                                    & = (x^{2}+y^{2})e^{-2z}e^{z} \\
		                                    & = (x^{2}+y^{2})e^{-z}.
	\end{align*}
\end{example}
\begin{example}[\hypertarget{transport}{Equação do Transporte}]
	Finalmente, chegamos ao exemplo da equação do transporte para resolver de vez esta EDP. Ela é dada por
	\[
		\left\{\begin{array}{ll}
			\frac{\partial^{}u}{\partial t^{}}(x, t) + b \nabla u(x, t) = f(x, t) \\
			u(x, 0) = g(x)
		\end{array}\right.
	\]
	Aqui, a variável que denotamos por x, normalmente unidimensional, passou a ser em duas na forma (x, t). Além disso, \(x'=x\).

	No primeiro passo, temos
	\begin{align*}
		 & x_{1r}'(s) = b_1, \: x_{1r}(0) = r_{1} \Rightarrow x_{r1}(s) = r_{1}+b_{1}s \\
		 & x_{2r}'(s) = b_2, \: x_{2r}(0) = r_{2} \Rightarrow x_{r2}(s) = r_{2}+b_{2}s \\
		 & \vdots                                                                      \\
		 & x_{nr}'(s) = b_n, \: x_{nr}(0) = r_{n} \Rightarrow x_{rn}(s) = r_{n}+b_{n}s \\
		 & t'(s) = 1, \: t(0) = 0.
	\end{align*}
	Sendo assim, nossa curva integral tem forma
	\[
		\gamma_{r}(s) = (r+sb, s) = (r_{1} + sb_{1}, \dotsc , r_{n}+sb_{n}, s).
	\]

	Logo,
	\[
		v_{r}'(s) = f(r+sb, s),\; v_{r}(0) = g(r),
	\]
	que pode ser resolvida colocando
	\[
		v_{r}(s) = g(r) + \int_{0}^{s}f(r+\tau b, \tau )d\tau
	\]

	Seguindo esta linha, temos
	\[
		G(r, s) = \gamma_{r}(s) = (r + sb, s) = (x, t),
	\]
	tal que s corresponde a t e r corresponde a x-tb, ou seja,
	\[
		G^{-1}(x, t) = (x-tb, t).
	\]
	Portanto,
	\[
		u(x, t) = g(x-tb) + \int_{0}^{t}f(x-tb + \tau b, \tau )d\tau .
	\]

	Uma coisa bem legal desse caso é que ele deixa a possibilidade de olhar para tipos particulares dele. Por exemplo,
	\[
		\left\{\begin{array}{ll}
			\frac{\partial^{}u}{\partial t^{}} + b \frac{\partial^{}u}{\partial x^{}} = 0 \\
			u(x, 0) = g(x)
		\end{array}\right.
	\]
	Com base no que encontramos durante o exemplo, a solução geral é da forma
	\[
		u(x, t) = g(x-tb) + \int_{0}^{t}f(x-tb + \tau b, \tau )d\tau,
	\]
	mas, como a função f(x, t) é 0 neste caso, isto significa que a função que resolve esta EDP é
	\[
		u(x, t) = g(x-tb).
	\]
	Podemos observar como é o comportamento das curvas integrais de acordo com a transformação de coordenadas
	\[
		\gamma_{r}(s) = (r+bs, s) = (x, t),
	\]
	tal que \(x = r + bt\) e, isolando o r,
	\[
		x-tb = r \Rightarrow t = \frac{x-c}{b} = A+\frac{x}{b}
	\]
	para alguma constante A. Em outras palavras, conseguimos uma descrição para corresponder cada ponto x a um ponto em t - sabendo uma ``posição'', podemos encontrar o ``tempo'' em que ela ocorre.

	Em particular, se f for nula, também acontece algo com a v definida:
	\[
		v_{r}'(s) = 0 \Rightarrow v_{r}(s) = k
	\]
	para alguma constante k, isto é, u é constante ao longo das curvas integrais.
\end{example}
\hypertarget{nextclass03}{\begin{example}}
	Considere agora a seguinte EDP:
	\[
		\left\{\begin{array}{ll}
			-y \frac{\partial^{}u}{\partial x^{}} + x \frac{\partial^{}u}{\partial y^{}} = 0 \\
			u(x, 0) = g(x)
		\end{array}\right.
	\]
	Aplicando o primeiro passo da \hyperlink{recipe}{\textit{receita}}, temos que resolver
	\begin{align*}
		 & x_{r}'(s) = -y_{r}(s), \: x_{r}(0) = r \\
		 & y_{r}'(s) = x_{r}(s), \: y_{r}(0) = 0.
	\end{align*}
	Derivando novamente, segue que
	\[
		x_{r}''(s) = -y_{r}'(s) = -x_{r}(s),
	\]
	cuja solução é
	\[
		x_{r}(s) = A \cos^{}{(s)} + B\sin^{}{(s)}.
	\]
	Analogamente,
	\[
		y_{r}(s) = C\cos^{}{(s)} + D\sin^{}{(s)}.
	\]
	Continua na próxima aula...
\end{example}

\end{document}

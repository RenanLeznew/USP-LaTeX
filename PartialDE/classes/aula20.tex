\documentclass[../pde_notes.tex]{subfiles}
\begin{document}
\section{Aula 20 - 21 de Maio, 2025}
\subsection{Motivações}
\begin{itemize}
	\item Consequências da Fórmula de Poisson;
	\item Princípio do Máximo Forte;
	\item Desigualdade de Harnack;
\end{itemize}
\subsection{Consequências da Fórmula de Poisson.}
Nosso objeto de estudo é a fórmula de Poisson, feita no contexto de uma bola de centro (a, b) e raio \(R>0\):
\[
	\left\{\begin{array}{ll}
		\Delta u = 0, \quad x\in B \\
		u|_{\partial \Omega } = h
	\end{array}\right.
\]
sendo que
\[
	u(a+r\cos^{}{(\theta )}, b + r\sin^{}{(\theta )}) = \frac{R^{2}-r^{2}}{2\pi }\int_{-\pi }^{\pi }\frac{h(\psi )}{R^{2}+r^{2}-2rR\cos^{}{(\theta -\psi )}} \mathrm{d\psi }.
\]
Colocando \(h = u|_{\partial B}\), temos
\[
	h(\psi ) = u(a+R\cos^{}{(\psi) }, b+R\sin^{}{(\psi) }).
\]
Logo,
\[
	u(a+r\cos^{}{(\psi )}, b+r\sin^{}{(\psi )}) = \frac{R^{2}-r^{2}}{2\pi R}\int_{-\pi }^{\pi }\frac{u(a+R\cos^{}{(\psi )}, b+R\sin^{}{(\psi )})}{R^{2}+r^{2}-2rR\cos^{}{(\theta -\psi )}} \mathrm{d\psi },
\]
que pode ser reescrita por meio da substituição
\[
	\left.\begin{array}{ll}
		\gamma (\psi ) = (a + R \cos^{}{(\psi )}, b+r\sin^{}{(\psi)}) \\
		\Vert \gamma ' \Vert d\psi = R d\psi
	\end{array}\right\} \int_{\partial B}^{}u \mathrm{ds} = \frac{R^{2}-r^{2}}{2\pi R}\int_{\partial B}^{}\frac{u}{R^{2}-r^{2}-2rR\cos^{}{(\theta -\psi )}} \mathrm{ds},
\]
onde \(ds = R d\psi \). Além disso, note que, se considerarmos
\[
	x = (a+r\cos^{}{(\theta) }, b + r\sin^{}{(\theta )}) \quad\&\quad x' = (a+R\cos^{}{(\psi )}, b + R \cos^{}{(\psi )}),
\]
então
\begin{align*}
	\Vert x-x' \Vert^{2} & = \Vert (R\cos^{}{(\psi )}-r\cos^{}{(\theta )}, R\sin^{}{(\psi )} - r\sin^{}{(\theta )}) \Vert^{2}                                                                              \\
	                     & = (R\cos^{}{(\psi )}-r\cos^{}{(\theta )})^{2} + (R\sin^{}{(\psi )}-r\sin^{}{(\theta )})^{2}                                                                                     \\
	                     & = R^{2}\cos^{2}{(\psi )}+r^{2}\cos^{2}{(\theta )}-2rR\cos^{}{(\psi )}\cos^{}{(\theta) } + R^{2}\sin^{2}{(\psi )}+r^{2}\sin^{2}{(\theta )}-2rR\sin^{}{(\theta )}\sin^{}{(\psi )} \\
	                     & =R^{2}+r^{2}-2rR\cos^{}{(\theta -\psi )} \Rightarrow r^{2} = \Vert x-p \Vert^{2},\; p = (a, b).
\end{align*}
Em conclusão, u pode ser escrita como
\[
	u(x) = \frac{R^{2}-\Vert x-p \Vert^{2}}{2\pi R}\int_{\partial B}^{}\frac{u(x')}{\Vert x-x' \Vert^{2}} \mathrm{dS(x')},
\]
que é uma generalização da fórmula vista na aula passada que funciona para bolas em qualquer dimensão finita. Uma das consequências é que, se x for igual a p, então
\[
	u(p) = \frac{R}{2\pi }\int_{\partial B}^{}\frac{u(x')}{\Vert p-x' \Vert^{2}} \mathrm{dS(x')} = \frac{1}{2\pi R}\int_{\partial B}^{}u(x') \mathrm{dS(x')},
\]
e portanto recuperamos o valor médio
\[
	u(p) = \frac{1}{|\partial B|}\int_{\partial B}^{}u(x') \mathrm{dS(x')}.
\]
Logo,
\[
	u(a, b) = \frac{1}{2\pi r}\int_{-\pi }^{\pi }u(a+r\cos^{}{(\theta )}, b+r\sin^{}{(\theta )})r \mathrm{d\theta },\quad \forall r > 0,\; B_r(a, b)\subseteq \Omega,
\]
sendo que
\[
	\partial B = \{(a+r\cos^{}{(\theta )}, b+r\sin^{}{(\theta )}):\; -\pi <\theta \leq \pi \}\quad\&\quad  dS = rd\theta.
\]
Consequentemente,
\[
	\int_{0}^{R}ru(a, b) \mathrm{dr} = \frac{1}{2\pi }\int_{0}^{R}\int_{-\pi }^{\pi }u(a+r\cos^{}{(\theta )}, b+r\sin^{}{(\theta )}) \mathrm{d\theta } \mathrm{dr},
\]
ou seja,
\[
	\frac{R^{2}}{2}u(a, b) = \frac{1}{2\pi } \iint_{B_{R}(a, b)}udxdy \Longleftrightarrow u(a, b) = \frac{1}{\pi R^{2}}\iint_{B}udxdy,
\]
e novamente recuperamos o valor médio
\[
	u(p) = \frac{1}{|B|}\int_{B}^{}u \mathrm{dx}.
\]

A segunda consequência disso é o princípio do máximo forte:
\begin{theorem*}[Princípio do Máximo Forte]
	Seja \(\Omega \) um subconjunto aberto, conexo e limitado de \(\mathbb{R}^{2}\), e seja u uma função de classe \(\mathcal{C}^{2}(\overline{\Omega })\) tal que
	\[
		\Delta u = 0.
	\]
	Se \(x_{0}\) for um ponto em \(\Omega \) e tal que \(u(x_{0}) \geq u(x)\) para todo x em \(\Omega \), então
	\[
		u(x) = u(x_{0}),\; \forall x\in \Omega .
	\]
\end{theorem*}
\begin{proof*}
	Vamos definir
	\[
		\mathcal{C}\coloneqq \{x\in \Omega :\; u(x) = u(x_{0})\},
	\]
	que é não-vazio pois pelo menos o \(x_{0}\) pertence a ele. Além disso, \(\mathcal{C}\) é fechado em \(\Omega \), já que se \(\{x_{n}\}_{n}\) for uma sequência de elementos de \(\mathcal{C}\) que converge para o ponto \(\tilde{x}\) em \(\Omega \), então
	\[
		u(\tilde{x}) = \lim_{n\to \infty}u(x_{n}) = \lim_{n\to \infty}u(x_{0}) = u(x_{0}),
	\]
	logo \(\tilde{x}\) pertence também a \(\mathcal{C}.\) Vale também que \(\mathcal{C}\) é aberto; de fato, dado \(\tilde{x}\) em \(\mathcal{C}\) e um r positivo tal que
	\[
		\overline{B_r(\tilde{x})}\subseteq \Omega ,
	\]
	temos
	\[
		\frac{1}{|B|}\int_{B}^{}u(\tilde{x}) \mathrm{dx} = u(\tilde{x}) = \frac{1}{|B|}\int_{B}^{}u \mathrm{dx},
	\]
	de forma que
	\[
		\int_{B}^{}\underbrace{u(\tilde{x})-u(x)}_{\geq 0} \mathrm{dx} = 0 \Rightarrow u(x) = u(\tilde{x}) = u(x_{0}),\; \forall x\in B_r(\tilde{x}).
	\]
	Consequentemente,
	\[
		B_r(\tilde{x})\subseteq \mathcal{C}.
	\]
	Concluímos, com isso, que \(\mathcal{C}\) é não-vazio, aberto E fechado no \textit{conexo} \(\Omega \). Portanto,
	\[
		\mathcal{C}=\Omega  \Rightarrow u(x) = u(x_{0}),\; \forall x\in \Omega . \text{ \qedsymbol}
	\]
\end{proof*}

Veremos também a desigualdade de Harnack e o que resulta dela, dada por
\hypertarget{harnack_inequality}{
	\begin{theorem*}[Desigualdade de Harnack]
		Seja \(u\in \mathcal{C}^{2}(\mathbb{R}^{2})\) uma solução de
		\[
			\Delta u = 0.
		\]
		Então, para todo raio positivo tal que \(\Vert x \Vert<R\), valem as desigualdades
		\[
			\frac{R-\Vert x \Vert}{R+\Vert x \Vert}u(0)\leq u(x)\leq \frac{R+\Vert x \Vert}{R-\Vert x \Vert}u(0).
		\]
	\end{theorem*}
}
\begin{proof*}
	Tomemos o ponto \(X=(x, y)\) de \(\mathbb{R}^{2}\); sabemos que, pela fórmula de Poisson,
	\[
		u(X) = \frac{R^{2}-\Vert X \Vert^{2}}{2rR}\int_{\partial B_{r}(0)}^{}\frac{u(x')}{\Vert x-x' \Vert^{2}} \mathrm{dS}.
	\]
	Colocando \(R = \Vert x' \Vert,\) segue que
	\begin{align*}
		R-\Vert x \Vert = \Vert x' \Vert-\Vert x \Vert \leq \Vert x'-x \Vert & \leq \Vert x' \Vert+\Vert x \Vert \\
		                                                                     & \leq R+\Vert x \Vert.
	\end{align*}
	Desta forma,
	\begin{align*}
		u(x) & = \frac{(R+\Vert x \Vert)(R-\Vert x \Vert)}{2\pi R}\int_{\partial B_r(0)}^{}\frac{u(x')}{\Vert x-x' \Vert^{2}} \mathrm{dS} \\
		     & \leq \frac{(R+\Vert x \Vert)(R-\Vert x \Vert)}{2\pi R}\int_{\partial B}^{}\frac{u(x')}{(R-\Vert x \Vert)^{2}} \mathrm{dS}  \\
		     & = \frac{R+\Vert x \Vert}{R-\Vert x \Vert}\frac{1}{2\pi R}\int_{\partial B}^{}u(x') \mathrm{dS}                             \\
		     & =\frac{R+\Vert x \Vert}{R-\Vert x \Vert}u(0)
	\end{align*}
	e
	\begin{align*}
		u(x) & = \frac{(R+\Vert x \Vert)(R-\Vert x \Vert)}{2\pi R}\int_{\partial B_r(0)}^{}\frac{u(x')}{\Vert x-x' \Vert^{2}} \mathrm{dS} \\
		     & \geq \frac{(R+\Vert x \Vert)(R-\Vert x \Vert)}{2\pi R}\int_{\partial B}^{}\frac{u(x')}{(R+\Vert x \Vert)^{2}} \mathrm{dS}  \\
		     & = \frac{R-\Vert x \Vert}{R+\Vert x \Vert}\frac{1}{2\pi R}\int_{\partial B}^{}u(x') \mathrm{ds}                             \\
		     & = \frac{R-\Vert x \Vert}{R+\Vert x \Vert}u(0).
	\end{align*}
	Portanto,
	\[
		\frac{R-\Vert x \Vert}{R+\Vert x \Vert}u(0)\leq u(x)\leq \frac{R+\Vert x \Vert}{R-\Vert x \Vert}u(0). \quad \text{\qedsymbol}
	\]
\end{proof*}
Essa desigualdade em si não é muito legal, mas a consequência dela sim! É o chamado
\hypertarget{liouville_theorem}{
	\begin{theorem*}[Teorema de Liouville]
		Seja u uma função de classe \(\mathcal{C}^{2}(\mathbb{R}^{2})\), sempre não-negativa (\(u\geq 0\)) e que satisfaça
		\[
			\Delta u = 0.
		\]
		Então, u é constante.
	\end{theorem*}
}
\begin{proof*}
	Pela \hyperlink{harnack_inequality}{\textit{Desigualdade de Harnack}},
	\[
		\frac{R-\Vert x \Vert}{R+\Vert x \Vert}u(0)\leq u(x)\leq \frac{R+\Vert x \Vert}{R-\Vert x \Vert}u(0),\quad \forall \Vert x \Vert<R.
	\]
	Fixando x e tomando o limite quando R tende a infinito, obtemos
	\[
		u(0)\leq u(x)\leq u(0) \Rightarrow u(x)=u(0). \quad \text{\qedsymbol}
	\]
\end{proof*}
\subsection{Uma Palavrinha sobre o Teorema da Divergência.}
O \hyperlink{divergence_theorem}{\textit{teorema da Divergência}} no cálculo é visto em \(\mathbb{R}^{3}\) como
\[
	\int_{\Omega }^{}\nabla \cdot F \mathrm{dx} = \int_{\partial \Omega }^{}F \cdot n \mathrm{dS},
\]
normalmente junto ao teorema de Green em \(\mathbb{R}^{2}\)
\hypertarget{green_theorem}{\[
		\int_{S}^{}\underbrace{P \mathrm{dx} + Q \mathrm{dy}}_{F_1 \mathrm{d}x + F_2 \mathrm{d}y} = \underbrace{\iint_{\Omega }\biggl(\frac{\partial^{}Q}{\partial x^{}} - \frac{\partial^{}P}{\partial y^{}}\biggr) \mathrm{d}x \mathrm{d}y}_{\iint_{\Omega }\nabla \cdot F dxdy}.
	\]}
Com isso, colocando
\[
	\gamma (t) = (x(t), y(t)),\; \gamma '(t) = (x'(t), y'(t)),\;\&\; n'(t) = \frac{(y'(t),-x'(t))}{\Vert (x'(t), y'(t)) \Vert},
\]
seguirá que
\begin{align*}
	\int_{\partial \Omega }^{}F \cdot n \mathrm{dS} & = \int_{\partial \Omega }^{}(F_1 y' - F_2x')\frac{1}{\Vert (x', y') \Vert}\Vert (x', y') \Vert \mathrm{dt} \\
	                                                & =\int_{\partial \Omega }^{}F_1 \mathrm{dy} - F_2 \mathrm{dx}                                               \\
	                                                & = \int_{\partial \Omega }^{}P \mathrm{dx} + Q \mathrm{dy},
\end{align*}
e ele também pode ser visto como um teorema de integração por partes, já que, supondo que temos um aberto em duas dimensões (mesma coisa para outras!), teremos
\begin{align*}
	\int_{\partial \Omega }^{}\frac{\partial^{}f}{\partial x^{}}g \mathrm{dx}\mathrm{dy} & = \int_{\Omega }^{}\biggl(\frac{\partial^{}}{\partial x^{}}(fg) - f \frac{\partial^{}g}{\partial x^{}}\biggr) \mathrm{dxdy} \\
	                                                                                     & =-\int_{\Omega }^{}f \frac{\partial^{}g}{\partial x^{}} \mathrm{dxdy} + \int_{\Omega }^{}\nabla \cdot (fg, 0) \mathrm{dxdy} \\
	                                                                                     & = - \int_{\Omega }^{}f \frac{\partial^{}g}{\partial x^{}} \mathrm{dxdy} + \int_{\partial \Omega }^{}fg n_{x} \mathrm{dS},
\end{align*}
em que usamos
\[
	\int_{\Omega }^{}\nabla \cdot (fg, 0) \mathrm{dxdy} = \int_{\partial \Omega }^{}(fg, 0) \cdot (n_x, ny) \mathrm{dS} = \int_{\partial \Omega }^{}fgn_x \mathrm{dS}.
\]
Analogamente, para y, temos
\[
	\int_{\Omega }^{}\frac{\partial^{}f}{\partial y^{}}g \mathrm{dxdy} = \int_{\partial \Omega }^{}fgn_y \mathrm{dS} - \int_{\Omega }^{}f \frac{\partial^{}g}{\partial y^{}} \mathrm{dxdy}.
\]
Juntando as duas formas com \(F_1\) e \(F_2\) nos papeis de f e 1 no papel de g, recuperamos o teorema da divergência:
\[
	\int_{\Omega }^{}\frac{\partial^{}F_1}{\partial x^{}} + \frac{\partial^{}F_2}{\partial y^{}} \mathrm{dxdy} = \int_{\Omega }^{}F_1n_x + F_2n_y \mathrm{dS} \Longleftrightarrow \int_{\Omega }^{}\nabla \cdot F \mathrm{dxdy} = \int_{\partial \Omega }^{}F \cdot n \mathrm{dS}.
\]

\end{document}
